\section{转置的性质}

本节, 我们讨论转置的一些性质.

设 \(A\) 是一个 \(m \times n\)~阵.
则 \(A\)~的转置 \(A^{\mathrm{T}}\) 是一个 \(n \times m\)~阵,
且对 \(i \leq n\) 与 \(j \leq m\),
有 \([A^{\mathrm{T}}]_{i,j} = [A]_{j,i}\).

我们已知, 若 \(A\) 是一个 \(m \times n\)~阵,
则 \((A^{\mathrm{T}})^{\mathrm{T}} = A\).
我们还知道, 若 \(A\) 是一个 \(n\)~级阵,
则 \(\det {(A)} = \det {(A^{\mathrm{T}})}\).

转置当然还有一些性质;
我只是还没提到它们.

\begin{theorem}
    设 \(A\), \(B\) 是 \(m \times n\) 阵.
    设 \(C\) 是 \(n \times s\) 阵.
    设 \(k\) 是数.
    则:

    (1)
    \((A + B)^{\mathrm{T}} = A^{\mathrm{T}} + B^{\mathrm{T}}\);

    (2)
    \((k A)^{\mathrm{T}} = k A^{\mathrm{T}}\);

    (3)
    \((A C)^{\mathrm{T}} = C^{\mathrm{T}} A^{\mathrm{T}}\).
\end{theorem}

\begin{proof}
    您验证, 等式 (1) 与 (2) 的二侧的阵的尺寸是一样的;
    我验证, 等式 (3) 的二侧的阵的尺寸是一样的.

    (1)
    对 \(i \leq n\) 与 \(j \leq m\),
    \begin{align*}
        [(A + B)^{\mathrm{T}}]_{i,j}
        = {} &
        [A + B]_{j,i}
        \\
        = {} &
        [A]_{j,i} + [B]_{j,i}
        \\
        = {} &
        [A^{\mathrm{T}}]_{i,j} + [B^{\mathrm{T}}]_{i,j}
        \\
        = {} &
        [A^{\mathrm{T}} + B^{\mathrm{T}}]_{i,j}.
    \end{align*}

    (2)
    对 \(i \leq n\) 与 \(j \leq m\),
    \begin{align*}
        [(k A)^{\mathrm{T}}]_{i,j}
        = {} &
        [k A]_{j,i}
        \\
        = {} &
        k [A]_{j,i}
        \\
        = {} &
        k [A^{\mathrm{T}}]_{i,j}
        \\
        = {} &
        [k A^{\mathrm{T}}]_{i,j}.
    \end{align*}

    (3)
    \(A\) 是 \(m \times n\)~的, \(C\) 是 \(n \times s\)~的,
    故 \(AC\) 是 \(m \times s\)~的,
    故 \((AC)^{\mathrm{T}}\) 是 \(s \times m\)~的.
    另一方面, \(C^{\mathrm{T}}\) 是 \(s \times n\)~的,
    \(A^{\mathrm{T}}\) 是 \(n \times m\)~的,
    故 \(C^{\mathrm{T}} A^{\mathrm{T}}\) 是 \(s \times m\)~的.
    对 \(i \leq s\) 与 \(j \leq m\),
    \begin{align*}
        [(AC)^{\mathrm{T}}]_{i,j}
        = {} &
        [AC]_{j,i}
        \\
        = {} &
        \sum_{p=1}^{n} {[A]_{j,p} [C]_{p,i}}
        \\
        = {} &
        \sum_{p=1}^{n}
        {[A^{\mathrm{T}}]_{p,j} [C^{\mathrm{T}}]_{i,p}}
        \\
        = {} &
        \sum_{p=1}^{n}
        {[C^{\mathrm{T}}]_{i,p} [A^{\mathrm{T}}]_{p,j}}
        \\
        = {} &
        [C^{\mathrm{T}} A^{\mathrm{T}}]_{i,j}.
        \qedhere
    \end{align*}
\end{proof}

为方便, 我记下一个简单的推广.
设 \(A\), \(B\), \(C\) 分别是
\(m \times s\), \(s \times t\), \(t \times n\)~阵.
则
\begin{align*}
    (ABC)^{\mathrm{T}}
    = {} &
    ((AB) C)^{\mathrm{T}}
    \\
    = {} &
    C^{\mathrm{T}}
    (AB)^{\mathrm{T}}
    \\
    = {} &
    C^{\mathrm{T}}
    (B^{\mathrm{T}} A^{\mathrm{T}})
    \\
    = {} &
    C^{\mathrm{T}}
    B^{\mathrm{T}}
    A^{\mathrm{T}}.
\end{align*}
一般地, 我们有

\begin{theorem}
    设 \(A_1\), \(A_2\), \(\dots\), \(A_n\)
    分别是 \(m_0 \times m_1\), \(m_1 \times m_2\), \(\dots\),
    \(m_{n-1} \times m_n\)~阵
    (也就是说, \(A_k\) 的列数等于 \(A_{k+1}\) 的行数,
    \(k = 1\), \(2\), \(\dots\), \(n-1\)).
    则
    \begin{align*}
        (A_1 A_2 \dots A_n)^{\mathrm{T}}
        = A_n^{\mathrm{T}} \dots A_2^{\mathrm{T}} A_1^{\mathrm{T}}.
    \end{align*}
\end{theorem}

\begin{proof}
    请允许我留此事为您的习题.
    您用数学归纳法即可.
\end{proof}

\section{辛阵}

设 \(2m\) 是一个不低于 \(2\) 的偶数.
作 \(2m\)~级阵
\begin{align*}
    K_m =
    \begin{bmatrix}
        0    & I_m \\
        -I_m & 0   \\
    \end{bmatrix},
\end{align*}
其中 \(0\) 是 \(m\)~级零阵;
具体地,
\begin{align*}
    [K_m]_{i,j} =
    \begin{cases}
        [0]_{i,j},      & \text{\(i \leq m\), \(j \leq m\)}; \\
        [I_m]_{i,j-m},  & \text{\(i \leq m\), \(j > m\)};    \\
        [-I_m]_{i-m,j}, & \text{\(i > m\), \(j \leq m\)};    \\
        [0]_{i-m,j-m},  & \text{\(i > m\), \(j > m\)}.
    \end{cases}
\end{align*}
更具体地,
\begin{align*}
    [K_m]_{i,j} =
    \begin{cases}
        1,  & 1 \leq i = j - m \leq m; \\
        -1, & 1 \leq i - m = j \leq m; \\
        0,  & \text{其他}.
    \end{cases}
\end{align*}
不难算出, \(K_m^{\mathrm{T}} = -K_m\),
且 \(K_m K_m = -I_{2m}\).

按列~\(1\) 展开 \(\det {(K_m)}\), 有
\begin{align*}
    \det {(K_m)}
    = (-1)^{m+1} (-1) \det {(K_m ({m+1}|1))}
    = (-1)^m \det {
        \begin{bmatrix}
            0        & I_m \\
            -I_{m-1} & 0   \\
        \end{bmatrix}
    }.
\end{align*}
再按行~\(1\) 展开 \(\det {(K_m ({m+1}|1))}\), 有
\begin{align*}
    \det {(K_m ({m+1}|1))}
    = {} &
    (-1)^{1+m-1}\, 1 \det {(K_m ({m+1,1}|{1,m+1}))}
    \\
    = {} &
    (-1)^m \det {(K_{m-1})}.
\end{align*}
则 \(\det {(K_m)} = \det {(K_{m-1})}\),
当 \(m > 1\).
不难算出 \(\det {(K_1)} = 1\).
则 \(\det {(K_m)} = 1\), 对 \(m \geq 1\).
% 则 \(\operatorname{adj} {(K_m)} K_m = \det {(K_m)} I_{2m} = I_{2m}\).
% 又因为 \((-K_m) K_m = -(K_m K_m) = -(-I_{2m}) = I_{2m}\),
% 故 \(\operatorname{adj} {(K_m)} = -K_m\).

现在, 我们可定义辛阵 (simplektika matrico) 了.

\begin{definition}
    设 \(A\) 是一个 \(2m\)~级阵.
    若 \(A^{\mathrm{T}} K_m A = K_m\),
    则 \(A\) 是一个\emph{辛阵}.
\end{definition}

我们可写出辛阵的一些性质.

不难看出, \(2m\)~级单位阵 \(I_{2m}\) 是一个辛阵.
由 \(K_m\) 的性质, 不难验证,
\(K_m\) 也是一个辛阵.

设 \(A\), \(B\) 是 \(2m\)~级辛阵.
则
\begin{align*}
    (AB)^{\mathrm{T}} K_m (AB)
    = {} &
    (B^{\mathrm{T}} A^{\mathrm{T}})
    (K_m A) B
    \\
    = {} &
    B^{\mathrm{T}}
    (A^{\mathrm{T}} K_m A)
    B
    \\
    = {} &
    B^{\mathrm{T}} K_m B
    \\
    = {} &
    K_m.
\end{align*}

因为
\begin{align*}
    \det {(K_m)}
    = \det {(A^{\mathrm{T}} K_m A)}
    = \det {(A^{\mathrm{T}})} \det {(K_m)} \det {(A)}
    = \det {(K_m)}\, (\det {(A)})^2,
\end{align*}
且 \(\det {(K_m)} = 1\),
故 \((\det {(A)})^2 = 1\).

若数 \(t\) 适合 \(t^2 = 1\),
则
\begin{align*}
    (t A)^{\mathrm{T}} K_m (t A)
    = (t A^{\mathrm{T}}) K_m (t A)
    = t^2 (A^{\mathrm{T}} K_m A)
    = 1 K_m = K_m.
\end{align*}

注意到
\(\det {(K_m A)} = \det {(K_m)} \det {(A)} \neq 0\),
且
\begin{align*}
    ((A^{\mathrm{T}})^{\mathrm{T}} K_m A^{\mathrm{T}})
    (K_m A)
    = {} &
    (A K_m A^{\mathrm{T}})
    (K_m A)
    \\
    = {} &
    (A K_m) (A^{\mathrm{T}} K_m A)
    \\
    = {} &
    (A K_m) K_m
    \\
    = {} &
    A (K_m K_m)
    \\
    = {} &
    A (-I_{2m})
    \\
    = {} &
    {(-I_{2m})} A
    \\
    = {} &
    (K_m K_m) A
    \\
    = {} &
    K_m (K_m A),
\end{align*}
故
\((A^{\mathrm{T}})^{\mathrm{T}} K_m A^{\mathrm{T}} = K_m\).

最后, 注意到
\begin{align*}
    (\operatorname{adj} {(A)})^{\mathrm{T}}\,
    K_m
    \operatorname{adj} {(A)}
    = {} &
    (\operatorname{adj} {(A)})^{\mathrm{T}}\,
    (A^{\mathrm{T}} K_m A)
    \operatorname{adj} {(A)}
    \\
    = {} &
    ((\operatorname{adj} {(A)})^{\mathrm{T}}\, A^{\mathrm{T}})\,
    K_m\,
    (A \operatorname{adj} {(A)})
    \\
    = {} &
    (A \operatorname{adj} {(A)})^{\mathrm{T}}\,
    K_m\,
    (A \operatorname{adj} {(A)})
    \\
    = {} &
    (\det {(A)}\, I_{2m})^{\mathrm{T}}\,
    K_m\,
    (\det {(A)}\, I_{2m})
    \\
    = {} &
    (\det {(A)}\, I_{2m}^{\mathrm{T}})\, K_m\,
    (\det {(A)}\, I_{2m})
    \\
    = {} &
    (\det {(A)})^2\, (I_{2m}^{\mathrm{T}} K_m I_{2m})
    \\
    = {} &
    K_m.
\end{align*}

总结这些结果, 我们有

\begin{theorem}
    (1)
    \(I_{2m}\) 与 \(K_m\) 是 \(2m\)~级辛阵.

    (2)
    \(2m\)~级辛阵 \(A\) 的行列式的平方为 \(1\).

    (3)
    设 \(A\), \(B\) 是 \(2m\)~级辛阵.
    设数 \(t\) 适合 \(t^2 = 1\).
    则 \(AB\), \(tA\),
    \(A^{\mathrm{T}}\),
    \(\operatorname{adj} {(A)}\)
    也是辛阵.
\end{theorem}

本书是关于行列式的.
于是, 一个自然的问题是, 辛阵的行列式是多少.
我们知道, 辛阵的行列式的平方是 \(1\),
故辛阵的行列式是 \(1\) 或 \(-1\).
不过, 不平凡地, 辛阵的行列式\emph{一定}是 \(1\).
我们现有的知识无法解决此事.
我会在后面的几节, 介绍更多的知识, 以解决它.

虽然如此, 我们还是能解决 \(2m = 2\) 的情形.
设
\begin{align*}
    A = \begin{bmatrix}
            a & b \\
            c & d \\
        \end{bmatrix}
\end{align*}
是一个 \(2\)~级辛阵.
由 \(A^{\mathrm{T}} K_1 A = K_1\), 知
\begin{align*}
    \begin{bmatrix}
        a & c \\
        b & d \\
    \end{bmatrix}
    \begin{bmatrix}
        0  & 1 \\
        -1 & 0 \\
    \end{bmatrix}
    \begin{bmatrix}
        a & b \\
        c & d \\
    \end{bmatrix}
    =
    \begin{bmatrix}
        0  & 1 \\
        -1 & 0 \\
    \end{bmatrix},
\end{align*}
即
\begin{align*}
    \begin{bmatrix}
        0          & ad - bc \\
        -(ad - bc) & 0       \\
    \end{bmatrix}
    =
    \begin{bmatrix}
        0  & 1 \\
        -1 & 0 \\
    \end{bmatrix}.
\end{align*}
故
\(\det {(A)} = ad - bc = 1\).

\section{反称阵}

从本节开始, 我们讨论反称阵与其性质.

\begin{definition}
    设 \(A\) 是一个 \(n\)~级阵.
    若 \([A]_{i,i} = 0\),
    且
    \([A]_{i,j} + [A]_{j,i} = 0\),
    则 \(A\) 是一个\emph{反称阵}.
\end{definition}

\begin{example}
    不难看出, \(1\)~级反称阵即为 \([0]\),
    \(2\)~级反称阵形如
    \begin{align*}
        \begin{bmatrix}
            0  & a \\
            -a & 0 \\
        \end{bmatrix},
    \end{align*}
    \(3\)~级反称阵形如
    \begin{align*}
        \begin{bmatrix}
            0  & a  & b \\
            -a & 0  & c \\
            -b & -c & 0 \\
        \end{bmatrix},
    \end{align*}
    \(4\)~级反称阵形如
    \begin{align*}
        \begin{bmatrix}
            0  & a  & b  & c \\
            -a & 0  & d  & e \\
            -b & -d & 0  & f \\
            -c & -e & -f & 0 \\
        \end{bmatrix}.
    \end{align*}
\end{example}

\begin{example}
    不难验证, \(2m\)~级阵
    \(
    K_m =
    \begin{bmatrix}
        0    & I_m \\
        -I_m & 0   \\
    \end{bmatrix}
    \)
    是反称阵.
\end{example}

不难看出,
\([A]_{i,j} + [A]_{j,i} = 0\)
相当于 \(A^{\mathrm{T}} = -A\):
注意到 \([A^{\mathrm{T}}]_{i,j} = [A]_{j,i}\)
与 \([-A]_{i,j} = -[A]_{i,j}\).
有时, 这更方便应用.

\begin{theorem}
    设 \(A\), \(B\) 是 \(n\)~级反称阵.
    设 \(k\) 是数.
    设 \(X\) 是 \(n \times m\)~阵.

    (1)
    \(n\)~级阵 \(0\) 是反称阵.

    (2)
    \(A + B\) 是反称阵.

    (3)
    \(kA\) 是反称阵;
    特别地, \((-1)A = -A = A^{\mathrm{T}}\) 也是反称阵.

    (4)
    \(X^{\mathrm{T}} A X\) 是反称阵.
\end{theorem}

\begin{proof}
    (1)
    \(0^{\mathrm{T}} = 0 = -0\),
    且 \([0]_{i,i} = 0\).

    (2)
    由转置的性质,
    \((A + B)^{\mathrm{T}} = A^{\mathrm{T}} + B^{\mathrm{T}}
    = (-A) + (-B) = -(A + B)\).
    再注意到
    \begin{align*}
        [A + B]_{i,i} = [A]_{i,i} + [B]_{i,i} = 0 + 0 = 0.
    \end{align*}

    (3)
    由转置的性质,
    \((kA)^{\mathrm{T}} = kA^{\mathrm{T}} = k(-A) = -(kA)\).
    再注意到
    \begin{align*}
        [kA]_{i,i} = k [A]_{i,i} = k 0 = 0.
    \end{align*}

    (4)
    \(X\) 是 \(n \times m\)~的, 则 \(AX\) 是 \(n \times m\) 的;
    \(X^{\mathrm{T}}\) 是 \(m \times n\)~的,
    则 \(X^{\mathrm{T}} A X\) 是 \(m \times m\)~的.
    由转置的性质,
    \begin{align*}
        (X^{\mathrm{T}} A X)^{\mathrm{T}}
        = X^{\mathrm{T}} A^{\mathrm{T}} (X^{\mathrm{T}})^{\mathrm{T}}
        = X^{\mathrm{T}} (-A) X
        = -(X^{\mathrm{T}} A X).
    \end{align*}
    由阵的积的定义,
    \begin{align*}
             &
        [X^{\mathrm{T}} A X]_{i,i}
        \\
        = {} &
        [X^{\mathrm{T}} (A X)]_{i,i}
        \\
        = {} &
        \sum_{\ell = 1}^{n}
        {[X^{\mathrm{T}}]_{i,\ell} [A X]_{\ell,i}}
        \\
        = {} &
        \sum_{\ell = 1}^{n}
        {[X]_{\ell,i} [A X]_{\ell,i}}
        \\
        = {} &
        \sum_{\ell = 1}^{n}
        {
        [X]_{\ell,i}
        \left(\sum_{k = 1}^{n} [A]_{\ell,k} [X]_{k,i}\right)
        }
        \\
        = {} &
        \sum_{\ell = 1}^{n}
        \sum_{k = 1}^{n}
        {[X]_{\ell,i} [A]_{\ell,k} [X]_{k,i}}
        \\
        = {} &
        \sum_{\ell, k = 1}^{n}
        {[X]_{\ell,i} [X]_{k,i} [A]_{\ell,k}}
        \\
        = {} & \hphantom{{} + {}}
        \sum_{1 \leq \ell = k \leq n}
        {[X]_{\ell,i} [X]_{k,i} [A]_{\ell,k}}
        +
        \sum_{1 \leq \ell < k \leq n}
        {[X]_{\ell,i} [X]_{k,i} [A]_{\ell,k}}
        \\
             & +
        \sum_{1 \leq k < \ell \leq n}
        {[X]_{\ell,i} [X]_{k,i} [A]_{\ell,k}}
        \\
        = {} & \hphantom{{} + {}}
        \sum_{1 \leq \ell = k \leq n}
        {[X]_{\ell,i} [X]_{k,i}\, 0}
        +
        \sum_{1 \leq \ell < k \leq n}
        {[X]_{\ell,i} [X]_{k,i} [A]_{\ell,k}}
        \\
             & +
        \sum_{1 \leq \ell < k \leq n}
        {[X]_{k,i} [X]_{\ell,i} [A]_{k,\ell}}
        \\
        = {} &
        0 +
        \sum_{1 \leq \ell < k \leq n}
        {[X]_{\ell,i} [X]_{k,i} ([A]_{\ell,k} + [A]_{k,\ell})}
        \\
        = {} &
        \sum_{1 \leq \ell < k \leq n}
        {[X]_{\ell,i} [X]_{k,i}\, 0}
        \\
        = {} &
        0.
    \end{align*}
    于是, \(X^{\mathrm{T}} A X\) 是一个 \(m\)~级反称阵.
\end{proof}

\section{\texorpdfstring{奇数级反称阵的行列式为 \(0\)}
  {奇数级反称阵的行列式为 0}}

我们讨论反称阵的行列式.

我们先计算小级反称阵的行列式.

\begin{example}
    设 \(A\) 是 \(1\)~级反称阵.
    则 \(A = [0]\).
    故 \(\det {(A)} = 0\).
\end{example}

\begin{example}
    设 \(A\) 是 \(2\)~级反称阵.
    则 \(A\) 形如
    \begin{align*}
        \begin{bmatrix}
            0  & a \\
            -a & 0 \\
        \end{bmatrix}.
    \end{align*}
    则 \(\det {(A)} = a^2\),
    即 \(\det {(A)} = [A]_{1,2}^2\).
\end{example}

\begin{example}
    设 \(A\) 是 \(3\)~级反称阵.
    则 \(A\) 形如
    \begin{align*}
        \begin{bmatrix}
            0  & a  & b \\
            -a & 0  & c \\
            -b & -c & 0 \\
        \end{bmatrix}.
    \end{align*}
    则
    \begin{align*}
        \det {(A)}
        = {} &
        0
        \det {\begin{bmatrix}
                      0  & c \\
                      -c & 0 \\
                  \end{bmatrix}}
        - (-a)
        \det {\begin{bmatrix}
                      a  & b \\
                      -c & 0 \\
                  \end{bmatrix}}
        + (-b)
        \det {\begin{bmatrix}
                      a & b \\
                      0 & c \\
                  \end{bmatrix}}
        \\
        = {} &
        a(bc) + (-b)(ac)
        \\
        = {} & 0.
    \end{align*}
\end{example}

\begin{example}
    设 \(A\) 是 \(4\)~级反称阵.
    则 \(A\) 形如
    \begin{align*}
        \begin{bmatrix}
            0  & a  & b  & c \\
            -a & 0  & d  & e \\
            -b & -d & 0  & f \\
            -c & -e & -f & 0 \\
        \end{bmatrix}.
    \end{align*}
    则
    \begin{align*}
             & \det {(A)}
        \\
        = {} &
        \hphantom{{} + {}}
        0
        \det {\begin{bmatrix}
                      0  & d  & e \\
                      -d & 0  & f \\
                      -e & -f & 0 \\
                  \end{bmatrix}}
        - (-a)
        \det {\begin{bmatrix}
                      a  & b  & c \\
                      -d & 0  & f \\
                      -e & -f & 0 \\
                  \end{bmatrix}}
        \\
             &
        + (-b)
        \det {\begin{bmatrix}
                      a  & b  & c \\
                      0  & d  & e \\
                      -e & -f & 0 \\
                  \end{bmatrix}}
        - (-c)
        \det {\begin{bmatrix}
                      a  & b & c \\
                      0  & d & e \\
                      -d & 0 & f \\
                  \end{bmatrix}}
        \\
        = {} &
        a (dfc - ebf + aff)
        - b (-ebe + afe + edc)
        + c (adf - dbe + ddc)
        \\
        = {} &
        af (af - be + cd)
        - be (af - be + cd)
        + cd (af - be + cd)
        \\
        = {} & (af - be + cd)^2,
    \end{align*}
    即
    \begin{align*}
        \det {(A)}
        = ([A]_{1,2} [A]_{3,4} - [A]_{1,3} [A]_{2,4}
        + [A]_{1,4} [A]_{2,3})^2.
    \end{align*}
\end{example}

我们发现, \(1\)~级反称阵与 \(3\)~级反称阵%
的行列式为 \(0\).
一般地, 我们有

\begin{theorem}
    设 \(A\) 是一个奇数级反称阵.
    则 \(\det {(A)} = 0\).
\end{theorem}

我的论证会用如下几个事实.

(1)
设 \(A\) 是一个 \(n\)~级阵.
则
\begin{align*}
    \det {(A)}
    = \sum_{i = 1}^{n}
    {(-1)^{i+1} [A]_{i,1} \det {(A(i|1))}}.
\end{align*}
这就是按列~\(1\) 展开行列式.

(2)
设 \(A\) 是一个 \(n\)~级阵.
则
\begin{align*}
    \det {(A)}
    = \sum_{j = 1}^{n}
    {(-1)^{1+j} [A]_{1,j} \det {(A(1|j))}}.
\end{align*}
这就是按行~\(1\) 展开行列式.

(3)
设 \(A\) 是一个 \(n\)~级阵 (\(n \geq 3\)).
同时用 (1) (2),
有
\begin{align*}
         & \det {(A)}
    \\
    = {} &
    [A]_{1,1} \det {(A(1|1))}
    +
    \sum_{j = 2}^{n} {
    (-1)^{1+j} [A]_{1,j} \det {(A(1|j))}
    }
    \\
    = {} &
    [A]_{1,1} \det {(A(1|1))}
    +
    \sum_{j = 2}^{n} {
    (-1)^{1+j} [A]_{1,j}
    \sum_{i = 2}^{n} {
    (-1)^{i-1+1} [A]_{i,1} \det {(A(1,i|j,1))}
    }
    }
    \\
    = {} &
    [A]_{1,1} \det {(A(1|1))}
    +
    \sum_{j = 2}^{n} {
    \sum_{i = 2}^{n} {
    (-1)^{1+j} [A]_{1,j}
    (-1)^{i-1+1} [A]_{i,1} \det {(A(1,i|j,1))}
    }
    }
    \\
    = {} &
    [A]_{1,1} \det {(A(1|1))}
    +
    \sum_{i, j = 2}^{n} {
    (-1)^{i+j-1}
        [A]_{i,1} [A]_{1,j} \det {(A(1,i|1,j))}
    }.
\end{align*}
特别地, 若 \(A\) 还是一个反称阵, 则
\begin{align*}
    \det {(A)}
    = \sum_{i, j = 2}^{n} {
    (-1)^{i+j}
        [A]_{1,i} [A]_{1,j} \det {(A(1,i|1,j))}
    }.
\end{align*}

(4)
设 \(A\) 是一个 \(n\)~级阵.
则 \(\det {(A^{\mathrm{T}})} = \det {(A)}\).
这就是行列式与转置的关系.

(5)
设 \(A\) 是一个 \(n\)~级阵.
设 \(u\) 是一个数.
则 \(\det {(uA)} = u^n \det {(A)}\).
% 这是我讲过的例题.
特别地,
\(\det {(-A)} = (-1)^n \det {(A)}\).

(6)
设 \(A\) 是一个 \(n\)~级反称阵 (\(n \geq 3\)).
设 \(r\) 是不超过 \(n\) 的正整数.
设 \(s\), \(t\) 是二个不超过 \(n\) 的正整数,
且 \(s \neq r\), \(t \neq r\).
则
\(A(r,t|r,s) = -(A(r,s|r,t))^{\mathrm{T}}\).
并且, \(A(r,s|r,s)\)
是一个 \(n-2\)~级反称阵.

(7)
每一个适合条件
``\(i\), \(j\) 都是不低于 \(p\), 且不低于 \(q\) 的整数''
的有序对 \((i, j)\) \emph{恰}适合以下三个条件之一:
(a)
\(p \leq i = j \leq q\);
(b)
\(p \leq i < j \leq q\);
(c)
\(p \leq j < i \leq q\).

(8)
任取一个正奇数 \(n\), 一定存在一个正整数 \(k\)
使 \(n = 2k - 1\).

\vspace{2ex}

介绍完这几件事后, 我总算可以证明定理了.

\begin{proof}
    记命题 \(P(k)\) 为
    \begin{quotation}
        每一个 \(2k - 1\)~级反称阵的行列式都是 \(0\).
    \end{quotation}
    我们用数学归纳法证明,
    对任何正整数~\(k\),
    \(P(k)\) 是正确的.

    不难验证 \(P(1)\) 是正确的.

    现在, 我们假定 \(P(k-1)\) 是正确的 (\(k \geq 2\)).
    我们要证 \(P(k)\) 也是正确的.
    记 \(n = 2k-1\).
    设 \(A\) 是一个 \(n\)~级反称阵.
    那么
    \begin{align*}
             & \det {(A)}     \\
        = {} &
        \sum_{i, j = 2}^{n} {
        (-1)^{i+j}
            [A]_{1,i} [A]_{1,j} \det {(A(1,i|1,j))}
        }
        \\
        = {} &
        \hphantom{{} + {}}
        \sum_{2 \leq i = j \leq n} {
            (-1)^{i+j}
                [A]_{1,i} [A]_{1,j} \det {(A(1,i|1,j))}
        }
        \\
             & +
        \sum_{2 \leq i < j \leq n} {
            (-1)^{i+j}
                [A]_{1,i} [A]_{1,j} \det {(A(1,i|1,j))}
        }
        \\
             & +
        \sum_{2 \leq j < i \leq n} {
            (-1)^{i+j}
                [A]_{1,i} [A]_{1,j} \det {(A(1,i|1,j))}
        }
        \\
        = {} &
        \hphantom{{} + {}}
        \sum_{2 \leq i \leq n} {
            (-1)^{i+i}
                [A]_{1,i} [A]_{1,i} \det {(A(1,i|1,i))}
        }
        \\
             & +
        \sum_{2 \leq i < j \leq n} {
            (-1)^{i+j}
                [A]_{1,i} [A]_{1,j} \det {(A(1,i|1,j))}
        }
        \\
             & +
        \sum_{2 \leq i < j \leq n} {
            (-1)^{j+i}
                [A]_{1,j} [A]_{1,i} \det {(A(1,j|1,i))}
        }
        \\
        = {} &
        \hphantom{{} + {}}
        \sum_{2 \leq i \leq n} {
            [A]_{1,i}^2 \det {(A(1,i|1,i))}
        }
        \\
             & +
        \sum_{2 \leq i < j \leq n} {
            (-1)^{i+j}
                [A]_{1,i} [A]_{1,j} \det {(A(1,i|1,j))}
        }
        \\
             & +
        \sum_{2 \leq i < j \leq n} {
            (-1)^{i+j}
                [A]_{1,i} [A]_{1,j}
            \det {(-(A(1,i|1,j))^{\mathrm{T}})}
        }
        \\
        = {} &
        \hphantom{{} + {}}
        \sum_{2 \leq i \leq n} {
            [A]_{1,i}^2 \det {(A(1,i|1,i))}
        }
        \\
             & +
        \sum_{2 \leq i < j \leq n} {
            (-1)^{i+j}
                [A]_{1,i} [A]_{1,j} \det {(A(1,i|1,j))}
        }
        \\
             & +
        \sum_{2 \leq i < j \leq n} {
            (-1)^{i+j}
                [A]_{1,i} [A]_{1,j}
            (-1)^{n-2} \det {((A(1,i|1,j))^{\mathrm{T}})}
        }
        \\
        = {} &
        \hphantom{{} + {}}
        \sum_{2 \leq i \leq n} {
            [A]_{1,i}^2 \det {(A(1,i|1,i))}
        }
        \\
             & +
        \sum_{2 \leq i < j \leq n} {
            (-1)^{i+j}
                [A]_{1,i} [A]_{1,j} \det {(A(1,i|1,j))}
        }
        \\
             & +
        \sum_{2 \leq i < j \leq n} {
            (-1)^{i+j}
                [A]_{1,i} [A]_{1,j}
            (-1)^{n} \det {(A(1,i|1,j))}
        }
        \\
        = {} &
        \hphantom{{} + {}}
        \sum_{2 \leq i \leq n} {
            [A]_{1,i}^2 \det {(A(1,i|1,i))}
        }
        \\
             & + (1 + (-1)^n)
        \sum_{2 \leq i < j \leq n} {
            (-1)^{i+j}
                [A]_{1,i} [A]_{1,j} \det {(A(1,i|1,j))}
        }.
    \end{align*}
    注意到 \(A(1,i|1,i)\) 是
    \(n - 2\)~级,
    即 \(2(k-1) - 1\)~级反称阵;
    由假定,
    其行列式为 \(0\).
    再注意到 \(n = 2k - 1\),
    故 \(1 + (-1)^n = 0\).
    所以, \(\det {(A)} = 0\).

    所以, \(P(k)\) 是正确的.
    由数学归纳法原理, 待证命题成立.
\end{proof}

我们还不了解偶数级反称阵的行列式.
不过, 我们之后会了解它的.

\section{阵的积与倍加}

前面, 在研究行列式的性质时, 我们引入了一种叫 ``倍加'' 的行为:

\DefinitionMultiplyAndAdd*

我们说, 有列的倍加, 也有行的倍加.
具体地, 加 \(A\)~的一行的倍于另一行,
且不改变其他的行, 是一次行的倍加.

回想起, 行列式有倍加不变性.
具体地, 若我们加方阵 \(A\)~的一列 (行) 的倍于另一列 (行),
且不改变其他的列 (行), 得方阵 \(B\),
则 \(\det {(B)} = \det {(A)}\).

本节, 我们讨论阵的积与倍加的关系.

设 \(A\) 是 \(m \times n\) 阵.
设 \(A\) 的%
列~\(1\), \(2\), \(\dots\), \(n\) 分别是
\(a_1\), \(a_2\), \(\dots\), \(a_n\).
则 \(A = [a_1, a_2, \dots, a_n]\).
再设 \(n\)~级单位阵 \(I_n\) 的%
列~\(1\), \(2\), \(\dots\), \(n\) 分别是
\(e_1\), \(e_2\), \(\dots\), \(e_n\).
则 \(I_n = [e_1, e_2, \dots, e_n]\).
由 \(A I_n = A\), 知
\begin{align*}
    [Ae_1, Ae_2, \dots, Ae_n]
    = A[e_1, e_2, \dots, e_n]
    = [a_1, a_2, \dots, a_n].
\end{align*}
故 \(Ae_j = a_j\),
对任何不超过 \(n\) 的正整数 \(j\).

设 \(p \neq q\).
我们加 \(A\) 的列~\(p\) 的 \(s\)~倍于列~\(q\),
不改变其他的列, 得 \(n\)~级阵 \(B\).
设 \(B\) 的%
列~\(1\), \(2\), \(\dots\), \(n\) 分别是
\(b_1\), \(b_2\), \(\dots\), \(b_n\).
则 \(B = [b_1, b_2, \dots, b_n]\).
则 \(j \neq q\) 时, 有 \(b_j = a_j\),
且 \(b_q = a_q + s a_p\).
则 \(j \neq q\) 时,
\begin{align*}
    B e_j = b_j = a_j = A e_j,
\end{align*}
且
\begin{align*}
    B e_q = b_q = a_q + s a_p = A e_q + s (A e_p) = A (e_q + s e_p).
\end{align*}
作 \(n\)~级阵
\(E(n; p, q; s) = [f_1, f_2, \dots, f_n]\),
其中
\begin{align*}
    f_j
    = \begin{cases}
          e_j,         & j \neq q; \\
          e_q + s e_p, & j = q.
      \end{cases}
\end{align*}
则 \(B e_j = A f_j\),
对任何不超过 \(n\) 的正整数 \(j\).
则
\begin{align*}
    B = BI_n = [B e_1, B e_2, \dots, B e_n]
    = [A f_1, A f_2, \dots, A f_n]
    = A E(n; p, q; s).
\end{align*}
不难写出
\begin{align*}
    [E(n; p, q; s)]_{i,j}
    = \begin{cases}
          s,     & \text{\(i = p\), 且 \(j = q\)}; \\
          [I_n], & \text{其他}.
      \end{cases}
\end{align*}
于是, 我们有

\begin{theorem}
    设 \(A\) 是 \(m \times n\)~阵.
    设 \(p\), \(q\) 是不超过 \(n\)~的正整数,
    且 \(p \neq q\).
    设 \(s\) 是数.
    作 \(n\)~级阵 \(E(n; p, q; s)\) 如下:
    \begin{align*}
        [E(n; p, q; s)]_{i,j}
        = \begin{cases}
              s,     & \text{\(i = p\), 且 \(j = q\)}; \\
              [I_n], & \text{其他}.
          \end{cases}
    \end{align*}
    (通俗地,
    加 \(I_n\) 的列~\(p\) 的 \(s\)~倍于列~\(q\),
    不改变其他的列, 得 \(n\)~级阵 \(E(n; p, q; s)\).)
    那么,
    加 \(A\) 的列~\(p\) 的 \(s\)~倍于列~\(q\),
    不改变其他的列, 得 \(m \times n\)~阵 \(A E(n; p, q; s)\).
\end{theorem}

\begin{proof}
    前面的说明就是一个证明.
    当然, 不考虑前面的说明, 我们也可直接用阵的积的定义验证它.
    毕竟, 我们的目标是
    \begin{equation*}
        [A E(n; p, q; s)]_{i,j}
        = \begin{cases}
            [A]_{i,j},              & j \neq q; \\
            [A]_{i,q} + s[A]_{i,p}, & j = q.
        \end{cases}
        \qedhere
    \end{equation*}
\end{proof}

列的倍加可用阵的积实现.
行的倍加当然也可用阵的积实现.
具体地,

\begin{theorem}
    设 \(A\) 是 \(m \times n\)~阵.
    设 \(p\), \(q\) 是不超过 \(m\)~的正整数,
    且 \(p \neq q\).
    设 \(s\) 是数.
    作 \(m\)~级阵 \(E(m; q, p; s)\) 如下:
    \begin{align*}
        [E(m; q, p; s)]_{i,j}
        = \begin{cases}
              s,     & \text{\(i = q\), 且 \(j = p\)}; \\
              [I_m], & \text{其他}.
          \end{cases}
    \end{align*}
    (通俗地,
    加 \(I_m\) 的行~\(p\) 的 \(s\)~倍于行~\(q\),
    不改变其他的行, 得 \(m\)~级阵 \(E(m; q, p; s)\).)
    那么,
    加 \(A\) 的列~\(p\) 的 \(s\)~倍于列~\(q\),
    不改变其他的列, 得 \(m \times n\)~阵 \(E(m; q, p; s) A\).
\end{theorem}

\begin{proof}
    请允许我留此事为您的习题.
    当然, 我想给您一个提示:
    证明此事的要点是验证
    \begin{equation*}
        [E(m; q, p; s) A]_{i,j}
        = \begin{cases}
            [A]_{i,j},              & i \neq q; \\
            [A]_{q,j} + s[A]_{p,j}, & i = q.
        \end{cases}
        \qedhere
    \end{equation*}
\end{proof}

我再说几件事.
设 \(p\), \(q\) 是不超过 \(n\) 的正整数,
且 \(p \neq q\).

不难看出, \(E(n; p, q; s)\) 的行列式为 \(1\):
毕竟, 这是对单位阵作一次倍加后得到的方阵,
且单位阵的行列式为 \(1\).

不难验证, \(E(n; p, q; s)\) 的转置是
\(E(n; q, p; s)\);
对比这二个阵的元即可.

最后, 我说, 用阵的积表示倍加是好的.
我们知道, 阵的积适合一些运算律;
于是, 我们或可用阵的运算律发现倍加的某些规律.
反过来, 我们也或可用倍加发现阵的一些性质.

\section{反称阵与倍加}

本节, 我们用倍加研究反称阵.

先看一个简单的结果.

\begin{theorem}
    设 \(A\) 是 \(n\)~级反称阵.
    设 \(p\), \(q\) 是不超过 \(n\)~的正整数,
    且 \(p \neq q\).
    设 \(s\) 是数.

    (1)
    加 \(A\) 的列~\(p\) 的 \(s\)~倍于列~\(q\),
    不改变其他的列, 得 \(n\)~级阵 \(B\).
    加 \(B\) 的行~\(p\) 的 \(s\)~倍于行~\(q\),
    不改变其他的行, 得 \(n\)~级阵 \(C\).
    则 \(C\) 是反称阵.
    % \begin{align*}
    %     [C]_{i,j} =
    %     \begin{cases}
    %         [A]_{q,j} + s [A]_{p,j},
    %          & \text{\(i = q\), 且 \(j \neq q\)}; \\
    %         [A]_{i,q} + s [A]_{i,p},
    %          & \text{\(j = q\), 且 \(i \neq q\)}; \\
    %         [A]_{i,j},
    %          & \text{其他}.
    %     \end{cases}
    % \end{align*}
    通俗地,
    对反称阵先作一次列的倍加,
    再作一次对应的行的倍加,
    则二次倍加后的阵仍是反称阵.

    (2)
    在 (1) 中, 先行后列不影响结果.
    具体地, 设
    加 \(A\) 的行~\(p\) 的 \(s\)~倍于行~\(q\),
    不改变其他的行, 得 \(n\)~级阵 \(F\).
    加 \(F\) 的列~\(p\) 的 \(s\)~倍于列~\(q\),
    不改变其他的列, 得 \(n\)~级阵 \(G\).
    则 \(G = C\).
\end{theorem}

\begin{proof}
    (1)
    我用二种方法证明此事.

    法~1:
    设加 \(A\) 的列~\(p\) 的 \(s\)~倍于列~\(q\),
    不改变其他的列, 得 \(n\)~级阵 \(B\).
    则
    \begin{align*}
        [B]_{i,j}
        = \begin{cases}
              [A]_{i,j},              & j \neq q; \\
              [A]_{i,q} + s[A]_{i,p}, & j = q.
          \end{cases}
    \end{align*}
    设加 \(B\) 的行~\(p\) 的 \(s\)~倍于行~\(q\),
    不改变其他的行, 得 \(n\)~级阵 \(C\).
    则
    \begin{align*}
        [C]_{i,j}
        = \begin{cases}
              [B]_{i,j},              & i \neq q; \\
              [B]_{q,j} + s[B]_{p,j}, & i = q.
          \end{cases}
    \end{align*}
    当 \(i = q\), 且 \(j \neq q\) 时,
    \begin{align*}
        [C]_{i,j}
            = [B]_{q,j} + s[B]_{p,j}
            = [A]_{q,j} + s[A]_{p,j};
    \end{align*}
    当 \(j = q\), 且 \(i \neq q\) 时,
    \begin{align*}
        [C]_{i,j}
            = [B]_{i,q}
            = [A]_{i,q} + s[A]_{i,p};
    \end{align*}
    当 \(i = q = j\) 时,
    \begin{align*}
        [C]_{i,j}
        = {} &
        [B]_{q,q} + s[B]_{p,q}
        \\
        = {} &
        ([A]_{q,q} + s[A]_{q,p}) + s([A]_{p,q} + s[A]_{p,p})
        \\
        = {} &
        [A]_{q,q} + s([A]_{q,p} + s[A]_{p,q}) + s^2 [A]_{p,p}
        \\
        = {} &
        0 + s 0 + s^2 0
        \\
        = {} &
        0
        \\
        = {} &
        [A]_{i,j};
    \end{align*}
    当 \(i \neq q\), 且 \(j \neq q\) 时,
    \begin{align*}
        [C]_{i,j} = [B]_{i,j} = [A]_{i,j}.
    \end{align*}
    由此, 不难验证, \(C\) 是一个反称阵.

    法~2:
    设 \(u\), \(v\) 是不超过 \(n\) 的正整数, 且 \(u \neq v\).
    作 \(n\)~级阵 \(E(n; u, v; s)\) 如下:
    \begin{align*}
        [E(n; u, v; s)]_{i,j}
        = \begin{cases}
              s,     & \text{\(i = u\), 且 \(j = v\)}; \\
              [I_n], & \text{其他}.
          \end{cases}
    \end{align*}
    于是, 加 \(A\) 的列~\(p\) 的 \(s\)~倍于列~\(q\),
    不改变其他的列, 得 \(n\)~级阵 \(B = A E(n; p, q; s)\).
    进一步地,
    加 \(B\) 的行~\(p\) 的 \(s\)~倍于行~\(q\),
    不改变其他的行, 得 \(n\)~级阵 \(C = E(n; q, p; s) B\).
    则
    \begin{align*}
        C = E(n; q, p; s) (A E(n; p, q; s))
        = (E(n; p, q; s))^{\mathrm{T}} A E(n; p, q; s).
    \end{align*}
    故 \(C\) 是一个反称阵.

    注意到, 法~2 利用了已有的阵的运算律:
    法~1 是较直接的; 法~2 是较聪明的.

    (2)
    不难写出,
    \begin{align*}
        F = E(n; q, p; s) A = (E(n; p, q; s))^{\mathrm{T}} A,
    \end{align*}
    且
    \begin{align*}
        G = F E(n; p, q; s)
        = ((E(n; p, q; s))^{\mathrm{T}} A) E(n; p, q; s).
    \end{align*}
    因为结合律, 我们有
    \begin{equation*}
        G
        = (E(n; p, q; s))^{\mathrm{T}} (A E(n; p, q; s))
        = C.
        \qedhere
    \end{equation*}
\end{proof}

利用此事, 我们可证明如下重要的定理.

\begin{theorem}
    设 \(A\) 是 \(n\)~级反称阵.
    利用若干次列的倍加, 与对应的行的倍加,
    我们可变 \(A\) 为形如
    \begin{equation}
        \begin{bmatrix}
            0      & b_1    & 0      & 0      & \cdots & 0      & 0      \\
            -b_1   & 0      & 0      & 0      & \cdots & 0      & 0      \\
            0      & 0      & 0      & b_2    & \cdots & 0      & 0      \\
            0      & 0      & -b_2   & 0      & \cdots & 0      & 0      \\
            \vdots & \vdots & \vdots & \vdots & \ddots & \vdots & \vdots \\
            0      & 0      & 0      & 0      & \cdots & 0      & b_m    \\
            0      & 0      & 0      & 0      & \cdots & -b_m   & 0      \\
        \end{bmatrix}
        \label{eq:C3201}
    \end{equation}
    (若 \(n\) 是偶数), 或
    \begin{equation}
        \begin{bmatrix}
            0      & b_1    & 0      & 0      & \cdots & 0      & 0      & 0      \\
            -b_1   & 0      & 0      & 0      & \cdots & 0      & 0      & 0      \\
            0      & 0      & 0      & b_2    & \cdots & 0      & 0      & 0      \\
            0      & 0      & -b_2   & 0      & \cdots & 0      & 0      & 0      \\
            \vdots & \vdots & \vdots & \vdots & \ddots & \vdots & \vdots & \vdots \\
            0      & 0      & 0      & 0      & \cdots & 0      & b_k    & 0      \\
            0      & 0      & 0      & 0      & \cdots & -b_k   & 0      & 0      \\
            0      & 0      & 0      & 0      & \cdots & 0      & 0      & 0      \\
        \end{bmatrix}
        \label{eq:C3202}
    \end{equation}
    (若 \(n\) 是奇数) 的反称阵.

    我们约定, 作倍加时, 我们先列后行, 交替地作.
    具体地, 我们先作一次列的倍加
    (比如, 加列~\(p\) 的 \(s\)~倍于列~\(q\),
    其中 \(p \neq q\)),
    然后立即作一次对应的行的倍加
    (加行~\(p\) 的 \(s\)~倍于行~\(q\)).
    然后再作一次列的, 且再作一次对应的行的 (若还有)
    \(\dots \dots\).

    当然, 若 \(n = 2\), 则 \(A\) 已形如
    \begin{align*}
        \begin{bmatrix}
            0  & a \\
            -a & 0 \\
        \end{bmatrix}
    \end{align*}
    (取式~\eqref{eq:C3201} 的 \(m\) 为 \(1\));
    若 \(n = 1\), 则 \(A\) 已形如
    \([0]\)
    (取式~\eqref{eq:C3202} 的 \(k\) 为 \(0\)).
\end{theorem}

\begin{proof}
    作命题 \(P(n)\):
    对任何 \(n\)~级反称阵 \(A\),
    存在若干次列的倍加, 与对应的行的倍加,
    变 \(A\) 为一个%
    形如式~\eqref{eq:C3201} (若 \(n\) 是偶数)
    或式~\eqref{eq:C3202} (若 \(n\) 是奇数)
    的反称阵.
    再作命题 \(Q(n)\):
    \(P(n-1)\) 与 \(P(n)\) 是对的.
    我们用数学归纳法证明:
    对任何高于 \(1\) 的整数 \(n\), \(Q(n)\) 是对的.

    \(Q(2)\) 是对的,
    因为 \(P(1)\) 与 \(P(2)\) 是对的;
    注意到,
    我们可加一列 (行) 的 \(0\)~倍于另一列 (行),
    从而 ``什么也不变''.

    我们设 \(Q(n-1)\) 是对的 (\(n \geq 3\)).
    则 \(P(n-2)\) 与 \(P(n-1)\) 是对的.
    我们要由此证明 \(Q(n)\) 是对的,
    即 \(P(n-1)\) 与 \(P(n)\) 是对的.
    \(P(n-1)\) 当然是对的, 由假定.
    所以, 我们由假定
    ``\(P(n-2)\) 与 \(P(n-1)\) 是对的''
    证 \(P(n)\) 是对的,
    即得 \(Q(n)\) 是对的.

    任取一个 \(n\)~级反称阵 \(A\).
    我们先说明:
    存在若干次列的倍加, 与对应的行的倍加,
    变 \(A\) 为一个反称阵 \(C\),
    其中
    \([C]_{1,j} = [C]_{2,j} = 0\),
    \([C]_{j,1} = [C]_{j,2} = 0\),
    对任何高于 \(2\), 且不超过 \(n\) 的正整数 \(j\).

    若对任何高于 \(2\), 且不超过 \(n\) 的正整数 \(j\),
    已有
    \([A]_{1,j} = [A]_{2,j} = 0\),
    \([A]_{j,1} = [A]_{j,2} = 0\),
    我们 ``什么也不变'', 取 \(C = A\) 即可.

    若 \([A]_{1,2} \neq 0\),
    则 \([A]_{2,1}\) 当然也不是零.
    我们加 \(A\) 的%
    列~\(2\) 的 \(-[A]_{1,3}/[A]_{1,2}\) 倍于列~\(3\),
    且加行~\(2\) 的 \(-[A]_{3,1}/[A]_{2,1} = -[A]_{1,3}/[A]_{1,2}\)
    倍于行~\(3\), 得阵 \(F_3\).
    则 \(F_3\) 是一个反称阵,
    \([F_3]_{1,3} = 0\),
    \([F_3]_{3,1} = 0\),
    且 \([F_3]_{1,j} = [A]_{1,j}\),
    \([F_3]_{j,1} = [A]_{j,1}\)
    (\(j \neq 3\)).
    接着, 我们加 \(F_3\) 的%
    列~\(2\) 的 \(-[F_3]_{1,4}/[F_3]_{1,2}\) 倍于列~\(4\),
    且加行~\(2\) 的 \(-[F_3]_{4,1}/[F_3]_{2,1}
    = -[F_3]_{1,4}/[F_3]_{1,2}\)
    倍于行~\(4\), 得阵 \(F_4\).
    则 \(F_4\) 是一个反称阵,
    \([F_4]_{1,3} = [F_4]_{1,4} = 0\),
    \([F_4]_{3,1} = [F_4]_{4,1} = 0\),
    且 \([F_4]_{1,j} = [F_3]_{1,j}\),
    \([F_4]_{j,1} = [F_3]_{j,1}\)
    (\(j \neq 4\)).
    \(\dots \dots\)
    接着, 我们加 \(F_{n-1}\) 的%
    列~\(2\) 的 \(-[F_{n-1}]_{1,n}/[F_{n-1}]_{1,2}\) 倍于列~\(n\),
    且加行~\(2\) 的 \(-[F_{n-1}]_{n,1}/[F_{n-1}]_{2,1}
    = -[F_{n-1}]_{1,n}/[F_{n-1}]_{1,2}\)
    倍于行~\(n\), 得阵 \(F_n\).
    则 \(F_n\) 是一个反称阵,
    \([F_n]_{1,j} = 0\),
    \([F_n]_{j,1} = 0\)
    (\(j = 3\), \(4\), \(\dots\)),
    且 \([F_n]_{1,2} = -[F_n]_{2,1} = [A]_{1,2} \neq 0\).
    然后, 我们加 \(F_n\) 的%
    列~\(1\) 的 \(-[F_n]_{2,3}/[F_n]_{2,1}\) 倍于列~\(3\),
    且加行~\(1\) 的 \(-[F_n]_{3,2}/[F_n]_{1,2} = -[F_n]_{2,3}/[F_n]_{2,1}\)
    倍于行~\(3\), 得阵 \(G_3\).
    则 \(G_3\) 是一个反称阵,
    \([G_3]_{2,3} = 0\),
    \([G_3]_{3,2} = 0\),
    且 \([G_3]_{2,j} = [F_n]_{2,j}\),
    \([G_3]_{j,2} = [F_n]_{j,2}\)
    (\(j \neq 3\)).
    接着, 我们加 \(G_3\) 的%
    列~\(1\) 的 \(-[G_3]_{2,4}/[G_3]_{2,1}\) 倍于列~\(4\),
    且加行~\(1\) 的 \(-[G_3]_{4,2}/[G_3]_{1,2} = -[G_3]_{2,4}/[G_3]_{2,1}\)
    倍于行~\(4\), 得阵 \(G_4\).
    则 \(G_4\) 是一个反称阵,
    \([G_4]_{2,3} = [G_4]_{2,4} = 0\),
    \([G_4]_{3,2} = [G_4]_{4,2} = 0\),
    且 \([G_4]_{2,j} = [F_n]_{2,j}\),
    \([G_4]_{j,2} = [F_n]_{j,2}\)
    (\(j \neq 4\)).
    \(\dots \dots\)
    最后, 我们加 \(G_{n-1}\) 的%
    列~\(1\) 的 \(-[G_{n-1}]_{2,n}/[G_{n-1}]_{2,1}\) 倍于列~\(n\),
    且加行~\(1\) 的 \(-[G_{n-1}]_{n,2}/[G_{n-1}]_{1,2}
    = -[G_{n-1}]_{2,n}/[G_{n-1}]_{2,1}\)
    倍于行~\(n\), 得阵 \(G_n\).
    则 \(G_n\) 是一个反称阵,
    \([G_n]_{1,j} = 0\),
    \([G_n]_{j,1} = 0\),
    \([G_n]_{2,j} = 0\),
    \([G_n]_{j,2} = 0\),
    (\(j = 3\), \(4\), \(\dots\)).
    取 \(C\) 为 \(G_n\) 即可.

    若 \([A]_{1,2} = -[A]_{2,1} = 0\),
    但有某个 \([A]_{\ell,j} \neq 0\)
    (\(\ell = 1\) 或 \(2\),
    且 \(j > 2\)),
    我们可加 \(A\) 的列~\(j\) 于列~\(3 - \ell\),
    且加行~\(j\) 于行~\(3 - \ell\)
    (注意到, 当 \(\ell = 1\) 或 \(2\) 时,
    \(3 - \ell = 2\) 或 \(1\), 分别地),
    得阵 \(D\).
    则 \([D]_{\ell,3 - \ell} \neq 0\),
    这就转化问题为前面讨论过的情形.

    综上, 作若干次列的倍加, 与对应的行的倍加,
    我们可变 \(A\) 为一个反称阵 \(C\),
    其中
    \([C]_{1,j} = [C]_{2,j} = 0\),
    \([C]_{j,1} = [C]_{j,2} = 0\),
    对任何高于 \(2\), 且不超过 \(n\) 的正整数 \(j\).

    考虑 \(C\) 的右下角的 \(n-2\)~级子阵 \(C({1,2}|{1,2})\).
    不难看出, 它是一个 \(n-2\)~级反称阵.
    由假定,
    作若干次列的倍加, 与对应的行的倍加,
    我们可变 \(C({1,2}|{1,2})\) 为一个反称阵 \(H\),
    其中 \(H\) 形如式~\eqref{eq:C3201} (若 \(n-2\) 是偶数)
    或式~\eqref{eq:C3202} (若 \(n-2\) 是奇数).

    注意到, 既然当 \(2 < j\) 时,
    \([C]_{1,j} = [C]_{2,j} = 0\),
    \([C]_{j,1} = [C]_{j,2} = 0\),
    那么,
    无论如何对 \(C\) 的不是列~\(1\) 或列~\(2\) 的列作倍加,
    也无论如何对 \(C\) 的不是行~\(1\) 或行~\(2\) 的行作倍加,
    所得的阵的
    \((1, j)\)-元, \((2, j)\)-元, \((j, 1)\)-元, \((j, 2)\)-元,
    一定是零.
    所以, 作若干次列的倍加, 与对应的行的倍加后,
    我们可变 \(C\) 为一个 \(n\)~级反称阵 \(B\),
    使
    \begin{align*}
        [B]_{i,j}
        = \begin{cases}
              [H]_{i-2,j-2}, & \text{\(i > 2\), 且 \(j > 2\)}; \\
              [A]_{1,2},     & \text{\(i = 1\), 且 \(j = 2\)}; \\
              [A]_{2,1},     & \text{\(i = 2\), 且 \(j = 1\)}; \\
              0,             & \text{其他}.
          \end{cases}
    \end{align*}
    于是, \(B\) 是%
    形如式~\eqref{eq:C3201} (若 \(n\) 是偶数)
    或式~\eqref{eq:C3202} (若 \(n\) 是奇数)
    的反称阵.

    所以, \(P(n)\) 是正确的.
    则 \(Q(n)\) 是正确的.
    由数学归纳法原理, 待证命题成立.
\end{proof}

由倍加与阵的积的关系, 我们有:

\begin{theorem}
    设 \(A\) 是 \(n\)~级反称阵.
    则存在若干个形如 \(E(n; p, q; s)\)
    (\(s\) 是一个数;
    \(p\), \(q\) 是不超过 \(n\) 的正整数,
    \(p \neq q\))
    的阵
    \(E_1\), \(E_2\), \(\dots\), \(E_w\),
    使 \(V^{\mathrm{T}} AV\) 是%
    形如式~\eqref{eq:C3201} (若 \(n\) 是偶数)
    或式~\eqref{eq:C3202} (若 \(n\) 是奇数)
    的反称阵,
    其中 \(V = E_1 E_2 \dots E_w\).
\end{theorem}

\begin{proof}
    由上个定理, 存在若干个形如 \(E(n; p, q; s)\)
    (\(s\) 是一个数;
    \(p\), \(q\) 是不超过 \(n\) 的正整数,
    \(p \neq q\))
    的阵
    \(E_1\), \(E_2\), \(\dots\), \(E_w\),
    使
    \begin{align*}
        E_w^{\mathrm{T}}
        (E_{w-1}^{\mathrm{T}}
        \dots
        (E_2^{\mathrm{T}}
        (E_1^{\mathrm{T}}
        A
        E_1)
        E_2)
        \dots
        E_{w-1})
        E_w
    \end{align*}
    是%
    形如式~\eqref{eq:C3201} (若 \(n\) 是偶数)
    或式~\eqref{eq:C3202} (若 \(n\) 是奇数)
    的反称阵.
    由结合律, 上式相当于
    \begin{align*}
        (E_w^{\mathrm{T}}
        E_{w-1}^{\mathrm{T}}
        \dots
        E_2^{\mathrm{T}}
        E_1^{\mathrm{T}})
        A
        (E_1
        E_2
        \dots
        E_{w-1}
        E_w).
    \end{align*}
    记 \(V = E_1 E_2 \dots E_{w-1} E_w\).
    由转置的性质,
    \(V^{\mathrm{T}}
    = E_w^{\mathrm{T}}
    E_{w-1}^{\mathrm{T}}
    \dots
    E_2^{\mathrm{T}}
    E_1^{\mathrm{T}}\).
    故上式相当于 \(V^{\mathrm{T}} A V\).
\end{proof}

我们计算如式~\eqref{eq:C3201} 所示的反称阵的行列式.
记
\begin{align*}
    B_m =
    \begin{bmatrix}
        0      & b_1    & 0      & 0      & \cdots & 0      & 0      \\
        -b_1   & 0      & 0      & 0      & \cdots & 0      & 0      \\
        0      & 0      & 0      & b_2    & \cdots & 0      & 0      \\
        0      & 0      & -b_2   & 0      & \cdots & 0      & 0      \\
        \vdots & \vdots & \vdots & \vdots & \ddots & \vdots & \vdots \\
        0      & 0      & 0      & 0      & \cdots & 0      & b_m    \\
        0      & 0      & 0      & 0      & \cdots & -b_m   & 0      \\
    \end{bmatrix}.
\end{align*}
按列~\(2m\) 展开, 有
\begin{align*}
    \det {(B_m)}
    = {} &
    (-1)^{2m-1+2m} [B_m]_{2m-1,2m} \det {(B_m ({2m-1}|{2m}))}
    \\
    = {} &
    {-b_m} \det {
        \begin{bmatrix}
            B_{m-1} & 0    \\
            0       & -b_m \\
        \end{bmatrix}
    }.
\end{align*}
再按行~\(2m-1\) 展开 \(\det {(B_m ({2m-1}|{2m}))}\), 有
\begin{align*}
    \det {(B_m ({2m-1}|{2m}))}
    = {} & (-1)^{2m-1+2m-1} (-b_m)
    \det {(B_{m-1})}                \\
    = {} & {-b_m} \det {(B_{m-1})}.
\end{align*}
则 \(\det {(B_m)} = \det {(B_{m-1})}\, b_m^2\),
当 \(m > 1\).
不难算出 \(\det {(B_1)} = b_1^2\).
则
\begin{align*}
    \det {(B_m)} = b_1^2 b_2^2 \dots b_m^2
    = (b_1 b_2 \dots b_m)^2.
\end{align*}

还有一件事值得一提.
注意到, 倍加不改变行列式.
由此, 我们立得, 奇数级反称阵的行列式为 \(0\):
毕竟, 利用倍加, 我们可变一个奇数级反称阵为%
一个至少有一列的元全为零的阵.

\section{Pfaffian}

我们说, 行列式是方阵的一个重要的属性.
反称阵是方阵, 故行列式也是反称阵的一个重要的属性.
本节, 我们学习反称阵的另一个属性.
它也是重要的, 且与行列式有关.

\begin{definition}[pfaffian]
    设 \(A\) 是 \(n\)~级\emph{反称阵}.
    定义 \(A\)~的 \emph{pfaffian} 为
    \begin{align*}
        \operatorname{pf} {(A)} =
        \begin{dcases}
            0,
             & n = 1;    \\
            [A]_{1,2},
             & n = 2;    \\
            \sum_{j = 2}^{n}
            {(-1)^{j} [A]_{1,j}
            \operatorname{pf} {(A({1,j}|{1,j}))}},
             & n \geq 3.
        \end{dcases}
    \end{align*}
\end{definition}

% 据说, 这是英国数学家 Arthur Cayley
% 用德国数学家 Johann Friedrich Pfaff 的姓%
% 命名的概念.

注意, 我们只对反称阵定义 pfaffian.

我们计算不高于 \(4\)~级的反称阵的 pfaffian.

\begin{example}
    设 \(A\) 是 \(1\)~级反称阵.
    则 \(\operatorname{pf} {(A)} = 0\).

    回想起, \(\det {(A)}\) 也是零.
\end{example}

\begin{example}
    设 \(A\) 是 \(2\)~级反称阵.
    则 \(\operatorname{pf} {(A)} = [A]_{1,2}\).

    回想起, \(\det {(A)} = [A]_{1,2}^2\);
    于是, \(\det {(A)} = (\operatorname{pf} {(A)})^2\).
\end{example}

\begin{example}
    设 \(A\) 是 \(3\)~级反称阵.
    则
    \begin{align*}
        \operatorname{pf} {(A)}
        = {} &
        (-1)^2 [A]_{1,2} \operatorname{pf} {(A({1,2}|{1,2}))}
        +
        (-1)^3 [A]_{1,3} \operatorname{pf} {(A({1,3}|{1,3}))}
        \\
        = {} &
        [A]_{1,2} 0 - [A]_{1,3} 0
        \\
        = {} &
        0.
    \end{align*}

    回想起, \(\det {(A)}\) 也是零.
\end{example}

\begin{example}
    设 \(A\) 是 \(4\)~级反称阵.
    则
    \begin{align*}
        \operatorname{pf} {(A)}
        = {} &
        \hphantom{{} + {}}
        (-1)^2 [A]_{1,2} \operatorname{pf} {(A({1,2}|{1,2}))}
        +
        (-1)^3 [A]_{1,3} \operatorname{pf} {(A({1,3}|{1,3}))}
        \\
             &
        +
        (-1)^4 [A]_{1,4} \operatorname{pf} {(A({1,4}|{1,4}))}
        \\
        = {} &
        [A]_{1,2} [A]_{3,4}
        - [A]_{1,3} [A]_{2,4}
        + [A]_{1,4} [A]_{2,3}.
    \end{align*}

    回想起, \(\det {(A)} =
    ([A]_{1,2} [A]_{3,4}
    - [A]_{1,3} [A]_{2,4}
    + [A]_{1,4} [A]_{2,3})^2\);
    于是, \(\det {(A)} = (\operatorname{pf} {(A)})^2\).
\end{example}

看来, 对不高于 \(4\)~级的反称阵 \(A\),
我们有 \(\det {(A)} = (\operatorname{pf} {(A)})^2\).
我们会说明, 这对任何级的反称阵 \(A\) 都是对的.
为此, 我们会证明 pfaffian 的一些性质,
再利用这些性质解决此事.

最后, 我们再看一些较特别的阵的 pfaffian 吧.

\begin{example}
    作 \(2m\)~级反称阵
    \begin{align*}
        B =
        \begin{bmatrix}
            0      & b_1    & 0      & 0      & \cdots & 0      & 0      \\
            -b_1   & 0      & 0      & 0      & \cdots & 0      & 0      \\
            0      & 0      & 0      & b_2    & \cdots & 0      & 0      \\
            0      & 0      & -b_2   & 0      & \cdots & 0      & 0      \\
            \vdots & \vdots & \vdots & \vdots & \ddots & \vdots & \vdots \\
            0      & 0      & 0      & 0      & \cdots & 0      & b_m    \\
            0      & 0      & 0      & 0      & \cdots & -b_m   & 0      \\
        \end{bmatrix}.
    \end{align*}
    我们计算 \(\operatorname{pf} (B)\).
    由 pfaffian 的定义,
    \begin{align*}
        \operatorname{pf} {(B)}
        = (-1)^2 [B]_{1,2} \operatorname{pf} {(B({1,2}|{1,2}))}
        = b_1 \operatorname{pf} {(B({1,2}|{1,2}))}
    \end{align*}
    (注意到, 对任何 \(j > 2\), 有 \([B]_{1,j} = 0\)).
    不难看出,
    \begin{align*}
        B({1,2}|{1,2}) =
        \begin{bmatrix}
            0      & b_2    & \cdots & 0      & 0      \\
            -b_2   & 0      & \cdots & 0      & 0      \\
            \vdots & \vdots & \ddots & \vdots & \vdots \\
            0      & 0      & \cdots & 0      & b_m    \\
            0      & 0      & \cdots & -b_m   & 0      \\
        \end{bmatrix}.
    \end{align*}
    则, 类似地,
    \begin{align*}
        \operatorname{pf} {(B({1,2}|{1,2}))}
        = b_2 \operatorname{pf} {(B({1,2,3,4}|{1,2,3,4}))}.
    \end{align*}
    故
    \begin{align*}
        \operatorname{pf} {(B)}
        = b_1 b_2 \operatorname{pf} {(B({1,2,3,4}|{1,2,3,4}))}.
    \end{align*}
    \(\dots \dots\)
    最后, 我们算出
    \begin{align*}
        \operatorname{pf} {(B)}
        = b_1 b_2 \dots b_{m-1}
        \operatorname{pf} {(
        B({1,2,\dots,2m-2}|{1,2,\dots,2m-2})
        )}
        = b_1 b_2 \dots b_{m-1} b_m.
    \end{align*}
    回想起, \(\det {(B)} = (b_1 b_2 \dots b_m)^2\);
    于是, \(\det {(B)} = (\operatorname{pf} {(B)})^2\).
\end{example}

\begin{example}
    设 \(A\) 是 \(n\)~级反称阵.
    设 \(D\) 是 \(t\)~级反称阵.
    作 \(n + t\)~级阵
    \begin{align*}
        M =
        \begin{bmatrix}
            A & 0 \\
            0 & D \\
        \end{bmatrix}.
    \end{align*}
    不难验证, \(M\) 也是一个反称阵.
    我们用数学归纳法证明, \(\operatorname{pf} {(M)}
    = \operatorname{pf} {(A)} \operatorname{pf} {(D)}\).

    作命题 \(P(n)\):
    对任何 \(n\)~级反称阵 \(A\),
    \begin{align*}
        \operatorname{pf} {
            \begin{bmatrix}
                A & 0 \\
                0 & D \\
            \end{bmatrix}
        } = \operatorname{pf} {(A)} \operatorname{pf} {(D)}.
    \end{align*}
    再作命题 \(Q(n)\):
    \(P(n-1)\) 与 \(P(n)\) 是对的.
    我们用数学归纳法证明:
    对任何高于 \(1\) 的整数 \(n\), \(Q(n)\) 是对的.

    首先, \(P(1)\) 是对的, 因为 \(n = 1\)~时,
    \begin{align*}
        \begin{bmatrix}
            A & 0 \\
            0 & D \\
        \end{bmatrix}
    \end{align*}
    的行~\(1\) 的元全是 \(0\),
    故由 pfaffian 的定义, 此阵的 pfaffian 是 \(0\).

    然后, \(P(2)\) 是对的.
    记 \(J =
    \begin{bmatrix}
        A & 0 \\
        0 & D \\
    \end{bmatrix}
    \),
    其中 \(A\) 是 \(2\)~级反称阵.
    由 pfaffian 的定义,
    \begin{align*}
        \operatorname{pf} {(J)}
        = (-1)^{2} [J]_{1,2} \operatorname{pf} {(J({1,2}|{1,2}))}
        = [A]_{1,2} \operatorname{pf} {(D)}
        = \operatorname{pf} {(A)} \operatorname{pf} {(D)}.
    \end{align*}

    由此可见, \(Q(2)\) 是对的.

    我们设 \(Q(n-1)\) 是对的 (\(n \geq 3\)).
    则 \(P(n-2)\) 与 \(P(n-1)\) 是对的.
    我们要由此证明 \(Q(n)\) 是对的,
    即 \(P(n-1)\) 与 \(P(n)\) 是对的.
    \(P(n-1)\) 当然是对的, 由假定.
    所以, 我们由假定
    ``\(P(n-2)\) 与 \(P(n-1)\) 是对的''
    证 \(P(n)\) 是对的,
    即得 \(Q(n)\) 是对的.

    任取一个 \(n\)~级反称阵 \(A\).
    记 \(M =
    \begin{bmatrix}
        A & 0 \\
        0 & D \\
    \end{bmatrix}
    \).
    则
    \begin{align*}
             &
        \operatorname{pf} {(M)}
        \\
        = {} &
        \sum_{2 \leq j \leq n+t}
        {(-1)^{j} [M]_{1,j} \operatorname{pf} {(M({1,j}|{1,j}))}}
        \\
        = {} &
        \sum_{2 \leq j \leq n}
        {(-1)^{j} [M]_{1,j} \operatorname{pf} {(M({1,j}|{1,j}))}}
        +
        \sum_{n+1 \leq j \leq n+t}
        {(-1)^{j} [M]_{1,j} \operatorname{pf} {(M({1,j}|{1,j}))}}
        \\
        = {} &
        \sum_{2 \leq j \leq n}
        {(-1)^{j} [A]_{1,j} \operatorname{pf} {(M({1,j}|{1,j}))}}
        +
        \sum_{n+1 \leq j \leq n+t}
        {(-1)^{j} 0 \operatorname{pf} {(M({1,j}|{1,j}))}}
        \\
        = {} &
        \sum_{2 \leq j \leq n}
        {(-1)^{j} [A]_{1,j} \operatorname{pf} {(M({1,j}|{1,j}))}}.
    \end{align*}
    注意到, \(2 \leq j \leq n\) 时,
    \begin{align*}
        M({1,j}|{1,j})
        = \begin{bmatrix}
              A({1,j}|{1,j}) & 0 \\
              0              & D \\
          \end{bmatrix}.
    \end{align*}
    由假定, \(\operatorname{pf} {(M({1,j}|{1,j}))}
    = \operatorname{pf} {(A({1,j}|{1,j}))}
    \operatorname{pf} {(D)}\).
    从而
    \begin{align*}
        \operatorname{pf} {(M)}
        = {} &
        \sum_{2 \leq j \leq n}
        {(-1)^{j} [A]_{1,j} \operatorname{pf} {(M({1,j}|{1,j}))}}
        \\
        = {} &
        \sum_{2 \leq j \leq n}
        {(-1)^{j} [A]_{1,j}
        \operatorname{pf} {(A({1,j}|{1,j}))} \operatorname{pf} {(D)}
        }
        \\
        = {} &
        \left(
        \sum_{2 \leq j \leq n}
        {(-1)^{j} [A]_{1,j}
        \operatorname{pf} {(A({1,j}|{1,j}))}}
        \right) \operatorname{pf} {(D)}
        \\
        = {} &
        \operatorname{pf} {(A)} \operatorname{pf} {(D)}.
    \end{align*}

    所以, \(P(n)\) 是正确的.
    则 \(Q(n)\) 是正确的.
    由数学归纳法原理, 待证命题成立.

    回想起, 当 \(A\) 是 \(n\)~级阵,
    \(D\) 是 \(t\)~级阵,
    \(C\) 是 \(n \times t\)~阵时,
    \begin{align*}
        \det {
            \begin{bmatrix}
                A & C \\
                0 & D \\
            \end{bmatrix}
        }
        = \det {(A)} \det {(D)}.
    \end{align*}
    那么, 当 \(A\), \(D\) 是反称阵,
    且 \(C = 0\) 时,
    \(\det {(M)}\)
    当然也是 \(\det {(A)} \det {(D)}\).
    不过, 用现有的知识,
    我们还不知道这个 \(M\) 的行列式与 pfaffian 的关系.
\end{example}

在看最后一个例前, 我们先计算一类阵的行列式.
设 \(A\), \(D\) 分别是 \(n\)~即阵与 \(t\)~级阵.
设 \(C\) 是 \(n \times t\)~阵.
作 \(n + t\)~级阵
\begin{align*}
    P = \begin{bmatrix}
            C & A \\
            D & 0 \\
        \end{bmatrix}.
\end{align*}
我们计算 \(\det {(P)}\).
为此, 我们可用反称性.
具体地, 我们交换 \(P\) 的列~\(t+1\) 与列~\(t\),
得阵~\(P_t\).
则 \(\det {(P_t)} = (-1) \det {(P)}\).
再交换 \(P_t\) 的列~\(t\) 与列~\(t-1\),
得阵~\(P_{t-1}\).
则 \(\det {(P_{t-1})} = (-1) \det {(P_t)}
= (-1)^2 \det {(P)}\).
\(\dots \dots\)
再交换 \(P_2\) 的列~\(2\) 与列~\(1\),
得阵~\(P_1\).
则 \(\det {(P_1)} = (-1) \det {(P_2)}
= (-1)^t \det {(P)}\).
注意到, 我们作了 \(t\) 次相邻的列的交换,
使 \(P\) 的列~\(t+1\) 到列~\(1\) 的位置,
且其他的列的相对位置不变.
类似地, 我们再分别对 \(P_1\)~的列~\(t+2\), \(\dots\), \(t+n\)
也相邻地向左移 \(t\)~次,
即可使 \(P\) 的列~\(t+1\), \(\dots\), \(t+n\)
分别到列~\(1\), \(\dots\), \(n\) 的位置,
变 \(P\) 为
\begin{align*}
    M = \begin{bmatrix}
            A & C \\
            0 & D \\
        \end{bmatrix}.
\end{align*}
我们一共作了 \(t + (n-1)t = nt\)~次列的交换.
故
\begin{align*}
    \det {
        \begin{bmatrix}
            C & A \\
            D & 0 \\
        \end{bmatrix}
    }
    = (-1)^{nt} \det {
        \begin{bmatrix}
            A & C \\
            0 & D \\
        \end{bmatrix}
    }
    = (-1)^{nt} \det {(A)} \det {(D)}.
\end{align*}
特别地, 若我们取 \(D = -A^{\mathrm{T}}\),
且 \(C = 0\),
则
\begin{align*}
    \det {
        \begin{bmatrix}
            0               & A \\
            -A^{\mathrm{T}} & 0 \\
        \end{bmatrix}
    }
    = {} &
    (-1)^{n^2} \det {(-A^{\mathrm{T}})} \det {(A)}
    \\
    = {} &
    (-1)^{n^2} (-1)^n \det {(A^{\mathrm{T}})} \det {(A)}
    \\
    = {} &
    (-1)^{n(n+1)} (\det {(A)})^2
    \\
    = {} &
    (\det {(A)})^2.
\end{align*}
(注意到, \(n(n+1)\) 一定是一个偶数.)

\begin{example}
    设 \(A\) 是一个 \(m\)~级阵.
    为方便, 我们记
    \begin{align*}
        S(A) = \begin{bmatrix}
                   0               & A \\
                   -A^{\mathrm{T}} & 0 \\
               \end{bmatrix}.
    \end{align*}
    不难看出, \(S(A)\) 是一个 \(2m\)~级反称阵.
    我们用数学归纳法证明,
    \(\operatorname{pf} {(A)} = (-1)^{m(m-1)/2} \det {(A)}\).
    这也说明, pfaffian 跟行列式有一定的联系.

    作命题 \(P(m)\):
    对任何 \(m\)~级阵 \(A\),
    \(\operatorname{pf} {(A)} = (-1)^{m(m-1)/2} \det {(A)}\).
    我们用数学归纳法证明:
    对任何正整数 \(m\), \(P(m)\) 是对的.

    \(P(1)\) 是对的.
    设 \(A = [a]\).
    则 \(S(A) = \begin{bmatrix}
        0  & a \\
        -a & 0 \\
    \end{bmatrix}\).
    由定义, \(\det {(A)} = a\),
    且
    \(\operatorname{pf} {(S(A))} = a = (-1)^{1(1-1)/2} \det {(A)}\).

    我们设 \(P(m-1)\) 是对的.
    我们要由此证明 \(P(m)\) 是对的.

    任取 \(m\)~级阵 \(A\).
    则
    \begin{align*}
        \operatorname{pf} {(S(A))}
        = {} &
        \sum_{j=2}^{2m}
        {(-1)^j [S(A)]_{1,j} \operatorname{pf} {((S(A))({1,j}|{1,j}))}}
        \\
        = {} &
        \hphantom{{} + {}}
        \sum_{2 \leq j \leq m}
        {(-1)^j [S(A)]_{1,j} \operatorname{pf} {((S(A))({1,j}|{1,j}))}}
        \\
             & +
        \sum_{m+1 \leq j \leq 2m}
        {(-1)^j [S(A)]_{1,j} \operatorname{pf} {((S(A))({1,j}|{1,j}))}}
        \\
        = {} &
        \hphantom{{} + {}}
        \sum_{2 \leq j \leq m}
        {(-1)^j 0 \operatorname{pf} {((S(A))({1,j}|{1,j}))}}
        \\
             & +
        \sum_{1 \leq k \leq m}
        {(-1)^{m+k} [S(A)]_{1,m+k}
        \operatorname{pf} {((S(A))({1,m+k}|{1,m+k}))}}
        \\
        = {} &
        \sum_{1 \leq k \leq m}
        {(-1)^{m+k} [A]_{1,k}
            \operatorname{pf} {(S(A(1|k)))}}
        \\
        = {} &
        \sum_{1 \leq k \leq m}
        {(-1)^{m-1} (-1)^{1+k} [A]_{1,k}
            \operatorname{pf} {(S(A(1|k)))}}
        \\
        = {} &
        \sum_{1 \leq k \leq m}
        {(-1)^{1+k} [A]_{1,k}
            (-1)^{m-1} \operatorname{pf} {(S(A(1|k)))}}.
    \end{align*}
    注意到, \(A(1|k)\) 是 \(m-1\)~级阵.
    由假定,
    \begin{align*}
        \operatorname{pf} {(S(A(1|k)))}
        = (-1)^{(m-1)(m-2)/2} \det {(A(1|k))}.
    \end{align*}
    则
    \begin{align*}
        \operatorname{pf} {(S(A))}
        = {} &
        \sum_{1 \leq k \leq m}
        {(-1)^{1+k} [A]_{1,k}
            (-1)^{m-1} \operatorname{pf} {(S(A(1|k)))}}
        \\
        = {} &
        \sum_{1 \leq k \leq m}
        {(-1)^{1+k} [A]_{1,k}
            (-1)^{m-1} (-1)^{(m-1)(m-2)/2} \det {(A(1|k))}}
        \\
        = {} &
        \sum_{1 \leq k \leq m}
        {(-1)^{1+k} [A]_{1,k}
            (-1)^{m(m-1)/2} \det {(A(1|k))}}
        \\
        = {} &
        (-1)^{m(m-1)/2}
        \sum_{1 \leq k \leq m}
        {(-1)^{1+k} [A]_{1,k}
            \det {(A(1|k))}}
        \\
        = {} &
        (-1)^{m(m-1)/2} \det {(A)}.
    \end{align*}

    所以, \(P(m)\) 是正确的.
    由数学归纳法原理, 待证命题成立.

    最后, 不难看出,
    \(\det {(S(A))} = (\det {(A)})^2
    = (\operatorname{pf} {(S(A))})^2\).
\end{example}

我们作一个小结.

\begin{theorem}
    (1)
    设 \(A\), \(D\) 是 \(n\)~级与 \(t\)~级反称阵.
    则
    \begin{align*}
        \operatorname{pf} {
            \begin{bmatrix}
                A & 0 \\
                0 & D
            \end{bmatrix}
        }
        = \operatorname{pf} {(A)} \operatorname{pf} {(D)}.
    \end{align*}

    (2)
    设 \(A\) 是 \(m\)~级阵.
    则
    \begin{align*}
        \operatorname{pf} {
            \begin{bmatrix}
                0               & A \\
                -A^{\mathrm{T}} & 0 \\
            \end{bmatrix}
        } = (-1)^{m(m-1)/2} \det {(A)}.
    \end{align*}
    特别地, 取 \(A = I_m\), 即知
    \(K_m = \begin{bmatrix}
        0    & I_m \\
        -I_m & 0   \\
    \end{bmatrix}\)
    的 pfaffian 是 \((-1)^{m(m-1)/2}\).

    (3)
    \begin{align*}
        \operatorname{pf} {
            \begin{bmatrix}
                0      & b_1    & 0      & 0      & \cdots & 0      & 0      \\
                -b_1   & 0      & 0      & 0      & \cdots & 0      & 0      \\
                0      & 0      & 0      & b_2    & \cdots & 0      & 0      \\
                0      & 0      & -b_2   & 0      & \cdots & 0      & 0      \\
                \vdots & \vdots & \vdots & \vdots & \ddots & \vdots & \vdots \\
                0      & 0      & 0      & 0      & \cdots & 0      & b_m    \\
                0      & 0      & 0      & 0      & \cdots & -b_m   & 0      \\
            \end{bmatrix}
        } = b_1 b_2 \dots b_m.
    \end{align*}
\end{theorem}

\section{Pfaffian 的性质}

本节, 我们讨论 pfaffian 的一些性质.

回想起, \(1\)~级反称阵与 \(3\)~级反称阵的 pfaffian 为 \(0\).
一般地,

\begin{theorem}
    奇数级反称阵的 pfaffian 为 \(0\).
\end{theorem}

\begin{proof}
    作命题 \(P(k)\):
    对任何 \(2k-1\)~级反称阵 \(A\),
    必 \(\operatorname{pf} {(A)} = 0\).
    我们用数学归纳法证明:
    对任何正整数 \(k\), \(P(k)\) 是对的.

    \(P(1)\) 不证自明.

    我们设 \(P(k-1)\) 是对的.
    我们要由此证明 \(P(k)\) 是对的.

    任取 \(2k-1\)~级反称阵 \(A\).
    则
    \begin{align*}
        \operatorname{pf} {(A)}
        = {} &
        \sum_{j = 2}^{2k-1}
        {(-1)^{j} [A]_{1,j}
        \operatorname{pf} {(A({1,j}|{1,j}))}}
        \\
        = {} &
        \sum_{j = 2}^{2k-1}
        {(-1)^{j} [A]_{1,j}\, 0}
        \\
        = {} &
        0.
    \end{align*}

    所以, \(P(k)\) 是正确的.
    由数学归纳法原理, 待证命题成立.
\end{proof}

在讨论较复杂的性质前, 我们先看一个简单的.

\begin{theorem}
    设 \(m\) 是整数.
    设 \(A\) 是 \(2m\)~级反称阵.
    设 \(x\) 是数.
    则 \(\operatorname{pf} {(xA)}
    = x^m \operatorname{pf} {(A)}\).
    特别地, \(A^{\mathrm{T}} = (-1)A\) 的 pfaffian
    是 \((-1)^m \operatorname{pf} {(A)}\).
\end{theorem}

\begin{proof}
    作命题 \(P(m)\):
    对任何 \(2m\)~级反称阵 \(A\),
    必 \(\operatorname{pf} {(xA)}
    = x^m \operatorname{pf} {(A)}\).
    我们用数学归纳法证明:
    对任何正整数 \(m\), \(P(m)\) 是对的.

    \(P(1)\) 是对的:
    \begin{align*}
        \operatorname{pf} {
            \begin{bmatrix}
                0   & xa \\
                -xa & 0  \\
            \end{bmatrix}
        }
        = xa
        = x^1 \operatorname{pf} {
            \begin{bmatrix}
                0  & a \\
                -a & 0 \\
            \end{bmatrix}
        }.
    \end{align*}

    我们设 \(P(m-1)\) 是对的.
    我们要由此证明 \(P(m)\) 是对的.

    任取 \(2m\)~级反称阵 \(A\).
    则
    \begin{align*}
        \operatorname{pf} {(xA)}
        = {} &
        \sum_{j = 2}^{2m}
        {(-1)^{j} [xA]_{1,j}
        \operatorname{pf} {((xA)({1,j}|{1,j}))}}
        \\
        = {} &
        \sum_{j = 2}^{2m}
        {(-1)^{j} x\, [A]_{1,j}
        \operatorname{pf} {(x\,A({1,j}|{1,j}))}}
        \\
        = {} &
        \sum_{j = 2}^{2m}
        {(-1)^{j} x\, [A]_{1,j}
        x^{m-1} \operatorname{pf} {(A({1,j}|{1,j}))}}
        \\
        = {} &
        x\,x^{m-1}
        \sum_{j = 2}^{2m}
        {(-1)^{j} [A]_{1,j}
        \operatorname{pf} {(A({1,j}|{1,j}))}}
        \\
        = {} &
        x^m \operatorname{pf} {(A)}.
    \end{align*}

    所以, \(P(m)\) 是正确的.
    由数学归纳法原理, 待证命题成立.
\end{proof}

\begin{theorem}
    设 \(A\), \(B\), \(C\) 是三个 \(n\)~级反称阵.
    设 \(q\) 是不超过 \(n\) 的正整数.
    设 \(s\), \(t\) 是数.
    设 \([A]_{i,j} = [B]_{i,j} = [C]_{i,j}\),
    对任何不等于 \(q\), 且不超过 \(n\) 的正整数 \(i\), \(j\);
    设 \([C]_{i,j} = s[A]_{i,j} + t[B]_{i,j}\),
    若 \(i = q\) 或 \(j = q\).
    则
    \begin{align*}
        \operatorname{pf} {(C)}
        = s \operatorname{pf} {(A)}
        + t \operatorname{pf} {(B)}.
    \end{align*}
\end{theorem}

此性质或许跟行列式的多线性有些像, 但不完全一样.

\begin{proof}
    作命题 \(P(n)\):
    \begin{quotation}
        对任何数 \(s\), \(t\),
        对任何不超过 \(n\) 的正整数 \(q\),
        对任何适合如下条件的三个 \(n\)~级反称阵 \(A\), \(B\), \(C\),
        必有
        \(
        \operatorname{pf} {(C)}
        = s \operatorname{pf} {(A)}
        + t \operatorname{pf} {(B)}
        \):

        (1)
        \([A]_{i,j} = [B]_{i,j} = [C]_{i,j}\),
        对任何不等于 \(q\), 且不超过 \(n\) 的正整数 \(i\), \(j\);

        (2)
        \([C]_{i,j} = s[A]_{i,j} + t[B]_{i,j}\),
        若 \(i = q\) 或 \(j = q\).
    \end{quotation}
    再作命题 \(Q(n)\):
    \(P(n-1)\) 与 \(P(n)\) 是对的.
    我们用数学归纳法证明:
    对任何高于 \(1\) 的整数 \(n\), \(Q(n)\) 是对的.

    \(P(1)\) 不证自明.

    \(P(2)\) 是简单的.
    无论 \(q = 1\) 或 \(q = 2\),
    我们都有 \([C]_{1,2} = s[A]_{1,2} + t[B]_{1,2}\).
    则 \(\operatorname{pf} {(C)}
    = s \operatorname{pf} {(A)}
    + t \operatorname{pf} {(B)}\).

    综上, \(Q(2)\) 是对的.

    我们设 \(Q(n-1)\) 是对的 (\(n \geq 3\)).
    则 \(P(n-2)\) 与 \(P(n-1)\) 是对的.
    我们要由此证明 \(Q(n)\) 是对的,
    即 \(P(n-1)\) 与 \(P(n)\) 是对的.
    \(P(n-1)\) 当然是对的, 由假定.
    所以, 我们由假定
    ``\(P(n-2)\) 与 \(P(n-1)\) 是对的''
    证 \(P(n)\) 是对的,
    即得 \(Q(n)\) 是对的.

    设 \(A\), \(B\), \(C\) 是三个 \(n\)~级反称阵.
    设 \(q\) 是不超过 \(n\) 的正整数.
    设 \(s\), \(t\) 是数.
    设 \([A]_{i,j} = [B]_{i,j} = [C]_{i,j}\),
    对任何不等于 \(q\), 且不超过 \(n\) 的正整数 \(i\), \(j\);
    设 \([C]_{i,j} = s[A]_{i,j} + t[B]_{i,j}\),
    若 \(i = q\) 或 \(j = q\).

    若 \(q = 1\), 则
    \([C]_{1,j} = s[A]_{1,j} + t[B]_{1,j}\),
    且 \(C({1,j}|{1,j}) = A({1,j}|{1,j}) = B({1,j}|{1,j})\),
    对 \(j > 1\).
    则
    \begin{align*}
             &
        \operatorname{pf} {(C)}
        \\
        = {} &
        \sum_{j=2}^{n}
        {(-1)^j [C]_{1,j}
        \operatorname{pf} {(C({1,j}|{1,j}))}}
        \\
        = {} &
        \sum_{j=2}^{n}
        {(-1)^j (s[A]_{1,j} + t[B]_{1,j})
        \operatorname{pf} {(C({1,j}|{1,j}))}}
        \\
        = {} &
        s \sum_{j=2}^{n}
        {(-1)^j [A]_{1,j}
        \operatorname{pf} {(C({1,j}|{1,j}))}}
        +
        t \sum_{j=2}^{n}
        {(-1)^j [B]_{1,j}
        \operatorname{pf} {(C({1,j}|{1,j}))}}
        \\
        = {} &
        s \sum_{j=2}^{n}
        {(-1)^j [A]_{1,j}
        \operatorname{pf} {(A({1,j}|{1,j}))}}
        +
        t \sum_{j=2}^{n}
        {(-1)^j [B]_{1,j}
        \operatorname{pf} {(B({1,j}|{1,j}))}}
        \\
        = {} &
        s \operatorname{pf} {(A)}
        + t \operatorname{pf} {(B)}.
    \end{align*}

    设 \(q \neq 1\).
    那么, 当 \(j \neq q\) 时,
    \([A]_{1,j} = [B]_{1,j} = [C]_{1,j}\),
    且 \(n-2\)~级反称阵
    \(A({1,j}|{1,j})\),
    \(B({1,j}|{1,j})\),
    \(C({1,j}|{1,j})\)
    适合:

    (1\ensuremath{'})
    \([A({1,j}|{1,j})]_{u,v}
    = [B({1,j}|{1,j})]_{u,v}
    = [C({1,j}|{1,j})]_{u,v}\),
    对任何不等于 \(q' = q - \rho(q, 1) - \rho(q, j)
    = q - 1 - \rho(q, j)\),
    且不超过 \(n-2\) 的正整数 \(u\), \(v\);

    (2\ensuremath{'})
    \([C({1,j}|{1,j})]_{u,v}
    = s [A({1,j}|{1,j})]_{u,v}
    + t [B({1,j}|{1,j})]_{u,v}\),
    若 \(u = q'\) 或 \(v = q'\).

    于是, 由假定, 当 \(j \neq q\) 时,
    \begin{align*}
        \operatorname{pf} {(C({1,j}|{1,j}))}
        = s \operatorname{pf} {(A({1,j}|{1,j}))}
        + t \operatorname{pf} {(B({1,j}|{1,j}))}.
    \end{align*}

    另一方面, 当 \(j = q\) 时,
    \([C]_{1,j} = s[A]_{1,j} + t[B]_{1,j}\),
    且 \(C({1,j}|{1,j}) = A({1,j}|{1,j}) = B({1,j}|{1,j})\).

    综上, 我们有
    \begin{align*}
             &
        \operatorname{pf} {(C)}
        \\
        = {} &
        \sum_{j=2}^{n}
        {(-1)^j [C]_{1,j}
        \operatorname{pf} {(C({1,j}|{1,j}))}}
        \\
        = {} &
        (-1)^q [C]_{1,q}
        \operatorname{pf} {(C({1,q}|{1,q}))}
        +
        \sum_{\substack{2 \leq j \leq n     \\j \neq q}}
        {(-1)^j [C]_{1,j}
        \operatorname{pf} {(C({1,j}|{1,j}))}}
        \\
        = {} &
        \hphantom{{} + {}}
        (-1)^q (s[A]_{1,j} + t[B]_{1,j})
        \operatorname{pf} {(C({1,q}|{1,q}))}
        \\
             &
        +
        \sum_{\substack{2 \leq j \leq n     \\j \neq q}}
        {(-1)^j [C]_{1,j}
        (s \operatorname{pf} {(A({1,j}|{1,j}))}
        + t \operatorname{pf} {(B({1,j}|{1,j}))})
        }
        \\
        = {} &
        \hphantom{{} + {}}
        s (-1)^q [A]_{1,j}
        \operatorname{pf} {(C({1,q}|{1,q}))}
        + t (-1)^q [B]_{1,j}
        \operatorname{pf} {(C({1,q}|{1,q}))}
        \\
             &
        +
        s \sum_{\substack{2 \leq j \leq n   \\j \neq q}}
        {(-1)^j [C]_{1,j}
        \operatorname{pf} {(A({1,j}|{1,j}))}
        }
        + t \sum_{\substack{2 \leq j \leq n \\j \neq q}}
        {(-1)^j [C]_{1,j}
        \operatorname{pf} {(B({1,j}|{1,j}))}
        }
        \\
        = {} &
        \hphantom{{} + {}}
        s (-1)^q [A]_{1,j}
        \operatorname{pf} {(A({1,q}|{1,q}))}
        + t (-1)^q [B]_{1,j}
        \operatorname{pf} {(B({1,q}|{1,q}))}
        \\
             &
        +
        s \sum_{\substack{2 \leq j \leq n   \\j \neq q}}
        {(-1)^j [A]_{1,j}
        \operatorname{pf} {(A({1,j}|{1,j}))}
        }
        + t \sum_{\substack{2 \leq j \leq n \\j \neq q}}
        {(-1)^j [B]_{1,j}
        \operatorname{pf} {(B({1,j}|{1,j}))}
        }
        \\
        = {} &
        s \sum_{j=2}^{n}
        {(-1)^j [A]_{1,j}
        \operatorname{pf} {(A({1,j}|{1,j}))}
        }
        +
        t \sum_{j=2}^{n}
        {(-1)^j [B]_{1,j}
        \operatorname{pf} {(B({1,j}|{1,j}))}
        }
        \\
        = {} &
        s \operatorname{pf} {(A)}
        + t \operatorname{pf} {(B)}.
    \end{align*}

    所以, \(P(n)\) 是正确的.
    则 \(Q(n)\) 是正确的.
    由数学归纳法原理, 待证命题成立.
\end{proof}

特别地, 若我们取 \(A = B\), 且 \(t = 0\),
我们有

\begin{restatable}{theorem}{TheoremPfaffianMulitply}
    设 \(A\), \(C\) 是二个 \(n\)~级反称阵.
    设 \(q\) 是不超过 \(n\) 的正整数.
    设 \(s\) 是数.
    设 \([A]_{i,j} = [C]_{i,j}\),
    对任何不等于 \(q\), 且不超过 \(n\) 的正整数 \(i\), \(j\);
    设 \([C]_{i,j} = s[A]_{i,j}\),
    若 \(i = q\) 或 \(j = q\).
    则
    \(\operatorname{pf} {(C)}
    = s \operatorname{pf} {(A)}\).
\end{restatable}

\begin{theorem}
    设 \(A\) 是 \(n\)~级反称阵.
    设 \(p\), \(q\) 是不超过 \(n\) 的正整数, 且 \(p \neq q\).
    设交换 \(A\) 的列~\(p\), \(q\), 不改变其他的列, 得阵~\(B\).
    设交换 \(B\) 的行~\(p\), \(q\), 不改变其他的行, 得阵~\(C\).
    则 \(C\) 是反称阵, 且
    \(
    \operatorname{pf} {(C)} = -\operatorname{pf} {(A)}
    \).
\end{theorem}

\begin{proof}
    我们先说明 \(C\) 是反称阵.
    不难写出
    \begin{align*}
        [B]_{i,j} =
        \begin{cases}
            [A]_{i,q}, & j = p;     \\
            [A]_{i,p}, & j = q;     \\
            [A]_{i,j}, & \text{其他}.
        \end{cases}
    \end{align*}
    也不难写出
    \begin{align*}
        [C]_{i,j} =
        \begin{cases}
            [B]_{q,j}, & i = p;     \\
            [B]_{p,j}, & i = q;     \\
            [B]_{i,j}, & \text{其他}.
        \end{cases}
    \end{align*}
    当 \(i \neq p\), \(i \neq q\),
    \(j \neq p\), \(j \neq q\) 时,
    \begin{align*}
        [C]_{i,j} = [B]_{i,j} = [A]_{i,j};
    \end{align*}
    当 \(i = p\), \(j \neq p\), \(j \neq q\) 时,
    \begin{align*}
        [C]_{i,j} = [B]_{q,j} = [A]_{q,j};
    \end{align*}
    当 \(i = q\), \(j \neq p\), \(j \neq q\) 时,
    \begin{align*}
        [C]_{i,j} = [B]_{p,j} = [A]_{p,j};
    \end{align*}
    当 \(j = p\), \(i \neq p\), \(i \neq q\) 时,
    \begin{align*}
        [C]_{i,j} = [B]_{i,p} = [A]_{i,q};
    \end{align*}
    当 \(j = q\), \(i \neq p\), \(i \neq q\) 时,
    \begin{align*}
        [C]_{i,j} = [B]_{i,q} = [A]_{i,p};
    \end{align*}
    当 \(i = p\), \(j = p\) 时,
    \begin{align*}
        [C]_{i,j} = [B]_{q,p} = [A]_{q,q};
    \end{align*}
    当 \(i = p\), \(j = q\) 时,
    \begin{align*}
        [C]_{i,j} = [B]_{q,q} = [A]_{q,p};
    \end{align*}
    当 \(i = q\), \(j = p\) 时,
    \begin{align*}
        [C]_{i,j} = [B]_{p,p} = [A]_{p,q};
    \end{align*}
    当 \(i = q\), \(j = q\) 时,
    \begin{align*}
        [C]_{i,j} = [B]_{p,q} = [A]_{p,p}.
    \end{align*}
    由此, 不难验证, \(C\) 是反称阵.

    我们再说明
    \(
    \operatorname{pf} {(C)} = -\operatorname{pf} {(A)}
    \).
    以下, 我们无妨设 \(p < q\).

    作命题 \(P(n)\):
    对任何 \(n\)~级反称阵 \(A\),
    对任何 \(1 \leq p < q \leq n\),
    记 \(B\) 是交换 \(A\) 的列~\(p\), \(q\), 且不改变其他的列得到的阵,
    记 \(C\) 是交换 \(B\) 的行~\(p\), \(q\), 且不改变其他的行得到的阵,
    则 \(
    \operatorname{pf} {(C)} = -\operatorname{pf} {(A)}
    \).

    再作命题 \(Q(n)\):
    \(P(n-1)\) 与 \(P(n)\) 是对的.
    我们用数学归纳法证明:
    对任何高于 \(1\) 的整数 \(n\), \(Q(n)\) 是对的.

    \(P(1)\) 不证自明.
    既然没有二列或二行可交换,
    我们认为, 它是平凡地对的.

    \(P(2)\) 是简单的.
    既然 \(n = 2\), 且 \(1 \leq p < q \leq n\),
    则 \(p = 1\), \(q = 2\).
    不难写出, 若 \(A =
    \begin{bmatrix}
        0  & a \\
        -a & 0 \\
    \end{bmatrix}
    \),
    则交换列~\(1\), \(2\)
    与行~\(1\), \(2\) 后,
    我们有 \(C =
    \begin{bmatrix}
        0 & -a \\
        a & 0  \\
    \end{bmatrix}
    \).
    则
    \begin{align*}
        \operatorname{pf} {(C)}
        = -a
        = -\operatorname{pf} {(A)}.
    \end{align*}

    综上, \(Q(2)\) 是对的.

    我们设 \(Q(n-1)\) 是对的 (\(n \geq 3\)).
    则 \(P(n-2)\) 与 \(P(n-1)\) 是对的.
    我们要由此证明 \(Q(n)\) 是对的,
    即 \(P(n-1)\) 与 \(P(n)\) 是对的.
    \(P(n-1)\) 当然是对的, 由假定.
    所以, 我们由假定
    ``\(P(n-2)\) 与 \(P(n-1)\) 是对的''
    证 \(P(n)\) 是对的,
    即得 \(Q(n)\) 是对的.

    设 \(A\) 是 \(n\)~级反称阵.
    设 \(1 \leq p < q \leq n\).
    设交换 \(A\) 的列~\(p\), \(q\), 不改变其他的列, 得阵~\(B\).
    设交换 \(B\) 的行~\(p\), \(q\), 不改变其他的行, 得阵~\(C\).
    我们分类讨论.

    (1)
    \(p = 1\), \(q = 2\).
    当 \(d_1 > 2\), \(d_2 > 2\), 且 \(d_2 \neq d_1\) 时,
    \([A]_{2,d_2}\) 是 \(A({1,d_1}|{1,d_1})\)
    的 \((1, d_2 - 1 - \rho (d_2, d_1))\) 元.
    则
    \begin{align*}
             &
        \operatorname{pf} {(A)}
        \\
        = {} &
        \sum_{2 \leq d_1 \leq n}
        {
        (-1)^{d_1} [A]_{1,d_1}
        \operatorname{pf} {(A({1,d_1}|{1,d_1}))}
        }
        \\
        = {} &
        (-1)^{2} [A]_{1,2}
        \operatorname{pf} {(A({1,2}|{1,2}))}
        +
        \sum_{3 \leq d_1 \leq n}
        {
        (-1)^{d_1} [A]_{1,d_1}
        \operatorname{pf} {(A({1,d_1}|{1,d_1}))}
        }
        \\
        = {} &
        \hphantom{{} + {}}
        [A]_{1,2}
        \operatorname{pf} {(A({1,2}|{1,2}))}
        \\
             &
        +
        \sum_{3 \leq d_1 \leq n}
        {
            (-1)^{d_1} [A]_{1,d_1}
        \sum_{\substack{3 \leq d_2 \leq n      \\d_2 \neq d_1}}
            {
                (-1)^{d_2 - 1 - \rho(d_2, d_1)}
            }
        }
        \\
             &
        \qquad \qquad \qquad
        \qquad \qquad \qquad
        \cdot
        [A]_{2,d_2}
        \operatorname{pf} {(A({1,2,d_1,d_2}|{1,2,d_1,d_2}))}
        \\
        = {} &
        \hphantom{{} + {}}
        [A]_{1,2}
        \operatorname{pf} {(A({1,2}|{1,2}))}
        \\
             &
        +
        \sum_{3 \leq d_1 \leq n}
        {
        \sum_{\substack{3 \leq d_2 \leq n      \\d_2 \neq d_1}}
            {
                (-1)^{d_1}
                    [A]_{1,d_1}
                (-1)^{d_2 - 1 - \rho(d_2, d_1)}
            }
        }
        \\
             &
        \qquad \qquad \qquad
        \qquad \qquad \qquad
        \cdot
        [A]_{2,d_2}
        \operatorname{pf} {(A({1,2,d_1,d_2}|{1,2,d_1,d_2}))}
        \\
        = {} &
        \hphantom{{} + {}}
        [A]_{1,2}
        \operatorname{pf} {(A({1,2}|{1,2}))}
        \\
             &
        +
        \sum_{\substack{3 \leq d_1, d_2 \leq n \\d_2 \neq d_1}}
        {
        (-1)^{d_1 + d_2 - 1 - \rho(d_2, d_1)}
            [A]_{1,d_1} [A]_{2,d_2}
        \operatorname{pf} {(A({1,2,d_1,d_2}|{1,2,d_1,d_2}))}
        }
        \\
        = {} &
        \hphantom{{} + {}}
        [A]_{1,2}
        \operatorname{pf} {(A({1,2}|{1,2}))}
        \\
             &
        +
        \sum_{3 \leq d_1 < d_2 \leq n}
        {
        (-1)^{d_1 + d_2 - 1 - \rho(d_2, d_1)}
            [A]_{1,d_1} [A]_{2,d_2}
        \operatorname{pf} {(A({1,2,d_1,d_2}|{1,2,d_1,d_2}))}
        }
        \\
             &
        +
        \sum_{3 \leq d_2 < d_1 \leq n}
        {
        (-1)^{d_1 + d_2 - 1 - \rho(d_2, d_1)}
            [A]_{1,d_1} [A]_{2,d_2}
        \operatorname{pf} {(A({1,2,d_1,d_2}|{1,2,d_1,d_2}))}
        }
        \\
        = {} &
        \hphantom{{} + {}}
        [A]_{1,2}
        \operatorname{pf} {(A({1,2}|{1,2}))}
        \\
             &
        +
        \sum_{3 \leq j < k \leq n}
        {
        (-1)^{j + k - 1 - 1}
            [A]_{1,j} [A]_{2,k}
        \operatorname{pf} {(A({1,2,j,k}|{1,2,j,k}))}
        }
        \\
             &
        +
        \sum_{3 \leq j < k \leq n}
        {
        (-1)^{k + j - 1}
            [A]_{1,k} [A]_{2,j}
        \operatorname{pf} {(A({1,2,k,j}|{1,2,k,j}))}
        }
        \\
        = {} &
        \hphantom{{} + {}}
        [A]_{1,2}
        \operatorname{pf} {(A({1,2}|{1,2}))}
        \\
             &
        +
        \sum_{3 \leq j < k \leq n}
        {
        (-1)^{j + k}
            [A]_{1,j} [A]_{2,k}
        \operatorname{pf} {(A({1,2,j,k}|{1,2,j,k}))}
        }
        \\
             &
        +
        \sum_{3 \leq j < k \leq n}
        {
        (-1)^{j + k}
        (-1) [A]_{1,k} [A]_{2,j}
        \operatorname{pf} {(A({1,2,j,k}|{1,2,j,k}))}
        }
        \\
        = {} &
        \hphantom{{} + {}}
        [A]_{1,2}
        \operatorname{pf} {(A({1,2}|{1,2}))}
        \\
             &
        +
        \sum_{3 \leq j < k \leq n}
        {
        (-1)^{j + k}\,
        ([A]_{1,j} [A]_{2,k} - [A]_{1,k} [A]_{2,j})
        \operatorname{pf} {(A({1,2,j,k}|{1,2,j,k}))}
        }
        \\
        = {} &
        \hphantom{{} + {}}
        [A]_{1,2}
        \operatorname{pf} {(A({1,2}|{1,2}))}
        \\
             &
        +
        \sum_{3 \leq j < k \leq n}
        {
        (-1)^{j + k}
        \det {
            \begin{bmatrix}
                [A]_{1,j} & [A]_{1,k} \\
                [A]_{2,j} & [A]_{2,k} \\
            \end{bmatrix}
        }
        \operatorname{pf} {(A({1,2,j,k}|{1,2,j,k}))}
        }.
    \end{align*}
    类似地,
    \begin{align*}
        %  &
        \operatorname{pf} {(C)}
        % \\
        = {} &
        \hphantom{{} + {}}
        [C]_{1,2}
        \operatorname{pf} {(C({1,2}|{1,2}))}
        \\
             &
        +
        \sum_{3 \leq j < k \leq n}
        {
        (-1)^{j + k}
        \det {
            \begin{bmatrix}
                [C]_{1,j} & [C]_{1,k} \\
                [C]_{2,j} & [C]_{2,k} \\
            \end{bmatrix}
        }
        \operatorname{pf} {(C({1,2,j,k}|{1,2,j,k}))}
        }.
    \end{align*}
    注意到, \([C]_{1,2} = -[A]_{1,2}\),
    且 \(A({1,2}|{1,2}) = C({1,2}|{1,2})\);
    再注意到, \(3 \leq j < k \leq n\) 时,
    \([C]_{1,j} = [A]_{2,j}\),
    \([C]_{1,k} = [A]_{2,k}\),
    \([C]_{2,j} = [A]_{1,j}\),
    \([C]_{2,k} = [A]_{1,k}\),
    且 \(A({1,2,j,k}|{1,2,j,k}) = C({1,2,j,k}|{1,2,j,k})\).
    故
    \begin{align*}
        %  &
        \operatorname{pf} {(C)}
        % \\
        = {} &
        \hphantom{{} + {}}
        {-[A]_{1,2}}
        \operatorname{pf} {(A({1,2}|{1,2}))}
        \\
             &
        +
        \sum_{3 \leq j < k \leq n}
        {
        (-1)^{j + k}
        \det {
            \begin{bmatrix}
                [A]_{2,j} & [A]_{2,k} \\
                [A]_{1,j} & [A]_{1,k} \\
            \end{bmatrix}
        }
        \operatorname{pf} {(A({1,2,j,k}|{1,2,j,k}))}
        }.
    \end{align*}
    由此可见, \(\operatorname{pf} {(C)}
    = -\operatorname{pf} {(A)}\).

    (2)
    \(2 \leq p = q - 1 \leq n - 1\).
    则
    \begin{align*}
             &
        \operatorname{pf} {(A)}
        \\
        = {} &
        \sum_{2 \leq j \leq n}
        {
        (-1)^{j} [A]_{1,j}
        \operatorname{pf} {(A({1,j}|{1,j}))}
        }
        \\
        = {} &
        \hphantom{{} + {}}
        (-1)^{p} [A]_{1,p}
        \operatorname{pf} {(A({1,p}|{1,p}))}
        +
        (-1)^{p+1} [A]_{1,p+1}
        \operatorname{pf} {(A({1,p+1}|{1,p+1}))}
        \\
             &
        +
        \sum_{\substack{2 \leq j \leq n \\ j \neq p, p+1}}
        {
        (-1)^{j} [A]_{1,j}
        \operatorname{pf} {(A({1,j}|{1,j}))}
        }.
    \end{align*}
    类似地,
    \begin{align*}
             &
        \operatorname{pf} {(C)}
        \\
        = {} &
        \hphantom{{} + {}}
        (-1)^{p} [C]_{1,p}
        \operatorname{pf} {(C({1,p}|{1,p}))}
        +
        (-1)^{p+1} [C]_{1,p+1}
        \operatorname{pf} {(C({1,p+1}|{1,p+1}))}
        \\
             &
        +
        \sum_{\substack{2 \leq j \leq n \\ j \neq p, p+1}}
        {
        (-1)^{j} [C]_{1,j}
        \operatorname{pf} {(C({1,j}|{1,j}))}
        }.
    \end{align*}
    注意到,
    \([C]_{1,p} = [A]_{1,p+1}\),
    \([C]_{1,p+1} = [A]_{1,p}\),
    且
    \begin{align*}
         & C({1,p}|{1,p}) = A({1,p+1}|{1,p+1}), \\
         & C({1,p+1}|{1,p+1}) = A({1,p}|{1,p});
    \end{align*}
    再注意到, \(j > 2\), 且 \(j \neq p\), \(j \neq p+1\) 时,
    \([C]_{1,j} = [A]_{1,j}\),
    且 \(C({1,j}|{1,j})\)
    可被认为是交换
    \(A({1,j}|{1,j})\) 的列~\(p'\), \(p'+1\)
    与行~\(p'\), \(p'+1\) 得到的反称阵
    (其中 \(p' = p - \rho (p, 1) - \rho(p, j)
    = p - 1 - \rho(p, j)\)).
    由假定,
    \(
    \operatorname{pf} {(C({1,j}|{1,j}))}
    = (-1) \operatorname{pf} {(A({1,j}|{1,j}))}
    \).
    故
    \begin{align*}
             &
        \operatorname{pf} {(C)}
        \\
        = {} &
        \hphantom{{} + {}}
        (-1)^{p} [A]_{1,p+1}
        \operatorname{pf} {(A({1,p+1}|{1,p+1}))}
        +
        (-1)^{p+1} [A]_{1,p}
        \operatorname{pf} {(A({1,p}|{1,p}))}
        \\
             &
        +
        \sum_{\substack{2 \leq j \leq n \\ j \neq p, p+1}}
        {
        (-1)^{j} [A]_{1,j}
        (-1) \operatorname{pf} {(A({1,j}|{1,j}))}
        }.
    \end{align*}
    由此可见, \(\operatorname{pf} {(C)}
    = -\operatorname{pf} {(A)}\).

    (3)
    其他的情形.
    不难看出, 我们已证,
    若列~\(p\), \(q\) (行~\(p\), \(q\)) 是相邻的
    (即 \(p = q - 1\);
    注意, 我们已约定 \(p < q\)),
    则 \(\operatorname{pf} {(C)}
    = -\operatorname{pf} {(A)}\).
    那么, 其他的情形自然是%
    当列~\(p\), \(q\) (行~\(p\), \(q\)) 不是相邻的时%
    的情形.

    我们交换 \(A\)~的列~\(p\), \(p+1\) 与行~\(p\), \(p+1\),
    得反称阵~\(C_1\).
    则 \(\operatorname{pf} {(C_1)} = -\operatorname{pf} {(A)}
    = (-1)^{1} \operatorname{pf} {(A)}\).
    我们交换 \(C_1\)~的列~\(p+1\), \(p+2\) 与行~\(p+1\), \(p+2\),
    得反称阵~\(C_2\).
    则 \(\operatorname{pf} {(C_2)} = -\operatorname{pf} {(C_1)}
    = (-1)^{2} \operatorname{pf} {(A)}\).
    \(\dots \dots\)
    我们交换 \(C_{q-p-1}\)~的列~\(q-1\), \(q\) 与行~\(q-1\), \(q\),
    得反称阵~\(C_{q-p}\).
    则 \(\operatorname{pf} {(C_{q-p})} = -\operatorname{pf} {(C_{q-p-1})}
    = (-1)^{q-p} \operatorname{pf} {(A)}\).

    记 \(G_0 = C_{q-p}\).
    则 \(\operatorname{pf} {(G_0)} = (-1)^{q-p} \operatorname{pf} {(A)}\).
    我们交换 \(G_0\)~的列~\(q-2\), \(q-1\) 与行~\(q-2\), \(q-1\),
    得反称阵~\(G_1\).
    则 \(\operatorname{pf} {(G_1)} = -\operatorname{pf} {(G_0)}
    = (-1)^{q-p+1} \operatorname{pf} {(A)}\).
    我们交换 \(G_1\)~的列~\(q-3\), \(q-2\) 与行~\(q-3\), \(q-2\),
    得反称阵~\(G_2\).
    则 \(\operatorname{pf} {(G_2)} = -\operatorname{pf} {(G_1)}
    = (-1)^{q-p+2} \operatorname{pf} {(A)}\).
    \(\dots \dots\)
    我们交换 \(G_{q-p-2}\)~的列~\(p\), \(p+1\) 与行~\(p\), \(p+1\),
    得反称阵~\(G_{q-p-1}\).
    则 \(\operatorname{pf} {(G_{q-p-1})} = -\operatorname{pf} {(G_{q-p-2})}
    = (-1)^{q-p+(q-p-1)} \operatorname{pf} {(A)}\).
    不难看出, \(C = G_{q-p-1}\).
    所以,
    \(\operatorname{pf} {(C)}
    = (-1)^{2(q-p)-1} \operatorname{pf} {(A)}
    = -\operatorname{pf} {(A)}\).

    所以, \(P(n)\) 是正确的.
    则 \(Q(n)\) 是正确的.
    由数学归纳法原理, 待证命题成立.
\end{proof}

利用此事, 我们可写出 pfaffian 的定义的变体.

\begin{theorem}
    设 \(A\) 是 \(n\)~级反称阵 (\(n > 2\)).
    设 \(i\) 是不超过 \(n\) 的正整数.
    则
    \begin{align*}
        \operatorname{pf} {(A)}
        =
        \sum_{\substack{1 \leq j \leq n \\j \neq i}}
        {
        (-1)^{i-1+j+\rho(i,j)}\, [A]_{i,j}
        \operatorname{pf} {(A({i,j}|{i,j}))}}.
    \end{align*}
\end{theorem}

\begin{proof}
    设 \(i = 1\).
    则 \(i - 1 + j + \rho(i, j) = j\), 若 \(j > 1\).
    所以, 这是对的.

    设 \(i > 1\).
    交换 \(A\) 的列~\(i-1\), \(i\) 与行~\(i-1\), \(i\),
    得反称阵~\(C_1\).
    则 \(\operatorname{pf} {(C_1)}
    = -\operatorname{pf} {(A)}
    = (-1)^1 \operatorname{pf} {(A)}\).
    交换 \(C_1\) 的列~\(i-2\), \(i-1\) 与行~\(i-2\), \(i-1\),
    得反称阵~\(C_2\).
    则 \(\operatorname{pf} {(C_2)}
    = -\operatorname{pf} {(C_1)}
    = (-1)^2 \operatorname{pf} {(A)}\).
    \(\dots \dots\)
    交换 \(C_{i-2}\) 的列~\(1\), \(2\) 与行~\(1\), \(2\),
    得反称阵~\(C_{i-1}\).
    则 \(\operatorname{pf} {(C_{i-1})}
    = -\operatorname{pf} {(C_{i-2})}
    = (-1)^{i-1} \operatorname{pf} {(A)}\).
    记 \(C = C_{i-1}\).
    则 \(\operatorname{pf} {(A)}
    = (-1)^{i-1} \operatorname{pf} {(C)}\).
    我们有如下发现.

    (1)
    \(A(i|i) = C(1|1)\).

    (2)
    当 \(j < i\) 时,
    \([A]_{i,j} = [C]_{1,j+1}\);
    当 \(j > i\) 时,
    \([A]_{i,j} = [C]_{1,j}\).
    于是,
    当 \(2 \leq j \leq i\) 时,
    \([C]_{1,j} = [A]_{i,j-1}\);
    当 \(j > i\) 时,
    \([C]_{1,j} = [A]_{i,j}\).

    (3)
    当 \(j < i\) 时,
    \(A({i,j}|{i,j}) = C({1,j+1}|{1,j+1})\);
    当 \(j > i\) 时,
    \(A({i,j}|{i,j}) = C({1,j}|{1,j})\).
    于是,
    当 \(2 \leq j \leq i\) 时,
    \(C({1,j}|{1,j}) = A({i,j-1}|{i,j-1})\);
    当 \(j > i\) 时,
    \(C({1,j}|{1,j}) = A({i,j}|{i,j})\).

    则
    \begin{align*}
             &
        \operatorname{pf} {(C)}
        \\
        = {} &
        \sum_{j = 2}^{n}
        {(-1)^{j} [C]_{1,j}
        \operatorname{pf} {(C({1,j}|{1,j}))}}
        \\
        = {} &
        \sum_{2 \leq j \leq i}
        {(-1)^{j} [C]_{1,j}
        \operatorname{pf} {(C({1,j}|{1,j}))}}
        +
        \sum_{i < j \leq n}
        {(-1)^{j} [C]_{1,j}
        \operatorname{pf} {(C({1,j}|{1,j}))}}
        \\
        = {} &
        \sum_{2 \leq j \leq i}
        {(-1)^{j} [A]_{i,j-1}
        \operatorname{pf} {(A({i,j-1}|{i,j-1}))}}
        +
        \sum_{i < j \leq n}
        {(-1)^{j} [A]_{1,j}
        \operatorname{pf} {(A({i,j}|{i,j}))}}
        \\
        = {} &
        \sum_{1 \leq j < i}
        {(-1)^{j+1} [A]_{i,j}
        \operatorname{pf} {(A({i,j}|{i,j}))}}
        +
        \sum_{i < j \leq n}
        {(-1)^{j} [A]_{1,j}
        \operatorname{pf} {(A({i,j}|{i,j}))}}
        \\
        = {} &
        \hphantom{{} + {}}
        \sum_{1 \leq j < i}
        {(-1)^{j+\rho(i,j)}\, [A]_{i,j}
        \operatorname{pf} {(A({i,j}|{i,j}))}}
        \\
             &
        +
        \sum_{i < j \leq n}
        {(-1)^{j+\rho(i,j)}\, [A]_{1,j}
        \operatorname{pf} {(A({i,j}|{i,j}))}}
        \\
        = {} &
        \sum_{\substack{1 \leq j \leq n \\j \neq i}}
        {(-1)^{j+\rho(i,j)}\, [A]_{i,j}
        \operatorname{pf} {(A({i,j}|{i,j}))}}.
    \end{align*}
    故
    \begin{align*}
        \operatorname{pf} {(A)}
        = {} &
        (-1)^{i-1}
        \operatorname{pf} {(C)}
        \\
        = {} &
        (-1)^{i-1}
        \sum_{\substack{1 \leq j \leq n \\j \neq i}}
        {(-1)^{j+\rho(i,j)}\, [A]_{i,j}
        \operatorname{pf} {(A({i,j}|{i,j}))}}
        \\
        = {} &
        \sum_{\substack{1 \leq j \leq n \\j \neq i}}
        {
        (-1)^{i-1}
        (-1)^{j+\rho(i,j)}\, [A]_{i,j}
        \operatorname{pf} {(A({i,j}|{i,j}))}}
        \\
        = {} &
        \sum_{\substack{1 \leq j \leq n \\j \neq i}}
        {
        (-1)^{i-1+j+\rho(i,j)}\, [A]_{i,j}
        \operatorname{pf} {(A({i,j}|{i,j}))}}.
        \qedhere
    \end{align*}
\end{proof}

\begin{theorem}
    设 \(A\) 是 \(n\)~级反称阵.
    设 \(p\), \(q\) 是不超过 \(n\) 的正整数, 且 \(p \neq q\).
    设 \(A\) 的列~\(p\), \(q\) 相等
    (于是, 显然, \(A\) 的行~\(p\), \(q\) 也相等).
    则 \(\operatorname{pf} {(A)} = 0\).
\end{theorem}

\begin{proof}
    我们先说明:
    \begin{quotation}
        设 \(A\) 是 \(n\)~级反称阵.
        设 \(p\), \(q\) 是不超过 \(n\) 的正整数, 且 \(p \neq q\).
        设 \(A\) 的列~\(p\), \(q\) 相等
        (于是, 显然, \(A\) 的行~\(p\), \(q\) 也相等).
        再设 \(u\), \(v\) 是不超过 \(n\) 的正整数, 且 \(u \neq v\).
        设交换 \(A\) 的列~\(u\), \(v\), 不改变其他的列, 得阵~\(B\).
        设交换 \(B\) 的行~\(u\), \(v\), 不改变其他的行,
        得反称阵~\(C\).
        则 \(C\) 仍有二列相等 (当然, 也有二行相等).
    \end{quotation}

    首先, 我们可具体地写下 \(C\) 的元:
    \begin{align*}
        [C]_{i,j} =
        \begin{cases}
            [A]_{i,j},
             &
            \text{\(i \neq u\), \(i \neq v\), \(j \neq u\), \(j \neq v\)};
            \\
            [A]_{v,j},
             &
            \text{\(i = u\), \(j \neq u\), \(j \neq v\)};
            \\
            [A]_{u,j},
             &
            \text{\(i = v\), \(j \neq u\), \(j \neq v\)};
            \\
            [A]_{i,v},
             &
            \text{\(j = u\), \(i \neq u\), \(i \neq v\)};
            \\
            [A]_{i,u},
             &
            \text{\(j = v\), \(i \neq u\), \(i \neq v\)};
            \\
            [A]_{v,v},
             &
            \text{\(i = j = u\)};
            \\
            [A]_{u,u},
             &
            \text{\(i = j = v\)};
            \\
            [A]_{v,u},
             &
            \text{\(i = u\), \(j = v\)};
            \\
            [A]_{u,v},
             &
            \text{\(i = v\), \(j = u\)}.
        \end{cases}
    \end{align*}
    然后, 我们可分类讨论.
    无妨设 \(p < q\), 且 \(u < v\).
    则:

    若 \(p \neq u\), \(p \neq v\),
    且 \(q \neq u\), \(q \neq v\),
    则我们可验证, \(C\) 的列~\(p\), \(q\) 相等;

    若 \(p = u\), 且 \(q = v\),
    则我们可验证, \(C\) 的列~\(p\), \(q\) 相等;

    若 \(p = u\), \(q \neq v\),
    则我们可验证, \(C\) 的列~\(q\), \(v\) 相等;

    若 \(p = v\),
    则我们可验证, \(C\) 的列~\(u\), \(q\) 相等;

    若 \(q = u\),
    则我们可验证, \(C\) 的列~\(p\), \(v\) 相等;

    若 \(q = v\), 且 \(p \neq u\),
    则我们可验证, \(C\) 的列~\(u\), \(p\) 相等.

    \vspace{2ex}

    好的.
    现在, 我们证原命题.

    无妨设 \(p < q\).

    先设 \(p = 1\), \(q = 2\).
    则
    \begin{align*}
        %  &
        \operatorname{pf} {(A)}
        % \\
        = {} &
        \hphantom{{} + {}}
        [A]_{1,2}
        \operatorname{pf} {(A({1,2}|{1,2}))}
        \\
             &
        +
        \sum_{3 \leq j < k \leq n}
        {
        (-1)^{j + k}
        \det {
            \begin{bmatrix}
                [A]_{1,j} & [A]_{1,k} \\
                [A]_{2,j} & [A]_{2,k} \\
            \end{bmatrix}
        }
        \operatorname{pf} {(A({1,2,j,k}|{1,2,j,k}))}
        }.
    \end{align*}
    因为 \(A\) 的列~\(1\), \(2\) 相等,
    故 \(A\) 的行~\(1\), \(2\) 也相等
    (注意, \(A\) 是反称阵),
    且 \([A]_{1,2} = [A]_{1,1} = 0\).
    则
    \begin{align*}
        %  &
        \operatorname{pf} {(A)}
        % \\
        = {} &
        \hphantom{{} + {}}
        0
        \operatorname{pf} {(A({1,2}|{1,2}))}
        \\
             &
        +
        \sum_{3 \leq j < k \leq n}
        {
        (-1)^{j + k}
        \,
        0
        \operatorname{pf} {(A({1,2,j,k}|{1,2,j,k}))}
        }
        \\
        = {} &
        0.
    \end{align*}

    设 \(p = 1\), 但 \(q > 2\).
    交换 \(A\) 的列~\(2\), \(q\), 不改变其他的列, 得阵~\(B_1\).
    交换 \(B_1\) 的行~\(2\), \(q\), 不改变其他的行,
    得反称阵~\(C_1\).
    则 \(C_1\) 的列~\(1\), \(2\) 相等,
    故 \(\operatorname{pf} {(C_1)} = 0\).
    另一方面, \(\operatorname{pf} {(C_1)}
    = -\operatorname{pf} {(A)}\),
    故 \(\operatorname{pf} {(A)} = 0\).

    设 \(p = 2\).
    交换 \(A\) 的列~\(1\), \(q\), 不改变其他的列, 得阵~\(B_2\).
    交换 \(B_2\) 的行~\(1\), \(q\), 不改变其他的行,
    得反称阵~\(C_2\).
    则 \(C_2\) 的列~\(1\), \(2\) 相等,
    故 \(\operatorname{pf} {(C_2)} = 0\).
    另一方面, \(\operatorname{pf} {(C_2)}
    = -\operatorname{pf} {(A)}\),
    故 \(\operatorname{pf} {(A)} = 0\).

    最后, 设 \(2 < p\).
    交换 \(A\) 的列~\(2\), \(p\), 不改变其他的列, 得阵~\(B_3\).
    交换 \(B_3\) 的行~\(2\), \(p\), 不改变其他的行,
    得反称阵~\(C_3\).
    则 \(C_3\) 的列~\(2\), \(q\) 相等,
    故 \(\operatorname{pf} {(C_3)} = 0\)
    (我们已证此情形).
    另一方面, \(\operatorname{pf} {(C_3)}
    = -\operatorname{pf} {(A)}\),
    故 \(\operatorname{pf} {(A)} = 0\).
\end{proof}

有了这些有较长的论证的性质, 我们可以证明%
倍加与反称阵的 pfaffian 的关系.
这是重要的.

\begin{restatable}{theorem}{TheoremPfaffianMulitplyAndAdd}
    设 \(A\) 是 \(n\)~级反称阵.
    加 \(A\) 的列~\(p\) 的 \(s\)~倍于列~\(q\)
    (\(p \neq q\)),
    不改变其他的列, 得 \(n\)~级阵 \(B\).
    加 \(B\) 的行~\(p\) 的 \(s\)~倍于行~\(q\),
    不改变其他的行, 得 \(n\)~级阵 \(C\).
    则 \(\operatorname{pf} {(C)} = \operatorname{pf} {(A)}\).
\end{restatable}

\begin{proof}
    不难写出 \(C\)~的元:
    \begin{align*}
        [C]_{i,j} =
        \begin{cases}
            [A]_{q,j} + s[A]_{p,j},
             & \text{\(i = q\), 且 \(j \neq q\)}; \\
            [A]_{i,q} + s[A]_{i,p},
             & \text{\(j = q\), 且 \(i \neq q\)}; \\
            [A]_{i,j},
             & \text{其他}.
        \end{cases}
    \end{align*}
    作 \(n\)~级阵 \(G\) 如下:
    \begin{align*}
        [G]_{i,j} =
        \begin{cases}
            [A]_{p,j}, & \text{\(i = q\), 且 \(j \neq q\)}; \\
            [A]_{i,p}, & \text{\(j = q\), 且 \(i \neq q\)}; \\
            [A]_{i,j}, & \text{其他}.
        \end{cases}
    \end{align*}
    则 \(G\) 是反称阵, \(G\) 的列~\(p\), \(q\) 相等,
    且:

    (1)
    \([A]_{i,j} = [G]_{i,j} = [C]_{i,j}\),
    对任何不等于 \(q\), 且不超过 \(n\) 的正整数 \(i\), \(j\);

    (2)
    \([C]_{i,j} = 1[A]_{i,j} + s[G]_{i,j}\),
    若 \(i = q\) 或 \(j = q\).

    所以,
    \begin{align*}
        \operatorname{pf} {(C)}
        = {} &
        1 \operatorname{pf} {(A)} + s \operatorname{pf} {(G)}
        \\
        = {} &
        \operatorname{pf} {(A)} + s \, 0
        \\
        = {} &
        \operatorname{pf} {(A)}.
        \qedhere
    \end{align*}
\end{proof}

现在, 我们可以证明,
反称阵的行列式与 pfaffian 有如下关系.

\begin{restatable}{theorem}{TheoremPfaffianSquareDet}
    设 \(A\) 是 \(n\)~级反称阵.
    则 \(\det {(A)}
    = (\operatorname{pf} {(A)})^2\).
\end{restatable}

\begin{proof}
    设 \(n\) 是奇数.
    则 \(\det {(A)} = 0\),
    且 \(\operatorname{pf} {(A)} = 0\).

    再设 \(n = 2m\) 是偶数.
    则利用若干次列的倍加, 与对应的行的倍加
    (``先列后行, 交替地作''),
    我们可变 \(A\) 为
    \begin{align*}
        B =
        \begin{bmatrix}
            0      & b_1    & 0      & 0      & \cdots & 0      & 0      \\
            -b_1   & 0      & 0      & 0      & \cdots & 0      & 0      \\
            0      & 0      & 0      & b_2    & \cdots & 0      & 0      \\
            0      & 0      & -b_2   & 0      & \cdots & 0      & 0      \\
            \vdots & \vdots & \vdots & \vdots & \ddots & \vdots & \vdots \\
            0      & 0      & 0      & 0      & \cdots & 0      & b_m    \\
            0      & 0      & 0      & 0      & \cdots & -b_m   & 0      \\
        \end{bmatrix}.
    \end{align*}
    则 \(\det {(B)} = \det {(A)}\),
    且 \(\operatorname{pf} {(B)} = \operatorname{pf} {(A)}\).
    最后, 注意到 \(\det {(B)}
    = (\operatorname{pf} {(B)})^2\),
    故 \(\det {(A)}
    = (\operatorname{pf} {(A)})^2\).
\end{proof}

我以一个应用结束本节.

设 \(S\) 是 ``整'', 或 ``有理'', 或 ``实'', 或 ``复'' 中的一个.
我们约定,
当 \(S\) 是 ``整'' 时, \(S\)-数是 ``整数'';
当 \(S\) 是 ``有理'' 时, \(S\)-数是 ``有理数'';
当 \(S\) 是 ``实'' 时, \(S\)-数是 ``实数'';
当 \(S\) 是 ``复'' 时, \(S\)-数是 ``复数''.
于是, 二个 \(S\)-数的和、差、积还是 \(S\)-数.

若一个阵的元全是 \(S\)-数,
我们说, 此阵是一个 \(S\)-阵.

现在, 我们设 \(A\) 是一个 \(S\)-反称阵
(\(A\) 是反称阵, 且 \(A\) 是 \(S\)-阵).
因为 pfaffian 的定义是由一些数的和、差、积作成的,
且不含除法,
故 \(\operatorname{pf} {(A)}\) 是一个 \(S\)-数.
则 \(\det {(A)}\) 是一个 \(S\)-数的平方.
具体地 (且较有趣地),
若 \(A\) 是一个整反称阵, 则 \(\det {(A)}\) 是一个整数的平方
(完全平方数);
若 \(A\) 是一个实反称阵, 则 \(\det {(A)}\) 是一个实数的平方
(非负实数).

\section{阵的积与倍加 (续)}

本节, 我们进一步地讨论阵的积与 (列的) 倍加的关系.

回想起, 我们有如下结论.

\TheoremMultiplyAndAdd*

那么, 特别地, 取 \(A\) 为方阵, 有

\begin{theorem}
    设 \(A\) 是 \(n\)~级阵.
    利用若干次列的倍加,
    我们可变 \(A\) 为 \(n\)~级阵 \(B\),
    使当 \(i < j\) 时,
    \([B]_{i,j} = 0\).
\end{theorem}

形象地, 对任何 \(n\)~级阵
\begin{align*}
    A =
    \begin{bmatrix}
        [A]_{1,1}   & [A]_{1,2}   & \cdots & [A]_{1,n-1}   & [A]_{1,n}   \\
        [A]_{2,1}   & [A]_{2,2}   & \cdots & [A]_{2,n-1}   & [A]_{2,n}   \\
        \vdots      & \vdots      & {}     & \vdots        & \vdots      \\
        [A]_{n-1,1} & [A]_{n-1,2} & \cdots & [A]_{n-1,n-1} & [A]_{n-1,n} \\
        [A]_{n,1}   & [A]_{n,2}   & \cdots & [A]_{n,n-1}   & [A]_{n,n}   \\
    \end{bmatrix},
\end{align*}
我们总可作列的倍加, 变 \(A\) 为
\begin{align*}
    B =
    \begin{bmatrix}
        [B]_{1,1}   & 0           & \cdots & 0             & 0         \\
        [B]_{2,1}   & [B]_{2,2}   & \cdots & 0             & 0         \\
        \vdots      & \vdots      & \ddots & \vdots        & \vdots    \\
        [B]_{n-1,1} & [B]_{n-1,2} & \cdots & [B]_{n-1,n-1} & 0         \\
        [B]_{n,1}   & [B]_{n,2}   & \cdots & [B]_{n,n-1}   & [B]_{n,n} \\
    \end{bmatrix}.
\end{align*}

因为 \(A\), \(B\) 是方阵,
故我们可说 \(A\), \(B\) 的行列式.
现在, 我说, 若 \(\det {(A)} \neq 0\),
则我们可用列的倍加, 进一步地变 \(B\) 为
\begin{align*}
    D =
    \begin{bmatrix}
        [B]_{1,1} & 0         & \cdots & 0             & 0         \\
        0         & [B]_{2,2} & \cdots & 0             & 0         \\
        \vdots    & \vdots    & \ddots & \vdots        & \vdots    \\
        0         & 0         & \cdots & [B]_{n-1,n-1} & 0         \\
        0         & 0         & \cdots & 0             & [B]_{n,n} \\
    \end{bmatrix};
\end{align*}
具体地, \(D\) 是一个 \(n\)~级阵,
且
\begin{align*}
    [D]_{i,j} =
    \begin{cases}
        [B]_{i,i}, & i = j;     \\
        0,         & \text{其他}.
    \end{cases}
\end{align*}

我们知道, 倍加不改变行列式.
于是, \(\det {(B)} = \det {(A)} \neq 0\).
另一方面, \(\det {(B)} = [B]_{1,1} [B]_{2,2} \dots [B]_{n,n}\),
故 \([B]_{1,1}\), \([B]_{2,2}\), \(\dots\), \([B]_{n,n}\)
都不是零.
那么, 我们加 \(B\) 的%
列~\(n\) 的 \(-[B]_{n,n-1}/[B]_{n,n}\)~倍于列~\(n-1\),
列~\(n\) 的 \(-[B]_{n,n-2}/[B]_{n,n}\)~倍于列~\(n-2\),
\(\dots \dots\),
列~\(n\) 的 \(-[B]_{n,1}/[B]_{n,n}\)~倍于列~\(1\),
得
\begin{align*}
    \begin{bmatrix}
        [B]_{1,1}   & 0           & \cdots & 0             & 0             & 0         \\
        [B]_{2,1}   & [B]_{2,2}   & \cdots & 0             & 0             & 0         \\
        \vdots      & \vdots      & \ddots & \vdots        & \vdots        & \vdots    \\
        [B]_{n-2,1} & [B]_{n-2,2} & \cdots & [B]_{n-2,n-2} & 0             & 0         \\
        [B]_{n-1,1} & [B]_{n-1,2} & \cdots & [B]_{n-1,n-2} & [B]_{n-1,n-1} & 0         \\
        0           & 0           & \cdots & 0             & 0             & [B]_{n,n} \\
    \end{bmatrix}.
\end{align*}
我们再加%
列~\(n-1\) 的 \(-[B]_{n-1,n-3}/[B]_{n-1,n-1}\)~倍于列~\(n-2\),
列~\(n-1\) 的 \(-[B]_{n-1,n-3}/[B]_{n-1,n-1}\)~倍于列~\(n-3\),
\(\dots \dots\),
列~\(n-1\) 的 \(-[B]_{n-1,1}/[B]_{n-1,n-1}\)~倍于列~\(1\),
得
\begin{align*}
    \begin{bmatrix}
        [B]_{1,1}   & 0           & \cdots & 0             & 0             & 0         \\
        [B]_{2,1}   & [B]_{2,2}   & \cdots & 0             & 0             & 0         \\
        \vdots      & \vdots      & \ddots & \vdots        & \vdots        & \vdots    \\
        [B]_{n-2,1} & [B]_{n-2,2} & \cdots & [B]_{n-2,n-2} & 0             & 0         \\
        0           & 0           & \cdots & 0             & [B]_{n-1,n-1} & 0         \\
        0           & 0           & \cdots & 0             & 0             & [B]_{n,n} \\
    \end{bmatrix}.
\end{align*}
\(\dots \dots\)
最后, 我们得
\begin{align*}
    \begin{bmatrix}
        [B]_{1,1} & 0         & \cdots & 0             & 0         \\
        0         & [B]_{2,2} & \cdots & 0             & 0         \\
        \vdots    & \vdots    & \ddots & \vdots        & \vdots    \\
        0         & 0         & \cdots & [B]_{n-1,n-1} & 0         \\
        0         & 0         & \cdots & 0             & [B]_{n,n} \\
    \end{bmatrix}.
\end{align*}

综上, 我们有

\begin{theorem}
    设 \(A\) 是 \(n\)~级阵,
    且 \(\det {(A)} \neq 0\).
    利用若干次列的倍加,
    我们可变 \(A\) 为 \(n\)~级阵 \(D\),
    使当 \(i \neq j\) 时,
    \([D]_{i,j} = 0\).
\end{theorem}

进一步地, 我说, 我们还可变 \(D\) 为
\begin{align*}
    M =
    \begin{bmatrix}
        \det {(A)} & 0      & \cdots & 0      & 0      \\
        0          & 1      & \cdots & 0      & 0      \\
        \vdots     & \vdots & \ddots & \vdots & \vdots \\
        0          & 0      & \cdots & 1      & 0      \\
        0          & 0      & \cdots & 0      & 1      \\
    \end{bmatrix};
\end{align*}
具体地, \(M\) 是一个 \(n\)~级阵,
且
\begin{align*}
    [M]_{i,j} =
    \begin{cases}
        \det {(A)}, & i = j = 1; \\
        1,          & i = j > 1; \\
        0,          & \text{其他}.
    \end{cases}
\end{align*}

设 \(D\) 形如
\begin{align*}
    \begin{bmatrix}
        [B]_{1,1} & 0         & \cdots & 0             & 0             & 0         \\
        0         & [B]_{2,2} & \cdots & 0             & 0             & 0         \\
        \vdots    & \vdots    & \ddots & \vdots        & \vdots        & \vdots    \\
        0         & 0         & \cdots & [B]_{n-2,n-2} & 0             & 0         \\
        0         & 0         & \cdots & 0             & [B]_{n-1,n-1} & 0         \\
        0         & 0         & \cdots & 0             & 0             & [B]_{n,n} \\
    \end{bmatrix},
\end{align*}
其中 \([B]_{1,1}\), \([B]_{2,2}\), \(\dots\), \([B]_{n,n}\)
都不是零.

加列~\(n\) 的 \((1 - [B]_{n,n})/[B]_{n,n}\)~倍于列~\(n-1\), 有
\begin{align*}
    \begin{bmatrix}
        [B]_{1,1} & 0         & \cdots & 0             & 0             & 0         \\
        0         & [B]_{2,2} & \cdots & 0             & 0             & 0         \\
        \vdots    & \vdots    & \ddots & \vdots        & \vdots        & \vdots    \\
        0         & 0         & \cdots & [B]_{n-2,n-2} & 0             & 0         \\
        0         & 0         & \cdots & 0             & [B]_{n-1,n-1} & 0         \\
        0         & 0         & \cdots & 0             & 1-[B]_{n,n}   & [B]_{n,n} \\
    \end{bmatrix}.
\end{align*}
加列~\(n-1\) 于列~\(n\), 有
\begin{align*}
    \begin{bmatrix}
        [B]_{1,1} & 0         & \cdots & 0             & 0             & 0             \\
        0         & [B]_{2,2} & \cdots & 0             & 0             & 0             \\
        \vdots    & \vdots    & \ddots & \vdots        & \vdots        & \vdots        \\
        0         & 0         & \cdots & [B]_{n-2,n-2} & 0             & 0             \\
        0         & 0         & \cdots & 0             & [B]_{n-1,n-1} & [B]_{n-1,n-1} \\
        0         & 0         & \cdots & 0             & 1-[B]_{n,n}   & 1             \\
    \end{bmatrix}.
\end{align*}
加列~\(n\) 的 \([B]_{n,n}-1\)~倍于列~\(n-1\), 有
\begin{align*}
    \begin{bmatrix}
        [B]_{1,1} & 0         & \cdots & 0             & 0                       & 0             \\
        0         & [B]_{2,2} & \cdots & 0             & 0                       & 0             \\
        \vdots    & \vdots    & \ddots & \vdots        & \vdots                  & \vdots        \\
        0         & 0         & \cdots & [B]_{n-2,n-2} & 0                       & 0             \\
        0         & 0         & \cdots & 0             & [B]_{n-1,n-1} [B]_{n,n} & [B]_{n-1,n-1} \\
        0         & 0         & \cdots & 0             & 0                       & 1             \\
    \end{bmatrix}.
\end{align*}
加列~\(n-1\) 的 \(-1/[B]_{n,n}\)~倍于列~\(n\), 有
\begin{align*}
    \begin{bmatrix}
        [B]_{1,1} & 0         & \cdots & 0             & 0                       & 0      \\
        0         & [B]_{2,2} & \cdots & 0             & 0                       & 0      \\
        \vdots    & \vdots    & \ddots & \vdots        & \vdots                  & \vdots \\
        0         & 0         & \cdots & [B]_{n-2,n-2} & 0                       & 0      \\
        0         & 0         & \cdots & 0             & [B]_{n-1,n-1} [B]_{n,n} & 0      \\
        0         & 0         & \cdots & 0             & 0                       & 1      \\
    \end{bmatrix}.
\end{align*}
为方便, 记 \(d_2 = [B]_{n-1,n-1} [B]_{n,n}\).
加列~\(n-1\) 的 \((1 - d_2)/d_2\)~倍于列~\(n-2\), 有
\begin{align*}
    \begin{bmatrix}
        [B]_{1,1} & 0         & \cdots & 0             & 0      & 0      \\
        0         & [B]_{2,2} & \cdots & 0             & 0      & 0      \\
        \vdots    & \vdots    & \ddots & \vdots        & \vdots & \vdots \\
        0         & 0         & \cdots & [B]_{n-2,n-2} & 0      & 0      \\
        0         & 0         & \cdots & 1 - d_2       & d_2    & 0      \\
        0         & 0         & \cdots & 0             & 0      & 1      \\
    \end{bmatrix}.
\end{align*}
加列~\(n-2\) 于列~\(n-1\), 有
\begin{align*}
    \begin{bmatrix}
        [B]_{1,1} & 0         & \cdots & 0             & 0             & 0      \\
        0         & [B]_{2,2} & \cdots & 0             & 0             & 0      \\
        \vdots    & \vdots    & \ddots & \vdots        & \vdots        & \vdots \\
        0         & 0         & \cdots & [B]_{n-2,n-2} & [B]_{n-2,n-2} & 0      \\
        0         & 0         & \cdots & 1 - d_2       & 1             & 0      \\
        0         & 0         & \cdots & 0             & 0             & 1      \\
    \end{bmatrix}.
\end{align*}
加列~\(n-1\) 的 \(d_2 - 1\)~倍于列~\(n-2\), 有
\begin{align*}
    \begin{bmatrix}
        [B]_{1,1} & 0         & \cdots & 0                 & 0             & 0      \\
        0         & [B]_{2,2} & \cdots & 0                 & 0             & 0      \\
        \vdots    & \vdots    & \ddots & \vdots            & \vdots        & \vdots \\
        0         & 0         & \cdots & [B]_{n-2,n-2} d_2 & [B]_{n-2,n-2} & 0      \\
        0         & 0         & \cdots & 0                 & 1             & 0      \\
        0         & 0         & \cdots & 0                 & 0             & 1      \\
    \end{bmatrix}.
\end{align*}
加列~\(n-2\) 的 \(-1/d_2\)~倍于列~\(n-1\), 有
\begin{align*}
    \begin{bmatrix}
        [B]_{1,1} & 0         & \cdots & 0                                     & 0      & 0      \\
        0         & [B]_{2,2} & \cdots & 0                                     & 0      & 0      \\
        \vdots    & \vdots    & \ddots & \vdots                                & \vdots & \vdots \\
        0         & 0         & \cdots & [B]_{n-2,n-2} [B]_{n-1,n-1} [B]_{n,n} & 0      & 0      \\
        0         & 0         & \cdots & 0                                     & 1      & 0      \\
        0         & 0         & \cdots & 0                                     & 0      & 1      \\
    \end{bmatrix}.
\end{align*}
\(\dots \dots\)
最后, 我们得
\begin{align*}
    \begin{bmatrix}
        [B]_{1,1} [B]_{2,2} \dots [B]_{n,n} & 0      & \cdots & 0      & 0      \\
        0                                   & 1      & \cdots & 0      & 0      \\
        \vdots                              & \vdots & \ddots & \vdots & \vdots \\
        0                                   & 0      & \cdots & 1      & 0      \\
        0                                   & 0      & \cdots & 0      & 1      \\
    \end{bmatrix};
\end{align*}
再注意到
\(\det {(A)} = [B]_{1,1} [B]_{2,2} \dots [B]_{n,n}\).

综上, 我们有

\begin{theorem}
    设 \(A\) 是 \(n\)~级阵,
    且 \(\det {(A)} \neq 0\).
    利用若干次列的倍加,
    我们可变 \(A\) 为 \(n\)~级阵 \(M\),
    使
    \begin{align*}
        [M]_{i,j} =
        \begin{cases}
            \det {(A)}, & i = j = 1; \\
            1,          & i = j > 1; \\
            0,          & \text{其他}.
        \end{cases}
    \end{align*}
\end{theorem}

我们知道, 我们可用阵的积表示倍加.
于是

\begin{theorem}
    设 \(A\) 是 \(n\)~级阵,
    且 \(\det {(A)} \neq 0\).
    则存在若干个形如 \(E(n; p, q; s)\)
    (\(s\) 是一个数;
    \(p\), \(q\) 是不超过 \(n\) 的正整数,
    \(p \neq q\))
    的阵
    \(E_1\), \(E_2\), \(\dots\), \(E_w\),
    使
    \begin{align*}
        [A E_1 E_2 \dots E_w]_{i,j} =
        \begin{cases}
            \det {(A)}, & i = j = 1; \\
            1,          & i = j > 1; \\
            0,          & \text{其他}.
        \end{cases}
    \end{align*}
\end{theorem}

回想起, \(E(n; p, q; s)\)
(\(s\) 是一个数;
\(p\), \(q\) 是不超过 \(n\) 的正整数,
\(p \neq q\))
是适合如下条件的 \(n\)~级阵:
\begin{align*}
    [E(n; p, q; s)]_{i,j}
    = \begin{cases}
          s,     & \text{\(i = p\), 且 \(j = q\)}; \\
          [I_n], & \text{其他}.
      \end{cases}
\end{align*}

\section{Pfaffian 的性质 (续)}

前面, 我们得到了 pfaffian 的不少性质.
以下三条是重要的:

\TheoremPfaffianMulitply*

\TheoremPfaffianMulitplyAndAdd*

\TheoremPfaffianSquareDet*

本节, 我们证明 pfaffian 的一个新的重要的性质.
为此, 我们先改写第~2~条性质.

\begin{theorem}
    设 \(A\) 是 \(n\)~级反称阵.
    设 \(s\) 是数.
    设 \(p\), \(q\) 是不超过 \(n\) 的正整数,
    且 \(p \neq q\).
    则
    \begin{align*}
        \operatorname{pf}{( (E(n; p, q; s))^{\mathrm{T}}
            A E(n; p, q; s) )}
        = \operatorname{pf} {(A)} \det {(E(n; p, q; s))}.
    \end{align*}
\end{theorem}

\begin{proof}
    加 \(A\) 的列~\(p\) 的 \(s\)~倍于列~\(q\),
    不改变其他的列, 得 \(n\)~级阵 \(B\).
    加 \(B\) 的行~\(p\) 的 \(s\)~倍于行~\(q\),
    不改变其他的行, 得 \(n\)~级阵 \(C\).
    则 \(\operatorname{pf} {(C)} = \operatorname{pf} {(A)}\).

    注意到 \(C = (E(n; p, q; s))^{\mathrm{T}}
    A E(n; p, q; s)\),
    故
    \begin{align*}
        \operatorname{pf}{( (E(n; p, q; s))^{\mathrm{T}}
            A E(n; p, q; s) )} = \operatorname{pf} {(A)}.
    \end{align*}
    再注意到 \(\det {(E(n; p, q; s))} = 1\),
    故
    \begin{equation*}
        %  &
        \operatorname{pf}{( (E(n; p, q; s))^{\mathrm{T}}
            A E(n; p, q; s) )}
        % \\
        % =
        % {} &
        % \operatorname{pf} {(A)}\, 1
        % \\
        =
        % {} &
        \operatorname{pf} {(A)}
        \det {(E(n; p, q; s))}.
        \qedhere
    \end{equation*}
\end{proof}

好的.
现在, 我们可以证明本节的主要结论了.

\begin{theorem}
    设 \(A\) 是 \(n\)~级反称阵.
    设 \(P\) 是 \(n\)~级阵.
    则
    \begin{align*}
        \operatorname{pf} {(P^{\mathrm{T}} A P)}
        = \operatorname{pf} {(A)} \det {(P)}.
    \end{align*}
\end{theorem}

\begin{proof}
    若 \(\det {(P)} = 0\),
    则 \(\det {(P^{\mathrm{T}} A P)}
    = \det {(P^{\mathrm{T}} A)} \det {(P)}
    = 0\).
    则 \(P^{\mathrm{T}} A P\) 的 pfaffian 为 \(0\).
    所以, 若 \(\det {(P)} = 0\),
    则命题是对的.

    若 \(\det {(P)} \neq 0\),
    则存在若干个形如 \(E(n; p, q; s)\)
    (\(s\) 是一个数;
    \(p\), \(q\) 是不超过 \(n\) 的正整数,
    \(p \neq q\))
    的阵
    \(E_1\), \(E_2\), \(\dots\), \(E_w\),
    使 \(M = P E_1 E_2 \dots E_w\)
    适合
    \begin{align*}
        [M]_{i,j} =
        \begin{cases}
            \det {(P)}, & i = j = 1; \\
            1,          & i = j > 1; \\
            0,          & \text{其他}.
        \end{cases}
    \end{align*}
    则
    \begin{align*}
        \operatorname{pf} {(P^{\mathrm{T}} A P)}
        = {} &
        \operatorname{pf} {(P^{\mathrm{T}} A P)}
        \, 1
        \\
        = {} &
        \operatorname{pf} {(P^{\mathrm{T}} A P)}
        \det {(E_1)}
        \det {(E_2)}
        \dots
        \det {(E_{w-1})}
        \det {(E_w)}
        \\
        = {} &
        \operatorname{pf}
        {(E_1^{\mathrm{T}} (P^{\mathrm{T}} A P) E_1)}
        \det {(E_2)}
        \dots
        \det {(E_{w-1})}
        \det {(E_w)}
        \\
        = {} &
        \dots \dots \dots \dots
        \dots \dots \dots \dots
        \dots \dots \dots \dots
        \dots \dots \dots \dots
        \\
        = {} &
        \operatorname{pf}
        {(
        E_{w-1}^{\mathrm{T}}
        \dots
        (E_2^{\mathrm{T}}
        (E_1^{\mathrm{T}}
        (P^{\mathrm{T}} A P)
        E_1)
        E_2)
        \dots
        E_{w-1}
        )}
        \det {(E_w)}
        \\
        = {} &
        \operatorname{pf}
        {(
        E_w^{\mathrm{T}}
        (E_{w-1}^{\mathrm{T}}
        \dots
        (E_2^{\mathrm{T}}
        (E_1^{\mathrm{T}}
        (P^{\mathrm{T}} A P)
        E_1)
        E_2)
        \dots
        E_{w-1})
        E_w
        )}
        \\
        = {} &
        \operatorname{pf}
        {(
        (
        E_w^{\mathrm{T}}
        E_{w-1}^{\mathrm{T}}
        \dots
        E_2^{\mathrm{T}}
        E_1^{\mathrm{T}}
        P^{\mathrm{T}}
        )
        A
        (
        P
        E_1
        E_2
        \dots
        E_{w-1}
        E_w
        )
        )}
        \\
        = {} &
        \operatorname{pf}
        {(
            (P E_1 E_2 \dots E_{w-1} E_w)^{\mathrm{T}}
            A
            (P E_1 E_2 \dots E_{w-1} E_w)
            )}
        \\
        = {} &
        \operatorname{pf}
        {(
            M^{\mathrm{T}} A M
            )}.
    \end{align*}
    我们计算 \(M^{\mathrm{T}} A M\):
    \begin{align*}
        %  &
        [M^{\mathrm{T}} A M]_{i,j}
        % \\
        = {} &
        \sum_{k = 1}^{n}
        {
        [M^{\mathrm{T}}]_{i,k} [A M]_{k,j}
        }
        \\
        = {} &
        \sum_{k = 1}^{n}
        {
        [M]_{k,i} [A M]_{k,j}
        }
        \\
        = {} &
        [M]_{i,i} [A M]_{i,j}
        +
        \sum_{\substack{1 \leq k \leq n    \\k \neq i}}
        {
            [M]_{k,i} [A M]_{k,j}
        }
        \\
        = {} &
        [M]_{i,i} [A M]_{i,j}
        +
        \sum_{\substack{1 \leq k \leq n    \\k \neq i}}
        {
            0\, [A M]_{k,j}
        }
        \\
        = {} &
        [M]_{i,i}
            [A M]_{i,j}
        \\
        = {} &
        [M]_{i,i}
        \sum_{\ell = 1}^{n}
        {[A]_{i,\ell} [M]_{\ell,j}}
        \\
        = {} &
        [M]_{i,i}
        \Bigg(
        [A]_{i,j} [M]_{j,j}
        +
        \sum_{\substack{1 \leq \ell \leq n \\ \ell \neq j}}
        {[A]_{i,\ell} [M]_{\ell,j}}
        \Bigg)
        \\
        = {} &
        [M]_{i,i}
        \Bigg(
        [A]_{i,j} [M]_{j,j}
        +
        \sum_{\substack{1 \leq \ell \leq n \\ \ell \neq j}}
        {[A]_{i,\ell} \, 0}
        \Bigg)
        \\
        = {} &
        [M]_{i,i}
            [A]_{i,j}
            [M]_{j,j}.
    \end{align*}
    当 \(i \neq 1\), \(j \neq 1\) 时,
    \begin{align*}
        [M]_{i,i}
        [A]_{i,j}
        [M]_{j,j}
        = 1\, [A]_{i,j}\, 1
            = [A]_{i,j};
    \end{align*}
    当 \(i = 1\), \(j \neq 1\) 时,
    \begin{align*}
        [M]_{i,i}
        [A]_{i,j}
        [M]_{j,j}
        = \det {(P)}\, [A]_{i,j}\, 1
        = \det {(P)}\, [A]_{i,j};
    \end{align*}
    当 \(i \neq 1\), \(j = 1\) 时,
    \begin{align*}
        [M]_{i,i}
        [A]_{i,j}
        [M]_{j,j}
        = 1\, [A]_{i,j} \det {(P)}
        = \det {(P)}\, [A]_{i,j};
    \end{align*}
    当 \(i = 1\), \(j = 1\) 时,
    \begin{align*}
        [M]_{i,i}
        [A]_{i,j}
        [M]_{j,j}
        = \det {(P)}\, 0 \det {(P)}
        = 0
        = \det {(P)}\, [A]_{i,j}.
    \end{align*}
    则
    \begin{equation*}
        \operatorname{pf} {(
            P^{\mathrm{T}} A P
            )}
        =
        \operatorname{pf} {(
            M^{\mathrm{T}} A M
            )}
        =
        \det {(P)} \operatorname{pf} {(A)}
        =
        \operatorname{pf} {(A)} \det {(P)}.
        \qedhere
    \end{equation*}
\end{proof}

% 最后, 注意到 \((P^{\mathrm{T}})^{\mathrm{T}} = P\),
% 且 \(\det {(P^{\mathrm{T}})} = \det {(P)}\),
% 我们有

% \begin{theorem}
%     设 \(A\) 是 \(n\)~级反称阵.
%     设 \(P\) 是 \(n\)~级阵.
%     则
%     \begin{align*}
%         \operatorname{pf} {(P A P^{\mathrm{T}})}
%         = \operatorname{pf} {(A)} \det {(P)}.
%     \end{align*}
% \end{theorem}

\section{\texorpdfstring{辛阵的行列式为 \(1\)}
  {辛阵的行列式为 1}}

现在, 我们可以证明, 辛阵的行列式为 \(1\).

回想起, 说一个 \(2m\)~级阵 \(A\) 是辛阵,
若 \(A^{\mathrm{T}} K_m A = K_m\),
其中 \(K_m\) 是形如
\begin{align*}
    \begin{bmatrix}
        0    & I_m \\
        -I_m & 0   \\
    \end{bmatrix}
\end{align*}
的 \(2m\)~级反称阵.

% 不难算出, \(\operatorname{pf} {(K_m)} = (-1)^{m(m-1)/2}\).
% 则 \(\operatorname{pf} {(K_m)} \operatorname{pf} {(K_m)} = 1\).
不难算出, \(\det {(K_m)} = 1\).
则 \(\operatorname{pf} {(K_m)} \operatorname{pf} {(K_m)}
= \det {(K_m)} = 1\).
由 \(\operatorname{pf} {(K_m)}
= \operatorname{pf} {(A^{\mathrm{T}} K_m A)}
= \operatorname{pf} {(K_m)} \det {(A)}\),
知 \(\operatorname{pf} {(K_m)} \operatorname{pf} {(K_m)}
= \operatorname{pf} {(K_m)} \operatorname{pf} {(K_m)} \det {(A)}\),
即 \(1 = 1 \det {(A)} = \det {(A)}\).

当然, 还有不少别的证明辛阵的行列式为 \(1\) 的方法,
但我就不在这儿说它们了.
