\section{绝对值的性质 (1)}

设 \(a\) 是实数.
则 \(a\) 的绝对值
\begin{align*}
    |a| = \begin{cases}
              a,  & a \geq 0; \\
              -a, & a < 0.
          \end{cases}
\end{align*}

如下命题是正确的.

(1)
\(|0| = 0\);
若实数 \(a\) 适合 \(|a| = 0\), 则 \(a = 0\).

因为 \(0 \geq 0\),
故 \(|0| = 0\).

若 \(a > 0\), 则 \(|a| = a > 0\);
若 \(a < 0\), 则 \(|a| = -a > 0\).
于是, 若 \(|a| = 0\), 则 \(a\) 不能是正数,
也不能是负数, 故 \(a = 0\).

(2)
对任何实数 \(a\), 必 \(|a| \geq 0\).

我们已知, 若 \(a > 0\) 或 \(a < 0\), 则 \(|a| > 0\).
另外, \(|0| = 0\).

(3)
对任何实数 \(a\), 必 \(|a| \geq a \geq -|a|\).

若 \(a \geq 0\), 则 \(|a| = a\).
故 \(|a| = a \geq -a = -|a|\);
若 \(a < 0\), 则 \(|a| = -a\).
故 \(|a| = -a > a = -|a|\).

(4)
设 \(a\), \(b\) 是\emph{非负实数}.
若 \(a \geq b\), 则 \(a^2 \geq b^2\);
反过来, 若 \(a^2 \geq b^2\), 则 \(a \geq b\).

若 \(a > b \geq 0\), 则 \(a a > b b\), 故 \(a^2 > b^2\);
若 \(a = b \geq 0\), 则 \(a a = b b\), 故 \(a^2 = b^2\);
若 \(0 \leq a < b\), 则 \(a a < b b\), 故 \(a^2 < b^2\).
所以, 若 \(a \geq b\), 则 \(a^2 \geq b^2\);
反过来, 若 \(a^2 \geq b^2\), 则因 \(a < b\)
无法推出 \(a^2 \geq b^2\), 故必 \(a \geq b\).

(5)
对任何实数 \(a\), 必 \(|a|^2 = a^2\).
所以, 对任何实数 \(a\), 必 \(\sqrt{a^2} = |a|\).

若 \(a \geq 0\), 则 \(|a|^2 = a^2\);
若 \(a < 0\), 则 \(|a|^2 = (-a)^2 = a^2\).
因为\emph{非负实数} \(|a|\) 适合 \(|a|^2 = a^2\),
故由 \(\surd\) 的定义, \(\sqrt{a^2} = |a|\).

(6)
对任何实数 \(a\), \(b\), 必 \(|ab| = |a|\, |b|\).

若 \(a \geq 0\), \(b \geq 0\),
则 \(ab \geq 0\),
故 \(|ab| = ab = |a|\, |b|\);
若 \(a \geq 0\), \(b < 0\),
则 \(ab \leq 0\),
故 \(|ab| = -ab = a(-b) =|a|\, |b|\);
若 \(a < 0\), \(b \geq 0\),
则 \(ab \leq 0\),
故 \(|ab| = -ab = (-a)b = |a|\, |b|\);
若 \(a < 0\), \(b < 0\),
则 \(ab \geq 0\),
故 \(|ab| = ab = (-a)(-b) = |a|\, |b|\).

(7)
对任何实数 \(a\), \(b\), 必 \(|{a + b}| \leq |a| + |b|\).

若 \(a + b \geq 0\), 则 \(|{a + b}| = a + b \leq |a| + |b|\);
若 \(a + b < 0\), 则 \(|{a + b}| = -(a + b)
= (-a) + (-b) \leq |{-a}| + |{-b}| = |a| + |b|\).

类似地, 若 \(a_1\), \(\dots\), \(a_n\) 是实数, 则
\begin{align*}
    |{a_1 + \dots + a_n}| \leq |a_1| + \dots + |a_n|.
\end{align*}
可用数学归纳法证它.

(8)
对任何实数 \(a\), \(b\), 必 \(|{a - b}| \geq |a| - |b|\).

因为 \(|{a - b}| + |b| \geq |{(a - b) + b}| = |a|\).

\vspace{2ex}

这些事实会是有用的.

\section{绝对值的性质 (2)}

设 \(z = a + \mathrm{i} b\) 是复数
(其中, \(\mathrm{i}\) 是虚数单位,
\(a\), \(b\) 是实数, 下同).
则 \(z\) 的绝对值
\begin{align*}
    |z| = \sqrt{a^2 + b^2}.
\end{align*}

如下命题是正确的.

(1)
\(|0| = 0\);
若复数 \(z\) 适合 \(|z| = 0\), 则 \(z = 0\).

首先, \(|0| = |{0 + \mathrm{i} 0}| = \sqrt{0^2 + 0^2} = 0\).

若 \(z = a + \mathrm{i} b\) 适合 \(|z| = 0\),
则 \(\sqrt{a^2 + b^2} = 0\).
则 \(a^2 + b^2 = 0\).
则 \(a = b = 0\).
则 \(z = 0\).

(2)
对任何复数 \(z\), 必 \(|z| \geq 0\).

由 \(\surd\) 的定义, 这是显然的.

(3)
对任何复数 \(z\), \(w\), 必 \(|zw| = |z|\, |w|\).

设 \(z = a + \mathrm{i} b\), 且 \(w = c + \mathrm{i} d\).
则
\begin{align*}
    |zw|^2
    = {} & |{(ac - bd) + \mathrm{i} (ad + bc)}|^2 \\
    = {} & (ac - bd)^2 + (ad + bc)^2              \\
    = {} & a^2 c^2 + b^2 d^2 + a^2 d^2 + b^2 c^2  \\
    = {} & (a^2 + b^2) (c^2 + d^2)                \\
    = {} & |z|^2 |w|^2                            \\
    = {} & (|z|\, |w|)^2.
\end{align*}
因为 \(|zw|\) 与 \(|z|\, |w|\) 是非负实数,
故 \(|zw| = |z|\, |w|\).

顺便, 注意到, 对任何实数 \(a\), \(b\), \(c\), \(d\),
有
\begin{align*}
    (a^2 + b^2) (c^2 + d^2)
    = (ac - bd)^2 + (ad + bc)^2
    \geq (ac - bd)^2.
\end{align*}

(4)
对任何复数 \(z\), \(w\), 必 \(|{z + w}| \leq |z| + |w|\).

设 \(z = a + \mathrm{i} b\), 且 \(w = c + \mathrm{i} d\).
则
\begin{align*}
    |{z + w}|^2
    = {}    & |{(a + c) + \mathrm{i} (b + d)}|^2      \\
    = {}    & (a + c)^2 + (b + d)^2                   \\
    = {}    & a^2 + b^2 + c^2 + d^2 + 2(ac + bd)      \\
    = {}    & |z|^2 + |w|^2 + 2(ac + bd)              \\
    \leq {} & |z|^2 + |w|^2 + 2|{ac + bd}|            \\
    = {}    & |z|^2 + |w|^2 + 2 \sqrt{(ac + bd)^2}    \\
    = {}    & |z|^2 + |w|^2 + 2 \sqrt{(ac - (-b)d)^2} \\
    \leq {} & |z|^2 + |w|^2
    + 2 \sqrt{(a^2 + (-b)^2) (c^2 + d^2)}             \\
    \leq {} & |z|^2 + |w|^2 + 2 \sqrt{(|z|\,|w|)^2}   \\
    \leq {} & |z|^2 + |w|^2 + 2 \sqrt{|z|^2 |w|^2}    \\
    = {}    & |z|^2 + |w|^2 + 2 |z|\, |w|             \\
    = {}    & (|z| + |w|)^2.
\end{align*}
因为 \(|{z + w}|\) 与 \(|z| + |w|\) 是非负实数,
故 \(|{z + w}| \leq |z| + |w|\).

类似地, 若 \(z_1\), \(\dots\), \(z_n\) 是复数, 则
\begin{align*}
    |{z_1 + \dots + z_n}| \leq |z_1| + \dots + |z_n|.
\end{align*}
可用数学归纳法证它.

(5)
对任何复数 \(z\), \(w\), 必 \(|{z - w}| \geq |z| - |w|\).

因为 \(|{z - w}| + |w| \geq |{(z - w) + w}| = |z|\).

\vspace{2ex}

这些事实会是有用的.

\section{绝对值的性质 (3)}

设 \(z\), \(w\) 是数.
设 \(z\) 的绝对值是 \(|z|\).
则:

(1)
\(|0| = 0\);
若数 \(z\) 适合 \(|z| = 0\), 则 \(z = 0\).

(2)
\(|z| \geq 0\).

(3)
\(|zw| = |z|\, |w|\).

(4)
\(|{z + w}| \leq |z| + |w|\).

(5)
\(|{z - w}| \geq |z| - |w|\).

\vspace{2ex}

这些事实会是有用的.

\section{关于实数的大小的几个事实}

设 \(y_1\), \(y_2\), \(\dots\), \(y_n\) 是若干个实数.
我们总可从大到小地排它们,
故

\begin{theorem}
    设 \(y_1\), \(y_2\), \(\dots\), \(y_n\) 是若干个实数.
    则存在不超过 \(n\) 的正整数 \(i\),
    使对任何不超过 \(n\) 的正整数 \(k\),
    必 \(y_i \geq y_k\).
\end{theorem}

\begin{theorem}
    设 \(y_1\), \(y_2\), \(\dots\), \(y_n\) 是若干个实数
    (\(n \geq 2\)).

    (1)
    存在不超过 \(n\), 且不等的正整数 \(i\), \(j\),
    使对任何不超过 \(n\), 且不等于 \(i\) 的正整数 \(k\),
    必 \(y_i \geq y_j \geq y_k\).

    (2)
    对任何不超过 \(n\), 且不等于 \(i\) 的正整数 \(k\),
    必 \(y_j \geq y_k\).

    (3)
    对任何不超过 \(n\) 的正整数 \(k\),
    必 \(y_i \geq y_k\).
\end{theorem}

设 \(y_1\), \(y_2\), \(\dots\), \(y_n\) 是若干个\emph{非负}实数,
且不全是零.
从而有某不超过 \(n\) 的正整数 \(p\), 使 \(y_p > 0\).
再设某不超过 \(n\) 的正整数 \(i\) 适合:
对任何不超过 \(n\) 的正整数 \(k\),
必 \(y_i \geq y_k\).
则 \(y_i \geq y_p > 0\).
所以

\begin{theorem}
    设 \(y_1\), \(y_2\), \(\dots\), \(y_n\) 是若干个\emph{非负}实数,
    且\emph{不全是零}.
    则存在不超过 \(n\) 的正整数 \(i\),
    使对任何不超过 \(n\) 的正整数 \(k\),
    必 \(y_i \geq y_k\),
    且 \(y_i > 0\).
\end{theorem}

\begin{theorem}
    设 \(y_1\), \(y_2\), \(\dots\), \(y_n\) 是若干个\emph{非负}实数,
    且\emph{不全是零}
    (\(n \geq 2\)).

    (1)
    存在不超过 \(n\), 且不等的正整数 \(i\), \(j\),
    使对任何不超过 \(n\), 且不等于 \(i\) 的正整数 \(k\),
    必 \(y_i \geq y_j \geq y_k\),
    且 \(y_i > 0\).

    (2)
    对任何不超过 \(n\), 且不等于 \(i\) 的正整数 \(k\),
    必 \(y_j \geq y_k\).

    (3)
    对任何不超过 \(n\) 的正整数 \(k\),
    必 \(y_i \geq y_k\).

    (4)
    有一个特别的情形值得一提.
    设 \(y_j = 0\).
    由 (2) 知,
    对任何不超过 \(n\), 且不等于 \(i\) 的正整数 \(k\),
    必 \(0 = y_j \geq y_k \geq 0\).
    故对任何不超过 \(n\), 且不等于 \(i\) 的正整数 \(k\),
    必 \(y_k = 0\).
\end{theorem}

\section{行列式为零的阵的性质}

本节, 我们讨论行列式为零的阵的几个性质.

\begin{theorem}
    设 \(A\) 是一个 \(n\)~级阵.
    设 \(\det {(A)} = 0\).
    则存在不超过 \(n\) 的正整数 \(i\), 使
    \begin{equation}
        |[A]_{i,i}| \leq
        \sum_{\substack{1 \leq u \leq n \\
            u \neq i}} {|[A]_{i,u}|}.
        \label{eq:NonzeroDet1}
    \end{equation}
\end{theorem}

\begin{proof}
    因为 \(\det {(A)} = 0\),
    故有非零的 \(n \times 1\)~阵 \(x\),
    使 \(Ax = 0\)
    (见第一章, 节~\malneprasekcio{23}).

    考虑不全为零的非负实数
    \(|[x]_{1,1}|\), \(|[x]_{2,1}|\), \(\dots\), \(|[x]_{n,1}|\).
    则有不超过 \(n\) 的正整数 \(i\),
    使对任何不超过 \(n\) 的正整数 \(u\),
    有 \(|[x]_{i,1}| \geq |[x]_{u,1}|\),
    且 \(|[x]_{i,1}| > 0\).
    则
    \begin{align*}
        0
        = {} &
        [0]_{i,1}
        \\
        = {} &
        [A x]_{i,1}
        \\
        = {} &
        \sum_{1 \leq u \leq n}
        {[A]_{i,u} [x]_{u,1}}
        \\
        = {} &
        \sum_{\substack{1 \leq u \leq n \\ u \neq i}}
        {[A]_{i,u} [x]_{u,1}}
        + [A]_{i,i} [x]_{i,1}.
    \end{align*}
    则
    \begin{align*}
        |[A]_{i,i}|\,|[x]_{i,1}|
        = {}    &
        |{[A]_{i,i} [x]_{i,1}}|
        \\
        = {}    &
        |{-[A]_{i,i} [x]_{i,1}}|
        \\
        = {}    &
        \Bigg|
        \sum_{\substack{1 \leq u \leq n \\ u \neq i}}
        {[A]_{i,u} [x]_{u,1}}
        \Bigg|
        \\
        \leq {} &
        \sum_{\substack{1 \leq u \leq n \\ u \neq i}}
        {|{[A]_{i,u} [x]_{u,1}}|}
        \\
        = {}    &
        \sum_{\substack{1 \leq u \leq n \\ u \neq i}}
        {|[A]_{i,u}|\, |[x]_{u,1}|}
        \\
        \leq {} &
        \sum_{\substack{1 \leq u \leq n \\ u \neq i}}
        {|[A]_{i,u}|\, |[x]_{i,1}|}
        \\
        = {}    &
        \Bigg(
        \sum_{\substack{1 \leq u \leq n \\ u \neq i}}
        {|[A]_{i,u}|}
        \Bigg) |[x]_{i,1}|.
    \end{align*}
    因为 \(|[x]_{i,1}| > 0\),
    故式~\eqref{eq:NonzeroDet1} 是对的.
\end{proof}

不过, 此事反过来不一定是对的.

\begin{example}
    设 \(A =
    \begin{bmatrix}
        0  & 1 \\
        -1 & 0 \\
    \end{bmatrix}\)
    是一个 \(2\)~级阵.
    取 \(i = 1\) 或 \(i = 2\), 即有
    \begin{align*}
        |[A]_{i,1}|
        = 0
        \leq 1
        = \sum_{\substack{1 \leq u \leq 2 \\u \neq i}} {|[A]_{i,u}|}.
    \end{align*}
    可是, \(\det {(A)} = 1 \neq 0\).
\end{example}

\begin{theorem}
    设 \(A\) 是一个 \(n\)~级阵 (\(n \geq 2\)).
    设 \(\det {(A)} = 0\).
    则存在不超过 \(n\), 且不等的正整数 \(j\), \(k\), 使
    \begin{equation}
        |[A]_{j,j}|\,|[A]_{k,k}| \leq
        \Bigg(
        \sum_{\substack{1 \leq p \leq n \\
                p \neq j}} {|[A]_{j,p}|}
        \Bigg)
        \Bigg(
        \sum_{\substack{1 \leq q \leq n \\
                q \neq k}} {|[A]_{k,q}|}
        \Bigg).
        \label{eq:NonzeroDet2}
    \end{equation}
\end{theorem}

\begin{proof}
    因为 \(\det {(A)} = 0\),
    故有非零的 \(n \times 1\)~阵 \(x\),
    使 \(Ax = 0\)
    (见第一章, 节~\malneprasekcio{23}).

    考虑不全为零的非负实数
    \(|[x]_{1,1}|\), \(|[x]_{2,1}|\), \(\dots\), \(|[x]_{n,1}|\).
    则有不超过 \(n\), 且不等的正整数 \(j\), \(k\),
    使对任何不超过 \(n\), 且不等于 \(j\) 的正整数 \(p\),
    有 \(|[x]_{j,1}| \geq |[x]_{k,1}| \geq |[x]_{p,1}|\),
    且 \(|[x]_{j,1}| > 0\).

    注意到, 对任何不超过 \(n\) 的正整数 \(u\), 有
    \begin{align*}
        0
        = {} &
        [0]_{u,1}
        \\
        = {} &
        [A x]_{u,1}
        \\
        = {} &
        \sum_{1 \leq v \leq n}
        {[A]_{u,v} [x]_{v,1}}
        \\
        = {} &
        \sum_{\substack{1 \leq v \leq n \\ v \neq u}}
        {[A]_{u,v} [x]_{v,1}}
        + [A]_{u,u} [x]_{u,1}.
    \end{align*}
    故
    \begin{align*}
        -[A]_{u,u} [x]_{u,1}
        = \sum_{\substack{1 \leq v \leq n \\ v \neq u}}
        {[A]_{u,v} [x]_{v,1}}.
    \end{align*}
    则
    \begin{align*}
        |[A]_{u,u}|\, |[x]_{u,1}|
        = {}    &
        |{[A]_{u,u} [x]_{u,1}}|
        \\
        = {}    &
        |{-[A]_{u,u} [x]_{u,1}}|
        \\
        = {}    &
        \Bigg|
        \sum_{\substack{1 \leq v \leq n \\ v \neq u}}
        {[A]_{u,v} [x]_{v,1}}
        \Bigg|
        \\
        \leq {} &
        \sum_{\substack{1 \leq v \leq n \\ v \neq u}}
        {|{[A]_{u,v} [x]_{v,1}}|}
        \\
        = {}    &
        \sum_{\substack{1 \leq v \leq n \\ v \neq u}}
        {|{[A]_{u,v}|\, |[x]_{v,1}}|}.
    \end{align*}
    则
    (取 \(u = j\),
    并注意到对任何不超过 \(n\), 且不等于 \(j\) 的正整数 \(p\),
    有 \(|[x]_{p,1}| \leq |[x]_{k,1}|\))
    \begin{align*}
        |[A]_{j,j}|\, |[x]_{j,1}|
        \leq {} &
        \sum_{\substack{1 \leq p \leq n \\ p \neq j}}
        {|{[A]_{j,p}|\, |[x]_{p,1}}|}
        \\
        \leq {} &
        \sum_{\substack{1 \leq p \leq n \\ p \neq j}}
        {|{[A]_{j,p}|\, |[x]_{k,1}}|}
        \\
        = {}    &
        \Bigg(
        \sum_{\substack{1 \leq p \leq n \\ p \neq j}}
        {|{[A]_{j,p}|}
        \Bigg)
        |[x]_{k,1}}|,
    \end{align*}
    且
    (取 \(u = k\),
    并注意到对任何不超过 \(n\) 的正整数 \(q\),
    有 \(|[x]_{q,1}| \leq |[x]_{j,1}|\))
    \begin{align*}
        |[A]_{k,k}|\, |[x]_{k,1}|
        \leq {} &
        \sum_{\substack{1 \leq q \leq n \\ q \neq k}}
        {|{[A]_{k,q}|\, |[x]_{q,1}}|}
        \\
        \leq {} &
        \sum_{\substack{1 \leq q \leq n \\ q \neq k}}
        {|{[A]_{k,q}|\, |[x]_{j,1}}|}
        \\
        = {}    &
        \Bigg(
        \sum_{\substack{1 \leq q \leq n \\ q \neq k}}
        {|{[A]_{k,q}|}
        \Bigg)
        |[x]_{j,1}}|.
    \end{align*}
    故
    \begin{align*}
                &
        (|[A]_{j,j}|\, |[A]_{k,k}|)\,
        (|[x]_{j,1}|\, |[x]_{k,1}|)
        \\
        \leq {} &
        \Bigg(
        \sum_{\substack{1 \leq p \leq n \\ p \neq j}}
        {|{[A]_{j,p}|}}
        \Bigg)
        \Bigg(
        \sum_{\substack{1 \leq q \leq n \\ q \neq k}}
        {|{[A]_{k,q}|}}
        \Bigg)
        (|[x]_{j,1}|\, |[x]_{k,1}|).
    \end{align*}
    因为 \(|[x]_{j,1}| > 0\),
    故
    \begin{align*}
        % &
        (|[A]_{j,j}|\, |[A]_{k,k}|)\,
        |[x]_{k,1}|
        % \\
        % \leq {} &
        \leq
        \Bigg(
        \sum_{\substack{1 \leq p \leq n \\ p \neq j}}
        {|{[A]_{j,p}|}}
        \Bigg)
        \Bigg(
        \sum_{\substack{1 \leq q \leq n \\ q \neq k}}
        {|{[A]_{k,q}|}}
        \Bigg)
        |[x]_{k,1}|.
    \end{align*}

    若 \(|[x]_{k,1}| > 0\),
    则式~\eqref{eq:NonzeroDet2} 当然是对的.
    若 \(|[x]_{k,1}| = 0\),
    则对任何不超过 \(n\), 且不等于 \(j\) 的正整数 \(p\),
    有 \(0 = |[x]_{k,1}| \geq |[x]_{p,1}| \geq 0\).
    故 \(|[x]_{p,1}| = 0\).
    故 \([x]_{p,1} = 0\).
    则
    \begin{align*}
        -[A]_{j,j} [x]_{j,1}
        = \sum_{\substack{1 \leq p \leq n \\ p \neq j}}
        {[A]_{j,p} [x]_{p,1}} = 0.
    \end{align*}
    因为 \(|[x]_{j,1}| > 0\),
    故 \([x]_{j,1} \neq 0\).
    从而 \([A]_{j,j} = 0\).
    则式~\eqref{eq:NonzeroDet2} 的左侧是零.
\end{proof}

此事反过来也不一定是对的:
考虑上个例的 \(A\) 即可.

我们说, 若存在不超过 \(n\), 且不等的正整数 \(j\), \(k\),
使式~\eqref{eq:NonzeroDet2} 是对的,
则存在不超过 \(n\) 的正整数 \(i\),
使式~\eqref{eq:NonzeroDet1} 是对的.
用反证法, 不难看出, 若式~\eqref{eq:NonzeroDet2} 是对的,
则
\begin{align*}
    |[A]_{j,j}| >
    \sum_{\substack{1 \leq p \leq n \\
        p \neq j}} {|[A]_{j,p}|}
\end{align*}
与
\begin{align*}
    |[A]_{k,k}| >
    \sum_{\substack{1 \leq q \leq n \\
        q \neq k}} {|[A]_{k,q}|}
\end{align*}
不能全是对的.
故
\begin{align*}
    |[A]_{j,j}| \leq
    \sum_{\substack{1 \leq p \leq n \\
        p \neq j}} {|[A]_{j,p}|}
\end{align*}
与
\begin{align*}
    |[A]_{k,k}| \leq
    \sum_{\substack{1 \leq q \leq n \\
        q \neq k}} {|[A]_{k,q}|}
\end{align*}
的至少一个是对的.
取 \(i\) 为 \(j\) 或 \(k\) 即可.

\vspace{2ex}

或许, 您会想:
\begin{quotation}
    设 \(A\) 是一个 \(n\)~级阵 (\(n \geq 3\)).
    设 \(\det {(A)} = 0\).
    则存在不超过 \(n\),
    且互不相同的正整数 \(j\), \(k\), \(\ell\), 使
    \begin{align*}
                &
        |[A]_{j,j}|\,|[A]_{k,k}|\, |[A]_{\ell,\ell}|
        \\
        \leq {} &
        \Bigg(
        \sum_{\substack{1 \leq p \leq n \\
                p \neq j}} {|[A]_{j,p}|}
        \Bigg)
        \Bigg(
        \sum_{\substack{1 \leq q \leq n \\
                q \neq k}} {|[A]_{k,q}|}
        \Bigg)
        \Bigg(
        \sum_{\substack{1 \leq r \leq n \\
                r \neq \ell}} {|[A]_{\ell,r}|}
        \Bigg).
    \end{align*}
\end{quotation}
不过, 这不是对的.

\begin{example}
    设 \(A =
    \begin{bmatrix}
        1 & 1 & 1  \\
        1 & 1 & 1  \\
        1 & 1 & 10 \\
    \end{bmatrix}\)
    是一个 \(3\)~级阵.
    不难算出, \(\det {(A)} = 0\).
    可是
    (此处 \(j\), \(k\), \(\ell\) 分别是 \(1\), \(2\), \(3\))
    \begin{align*}
             &
        |[A]_{j,j}|\,|[A]_{k,k}|\, |[A]_{\ell,\ell}|
        \\
        = {} &
        1 \cdot 1 \cdot 10
        \\
        > {} &
        2 \cdot 2 \cdot 2
        \\
        = {} &
        \Bigg(
        \sum_{\substack{1 \leq p \leq 3 \\
                p \neq j}} {|[A]_{j,p}|}
        \Bigg)
        \Bigg(
        \sum_{\substack{1 \leq q \leq 3 \\
                q \neq k}} {|[A]_{k,q}|}
        \Bigg)
        \Bigg(
        \sum_{\substack{1 \leq r \leq 3 \\
                r \neq \ell}} {|[A]_{\ell,r}|}
        \Bigg).
    \end{align*}
    \(j\), \(k\), \(\ell\) 取其他的数时,
    也有类似的结果.
\end{example}

最后, 注意到, 一个阵与其转置的行列式相等:
若 \(\det {(A)} = 0\),
则 \(\det {(A^{\mathrm{T}})} = \det {(A)} = 0\).
应用前二个定理于 \(A^{\mathrm{T}}\), 立得

\begin{theorem}
    设 \(A\) 是一个 \(n\)~级阵.
    设 \(\det {(A)} = 0\).
    则存在不超过 \(n\) 的正整数 \(i\), 使
    \begin{align*}
        |[A]_{i,i}| \leq
        \sum_{\substack{1 \leq u \leq n \\
            u \neq i}} {|[A]_{u,i}|}.
    \end{align*}
\end{theorem}

\begin{theorem}
    设 \(A\) 是一个 \(n\)~级阵 (\(n \geq 2\)).
    设 \(\det {(A)} = 0\).
    则存在不超过 \(n\), 且不等的正整数 \(j\), \(k\), 使
    \begin{align*}
        |[A]_{j,j}|\,|[A]_{k,k}| \leq
        \Bigg(
        \sum_{\substack{1 \leq p \leq n \\
                p \neq j}} {|[A]_{p,j}|}
        \Bigg)
        \Bigg(
        \sum_{\substack{1 \leq q \leq n \\
                q \neq k}} {|[A]_{q,k}|}
        \Bigg).
    \end{align*}
\end{theorem}

\section{判断阵的行列式不是零的方法}

我们知道, 若一个阵的行列式是零,
则有形如式~\eqref{eq:NonzeroDet1}, \eqref{eq:NonzeroDet2} 的不等式成立.
利用反证法, 我们立得判断阵的行列式不是零的方法:

\begin{theorem}
    设 \(A\) 是一个 \(n\)~级阵.
    设对任何不超过 \(n\) 的正整数 \(i\), 必
    \begin{align*}
        |[A]_{i,i}| >
        \sum_{\substack{1 \leq u \leq n \\
            u \neq i}} {|[A]_{i,u}|}.
    \end{align*}
    则 \(\det {(A)} \neq 0\).
\end{theorem}

\begin{theorem}
    设 \(A\) 是一个 \(n\)~级阵 (\(n \geq 2\)).
    设对任何不超过 \(n\), 且不等的正整数 \(j\), \(k\), 必
    \begin{align*}
        |[A]_{j,j}|\,|[A]_{k,k}| >
        \Bigg(
        \sum_{\substack{1 \leq p \leq n \\
                p \neq j}} {|[A]_{j,p}|}
        \Bigg)
        \Bigg(
        \sum_{\substack{1 \leq q \leq n \\
                q \neq k}} {|[A]_{k,q}|}
        \Bigg).
    \end{align*}
    则 \(\det {(A)} \neq 0\).
\end{theorem}

当然, 如下命题也是正确的 (利用转置):

\begin{theorem}
    设 \(A\) 是一个 \(n\)~级阵.
    设对任何不超过 \(n\) 的正整数 \(i\), 必
    \begin{align*}
        |[A]_{i,i}| >
        \sum_{\substack{1 \leq u \leq n \\
            u \neq i}} {|[A]_{u,i}|}.
    \end{align*}
    则 \(\det {(A)} \neq 0\).
\end{theorem}

\begin{theorem}
    设 \(A\) 是一个 \(n\)~级阵 (\(n \geq 2\)).
    设对任何不超过 \(n\), 且不等的正整数 \(j\), \(k\), 必
    \begin{align*}
        |[A]_{j,j}|\,|[A]_{k,k}| >
        \Bigg(
        \sum_{\substack{1 \leq p \leq n \\
                p \neq j}} {|[A]_{p,j}|}
        \Bigg)
        \Bigg(
        \sum_{\substack{1 \leq q \leq n \\
                q \neq k}} {|[A]_{q,k}|}
        \Bigg).
    \end{align*}
    则 \(\det {(A)} \neq 0\).
\end{theorem}

这些定理的一个应用或许是,
对于一类阵,
即使我们不计算它的行列式,
我们也可判断它的行列式不是零.
这是好的.

\begin{example}\label{emp:NonzeroDet1}
    设 \(A =
    \begin{bmatrix}
        9 & 6 & 2 \\
        4 & 8 & 3 \\
        5 & 1 & 7 \\
    \end{bmatrix}\)
    是一个 \(3\)~级阵.
    不难算出, \(|9| > |6| + |2|\),
    \(|8| > |4| + |3|\),
    且 \(|7| > |5| + |1|\).
    于是, \(\det {(A)} \neq 0\).
    (其实, 不难算出, \(\det {(A)} = 327.\))
\end{example}

有时, 一个方阵 \(A\) 可能不适合定理的条件, 故我们无法直接地用定理.
不过, 既然我们研究是否 \(\det {(A)} \neq 0\),
我们可找二个跟 \(A\) 同尺寸的阵 \(B\), \(C\),
使 \(BAC\) 适合定理的条件.
则 \(\det {(BAC)} \neq 0\).
因为 \(0 \neq \det {(BAC)} = \det {(B)} \det {(A)} \det {(C)}\),
必 \(\det {(A)} \neq 0\).

\begin{example}\label{emp:NonzeroDet2}
    设 \(A =
    \begin{bmatrix}
        20 & 11 & 1 \\
        4  & 15 & 1 \\
        13 & 15 & 3 \\
    \end{bmatrix}\)
    是一个 \(3\)~级阵.
    不难验算, 我们无法直接地用前 4~个定理的任何一个%
    判断 \(\det {(A)}\) 是否非零.

    取
    \begin{align*}
        B = \begin{bmatrix}
                1 & 0 & 0 \\
                0 & 2 & 0 \\
                0 & 0 & 1 \\
            \end{bmatrix},
        \quad
        C = \begin{bmatrix}
                1 & 0 & 0  \\
                0 & 1 & 0  \\
                0 & 0 & 10 \\
            \end{bmatrix}.
    \end{align*}
    则
    \begin{align*}
        BAC = \begin{bmatrix}
                  20 & 11 & 10 \\
                  8  & 30 & 20 \\
                  13 & 15 & 30 \\
              \end{bmatrix}.
    \end{align*}
    不难算出
    \begin{align*}
         & R_1 = \sum_{\substack{1 \leq u \leq 3 \\u \neq 1}}
        {|[BAC]_{1,u}|} = |11| + |10| = 21,      \\
         & R_2 = \sum_{\substack{1 \leq u \leq 3 \\u \neq 2}}
        {|[BAC]_{2,u}|} = |8| + |20| = 28,       \\
         & R_3 = \sum_{\substack{1 \leq u \leq 3 \\u \neq 3}}
        {|[BAC]_{3,u}|} = |13| + |15| = 28,
    \end{align*}
    且
    \begin{align*}
         & |[BAC]_{1,1}|\,|[BAC]_{2,2}|
        = 600 > 588 = R_1 R_2,          \\
         & |[BAC]_{1,1}|\,|[BAC]_{3,3}|
        = 600 > 588 = R_1 R_3,          \\
         & |[BAC]_{2,2}|\,|[BAC]_{3,3}|
        = 30^2 > 28^2 = R_2 R_3.
    \end{align*}
    (注意到, 既然我们已算出,
    每个 \(|[BAC]_{j,j}|\,|[BAC]_{k,k}|\)
    都大于 \(R_j R_k\)
    (\(j < k\)),
    由乘法的交换律,
    我们不必再判断%
    每个 \(|[BAC]_{j,j}|\,|[BAC]_{k,k}|\)
    是否都大于 \(R_j R_k\)
    (\(j > k\)).)
    故 \(\det {(BAC)} \neq 0\).
    则 \(\det {(A)} \neq 0\).
    (其实, 不难算出, \(\det {(A)} = 476.\))
\end{example}

\begin{example}\label{emp:NonzeroDet3}
    设 \(n \geq 2\).
    设 \(n\)~级阵 \(A\) 适合
    \begin{align*}
        [A]_{i,j} =
        \begin{cases}
            2,
             & 1 \leq i = j \leq n;         \\
            \text{\(1\) 或 \(-1\)},
             & 1 \leq i = j - 1 \leq n - 1; \\
            \text{\(1\) 或 \(-1\)},
             & 2 \leq i = j + 1 \leq n;     \\
            0,
             & \text{其他}.
        \end{cases}
    \end{align*}
    形象地, 当 \(n = 4\) 时, \(A\) 形如
    \begin{align*}
        \begin{bmatrix}
            2     & \pm 1 & 0     & 0     \\
            \pm 1 & 2     & \pm 1 & 0     \\
            0     & \pm 1 & 2     & \pm 1 \\
            0     & 0     & \pm 1 & 2     \\
        \end{bmatrix}.
    \end{align*}
    (这里, 正负号可自由地组合.)

    不难算出, 对 \(i = 1\) 或 \(i = n\), 有
    \begin{align*}
        |[A]_{i,i}| = 2 > 1
        = \sum_{\substack{1 \leq u \leq n \\ u \neq i}}
        {|[A]_{i,u}|};
    \end{align*}
    对 \(1 < i < n\), 有
    \begin{align*}
        |[A]_{i,i}| = 2 = 2
        = \sum_{\substack{1 \leq u \leq n \\ u \neq i}}
        {|[A]_{i,u}|}.
    \end{align*}
    于是, 对 \(n = 2\), 我们可用第~1~个定理,
    得到 \(\det {(A)} \neq 0\);
    对 \(n = 3\), 我们可用第~2~个定理,
    得到 \(\det {(A)} \neq 0\);
    对 \(n \geq 4\), 这 4~个定理无法直接地被使用.

    我们试找一个 \(n\)~级阵 \(C\),
    使 \(AC\) 适合某个定理的条件,
    从而有 \(\det {(A)} \neq 0\).
    无妨设
    \begin{align*}
        C =
        \begin{bmatrix}
            c_1    & 0      & \cdots & 0      \\
            0      & c_2    & \cdots & 0      \\
            \vdots & \vdots & \ddots & \vdots \\
            0      & 0      & \cdots & c_n    \\
        \end{bmatrix},
    \end{align*}
    其中 \(c_1\), \(c_2\), \(\dots\), \(c_n\) 是待定的非零数.
    具体地,
    \begin{align*}
        [C]_{i,j}
        = \begin{cases}
              c_i, & 1 \leq i = j \leq n; \\
              0,   & \text{其他}.
          \end{cases}
    \end{align*}
    考虑到, 在后面的计算中, \(c_i\) 会被经常地取绝对值.
    既然如此, 为方便, 我们不如要求 \(c_i > 0\).
    不难算出, \([AC]_{i,j} = c_j [A]_{i,j}\);
    特别地, \(|[AC]_{i,i}| = 2c_i\).
    记
    \begin{align*}
        R_i = \sum_{\substack{1 \leq u \leq n \\ u \neq i}}
        {|[AC]_{i,u}|}.
    \end{align*}
    则
    \begin{align*}
        R_i =
        \begin{cases}
            c_2,               & i = 1;     \\
            c_{i-1} + c_{i+1}, & 1 < i < n; \\
            c_{n-1},           & i = n.
        \end{cases}
    \end{align*}
    我们希望, 找一组正数 \(c_1\), \(c_2\), \(\dots\), \(c_n\),
    使 \(|[AC]_{i,i}| > R_i\), 即
    \begin{align*}
        2c_1     & > c_2,           \\
        2c_2     & > c_1 + c_3,     \\
        2c_3     & > c_2 + c_4,     \\
        \dots \dots \dots
                 &
        \dots \dots \dots \dots,    \\
        2c_{n-1} & > c_{n-2} + c_n, \\
        2c_n     & > c_{n-1}.
    \end{align*}
    这些不等式相当于
    \begin{align*}
        c_1               & > c_2 - c_1,     \\
        c_2 - c_1         & > c_3 - c_2,     \\
        c_3 - c_2         & > c_4 - c_3,     \\
        \dots \dots \dots \dots
                          &
        \dots \dots \dots \dots \dots,       \\
        c_{n-1} - c_{n-2} & > c_n - c_{n-1}, \\
        c_n - c_{n-1}     & > -c_n.
    \end{align*}
    我们可试取
    \begin{align*}
        c_1               &
        = \frac{1}{1 \cdot 2}
        = 1 - \frac{1}{2},                   \\
        c_2 - c_1         &
        = \frac{1}{2 \cdot 3}
        = \frac{1}{2} - \frac{1}{3},         \\
        c_3 - c_2         &
        = \frac{1}{3 \cdot 4}
        = \frac{1}{3} - \frac{1}{4},         \\
        \dots \dots \dots \dots
                          &
        \dots \dots \dots \dots \dots \dots, \\
        c_{n-1} - c_{n-2} &
        = \frac{1}{(n-1)n}
        = \frac{1}{n-1} - \frac{1}{n},       \\
        c_n - c_{n-1}     &
        = \frac{1}{n(n+1)}
        = \frac{1}{n} - \frac{1}{n+1}.
    \end{align*}
    即
    \begin{align*}
        c_k = 1 - \frac{1}{k+1} = \frac{k}{k+1},
        \quad
        \text{\(k = 1\), \(2\), \(\dots\), \(n\)}.
    \end{align*}
    不难验证, \(c_k\) 是正数, 且
    \begin{align*}
        |[AC]_{1,1}| - R_1
         & = 2c_1 - c_2
        = 1 - \frac{2}{3} > 0,                 \\
        |[AC]_{n,n}| - R_n
         & = 2c_n - c_{n-1}
        = \frac{n-1}{n+1} + \frac{1}{n} > 0,   \\
        |[AC]_{i,i}| - R_i
         & = (c_i - c_{i-1}) - (c_{i+1} - c_i)
        = \frac{1}{i(i+1)} - \frac{1}{(i+1)(i+2)} > 0.
    \end{align*}
    故 \(AC\) 适合第~1~个定理的条件.
    则 \(\det {(AC)} \neq 0\).
    故 \(\det {(A)} \neq 0\).
\end{example}

\section{判断实阵的行列式大于零的方法}

最后, 我介绍判断实阵的行列式大于零的方法.

\begin{theorem}
    设 \(A\) 是一个 \(n\)~级\emph{实阵}.
    设对任何不超过 \(n\) 的正整数 \(i\), 必
    \begin{align*}
        [A]_{i,i} >
        \sum_{\substack{1 \leq u \leq n \\
            u \neq i}} {|[A]_{i,u}|}.
    \end{align*}
    则 \(\det {(A)} > 0\).
\end{theorem}

% 若我们巧用微积分, 此事是简单的.

% \begin{proof}
%     既然 \([A]_{i,i}\) 大于一个非负数,
%     则 \(0 < [A]_{i,i} = |[A]|_{i,i}|\).

%     作 \(n \times n\)~实阵 \(A_t\) 如下:
%     \begin{align*}
%         [A_t]_{i,j} =
%         \begin{cases}
%             [A]_{i,i},   & 1 \leq i = j \leq n; \\
%             t [A]_{i,j}, & \text{其他}.
%         \end{cases}
%     \end{align*}
%     作函数
%     \begin{align*}
%         \text{\(f\):} \quad
%         [0, 1] & \to \mathbb{R},       \\
%         t      & \mapsto \det {(A_t)}.
%     \end{align*}
%     对任何 \(0 \leq t \leq 1\),
%     与任何不超过 \(n\) 的正整数 \(i\),
%     \begin{align*}
%         |[A_t]_{i,i}|
%         = {}    &
%         |[A]_{i,i}|
%         \\
%         = {}    &
%         [A]_{i,i}
%         \\
%         > {}    &
%         \sum_{\substack{1 \leq u \leq n   \\
%             u \neq i}} {|[A]_{i,u}|}
%         \\
%         \geq {} &
%         t \sum_{\substack{1 \leq u \leq n \\
%             u \neq i}} {|[A]_{i,u}|}.
%         \\
%         = {}    &
%         \sum_{\substack{1 \leq u \leq n   \\
%             u \neq i}} {t |[A]_{i,u}|}.
%         \\
%         = {}    &
%         \sum_{\substack{1 \leq u \leq n   \\
%             u \neq i}} {|t|\, |[A]_{i,u}|}.
%         \\
%         = {}    &
%         \sum_{\substack{1 \leq u \leq n   \\
%             u \neq i}} {|[A_t]_{i,u}|}.
%     \end{align*}
%     故 \(f(t) = \det {(A_t)} \neq 0\),
%     对任何 \(0 \leq t \leq 1\).

%     因为 \(\det {(A_t)}\) 是关于 \(t\) 的整式,
%     故 \(f\) 是连续函数.
%     注意到, \(f(0) = [A]_{1,1} [A]_{2,2} \dots [A]_{n,n} > 0\);
%     又注意到, \(f(1) = \det {(A)}\).

%     我们用反证法说明, \(f(1) > 0\).
%     反设 \(f(1) \leq 0\).
%     若 \(f(1) = 0\), 这是矛盾.
%     若 \(f(1) < 0\), 则因 \(f(1) > 0\),
%     必有某 \(0 < s < 1\) 使 \(f(s) = 0\);
%     这又是矛盾.

%     故 \(f(1) > 0\).
% \end{proof}

% 不过, 若我们不用微积分, 此事是较复杂的.

我想给二个证明.
第~1~个证明的计算较多, 但它是较直接的.

\begin{proof}[法~1]
    作命题 \(P(n)\):
    \begin{quotation}
        对任何适合如下条件的 \(n\)~级实阵 \(A\),
        必 \(\det {(A)} > 0\):
        \begin{quotation}
            对任何不超过 \(n\)~的正整数 \(i\),
            必
            \begin{align*}
                [A]_{i,i} >
                \sum_{\substack{1 \leq u \leq n \\
                    u \neq i}} {|[A]_{i,u}|}.
            \end{align*}
        \end{quotation}
    \end{quotation}
    我们用数学归纳法证明,
    对任何正整数 \(n\),
    \(P(n)\) 是对的.

    \(P(1)\) 是对的; 这是显然的.

    \(P(2)\) 是对的.
    任取 \(2\)~级实阵
    \begin{align*}
        A = \begin{bmatrix}
                a & b \\
                c & d \\
            \end{bmatrix},
    \end{align*}
    其中 \(a > |b|\), 且 \(d > |c|\).
    则
    \begin{align*}
        \det {(A)} = ad - bc > |b|\,|c| - bc
        = |bc| - bc \geq 0.
    \end{align*}

    设 \(P(n-1)\) 是对的.
    我们要由此证 \(P(n)\) 是对的.

    设 \(A\) 是一个 \(n\)~级实阵,
    且对任何不超过 \(n\) 的正整数 \(i\), 必
    \begin{align*}
        [A]_{i,i} >
        \sum_{\substack{1 \leq u \leq n \\
            u \neq i}} {|[A]_{i,u}|}.
    \end{align*}
    因为 \([A]_{i,i}\) 大于一个非负数,
    故 \(0 < [A]_{i,i} = |[A]_{i,i}|\).
    作 \(n\)~级实阵 \(B_1 = A\).
    则 \([B_1]_{i,j} = [A]_{i,j}\),
    且 \(\det {(B_1)} = \det {(A)}\).

    我们加 \(B_1\) 的行~\(1\) 的 \(-[A]_{2,1} / [A]_{1,1}\)~倍%
    到 \(B_1\) 的行~\(2\), 得 \(n\)~级实阵 \(B_2\).
    则
    \begin{align*}
        [B_2]_{i,j}  & = [B_1]_{i,j},
                     & \quad i < 2;                        \\
        [B_2]_{2,1}  & = 0;                                \\
        [B_2]_{2,j}  & =
        [A]_{2,j} - \frac{[A]_{2,1}}{[A]_{1,1}} [A]_{1,j}; \\
        \det {(B_2)} & =
        \det {(B_1)} = \det {(A)}.
    \end{align*}

    我们加 \(B_2\) 的行~\(1\) 的 \(-[A]_{3,1} / [A]_{1,1}\)~倍%
    到 \(B_2\) 的行~\(3\), 得 \(n\)~级实阵 \(B_3\).
    则
    \begin{align*}
        [B_3]_{i,j}  & = [B_2]_{i,j},
                     & \quad i < 3;                        \\
        [B_3]_{3,1}  & = 0;                                \\
        [B_3]_{3,j}  & =
        [A]_{3,j} - \frac{[A]_{3,1}}{[A]_{1,1}} [A]_{1,j}; \\
        \det {(B_3)} & =
        \det {(B_2)} = \det {(A)}.
    \end{align*}

    \(\dots \dots\)

    我们加 \(B_{n-1}\) 的行~\(1\) 的 \(-[A]_{n,1} / [A]_{1,1}\)~倍%
    到 \(B_{n-1}\) 的行~\(n\), 得 \(n\)~级实阵 \(B_n\).
    则
    \begin{align*}
        [B_n]_{i,j}  & = [B_{n-1}]_{i,j},
                     & \quad i < n;                        \\
        [B_n]_{n,1}  & = 0;                                \\
        [B_n]_{n,j}  & =
        [A]_{n,j} - \frac{[A]_{n,1}}{[A]_{1,1}} [A]_{1,j}; \\
        \det {(B_n)} & =
        \det {(B_{n-1})} = \det {(A)}.
    \end{align*}

    综上, 我们作出了一个 \(n\)~级实阵 \(B_n\) 适合如下条件:
    \begin{align*}
        [B_n]_{1,j}  & = [A]_{1,j};                 \\
        [B_n]_{i,1}  & = 0,          & \quad i > 1; \\
        [B_n]_{i,j}  & =
        [A]_{i,j} - \frac{[A]_{i,1}}{[A]_{1,1}} [A]_{1,j},
                     & \quad i > 1;                 \\
        \det {(B_n)} & = \det {(A)}.
    \end{align*}
    作 \(n-1\)~级实阵 \(C = B_n (1|1)\).
    则
    \begin{align*}
        [C]_{i,j}  & = [B_n]_{i+1,j+1}
        = [A]_{i+1,j+1} - \frac{[A]_{i+1,1}}{[A]_{1,1}} [A]_{1,j+1};
        \\
        \det {(A)} & = \det {(B_n)} = [B_n]_{1,1} \det {(B_n (1|1))}
            = [A]_{1,1} \det {(C)}.
    \end{align*}

    注意到, 若 \(j \neq k\), 则
    \begin{align*}
        [A]_{k,k} >
        \sum_{\substack{1 \leq u \leq n \\
            u \neq k}} {|[A]_{k,u}|}
        \geq |[A]_{k,j}|,
    \end{align*}
    故
    \begin{align*}
        [C]_{i,i}
        = {}    &
        [A]_{i+1,i+1} - \frac{[A]_{i+1,1}}{[A]_{1,1}} [A]_{1,i+1}
        \\
        = {}    &
        \frac{[A]_{i+1,i+1} [A]_{1,1} - [A]_{1,i+1} [A]_{i+1,1}}
        {[A]_{1,1}}
        \\
        > {}    &
        \frac{|[A]_{1,i+1}|\, |[A]_{i+1,1}| - [A]_{1,i+1} [A]_{i+1,1}}
        {[A]_{1,1}}
        \\
        = {}    &
        \frac{|{[A]_{1,i+1} [A]_{i+1,1}}| - [A]_{1,i+1} [A]_{i+1,1}}
        {[A]_{1,1}}
        \\
        \geq {} &
        0.
    \end{align*}
    故 \(0 < [C]_{i,i} = |[C]_{i,i}|\).

    注意到
    \begin{align*}
        |[C]_{i,i}|
        = {}    &
        \Bigg|
        [A]_{i+1,i+1} - \frac{[A]_{i+1,1}}{[A]_{1,1}} [A]_{1,i+1}
        \Bigg|
        \\
        \geq {} &
        |[A]_{i+1,i+1}| -
        \Bigg|
        \frac{[A]_{i+1,1}}{[A]_{1,1}} [A]_{1,i+1}
        \Bigg|
        \\
        = {}    &
        |[A]_{i+1,i+1}| -
        \frac{|[A]_{i+1,1}|}{|[A]_{1,1}|} |[A]_{1,i+1}|,
    \end{align*}
    且
    \begin{align*}
                &
        \sum_{\substack{1 \leq v \leq n-1  \\ v \neq i}}
        {|[C]_{i,v}|}
        \\
        = {}    &
        \sum_{\substack{2 \leq \ell \leq n \\ \ell \neq i+1}}
        {|[C]_{i,\ell-1}|}
        \\
        = {}    &
        \sum_{\substack{2 \leq \ell \leq n \\ \ell \neq i+1}}
        {\Bigg|
        [A]_{i+1,\ell} - \frac{[A]_{i+1,1}}{[A]_{1,1}} [A]_{1,\ell}
        \Bigg|}
        \\
        \leq {} &
        \sum_{\substack{2 \leq \ell \leq n \\ \ell \neq i+1}}
        {\Bigg(
            |[A]_{i+1,\ell}| +
            \Bigg|
            \frac{[A]_{i+1,1}}{[A]_{1,1}} [A]_{1,\ell}
            \Bigg|
            \Bigg)}
        \\
        = {}    &
        \sum_{\substack{2 \leq \ell \leq n \\ \ell \neq i+1}}
        {\Bigg(
            |[A]_{i+1,\ell}| +
            \frac{|[A]_{i+1,1}|}{|[A]_{1,1}|} |[A]_{1,\ell}|
            \Bigg)}
        \\
        = {}    &
        \sum_{\substack{2 \leq \ell \leq n \\ \ell \neq i+1}}
        {
        |[A]_{i+1,\ell}|
        }
        +
        \frac{|[A]_{i+1,1}|}{|[A]_{1,1}|}
        \sum_{\substack{2 \leq \ell \leq n \\ \ell \neq i+1}}
        {
        |[A]_{1,\ell}|
        }
        \\
        = {}    &
        \sum_{\substack{1 \leq \ell \leq n \\ \ell \neq i+1}}
        {
        |[A]_{i+1,\ell}|
        }
        - |[A]_{i+1,1}|
        +
        \frac{|[A]_{i+1,1}|}{|[A]_{1,1}|}
        \sum_{2 \leq \ell \leq n}
        {
        |[A]_{1,\ell}|
        }
        \\
        {}      &
        \hphantom{=}
        -
        \frac{|[A]_{i+1,1}|}{|[A]_{1,1}|}
        |[A]_{1,i+1}|
        \\
        = {}    &
        \sum_{\substack{1 \leq \ell \leq n \\ \ell \neq i+1}}
        {
        |[A]_{i+1,\ell}|
        }
        - \frac{|[A]_{i+1,1}|}{|[A]_{1,1}|} |[A]_{1,1}|
        +
        \frac{|[A]_{i+1,1}|}{|[A]_{1,1}|}
        \sum_{2 \leq \ell \leq n}
        {
        |[A]_{1,\ell}|
        }
        \\
        {}      &
        \hphantom{=}
        -
        \frac{|[A]_{i+1,1}|}{|[A]_{1,1}|}
        |[A]_{1,i+1}|
        \\
        = {}    &
        \sum_{\substack{1 \leq \ell \leq n \\ \ell \neq i+1}}
        {
        |[A]_{i+1,\ell}|
        }
        - \frac{|[A]_{i+1,1}|}{|[A]_{1,1}|}
        \Bigg(
        |[A]_{1,1}|
        -
        \sum_{2 \leq \ell \leq n}
        {
            |[A]_{1,\ell}|
        }
        \Bigg)
        \\
        {}      &
        \hphantom{=}
        -
        \frac{|[A]_{i+1,1}|}{|[A]_{1,1}|}
        |[A]_{1,i+1}|
        \\
        \leq {} &
        \sum_{\substack{1 \leq \ell \leq n \\ \ell \neq i+1}}
        {
        |[A]_{i+1,\ell}|
        }
        -
        \frac{|[A]_{i+1,1}|}{|[A]_{1,1}|}
        |[A]_{1,i+1}|.
    \end{align*}
    故
    \begin{align*}
        |[C]_{i,i}| -
        \sum_{\substack{1 \leq v \leq n-1  \\ v \neq i}}
        {|[C]_{i,v}|}
        \geq
        |[A]_{i+1,i+1}| -
        \sum_{\substack{1 \leq \ell \leq n \\ \ell \neq i+1}}
        {
        |[A]_{i+1,\ell}|
        }
        > 0.
    \end{align*}

    综上, 由假定, \(\det {(C)} > 0\).

    既然 \([A]_{1,1} > 0\),
    且 \(\det {(C)} > 0\),
    故 \(\det {(A)} = [A]_{1,1} \det {(C)} > 0\).

    所以, \(P(n)\) 是正确的.
    由数学归纳法原理, 待证命题成立.
\end{proof}

第~2~个证明的计算较少.
不过, 我们要注意一件小事.

\begin{theorem}
    设 \(m\), \(b\) 是实数.
    设 \(b > 0\), 且对任何不超过 \(1\) 的正数 \(t\),
    有 \(mt + b \neq 0\).
    则 \(m + b > 0\).
\end{theorem}

\begin{proof}
    用反证法.
    反设 \(m + b \leq 0\).
    则 \(0 < b \leq -m\).
    设 \(s = b/(-m)\).
    则 \(0 < s \leq 1\).
    可是, \(ms + b = 0\).
    这是矛盾.
\end{proof}

\begin{proof}[法~2]
    作命题 \(P(n)\):
    \begin{quotation}
        对任何适合如下条件的 \(n\)~级实阵 \(A\),
        必 \(\det {(A)} > 0\):
        \begin{quotation}
            对任何不超过 \(n\)~的正整数 \(i\),
            必
            \begin{align*}
                [A]_{i,i} >
                \sum_{\substack{1 \leq u \leq n \\
                    u \neq i}} {|[A]_{i,u}|}.
            \end{align*}
        \end{quotation}
    \end{quotation}
    我们用数学归纳法证明,
    对任何正整数 \(n\),
    \(P(n)\) 是对的.

    \(P(1)\) 是对的; 这是显然的.

    设 \(P(n-1)\) 是对的.
    我们要由此证 \(P(n)\) 是对的.

    设 \(A\) 是一个 \(n\)~级实阵,
    且对任何不超过 \(n\) 的正整数 \(i\), 必
    \begin{align*}
        [A]_{i,i} >
        \sum_{\substack{1 \leq u \leq n \\
            u \neq i}} {|[A]_{i,u}|}.
    \end{align*}
    因为 \([A]_{i,i}\) 大于一个非负数,
    故 \(0 < [A]_{i,i} = |[A]_{i,i}|\).

    作 \(n \times n\)~实阵 \(A_t\) 如下:
    \begin{align*}
        [A_t]_{i,j} =
        \begin{cases}
            [A]_{i,j},     & i \neq 1;  \\
            [A]_{1,1},     & i = j = 1; \\
            t\, [A]_{1,j}, & \text{其他}.
        \end{cases}
    \end{align*}
    我们考虑 \(A_t\)~的行列式.
    按行~\(1\) 展开, 有
    \begin{align*}
        \det {(A_t)}
        = {} &
        \sum_{1 \leq j \leq n}
        {(-1)^{1 + j} [A_t]_{1,j} \det {(A_t (1 | j))}}
        \\
        = {} &
        \sum_{1 \leq j \leq n}
        {(-1)^{1 + j} [A_t]_{1,j} \det {(A (1 | j))}}
        \\
        = {} &
        \sum_{\substack{1 \leq j \leq n \\j \neq 1}}
        {(-1)^{1 + j} [A_t]_{1,j} \det {(A (1 | j))}}
        +
        (-1)^{1 + 1} [A_t]_{1,1} \det {(A (1 | 1))}
        \\
        = {} &
        \sum_{\substack{1 \leq j \leq n \\j \neq 1}}
        {(-1)^{1 + j} t\, [A]_{1,j} \det {(A (1 | j))}}
        +
        [A]_{1,1} \det {(A (1 | 1))}
        \\
        = {} &
        t
        \sum_{\substack{1 \leq j \leq n \\j \neq 1}}
        {(-1)^{1 + j} [A]_{1,j} \det {(A (1 | j))}}
        +
        [A]_{1,1} \det {(A (1 | 1))}.
    \end{align*}
    记
    \begin{align*}
        m & =
        \sum_{\substack{1 \leq j \leq n \\j \neq 1}}
        {(-1)^{1 + j} [A]_{1,j} \det {(A (1 | j))}},
        \\
        b & =
        [A]_{1,1} \det {(A (1 | 1))}.
    \end{align*}
    则 \(m\), \(b\) 是实数, 且
    \(\det {(A_t)} = mt + b\).

    作 \(n-1\)~级实阵 \(C = A(1 | 1)\),
    则 \([C]_{w,y} = [A]_{w+1,y+1}\).
    且对 \(w < n\),
    \begin{align*}
        [C]_{w,w}
        = {}    &
        [A]_{w+1,w+1}
        \\
        > {}    &
        \sum_{\substack{1 \leq u \leq n   \\u \neq w+1}}
        {|[A]_{w+1,u}|}
        \\
        \geq {} &
        \sum_{\substack{1 \leq u \leq n   \\u \neq 1, w+1}}
        {|[A]_{w+1,u}|}
        \\
        = {}    &
        \sum_{\substack{1 \leq y \leq n-1 \\y \neq w}}
        {|[A]_{w+1,y+1}|}
        \\
        = {}    &
        \sum_{\substack{1 \leq y \leq n-1 \\y \neq w}}
        {|[C]_{w,y}|}.
    \end{align*}
    所以, 由假定, \(\det {(A(1|1))} = \det {(C)} > 0\).
    因为 \([A]_{1,1} > 0\), 我们有 \(b > 0\).

    设 \(t\) 是不超过 \(1\)~的正数.
    则 \(0 \leq |t| = t \leq 1\).
    任取不超过 \(n\) 的正整数 \(i\).
    若 \(i \neq 1\), 则
    \begin{align*}
        |[A_t]_{i,i}|
        = {} &
        |[A]_{i,i}|
        \\
        = {} &
        [A]_{i,i}
        \\
        > {} &
        \sum_{\substack{1 \leq u \leq n \\u \neq i}}
        {|[A]_{i,u}|}
        \\
        = {} &
        \sum_{\substack{1 \leq u \leq n \\u \neq i}}
        {|[A_t]_{i,u}|}.
    \end{align*}
    若 \(i = 1\), 则
    \begin{align*}
        |[A_t]_{i,i}|
        = {}    &
        |[A]_{i,i}|
        \\
        = {}    &
        [A]_{i,i}
        \\
        > {}    &
        \sum_{\substack{1 \leq u \leq n     \\u \neq i}}
        {|[A]_{i,u}|}
        \\
        \geq {} &
        |t| \sum_{\substack{1 \leq u \leq n \\u \neq i}}
        {|[A]_{i,u}|}
        \\
        = {}    &
        \sum_{\substack{1 \leq u \leq n     \\u \neq i}}
        {|t|\, |[A]_{i,u}|}
        \\
        = {}    &
        \sum_{\substack{1 \leq u \leq n     \\u \neq i}}
        {|t\, [A]_{i,u}|}
        \\
        = {}    &
        \sum_{\substack{1 \leq u \leq n     \\u \neq i}}
        {|[A_t]_{i,u}|}.
    \end{align*}
    则 \(\det {(A_t)} \neq 0\).
    从而,
    对任何不超过 \(1\) 的正数 \(t\),
    有 \(mt + b \neq 0\).

    既然 \(b > 0\),
    且对任何不超过 \(1\) 的正数 \(t\),
    有 \(mt + b \neq 0\),
    我们能推出 \(m + b > 0\).
    故 \(\det {(A)} = \det {(A_1)} = m + b > 0\).

    所以, \(P(n)\) 是正确的.
    由数学归纳法原理, 待证命题成立.
\end{proof}

当然, 如下命题也是正确的 (利用转置):

\begin{theorem}
    设 \(A\) 是一个 \(n\)~级\emph{实阵}.
    设对任何不超过 \(n\) 的正整数 \(i\), 必
    \begin{align*}
        [A]_{i,i} >
        \sum_{\substack{1 \leq u \leq n \\
            u \neq i}} {|[A]_{u,i}|}.
    \end{align*}
    则 \(\det {(A)} > 0\).
\end{theorem}

\begingroup

我们回看上节的例.

\renewcommand\thmcontinues[1]{续}

\begin{example}[continues=emp:NonzeroDet1]
    设 \(A =
    \begin{bmatrix}
        9 & 6 & 2 \\
        4 & 8 & 3 \\
        5 & 1 & 7 \\
    \end{bmatrix}\)
    是一个 \(3\)~级实阵.
    不难算出, \(9 > |6| + |2|\),
    \(8 > |4| + |3|\),
    且 \(7 > |5| + |1|\).
    于是, \(\det {(A)} > 0\).
    (其实, 不难算出, \(\det {(A)} = 327.\))
\end{example}

有时, 一个实方阵 \(A\) 可能不适合定理的条件, 故我们无法直接地用定理.
不过, 既然我们研究是否 \(\det {(A)} > 0\),
我们可找二个跟 \(A\) 同尺寸的实阵 \(B\), \(C\),
使 \(\det {(B)} \det {(C)} > 0\),
且 \(BAC\) 适合定理的条件.
则 \(\det {(BAC)} > 0\).
因为 \(0 < \det {(BAC)} = \det {(B)} \det {(A)} \det {(C)}\),
且 \(\det {(B)} \det {(C)} > 0\),
故 \(\det {(A)} > 0\).

\begin{example}[continues=emp:NonzeroDet3]
    设 \(n \geq 2\).
    设 \(n\)~级实阵 \(A\) 适合
    \begin{align*}
        [A]_{i,j} =
        \begin{cases}
            2,
             & 1 \leq i = j \leq n;         \\
            \text{\(1\) 或 \(-1\)},
             & 1 \leq i = j - 1 \leq n - 1; \\
            \text{\(1\) 或 \(-1\)},
             & 2 \leq i = j + 1 \leq n;     \\
            0,
             & \text{其他}.
        \end{cases}
    \end{align*}
    我们无法直接地用定理说明 \(\det {(A)} > 0\).

    记 \(c_k = 1 - 1/(k+1) = k/(k+1) > 0\),
    \(k = 1\), \(2\), \(\dots\), \(n\).
    作 \(n\)~级实阵
    \begin{align*}
        C =
        \begin{bmatrix}
            c_1    & 0      & \cdots & 0      \\
            0      & c_2    & \cdots & 0      \\
            \vdots & \vdots & \ddots & \vdots \\
            0      & 0      & \cdots & c_n    \\
        \end{bmatrix};
    \end{align*}
    具体地,
    \begin{align*}
        [C]_{i,j}
        = \begin{cases}
              c_i, & 1 \leq i = j \leq n; \\
              0,   & \text{其他}.
          \end{cases}
    \end{align*}
    不难算出, \(\det {(C)} > 0\)
    (其实, \(\det {(C)} = 1/(n+1)\)).

    现在, 我们证明, \(\det {(AC)} > 0\).

    首先, \(AC\) 当然是一个实阵.
    不难算出, \([AC]_{i,j} = c_j\, [A]_{i,j}\);
    特别地, \([AC]_{i,i} = 2c_i\).
    记
    \begin{align*}
        R_i = \sum_{\substack{1 \leq u \leq n \\ u \neq i}}
        {|[AC]_{i,u}|}.
    \end{align*}
    则
    \begin{align*}
        R_i =
        \begin{dcases}
            c_2,               & i = 1;     \\
            c_{i-1} + c_{i+1}, & 1 < i < n; \\
            c_{n-1},           & i = n.
        \end{dcases}
    \end{align*}
    不难验证,
    \begin{align*}
        [AC]_{1,1} - R_1
         & = 2c_1 - c_2
        = 1 - \frac{2}{3} > 0,                 \\
        [AC]_{n,n} - R_n
         & = 2c_n - c_{n-1}
        = \frac{n-1}{n+1} + \frac{1}{n} > 0,   \\
        [AC]_{i,i} - R_i
         & = (c_i - c_{i-1}) - (c_{i+1} - c_i)
        = \frac{1}{i(i+1)} - \frac{1}{(i+1)(i+2)} > 0.
    \end{align*}
    于是, \(\det {(AC)} > 0\).

    既然 \(0 < \det {(AC)} = \det {(A)} \det {(C)}\),
    且 \(\det {(C)} > 0\),
    故 \(\det {(A)} > 0\).
\end{example}

我们还有如下定理.

\begin{theorem}
    设 \(A\) 是一个 \(n\)~级\emph{实阵} (\(n \geq 2\)).
    设对任何不超过 \(n\) 的正整数 \(i\), 必
    \([A]_{i,i} > 0\).
    设对任何不超过 \(n\), 且不等的正整数 \(j\), \(k\), 必
    \begin{align*}
        [A]_{j,j}\,[A]_{k,k} >
        \Bigg(
        \sum_{\substack{1 \leq p \leq n \\
                p \neq j}} {|[A]_{j,p}|}
        \Bigg)
        \Bigg(
        \sum_{\substack{1 \leq q \leq n \\
                q \neq k}} {|[A]_{k,q}|}
        \Bigg).
    \end{align*}
    则 \(\det {(A)} > 0\).
\end{theorem}

\begin{proof}
    作命题 \(P(n)\):
    \begin{quotation}
        对任何适合如下条件的 \(n\)~级实阵 \(A\),
        必 \(\det {(A)} > 0\):
        \begin{quotation}
            (1)
            对任何不超过 \(n\)~的正整数 \(i\),
            必 \([A]_{i,i} > 0\);

            (2)
            对任何不超过 \(n\), 且不等的正整数 \(j\), \(k\), 必
            \begin{align*}
                [A]_{j,j}\,[A]_{k,k} >
                \Bigg(
                \sum_{\substack{1 \leq p \leq n \\
                        p \neq j}} {|[A]_{j,p}|}
                \Bigg)
                \Bigg(
                \sum_{\substack{1 \leq q \leq n \\
                        q \neq k}} {|[A]_{k,q}|}
                \Bigg).
            \end{align*}
        \end{quotation}
    \end{quotation}
    我们用数学归纳法证明,
    对任何高于 \(1\)~的正整数 \(n\),
    \(P(n)\) 是对的.

    \(P(2)\) 是对的.
    任取 \(2\)~级实阵
    \begin{align*}
        A = \begin{bmatrix}
                a & b \\
                c & d \\
            \end{bmatrix},
    \end{align*}
    其中 \(a d > |b|\,|c|\).
    则
    \begin{align*}
        \det {(A)} = ad - bc > |b|\,|c| - bc
        = |bc| - bc \geq 0.
    \end{align*}

    设 \(P(n-1)\) 是对的.
    我们要由此证 \(P(n)\) 是对的.

    设 \(A\) 是一个 \(n\)~级实阵.
    设对任何不超过 \(n\) 的正整数 \(i\), 必
    \([A]_{i,i} > 0\).
    设对任何不超过 \(n\), 且不等的正整数 \(j\), \(k\), 必
    \begin{align*}
        [A]_{j,j}\,[A]_{k,k} >
        \Bigg(
        \sum_{\substack{1 \leq p \leq n \\
                p \neq j}} {|[A]_{j,p}|}
        \Bigg)
        \Bigg(
        \sum_{\substack{1 \leq q \leq n \\
                q \neq k}} {|[A]_{k,q}|}
        \Bigg).
    \end{align*}

    作 \(n \times n\)~实阵 \(A_t\) 如下:
    \begin{align*}
        [A_t]_{i,j} =
        \begin{cases}
            [A]_{i,j},     & i \neq 1;  \\
            [A]_{1,1},     & i = j = 1; \\
            t\, [A]_{1,j}, & \text{其他}.
        \end{cases}
    \end{align*}
    我们考虑 \(A_t\)~的行列式.
    按行~\(1\) 展开, 有
    \begin{align*}
        \det {(A_t)}
        = {} &
        \sum_{1 \leq j \leq n}
        {(-1)^{1 + j} [A_t]_{1,j} \det {(A_t (1 | j))}}
        \\
        = {} &
        \sum_{1 \leq j \leq n}
        {(-1)^{1 + j} [A_t]_{1,j} \det {(A (1 | j))}}
        \\
        = {} &
        \sum_{\substack{1 \leq j \leq n \\j \neq 1}}
        {(-1)^{1 + j} [A_t]_{1,j} \det {(A (1 | j))}}
        +
        (-1)^{1 + 1} [A_t]_{1,1} \det {(A (1 | 1))}
        \\
        = {} &
        \sum_{\substack{1 \leq j \leq n \\j \neq 1}}
        {(-1)^{1 + j} t\, [A]_{1,j} \det {(A (1 | j))}}
        +
        [A]_{1,1} \det {(A (1 | 1))}
        \\
        = {} &
        t
        \sum_{\substack{1 \leq j \leq n \\j \neq 1}}
        {(-1)^{1 + j} [A]_{1,j} \det {(A (1 | j))}}
        +
        [A]_{1,1} \det {(A (1 | 1))}.
    \end{align*}
    记
    \begin{align*}
        m & =
        \sum_{\substack{1 \leq j \leq n \\j \neq 1}}
        {(-1)^{1 + j} [A]_{1,j} \det {(A (1 | j))}},
        \\
        b & =
        [A]_{1,1} \det {(A (1 | 1))}.
    \end{align*}
    则 \(m\), \(b\) 是实数, 且
    \(\det {(A_t)} = mt + b\).

    作 \(n-1\)~级实阵 \(C = A(1 | 1)\).
    则 \([C]_{w,y} = [A]_{w+1,y+1}\),
    且对 \(v\), \(w < n\),
    且 \(v \neq w\),
    \begin{align*}
        [C]_{v,v}\, [C]_{w,w}
        = {}    &
        [A]_{v+1,v+1}\, [A]_{w+1,w+1}
        \\
        > {}    &
        \Bigg(
        \sum_{\substack{1 \leq u \leq n   \\u \neq v+1}}
        {|[A]_{v+1,u}|}
        \Bigg)
        \Bigg(
        \sum_{\substack{1 \leq u \leq n   \\u \neq w+1}}
        {|[A]_{w+1,u}|}
        \Bigg)
        \\
        \geq {} &
        \Bigg(
        \sum_{\substack{1 \leq u \leq n   \\u \neq 1, v+1}}
        {|[A]_{v+1,u}|}
        \Bigg)
        \Bigg(
        \sum_{\substack{1 \leq u \leq n   \\u \neq 1, w+1}}
        {|[A]_{w+1,u}|}
        \Bigg)
        \\
        = {}    &
        \Bigg(
        \sum_{\substack{1 \leq y \leq n-1 \\y \neq v}}
        {|[A]_{v+1,y+1}|}
        \Bigg)
        \Bigg(
        \sum_{\substack{1 \leq y \leq n-1 \\y \neq w}}
        {|[A]_{w+1,y+1}|}
        \Bigg)
        \\
        = {}    &
        \Bigg(
        \sum_{\substack{1 \leq y \leq n-1 \\y \neq v}}
        {|[C]_{v,y}|}
        \Bigg)
        \Bigg(
        \sum_{\substack{1 \leq y \leq n-1 \\y \neq w}}
        {|[C]_{w,y}|}
        \Bigg).
    \end{align*}
    所以, 由假定, \(\det {(A(1|1))} = \det {(C)} > 0\).
    因为 \([A]_{1,1} > 0\), 我们有 \(b > 0\).

    设 \(t\) 是不超过 \(1\)~的正数.
    则 \(0 \leq |t| = t \leq 1\).
    任取不超过 \(n\) 的正整数 \(j\), \(k\),
    且 \(j \neq k\).
    若 \(j \neq 1\) 且 \(k \neq 1\), 则
    \begin{align*}
        |[A_t]_{j,j}|\,|[A_t]_{k,k}|
        = {} &
        |[A]_{j,j}|\,|[A]_{k,k}|
        \\
        = {} &
        [A]_{j,j}\,[A]_{k,k}
        \\
        > {} &
        \Bigg(
        \sum_{\substack{1 \leq p \leq n \\
                p \neq j}} {|[A]_{j,p}|}
        \Bigg)
        \Bigg(
        \sum_{\substack{1 \leq q \leq n \\
                q \neq k}} {|[A]_{k,q}|}
        \Bigg)
        \\
        = {} &
        \Bigg(
        \sum_{\substack{1 \leq p \leq n \\
                p \neq j}} {|[A_t]_{j,p}|}
        \Bigg)
        \Bigg(
        \sum_{\substack{1 \leq q \leq n \\
                q \neq k}} {|[A_t]_{k,q}|}
        \Bigg).
    \end{align*}
    若 \(j = 1\) 且 \(k \neq j\), 则
    \begin{align*}
        |[A_t]_{j,j}|\,|[A_t]_{k,k}|
        = {}    &
        |[A]_{j,j}|\,|[A]_{k,k}|
        \\
        = {}    &
        [A]_{j,j}\,[A]_{k,k}
        \\
        > {}    &
        \Bigg(
        \sum_{\substack{1 \leq p \leq n \\
                p \neq j}} {|[A]_{j,p}|}
        \Bigg)
        \Bigg(
        \sum_{\substack{1 \leq q \leq n \\
                q \neq k}} {|[A]_{k,q}|}
        \Bigg)
        \\
        \geq {} &
        |t|\,\Bigg(
        \sum_{\substack{1 \leq p \leq n \\
                p \neq j}} {|[A]_{j,p}|}
        \Bigg)
        \Bigg(
        \sum_{\substack{1 \leq q \leq n \\
                q \neq k}} {|[A]_{k,q}|}
        \Bigg)
        \\
        = {}    &
        \Bigg(
        \sum_{\substack{1 \leq p \leq n \\
                p \neq j}} {|t|\,|[A]_{j,p}|}
        \Bigg)
        \Bigg(
        \sum_{\substack{1 \leq q \leq n \\
                q \neq k}} {|[A]_{k,q}|}
        \Bigg)
        \\
        = {}    &
        \Bigg(
        \sum_{\substack{1 \leq p \leq n \\
                p \neq j}} {|t\, [A]_{j,p}|}
        \Bigg)
        \Bigg(
        \sum_{\substack{1 \leq q \leq n \\
                q \neq k}} {|[A]_{k,q}|}
        \Bigg)
        \\
        = {}    &
        \Bigg(
        \sum_{\substack{1 \leq p \leq n \\
                p \neq j}} {|[A_t]_{j,p}|}
        \Bigg)
        \Bigg(
        \sum_{\substack{1 \leq q \leq n \\
                q \neq k}} {|[A_t]_{k,q}|}
        \Bigg).
    \end{align*}
    若 \(k = 1\) 且 \(j \neq k\),
    类似地, 我们也有
    \begin{align*}
        |[A_t]_{j,j}|\,|[A_t]_{k,k}|
        > {} &
        \Bigg(
        \sum_{\substack{1 \leq p \leq n \\
                p \neq j}} {|[A_t]_{j,p}|}
        \Bigg)
        \Bigg(
        \sum_{\substack{1 \leq q \leq n \\
                q \neq k}} {|[A_t]_{k,q}|}
        \Bigg).
    \end{align*}
    则 \(\det {(A_t)} \neq 0\).
    从而,
    对任何不超过 \(1\) 的正数 \(t\),
    有 \(mt + b \neq 0\).

    既然 \(b > 0\),
    且对任何不超过 \(1\) 的正数 \(t\),
    有 \(mt + b \neq 0\),
    我们能推出 \(m + b > 0\).
    故 \(\det {(A)} = \det {(A_1)} = m + b > 0\).

    所以, \(P(n)\) 是正确的.
    由数学归纳法原理, 待证命题成立.
\end{proof}

% the old proof, which works but is not so good
% \begin{proof}
%     设 \(A\) 是一个 \(n\)~级\emph{实阵} (\(n \geq 2\)).
%     设对任何不超过 \(n\) 的正整数 \(i\), 必
%     \([A]_{i,i} > 0\).
%     设对任何不超过 \(n\), 且不等的正整数 \(j\), \(k\), 必
%     \begin{align*}
%         [A]_{j,j}\,[A]_{k,k} >
%         \Bigg(
%         \sum_{\substack{1 \leq p \leq n \\
%                 p \neq j}} {|[A]_{j,p}|}
%         \Bigg)
%         \Bigg(
%         \sum_{\substack{1 \leq q \leq n \\
%                 q \neq k}} {|[A]_{k,q}|}
%         \Bigg).
%     \end{align*}
%     我们知道, 一定有
%     \(n-1\)~个互不相同的, 且不超过 \(n\) 的正整数
%     \(i_1\), \(i_2\), \(\dots\), \(i_{n-1}\),
%     使对任何不超过 \(n-1\) 的正整数 \(w\),
%     有
%     \begin{align*}
%         [A]_{i_w,i_w} >
%         \sum_{\substack{1 \leq u \leq n \\
%             u \neq i_w}} {|[A]_{i_w,u}|}.
%     \end{align*}
%     (无妨设 \(i_1\), \(i_2\), \(\dots\), \(i_{n-1}\)
%     是从小到大的.)
%     (可用反证法证明此事:
%     反设有不等的 j, k 不适合此不等式,
%     则可推出矛盾.)
%     从 \(1\), \(2\), \(\dots\), \(n\)
%     去除 \(i_1\), \(\dots\), \(i_{n-1}\) 后,
%     还剩一个数.
%     我们叫它 \(i_n\).
%     作 \(n \times n\)~实阵 \(A_t\) 如下:
%     \begin{align*}
%         [A_t]_{i,j} =
%         \begin{cases}
%             [A]_{i,j},       & i \neq i_n;  \\
%             [A]_{i_n,i_n},   & i = j = i_n; \\
%             t\, [A]_{i_n,j}, & \text{其他}.
%         \end{cases}
%     \end{align*}
%     我们考虑 \(A_t\)~的行列式.
%     按行~\(i_n\) 展开, 有
%     \begin{align*}
%         \det {(A_t)}
%         = {} &
%         \sum_{1 \leq j \leq n}
%         {(-1)^{i_n + j} [A_t]_{i_n,j} \det {(A_t (i_n | j))}}
%         \\
%         = {} &
%         \sum_{1 \leq j \leq n}
%         {(-1)^{i_n + j} [A_t]_{i_n,j} \det {(A (i_n | j))}}
%         \\
%         = {} &
%         \sum_{\substack{1 \leq j \leq n \\j \neq i_n}}
%         {(-1)^{i_n + j} [A_t]_{i_n,j} \det {(A (i_n | j))}}
%         +
%         (-1)^{i_n + i_n} [A_t]_{i_n,i_n} \det {(A (i_n | i_n))}
%         \\
%         = {} &
%         \sum_{\substack{1 \leq j \leq n \\j \neq i_n}}
%         {(-1)^{i_n + j} t\, [A]_{i_n,j} \det {(A (i_n | j))}}
%         +
%         [A]_{i_n,i_n} \det {(A (i_n | i_n))}
%         \\
%         = {} &
%         t
%         \sum_{\substack{1 \leq j \leq n \\j \neq i_n}}
%         {(-1)^{i_n + j} [A]_{i_n,j} \det {(A (i_n | j))}}
%         +
%         [A]_{i_n,i_n} \det {(A (i_n | i_n))}.
%     \end{align*}
%     记
%     \begin{align*}
%         m & =
%         \sum_{\substack{1 \leq j \leq n \\j \neq i_n}}
%         {(-1)^{i_n + j} [A]_{i_n,j} \det {(A (i_n | j))}},
%         \\
%         b & =
%         [A]_{i_n,i_n} \det {(A (i_n | i_n))}.
%     \end{align*}
%     则 \(m\), \(b\) 是实数, 且
%     \(\det {(A_t)} = mt + b\).

%     作 \(n-1\)~级实阵 \(C = A(i_n | i_n)\).
%     则 \([C]_{w,y} = [A]_{i_w,i_y}\)
%     (因为已设 \(i_1 < i_2 < \dots < i_{n-1}\)),
%     且对 \(w < n\),
%     \begin{align*}
%         [C]_{w,w}
%         = {}    &
%         [A]_{i_w,i_w}
%         \\
%         > {}    &
%         \sum_{\substack{1 \leq u \leq n   \\u \neq i_w}}
%         {|[A]_{i_w,u}|}
%         \\
%         \geq {} &
%         \sum_{\substack{1 \leq u \leq n   \\u \neq i_w, i_n}}
%         {|[A]_{i_w,u}|}
%         \\
%         = {}    &
%         \sum_{\substack{1 \leq y \leq n-1 \\y \neq w}}
%         {|[A]_{i_w,i_y}|}
%         \\
%         = {}    &
%         \sum_{\substack{1 \leq y \leq n-1 \\y \neq w}}
%         {|[C]_{w,y}|}.
%     \end{align*}
%     故 \(\det {(A(i_n | i_n))} = \det {(C)} > 0\),
%     由本节的第~1~个定理.
%     则 \(b > 0\).

%     设 \(t\) 是不超过 \(1\)~的正数.
%     则 \(0 \leq |t| = t \leq 1\).
%     任取不超过 \(n\), 且不等的正整数 \(j\), \(k\).
%     若 \(j\), \(k\) 的任何一个都不是 \(i_n\), 则
%     \begin{align*}
%         [A_t]_{j,j}\,[A_t]_{k,k}
%         = {} &
%         [A]_{j,j}\,[A]_{k,k}
%         \\
%         > {} &
%         \Bigg(
%         \sum_{\substack{1 \leq p \leq n \\
%                 p \neq j}} {|[A]_{j,p}|}
%         \Bigg)
%         \Bigg(
%         \sum_{\substack{1 \leq q \leq n \\
%                 q \neq k}} {|[A]_{k,q}|}
%         \Bigg)
%         \\
%         = {} &
%         \Bigg(
%         \sum_{\substack{1 \leq p \leq n \\
%                 p \neq j}} {|[A_t]_{j,p}|}
%         \Bigg)
%         \Bigg(
%         \sum_{\substack{1 \leq q \leq n \\
%                 q \neq k}} {|[A_t]_{k,q}|}
%         \Bigg).
%     \end{align*}
%     若 \(j = i_n\), 且 \(k \neq j\), 则
%     \begin{align*}
%         [A_t]_{j,j}\,[A_t]_{k,k}
%         = {}    &
%         [A]_{j,j}\,[A]_{k,k}
%         \\
%         > {}    &
%         \Bigg(
%         \sum_{\substack{1 \leq p \leq n \\
%                 p \neq j}} {|[A]_{j,p}|}
%         \Bigg)
%         \Bigg(
%         \sum_{\substack{1 \leq q \leq n \\
%                 q \neq k}} {|[A]_{k,q}|}
%         \Bigg)
%         \\
%         \geq {} &
%         |t|\,\Bigg(
%         \sum_{\substack{1 \leq p \leq n \\
%                 p \neq j}} {|[A]_{j,p}|}
%         \Bigg)
%         \Bigg(
%         \sum_{\substack{1 \leq q \leq n \\
%                 q \neq k}} {|[A]_{k,q}|}
%         \Bigg)
%         \\
%         = {}    &
%         \Bigg(
%         \sum_{\substack{1 \leq p \leq n \\
%                 p \neq j}} {|t|\,|[A]_{j,p}|}
%         \Bigg)
%         \Bigg(
%         \sum_{\substack{1 \leq q \leq n \\
%                 q \neq k}} {|[A]_{k,q}|}
%         \Bigg)
%         \\
%         = {}    &
%         \Bigg(
%         \sum_{\substack{1 \leq p \leq n \\
%                 p \neq j}} {|[A_t]_{j,p}|}
%         \Bigg)
%         \Bigg(
%         \sum_{\substack{1 \leq q \leq n \\
%                 q \neq k}} {|[A_t]_{k,q}|}
%         \Bigg).
%     \end{align*}
%     若 \(k = i_n\), 且 \(j \neq k\),
%     类似地, 我们也有
%     \begin{align*}
%         [A_t]_{j,j}\,[A_t]_{k,k}
%         > {} &
%         \Bigg(
%         \sum_{\substack{1 \leq p \leq n \\
%                 p \neq j}} {|[A_t]_{j,p}|}
%         \Bigg)
%         \Bigg(
%         \sum_{\substack{1 \leq q \leq n \\
%                 q \neq k}} {|[A_t]_{k,q}|}
%         \Bigg).
%     \end{align*}
%     注意到 \(0 < [A]_{i,i} = [A_t]_{i,i}\),
%     故 \(|[A_t]_{i,i}| = [A_t]_{i,i}\).
%     则 \(\det {(A_t)} \neq 0\);
%     也就是说,
%     对任何不超过 \(1\) 的正数 \(t\),
%     有 \(mt + b \neq 0\).

%     既然 \(b > 0\),
%     且对任何不超过 \(1\) 的正数 \(t\),
%     有 \(mt + b \neq 0\),
%     我们能推出 \(m + b > 0\).
%     故 \(\det {(A)} = \det {(A_1)} = m + b > 0\).
% \end{proof}

当然, 如下命题也是正确的 (利用转置):

\begin{theorem}
    设 \(A\) 是一个 \(n\)~级\emph{实阵} (\(n \geq 2\)).
    设对任何不超过 \(n\) 的正整数 \(i\), 必
    \([A]_{i,i} > 0\).
    设对任何不超过 \(n\), 且不等的正整数 \(j\), \(k\), 必
    \begin{align*}
        [A]_{j,j}\,[A]_{k,k} >
        \Bigg(
        \sum_{\substack{1 \leq p \leq n \\
                p \neq j}} {|[A]_{p,j}|}
        \Bigg)
        \Bigg(
        \sum_{\substack{1 \leq q \leq n \\
                q \neq k}} {|[A]_{q,k}|}
        \Bigg).
    \end{align*}
    则 \(\det {(A)} > 0\).
\end{theorem}

\begin{example}[continues=emp:NonzeroDet2]
    设 \(A =
    \begin{bmatrix}
        20 & 11 & 1 \\
        4  & 15 & 1 \\
        13 & 15 & 3 \\
    \end{bmatrix}\)
    是一个 \(3\)~级实阵.

    取实阵
    \begin{align*}
        B = \begin{bmatrix}
                1 & 0 & 0 \\
                0 & 2 & 0 \\
                0 & 0 & 1 \\
            \end{bmatrix},
        \quad
        C = \begin{bmatrix}
                1 & 0 & 0  \\
                0 & 1 & 0  \\
                0 & 0 & 10 \\
            \end{bmatrix}.
    \end{align*}
    不难算出, \(\det {(B)} > 0\),
    且 \(\det {(C)} > 0\).

    不难算出
    \begin{align*}
        BAC = \begin{bmatrix}
                  20 & 11 & 10 \\
                  8  & 30 & 20 \\
                  13 & 15 & 30 \\
              \end{bmatrix}.
    \end{align*}
    由此可见, \(BAC\) 是实阵, 且 \([BAC]_{i,i} > 0\).

    不难算出
    \begin{align*}
         & R_1 = \sum_{\substack{1 \leq u \leq 3 \\u \neq 1}}
        {|[BAC]_{1,u}|} = |11| + |10| = 21,      \\
         & R_2 = \sum_{\substack{1 \leq u \leq 3 \\u \neq 2}}
        {|[BAC]_{2,u}|} = |8| + |20| = 28,       \\
         & R_3 = \sum_{\substack{1 \leq u \leq 3 \\u \neq 3}}
        {|[BAC]_{3,u}|} = |13| + |15| = 28,
    \end{align*}
    且
    \begin{align*}
         & [BAC]_{1,1}\,[BAC]_{2,2}
        = 600 > 588 = R_1 R_2,      \\
         & [BAC]_{1,1}\,[BAC]_{3,3}
        = 600 > 588 = R_1 R_3,      \\
         & [BAC]_{2,2}\,[BAC]_{3,3}
        = 30^2 > 28^2 = R_2 R_3.
    \end{align*}
    (注意到, 既然我们已算出,
    每个 \([BAC]_{j,j}\,[BAC]_{k,k}\)
    都大于 \(R_j R_k\)
    (\(j < k\)),
    由乘法的交换律,
    我们不必再判断%
    每个 \([BAC]_{j,j}\,[BAC]_{k,k}\)
    是否都大于 \(R_j R_k\)
    (\(j > k\)).)
    故 \(\det {(BAC)} > 0\).

    既然 \(0 < \det {(BAC)} = \det {(B)} \det {(A)} \det {(C)}\),
    且 \(\det {(B)} > 0\), \(\det {(C)} > 0\),
    故 \(\det {(A)} > 0\).
    (其实, 不难算出, \(\det {(A)} = 476.\))
\end{example}

\endgroup
