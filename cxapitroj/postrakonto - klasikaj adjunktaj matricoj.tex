\section{消去律}

本节, 我们讨论一些运算的消去律.

\begin{theorem}
    设 \(x\), \(y\), \(z\) 是数.
    若 \(x + y = x + z\), 或 \(y + x = z + x\),
    则 \(y = z\).
\end{theorem}

我们知道:

(1)
存在数 \(0\), 使对任何数 \(a\), 必 \(0 + a = a = a + 0\);

(2)
对任何数 \(a\), 必存在数 \(-a\), 使 \((-a) + a = 0 = a + (-a)\).

那么, 若 \(x + y = x + z\),
则
\begin{align*}
    (-x) + (x + y) = (-x) + (x + z).
\end{align*}
由结合律, 有
\begin{align*}
    ((-x) + x) + y = ((-x) + x) + z,
\end{align*}
即
\begin{align*}
    0 + y = 0 + z.
\end{align*}
故 \(y = z\).

类似地, 我们可证,
若 \(y + x = z + x\), 则 \(y = z\).

类似地:

(3)
存在 \(m \times n\)~阵 \(0\),
使对任何 \(m \times n\)~阵 \(X\),
必 \(0 + X = X = X + 0\);

(4)
对任何 \(m \times n\)~阵 \(X\),
必存在 \(m \times n\)~阵 \(-X\),
使 \((-X) + X = 0 = X + (-X)\).

于是, 我们也有,

\begin{theorem}
    设 \(A\), \(B\), \(C\) 都是 \(m \times n\)~阵.
    若 \(A + B = A + C\), 或 \(B + A = C + A\),
    则 \(B = C\).
\end{theorem}

又设 \(x\), \(y\), \(z\) 是数.
那么, 若 \(xy = xz\), 或 \(yx = zx\),
还有 \(y = z\) 吗?
不一定.
毕竟, 对任何数 \(a\), 必 \(0a = 0 = a0\).
于是, 虽然 \(0 \cdot 1 = 0 \cdot 2\),
但 \(1 \neq 2\).

可是, 若我们还要求 \(x \neq 0\), 则 \(y = z\) 是对的.
毕竟:

(5)
存在数 \(1\), 使对任何数 \(a\), 必 \(1a = a = a1\);

(6)
对任何非零的数 \(a\), 必存在数 \(a^{-1}\), 使 \(a^{-1} a = 1 = a a^{-1}\).

设 \(x \neq 0\), 且 \(xy = xz\).
则 \(x^{-1} (xy) = x^{-1} (xz)\).
由结合律, 有 \((x^{-1} x) y = (x^{-1} x) z\).
则 \(1y = 1z\).
则 \(y = z\).

类似地, 可证,
若 \(x \neq 0\), 且 \(yx = zx\), 则 \(y = z\).
总之,

\begin{theorem}
    设 \(x\), \(y\), \(z\) 是数, 且 \(x \neq 0\).
    若 \(xy = xz\), 或 \(yx = zx\), 则 \(y = z\).
\end{theorem}

阵的积也有类似的性质.
不过, 由于阵的积是较复杂的, 相关的事实的论证也是较不简单的.

先看一个较简单的事实.

\begin{theorem}
    设 \(A\), \(B\) 是 \(m \times n\)~阵.
    设 \(x \neq 0\) 是一个数.
    若 \(xA = xB\), 则 \(A = B\).
\end{theorem}

\begin{proof}
    任取正整数 \(i \leq m\) 与 \(j \leq n\).
    则
    \begin{align*}
        x [A]_{i,j} = [xA]_{i,j} = [xB]_{i,j} = x [B]_{i,j}.
    \end{align*}
    由数的积的性质, 既然 \(x \neq 0\),
    则必 \([A]_{i,j} = [B]_{i,j}\).
\end{proof}

以下是本节的主要结论.

\begin{theorem}
    设 \(A\) 是一个 \(n\)~级阵, 且 \(\det {(A)} \neq 0\).

    设 \(n \times m\)~阵 \(B\), \(C\) 适合 \(AB = AC\).
    则 \(B = C\).

    设 \(m \times n\)~阵 \(F\), \(G\) 适合 \(FA = GA\).
    则 \(F = G\).
\end{theorem}

\begin{proof}
    我证第~1~个; 我留第~2~个为您的习题.

    设 \(A\) 是一个 \(n\)~级阵, 且 \(\det {(A)} \neq 0\).
    又设 \(n \times m\)~阵 \(B\), \(C\) 适合 \(AB = AC\).
    则
    \begin{align*}
        \operatorname{adj} {(A)}\, (AB)
        = \operatorname{adj} {(A)}\, (AC).
    \end{align*}
    由结合律,
    \begin{align*}
        (\operatorname{adj} {(A)}\, A) B
        = (\operatorname{adj} {(A)}\, A) C.
    \end{align*}
    由古伴的性质,
    \begin{align*}
        (\det {(A)}\, I) B = (\det {(A)}\, I) C,
    \end{align*}
    即
    \begin{align*}
        \det {(A)}\, (I B) = \det {(A)}\, (I C),
    \end{align*}
    即
    \begin{align*}
        \det {(A)}\, B = \det {(A)}\, C.
    \end{align*}
    因为 \(\det {(A)} \neq 0\),
    我们有 \(B = C\).
\end{proof}

以上结论也可被推广.
不过, 还是以上结论更常用, 更常见.

\begin{theorem}
    设 \(A\) 是一个 \(s \times n\)~阵.
    设 \(A\) 有一个行列式非零的 \(n\)~级子阵
    \begin{align*}
        A\binom{i_1,i_2,\dots,i_n}{1,2,\dots,n},
    \end{align*}
    其中 \(1 \leq i_1 < \dots < i_n \leq s\).
    设 \(n \times m\)~阵 \(B\), \(C\) 适合 \(AB = AC\).
    则 \(B = C\).

    设 \(H\) 是一个 \(n \times t\)~阵.
    设 \(H\) 有一个行列式非零的 \(n\)~级子阵
    \begin{align*}
        H\binom{1,2,\dots,n}{j_1,j_2,\dots,j_n},
    \end{align*}
    其中 \(1 \leq j_1 < \dots < j_n \leq t\).
    设 \(m \times n\)~阵 \(F\), \(G\) 适合 \(FH = GH\).
    则 \(F = G\).
\end{theorem}

\begin{proof}
    我证第~1~个; 我留第~2~个为您的习题.

    记
    \begin{align*}
        L = A\binom{i_1,i_2,\dots,i_n}{1,2,\dots,n}.
    \end{align*}
    注意到, \([L]_{p,v} = [A]_{i_p,v}\).

    从 \(1\), \(2\), \(\dots\), \(s\)
    去除 \(i_1\), \(i_2\), \(\dots\), \(i_n\)
    后, 还剩 \(s - n\)~个数.
    我们从小到大地叫这 \(s - n\)~个数为
    \(i_{n+1}\), \(\dots\), \(i_s\).

    作 \(n \times s\)~阵 \(X\) 如下:
    \begin{align*}
        [X]_{u,i_p}
        = \begin{cases}
              [\operatorname{adj} {(L)}]_{u,p},
                 & p \leq n; \\
              0, & p > n.
          \end{cases}
    \end{align*}
    则
    \begin{align*}
        [XA]_{u,v}
        = {} &
        \sum_{p = 1}^{s} {[X]_{u,p} [A]_{p,v}}
        \\
        = {} &
        \sum_{p = 1}^{s} {[X]_{u,i_p} [A]_{i_p,v}}
        \\
        = {} &
        \sum_{p = 1}^{n} {[X]_{u,i_p} [A]_{i_p,v}}
        + \sum_{p = n+1}^{s} {[X]_{u,i_p} [A]_{i_p,v}}
        \\
        = {} &
        \sum_{p = 1}^{n}
        {[\operatorname{adj} {(L)}]_{u,p} [L]_{p,v}}
        + \sum_{p = n+1}^{s} {0\, [A]_{i_p,v}}
        \\
        = {} &
        [\operatorname{adj} {(L)}\, L]_{u,v} + 0
        \\
        = {} &
        [\det {(L)}\,I_n]_{u,v}.
    \end{align*}
    故 \(X A = \det {(L)}\, I_n\).

    由 \(AB = AC\), 知
    \begin{align*}
        X (AB) = X (AC),
    \end{align*}
    即
    \begin{align*}
        (X A) B = (X A) C,
    \end{align*}
    即
    \begin{align*}
        (\det {(L)}\,I_n) B = (\det {(L)}\,I_n) C,
    \end{align*}
    即
    \begin{align*}
        \det {(L)}\,(I_n B) = \det {(L)}\,(I_n C),
    \end{align*}
    即
    \begin{align*}
        \det {(L)}\,B = \det {(L)}\,C.
    \end{align*}
    因为 \(\det {(L)} \neq 0\),
    故 \(B = C\).
\end{proof}

\section{古伴的性质 (1)}

本节, 我们认识古伴的二个性质.

一个 \(n\)~级阵 \(A\) 的古伴 \(\operatorname{adj} {(A)}\)
是一个 \(n\)~级阵,
其 \((i, j)\)-元
\begin{align*}
    [\operatorname{adj} {(A)}]_{i,j} = (-1)^{j+i} \det {(A(j|i))}.
\end{align*}
对任何方阵 \(A\), 有
\begin{align*}
    \operatorname{adj} {(A)}\, A
    = \det {(A)}\, I
    = A \operatorname{adj} {(A)}.
\end{align*}

\begin{theorem}
    设 \(A\) 是 \(n\)~级阵.
    则 \(\operatorname{adj} {(A^{\mathrm{T}})}
    = (\operatorname{adj} {(A)})^{\mathrm{T}}\).
\end{theorem}

\begin{proof}
    注意到
    \(A^{\mathrm{T}} (j|i) = (A (i|j))^{\mathrm{T}}\),
    故
    \begin{align*}
        [\operatorname{adj} {(A^{\mathrm{T}})}]_{i,j}
        = {} &
        (-1)^{j+i} \det {(A^{\mathrm{T}} (j|i))}
        \\
        = {} &
        (-1)^{j+i} \det {((A (i|j))^{\mathrm{T}})}
        \\
        = {} &
        (-1)^{i+j} \det {(A (i|j))}
        \\
        = {} &
        [\operatorname{adj} {(A)}]_{j,i}
        \\
        = {} &
        [(\operatorname{adj} {(A)})^{\mathrm{T}}]_{i,j}.
        \qedhere
    \end{align*}
\end{proof}

\begin{theorem}
    设 \(A\) 是 \(n\)~级阵 (\(n \geq 2\)).
    则
    \(\det {(\operatorname{adj} {(A)})} = (\det {(A)})^{n-1}\).
\end{theorem}

\begin{proof}
    不难看出, 若 \(B\) 是 \(n\)~级阵, \(k\) 是数, 则
    \(\det {(kB)} = k^n \det {(B)}\).
    于是
    \begin{align*}
        \det {(A)}\, (\det {(A)})^{n-1}
        = {} &
        (\det {(A)})^n
        \\
        = {} &
        (\det {(A)})^n \det {(I)}
        \\
        = {} &
        \det {(\det {(A)} I)}
        \\
        = {} &
        \det {(A \operatorname{adj} {(A)})}
        \\
        = {} &
        \det {(A)} \det {(\operatorname{adj} {(A)})}.
    \end{align*}
    若 \(\det {(A)} \neq 0\),
    我们可在等式的二侧消去它,
    即得结论.

    % 若 \(\det {(A)} = 0\),
    % 我们用别的方法说明 \(\det {(\operatorname{adj} {(A)})} = 0\).

    若 \(A = 0\), 则 \(\operatorname{adj} {(A)} = 0\),
    故 \(\det {(\operatorname{adj} {(A)})} = 0\).

    若 \(A \neq 0\), 且 \(\det {(A)} = 0\),
    我们用反证法说明
    \(\det {(\operatorname{adj} {(A)})} = 0\).
    反设
    \(\det {(\operatorname{adj} {(A)})} \neq 0\).
    则
    \begin{align*}
        \operatorname{adj} {(A)}\, A
        = \det {(A)}\, I
        = 0
        = \operatorname{adj} {(A)}\, 0.
    \end{align*}
    因为
    \(\det {(\operatorname{adj} {(A)})} \neq 0\),
    由消去律, 有 \(A = 0\).
    这是矛盾.
\end{proof}

\section{古伴的性质 (2)}

本节, 我们讨论古伴的子阵的行列式.

% \(A\) 的古伴来自 \(A\) 的子阵.
% 自然地, 我们会思考
% \(A\) 的古伴的子阵与 \(A\) 的子阵的关系.

\begin{theorem}
    设 \(A\) 是 \(n\)~级阵.
    设 \(k\) 是不超过 \(n\) 的正整数.
    设正整数 \(i_1\), \(i_2\), \(\dots\), \(i_k\)
    不超过 \(n\), 且是从小到大的.
    设正整数 \(j_1\), \(j_2\), \(\dots\), \(j_k\)
    不超过 \(n\), 且是从小到大的.
    则
    \begin{equation}
        \begin{aligned}
                 & \det {\left(
                (\operatorname{adj} {(A)})
                \binom{i_1,\dots,i_k}{j_1,\dots,j_k}
                \right)}
            \\
            = {} &
            (\det {(A)})^{k-1}\,
            (-1)^{j_1 + \dots + j_k + i_1 + \dots + i_k}
            \det {(A({j_1,\dots,j_k}|{i_1,\dots,i_k}))}.
        \end{aligned}
        \label{eq:C2401}
    \end{equation}
\end{theorem}

此事对 \(k = 1\) 是正确的.
此时, 等式的左侧即为 \(\operatorname{adj} {(A)}\)
的 \((i_1, j_1)\)-元,
而等式的右侧是
\(1 \, (-1)^{j_1 + i_1} \det {(A(j_1|i_1))}\).
这正是古伴的定义.

以下设 \(k > 1\).

从 \(1\), \(2\), \(\dots\), \(n\)
去除 \(i_1\), \(\dots\), \(i_k\) 后,
还剩 \(n - k\) 个数.
我们从小到大地叫这 \(n - k\) 个数为
\(i_{k+1}\), \(\dots\), \(i_n\).
类似地, 从 \(1\), \(2\), \(\dots\), \(n\)
去除 \(j_1\), \(\dots\), \(j_k\) 后,
还剩 \(n - k\) 个数.
我们从小到大地叫这 \(n - k\) 个数为
\(j_{k+1}\), \(\dots\), \(j_n\).
作 \(n\)~级阵 \(B\) 如下:
\begin{align*}
    [B]_{i,j_q}
    = \begin{cases}
          [\operatorname{adj} {(A)}]_{i,j_q},
           & q \leq k; \\
          [I_n]_{i,i_q},
           & q > k.
      \end{cases}
\end{align*}
形象地, \(B\) 的列 \(j_1\), \(\dots\), \(j_k\)
分别与 \(\operatorname{adj} {(A)}\) 的列 \(j_1\), \(\dots\), \(j_k\)
相等,
但 \(B\) 的列 \(j_{k+1}\), \(\dots\), \(j_n\)
分别是 \(n\)~级单位阵 \(I_n\) 的列 \(i_{k+1}\), \(\dots\), \(i_n\).
由此可见,
\begin{align*}
    B\binom{i_1,\dots,i_k}{j_1,\dots,j_k}
    = B({i_{k+1},\dots,i_n}|{j_{k+1},\dots,j_n})
    = (\operatorname{adj} {(A)})
    \binom{i_1,\dots,i_k}{j_1,\dots,j_k},
\end{align*}
且
\begin{align*}
    B({i_1,\dots,i_k}|{j_1,\dots,j_k})
    = B\binom{i_{k+1},\dots,i_n}{j_{k+1},\dots,j_n}
    = I_{n-k}.
\end{align*}
% 其中 \(I_{n-k}\) 指 \(n-k\)~级单位阵.

按列 \(j_{k+1}\) 展开 \(\det {(B)}\), 有
\begin{align*}
    \det {(B)} = (-1)^{i_{k+1}+j_{k+1}} \, 1
    \det {(B(i_{k+1}|j_{j+1}))}.
\end{align*}
注意到 \(B\) 的列 \(j_{k+2}\) 即为
\(B(i_{k+1}|j_{k+1})\) 的列 \(j_{k+2}-1\).
按列 \(j_{k+2}-1\) 展开 \(\det {(B(i_{k+1}|j_{k+1}))}\),
有
\begin{align*}
    \det {(B(i_{k+1}|j_{k+1}))} = (-1)^{i_{k+2}-1+j_{k+2}-1} \, 1
    \det {(B({i_{k+1},i_{k+2}}|{j_{k+1},j_{k+2}}))}.
\end{align*}
故
\begin{align*}
    \det {(B)} = (-1)^{i_{k+1}+i_{k+2}+j_{k+1}+j_{k+2}}
    \det {(B({i_{k+1},i_{k+2}}|{j_{k+1},j_{k+2}}))}.
\end{align*}
\(\dots \dots\)
最后, 我们有
\begin{align*}
    \det {(B)}
    = (-1)^{i_{k+1}+\dots+i_n+j_{k+1}+\dots+j_n}
    \det {(B({i_{k+1},\dots,i_n}|{j_{k+1},\dots,j_n}))}.
\end{align*}

我们考虑 \(AB\) 的行列式.
一方面,
\begin{align*}
    \det {(AB)}
    = {} & \det {(A)} \det {(B)}
    \\
    = {} & (-1)^{i_{k+1}+\dots+i_n+j_{k+1}+\dots+j_n}
    \det {(A)}
    \det {\left(
        (\operatorname{adj} {(A)})
        \binom{i_1,\dots,i_k}{j_1,\dots,j_k}
        \right)}.
\end{align*}
另一方面, 当 \(q \leq k\) 时,
\begin{align*}
    [AB]_{i,j_q}
    = {} &
    \sum_{u=1}^{n} {[A]_{i,u} [B]_{u,j_q}}
    \\
    = {} &
    \sum_{u=1}^{n} {[A]_{i,u} [\operatorname{adj} {(A)}]_{u,j_q}}
    \\
    = {} &
    [A \operatorname{adj} {(A)}]_{i,j_q}
    \\
    = {} &
    [\det {(A)}\,I_n]_{i,j_q};
\end{align*}
当 \(q > k\) 时,
\begin{align*}
    [AB]_{i,j_q}
    = {} &
    \sum_{u=1}^{n} {[A]_{i,u} [B]_{u,j_q}}
    \\
    = {} &
    \sum_{u=1}^{n} {[A]_{i,u} [I_n]_{u,i_q}}
    \\
    = {} &
    [A I_n]_{i,i_q}
    \\
    = {} &
    [A]_{i,i_q}.
\end{align*}
形象地, \(AB\) 的列 \(j_1\), \(\dots\), \(j_k\)
分别与 \(\det {(A)}\,I_n\) 的列 \(j_1\), \(\dots\), \(j_k\)
相等,
但 \(AB\) 的列 \(j_{k+1}\), \(\dots\), \(j_n\)
分别是 \(A\) 的列 \(i_{k+1}\), \(\dots\), \(i_n\).
由此可见,
\begin{align*}
    (AB)\binom{j_1,\dots,j_k}{j_1,\dots,j_k}
    = (AB)({j_{k+1},\dots,j_n}|{j_{k+1},\dots,j_n})
    = \det {(A)}\, I_k,
\end{align*}
且
\begin{align*}
    (AB)({j_1,\dots,j_k}|{j_1,\dots,j_k})
    = (AB)\binom{j_{k+1},\dots,j_n}{j_{k+1},\dots,j_n}
    = A({j_1,\dots,j_k}|{i_1,\dots,i_k}).
\end{align*}

按列 \(j_1\) 展开 \(\det {(AB)}\), 有
\begin{align*}
    \det {(AB)} = (-1)^{j_1+j_1} \det {(A)}
    \det {((AB)(j_1|j_1))}.
\end{align*}
注意到 \(AB\) 的列 \(j_2\) 即为
\((AB)(j_1|j_1)\) 的列 \(j_2-1\).
按列 \(j_2-1\) 展开 \(\det {((AB)(j_1|j_1))}\),
有
\begin{align*}
    \det {((AB)(j_1|j_1))}
    = (-1)^{j_2-1+j_2-1} \det {(A)}
    \det {((AB)({j_1,j_2}|{j_1,j_2}))}.
\end{align*}
故
\begin{align*}
    \det {(AB)} = (\det {(A)})^2
    \det {((AB)({i_1,i_2}|{j_1,j_2}))}.
\end{align*}
\(\dots \dots\)
最后, 我们有
\begin{align*}
    \det {(AB)}
    = (\det {(A)})^k
    \det {((AB)({j_1,\dots,j_k}|{j_1,\dots,j_k}))}.
\end{align*}

比较二次计算的结果, 我们应有
\begin{align*}
         &
    (-1)^{i_{k+1}+\dots+i_n+j_{k+1}+\dots+j_n}
    \det {(A)}
    \det {\left(
        (\operatorname{adj} {(A)})
        \binom{i_1,\dots,i_k}{j_1,\dots,j_k}
        \right)}
    \\
    = {} &
    (\det {(A)})^k
    \det {(A({j_1,\dots,j_k}|{i_1,\dots,i_k}))}.
\end{align*}
注意到
\begin{align*}
         &
    i_{k+1} + \dots + i_n + j_{k+1} + \dots + j_n
    \\
    = {} &
    ((1 + 2 + \dots + n) - (i_1 + \dots + i_k))
    + ((1 + 2 + \dots + n) - (j_1 + \dots + j_k))
    \\
    = {} &
    2(1 + 2 + \dots + n) - (j_1 + \dots + j_k + i_1 + \dots + i_k),
\end{align*}
故
\begin{align*}
         &
    \det {(A)}\,
    \det {\left(
        (\operatorname{adj} {(A)})
        \binom{i_1,\dots,i_k}{j_1,\dots,j_k}
        \right)}
    \\
    = {} &
    \det {(A)}\, (\det {(A)})^{k-1}\,
    (-1)^{j_1+\dots+j_k+i_1+\dots+i_k}
    \det {(A({j_1,\dots,j_k}|{i_1,\dots,i_k}))}.
\end{align*}

若 \(\det {(A)} \neq 0\),
我们可在等式的二侧消去它, 得式~\eqref{eq:C2401}.
不过, 若 \(\det {(A)} = 0\),
会发生什么?

\begin{theorem}
    设 \(A\) 是一个 \(n\)~级阵.
    设 \(\det {(A)} = 0\).
    则存在不超过 \(n\) 的正整数 \(s\),
    使对任何不超过 \(n\) 的正整数 \(j\),
    存在数 \(m_j\),
    使对任何不超过 \(n\) 的正整数 \(i\),
    有
    \([\operatorname{adj} {(A)}]_{i,j}
    = m_j [\operatorname{adj} {(A)}]_{i,s}\);
    通俗地,
    存在不超过 \(n\) 的正整数 \(s\),
    使 \(\operatorname{adj} {(A)}\) 的任何一列
    都是 \(\operatorname{adj} {(A)}\) 的列~\(s\) 的数乘.
\end{theorem}

有了此事, 不难看出, 若 \(\det {(A)} = 0\),
且 \(k > 1\), 则
\begin{align*}
    \det {\left(
        (\operatorname{adj} {(A)})
        \binom{i_1,\dots,i_k}{j_1,\dots,j_k}
        \right)} = 0,
\end{align*}
故式~\eqref{eq:C2401} 仍是正确的.

那么, 若我们证明了此事, 则我们也就证明了式~\eqref{eq:C2401}.

\begin{proof}
    若 \(\operatorname{adj} {(A)} = 0\),
    则 \(\operatorname{adj} {(A)}\) 的每一列都是零.
    从而 \(\operatorname{adj} {(A)}\)
    的每一列都是
    \(\operatorname{adj} {(A)}\) 的列~\(1\) 的数乘.

    下设 \(\operatorname{adj} {(A)} \neq 0\).
    则有不超过 \(n\) 的正整数 \(s\), \(t\),
    使 \([\operatorname{adj} {(A)}]_{t,s} \neq 0\).
    则 \((-1)^{s+t} \det {(A(s|t))} \neq 0\).
    作 \(n-1\)~级阵 \(C = A(s|t)\).
    则 \(\det {(C)} \neq 0\).

    从 \(1\), \(2\), \(\dots\), \(n\)
    去除 \(t\) 后, 还剩 \(n - 1\) 个数.
    我们从小到大地叫这 \(n - 1\) 个数为
    \(i_1\), \(\dots\), \(i_{n-1}\).
    类似地, 从 \(1\), \(2\), \(\dots\), \(n\)
    去除 \(s\) 后, 还剩 \(n - 1\) 个数.
    我们从小到大地叫这 \(n - 1\) 个数为
    \(\ell_1\), \(\dots\), \(\ell_{n-1}\).
    则 \([C]_{p,q} = [A]_{\ell_p,i_q}\).
    再记 \(i_n = t\), 且 \(\ell_n = s\).

    设 \(j\) 是不超过 \(n\) 的正整数.
    作 \((n-1) \times 1\)~阵 \(y_j\) 如下:
    \begin{align*}
        [y_j]_{u,1} = [\operatorname{adj} {(A)}]_{i_u,j};
    \end{align*}
    通俗地, 若 \(D_j\) 是
    \(\operatorname{adj} {(A)}\) 的列~\(j\),
    则 \(y_j\) 是去除 \(D_j\) 的行~\(t\) 后的那个
    \((n-1) \times 1\)~阵.
    则对 \(p < n\),
    \begin{align*}
        [C y_j]_{p,1}
        = {} &
        \sum_{u=1}^{n-1} {[C]_{p,u} [y_j]_{u,1}}
        \\
        = {} &
        \sum_{u=1}^{n-1}
        {[A]_{\ell_p,i_u} [\operatorname{adj} {(A)}]_{i_u,j}}
        \\
        = {} &
        \sum_{u=1}^{n}
        {[A]_{\ell_p,i_u} [\operatorname{adj} {(A)}]_{i_u,j}}
        - [A]_{\ell_p,i_n} [\operatorname{adj} {(A)}]_{i_n,j}
        \\
        = {} &
        \sum_{u=1}^{n}
        {[A]_{\ell_p,u} [\operatorname{adj} {(A)}]_{u,j}}
        - [A]_{\ell_p,t} [\operatorname{adj} {(A)}]_{t,j}
        \\
        = {} &
        [A \operatorname{adj} {(A)}]_{\ell_p,j}
        - [A]_{\ell_p,t} [\operatorname{adj} {(A)}]_{t,j}
        \\
        = {} &
        [\det {(A)}\, I_n]_{\ell_p,j}
        - [A]_{\ell_p,t} [\operatorname{adj} {(A)}]_{t,j}
        \\
        = {} &
        {- [\operatorname{adj} {(A)}]_{t,j} [A]_{\ell_p,t}}.
    \end{align*}

    作 \((n-1) \times 1\)~阵 \(z\) 如下:
    \begin{align*}
        [z]_{u,1} = [A]_{\ell_u,t};
    \end{align*}
    通俗地, 若 \(w\) 是 \(A\) 的列~\(t\),
    则 \(z\) 是去除 \(w\) 的行~\(s\) 后的那个
    \((n-1) \times 1\)~阵.

    由前面的计算,
    \(C y_j = -[\operatorname{adj} {(A)}]_{t,j} z\),
    且
    \(C y_s = -[\operatorname{adj} {(A)}]_{t,s} z\).
    记
    \begin{align*}
        m_j =
        \frac{-[\operatorname{adj} {(A)}]_{t,j}}
        {-[\operatorname{adj} {(A)}]_{t,s}}.
    \end{align*}
    则
    \begin{align*}
        C (m_j y_s)
        = {} & m_j (C y_s)
        = m_j (-[\operatorname{adj} {(A)}]_{t,s} z)
        = (m_j (-[\operatorname{adj} {(A)}]_{t,s})) z
        \\
        = {} &
        {-[\operatorname{adj} {(A)}]_{t,j} z}
        \\
        = {} &
        C y_j.
    \end{align*}
    % % 于是, \(X = y_j\) 与 \(X = m_j y_s\) 都是线性方程组
    % % \(CX = -[\operatorname{adj} {(A)}]_{t,j} z\)
    % % 的解.
    % % 因为 \(\det {(C)} \neq 0\),
    % % 故 \(y_j = m_j y_s\).
    % 则
    % \begin{align*}
    %     \operatorname{adj} {(C)}\, (C (m_j y_s))
    %     = \operatorname{adj} {(C)}\, (C y_j),
    % \end{align*}
    % 即
    % \begin{align*}
    %     (\operatorname{adj} {(C)}\, C) (m_j y_s)
    %     = (\operatorname{adj} {(C)}\, C) (y_j),
    % \end{align*}
    % 即
    % \begin{align*}
    %     (\det {(C)}\, I_{n-1}) (m_j y_s)
    %     = (\det {(C)}\, I_{n-1}) (y_j),
    % \end{align*}
    % 即
    % \begin{align*}
    %     \det {(C)}\, (I_{n-1} (m_j y_s))
    %     = \det {(C)}\, (I_{n-1} y_j),
    % \end{align*}
    % 即
    % \begin{align*}
    %     \det {(C)}\, (m_j y_s)
    %     = \det {(C)}\, y_j.
    % \end{align*}
    % 因为数 \(\det {(C)} \neq 0\),
    因为 \(\det {(C)} \neq 0\),
    故 \(m_j y_s = y_j\).
    故 \(m_j [\operatorname{adj} {(A)}]_{i_u,s}
        = [\operatorname{adj} {(A)}]_{i_u,j}\),
    对 \(u < n\).
    则 \(m_j [\operatorname{adj} {(A)}]_{i,s}
        = [\operatorname{adj} {(A)}]_{i,j}\),
    对 \(i \leq n\).
\end{proof}

% 最后, 为方便, 我写下式~\eqref{eq:C2401} 的一个特别的情形.

% 设 \(A\) 是 \(n\)~级阵 (\(n \geq 3\)).

% 设 \(1 \leq u < v \leq n\).
% 取 \(k = 2\),
% \(i_1 = j_1 = u\),
% \(i_2 = j_2 = v\),
% 有
% \begin{align*}
%     \det {\left(
%         (\operatorname{adj} {(A)})
%         \binom{u,v}{u,v}
%         \right)}
%     = (\det {(A)})^{2-1}\,
%     (-1)^{u+v+u+v}
%     \det {(A({u,v}|{u,v}))}.
% \end{align*}
% 不难算出
% \begin{align*}
%     \det {(A(u|u))} \det {(A(v|v))}
%     - \det {(A(u|v))} \det {(A(v|u))}
%     = \det {(A)} \det {(A({u,v}|{u,v}))}.
% \end{align*}

% 若 \(1 \leq v < u \leq n\),
% 则, 类似地,
% \begin{align*}
%     \det {(A(v|v))} \det {(A(u|u))}
%     - \det {(A(v|u))} \det {(A(u|v))}
%     = \det {(A)} \det {(A({v,u}|{v,u}))},
% \end{align*}
% 也就是
% \begin{align*}
%     \det {(A(u|u))} \det {(A(v|v))}
%     - \det {(A(u|v))} \det {(A(v|u))}
%     = \det {(A)} \det {(A({u,v}|{u,v}))}.
% \end{align*}

% 总结这些结果, 有

% \begin{theorem}
%     设 \(A\) 是 \(n\)~级阵 (\(n \geq 3\)).
%     设 \(u\), \(v\) 是不超过 \(n\) 的正整数, 且互不相同.
%     则
%     \begin{align*}
%         \det {(A(u|u))} \det {(A(v|v))}
%         = \det {(A(u|v))} \det {(A(v|u))}
%         + \det {(A)} \det {(A({u,v}|{u,v}))}.
%     \end{align*}
%     特别地, 若 \(\det {(A)} = 0\), 则
%     \begin{align*}
%         \det {(A(u|u))} \det {(A(v|v))}
%         = \det {(A(u|v))} \det {(A(v|u))}.
%     \end{align*}
% \end{theorem}

\section{古伴的性质 (3)}

本节, 我们再认识古伴的二个性质.

\begin{theorem}
    设 \(A\) 是 \(n\)~级阵 (\(n \geq 2\)).
    则
    \(\operatorname{adj} {(\operatorname{adj} {(A)})}
    = (\det {(A)})^{n-2}\,A\).
\end{theorem}

\begin{proof}
    设 \(n = 2\).
    不难算出, \(2\)~级阵
    \(A =
    \begin{bmatrix}
        a & b \\
        c & d \\
    \end{bmatrix}
    \)
    的古伴是
    \(
    \begin{bmatrix}
        d  & -b \\
        -c & a  \\
    \end{bmatrix}
    \).
    则
    \(
    \begin{bmatrix}
        d  & -b \\
        -c & a  \\
    \end{bmatrix}
    \)
    的古伴是
    \(
    \begin{bmatrix}
        a     & -(-b) \\
        -(-c) & d     \\
    \end{bmatrix}
    \),
    即 \(A\).
    注意到,
    \((\det {(A)})^{2-2} A = 1 A = A\).

    下设 \(n > 2\).
    一方面,
    \begin{align*}
        \operatorname{adj} {(A)}
        \operatorname{adj} {(\operatorname{adj} {(A)})}
        = \det {(\operatorname{adj} {(A)})}\, I_n
        = (\det {(A)})^{n-1}\, I_n;
    \end{align*}
    另一方面,
    \begin{align*}
        \operatorname{adj} {(A)}\,
        ((\det {(A)})^{n-2}\, A)
        = {} &
        (\det {(A)})^{n-2}\,
        (\operatorname{adj} {(A)} A)
        \\
        = {} &
        (\det {(A)})^{n-2}\, (\det {(A)}\,I_n)
        \\
        = {} &
        (\det {(A)})^{n-1}\, I_n.
    \end{align*}
    故
    \begin{align*}
        \operatorname{adj} {(A)}
        \operatorname{adj} {(\operatorname{adj} {(A)})}
        =
        \operatorname{adj} {(A)}\,
        ((\det {(A)})^{n-2} A).
    \end{align*}
    若 \(\det {(A)} \neq 0\),
    则 \(\det {(\operatorname{adj} {(A)})}
    = (\det {(A)})^{n-1} \neq 0\).
    由消去律,
    我们可在等式的二侧消去 \(\operatorname{adj} {(A)}\),
    即得结论.

    若 \(\det {(A)} = 0\),
    由上节的结论, \(\operatorname{adj} {(A)}\)~的%
    每一个 \(n-1\)~级子阵的行列式都是 \(0\).
    故 \(\operatorname{adj} {(\operatorname{adj} {(A)})}
    = 0 = 0A = (\det {(A)})^{n-2}\,A\).
\end{proof}

\begin{theorem}
    设 \(A\), \(B\) 是 \(n\)~级阵.
    则
    \begin{align*}
        \operatorname{adj} {(BA)}
        = \operatorname{adj} {(A)} \operatorname{adj} {(B)}.
    \end{align*}
\end{theorem}

\begin{proof}
    注意到
    \begin{align*}
        (BA) \operatorname{adj} {(BA)}
        = {} &
        \det {(BA)}\, I
        \\
        = {} &
        (\det {(B)} \det {(A)})\, I
        \\
        = {} &
        (\det {(A)} \det {(B)})\, I
        \\
        = {} &
        \det {(A)}\, (\det {(B)}\, I)
        \\
        = {} &
        \det {(A)}\, (B \operatorname{adj} {(B)})
        \\
        = {} &
        B (\det {(A)} \operatorname{adj} {(B)})
        \\
        = {} &
        B (\det {(A)}\, (I \operatorname{adj} {(B)}))
        \\
        = {} &
        B ((\det {(A)}\, I) \operatorname{adj} {(B)})
        \\
        = {} &
        B ((A \operatorname{adj} {(A)})\, \operatorname{adj} {(B)})
        \\
        = {} &
        B (A\, (\operatorname{adj} {(A)} \operatorname{adj} {(B)}))
        \\
        = {} &
        (B A)\, (\operatorname{adj} {(A)} \operatorname{adj} {(B)}),
    \end{align*}
    若 \(\det {(BA)} \neq 0\),
    由消去律, 即得结论.

    若 \(\det {(BA)} = 0\), 我们这么作.
    由阵的积的定义,
    \begin{align*}
        [\operatorname{adj} {(A)} \operatorname{adj} {(B)}]_{i,j}
        = {} &
        \sum_{u = 1}^{n}
        {[\operatorname{adj} {(A)}]_{i,u}
            [\operatorname{adj} {(B)}]_{u,j}}
        \\
        = {} &
        \sum_{u = 1}^{n}
        {(-1)^{u+i} \det {(A(u|i))}\,
        (-1)^{j+u} \det {(B(j|u))}}
        \\
        = {} &
        \sum_{u = 1}^{n}
        {(-1)^{u+i} (-1)^{j+u}
        \det {(A(u|i))} \det {(B(j|u))}}
        \\
        = {} &
        \sum_{u = 1}^{n}
        {(-1)^{j+i} \det {(B(j|u))} \det {(A(u|i))}}
        \\
        = {} &
        (-1)^{j+i}
        \sum_{u = 1}^{n}
        {\det {(B(j|u))} \det {(A(u|i))}}
        \\
        = {} &
        (-1)^{j+i}
        \det {((BA)(j|i))}
        \\
        = {} &
        [\operatorname{adj} {(BA)}]_{i,j}.
    \end{align*}
    这里, 我们用了 Binet--Cauchy 公式的推广.
\end{proof}

% 为方便, 我记下一个简单的推广.
% 设 \(A\), \(B\), \(C\) 是 \(n\)~级阵.
% 设 \(\det {(CBA)} \neq 0\).
% 因为 \(\det {(CBA)} = \det {(C)} \det {(BA)}\),
% 故 \(\det {(BA)} \neq 0\).
% 故
% \begin{align*}
%     \operatorname{adj} {(BA)}
%     = \operatorname{adj} {(A)} \operatorname{adj} {(B)}.
% \end{align*}
% 则
% \begin{align*}
%     \operatorname{adj} {(CBA)}
%     = {} &
%     \operatorname{adj} {(C(BA))}
%     \\
%     = {} &
%     \operatorname{adj} {(BA)}
%     \operatorname{adj} {(C)}
%     \\
%     = {} &
%     (\operatorname{adj} {(A)} \operatorname{adj} {(B)})
%     \operatorname{adj} {(C)}
%     \\
%     = {} &
%     \operatorname{adj} {(A)}
%     \operatorname{adj} {(B)}
%     \operatorname{adj} {(C)}.
% \end{align*}

\section{阵的积, 分块地写}

我们知道, 若 \(B\) 是 \(s \times m\)~阵,
且 \(a_1\), \(a_2\), \(\dots\), \(a_n\)
是 \(m \times 1\) 阵,
则
\begin{align*}
    B\,[a_1, a_2, \dots, a_n] = [Ba_1, Ba_2, \dots, Ba_n].
\end{align*}
形象地, 这是分块地写阵的积.
证明 \(\det {(BA)} = \det {(B)} \det {(A)}\)
(\(A\), \(B\) 是同级的方阵) 时,
我们用了此事, 使推理更简单.

我要介绍另一种分块地写阵的积的方式.

\begin{theorem}
    设
    \(A_{1,1}\), \(A_{1,2}\), \(A_{2,1}\), \(A_{2,2}\),
    \(B_{1,1}\), \(B_{1,2}\), \(B_{2,1}\), \(B_{2,2}\)
    是阵.
    设 \(A_{1,1}\), \(A_{1,2}\) 都有 \(r_1\) 行;
    设 \(A_{2,1}\), \(A_{2,2}\) 都有 \(r_2\) 行;
    设 \(A_{1,1}\), \(A_{2,1}\) 都有 \(u_1\) 列;
    设 \(A_{1,2}\), \(A_{2,2}\) 都有 \(u_2\) 列.
    设 \(B_{1,1}\), \(B_{1,2}\) 都有 \(u_1\) 行;
    设 \(B_{2,1}\), \(B_{2,2}\) 都有 \(u_2\) 行;
    设 \(B_{1,1}\), \(B_{2,1}\) 都有 \(c_1\) 列;
    设 \(B_{1,2}\), \(B_{2,2}\) 都有 \(c_2\) 列.
    作 \((r_1 + r_2) \times (u_1 + u_2)\)~阵
    \begin{align*}
        A = \begin{bmatrix}
                A_{1,1} & A_{1,2} \\
                A_{2,1} & A_{2,2} \\
            \end{bmatrix};
    \end{align*}
    具体地,
    \begin{align*}
        [A]_{i,j}
        = \begin{cases}
              [A_{1,1}]_{i,j},
               & \text{\(i \leq r_1\), \(j \leq u_1\)}; \\
              [A_{1,2}]_{i,j-u_1},
               & \text{\(i \leq r_1\), \(j > u_1\)};    \\
              [A_{2,1}]_{i-r_1,j},
               & \text{\(i > r_1\), \(j \leq u_1\)};    \\
              [A_{2,2}]_{i-r_1,j-u_1},
               & \text{\(i > r_1\), \(j > u_1\)}.
          \end{cases}
    \end{align*}
    作 \((u_1 + u_2) \times (c_1 + c_2)\)~阵
    \begin{align*}
        B = \begin{bmatrix}
                B_{1,1} & B_{1,2} \\
                B_{2,1} & B_{2,2} \\
            \end{bmatrix};
    \end{align*}
    具体地,
    \begin{align*}
        [B]_{i,j}
        = \begin{cases}
              [B_{1,1}]_{i,j},
               & \text{\(i \leq u_1\), \(j \leq c_1\)}; \\
              [B_{1,2}]_{i,j-c_1},
               & \text{\(i \leq u_1\), \(j > c_1\)};    \\
              [B_{2,1}]_{i-u_1,j},
               & \text{\(i > u_1\), \(j \leq c_1\)};    \\
              [B_{2,2}]_{i-u_1,j-c_1},
               & \text{\(i > u_1\), \(j > c_1\)}.
          \end{cases}
    \end{align*}
    因为 \(A\) 的列数等于 \(B\) 的行数,
    故 \(AB\) 有意义.

    记 \(C_{i,j} = A_{i,1} B_{1,j} + A_{i,2} B_{2,j}\),
    \(i\), \(j = 1\), \(2\).
    不难看出, \(C_{i,j}\) 是 \(r_i \times c_j\)~阵.
    作 \((r_1+r_2) \times (c_1+c_2)\)~阵
    \begin{align*}
        C = \begin{bmatrix}
                C_{1,1} & C_{1,2} \\
                C_{2,1} & C_{2,2} \\
            \end{bmatrix};
    \end{align*}
    具体地,
    \begin{align*}
        [C]_{i,j}
        = \begin{cases}
              [C_{1,1}]_{i,j},
               & \text{\(i \leq r_1\), \(j \leq c_1\)}; \\
              [C_{1,2}]_{i,j-c_1},
               & \text{\(i \leq r_1\), \(j > c_1\)};    \\
              [C_{2,1}]_{i-r_1,j},
               & \text{\(i > r_1\), \(j \leq c_1\)};    \\
              [C_{2,2}]_{i-r_1,j-c_1},
               & \text{\(i > r_1\), \(j > c_1\)}.
          \end{cases}
    \end{align*}
    则 \(AB = C\).

    形象地,
    \begin{align*}
        \begin{bmatrix}
            A_{1,1} & A_{1,2} \\
            A_{2,1} & A_{2,2} \\
        \end{bmatrix}
        \begin{bmatrix}
            B_{1,1} & B_{1,2} \\
            B_{2,1} & B_{2,2} \\
        \end{bmatrix}
        =
        \begin{bmatrix}
            A_{1,1} B_{1,1} + A_{1,2} B_{2,1}
             & A_{1,1} B_{1,2} + A_{1,2} B_{2,2} \\
            A_{2,1} B_{1,1} + A_{2,2} B_{2,1}
             & A_{2,1} B_{1,2} + A_{2,2} B_{2,2} \\
        \end{bmatrix}.
    \end{align*}
    于是, 形式地,
    分块地写阵的积与直接用定义写阵的积没有区别.

    % 特别地,
    % 若取 \(A_{1,2}\), \(A_{2,1}\), \(B_{1,2}\), \(B_{2,1}\)
    % 为零阵,
    % 有
    % \begin{align*}
    %     \begin{bmatrix}
    %         A_{1,1} & 0       \\
    %         0       & A_{2,2} \\
    %     \end{bmatrix}
    %     \begin{bmatrix}
    %         B_{1,1} & 0       \\
    %         0       & B_{2,2} \\
    %     \end{bmatrix}
    %     =
    %     \begin{bmatrix}
    %         A_{1,1} B_{1,1} & 0               \\
    %         0               & A_{2,2} B_{2,2} \\
    %     \end{bmatrix}.
    % \end{align*}
\end{theorem}

\begin{proof}
    注意到
    \begin{align*}
        [AB]_{i,j}
        = {} &
        \sum_{k=1}^{u_1+u_2}
        {[A]_{i,k} [B]_{k,j}}
        \\
        = {} &
        \sum_{k=1}^{u_1}
        {[A]_{i,k} [B]_{k,j}}
        +
        \sum_{k=u_1+1}^{u_1+u_2}
        {[A]_{i,k} [B]_{k,j}}.
    \end{align*}

    若 \(j \leq c_1\), 则
    \begin{align*}
        [AB]_{i,j}
        = {} &
        \sum_{k=1}^{u_1}
        {[A]_{i,k} [B]_{k,j}}
        +
        \sum_{k=u_1+1}^{u_1+u_2}
        {[A]_{i,k} [B]_{k,j}}
        \\
        = {} &
        \sum_{k=1}^{u_1}
        {[A]_{i,k} [B_{1,1}]_{k,j}}
        +
        \sum_{k=u_1+1}^{u_1+u_2}
        {[A]_{i,k} [B_{2,1}]_{k-u_1,j}}.
    \end{align*}
    当 \(i \leq r_1\) 时,
    \begin{align*}
        [AB]_{i,j}
        = {} &
        \sum_{k=1}^{u_1}
        {[A]_{i,k} [B_{1,1}]_{k,j}}
        +
        \sum_{k=u_1+1}^{u_1+u_2}
        {[A]_{i,k} [B_{2,1}]_{k-u_1,j}}
        \\
        = {} &
        \sum_{k=1}^{u_1}
        {[A_{1,1}]_{i,k} [B_{1,1}]_{k,j}}
        +
        \sum_{k=u_1+1}^{u_1+u_2}
        {[A_{1,2}]_{i,k-u_1} [B_{2,1}]_{k-u_1,j}}
        \\
        = {} &
        [A_{1,1} B_{1,1}]_{i,j}
        + [A_{1,2} B_{2,1}]_{i,j}
        \\
        = {} &
        [A_{1,1} B_{1,1} + A_{1,2} B_{2,1}]_{i,j}
        \\
        = {} &
        [C_{1,1}]_{i,j}
        \\
        = {} &
        [C]_{i,j}.
    \end{align*}
    当 \(i > r_1\) 时,
    \begin{align*}
        [AB]_{i,j}
        = {} &
        \sum_{k=1}^{u_1}
        {[A]_{i,k} [B_{1,1}]_{k,j}}
        +
        \sum_{k=u_1+1}^{u_1+u_2}
        {[A]_{i,k} [B_{2,1}]_{k-u_1,j}}
        \\
        = {} &
        \sum_{k=1}^{u_1}
        {[A_{2,1}]_{i-r_1,k} [B_{1,1}]_{k,j}}
        +
        \sum_{k=u_1+1}^{u_1+u_2}
        {[A_{2,2}]_{i-r_1,k-u_1} [B_{2,1}]_{k-u_1,j}}
        \\
        = {} &
        [A_{2,1} B_{1,1}]_{i-r_1,j}
        + [A_{2,2} B_{2,1}]_{i-r_1,j}
        \\
        = {} &
        [A_{2,1} B_{1,1} + A_{2,2} B_{2,1}]_{i-r_1,j}
        \\
        = {} &
        [C_{2,1}]_{i-r_1,j}
        \\
        = {} &
        [C]_{i,j}.
    \end{align*}

    若 \(j > c_1\), 则
    \begin{align*}
        [AB]_{i,j}
        = {} &
        \sum_{k=1}^{u_1}
        {[A]_{i,k} [B]_{k,j}}
        +
        \sum_{k=u_1+1}^{u_1+u_2}
        {[A]_{i,k} [B]_{k,j}}
        \\
        = {} &
        \sum_{k=1}^{u_1}
        {[A]_{i,k} [B_{1,2}]_{k,j-c_1}}
        +
        \sum_{k=u_1+1}^{u_1+u_2}
        {[A]_{i,k} [B_{2,2}]_{k-u_1,j-c_1}}.
    \end{align*}
    当 \(i \leq r_1\) 时,
    \begin{align*}
        [AB]_{i,j}
        = {} &
        \sum_{k=1}^{u_1}
        {[A]_{i,k} [B_{1,2}]_{k,j-c_1}}
        +
        \sum_{k=u_1+1}^{u_1+u_2}
        {[A]_{i,k} [B_{2,2}]_{k-u_1,j-c_1}}
        \\
        = {} &
        \sum_{k=1}^{u_1}
        {[A_{1,1}]_{i,k} [B_{1,2}]_{k,j-c_1}}
        +
        \sum_{k=u_1+1}^{u_1+u_2}
        {[A_{1,2}]_{i,k-u_1} [B_{2,2}]_{k-u_1,j-c_1}}
        \\
        = {} &
        [A_{1,1} B_{1,2}]_{i,j-c_1}
        + [A_{1,2} B_{2,2}]_{i,j-c_1}
        \\
        = {} &
        [A_{1,1} B_{1,2} + A_{1,2} B_{2,2}]_{i,j-c_1}
        \\
        = {} &
        [C_{1,2}]_{i,j-c_1}
        \\
        = {} &
        [C]_{i,j}.
    \end{align*}
    当 \(i > r_1\) 时,
    \begin{align*}
        [AB]_{i,j}
        = {} &
        \sum_{k=1}^{u_1}
        {[A]_{i,k} [B_{1,2}]_{k,j-c_1}}
        +
        \sum_{k=u_1+1}^{u_1+u_2}
        {[A]_{i,k} [B_{2,2}]_{k-u_1,j-c_1}}
        \\
        = {} &
        \sum_{k=1}^{u_1}
        {[A_{2,1}]_{i-r_1,k} [B_{1,2}]_{k,j-c_1}}
        +
        \sum_{k=u_1+1}^{u_1+u_2}
        {[A_{2,2}]_{i-r_1,k-u_1} [B_{2,2}]_{k-u_1,j-c_1}}
        \\
        = {} &
        [A_{2,1} B_{1,2}]_{i-r_1,j-c_1}
        + [A_{2,2} B_{2,2}]_{i-r_1,j-c_1}
        \\
        = {} &
        [A_{2,1} B_{1,2} + A_{2,2} B_{2,2}]_{i-r_1,j-c_1}
        \\
        = {} &
        [C_{2,2}]_{i-r_1,j-c_1}
        \\
        = {} &
        [C]_{i,j}.
        \qedhere
    \end{align*}
\end{proof}
