\chapter{后日谈}

我其实还有一些想说的话.
不过, 这些话较不适合在第一章讲;
当然, 也较不适合在附录~A 或附录~B 讲.
所以, 我就在这儿讲.

% \vfill
% \begin{ilustrajxo*}[h!]%
%     \includegraphics[height=12.5cm]{4}%
%     \centering%
% \end{ilustrajxo*}
% \vfill
\clearpage

\section{我要如何定义行列式?}

行列式是什么?
% 粗糙地,
我认为,
就像迹那样
(注: \(n\)~级阵 \(A\)~的\emph{迹}是
\([A]_{1,1} + [A]_{2,2} + \dots + [A]_{n,n}\)),
行列式也只是方阵一个属性而已.
不过, 这个看法, 多少有些不全面;
毕竟, 这可能会使人认为,
``行列式就是个 `新定义运算' 而已''.
行列式是有用的;
至少, 不说线性代数 (或高等代数),
它在微积分与几何里, 也是有用的.

出于多种原因, 我决定,
写一本行列式的入门教材.
既然是\emph{入门}教材,
它自然要简单一些.
我能想到至少三种 (差别较大的) 定义方式:

\begin{definition}[归纳定义]
    设 \(A\) 是 \(n\)~级阵 (\(n \geq 1\)).
    定义 \(A\) 的行列式
    \begin{align*}
        \det {(A)}
        =
        \begin{dcases}
            [A]_{1,1},
             & n = 1;    \\
            \sum_{i = 1}^{n}
            {(-1)^{i+1} [A]_{i,1} \det {(A(i|1))}},
             & n \geq 2.
        \end{dcases}
    \end{align*}
\end{definition}

\begin{definition}[组合定义]
    设 \(A\) 是 \(n\)~级阵 (\(n \geq 1\)).
    定义 \(A\) 的行列式
    \begin{align*}
        \det {(A)}
        = {} &
        \sum_{\substack{
        1 \leq i_1, i_2, \dots, i_n \leq n \\
                i_1, i_2, \dots, i_n\,\text{互不相同}
            }}
        {s(i_1, i_2, \dots, i_n)\,
            [A]_{i_1,1} [A]_{i_2,2} \dots [A]_{i_n,n}},
    \end{align*}
    其中
    \begin{align*}
        s(i_1, i_2, \dots, i_n)
        = \prod_{1 \leq p < q \leq n}
        {\operatorname{sgn} {(i_q - i_p)}}
    \end{align*}
    是文字列
    (或者, ``排列'',
    因为这里的 \(i_1\), \(i_2\), \(\dots\), \(i_n\)
    是互不相同的)
    \(i_1\), \(i_2\), \(\dots\), \(i_n\) 的符号.
\end{definition}

\begin{definition}[公理定义]
    设 \(f\) 是定义在全体 \(n\)~级阵上的函数.
    若 \(f\) 适合如下三条, 则说
    \(f\) 是 (\(n\)~级阵的) \emph{一个}行列式函数
    (自然地, 若 \(A\) 是 \(n\)~级阵,
    则 \(f(A)\) 是 \(A\) 的\emph{一个}行列式):

    (1)
    (规范性)
    若 \(I\) 是 \(n\)~级单位阵,
    则 \(f(I) = 1\).

    (2)
    (多线性)
    对任何不超过 \(n\) 的正整数 \(j\),
    任何 \(n-1\)~个 \(n \times 1\)~阵
    \(a_1\), \(\dots\), \(a_{j-1}\),
    \(a_{j+1}\), \(\dots\), \(a_n\),
    任何二个 \(n \times 1\)~阵 \(x\), \(y\),
    任何二个数 \(s\), \(t\),
    有
    \begin{align*}
             & f
            {([a_1, \dots, a_{j-1}, sx + ty,
                        a_{j+1}, \dots, a_n])}
        \\
        = {} &
        s
        f {([a_1, \dots, a_{j-1}, x, a_{j+1}, \dots, a_n])}
        +
        t
        f {([a_1, \dots, a_{j-1}, y, a_{j+1}, \dots, a_n])}.
    \end{align*}

    (3)
    (交错性)
    若 \(n\)~级阵 \(A\) 有二列完全相同,
    则 \(f {(A)} = 0\).
\end{definition}

值得注意的是, 这里的定义是关于列的.
我们知道, 一个阵的行列式等于其转置的行列式,
故阵的行与阵的列在行列式里的地位是一样的,
进而我们也可用关于行的版本定义行列式.
% (我就不单独列出了).
这, 我认为, 只是个人的喜好而已.
毕竟, 行或列不是本质的.
规范性、多线性、交错性是本质.

粗略地, 我想到的三种定义,
对初学行列式的人,
都是有一定挑战的,
因为它们都涉及了 ``非高中数学内容''.
归纳定义不好, 因为学生不一定熟悉数学归纳法.
当我是高中生时, 数学归纳法至少是必学的;
可是, 过了几年, 新教材里的数学归纳法是选学内容,
且新高考也不再考它.
组合定义不好, 因为学生不一定熟悉
(比数学归纳法抽象的) 排列或置换.
并且,
% 尴尬地,
在一些线性代数 (或高等代数) 教材里,
% 排列或置换的理论似乎只为行列式服务.
排列或置换的理论似乎只为行列式所用.
公理定义不好, 因为它有些抽象.
这要一定的准备.
据说, 老的中学数学有
``\(2\)~级行列式'' ``\(3\)~级行列式''
(即, \(2\)~级阵的行列式与 \(3\)~级阵的行列式);
可是, 我是高中生时, (必学的) 教材没有了行列式;
(必学的) 新教材自然也没有行列式.
我认为, 这么讲行列式, 会使更多的初学者不解
(若学生没有什么准备知识).

虽然这三个定义都对初学者有一定的挑战,
我还是选了归纳定义;
毕竟, 我想, 数学归纳法应是%
每一个 (学数学的) 人都了解的 (基础的) 原理.

于是, 我开始写本书的第一章.

\section{我要讲阵吗?}

理论地, 我可不用阵讲行列式.
具体地, 我可以这么定义行列式.

\begin{definition}
    我们叫下面用二条竖线围起来的%
    由 \(n\)~行 \(n\)~列元%
    作成的式为一个 \emph{\(n\)~级行列式}:
    \begin{equation}
        D =
        \begin{vmatrix}
            a_{1,1} & a_{1,2} & \cdots & a_{1,n} \\
            a_{2,1} & a_{2,2} & \cdots & a_{2,n} \\
            \vdots  & \vdots  & {}     & \vdots  \\
            a_{n,1} & a_{n,2} & \cdots & a_{n,n} \\
        \end{vmatrix}.
        \label{eq:C0201}
    \end{equation}
    它由 \(n\)~行 \(n\)~列, 共 \(n^2\)~个元作成.
    我们叫行~\(i\) 的 \(n\)~个元
    \(a_{i,1}\), \(a_{i,2}\), \(\dots\), \(a_{i,n}\)
    为行列式~\(D\) 的行~\(i\),
    叫列~\(j\) 的 \(n\)~个元
    \(a_{1,j}\), \(a_{2,j}\), \(\dots\), \(a_{n,j}\)
    为行列式~\(D\) 的列~\(j\).
    我们叫行~\(i\), 列~\(j\) 交点上的元 \(a_{i,j}\)
    为行列式~\(D\) 的 \((i,j)\)-元.

    我们定义 \(a_{i,j}\)~的\emph{余子式} \(M_{i,j}\)
    为由行列式~\(D\) 中%
    去除行~\(i\) 列~\(j\) 后%
    剩下的 \(n-1\)~行与 \(n-1\)~列元%
    作成的行列式:
    \begin{align*}
        M_{i,j} =
        \begin{vmatrix}
            a_{1,1}     & \cdots & a_{1,j-1}   &
            a_{1,j+1}   & \cdots & a_{1,n}       \\
            \vdots      & {}     & \vdots      &
            \vdots      & {}     & \vdots        \\
            a_{i-1,1}   & \cdots & a_{i-1,j-1} &
            a_{i-1,j+1} & \cdots & a_{i-1,n}     \\
            a_{i+1,1}   & \cdots & a_{i+1,j-1} &
            a_{i+1,j+1} & \cdots & a_{i+1,n}     \\
            \vdots      & {}     & \vdots      &
            \vdots      & {}     & \vdots        \\
            a_{n,1}     & \cdots & a_{n,j-1}   &
            a_{n,j+1}   & \cdots & a_{n,n}       \\
        \end{vmatrix}.
    \end{align*}

    当 \(n = 1\) 时, 定义式~\eqref{eq:C0201} 为
    \begin{equation}
        D = a_{1,1}.
        \label{eq:C0202}
    \end{equation}
    当 \(n \geq 2\) 时, 归纳地定义式~\eqref{eq:C0201} 为
    \begin{equation}
        \begin{aligned}
            D
            = {} &
            a_{1,1} M_{1,1} - a_{2,1} M_{2,1} + \dots +
            (-1)^{n+1} a_{n,1} M_{n,1}
            \\
            = {} &
            \sum_{i = 1}^{n} {(-1)^{i+1} a_{i,1} M_{i,1}}.
        \end{aligned}
        \label{eq:C0203}
    \end{equation}
\end{definition}

可以看到, 在这个定义里,
``行列式'' 至少有二种意思:
一是形如式~\eqref{eq:C0201} 的方形数表的式,
二是由式~\eqref{eq:C0202}, \eqref{eq:C0203}
定义的一个数.
既然式~\eqref{eq:C0201} 只是一个式,
又因为二个式相等是指它们的结果相等,
故, 有着不一样的元的二个 \(n\)~级行列式可能相等.

我如何定义行列式?
我先定义阵 (矩形数表),
再定义方阵 (方形数表) 的行列式%
是施一定的规则于方阵得到的数.
(具体地, 您看第一章, 节~\sekcio{5}, \sekcio{6} 即知.)

由此可见, 这个定义跟第一章的定义,
在思想上, 是有一些区别的.
我用的定义视行列式为方阵的一个属性,
而这个定义,
% 单纯地,
是一个
``形如式~\eqref{eq:C0201} 的方形数表的式'',
或
``由式~\eqref{eq:C0202}, \eqref{eq:C0203} 确定的数''.

历史地, 行列式比阵早出现.
所以, 这种不涉及阵的定义, 是较古典的.
逻辑地, 它没有什么问题.
不过, 教学地, 有的学生区分行列式与阵是有些挑战的.
毕竟,
% 二者相貌类似,
二者长得差不多,
且阵 (的运算) 与行列式对%
线性代数 (或高等代数) 的初学者来说%
都是有一定挑战的.

我想到的解决此问题的方法就是先讲阵
(至少, 一定要先讲阵的记号),
再讲行列式.
这样, 初学者更能体会,
行列式是方阵的一个属性.
至少, 我不想在入门课给学生较多挑战.

类似地, 历史地, 对数比指数早出现.
可是, 我们在高中学数学时,
也并没有先讲对数, 再讲指数.
相反, 教材用指数讲对数.

\vspace{2ex}

最后, 我以一个较形象的例结束本节.

我在前面说过,
\(n\)~级阵 \(A\)~的迹是
\([A]_{1,1} + [A]_{2,2} + \dots + [A]_{n,n}\).
能否不用阵定义迹?
我想, 理论地, 当然可以.
我试作了一个定义.
您看看它如何吧.

\begin{definition}
    我们叫下面用括号 \(\{ \ \}\)
    (注意, 这只是我自己选的一个记号)
    围起来的%
    由 \(n\)~行 \(n\)~列元%
    作成的式为一个 \emph{\(n\)~级迹}:
    \begin{equation}
        T =
        \begin{Bmatrix}
            a_{1,1} & a_{1,2} & \cdots & a_{1,n} \\
            a_{2,1} & a_{2,2} & \cdots & a_{2,n} \\
            \vdots  & \vdots  & {}     & \vdots  \\
            a_{n,1} & a_{n,2} & \cdots & a_{n,n} \\
        \end{Bmatrix}.
        \label{eq:C0204}
    \end{equation}
    它由 \(n\)~行 \(n\)~列, 共 \(n^2\)~个元作成.
    我们叫行~\(i\) 的 \(n\)~个元
    \(a_{i,1}\), \(a_{i,2}\), \(\dots\), \(a_{i,n}\)
    为迹~\(T\) 的行~\(i\),
    叫列~\(j\) 的 \(n\)~个元
    \(a_{1,j}\), \(a_{2,j}\), \(\dots\), \(a_{n,j}\)
    为迹~\(T\) 的列~\(j\).
    我们叫行~\(i\), 列~\(j\) 交点上的元 \(a_{i,j}\)
    为迹~\(T\) 的 \((i,j)\)-元.

    我们定义式~\eqref{eq:C0204} 为
    \begin{align*}
        T = a_{1,1} + a_{2,2} + \dots + a_{n,n}.
    \end{align*}
\end{definition}

% 这是一个开放的问题;
% 或者, 用《周易》的话,
% ``仁者见之谓之仁, 智者见之谓之智''.
% 这是一个
% ``仁者见之谓之仁, 智者见之谓之智''
% 的问题;
这是一个个人喜好问题.
不同的人, 会有不同的想法.


\section{行列式的性质}
% \section{我要介绍行列式的哪些性质?}

% 这是一个好问题.

% 我初学行列式时, 教材讲了好几条性质.
% 我姑且用第一章的话译它们
% 这里,
设 \(a_1\), \(a_2\), \(\dots\), \(a_n\), \(x\), \(y\)
是 \(n+2\)~个 \(n \times 1\)~阵.
设 \(A = [a_1, a_2, \dots, a_n]\).
设 \(s\) 是一个数.
行列式有如下性质:

\vspace{2ex}

(1)
一个方阵与其转置的行列式相等.
于是, 我不必同时讲行的性质与列的性质,
因为您总可用转置,
译列的性质为行的性质
(或译行的性质为列的性质).

(2)
若一个阵的某一列是二组数的和,
那么此阵的行列式等于二个阵的行列式的和,
而这二个阵, 除这一列以外, 全与原阵的对应的列一样.
用公式写, 就是
\begin{align*}
         & \det {[\dots, a_{j-1}, x + y, a_{j+1}, \dots]}
    \\
    = {} & \det {[\dots, a_{j-1}, x, a_{j+1}, \dots]}
    + \det {[\dots, a_{j-1}, y, a_{j+1}, \dots]},
\end{align*}
其中, 未写的列是不变的, 下同.

(3)
以一个数乘阵的一列后得到的阵的行列式%
等于以此数乘原阵的行列式.
用公式写, 就是
\begin{align*}
    \det {[\dots, a_{j-1}, sx, a_{j+1}, \dots]}
    = s \det {[\dots, a_{j-1}, x, a_{j+1}, \dots]}.
\end{align*}

(4)
若一个阵有二列相同, 则其行列式为零.

(5)
若交换一个阵的二列, 则其行列式变号.

(6)
若一个阵有二列成比例, 则其行列式为零.
(利用性质~(3) (4) 即知.)

(7)
若加一个阵的一列的倍于另一列, 则其行列式不变.
(利用性质~(2) (6) 即知.)

(8)
单位阵的行列式是 \(1\).

(9)
设 \(j\) 是不超过 \(n\) 的正整数.
则
\begin{align*}
    \det {(A)} = \sum_{i = 1}^{n}
    {(-1)^{i+j} [A]_{i,j} \det {(A(i|j))}}.
\end{align*}

\vspace{2ex}

不难看出, (2) (3) 的联合是多线性,
且 (4) 就是交错性.

% 我在第一章讲了行列式的 5~个性质.
% 用行列式与转置的关系可译行列式的关于阵的列的命题%
% 为其关于行的命题.
% 规范性指出, 单位阵的行列式是 \(1\);
% 这是前面 7~条性质未指出的.
% 多线性就是 (2) (3) 的联合.
% 交错性就是 (4).
% 反称性就是 (5).
% 我没介绍 (6) (7); 它没介绍规范性:
% 我们二个都有好的未来.
% (这句话里包含三个分句,
% 前面二个是并列的,
% 末了儿一个是总起来说的,
% 就得在 ``规范性'' 之后用冒号,
% 表示并列的二个分句已经完了,
% 还有总起来说的一个分句在后面.
% 这同样是提示的意思.
% 如此用冒号, 有些人还没有养成习惯.
% 可是, 这个用法可明白地表示分句与分句的关系,
% 应学会它.)

这些性质是有用的.
我们看几个例.

\begin{example}
    设 \(n\)~级阵 \(A\) 适合:
    当 \(1 \leq j < i \leq n\) 时,
    \([A]_{i,j} = 0\).
    形象地,
    \begin{align*}
        A =
        \begin{bmatrix}
            [A]_{1,1} & [A]_{1,2} & \cdots & [A]_{1,n-1}   & [A]_{1,n}   \\
            0         & [A]_{2,2} & \cdots & [A]_{2,n-1}   & [A]_{2,n}   \\
            \vdots    & \vdots    & \ddots & \vdots        & \vdots      \\
            0         & 0         & \cdots & [A]_{n-1,n-1} & [A]_{n-1,n} \\
            0         & 0         & \cdots & 0             & [A]_{n,n}   \\
        \end{bmatrix}.
    \end{align*}
    我们计算 \(\det {(A)}\).

    按列~\(1\) 展开, 有
    \begin{align*}
        \det {(A)}
        = {} &
        \sum_{i=1}^{n}
        {(-1)^{i+1} [A]_{i,1} \det {(A(i|1))}}
        \\
        = {} &
        (-1)^{1+1} [A]_{1,1} \det {(A(1|1))}
        + \sum_{i=2}^{n}
        {(-1)^{i+1} [A]_{i,1} \det {(A(i|1))}}
        \\
        = {} &
        [A]_{1,1} \det {(A(1|1))}
        + \sum_{i=2}^{n}
        {(-1)^{i+1}\, 0 \det {(A(i|1))}}
        \\
        = {} &
        [A]_{1,1} \det {(A(1|1))}.
    \end{align*}

    不难看出, 当 \(1 \leq j < i \leq n-1\) 时,
    也有 \([A(1|1)]_{i,j} = 0\).
    于是, 类似地,
    \begin{align*}
        \det {(A(1|1))}
        = [A(1|1)]_{1,1} \det {((A(1|1))(1|1))}
            = [A]_{2,2} \det {(A({1,2}|{1,2}))}.
    \end{align*}
    故
    \begin{align*}
        \det {(A)}
        = [A]_{1,1} [A]_{2,2} \det {(A({1,2}|{1,2}))}.
    \end{align*}

    \(\dots \dots\)

    最后, 我们得到
    \begin{align*}
        \det {(A)}
        = {} &
        [A]_{1,1} [A]_{2,2} \dots [A]_{n-1,n-1}
        \det {(A({1,2,\dots,n-1}|{1,2,\dots,n-1}))}
        \\
        = {} &
        [A]_{1,1} [A]_{2,2} \dots [A]_{n-1,n-1} [A]_{n,n}
        \\
        = {} &
        [A]_{1,1} [A]_{2,2} \dots [A]_{n,n}.
    \end{align*}
\end{example}

\begin{example}
    设 \(n\)~级阵 \(A\) 适合:
    当 \(1 \leq i < j \leq n\) 时,
    \([A]_{i,j} = 0\).
    形象地,
    \begin{align*}
        A =
        \begin{bmatrix}
            [A]_{1,1}   & 0           & \cdots & 0             & 0         \\
            [A]_{2,1}   & [A]_{2,2}   & \cdots & 0             & 0         \\
            \vdots      & \vdots      & \ddots & \vdots        & \vdots    \\
            [A]_{n-1,1} & [A]_{n-1,2} & \cdots & [A]_{n-1,n-1} & 0         \\
            [A]_{n,1}   & [A]_{n,2}   & \cdots & [A]_{n,n-1}   & [A]_{n,n} \\
        \end{bmatrix}.
    \end{align*}
    我们计算 \(\det {(A)}\).

    不难看出, \(A\) 的转置 \(A^{\mathrm{T}}\) 适合:
    当 \(1 \leq j < i \leq n\) 时,
    \([A^{\mathrm{T}}]_{i,j} = [A]_{j,i} = 0\).
    由性质~(1) 与上例的结果,
    \begin{align*}
        \det {(A)}
        = {} &
        \det {(A^{\mathrm{T}})}
        \\
        = {} &
        [A^{\mathrm{T}}]_{1,1} [A^{\mathrm{T}}]_{2,2}
        \dots [A^{\mathrm{T}}]_{n,n}
        \\
        = {} &
        [A]_{1,1} [A]_{2,2} \dots [A]_{n,n}.
    \end{align*}
\end{example}

\begin{example}
    运用性质~(7), 可作出一些零, 简化计算.
    比如,
    设
    \begin{align*}
        A = \begin{bmatrix}
                4 & 9 & 2 \\
                3 & 5 & 7 \\
                8 & 1 & 6 \\
            \end{bmatrix}.
    \end{align*}
    则
    \begin{align*}
        \det {(A)}
        = {} &
        \det {
            \begin{bmatrix}
                4 & 9 & 2 + 9 \\
                3 & 5 & 7 + 5 \\
                8 & 1 & 6 + 1 \\
            \end{bmatrix}
        }
        \\
        = {} &
        \det {
            \begin{bmatrix}
                4 & 9 & 2 + 9 + 4 \\
                3 & 5 & 7 + 5 + 3 \\
                8 & 1 & 6 + 1 + 8 \\
            \end{bmatrix}
        }
        \\
        = {} &
        \det {
            \begin{bmatrix}
                4 & 9 & 15 \\
                3 & 5 & 15 \\
                8 & 1 & 15 \\
            \end{bmatrix}
        }
        \\
        = {} &
        \det {
            \begin{bmatrix}
                4 & 9 - 15 \cdot \frac{1}{15} & 15 \\
                3 & 5 - 15 \cdot \frac{1}{15} & 15 \\
                8 & 1 - 15 \cdot \frac{1}{15} & 15 \\
            \end{bmatrix}
        }
        \\
        = {} &
        \det {
            \begin{bmatrix}
                4 - 15 \cdot \frac{8}{15} & 8 & 15 \\
                3 - 15 \cdot \frac{8}{15} & 4 & 15 \\
                8 - 15 \cdot \frac{8}{15} & 0 & 15 \\
            \end{bmatrix}
        }
        \\
        = {} &
        \det {
            \begin{bmatrix}
                -4 & 8 & 15 \\
                -5 & 4 & 15 \\
                0  & 0 & 15 \\
            \end{bmatrix}
        }
        \\
        = {} &
        \det {
            \begin{bmatrix}
                -4 + 8 \cdot \frac{5}{4} & 8 & 15 \\
                -5 + 4 \cdot \frac{5}{4} & 4 & 15 \\
                0 + 0 \cdot \frac{5}{4}  & 0 & 15 \\
            \end{bmatrix}
        }
        \\
        = {} &
        \det {
            \begin{bmatrix}
                6 & 8 & 15 \\
                0 & 4 & 15 \\
                0 & 0 & 15 \\
            \end{bmatrix}
        }
        \\
        = {} &
        6 \cdot 4 \cdot 15
        \\
        = {} &
        360.
    \end{align*}
    我们用 \(3\)~级阵的行列式的公式验证结果:
    \begin{align*}
        \det {(A)}
        = {} &
        4 \cdot 5 \cdot 6
        + 3 \cdot 1 \cdot 2
        + 8 \cdot 9 \cdot 7
        - 4 \cdot 1 \cdot 7
        - 3 \cdot 9 \cdot 6
        - 8 \cdot 5 \cdot 2
        \\
        = {} &
        120 + 6 + 504 - 28 - 162 - 80
        \\
        = {} & 360.
    \end{align*}
\end{example}

\begin{example}
    设
    \begin{align*}
        A = \begin{bmatrix}
                16 & 3  & 2  & 13 \\
                5  & 10 & 11 & 8  \\
                9  & 6  & 7  & 12 \\
                4  & 15 & 14 & 1  \\
            \end{bmatrix}.
    \end{align*}
    则
    \begin{align*}
        \det {(A)}
        = {} &
        \det {
            \begin{bmatrix}
                16 & 3 - 2   & 2  & 13 \\
                5  & 10 - 11 & 11 & 8  \\
                9  & 6 - 7   & 7  & 12 \\
                4  & 15 - 14 & 14 & 1  \\
            \end{bmatrix}
        }
        \\
        = {} &
        \det {
            \begin{bmatrix}
                16 - 13 & 3 - 2   & 2  & 13 \\
                5 - 8   & 10 - 11 & 11 & 8  \\
                9 - 12  & 6 - 7   & 7  & 12 \\
                4 - 1   & 15 - 14 & 14 & 1  \\
            \end{bmatrix}
        }
        \\
        = {} &
        \det {
            \begin{bmatrix}
                3  & 1  & 2  & 13 \\
                -3 & -1 & 11 & 8  \\
                -3 & -1 & 7  & 12 \\
                3  & 1  & 14 & 1  \\
            \end{bmatrix}
        }.
    \end{align*}
    利用性质~(6) 可知, 最后一个阵的行列式为 \(0\).
    所以, \(A\)~的行列式也是 \(0\).

    我们当然也可用
    \(4\)~级阵的行列式的公式验证结果.
    不过, 这较复杂.
    首先, 根据定义,
    \begin{align*}
             & \det {(A)} \\
        = {} &
        \hphantom{{} + {}}
        (-1)^{1+1} [A]_{1,1} \det {(A(1|1))}
        + (-1)^{2+1} [A]_{2,1} \det {(A(2|1))}
        \\
             &
        + (-1)^{3+1} [A]_{3,1} \det {(A(3|1))}
        + (-1)^{4+1} [A]_{4,1} \det {(A(4|1))}.
    \end{align*}
    可算出
    \begin{align*}
         & \det {(A(1|1))}
        = \det {\begin{bmatrix}
                        10 & 11 & 8  \\
                        6  & 7  & 12 \\
                        15 & 14 & 1  \\
                    \end{bmatrix}}
        = 136,             \\
         & \det {(A(2|1))}
        = \det {\begin{bmatrix}
                        3  & 2  & 13 \\
                        6  & 7  & 12 \\
                        15 & 14 & 1  \\
                    \end{bmatrix}}
        = -408,            \\
         & \det {(A(3|1))}
        = \det {\begin{bmatrix}
                        3  & 2  & 13 \\
                        10 & 11 & 8  \\
                        15 & 14 & 1  \\
                    \end{bmatrix}}
        = -408,            \\
         & \det {(A(4|1))}
        = \det {\begin{bmatrix}
                        3  & 2  & 13 \\
                        10 & 11 & 8  \\
                        6  & 7  & 12 \\
                    \end{bmatrix}}
        = 136.
    \end{align*}
    所以
    \begin{align*}
        \det {(A)}
        = 16 \cdot 136 - 5 \cdot (-408)
        + 9 \cdot (-408) - 4 \cdot 136
        = 0.
    \end{align*}
\end{example}

\begin{example}
    设 \(x\), \(y\) 是数.
    作 \(n\)~级阵 \(A\) 如下:
    \begin{align*}
        [A]_{i,j} =
        \begin{cases}
            x, & i = j;    \\
            y, & i \neq j.
        \end{cases}
    \end{align*}
    形象地,
    \begin{align*}
        A =
        \begin{bmatrix}
            x      & y      & \cdots & y      & y      \\
            y      & x      & \cdots & y      & y      \\
            \vdots & \vdots & {}     & \vdots & \vdots \\
            y      & y      & \cdots & x      & y      \\
            y      & y      & \cdots & y      & x      \\
        \end{bmatrix}.
    \end{align*}
    我们计算 \(\det {(A)}\).

    我们加 \(A\)~的%
    列~\(1\), \(2\), \(\dots\), \(n-1\) 于列~\(n\),
    得 \(n\)~级阵
    \begin{align*}
        A_1 =
        \begin{bmatrix}
            x      & y      & \cdots & y      & x + (n-1)y \\
            y      & x      & \cdots & y      & x + (n-1)y \\
            \vdots & \vdots & {}     & \vdots & \vdots     \\
            y      & y      & \cdots & x      & x + (n-1)y \\
            y      & y      & \cdots & y      & x + (n-1)y \\
        \end{bmatrix}.
    \end{align*}
    注意到, \(A_1\) 的前 \(n-1\)~列%
    与 \(A\) 的前 \(n-1\)~列一样,
    但 \(A_1\) 的列~\(n\) 的元全是 \(x + (n-1)y\).
    用 \(n-1\)~次性质~(7), 有
    \(\det {(A)} = \det {(A_1)}\).

    作 \(n\)~级阵
    \begin{align*}
        A_2 =
        \begin{bmatrix}
            x      & y      & \cdots & y      & 1      \\
            y      & x      & \cdots & y      & 1      \\
            \vdots & \vdots & {}     & \vdots & \vdots \\
            y      & y      & \cdots & x      & 1      \\
            y      & y      & \cdots & y      & 1      \\
        \end{bmatrix}.
    \end{align*}
    注意到, \(A_2\) 的前 \(n-1\)~列%
    与 \(A_1\) 的前 \(n-1\)~列一样,
    但 \(A_2\) 的列~\(n\) 的元全是 \(1\).
    用性质~(3), 有
    \(\det {(A)} = \det {(A_1)}
    = (x + (n-1)y) \det {(A_2)}\).

    我们加 \(A_2\)~的%
    列~\(n\) 的 \(-y\)~倍于列~\(1\),
    列~\(n\) 的 \(-y\)~倍于列~\(2\),
    \(\dots \dots\),
    列~\(n\) 的 \(-y\)~倍于列~\(n-1\),
    得 \(n\)~级阵
    \begin{align*}
        A_3 =
        \begin{bmatrix}
            x-y    & 0      & \cdots & 0      & 1      \\
            0      & x-y    & \cdots & 0      & 1      \\
            \vdots & \vdots & {}     & \vdots & \vdots \\
            0      & 0      & \cdots & x-y    & 1      \\
            0      & 0      & \cdots & 0      & 1      \\
        \end{bmatrix}.
    \end{align*}
    用 \(n-1\)~次性质~(7), 有
    \(\det {(A)}
    = (x + (n-1)y) \det {(A_2)}
    = (x + (n-1)y) \det {(A_3)}\).
    注意到,
    \(1 \leq j < i \leq n\) 时,
    有 \([A_3]_{i,j} = 0\).
    故
    \begin{align*}
        \det {(A_3)} = (x-y)^{n-1} \cdot 1 = (x-y)^{n-1}.
    \end{align*}
    故
    \begin{align*}
        \det {(A)}
        = (x + (n-1)y) \det {(A_3)}
        = (x + (n-1)y) (x-y)^{n-1}.
    \end{align*}
\end{example}

\begin{example}
    设 \(x_1\), \(x_2\), \(\dots\), \(x_n\) 是数.
    作 \(n\)~级阵 \(V(x_1, x_2, \dots, x_n)\) 如下:
    \begin{align*}
        [V(x_1, x_2, \dots, x_n)]_{i,j}
        = x_i^{j-1}.
    \end{align*}
    形象地,
    \begin{align*}
        V(x_1, x_2, \dots, x_n)
        =
        \begin{bmatrix}
            1      & x_1     & x_1^2     & \cdots & x_1^{n-2}     & x_1^{n-1}     \\
            1      & x_2     & x_2^2     & \cdots & x_2^{n-2}     & x_2^{n-1}     \\
            1      & x_3     & x_3^2     & \cdots & x_3^{n-2}     & x_3^{n-1}     \\
            \vdots & \vdots  & \vdots    & {}     & \vdots        & \vdots        \\
            1      & x_{n-1} & x_{n-1}^2 & \cdots & x_{n-1}^{n-2} & x_{n-1}^{n-1} \\
            1      & x_n     & x_n^2     & \cdots & x_n^{n-2}     & x_n^{n-1}     \\
        \end{bmatrix}.
    \end{align*}
    我们计算 \(\det {(V(x_1, x_2, \dots, x_n))}\).

    我们加 \(V(x_1, x_2, \dots, x_n)\)~的%
    列~\(n-1\) 的 \(-x_n\) 倍于列~\(n\),
    列~\(n-2\) 的 \(-x_n\) 倍于列~\(n-1\),
    \(\dots \dots\),
    列~\(1\) 的 \(-x_n\) 倍于列~\(2\),
    得 \(n\)~级阵
    \begin{align*}
             & A
        \\
        = {} &
        \begin{bmatrix}
            1      & x_1 - x_n     & x_1^2 - x_1 x_n         & \cdots & x_1^{n-2} - x_1^{n-3} x_n         & x_1^{n-1} - x_1^{n-2} x_n         \\
            1      & x_2 - x_n     & x_2^2 - x_2 x_n         & \cdots & x_2^{n-2} - x_2^{n-3} x_n         & x_2^{n-1} - x_2^{n-2} x_n         \\
            1      & x_3 - x_n     & x_3^2 - x_3 x_n         & \cdots & x_3^{n-2} - x_3^{n-3} x_n         & x_3^{n-1} - x_3^{n-2} x_n         \\
            \vdots & \vdots        & \vdots                  & {}     & \vdots                            & \vdots                            \\
            1      & x_{n-1} - x_n & x_{n-1}^2 - x_{n-1} x_n & \cdots & x_{n-1}^{n-2} - x_{n-1}^{n-3} x_n & x_{n-1}^{n-1} - x_{n-1}^{n-2} x_n \\
            1      & 0             & 0                       & \cdots & 0                                 & 0                                 \\
        \end{bmatrix}
        \\
        = {} &
        \begin{bmatrix}
            1      & x_1 - x_n     & x_1 (x_1 - x_n)         & \cdots & x_1^{n-3} (x_1 - x_n)         & x_1^{n-2} (x_1 - x_n)         \\
            1      & x_2 - x_n     & x_2 (x_2 - x_n)         & \cdots & x_2^{n-3} (x_2 - x_n)         & x_2^{n-2} (x_2 - x_n)         \\
            1      & x_3 - x_n     & x_3 (x_3 - x_n)         & \cdots & x_3^{n-3} (x_3 - x_n)         & x_3^{n-2} (x_3 - x_n)         \\
            \vdots & \vdots        & \vdots                  & {}     & \vdots                        & \vdots                        \\
            1      & x_{n-1} - x_n & x_{n-1} (x_{n-1} - x_n) & \cdots & x_{n-1}^{n-3} (x_{n-1} - x_n) & x_{n-1}^{n-2} (x_{n-1} - x_n) \\
            1      & 0             & 0                       & \cdots & 0                             & 0                             \\
        \end{bmatrix};
    \end{align*}
    具体地,
    \begin{align*}
        [A]_{i,j}
        =
        \begin{cases}
            1,
             & j = 1; \\
            x_i^{j-1} - x_i^{j-2} x_n
            = x_i^{j-2} (x_i - x_n),
             & j > 2.
        \end{cases}
    \end{align*}
    由性质~(7), 有
    \begin{align*}
        \det {(V(x_1, x_2, \dots, x_n))} = \det {(A)}.
    \end{align*}

    按行~\(n\) 展开, 有
    \begin{align*}
        \det {(V(x_1, x_2, \dots, x_n))}
        = \det {(A)}
        = (-1)^{n+1} \det {(B)},
    \end{align*}
    其中 \(B = A(n|1)\),
    且 \([B]_{i,j} = x_i^{j-1} (x_i - x_n)\);
    形象地,
    \begin{align*}
        B =
        \begin{bmatrix}
            x_1 - x_n     & x_1 (x_1 - x_n)         & \cdots & x_1^{n-3} (x_1 - x_n)         & x_1^{n-2} (x_1 - x_n)         \\
            x_2 - x_n     & x_2 (x_2 - x_n)         & \cdots & x_2^{n-3} (x_2 - x_n)         & x_2^{n-2} (x_2 - x_n)         \\
            x_3 - x_n     & x_3 (x_3 - x_n)         & \cdots & x_3^{n-3} (x_3 - x_n)         & x_3^{n-2} (x_3 - x_n)         \\
            \vdots        & \vdots                  & {}     & \vdots                        & \vdots                        \\
            x_{n-1} - x_n & x_{n-1} (x_{n-1} - x_n) & \cdots & x_{n-1}^{n-3} (x_{n-1} - x_n) & x_{n-1}^{n-2} (x_{n-1} - x_n) \\
        \end{bmatrix}.
    \end{align*}

    用 \(n-1\)~次性质~(3) (关于行), 有
    \begin{align*}
        \det {(B)}
        = {} &
        (x_1 - x_n) (x_2- x_n) \dots (x_{n-1} - x_n)
        \det {(V(x_1, x_2, \dots, x_{n-1}))}
        \\
        = {} &
        (-1)^{n-1} (x_n - x_1) (x_n - x_2) \dots (x_n - x_{n-1})
        \det {(V(x_1, x_2, \dots, x_{n-1}))}.
    \end{align*}
    故
    \begin{align*}
             &
        \det {(V(x_1, x_2, \dots, x_n))}
        \\
        = {} &
        (-1)^{n+1} \det {(B)}
        \\
        = {} &
        (-1)^{n+1}
        (-1)^{n-1} (x_n - x_1) (x_n - x_2) \dots (x_n - x_{n-1})
        \det {(V(x_1, x_2, \dots, x_{n-1}))}
        \\
        = {} &
        (x_n - x_1) (x_n - x_2) \dots (x_n - x_{n-1})
        \det {(V(x_1, x_2, \dots, x_{n-1}))}.
    \end{align*}

    类似地,
    \begin{align*}
             &
        \det {(V(x_1, x_2, \dots, x_{n-1}))}
        \\
        = {} &
        (x_{n-1} - x_1) (x_{n-1} - x_2) \dots (x_{n-1} - x_{n-2})
        \det {(V(x_1, x_2, \dots, x_{n-2}))}.
    \end{align*}
    故
    \begin{align*}
        \det {(V(x_1, x_2, \dots, x_n))}
        = {} & \hphantom{\cdot\,\,}
        (x_n - x_1) (x_n - x_2) \dots (x_n - x_{n-1})
        \\
             & \cdot
        (x_{n-1} - x_1) (x_{n-1} - x_2) \dots (x_{n-1} - x_{n-2})
        \\
             & \cdot
        \det {(V(x_1, x_2, \dots, x_{n-2}))}.
    \end{align*}

    \(\dots \dots\)

    最后, 我们有
    \begin{align*}
        \det {(V(x_1, x_2, \dots, x_n))}
        = {} & \hphantom{\cdot\,\,}
        (x_n - x_1) (x_n - x_2) \dots (x_n - x_{n-1})
        \\
             & \cdot
        (x_{n-1} - x_1) (x_{n-1} - x_2) \dots (x_{n-1} - x_{n-2})
        \\
             & \cdot
        \dots \dots \dots \dots
        \dots \dots \dots \dots
        \\
             & \cdot
        (x_3 - x_1) (x_3 - x_2)
        \\
             & \cdot
        (x_2 - x_1)
        \\
             & \cdot
        \det {(V(x_1))}
        \\
        = {} & \hphantom{\cdot\,\,}
        (x_n - x_1) (x_n - x_2) \dots (x_n - x_{n-1})
        \\
             & \cdot
        (x_{n-1} - x_1) (x_{n-1} - x_2) \dots (x_{n-1} - x_{n-2})
        \\
             & \cdot
        \dots \dots \dots \dots
        \dots \dots \dots \dots
        \\
             & \cdot
        (x_3 - x_1) (x_3 - x_2)
        \\
             & \cdot
        (x_2 - x_1)
        \\
             & \cdot
        1
        \\
        = {} &
        \prod_{2 \leq v \leq n}
        {
            \prod_{1 \leq u \leq v-1} {(x_v - x_u)}
        }
        \\
        = {} &
        \prod_{1 \leq u < v \leq n} {(x_v - x_u)}.
    \end{align*}

    不难看出,
    若 \(V(x_1, x_2, \dots, x_n)\) 的行列式非零,
    则 \(x_1\), \(x_2\), \(\dots\), \(x_n\) 互不相同;
    反过来,
    若 \(x_1\), \(x_2\), \(\dots\), \(x_n\) 互不相同,
    则 \(V(x_1, x_2, \dots, x_n)\) 的行列式非零.
\end{example}

\begin{example}
    设 \(a_1\), \(a_2\), \(\dots\), \(a_n\) 是 \(n\)~个数.
    作 \(n\)~级阵 \(C(a_1, a_2, \dots, a_n)\) 如下:
    \begin{align*}
        [C(a_1, a_2, \dots, a_n)]_{i,j}
        = \begin{cases}
              -a_{n-i+1}, & j = n;                   \\
              1,          & 2 \leq i = j + 1 \leq n; \\
              0,          & \text{其他}.
          \end{cases}
    \end{align*}
    形象地,
    \begin{align*}
        C(a_1, a_2, \dots, a_n)
        = \begin{bmatrix}
              0      & 0      & \cdots & 0      & 0      & -a_n     \\
              1      & 0      & \cdots & 0      & 0      & -a_{n-1} \\
              0      & 1      & \cdots & 0      & 0      & -a_{n-2} \\
              \vdots & \vdots & {}     & \vdots & \vdots & \vdots   \\
              0      & 0      & \cdots & 1      & 0      & -a_2     \\
              0      & 0      & \cdots & 0      & 1      & -a_1     \\
          \end{bmatrix}
    \end{align*}.
    设 \(x\) 是一个数.
    记 \(P(x; a_1, a_2, \dots, a_n) = xI_n - C(a_1, a_2, \dots, a_n)\).
    形象地,
    \begin{align*}
        P(x; a_1, a_2, \dots, a_n)
        = \begin{bmatrix}
              x      & 0      & \cdots & 0      & 0      & a_n     \\
              -1     & x      & \cdots & 0      & 0      & a_{n-1} \\
              0      & -1     & \cdots & 0      & 0      & a_{n-2} \\
              \vdots & \vdots & {}     & \vdots & \vdots & \vdots  \\
              0      & 0      & \cdots & -1     & x      & a_2     \\
              0      & 0      & \cdots & 0      & -1     & x + a_1 \\
          \end{bmatrix}.
    \end{align*}
    我们计算 \(\det {(P(x; a_1, a_2, \dots, a_n))}\).

    注意到, 当 \(1 < j < n\) 时,
    \([P(x; a_1, a_2, \dots, a_n)]_{1,j} = 0\).
    于是, 按行~\(1\) 展开, 有
    \begin{align*}
             &
        \det {(P(x; a_1, a_2, \dots, a_n))}
        \\
        = {} &
        \hphantom{{} + {}}
        (-1)^{1+1} [P(x; a_1, a_2, \dots, a_n)]_{1,1}
        \det {((P(x; a_1, a_2, \dots, a_n))(1|1))}
        \\
             &
        + (-1)^{1+n} [P(x; a_1, a_2, \dots, a_n)]_{1,n}
        \det {((P(x; a_1, a_2, \dots, a_n))(1|n))}.
    \end{align*}
    不难写出,
    \begin{align*}
        [P(x; a_1, a_2, \dots, a_n)]_{1,1} = x,
        \quad
        [P(x; a_1, a_2, \dots, a_n)]_{1,n} = a_n.
    \end{align*}
    不难写出,
    \begin{align*}
        (P(x; a_1, a_2, \dots, a_n))(1|1)
        = {} &
        \begin{bmatrix}
            x      & \cdots & 0      & 0      & a_{n-1} \\
            -1     & \cdots & 0      & 0      & a_{n-2} \\
            \vdots & {}     & \vdots & \vdots & \vdots  \\
            0      & \cdots & -1     & x      & a_2     \\
            0      & \cdots & 0      & -1     & x + a_1 \\
        \end{bmatrix}
        \\
        = {} &
        P(x; a_1, a_2, \dots, a_{n-1}).
    \end{align*}
    故
    \begin{align*}
        \det {((P(x; a_1, a_2, \dots, a_n))(1|1))}
        = \det {(P(x; a_1, a_2, \dots, a_{n-1}))}.
    \end{align*}
    不难写出,
    \begin{align*}
        (P(x; a_1, a_2, \dots, a_n))(1|n)
        = \begin{bmatrix}
              -1     & x      & \cdots & 0      & 0      \\
              0      & -1     & \cdots & 0      & 0      \\
              \vdots & \vdots & {}     & \vdots & \vdots \\
              0      & 0      & \cdots & -1     & x      \\
              0      & 0      & \cdots & 0      & -1     \\
          \end{bmatrix};
    \end{align*}
    具体地,
    \begin{align*}
        [(P(x; a_1, a_2, \dots, a_n))(1|n)]_{i,j}
        = \begin{cases}
              -1, & 1 \leq i = j \leq n-1;     \\
              x,  & 1 \leq i = j - 1 \leq n-2; \\
              0,  & \text{其他}.
          \end{cases}
    \end{align*}
    故 \(1 \leq j < i \leq n-1\) 时, 有
    \([(P(x; a_1, a_2, \dots, a_n))(1|n)]_{i,j} = 0\).
    则
    \begin{align*}
        \det {((P(x; a_1, a_2, \dots, a_n))(1|n))}
        = (-1)^{n-1}.
    \end{align*}

    综上,
    \begin{align*}
             &
        \det {(P(x; a_1, a_2, \dots, a_n))}
        \\
        = {} &
        \hphantom{{} + {}}
        (-1)^{1+1} [P(x; a_1, a_2, \dots, a_n)]_{1,1}
        \det {((P(x; a_1, a_2, \dots, a_n))(1|1))}
        \\
             &
        + (-1)^{1+n} [P(x; a_1, a_2, \dots, a_n)]_{1,n}
        \det {((P(x; a_1, a_2, \dots, a_n))(1|n))}
        \\
        = {} &
        x \det {(P(x; a_1, a_2, \dots, a_{n-1}))}
        + (-1)^{1+n} a_n (-1)^{n-1}
        \\
        = {} &
        x \det {(P(x; a_1, a_2, \dots, a_{n-1}))} + a_n.
    \end{align*}

    类似地,
    \begin{align*}
        \det {(P(x; a_1, a_2, \dots, a_{n-1}))}
        = x \det {(P(x; a_1, a_2, \dots, a_{n-2}))} + a_{n-1}.
    \end{align*}
    故
    \begin{align*}
        \det {(P(x; a_1, a_2, \dots, a_n))}
        = {} &
        x \det {(P(x; a_1, a_2, \dots, a_{n-1}))} + a_n
        \\
        = {} &
        x (x \det {(P(x; a_1, a_2, \dots, a_{n-2}))} + a_{n-1}) + a_n
        \\
        = {} &
        x^2 \det {(P(x; a_1, a_2, \dots, a_{n-2}))}
        + a_{n-1} x + a_n.
    \end{align*}

    \(\dots \dots\)

    最后, 我们有
    \begin{align*}
        \det {(P(x; a_1, a_2, \dots, a_n))}
        = {} &
        x \det {(P(x; a_1, a_2, \dots, a_{n-1}))} + a_n
        \\
        = {} &
        x^2 \det {(P(x; a_1, a_2, \dots, a_{n-2}))} + a_{n-1} x + a_n
        \\
        = {} &
        \dots \dots \dots \dots
        \\
        = {} &
        x^{n-1} \det {(P(x; a_1))} + a_2 x^{n-2} + \dots + a_{n-1} x + a_n
        \\
        = {} &
        x^{n-1} (x + a_1) + a_2 x^{n-2} + \dots + a_{n-1} x + a_n
        \\
        = {} &
        x^n + a_1 x^{n-1} + a_2 x^{n-2} + \dots + a_{n-1} x + a_n.
    \end{align*}
\end{example}

虽然我写了不少例,
我并不想在本书具体地介绍%
如何计算一个阵的行列式;
那或许不是我的事.
我的目标只是带您入门行列式;
这些例的目的是使您更好地理解行列式的性质.
若您对计算一个阵的行列式的方法感兴趣,
您可以见一些线性代数 (或高等代数) 教材,
或者找相关的文献.

\section{排列}

一般地, 在以组合定义定义行列式的教材里,
排列 (或置换) 的一些基本的性质是必要的,
因为它们对论证行列式的性质有用
(通俗地, 这些教材用排列研究行列式).
我想在本节用行列式研究排列
(通俗地, 我要反着干).

研究排列时, 有一种行为相当重要.

\begin{definition}[对换]
    设 \(a_1\), \(a_2\), \(\dots\), \(a_n\) 是一个排列.
    设 \(i\), \(j\) 是不超过 \(n\) 的二个不等的正整数.
    我们交换此排列的第~\(i\) 个文字与第~\(j\) 个文字
    (也就是说, 交换文字 \(a_i\), \(a_j\)),
    且不改变其他的文字,
    则, 我们得到了一个新的排列
    \(b_1\), \(b_2\), \(\dots\), \(b_n\),
    其中
    \begin{align*}
        b_k
        = \begin{cases}
              a_j, & k = i;     \\
              a_i, & k = j;     \\
              a_k, & \text{其他}.
          \end{cases}
    \end{align*}
    我们说, 变 \(a_1\), \(a_2\), \(\dots\), \(a_n\)
    为 \(b_1\), \(b_2\), \(\dots\), \(b_n\)
    的行为是一次\emph{对换}.
    特别地, 若 \(j = i + 1\) 或
    \(j = i - 1\)
    (通俗地, 这就是说,
    \(a_i\), \(a_j\), 或 \(a_j\), \(a_i\), 在
    \(a_1\), \(a_2\), \(\dots\), \(a_n\) 里,
    是相邻的二个文字),
    我们说, 这一次对换是一次\emph{相邻对换}.
\end{definition}

\begin{example}
    考虑排列~I: \(3\), \(2\), \(5\), \(1\), \(4\).
    交换 \(3\) 与 \(1\) 的位置,
    得到排列~II: \(1\), \(2\), \(5\), \(3\), \(4\).
    我们说, 变 I 为 II 的行为是一次对换.

    还是考虑排列~I.
    交换 \(1\) 与 \(4\) 的位置,
    得到排列~III: \(3\), \(2\), \(5\), \(4\), \(1\).
    变 I 为 III 的行为当然也是一次对换.
    不过, 因为 \(1\), \(4\) 在 I 里是%
    相邻的二个文字,
    故我们也可说,
    变 I 为 III 的行为是一次相邻对换.

    值得注意的是,
    一次对换的结果可以跟多次相邻对换的结果一样.
    比如, 为了变 I 为 II,
    我们可以这么干:

    交换 \(3\) 与 \(2\) 的位置,
    得排列~IV: \(2\), \(3\), \(5\), \(1\), \(4\);

    交换 \(3\) 与 \(5\) 的位置,
    得排列~V: \(2\), \(5\), \(3\), \(1\), \(4\);

    交换 \(3\) 与 \(1\) 的位置,
    得排列~VI: \(2\), \(5\), \(1\), \(3\), \(4\);

    交换 \(5\) 与 \(1\) 的位置,
    得排列~VII: \(2\), \(1\), \(5\), \(3\), \(4\);

    交换 \(2\) 与 \(1\) 的位置,
    得排列~VIII: \(1\), \(2\), \(5\), \(3\), \(4\).

    可以看到,
    我们利用 \(5\)~次相邻对换实现了%
    一次 (不是相邻的) 对换.
    注意到, 这是一个奇数.

    其实, 不难看出这么一件事.
    设 \(a_1\), \(a_2\), \(\dots\), \(a_n\) 是一个排列,
    \(i\), \(j\) 是不超过 \(n\) 的二个不等的正整数,
    且 \(i < j\).
    则, 我们可作 \(2(j - i) - 1\)~次相邻对换%
    以交换此排列的第~\(i\) 个文字与第~\(j\) 个文字.
\end{example}

相邻对换与逆序数有如下关系.

\begin{theorem}
    设 \(a_1\), \(a_2\), \(\dots\), \(a_n\)
    是互不相同的 \(n\)~个整数.
    设排列~I: \(a_1\), \(a_2\), \(\dots\), \(a_n\)
    的逆序数 \(\tau (a_1, a_2, \dots, a_n) = T\).
    那么, 我们可作 \(T\)~次相邻对换%
    变排列~I 为其自然排列.
\end{theorem}

% 我宣布一件事.
% 从现在开始, 我所说的 ``文字列'' 的文字全是整数;
% 自然地, 我所说的 ``排列'' 的文字也全是整数.

\begin{proof}
    我们先设 \(a_1\), \(a_2\), \(\dots\), \(a_n\)
    是 \(1\), \(2\), \(\dots\), \(n\) 的排列.
    作命题 \(P(n)\):
    \begin{quotation}
        任取 \(1\), \(2\), \(\dots\), \(n\)
        的一个排列 \(a_1\), \(a_2\), \(\dots\), \(a_n\),
        我们可作 \(\tau (a_1, a_2, \dots, a_n)\)~次相邻对换%
        变其为自然排列 \(1\), \(2\), \(\dots\), \(n\).
    \end{quotation}
    我们用数学归纳法证明,
    对任何正整数 \(n\), \(P(n)\) 成立.

    \(P(1)\) 是正确的.
    毕竟, \(1\) 只有 \(1\)~个排列: \(1\).
    我们不必作任何对换, 它已是自然排列.
    排列 \(1\)~的逆序数自然是 \(0\).

    现在, 我们设 \(P(m-1)\) 是正确的.
    我们证 \(P(m)\) 也是正确的.
    任取 \(1\), \(2\), \(\dots\), \(m\) 的一个排列
    \(a_1\), \(a_2\), \(\dots\), \(a_m\).
    那么, 一定存在, 且只存在一个不超过 \(m\) 的正整数 \(k\),
    使 \(a_k = m\).
    我们交换 \(m\) 与 \(a_{k+1}\),
    再交换 \(m\) 与 \(a_{k+2}\),
    \(\dots \dots\),
    再交换 \(m\) 与 \(a_m\),
    从而可变 \(a_1\), \(a_2\), \(\dots\), \(a_m\)
    为 \(a_1\), \(\dots\), \(a_{k-1}\),
    \(a_{k+1}\), \(\dots\), \(a_m\), \(m\).
    这 \(m - k\)~次相邻对换%
    使 \(m\) 是最后一个数
    % 使 \(m\) 在最末的位置
    (若 \(a_m = m\), 则不必作对换了,
    故此关系仍成立),
    且新排列的前 \(m-1\)~个整数
    \(a_1\), \(\dots\), \(a_{k-1}\),
    \(a_{k+1}\), \(\dots\), \(a_m\)
    是 \(1\), \(2\), \(\dots\), \(m-1\) 的排列.
    由假定, 我们可作
    \(
    T_{m-1} = \tau (a_1, \dots, a_{k-1}, a_{k+1}, \dots, a_m)
    \)
    次相邻对换,
    变 \(a_1\), \(\dots\), \(a_{k-1}\),
    \(a_{k+1}\), \(\dots\), \(a_m\)
    为其自然排列 \(1\), \(2\), \(\dots\), \(m-1\).
    所以, 我们可作 \(T_{m-1} + (m - k)\)~次相邻对换%
    变 \(a_1\), \(a_2\), \(\dots\), \(a_m\)
    为其自然排列 \(1\), \(2\), \(\dots\), \(m\).

    注意到, 每一个适合条件 \(1 \leq i < j \leq m\)
    的有序整数对 \((i, j)\)
    \emph{恰}适合以下三个条件之一:
    (a)
    \(i \neq k\) 且 \(j \neq k\);
    (b)
    \(i = k\) 且 \(k < j \leq m\);
    (c)
    \(1 \leq i < k\) 且 \(j = k\).
    再注意到, 对每一个不等于 \(k\),
    且不超过 \(m\) 的正整数 \(\ell\),
    必有 \(a_\ell < m\).
    故
    \begin{align*}
             & \tau (a_1, a_2, \dots, a_m)
        \\
        = {} &
        \sum_{1 \leq i < j \leq m} {\rho(a_i, a_j)}
        \\
        = {} &
        \sum_{\substack{1 \leq i < j \leq m \\i,j \neq k}}
        {\rho(a_i, a_j)}
        + \sum_{k < j \leq m} {\rho(a_k, a_j)}
        + \sum_{1 \leq i < k} {\rho(a_i, a_k)}
        \\
        = {} &
        \tau (a_1, \dots, a_{k-1}, a_{k+1}, \dots, a_m)
        + \sum_{k < j \leq m} {1}
        + \sum_{1 \leq i < k} {0}
        \\
        = {} &
        T_{m-1} + (m - k).
    \end{align*}
    从而, 我们可作
    \(\tau (a_1, a_2, \dots, a_m)\)~次相邻对换%
    变 \(a_1\), \(a_2\), \(\dots\), \(a_m\)
    为其自然排列 \(1\), \(2\), \(\dots\), \(m\).

    所以, \(P(m)\) 是正确的.
    由数学归纳法原理,
    对任何正整数 \(n\), \(P(n)\) 成立.

    最后, 我们看一般的情形.

    设 \(a_1\), \(a_2\), \(\dots\), \(a_n\)
    的自然排列是 \(c_1\), \(c_2\), \(\dots\), \(c_n\).
    我们记 \(f(c_i) = i\)
    (\(i = 1\), \(2\), \(\dots\), \(n\)).
    (通俗地, \(f(a_i)\) 表示
    \(a_i\) 是第~\(f(a_i)\) \emph{小}的数.)
    那么, 不难看出,
    若 \(d_1\), \(d_2\), \(\dots\), \(d_n\)
    是 \(c_1\), \(c_2\), \(\dots\), \(c_n\)
    的一个排列,
    则 \(f(d_1)\), \(f(d_2)\), \(\dots\), \(f(d_n)\)
    是 \(1\), \(2\), \(\dots\), \(n\)
    的一个排列;
    反过来, 若 \(v_1\), \(v_2\), \(\dots\), \(v_n\) 是
    \(1\), \(2\), \(\dots\), \(n\) 的一个排列,
    则 \(c_{v_1}\), \(c_{v_2}\), \(\dots\), \(c_{v_n}\)
    一定是 \(c_1\), \(c_2\), \(\dots\), \(c_n\) 的一个排列.
    也不难看出, 若 \(d_t < d_v\), 必有 \(f(d_t) < f(d_v)\).
    反过来, 若 \(f(d_t) < f(d_v)\), 必有 \(d_t < d_v\).
    从而, \(f(a_1)\), \(f(a_2)\), \(\dots\), \(f(a_n)\)
    的逆序数等于 \(a_1\), \(a_2\), \(\dots\), \(a_n\)
    的逆序数.
    我们分别为 \(a_1\), \(a_2\), \(\dots\), \(a_n\)
    取 ``小名'' \(f(a_1)\), \(f(a_2)\), \(\dots\), \(f(a_n)\),
    即可用老方法,
    作跟 \(a_1\), \(a_2\), \(\dots\), \(a_n\)
    的逆序数一样多的次数的相邻对换,
    变 \(a_1\), \(a_2\), \(\dots\), \(a_n\) 为其%
    自然排列 \(c_1\), \(c_2\), \(\dots\), \(c_n\).
    (注意到, 我们总是可用类似的手法,
    变一般的排列的研究为
    \(1\), \(2\), \(\dots\), \(n\) 的排列的研究.
    所以, 当我们得到
    \(1\), \(2\), \(\dots\), \(n\)
    的排列的\emph{只跟逆序有关}的性质后,
    我们总可译其为一般的排列的性质.)
\end{proof}

\begin{theorem}
    设排列~I: \(a_1\), \(a_2\), \(\dots\), \(a_n\).
    施一次对换于排列~I,
    得排列~II: \(b_1\), \(b_2\), \(\dots\), \(b_n\).
    则
    \begin{align*}
        s(b_1, b_2, \dots, b_n) = -s(a_1, a_2, \dots, a_n).
    \end{align*}
    (通俗地, 一次对换变排列的号.)
\end{theorem}

\begin{proof}
    无妨设排列~I, II 都是
    \(1\), \(2\), \(\dots\), \(n\) 的排列;
    可用我之前提到的手法推广此事于任何的排列.

    设 \(n\)~级单位阵
    \(I = [e_1, e_2, \dots, e_n]\).
    作二个 \(n\)~级阵
    \(A = [e_{a_1}, e_{a_2}, \dots, e_{a_n}]\),
    \(B = [e_{b_1}, e_{b_2}, \dots, e_{b_n}]\).
    那么, \(B\) 可被认为是交换了 \(A\)~的二列,
    且不变其他列得到的阵.
    从而,
    \begin{align*}
             & s(b_1, b_2, \dots, b_n)
        \\
        = {} & s(b_1, b_2, \dots, b_n) \det {(I)}
        \\
        = {} & \det {(B)}
        \\
        = {} & {-\det {(A)}}
        \\
        = {} & {-s(a_1, a_2, \dots, a_n)} \det {(I)}
        \\
        = {} & {-s(a_1, a_2, \dots, a_n)}.
        \qedhere
    \end{align*}
\end{proof}

\begin{theorem}
    设 \(n \geq 2\).
    设 \(a_1\), \(a_2\), \(\dots\), \(a_n\) 是
    \(n\)~个互不相同的整数.
    那么, 在
    \(a_1\), \(a_2\), \(\dots\), \(a_n\) 的全部排列中,
    符号为 \(1\) 的排列%
    跟符号为 \(-1\) 的排列%
    的数目相等.
\end{theorem}

\begin{proof}
    设 \(a_1\), \(a_2\), \(\dots\), \(a_n\)
    的自然排列是 \(c_1\), \(c_2\), \(\dots\), \(c_n\).
    不难看出,
    每一个 \(a_1\), \(a_2\), \(\dots\), \(a_n\) 的排列%
    都是 \(c_1\), \(c_2\), \(\dots\), \(c_n\) 的排列,
    且每一个 \(c_1\), \(c_2\), \(\dots\), \(c_n\) 的排列%
    都是 \(a_1\), \(a_2\), \(\dots\), \(a_n\) 的排列,

    无妨设 \(c_1\), \(c_2\), \(\dots\), \(c_n\)
    分别为 \(1\), \(2\), \(\dots\), \(n\);
    可用我之前提到的手法推广此事于任何的排列.

    作 \(n\)~级阵 \(U\),
    其中 \([U]_{i,j} = 1\)
    (通俗地, \(U\) 的每一个元都是 \(1\)).
    根据交错性, \(\det {(U)} = 0\).
    根据完全展开
    (取 \(j_1\), \(j_2\), \(\dots\), \(j_n\)
    为 \(1\), \(2\), \(\dots\), \(n\)),
    \begin{align*}
        0
        = {} & \det {(U)}
        \\
        = {} & \sum_{\substack{
        1 \leq i_1, i_2, \dots, i_n \leq n \\
                i_1, i_2, \dots, i_n\,\text{互不相同}
            }}
        {s(i_1, i_2, \dots, i_n)\,
            [U]_{i_1,1} [U]_{i_2,2} \dots [U]_{i_n,n}}
        \\
        = {} & \sum_{\substack{
        1 \leq i_1, i_2, \dots, i_n \leq n \\
                i_1, i_2, \dots, i_n\,\text{互不相同}
            }}
        {s(i_1, i_2, \dots, i_n)}.
    \end{align*}
    我问一个简单的问题:
    设有 \(k\)~个数, 每一个都是 \(1\) 或 \(-1\).
    若它们的和为零, 请问, 有几个 \(1\), 有几个 \(-1\)?
    我想, 您能答出,
    \(1\) 跟 \(-1\) 的个数都是 \(k/2\).
\end{proof}

由此可见,
完全展开 \(n\)~级阵 (\(n \geq 2\)) 的行列式的公式里,
符号为 \(1\)~的项跟符号为 \(-1\) 的项%
的数目都是
\((n \cdot (n - 1) \cdot \dots \cdot 2 \cdot 1)/2\).

\vspace{2ex}

最后, 我想说公理定义的事.

或许, 您还记得,
我在第一章, 节~\sekcio{14},
``用行列式的性质确定行列式'' 里说过,
行列式的规范性、多线性、交错性可确定行列式:

\begin{theorem}
    设定义在全体 \(n\)~级阵上的函数 \(f\) 适合:

    (1)
    (多线性)
    对任何不超过 \(n\) 的正整数 \(j\),
    任何 \(n-1\)~个 \(n \times 1\)~阵
    \(a_1\), \(\dots\), \(a_{j-1}\),
    \(a_{j+1}\), \(\dots\), \(a_n\),
    任何二个 \(n \times 1\)~阵 \(x\), \(y\),
    任何二个数 \(s\), \(t\),
    有
    \begin{align*}
             & f
            {([a_1, \dots, a_{j-1}, sx + ty,
                        a_{j+1}, \dots, a_n])}
        \\
        = {} &
        s
        f {([a_1, \dots, a_{j-1}, x, a_{j+1}, \dots, a_n])}
        +
        t
        f {([a_1, \dots, a_{j-1}, y, a_{j+1}, \dots, a_n])}.
    \end{align*}

    (2)
    (交错性)
    若 \(n\)~级阵 \(A\) 有二列完全相同,
    则 \(f {(A)} = 0\).

    那么, 对任何 \(n\)~级阵 \(A\),
    \(f(A) = f(I) \det {(A)}\).

    特别地, 若 \(f(I) = 1\) (规范性),
    则 \(f\) 就是行列式.
\end{theorem}

还记得我如何证此事吗?
我作了一个辅助函数
\begin{align*}
    g(A) = f(A) - f(I) \det {(A)}.
\end{align*}
可以验证, \(g\) 是多线性的、交错性的,
且 \(g(I) = 0\).
于是, 设法证对任何 \(n\)~级阵 \(A\),
\(g(A) = 0\) 即可.
我为什么要作这个 \(g\)?

为解释此事, 我们看,
若我不作此 \(g\), 会如何.
首先, 因为 \(f\) 的多线性与交错性,
我们可类似地写出
\begin{align*}
    f(A)
    = \sum_{\substack{
    1 \leq i_1, i_2, \dots, i_n \leq n \\
            i_1, i_2, \dots, i_n\,\text{互不相同}
        }}
    {[A]_{i_1,1} [A]_{i_2,2} \dots [A]_{i_n,n}\,
        f([e_{i_1}, e_{i_2}, \dots, e_{i_n}])},
\end{align*}
其中 \(e_k\) 是 \(n\)~级单位阵~\(I\) 的列~\(k\).

接着, 我们要确定
\(f([e_{i_1}, e_{i_2}, \dots, e_{i_n}])\).
由多线性与交错性, 我们可推出反称性.
这样, 我们可以适当地交换
\([e_{i_1}, e_{i_2}, \dots, e_{i_n}]\)~的列,
可变其为 \([e_1, e_2, \dots, e_n]\),
即 \(n\)~级单位阵 \(I\).
从而, 存在一个由 \(i_1\), \(i_2\), \(\dots\), \(i_n\)
确定的数 \(\sigma (i_1, i_2, \dots, i_n)\),
它等于 \(1\) 或 \(-1\),
使
\begin{align*}
    f([e_{i_1}, e_{i_2}, \dots, e_{i_n}])
    = \sigma (i_1, i_2, \dots, i_n)\, f(I).
\end{align*}
故
\begin{align*}
    f(A)
    = f(I) \sum_{\substack{
    1 \leq i_1, i_2, \dots, i_n \leq n \\
            i_1, i_2, \dots, i_n\,\text{互不相同}
        }}
    {\sigma (i_1, i_2, \dots, i_n)\,
        [A]_{i_1,1} [A]_{i_2,2} \dots [A]_{i_n,n}}.
\end{align*}
问题来了: \(\sigma (i_1, i_2, \dots, i_n)\)
究竟是什么?

我们回想,
任给 \(1\), \(2\), \(\dots\), \(n\)
的一个排列 \(i_1\), \(i_2\), \(\dots\), \(i_n\),
我们总可作 \(\tau (i_1, i_2, \dots, i_n)\)~次%
相邻对换变其为自然排列.
所以, 我们可作
\(\tau (i_1, i_2, \dots, i_n)\)~次列的交换
(每次交换二列),
变 \([e_{i_1}, e_{i_2}, \dots, e_{i_n}]\) 为 \(I\).
由反称性,
\begin{align*}
         & f([e_{i_1}, e_{i_2}, \dots, e_{i_n}])
    \\
    = {} & (-1)^{\tau (i_1, i_2, \dots, i_n)}
    f([e_1, e_2, \dots, e_n])                    \\
    = {} & s(i_1, i_2, \dots, i_n)\, f(I).
\end{align*}
所以,
% 神秘的
\(\sigma (i_1, i_2, \dots, i_n)\)
% 实为老朋友
其实就是
\(s(i_1, i_2, \dots, i_n)\).
再根据完全展开
(取 \(j_1\), \(j_2\), \(\dots\), \(j_n\)
为 \(1\), \(2\), \(\dots\), \(n\)),
可知 \(f(A) = f(I) \det {(A)}\).

由此可见, 理论地, 不作辅助函数 \(g\) 也是可证此事的.
但是, 跟作辅助函数 \(g\) 的论证相比,
这个试直接计算 \(f(A)\) 的论证至少用到了二件事:

(1)
作跟逆序数一样多的次数的相邻对换,
可变一个排列为其自然排列;

(2)
完全展开行列式的公式.

注意到, (1) 是我已提到的, 而 (2)
(在我的教学里) 是选学的内容.
所以, 为了使论证更好地被理解,
也为了简化论证,
我作了辅助函数.
\(g\) 的一个特点是 \(g(I) = 0\).
所以, 利用反称性%
变 \(g([e_{i_1}, e_{i_2}, \dots, e_{i_n}])\)
为 \(g(I)\) 时,
我们不必关注变了几次号,
也不必关注如何变
\(i_1\), \(i_2\), \(\dots\), \(i_n\)
为 \(1\), \(2\), \(\dots\), \(n\)
(比如, 要变 \(3\), \(2\), \(5\), \(1\), \(4\)
为 \(1\), \(2\), \(3\), \(4\), \(5\),
可以考虑交换 \(5\), \(4\) 的位置,
再交换 \(4\), \(1\) 的位置,
再交换 \(3\), \(1\) 的位置);
我们知道, \emph{可适当地交换文字的位置},
变 \(i_1\), \(i_2\), \(\dots\), \(i_n\)
为 \(1\), \(2\), \(\dots\), \(n\),
即可.
``零跟任何数的积都是零''
起了较大的作用.

用同样的手法, 可证

\begin{theorem}
    设 \(c\) 是常数.
    设定义在全体 \(n\)~级阵上的函数 \(f\) 适合:

    (1)
    (多线性)
    对任何不超过 \(n\) 的正整数 \(j\),
    任何 \(n-1\)~个 \(n \times 1\)~阵
    \(a_1\), \(\dots\), \(a_{j-1}\),
    \(a_{j+1}\), \(\dots\), \(a_n\),
    任何二个 \(n \times 1\)~阵 \(x\), \(y\),
    任何二个数 \(s\), \(t\),
    有
    \begin{align*}
             & f
            {([a_1, \dots, a_{j-1}, sx + ty,
                        a_{j+1}, \dots, a_n])}
        \\
        = {} &
        s
        f {([a_1, \dots, a_{j-1}, x, a_{j+1}, \dots, a_n])}
        +
        t
        f {([a_1, \dots, a_{j-1}, y, a_{j+1}, \dots, a_n])}.
    \end{align*}

    (2)
    (交错性)
    若 \(n\)~级阵 \(A\) 有二列完全相同,
    则 \(f {(A)} = 0\).

    (3)
    (``规范性'')
    \(f(I) = c\).

    设定义在全体 \(n\)~级阵上的函数 \(h\)
    也适合这三条性质.
    则对任何 \(n\)~级阵 \(A\), \(f(A) = h(A)\).
\end{theorem}

\begin{proof}
    作辅助函数 \(g(A) = f(A) - h(A)\).
    此 \(g\) 也有多线性与交错性,
    且 \(g(I) = 0\).
    然后, 用我 (在第一章, 节~\sekcio{14}) 用过的手法即可.
    (注意, 此事甚至没提到
    ``行列式''
    ``det''
    等文字.)
\end{proof}

\section{Binet--Cauchy 公式}

我要介绍证 Binet--Cauchy 公式的另一种方法.

在第一章, 节~\malneprasekcio{30} 里,
我用完全展开行列式的公式证明了它.
这种证法至少有二个好处:

(1)
记号较简单;

(2)
方便您对比选学内容
``Binet--Cauchy 公式''
与必学内容
``Binet--Cauchy 公式 (青春版)''.
% 体会论证的异同
% (其实, 它们大同小异).

当然, 此证法的一个较大的缺点是
``用到了完全展开
(或, 组合定义相关的公式)''.
% 我讲课时, 会尽可能地不跳过 ``选学内容'';
% 但是,
若您真地不会 (或, 不想用) 完全展开,
且您又想了解如何证明 Binet--Cauchy 公式,
我在这儿给一个利用按多列展开行列式的公式的证明.
至少, 按多列展开行列式 (及其论证) 不涉及
``排列'' ``逆序数'' ``符号''
等概念.

这个论证是我初学 Binet--Cauchy 公式时学到的证明.
这个论证体现了重要思想 ``算二次 (dufoje)'';
聪明的数学家利用此原则, 得到了不少有名的等式.

为方便, 请允许我引用%
按多列展开行列式公式与 Binet--Cauchy 公式.

\begin{theorem}
    设 \(A\) 是 \(n\)~级阵 (\(n \geq 1\)).
    设 \(k\) 是不超过 \(n\) 的正整数.
    设 \(j_1\), \(j_2\), \(\dots\), \(j_k\) 是%
    不超过 \(n\) 的正整数,
    且 \(j_1 < j_2 < \dots < j_k\).
    则
    \begin{align*}
         &
        \det {(A)}
        = \sum_{1 \leq i_1 < i_2 < \dots < i_k \leq n}
        {\det {\left(
                A\binom{i_1, i_2, \dots, i_k}
                {j_1, j_2, \dots, j_k}
                \right)}}
        \\
         &
        \qquad \qquad \qquad
        \cdot (-1)^{i_1 + i_2 + \dots + i_k
            + j_1 + j_2 + \dots + j_k}
        \det {(A({i_1,i_2,\dots,i_k}|{j_1,j_2,\dots,j_k}))}.
    \end{align*}
\end{theorem}

\begin{theorem}[Binet--Cauchy 公式]
    设 \(A\), \(B\) 分别是 \(m \times n\), \(n \times m\)~阵.

    (1)
    若 \(n > m\), 则 \(\det {(BA)} = 0\).

    (2)
    若 \(n \leq m\),
    则
    \begin{align*}
             & \det {(BA)}
        \\
        = {} & \sum_{1 \leq j_1 < j_2 < \dots < j_n \leq m}
        {
            \det {\left(
                B\binom{1, 2, \dots, n}{j_1, j_2, \dots, j_n}
                \right)}
            \det {\left(
                A\binom{j_1, j_2, \dots, j_n}{1, 2, \dots, n}
                \right)}
        }.
    \end{align*}
    特别地, 若 \(n = m\), 则因适合条件
    \(1 \leq j_1 < j_2 < \dots < j_n \leq m\)
    的 \(j_1\), \(j_2\), \(\dots\), \(j_n\),
    是, 且只能是,
    \(1\), \(2\), \(\dots\), \(n\),
    故
    \begin{align*}
        \det {(BA)}
        = {} &
        \det {\left(
            B\binom{1, 2, \dots, n}{1, 2, \dots, n}
            \right)}
        \det {\left(
            A\binom{1, 2, \dots, n}{1, 2, \dots, n}
            \right)}
        \\
        = {} & \det {(B)} \det {(A)}.
    \end{align*}
\end{theorem}

有几件较重要小事值得一提.

(1)
若一个方阵有一行的元全是 \(0\),
则其行列式为 \(0\).

可以直接用定义 (按列~\(1\) 展开行列式) 论证此事;
可用按一行展开行列式的公式论证此事;
当然, 也可利用 (关于行的) 多线性论证此事.
我就不在这儿证了.

(2)
若加一个阵的一列的倍于另一列, 则其行列式不变.

这就是我在前面提到的行列式的一个性质.

(3)
设 \(A\), \(B\), \(C\), \(D\)
分别是 \(m \times s\), \(n \times s\),
\(m \times t\), \(n \times t\)~阵.
那么,
\begin{align*}
    \begin{bmatrix}
        A & C \\
        B & D \\
    \end{bmatrix}
\end{align*}
表示一个 \((m + n) \times (s + t)\)-阵 \(M\),
且对任何不超过 \(m + n\)~的正整数 \(i\)
与不超过 \(s + t\)~的正整数 \(j\),
\begin{align*}
    [M]_{i,j}
    = \begin{cases}
          [A]_{i,j},
           & \text{\(i \leq m\), \(j \leq s\)}; \\
          [B]_{i-m,j},
           & \text{\(i > m\), \(j \leq s\)};    \\
          [C]_{i,j-s},
           & \text{\(i \leq m\), \(j > s\)};    \\
          [D]_{i-m,j-s},
           & \text{\(i > m\), \(j > s\)}.
      \end{cases}
\end{align*}

这是我们在第一章, 节~\sekcio{31} 里见过的记号.
现在, 我要进一步地发展此记号.
具体地,
设 \(A\), \(B\)
分别是 \(m \times s\), \(n \times s\)~阵.
那么,
\begin{align*}
    \begin{bmatrix}
        A \\
        B \\
    \end{bmatrix}
\end{align*}
表示一个 \((m + n) \times s\)-阵 \(L\),
且对任何不超过 \(m + n\)~的正整数 \(i\)
与不超过 \(s\)~的正整数 \(j\),
\begin{align*}
    [L]_{i,j}
    = \begin{cases}
          [A]_{i,j},
           & \text{\(i \leq m\)}; \\
          [B]_{i-m,j},
           & \text{\(i > m\)}.
      \end{cases}
\end{align*}

类似地, 设 \(A\), \(C\)
分别是 \(m \times s\), \(m \times t\)~阵.
那么,
\begin{align*}
    \begin{bmatrix}
        A & C \\
    \end{bmatrix}
\end{align*}
表示一个 \(m \times (s + t)\)-阵 \(H\),
且对任何不超过 \(m\)~的正整数 \(i\)
与不超过 \(s + t\)~的正整数 \(j\),
\begin{align*}
    [H]_{i,j}
    = \begin{cases}
          [A]_{i,j},
           & \text{\(j \leq s\)}; \\
          [C]_{i,j-s},
           & \text{ \(j > s\)}.
      \end{cases}
\end{align*}

\begin{proof}
    设 \(A\), \(B\) 分别是
    \(m \times n\), \(n \times m\)~阵.
    作 \(m + n\)~级阵
    \begin{align*}
        J =
        \begin{bmatrix}
            I_m & A \\
            -B  & 0 \\
        \end{bmatrix},
    \end{align*}
    其中左上角的 \(I_m\) 是 \(m\)~级单位阵,
    右下角的 \(0\) 是 \(n \times n\)~零阵
    (元全为 \(0\) 的阵).

    我们用二种方式计算 \(J\) 的行列式.

    一方面, 我们按%
    列~\(m + 1\), \(m + 2\), \(\dots\), \(m + n\) 展开,
    有
    \begin{align*}
         &
        \det {(J)}
        = \sum_{1 \leq i_1 < i_2 < \dots < i_n \leq m+n}
        {\det {\left(
                J\binom{i_1,i_2,\dots,i_n}{m+1,m+2,\dots,m+n}
                \right)}}
        \\
         &
        \qquad \qquad \qquad
        \cdot (-1)^{i_1 + i_2 + \dots + i_n
            + (m+1) + (m+2) + \dots + (m+n)}
        \\
         &
        \qquad \qquad \qquad
        \cdot \det {(J({i_1,i_2,\dots,i_n}|{m+1,m+2,\dots,m+n}))}.
    \end{align*}
    注意到, 若 \(i_n > m\), 则因
    \(\displaystyle
    J\binom{i_1,i_2,\dots,i_n}{m+1,m+2,\dots,m+n}\)
    的行~\(n\) 的元全是 \(0\),
    故这个子阵的行列式为 \(0\).
    特别地, 若 \(n > m\), 则 \(i_n\) 必高于 \(m\)
    (抽屉原理).
    故 \(n > m\) 时, \(J\) 的行列式为 \(0\).

    当 \(n \leq m\) 时,
    我们去除使 \(i_n > m\) 的
    \(i_1\), \(i_2\), \(\dots\), \(i_n\),
    有
    \begin{align*}
         &
        \det {(J)}
        = \sum_{1 \leq i_1 < i_2 < \dots < i_n \leq m}
        {\det {\left(
                J\binom{i_1,i_2,\dots,i_n}{m+1,m+2,\dots,m+n}
                \right)}}
        \\
         &
        \qquad \qquad \qquad
        \cdot (-1)^{i_1 + i_2 + \dots + i_n
            + (m+1) + (m+2) + \dots + (m+n)}
        \\
         &
        \qquad \qquad \qquad
        \cdot \det {(J({i_1,i_2,\dots,i_n}|{m+1,m+2,\dots,m+n}))}.
    \end{align*}
    不难看出,
    因为 \(1 \leq i_1 < i_2 < \dots < i_n \leq m\),
    \begin{align*}
        J\binom{i_1,i_2,\dots,i_n}{m+1,m+2,\dots,m+n}
        = A\binom{i_1,i_2,\dots,i_n}{1,2,\dots,n},
    \end{align*}
    且
    \begin{align*}
        J({i_1,i_2,\dots,i_n}|{m+1,m+2,\dots,m+n})
        =
        \begin{bmatrix}
            I_m (i_1,i_2,\dots,i_n|) \\
            -B
        \end{bmatrix},
    \end{align*}
    其中
    \(I_m (i_1,i_2,\dots,i_n|)\)
    表示去除 \(I_m\)~的%
    行~\(i_1\), \(i_2\), \(\dots\), \(i_n\)
    后 (不改变列) 得到的阵.
    按列~\(i_1\), \(i_2\), \(\dots\), \(i_n\) 展开
    \(T_{i_1,i_2,\dots,i_n}
    = J({i_1,i_2,\dots,i_n}|{m+1,m+2,\dots,m+n})\)~的%
    行列式, 有
    \begin{align*}
             & \det {(T_{i_1,i_2,\dots,i_n})}
        \\
        = {} &
        \sum_{1 \leq k_1 < k_2 < \dots < k_n \leq m}
        {\det {\left(
                T_{i_1,i_2,\dots,i_n}
                \binom{k_1, k_2, \dots, k_n}{i_1, i_2, \dots, i_n}
                \right)}}
        \\
             &
        \qquad
        \cdot (-1)^{k_1 + k_2 + \dots + k_n
            + i_1 + i_2 + \dots + i_n}
        \det {(T_{i_1,i_2,\dots,i_n}
        ({k_1,k_2,\dots,k_n}|{i_1,i_2,\dots,i_n}))}.
    \end{align*}
    注意到, 若 \(k_1 \leq m-n\),
    则因 \(\displaystyle T_{i_1,i_2,\dots,i_n}
    \binom{k_1,k_2,\dots,k_n}{i_1,i_2,\dots,i_n}\)
    的行~\(1\) 的元全是 \(0\),
    故这个子阵的行列式为 \(0\).
    并且, 若 \(k_1 > m-n\),
    则 \(k_1\), \(k_2\), \(\dots\), \(k_n\)
    是, 且只能是
    \(m-n+1\), \(m-n+2\), \(\dots\), \(m\).
    故
    \begin{align*}
             & \det {(T_{i_1,i_2,\dots,i_n})}
        \\
        = {} &
        \hphantom{\cdot\,\,}
        \det {\left(
            T_{i_1,i_2,\dots,i_n}
            \binom{m-n+1,m-n+2,\dots,m}{i_1,i_2,\dots,i_n}
            \right)}
        \\
             &
        \cdot (-1)^{(m-n+1) + (m-n+2) + \dots + m
            + i_1 + i_2 + \dots + i_n}
        \\
             &
        \cdot
        \det {(T_{i_1,i_2,\dots,i_n}
        ({m-n+1,m-n+2,\dots,m}|{i_1,i_2,\dots,i_n}))}
        \\
        = {} &
        \hphantom{\cdot\,\,}
        \det {\left(
            (-B)
            \binom{1,2,\dots,n}{i_1,i_2,\dots,i_n}
            \right)}
        \\
             & \cdot
        (-1)^{(m-n+1) + (m-n+2) + \dots + (m-n+n)
            + i_1 + i_2 + \dots + i_n}
        \\
             & \cdot
        \det {(I_m
        ({i_1,i_2,\dots,i_n}|{i_1,i_2,\dots,i_n}))}
        \\
        = {} &
        (-1)^{(m-n+1) + (m-n+2) + \dots + (m-n+n)
                + i_1 + i_2 + \dots + i_n}\,
        (-1)^n
        \det {\left(
            B
            \binom{1,2,\dots,n}{i_1,i_2,\dots,i_n}
            \right)}.
    \end{align*}
    从而
    \begin{align*}
        \det {(J)}
        = {} & \sum_{1 \leq i_1 < i_2 < \dots < i_n \leq m}
        {\det {\left(
                A\binom{i_1,i_2,\dots,i_n}{1,2,\dots,n}
                \right)}}
        \\
             & \qquad
        \cdot
        (-1)^{i_1 + i_2 + \dots + i_n
            + (m+1) + (m+2) + \dots + (m+n)}
        \\
             & \qquad
        \cdot
        (-1)^{(m-n+1) + (m-n+2) + \dots + (m-n+n)
            + i_1 + i_2 + \dots + i_n}\, (-1)^n
        \\
             & \qquad
        \cdot
        \det {\left(
            B
            \binom{1,2,\dots,n}{i_1,i_2,\dots,i_n}
            \right)}.
    \end{align*}
    注意到
    \begin{align*}
             &
        \hphantom{\cdot\,\,}
        (-1)^{i_1 + i_2 + \dots + i_n
            + (m+1) + (m+2) + \dots + (m+n)}
        \\
             &
        \cdot
        (-1)^{(m-n+1) + (m-n+2) + \dots + (m-n+n)
            + i_1 + i_2 + \dots + i_n}
        \cdot
        (-1)^n
        \\
        = {} & (-1)^{2(i_1+i_2+\dots+i_n)}
        \cdot (-1)^{2mn}
        \cdot (-1)^{2(1+2+\dots+n)}
        \cdot (-1)^{n-n^2}
        \\
        = {} & (-1)^{-n(n-1)}
        \\
        = {} & 1,
    \end{align*}
    故
    \begin{align*}
        \det {(J)}
        = \sum_{1 \leq i_1 < i_2 < \dots < i_n \leq m}
        {\det {\left(
                B
                \binom{1,2,\dots,n}{i_1,i_2,\dots,i_n}
                \right)}
            \det {\left(
                A\binom{i_1,i_2,\dots,i_n}{1,2,\dots,n}
                \right)}}.
    \end{align*}

    另一方面, 我们也可用行列式的性质,
    施适当的变换于 \(J\),
    得一个新阵 \(K\),
    且 \(J\), \(K\) 的行列式相等.
    具体地, 我们加 \(J\) 的%
    列~\(1\) 的 \(-[A]_{1,1}\)~倍于其列~\(m+1\),
    列~\(2\) 的 \(-[A]_{2,1}\)~倍于其列~\(m+1\),
    \(\dots \dots\),
    列~\(m\) 的 \(-[A]_{m,1}\)~倍于其列~\(m+1\),
    得
    \begin{align*}
        J_1 =
        \begin{bmatrix}
            1          & 0          & \cdots & 0          &
            0          & [A]_{1,2}  & \cdots & [A]_{1,n}    \\
            0          & 1          & \cdots & 0          &
            0          & [A]_{2,2}  & \cdots & [A]_{2,n}    \\
            \vdots     & \vdots     & {}     & \vdots     &
            \vdots     & \vdots     & {}     & \vdots       \\
            0          & 0          & \cdots & 1          &
            0          & [A]_{m,2}  & \cdots & [A]_{m,n}    \\
            -[B]_{1,1} & -[B]_{1,2} & \cdots & -[B]_{1,m} &
            p_{1,1}    & 0          & \cdots & 0            \\
            -[B]_{2,1} & -[B]_{2,2} & \cdots & -[B]_{2,m} &
            p_{2,1}    & 0          & \cdots & 0            \\
            \vdots     & \vdots     & {}     & \vdots     &
            \vdots     & \vdots     & {}     & \vdots       \\
            -[B]_{n,1} & -[B]_{n,2} & \cdots & -[B]_{n,m} &
            p_{n,1}    & 0          & \cdots & 0            \\
        \end{bmatrix},
    \end{align*}
    其中
    \begin{align*}
        p_{k,1}
        = {} &
        (-[B]_{k,1})(-[A]_{1,1})
        + (-[B]_{k,2})(-[A]_{2,1})
        + \dots
        + (-[B]_{k,m})(-[A]_{m,1})
        \\
        = {} &
        [B]_{k,1} [A]_{1,1}
        + [B]_{k,2} [A]_{2,1}
        + \dots
        + [B]_{k,m} [A]_{m,1}
        \\
        = {} &
        [BA]_{k,1}.
    \end{align*}
    根据行列式的性质,
    \(\det {(J)} = \det {(J_1)}\).

    我们加 \(J_1\) 的%
    列~\(1\) 的 \(-[A]_{1,2}\)~倍于其列~\(m+2\),
    列~\(2\) 的 \(-[A]_{2,2}\)~倍于其列~\(m+2\),
    \(\dots \dots\),
    列~\(m\) 的 \(-[A]_{m,2}\)~倍于其列~\(m+2\),
    得
    \begin{align*}
        J_2 =
        \begin{bmatrix}
            1          & 0          & \cdots & 0          &
            0          & 0          & \cdots & [A]_{1,n}    \\
            0          & 1          & \cdots & 0          &
            0          & 0          & \cdots & [A]_{2,n}    \\
            \vdots     & \vdots     & {}     & \vdots     &
            \vdots     & \vdots     & {}     & \vdots       \\
            0          & 0          & \cdots & 1          &
            0          & 0          & \cdots & [A]_{m,n}    \\
            -[B]_{1,1} & -[B]_{1,2} & \cdots & -[B]_{1,m} &
            p_{1,1}    & p_{1,2}    & \cdots & 0            \\
            -[B]_{2,1} & -[B]_{2,2} & \cdots & -[B]_{2,m} &
            p_{2,1}    & p_{2,2}    & \cdots & 0            \\
            \vdots     & \vdots     & {}     & \vdots     &
            \vdots     & \vdots     & {}     & \vdots       \\
            -[B]_{n,1} & -[B]_{n,2} & \cdots & -[B]_{n,m} &
            p_{n,1}    & p_{n,2}    & \cdots & 0            \\
        \end{bmatrix},
    \end{align*}
    其中
    \begin{align*}
        p_{k,2}
        = {} &
        (-[B]_{k,1})(-[A]_{1,2})
        + (-[B]_{k,2})(-[A]_{2,2})
        + \dots
        + (-[B]_{k,m})(-[A]_{m,2})
        \\
        = {} &
        [B]_{k,1} [A]_{1,2}
        + [B]_{k,2} [A]_{2,2}
        + \dots
        + [B]_{k,m} [A]_{m,2}
        \\
        = {} &
        [BA]_{k,2}.
    \end{align*}
    根据行列式的性质,
    \(\det {(J_1)} = \det {(J_2)}\),
    故
    \(\det {(J)} = \det {(J_2)}\).

    \(\dots \dots\)

    我们加 \(J_{n-1}\) 的%
    列~\(1\) 的 \(-[A]_{1,n}\)~倍于其列~\(m+n\),
    列~\(2\) 的 \(-[A]_{2,n}\)~倍于其列~\(m+n\),
    \(\dots \dots\),
    列~\(m\) 的 \(-[A]_{m,n}\)~倍于其列~\(m+n\),
    得
    \begin{align*}
        J_n
        = {} &
        \begin{bmatrix}
            1          & 0          & \cdots & 0          &
            0          & 0          & \cdots & 0            \\
            0          & 1          & \cdots & 0          &
            0          & 0          & \cdots & 0            \\
            \vdots     & \vdots     & {}     & \vdots     &
            \vdots     & \vdots     & {}     & \vdots       \\
            0          & 0          & \cdots & 1          &
            0          & 0          & \cdots & 0            \\
            -[B]_{1,1} & -[B]_{1,2} & \cdots & -[B]_{1,m} &
            p_{1,1}    & p_{1,2}    & \cdots & p_{1,n}      \\
            -[B]_{2,1} & -[B]_{2,2} & \cdots & -[B]_{2,m} &
            p_{2,1}    & p_{2,2}    & \cdots & p_{2,n}      \\
            \vdots     & \vdots     & {}     & \vdots     &
            \vdots     & \vdots     & {}     & \vdots       \\
            -[B]_{n,1} & -[B]_{n,2} & \cdots & -[B]_{n,m} &
            p_{n,1}    & p_{n,2}    & \cdots & p_{n,n}      \\
        \end{bmatrix}
        \\
        = {} &
        \begin{bmatrix}
            I_m & 0  \\
            -B  & BA \\
        \end{bmatrix},
    \end{align*}
    其中右上角的 \(0\) 是 \(m \times n\)~零阵,
    且
    \begin{align*}
        p_{k,n}
        = {} &
        (-[B]_{k,1})(-[A]_{1,n})
        + (-[B]_{k,2})(-[A]_{2,n})
        + \dots
        + (-[B]_{k,m})(-[A]_{m,n})
        \\
        = {} &
        [B]_{k,1} [A]_{1,n}
        + [B]_{k,2} [A]_{2,n}
        + \dots
        + [B]_{k,m} [A]_{m,n}
        \\
        = {} &
        [BA]_{k,n}.
    \end{align*}
    根据行列式的性质,
    \(\det {(J_n)} = \det {(J_{n-1})}\),
    故
    \(\det {(J)} = \det {(J_n)}\).

    我们要如何计算 \(J_n\)~的行列式?
    我给您三条路:
    (a)
    用第一章, 节~\sekcio{31} 的第~2~个例
    (计算 \(J_n\) 的转置的行列式);
    (b)
    按行 \(1\), \(2\), \(\dots\), \(m\) 展开;
    (c)
    按列 \(m+1\), \(m+2\), \(\dots\), \(m+n\) 展开.
    总之, 若您没错, 您可算出,
    \begin{align*}
        \det {(J)}
        = \det {(J_n)}
        = \det {(I_m)} \det {(BA)}
        = \det {(BA)}.
    \end{align*}

    现在, 我们比较这二次计算的结果.
    这就是 Binet--Cauchy 公式.
\end{proof}

我还想说一些话.

设 \(A\), \(B\) 分别是 \(s \times n\) 与 \(m \times s\)~阵.
不难看出, \(BA\) 有意义, 且 \(BA\) 是一个 \(m \times n\)~阵.
\(BA\) 可能不是一个方阵, 故说 \(BA\) 的行列式可能是无意义的.
但是, 我们仍可说 \(BA\)~的子阵的行列式, 若这个子阵是正方的.
设 \(k\) 是一个既不超过 \(m\), 也不超过 \(n\) 的正整数.
设 \(1 \leq i_1 < \dots < i_k \leq m\);
设 \(1 \leq \ell_1 < \dots < \ell_k \leq n\).
我们说, 我们可以用 \(B\) 与 \(A\) 的 \(k\)~级子阵的行列式%
写出 \(BA\)~的 \(k\)~级子阵
\begin{align*}
    F = (BA)\binom{i_1,\dots,i_k}{\ell_1,\dots,\ell_k}
\end{align*}
的行列式.

首先, 我们注意到
\begin{align*}
    (BA)\binom{i_1,\dots,i_k}{\ell_1,\dots,\ell_k}
    =
    B\binom{i_1,\dots,i_k}{1,\dots,s}
    \,
    A\binom{1,\dots,s}{\ell_1,\dots,\ell_k}.
\end{align*}
为说明此事, 我们设
\begin{align*}
    D = B\binom{i_1,\dots,i_k}{1,\dots,s},
    \quad
    C = A\binom{1,\dots,s}{\ell_1,\dots,\ell_k}.
\end{align*}
则 \(D\), \(C\) 分别是 \(k \times s\), \(s \times k\)~阵,
且 \([D]_{u,j} = [B]_{i_u,j}\),
\([C]_{i,v} = [A]_{i,\ell_v}\).
则
\begin{align*}
    [DC]_{u,v}
    = {} &
    \sum_{p = 1}^{s} {[D]_{u,j} [C]_{j,v}}
    \\
    = {} &
    \sum_{p = 1}^{s} {[B]_{i_u,j} [A]_{j,\ell_v}}
    \\
    = {} &
    [BA]_{i_u,\ell_v}
    \\
    = {} &
    [F]_{u,v}.
\end{align*}

现在, 我们可以对 \(C\), \(D\) 用 Binet--Cauchy 公式了.
我再说一次, \(C\) 是 \(s \times k\)~的,
且 \(D\) 是 \(k \times s\)~的.
于是:

(1)
若 \(k > s\), 则 \(\det {(DC)} = 0\);

(2)
若 \(k \leq s\), 则
\begin{align*}
         & \det {(DC)}
    \\
    = {} & \sum_{1 \leq j_1 < \dots < j_k \leq s}
    {
        \det {\left(
            D\binom{1, \dots, k}{j_1, \dots, j_k}
            \right)}
        \det {\left(
            C\binom{j_1, \dots, j_k}{1, \dots, k}
            \right)}
    }
    \\
    = {} & \sum_{1 \leq j_1 < \dots < j_k \leq s}
    {
        \det {\left(
            B\binom{i_1, \dots, i_k}{j_1, \dots, j_k}
            \right)}
        \det {\left(
            A\binom{j_1, \dots, j_k}{\ell_1, \dots, \ell_k}
            \right)}
    }.
\end{align*}

综上, 我们有

\begin{theorem}
    设 \(A\), \(B\) 分别是 \(s \times n\) 与 \(m \times s\)~阵.
    设 \(k\) 是一个既不超过 \(m\), 也不超过 \(n\) 的正整数.
    设 \(1 \leq i_1 < \dots < i_k \leq m\);
    设 \(1 \leq \ell_1 < \dots < \ell_k \leq n\).

    (1)
    若 \(k > s\), 则
    \begin{align*}
        \det {\left(
            (BA)\binom{i_1,\dots,i_k}{\ell_1,\dots,\ell_k}
            \right)} = 0;
    \end{align*}

    (2)
    若 \(k \leq s\), 则
    \begin{align*}
             & \det {\left(
            (BA)\binom{i_1,\dots,i_k}{\ell_1,\dots,\ell_k}
            \right)}
        \\
        = {} & \sum_{1 \leq j_1 < \dots < j_k \leq s}
        {
            \det {\left(
                B\binom{i_1, \dots, i_k}{j_1, \dots, j_k}
                \right)}
            \det {\left(
                A\binom{j_1, \dots, j_k}{\ell_1, \dots, \ell_k}
                \right)}
        }.
    \end{align*}
\end{theorem}

可视此事为 Binet--Cauchy 公式的一个推广:
若 \(m = n\), \(k = n\),
且
\(i_u = \ell_u = u\) (\(u = 1\), \(2\), \(\dots\), \(n\)),
则这就是 Binet--Cauchy 公式.

\vspace{2ex}

若我们取 \(m = n = s\), 则

\begin{theorem}
    设 \(A\), \(B\) 是 \(n\)~级阵.
    设 \(k\) 是一个不超过 \(n\) 的正整数.
    设 \(1 \leq i_1 < \dots < i_k \leq n\);
    设 \(1 \leq \ell_1 < \dots < \ell_k \leq n\).
    则
    \begin{align*}
             & \det {\left(
            (BA)\binom{i_1,\dots,i_k}{\ell_1,\dots,\ell_k}
            \right)}
        \\
        = {} & \sum_{1 \leq j_1 < \dots < j_k \leq n}
        {
            \det {\left(
                B\binom{i_1, \dots, i_k}{j_1, \dots, j_k}
                \right)}
            \det {\left(
                A\binom{j_1, \dots, j_k}{\ell_1, \dots, \ell_k}
                \right)}
        }.
    \end{align*}
\end{theorem}

我们知道, \(A\) 的一个子阵不但可被认为是%
取 \(A\) 的一些行与一些列交叉处的元%
按原来的次序作成的新阵,
还可被认为是%
去除 \(A\) 的一些行与一些列后,
剩下的元按原来的次序作成的阵.
所以, 对称地, 我们当然可如此改写前面的结论:

\begin{theorem}
    设 \(A\), \(B\) 是 \(n\)~级阵.
    设 \(k\) 是一个不超过 \(n\) 的正整数.
    设 \(1 \leq i_1 < \dots < i_k \leq n\);
    设 \(1 \leq \ell_1 < \dots < \ell_k \leq n\).
    则
    \begin{align*}
             & \det {(
        (BA)({i_1,\dots,i_k}|{\ell_1,\dots,\ell_k})
        )}
        \\
        = {} & \sum_{1 \leq j_1 < \dots < j_k \leq n}
        {
        \det {(
        B({i_1, \dots, i_k}|{j_1, \dots, j_k})
        )}
        \det {(
        A({j_1, \dots, j_k}|{\ell_1, \dots, \ell_k})
        )}
        }.
    \end{align*}
\end{theorem}

% Stories about properties of determinants
\section{按一行展开行列式}

本节, 我想用定义 (按列~\(1\) 展开),
证明按一行展开行列式的公式.

\begin{theorem}
    设 \(A\) 是 \(n\)~级阵 (\(n \geq 1\)).
    设 \(i\) 为整数, 且 \(1 \leq i \leq n\).
    则
    \begin{align*}
        \det {(A)} = \sum_{j = 1}^{n}
        {(-1)^{i+j} [A]_{i,j} \det {(A(i|j))}}.
    \end{align*}
\end{theorem}

\begin{proof}
    我们用数学归纳法证明此事.
    具体地, 设 \(P(n)\) 为命题
    \begin{quotation}
        对任何 \(n\)~级阵 \(A\),
        对任何不超过 \(n\) 的正整数 \(i\),
        \begin{align*}
            \det {(A)} = \sum_{j = 1}^{n}
            {(-1)^{i+j} [A]_{i,j} \det {(A(i|j))}}.
        \end{align*}
    \end{quotation}
    则, 我们的目标是:
    对任何正整数 \(n\), \(P(n)\) 是正确的.

    \(P(1)\) 显然是正确的.

    可以验证, \(P(2)\) 也是正确的 (习题).

    现在, 我们假定 \(P(m-1)\) 是正确的.
    我们要证 \(P(m)\) 也是正确的.
    任取一个 \(m\)~级阵 \(A\).
    任取不超过 \(m\) 的正整数 \(i\).
    为方便, 我们记
    \(f = (-1)^{i+1} [A]_{i,1} \det {(A(i|1))}\).
    则 (第~2~个等号利用了假定,
    并注意到,
    \(A\)~的行~\(i\) (其中 \(k \neq i\)) 对应
    \(A(k|1)\)~的行~\(i - \rho(i,k)\),
    \(A\)~的列~\(j\) (其中 \(j \neq 1\)) 对应
    \(A(k|1)\)~的列~\(j - 1\))
    \begin{align*}
             & \det {(A)}
        \\
        = {} &
        f + \sum_{\substack{1 \leq k \leq m \\k \neq i}}
        {(-1)^{k+1} [A]_{k,1} \det {(A(k|1))}}
        \\
        = {} &
        f + \sum_{\substack{1 \leq k \leq m \\k \neq i}}
        {(-1)^{k+1} [A]_{k,1}
        \sum_{j = 2}^{m}
        {(-1)^{i - \rho(i,k) + j - 1}
            [A]_{i,j} \det {(A(k,i|1,j))}}}
        \\
        = {} &
        f + \sum_{\substack{1 \leq k \leq m \\k \neq i}}
        {\sum_{j = 2}^{m}
        {(-1)^{k+1} [A]_{k,1}
        (-1)^{i - \rho(i,k) + j - 1}
            [A]_{i,j} \det {(A(k,i|1,j))}}}
        \\
        = {} &
        f + \sum_{j = 2}^{m}
        {\sum_{\substack{1 \leq k \leq m    \\k \neq i}}
        {(-1)^{k+1} [A]_{k,1}
            (-1)^{i - \rho(i,k) + j - 1}
                [A]_{i,j} \det {(A(k,i|1,j))}}}
        \\
        = {} &
        f + \sum_{j = 2}^{m}
        {\sum_{\substack{1 \leq k \leq m    \\k \neq i}}
        {(-1)^{i+j} [A]_{i,j}
            (-1)^{k - \rho(k,i) + 1}
                [A]_{k,1} \det {(A(k,i|1,j))}}}
        \\
        = {} &
        f + \sum_{j = 2}^{m}
        {(-1)^{i+j} [A]_{i,j}
        \sum_{\substack{1 \leq k \leq m     \\k \neq i}}
        {(-1)^{k - \rho(k,i) + 1}
                [A]_{k,1} \det {(A(k,i|1,j))}}}
        \\
        = {} &
        f + \sum_{j = 2}^{m}
        {(-1)^{i+j} [A]_{i,j} \det {(A(i|j))}}.
    \end{align*}
    所以, \(P(m)\) 是正确的.
    由数学归纳法原理, 待证命题成立.
\end{proof}

\section{方阵与其转置的行列式相等}

本节, 我想证明: 一个方阵与其转置的行列式相等.

其实, 在第一章, 节~\sekcio{15},
我已用定义 (按列~\(1\) 展开)
\emph{与}按行~\(1\) 展开证明了它.
不过, 我想, 知道别的证明也好.

% 我先引用定理:

% \begin{theorem}
%     设 \(A\) 是 \(n\)~级阵 (\(n \geq 1\)).
%     则 \(A\)~的转置, \(A^{\mathrm{T}}\),
%     的行列式等于 \(A\)~的行列式.
% \end{theorem}

以下, 设 \(P(n)\) 为命题
\begin{quotation}
    对\emph{任何} \(n\)~级阵 \(A\),
    \begin{align*}
        \det {(A^{\mathrm{T}})} = \det {(A)}.
    \end{align*}
\end{quotation}
则, 我们的目标是:
对任何正整数 \(n\), \(P(n)\) 是正确的.

\begin{proof}[同时用按列~\(1\) 展开与按行~\(1\) 展开]
    见第一章, 节~\sekcio{15}.
\end{proof}

\begin{proof}[用按一列展开]
    \(P(1)\) 是正确的.
    毕竟, \(1\)~级阵的转置就是自己.

    现在, 我们假定 \(P(m-1)\) 是正确的.
    我们要证 \(P(m)\) 也是正确的.

    任取一个 \(m\)~级阵 \(A\).
    一方面, 我们知道,
    对任何数 \(x\), 与任何正整数 \(s\),
    必有
    \begin{align*}
        x = \frac{1}{s} \sum_{j = 1}^{s} {x}.
    \end{align*}
    另一方面, 我们已知,
    对每一个 \(s\) 级阵 \(B\),
    与每一个不超过 \(s\) 的正整数 \(j\),
    \begin{align*}
        \det {(B)}
        = \sum_{i = 1}^{s} {(-1)^{i+j} [B]_{i,j} \det {(B(i|j))}}.
    \end{align*}
    最后, 注意到, 既然
    \([A]_{i,j} = [A^{\mathrm{T}}]_{j,i}\),
    则, 不难验证,
    \(A^{\mathrm{T}}(i|j) = (A(j|i))^{\mathrm{T}}\).
    结合这三件事,
    与用过多次的加法的结合律与交换律,
    并利用假定 (第~4~个等号), 我们有
    \begin{align*}
             & \det {(A^{\mathrm{T}})}
        \\
        = {} &
        \frac{1}{m}
        \sum_{j=1}^{m}
        {\det {(A^{\mathrm{T}})}}
        \\
        = {} &
        \frac{1}{m}
        \sum_{j=1}^{m} {
        \sum_{i=1}^{m} {
        (-1)^{i+j} [A^{\mathrm{T}}]_{i,j}
        \det {(A^{\mathrm{T}}(i|j))}}}
        \\
        = {} &
        \frac{1}{m}
        \sum_{j=1}^{m} {
        \sum_{i=1}^{m} {
        (-1)^{i+j} [A]_{j,i}
        \det {((A(j|i))^{\mathrm{T}})}}}
        \\
        = {} &
        \frac{1}{m}
        \sum_{j=1}^{m} {
        \sum_{i=1}^{m} {
        (-1)^{i+j} [A]_{j,i}
        \det {(A(j|i))}}}
        \\
        = {} &
        \frac{1}{m}
        \sum_{j=1}^{m} {
        \sum_{i=1}^{m} {
        (-1)^{j+i} [A]_{j,i}
        \det {(A(j|i))}}}
        \\
        = {} &
        \frac{1}{m}
        \sum_{i=1}^{m} {
        \sum_{j=1}^{m} {
        (-1)^{j+i} [A]_{j,i}
        \det {(A(j|i))}}}
        \\
        = {} &
        \frac{1}{m}
        \sum_{i=1}^{m}
        {\det {(A)}}
        \\
        = {} & \det {(A)}.
    \end{align*}
    所以, \(P(m)\) 是正确的.
    由数学归纳法原理, 待证命题成立.
\end{proof}

\begin{proof}[用按列~\(1\) 展开]
    作辅助命题 \(Q(n)\):
    \begin{quotation}
        \(P(n-1)\) 与 \(P(n)\) 是正确的.
    \end{quotation}
    我们用数学归纳法证明:
    对任何高于 \(1\) 的整数 \(n\), \(Q(n)\) 是正确的.
    由此, 我们可知, 对任何正整数 \(n\),
    \(P(n)\) 是正确的.

    不难验证 \(Q(2)\) 是正确的.

    现在, 我们假定 \(Q(m-1)\) 是正确的.
    我们要证 \(Q(m)\) 也是正确的.
    既然 \(Q(m-1)\) 是正确的,
    则 \(P(m-2)\) 与 \(P(m-1)\) 是正确的.
    所以, 若我们能由此证明 \(P(m)\) 是正确的,
    则 \(Q(m)\) 是正确的.

    任取一个 \(m\)~级阵 \(A\).
    则
    \begin{align*}
             & \det {(A^{\mathrm{T}})}
        \\
        = {} &
        \sum_{i = 1}^{m} {
        (-1)^{i+1} [A^{\mathrm{T}}]_{i,1}
        \det {(A^{\mathrm{T}}(i|1))}
        }
        \\
        = {} &
        \hphantom{{} + {}}
        (-1)^{1+1} [A^{\mathrm{T}}]_{1,1}
        \det {(A^{\mathrm{T}}(1|1))}
        \\
             &
        + \sum_{i = 2}^{m} {
        (-1)^{i+1} [A^{\mathrm{T}}]_{i,1}
        \det {(A^{\mathrm{T}}(i|1))}
        }
        \\
        = {} &
        \hphantom{{} + {}}
        (-1)^{1+1} [A]_{1,1} \det {(A(1|1))}
        \tag*{(1)}
        \\
             &
        + \sum_{i = 2}^{m} {
        (-1)^{i+1} [A^{\mathrm{T}}]_{i,1}
        \det {(A(1|i))}
        }
        \tag*{(2)}
        \\
        = {} &
        \hphantom{{} + {}}
        (-1)^{1+1} [A]_{1,1} \det {(A(1|1))}
        \\
             &
        + \sum_{i = 2}^{m} {
        (-1)^{i+1} [A^{\mathrm{T}}]_{i,1}
        \sum_{k = 2}^{m} {
        (-1)^{k-1+1} [A]_{k,1} \det {(A({1,k}|{i,1}))}
        }
        }
        \tag*{(3)}
        \\
        = {} &
        \hphantom{{} + {}}
        (-1)^{1+1} [A]_{1,1} \det {(A(1|1))}
        \\
             &
        + \sum_{i = 2}^{m} {
        \sum_{k = 2}^{m} {
        (-1)^{i+1} [A^{\mathrm{T}}]_{i,1}
        (-1)^{k-1+1} [A]_{k,1} \det {(A({1,k}|{i,1}))}
        }
        }
        \\
        = {} &
        \hphantom{{} + {}}
        (-1)^{1+1} [A]_{1,1} \det {(A(1|1))}
        \\
             &
        + \sum_{k = 2}^{m} {
        \sum_{i = 2}^{m} {
        (-1)^{i+1} [A^{\mathrm{T}}]_{i,1}
        (-1)^{k-1+1} [A]_{k,1} \det {(A({1,k}|{i,1}))}
        }
        }
        \\
        = {} &
        \hphantom{{} + {}}
        (-1)^{1+1} [A]_{1,1} \det {(A(1|1))}
        \\
             &
        + \sum_{k = 2}^{m} {
        \sum_{i = 2}^{m} {
        (-1)^{k+1} [A]_{k,1}
        (-1)^{i-1+1} [A^{\mathrm{T}}]_{i,1}
        \det {(A({1,k}|{i,1}))}
        }
        }
        \\
        = {} &
        \hphantom{{} + {}}
        (-1)^{1+1} [A]_{1,1} \det {(A(1|1))}
        \\
             &
        + \sum_{k = 2}^{m} {
        (-1)^{k+1} [A]_{k,1}
        \sum_{i = 2}^{m} {
        (-1)^{i-1+1} [A^{\mathrm{T}}]_{i,1}
        \det {(A({1,k}|{i,1}))}
        }
        }
        \\
        = {} &
        \hphantom{{} + {}}
        (-1)^{1+1} [A]_{1,1} \det {(A(1|1))}
        \\
             &
        + \sum_{k = 2}^{m} {
        (-1)^{k+1} [A]_{k,1}
        \sum_{i = 2}^{m} {
        (-1)^{i-1+1} [A^{\mathrm{T}}]_{i,1}
        \det {(A^{\mathrm{T}} ({i,1}|{1,k}))}
        }
        }
        \tag*{(4)}
        \\
        = {} &
        \hphantom{{} + {}}
        (-1)^{1+1} [A]_{1,1} \det {(A(1|1))}
        \\
             &
        + \sum_{k = 2}^{m} {
        (-1)^{k+1} [A]_{k,1}
        \det {(A^{\mathrm{T}}(1|k))}
        }
        \tag*{(5)}
        \\
        = {} &
        \hphantom{{} + {}}
        (-1)^{1+1} [A]_{1,1} \det {(A(1|1))}
        \\
             &
        + \sum_{k = 2}^{m} {
        (-1)^{k+1} [A]_{k,1}
        \det {(A(k|1))}
        }
        \tag*{(6)}
        \\
        = {} &
        \sum_{k = 1}^{m} {
        (-1)^{k+1} [A]_{k,1}
        \det {(A(k|1))}
        }
        \\
        = {} &
        \det {(A)}.
    \end{align*}
    所以, \(P(m)\) 是正确的.
    所以, \(Q(m)\) 是正确的.
    由数学归纳法原理, 待证命题成立.

    我简单地作一些解释吧.

    (1) (2)
    都利用了假定 \(P(m-1)\).
    注意, \(A^{\mathrm{T}} (i|1)\)
    的转置是 \(A(1|i)\).

    (3)
    利用了按列~\(1\) 展开.
    注意, \([A]_{k,1}\) 是 \(A(1|i)\)~的
    \((k-1, 1)\)-元
    (\(i, k > 1\)).

    (4)
    利用了假定 \(P(m-2)\).
    注意, \(A({1,k}|{i,1})\)
    的转置是 \(A^{\mathrm{T}} ({i,1}|{1,k})\).

    (5)
    利用了按列~\(1\) 展开.
    注意, \([A^{\mathrm{T}}]_{i,1}\) 是
    \(A^{\mathrm{T}} (1|k)\)~的
    \((i-1, 1)\)-元
    (\(i, k > 1\)).

    (6)
    利用了假定 \(P(m-1)\).
    注意, \(A^{\mathrm{T}} (1|k)\)
    的转置是 \(A(k|1)\).
\end{proof}

\section{(关于列的) 多线性}

本节, 我想证明 (关于列的) 多线性.

其实, 在第一章, 节~\sekcio{13},
我已用按一列展开行列式的公式证明了它.
不过, 我想, 知道别的证明也好.

% 我先引用定理:

% \begin{theorem}
%     行列式 (关于列) 是多线性的.
%     具体地, 对任何不超过 \(n\) 的正整数 \(j\),
%     任何 \(n-1\)~个 \(n \times 1\)~阵
%     \(a_1\), \(\dots\), \(a_{j-1}\),
%     \(a_{j+1}\), \(\dots\), \(a_n\),
%     任何二个 \(n \times 1\)~阵 \(x\), \(y\),
%     任何二个数 \(s\), \(t\),
%     有
%     \begin{align*}
%              & \det
%         {[a_1, \dots, a_{j-1}, sx + ty, a_{j+1}, \dots, a_n]}
%         \\
%         = {} &
%         s
%         \det {[a_1, \dots, a_{j-1}, x, a_{j+1}, \dots, a_n]}
%         +
%         t
%         \det {[a_1, \dots, a_{j-1}, y, a_{j+1}, \dots, a_n]}.
%     \end{align*}
%     (若 \(j = 1\), 则 \(a_1\) 不出现;
%     若 \(j = n\), 则 \(a_n\) 不出现.
%     下同.)
% \end{theorem}

以下, 设 \(P(n)\) 为命题
\begin{quotation}
    对任何不超过 \(n\) 的正整数 \(j\),
    任何 \(n-1\)~个 \(n \times 1\)~阵
    \(a_1\), \(\dots\), \(a_{j-1}\),
    \(a_{j+1}\), \(\dots\), \(a_n\),
    任何二个 \(n \times 1\)~阵 \(x\), \(y\),
    任何二个数 \(s\), \(t\),
    有
    \begin{align*}
             & \det
        {[a_1, \dots, a_{j-1}, sx + ty, a_{j+1}, \dots, a_n]}
        \\
        = {} &
        s
        \det {[a_1, \dots, a_{j-1}, x, a_{j+1}, \dots, a_n]}
        +
        t
        \det {[a_1, \dots, a_{j-1}, y, a_{j+1}, \dots, a_n]}.
    \end{align*}
\end{quotation}
则, 我们的目标是:
对任何正整数 \(n\), \(P(n)\) 是正确的.

\begin{proof}[用按一列展开]
    见第一章, 节~\sekcio{13}.
\end{proof}

\begin{proof}[用按列~\(1\) 展开]
    不难验证 \(P(1)\) 是正确的.

    现在, 我们假定 \(P(m-1)\) 是正确的.
    我们要证 \(P(m)\) 也是正确的.

    任取不超过 \(m\) 的正整数 \(j\).
    任取 \(m-1\)~个 \(m \times 1\)~阵
    \(a_1\), \(\dots\), \(a_{j-1}\),
    \(a_{j+1}\), \(\dots\), \(a_m\).
    任取二个 \(m \times 1\)~阵 \(x\), \(y\),
    任取二个数 \(s\), \(t\).
    作三个 \(m\)~级阵 \(A\), \(B\), \(C\):
    \(A\), \(B\), \(C\)~的列~\(k\) 为 \(a_k\) (\(k \neq j\));
    \(A\)~的列~\(j\) 为 \(x\);
    \(B\)~的列~\(j\) 为 \(y\);
    \(C\)~的列~\(j\) 为 \(sx + ty\).
    那么,
    \begin{align*}
         & \det {(A)}
        = \det
        {[a_1, \dots, a_{j-1}, x, a_{j+1}, \dots, a_m]},
        \\
         & \det {(B)}
        = \det
        {[a_1, \dots, a_{j-1}, y, a_{j+1}, \dots, a_m]},
        \\
         & \det {(C)}
        = \det
        {[a_1, \dots, a_{j-1}, sx + ty, a_{j+1}, \dots, a_m]}.
    \end{align*}
    不难发现, 若 \(k \neq j\), 则
    \(
    [A]_{i,k} = [B]_{i,k} = [C]_{i,k},
    \)
    故
    \(
    A(i|j) = B(i|j) = C(i|j).
    \)

    设 \(j = 1\).
    那么
    \begin{align*}
             & \det {(C)}
        \\
        = {} &
        \sum_{i = 1}^{m} {
        (-1)^{i+j} [C]_{i,j} \det {(C(i|j))}
        }
        \\
        = {} &
        \sum_{i = 1}^{m} {
        (-1)^{i+j} (s[A]_{i,j} + t[B]_{i,j}) \det {(C(i|j))}
        }
        \\
        = {} &
        s\sum_{i = 1}^{m} {
        (-1)^{i+j} [A]_{i,j} \det {(C(i|j))}
        }
        +
        t\sum_{i = 1}^{m} {
        (-1)^{i+j} [B]_{i,j} \det {(C(i|j))}
        }
        \\
        = {} &
        s\sum_{i = 1}^{m} {
        (-1)^{i+j} [A]_{i,j} \det {(A(i|j))}
        }
        +
        t\sum_{i = 1}^{m} {
        (-1)^{i+j} [B]_{i,j} \det {(B(i|j))}
        }
        \\
        = {} &
        s \det{(A)} + t \det{(B)}.
    \end{align*}

    设 \(j > 1\).
    设 \(A(i|1)\), \(B(i|1)\), \(C(i|1)\)
    的列~\(j-1\) 分别是 \(p\), \(q\), \(r\).
    则 \(r = sp + tq\).
    由假定,
    \(\det {(C(i|1))} = s\det {(A(i|1))} + t\det {(B(i|1))}\).
    从而
    \begin{align*}
             & \det {(C)}
        \\
        = {} &
        \sum_{i = 1}^{m}
        {(-1)^{i+1} [C]_{i,1} \det {(C(i|1))}}
        \\
        = {} &
        \sum_{i = 1}^{m}
        {(-1)^{i+1} [C]_{i,1}
        (s\det {(A(i|1))} + t\det {(B(i|1))})}
        \\
        = {} &
        s \sum_{i = 1}^{m}
        {(-1)^{i+1} [C]_{i,1} \det {(A(i|1))}}
        +
        t \sum_{i = 1}^{m}
        {(-1)^{i+1} [C]_{i,1} \det {(B(i|1))}}
        \\
        = {} &
        s \sum_{i = 1}^{m}
        {(-1)^{i+1} [A]_{i,1} \det {(A(i|1))}}
        +
        t \sum_{i = 1}^{m}
        {(-1)^{i+1} [B]_{i,1} \det {(B(i|1))}}
        \\
        = {} &
        s \det {(A)} + t \det {(B)}.
    \end{align*}
    所以, \(P(m)\) 是正确的.
    由数学归纳法原理, 待证命题成立.
\end{proof}

\begin{proof}[用按行~\(1\) 展开]
    不难验证 \(P(1)\) 是正确的.

    现在, 我们假定 \(P(m-1)\) 是正确的.
    我们要证 \(P(m)\) 也是正确的.

    任取不超过 \(m\) 的正整数 \(j\).
    任取 \(m-1\)~个 \(m \times 1\)~阵
    \(a_1\), \(\dots\), \(a_{j-1}\),
    \(a_{j+1}\), \(\dots\), \(a_m\).
    任取二个 \(m \times 1\)~阵 \(x\), \(y\),
    任取二个数 \(s\), \(t\).
    作三个 \(m\)~级阵 \(A\), \(B\), \(C\):
    \(A\), \(B\), \(C\)~的列~\(k\) 为 \(a_k\) (\(k \neq j\));
    \(A\)~的列~\(j\) 为 \(x\);
    \(B\)~的列~\(j\) 为 \(y\);
    \(C\)~的列~\(j\) 为 \(sx + ty\).
    那么,
    \begin{align*}
         & \det {(A)}
        = \det
        {[a_1, \dots, a_{j-1}, x, a_{j+1}, \dots, a_m]},
        \\
         & \det {(B)}
        = \det
        {[a_1, \dots, a_{j-1}, y, a_{j+1}, \dots, a_m]},
        \\
         & \det {(C)}
        = \det
        {[a_1, \dots, a_{j-1}, sx + ty, a_{j+1}, \dots, a_m]}.
    \end{align*}

    一方面, 若 \(k = j\), 则
    \([C]_{1,k} = x[A]_{1,k} + y[B]_{1,k}\),
    且
    \(A(1|k) = B(1|k) = C(1|k)\).
    另一方面, 若 \(k \neq j\),
    则
    \([A]_{i,k} = [B]_{i,k} = [C]_{i,k}\).
    设 \(A(1|k)\), \(B(1|k)\), \(C(1|k)\)
    的列~\(j - \rho(j, k)\)
    分别是
    \(p\), \(q\), \(r\).
    则
    \(r = sp + tq\).
    由假定,
    \(\det {(C(1|k))} = s \det {(A(1|k))} + t \det {(B(1|k))}\).
    从而
    \begin{align*}
             & \det {(C)}
        \\
        = {} &
        \sum_{k = 1}^{m}
        {(-1)^{1 + k} [C]_{1,k} \det {(C(1|k))}}
        \\
        = {} &
        (-1)^{1 + j} [C]_{1,j} \det {(C(1|j))}
        +
        \sum_{\substack{1 \leq k \leq m     \\k \neq j}}
        {(-1)^{1 + k} [C]_{1,k} \det {(C(1|k))}}
        \\
        = {} &
        \hphantom{{} + {}}
        (-1)^{1 + j} (s[A]_{1,j} + t[B]_{1,j}) \det {(C(1|j))}
        \\
             &
        +
        \sum_{\substack{1 \leq k \leq m     \\k \neq j}}
        {(-1)^{1 + k} [C]_{1,k}
            (s \det {(A(1|k))} + t \det {(B(1|k))})}
        \\
        = {} &
        \hphantom{{} + {}}
        s (-1)^{1 + j} [A]_{1,j} \det {(C(1|j))}
        + t (-1)^{1 + j} [B]_{1,j} \det {(C(1|j))}
        \\
             &
        + s \sum_{\substack{1 \leq k \leq m \\k \neq j}}
        {(-1)^{1 + k} [C]_{1,k} \det {(A(1|k))}}
        \\
             &
        + t \sum_{\substack{1 \leq k \leq m \\k \neq j}}
        {(-1)^{1 + k} [C]_{1,k} \det {(B(1|k))}}
        \\
        = {} &
        \hphantom{{} + {}}
        s (-1)^{1 + j} [A]_{1,j} \det {(A(1|j))}
        + t (-1)^{1 + j} [B]_{1,j} \det {(B(1|j))}
        \\
             &
        + s \sum_{\substack{1 \leq k \leq m \\k \neq j}}
        {(-1)^{1 + k} [A]_{1,k} \det {(A(1|k))}}
        \\
             &
        + t \sum_{\substack{1 \leq k \leq m \\k \neq j}}
        {(-1)^{1 + k} [B]_{1,k} \det {(B(1|k))}}
        \\
        = {} &
        \hphantom{{} + {}}
        s (-1)^{1 + j} [A]_{1,j} \det {(A(1|j))}
        + s \sum_{\substack{1 \leq k \leq m \\k \neq j}}
        {(-1)^{1 + k} [A]_{1,k} \det {(A(1|k))}}
        \\
             &
        + t (-1)^{1 + j} [B]_{1,j} \det {(B(1|j))}
        + t \sum_{\substack{1 \leq k \leq m \\k \neq j}}
        {(-1)^{1 + k} [B]_{1,k} \det {(B(1|k))}}
        \\
        = {} &
        s \sum_{k = 1}^{m}
        {(-1)^{1 + k} [A]_{1,k} \det {(A(1|k))}}
        + t \sum_{k = 1}^{m}
        {(-1)^{1 + k} [B]_{1,k} \det {(B(1|k))}}
        \\
        = {} & s \det {(A)} + t \det {(B)}.
    \end{align*}
    所以, \(P(m)\) 是正确的.
    由数学归纳法原理, 待证命题成立.
\end{proof}

\section{按前二列展开行列式}

本节, 我想证明一个公式.
它在交错性的一个证明%
与反称性的一个证明里有用.

\begin{theorem}
    设 \(A\) 是 \(n\)~级阵 (\(n \geq 2\)).
    则
    \begin{align*}
        \det {(A)}
        = \sum_{1 \leq i < k \leq n}
        {\det {
                \begin{bmatrix}
                    [A]_{i,1} & [A]_{i,2} \\
                    [A]_{k,1} & [A]_{k,2} \\
                \end{bmatrix}
            }
            (-1)^{i+k+1+2}
            \det {(A(i,k|1,2))}}.
    \end{align*}
\end{theorem}

\begin{proof}
    注意到, 当 \(d_2 \neq d_1\) 时,
    \([A]_{d_2,2}\) 是
    \(A(d_1|1)\) 的
    \((d_2 - \rho(d_2, d_1), 1)\)-元.
    从而
    \begin{align*}
             & \det {(A)}
        \\
        = {} &
        \sum_{d_1 = 1}^{n} {(-1)^{d_1+1} [A]_{d_1,1}
        \det {(A(d_1|1))}}
        \\
        = {} &
        \sum_{d_1 = 1}^{n} {(-1)^{d_1+1} [A]_{d_1,1}
        \sum_{\substack{1 \leq d_2 \leq n     \\d_2 \neq d_1}}
        {(-1)^{d_2 - \rho(d_2,d_1) + 1} [A]_{d_2,2}
            \det {(A(d_1,d_2|1,2))}}}
        \\
        = {} &
        \sum_{d_1 = 1}^{n}
        {\sum_{\substack{1 \leq d_2 \leq n    \\d_2 \neq d_1}}
        {(-1)^{d_1+1} [A]_{d_1,1}
            (-1)^{d_2 - \rho(d_2,d_1) + 1} [A]_{d_2,2}
            \det {(A(d_1,d_2|1,2))}}}
        \\
        = {} &
        \sum_{\substack{1 \leq d_1,d_2 \leq n \\d_1 \neq d_2}}
        {(-1)^{d_1+1} [A]_{d_1,1}
            (-1)^{d_2 - \rho(d_2,d_1) + 1} [A]_{d_2,2}
            \det {(A(d_1,d_2|1,2))}}
        \\
        = {} &
        \sum_{\substack{1 \leq d_1,d_2 \leq n \\d_1 \neq d_2}}
        {(-1)^{\rho(d_1,d_2)}
                [A]_{d_1,1} [A]_{d_2,2}
            (-1)^{d_1+d_2+1+2}
            \det {(A(d_1,d_2|1,2))}}
        \\
        = {} &
        \hphantom{{} + {}}
        \sum_{\substack{1 \leq d_1,d_2 \leq n \\d_1 < d_2}}
        {(-1)^{\rho(d_1,d_2)}
                [A]_{d_1,1} [A]_{d_2,2}
            (-1)^{d_1+d_2+1+2}
            \det {(A(d_1,d_2|1,2))}}
        \\
             &
        +
        \sum_{\substack{1 \leq d_1,d_2 \leq n \\d_1 > d_2}}
        {(-1)^{\rho(d_1,d_2)}
                [A]_{d_1,1} [A]_{d_2,2}
            (-1)^{d_1+d_2+1+2}
            \det {(A(d_1,d_2|1,2))}}
        \\
        = {} &
        \hphantom{{} + {}}
        \sum_{\substack{1 \leq i,k \leq n     \\i < k}}
        {(-1)^{\rho(i,k)}
                [A]_{i,1} [A]_{k,2}
            (-1)^{i+k+1+2}
            \det {(A(i,k|1,2))}}
        \\
             &
        +
        \sum_{\substack{1 \leq k,i \leq n     \\k > i}}
        {(-1)^{\rho(k,i)}
                [A]_{k,1} [A]_{i,2}
            (-1)^{k+i+1+2}
            \det {(A(k,i|1,2))}}
        \\
        = {} &
        \sum_{1 \leq i < k \leq n}
        {([A]_{i,1} [A]_{k,2} - [A]_{k,1} [A]_{i,2})\,
        (-1)^{i+k+1+2}
        \det {(A(i,k|1,2))}}
        \\
        = {} &
        \sum_{1 \leq i < k \leq n}
        {\det {
                \begin{bmatrix}
                    [A]_{i,1} & [A]_{i,2} \\
                    [A]_{k,1} & [A]_{k,2} \\
                \end{bmatrix}
            }
            (-1)^{i+k+1+2}
            \det {(A(i,k|1,2))}}.
        \qedhere
    \end{align*}
\end{proof}

\section{(关于列的) 交错性}

本节, 我想证明 (关于列的) 交错性.

其实, 在第一章, 节~\sekcio{13},
我已用按一列展开行列式的公式证明了它.
不过, 我想, 知道别的证明也好.

% 我先引用定理:

% \begin{theorem}
%     行列式 (关于列) 是交错性的.
%     具体地,
%     若 \(n\)~级阵 \(A\) 有二列完全相同,
%     则 \(\det {(A)} = 0\).
% \end{theorem}

以下, 设 \(P(n)\) 为命题
\begin{quotation}
    对每一个有完全相同的二列的 \(n\)~级阵 \(A\),
    其行列式必为零.
\end{quotation}
则, 我们的目标是:
对任何正整数 \(n\), \(P(n)\) 是正确的.

\begin{proof}[用按一列展开]
    见第一章, 节~\sekcio{13}.
\end{proof}

在介绍其他的证明前, 我要介绍二个有用的事实.

\begin{theorem}
    设命题~I:
    若一个方阵有\emph{相邻的}二列相同, 则其行列式为零.
    设命题~II:
    交换方阵 \(A\) 的\emph{相邻的}二列, 得方阵 \(B\),
    则 \(\det {(B)} = -\det {(A)}\).
    则, 用行列式的 (关于列的) 多线性,
    我们可由 I 推出 II.
\end{theorem}

\begin{proof}
    设 \(A\) 是一个 \(n\)~级阵 (\(n \geq 2\)).
    设 \(A\) 的列 \(1\), \(2\), \(\dots\), \(n\)
    为 \(a_1\), \(a_2\), \(\dots\), \(a_n\).
    设 \(1 \leq j < n\).
    交换 \(A\) 的列 \(j\), \(j+1\), 得 \(B\).
    记
    \begin{align*}
        f_j (x, y)
        = \det {[\dots, a_{j-1}, x, y, a_{j+2}, \dots]}.
    \end{align*}
    那么,
    \(f_j (a_j, a_{j+1}) = \det {(A)}\),
    \(f_j (a_{j+1}, a_j) = \det {(B)}\).
    则
    \begin{align*}
        0
        = {} & f_j (a_j + a_{j+1}, a_j + a_{j+1})    \\
        = {} & f_j (a_j, a_j + a_{j+1})
        + f_j (a_{j+1}, a_j + a_{j+1})               \\
        = {} & (f_j (a_j, a_j) + f_j (a_j, a_{j+1}))
        + (f_j (a_{j+1}, a_j)
        + f_j (a_{j+1}, a_{j+1}))                    \\
        = {} & f_j (a_j, a_{j+1})
        + f_j (a_{j+1}, a_j)                         \\
        = {} & \det {(A)} + \det {(B)}.
        \qedhere
    \end{align*}
\end{proof}

\begin{restatable}[]{theorem}{TheoremSwapTwoAdjacentColumns}
    设命题~II:
    交换方阵 \(A\) 的\emph{相邻的}二列, 得方阵 \(B\),
    则 \(\det {(B)} = -\det {(A)}\).
    设命题~III:
    交换方阵 \(A\) 的二列, 得方阵 \(B\),
    则 \(\det {(B)} = -\det {(A)}\).
    则, 我们可由 II 推出 III.
\end{restatable}

\begin{proof}
    设 \(A\) 是 \(n\)~级阵,
    设交换 \(A\)~的列~\(p\), \(q\) 后得到的阵为 \(B\)
    (\(p < q\)).
    我们证明:
    \(\det {(B)} = -\det {(A)}\).

    我们交换 \(A\)~的列~\(p\), \(p+1\), 得阵~\(B_1\).
    则 \(\det {(B_1)} = -\det {(A)}
    (-1)^{1} \det {(A)}\).
    我们交换 \(B_1\)~的列~\(p+1\), \(p+2\), 得阵~\(B_2\).
    则 \(\det {(B_2)} = -\det {(B_1)}
    = (-1)^{2} \det {(A)}\).
    \(\dots \dots\)
    我们交换 \(B_{q-p-1}\)~的列~\(q-1\), \(q\), 得阵~\(B_{q-p}\).
    则 \(\det {(B_{q-p})} = -\det {(B_{q-p-1})}
    = (-1)^{q-p} \det {(A)}\).

    记 \(C_0 = B_{q-p}\).
    则 \(\det {(C_0)} = (-1)^{q-p} \det {(A)}\).
    我们交换 \(C_0\)~的列~\(q-2\), \(q-1\), 得阵~\(C_1\).
    则 \(\det {(C_1)} = -\det {(C_0)}
    = (-1)^{q-p+1} \det {(A)}\).
    我们交换 \(C_1\)~的列~\(q-3\), \(q-2\), 得阵~\(C_2\).
    则 \(\det {(C_2)} = -\det {(C_1)}
    = (-1)^{q-p+2} \det {(A)}\).
    \(\dots \dots\)
    我们交换 \(C_{q-p-2}\)~的列~\(p\), \(p+1\),
    得阵~\(C_{q-p-1}\).
    则 \(\det {(C_{q-p-1})} = -\det {(C_{q-p-2})}
    = (-1)^{q-p+(q-p-1)} \det {(A)}\).
    不难看出, \(B = C_{q-p-1}\).
    所以,
    \(\det {(B)}
    = (-1)^{2(q-p)-1} \det {(A)}
    = -\det {(A)}\).
\end{proof}

好的.
现在, 我介绍交错性的其他的证明.

\begin{proof}[同时用按列~\(1\) 展开与按前二列展开]
    \(P(1)\) 不证自明.

    不难验证 \(P(2)\) 是正确的.

    现在, 我们假定 \(P(m-1)\) 是正确的 (\(m \geq 3\)).
    我们要证 \(P(m)\) 也是正确的.

    任取一个 \(m\)~级阵 \(A\).

    设 \(A\) 的列~\(1\), \(2\) 相同.
    则
    \begin{align*}
        \det {(A)}
        = {} &
        \sum_{1 \leq i < k \leq m}
        {\det {
                \begin{bmatrix}
                    [A]_{i,1} & [A]_{i,2} \\
                    [A]_{k,1} & [A]_{k,2} \\
                \end{bmatrix}
            }
            (-1)^{i+k+1+2}
            \det {(A(i,k|1,2))}}
        \\
        = {} &
        \sum_{1 \leq i < k \leq m}
        {\det {
                \begin{bmatrix}
                    [A]_{i,1} & [A]_{i,1} \\
                    [A]_{k,1} & [A]_{k,1} \\
                \end{bmatrix}
            }
            (-1)^{i+k+1+2}
            \det {(A(i,k|1,2))}}
        \\
        = {} &
        \sum_{1 \leq i < k \leq m}
        {0\, (-1)^{i+k+1+2} \det {(A(i,k|1,2))}}
        \\
        = {} & 0.
    \end{align*}

    设 \(A\) 的列~\(j\), \(j+1\) 相同 (\(1 < j < m\)).
    则
    \begin{align*}
        \det {(A)}
        = \sum_{i = 1}^{m} {(-1)^{i + 1} [A]_{i,1}
        \det {(A(i|1))}}.
    \end{align*}
    注意, \(A(i|1)\) 的列~\(j-1\), \(j\) 相同.
    由假定, \(\det {(A(i|1))} = 0\).
    故 \(\det {(A)} = 0\).

    综上,
    若 \(A\) 的\emph{相邻的}二列相同,
    则其行列式为零.

    现假定 \(A\) 的列 \(p\), \(q\) 相同
    (\(1 \leq p < q \leq m\)).
    若 \(q - p = 1\), 则这是相邻的二列,
    故 \(\det {(A)} = 0\).
    若 \(q - p > 1\), 则交换列 \(p\), \(q-1\),
    得阵~\(B\).
    \(B\) 有相邻的二列相同, 故 \(\det {(B)} = 0\).
    从而, 由 ``有用的事实'',
    \(\det {(A)} = -\det {(B)} = 0\).

    所以, \(P(m)\) 是正确的.
    由数学归纳法原理, 待证命题成立.
\end{proof}

注意到, 因为在按列~\(1\) 展开行列式的公式里,
列~\(1\) 与其他列的地位不一样,
用它证明一个关于二列的性质%
对列~\(1\) 与其他列成立是有些挑战的.
不过, 若我们用按行~\(1\) 展开行列式的公式,
则这是较简单的.

\begin{proof}[用按行~\(1\) 展开]
    \(P(1)\) 不证自明.

    不难验证 \(P(2)\) 是正确的.

    现在, 我们假定 \(P(m-1)\) 是正确的 (\(m \geq 3\)).
    我们要证 \(P(m)\) 也是正确的.

    设 \(A\) 的列~\(j\), \(j+1\) 相同 (\(1 \leq j < m\)).

    注意, \([A]_{1,j} = [A]_{1,j+1}\),
    且 \(A(1|j) = A(1|j+1)\).
    再注意, 若 \(k \neq j\), \(j+1\),
    则 \(A(1|k)\) 有 (相邻的) 二列相同.
    由假定, \(\det {(A(1|k))} = 0\).
    从而
    \begin{align*}
             & \det {(A)}
        \\
        = {} &
        \sum_{k = 1}^{m}
        {(-1)^{1 + k} [A]_{1,k} \det {(A(1|k))}}
        \\
        = {} &
        \hphantom{{} + {}}
        (-1)^{1 + j} [A]_{1,j} \det {(A(1|j))}
        + (-1)^{1 + j+1} [A]_{1,j+1} \det {(A(1|j+1))}
        \\
             & +
        \sum_{\substack{1 \leq k \leq m \\k \neq j, j+1}}
        {(-1)^{1 + k} [A]_{1,k} \det {(A(1|k))}}
        \\
        = {} &
        \hphantom{{} + {}}
        (-1)^{1 + j} [A]_{1,j} \det {(A(1|j))}
        + (-1)^{1 + j+1} [A]_{1,j} \det {(A(1|j))}
        \\
             & +
        \sum_{\substack{1 \leq k \leq m \\k \neq j, j+1}}
        {(-1)^{1 + k} [A]_{1,k}\,0 }
        \\
        = {} &
        (-1)^{1 + j} [A]_{1,j} \det {(A(1|j))}
        - (-1)^{1 + j} [A]_{1,j} \det {(A(1|j))}
        \\
        = {} & 0.
    \end{align*}

    综上,
    若 \(A\) 的\emph{相邻的}二列相同,
    则其行列式为零.

    现假定 \(A\) 的列 \(p\), \(q\) 相同
    (\(1 \leq p < q \leq m\)).
    若 \(q - p = 1\), 则这是相邻的二列,
    故 \(\det {(A)} = 0\).
    若 \(q - p > 1\), 则交换列 \(p\), \(q-1\),
    得阵~\(B\).
    \(B\) 有相邻的二列相同, 故 \(\det {(B)} = 0\).
    从而, 由 ``有用的事实'',
    \(\det {(A)} = -\det {(B)} = 0\).

    所以, \(P(m)\) 是正确的.
    由数学归纳法原理, 待证命题成立.
\end{proof}

\section{(关于列的) 反称性}

本节, 我想证明 (关于列的) 反称性.

其实, 在第一章, 节~\sekcio{13},
我已用多线性与交错性证明了它.
不过, 我想, 知道别的证明也好.

% 我先引用定理:

% \begin{theorem}[反称性]
%     行列式 (关于列) 是反称性的.
%     具体地,
%     设 \(A\) 是 \(n\)~级阵,
%     设交换 \(A\)~的列~\(p\) 与列~\(q\) 后得到的阵为 \(B\)
%     (\(p < q\)).
%     则 \(\det {(B)} = -\det {(A)}\).
%     (通俗地, 交换方阵的二列, 则其行列式变号.)
% \end{theorem}

以下, 设 \(P(n)\) 为命题
\begin{quotation}
    对任何 \(n\)~级阵 \(A\),
    对任何不超过 \(n\) 且高于 \(1\) 的整数 \(q\),
    对任何低于 \(q\) 的正整数 \(p\),
    必 \(\det {(B)} = -\det {(A)}\),
    其中 \(B\) 是交换 \(A\)~的%
    列~\(p\), \(q\) 后得到的阵.
\end{quotation}
则, 我们的目标是:
对任何正整数 \(n\), \(P(n)\) 是正确的.

\begin{proof}[用按一列展开]
    \(P(1)\) 不证自明.

    不难验证 \(P(2)\) 是正确的.

    现在, 我们假定 \(P(m-1)\) 是正确的 (\(m \geq 3\)).
    我们要证 \(P(m)\) 也是正确的.

    任取一个 \(m\)~级阵 \(A\).
    任取一个不超过 \(m\) 且高于 \(1\) 的整数 \(q\).
    任取一个低于 \(q\) 的正整数 \(p\).
    交换 \(A\)~的列~\(p\), \(q\), 得 \(B\).
    在 \(1\), \(2\), \(\dots\), \(m\) 这 \(m\)~个数里,
    我们必定能找到一个数 \(j\),
    它既不等于 \(p\), 也不等于 \(q\).
    则
    \begin{align*}
        \det {(B)}
        = \sum_{i = 1}^{m}
        {(-1)^{i + j} [B]_{i,j} \det {(B(i|j))}}.
    \end{align*}
    注意, \([B]_{i,j} = [A]_{i,j}\).
    再注意, \(B(i|j)\) 可被认为是%
    交换 \(A(i|j)\) 的二列所得到的 \(m-1\)~级阵.
    由假定, \(\det {(B(i|j))} = -\det {(A(i|j))}\).
    从而
    \begin{align*}
        \det {(B)}
        = {} &
        \sum_{i = 1}^{m}
        {(-1)^{i + j} [A]_{i,j} (-\det {(A(i|j))})}
        \\
        = {} &
        {-\sum_{i = 1}^{m}
        {(-1)^{i + j} [A]_{i,j} \det {(A(i|j))}}}
        \\
        = {} &
        {-\det {(A)}}.
    \end{align*}

    所以, \(P(m)\) 是正确的.
    由数学归纳法原理, 待证命题成立.
\end{proof}

在介绍其他的证明前, 我要介绍一个有用的事实.

\TheoremSwapTwoAdjacentColumns*

\begin{proof}
    略.
\end{proof}

好的.
现在, 我介绍反称性的其他的证明.

\begin{proof}[同时用按列~\(1\) 展开与按前二列展开]
    \(P(1)\) 不证自明.

    不难验证 \(P(2)\) 是正确的.

    现在, 我们假定 \(P(m-1)\) 是正确的 (\(m \geq 3\)).
    我们要证 \(P(m)\) 也是正确的.

    任取一个 \(m\)~级阵 \(A\).
    任取一个低于 \(m\) 的正整数 \(j\).
    交换 \(A\)~的列~\(j\), \(j+1\), 得 \(B\).

    设 \(j = 1\).
    注意到, \(i \neq k\) 时,
    \(A(i,k|1,2) = B(i,k|1,2)\).
    则
    \begin{align*}
        \det {(B)}
        = {} &
        \sum_{1 \leq i < k \leq m}
        {\det {
                \begin{bmatrix}
                    [B]_{i,1} & [B]_{i,2} \\
                    [B]_{k,1} & [B]_{k,2} \\
                \end{bmatrix}
            }
            (-1)^{i+k+1+2}
            \det {(B(i,k|1,2))}}
        \\
        = {} &
        \sum_{1 \leq i < k \leq m}
        {\det {
                \begin{bmatrix}
                    [A]_{i,2} & [A]_{i,1} \\
                    [A]_{k,2} & [A]_{k,1} \\
                \end{bmatrix}
            }
            (-1)^{i+k+1+2}
            \det {(A(i,k|1,2))}}
        \\
        = {} &
        \sum_{1 \leq i < k \leq m}
        {\left(-\det {
                \begin{bmatrix}
                    [A]_{i,1} & [A]_{i,2} \\
                    [A]_{k,1} & [A]_{k,2} \\
                \end{bmatrix}
            }\right)
            (-1)^{i+k+1+2}
            \det {(A(i,k|1,2))}}
        \\
        = {} & {-\det {(A)}}.
    \end{align*}

    设 \(j > 1\).
    则
    \begin{align*}
        \det {(B)}
        = \sum_{i = 1}^{m}
        {(-1)^{i + 1} [B]_{i,1} \det {(B(i|1))}}.
    \end{align*}
    注意, \([B]_{i,1} = [A]_{i,1}\).
    再注意, \(B(i|1)\) 可被认为是%
    交换 \(A(i|1)\) 的二列所得到的 \(m-1\)~级阵.
    由假定, \(\det {(B(i|1))} = -\det {(A(i|1))}\).
    从而
    \begin{align*}
        \det {(B)}
        = {} &
        \sum_{i = 1}^{m}
        {(-1)^{i + 1} [A]_{i,1} (-\det {(A(i|1))})}
        \\
        = {} &
        {-\sum_{i = 1}^{m}
        {(-1)^{i + 1} [A]_{i,1} \det {(A(i|1))}}}
        \\
        = {} &
        {-\det {(A)}}.
    \end{align*}

    综上,
    交换方阵的\emph{相邻的}二列,
    则其行列式变号.

    由 ``有用的事实'',
    交换方阵的二列,
    则其行列式变号.

    所以, \(P(m)\) 是正确的.
    由数学归纳法原理, 待证命题成立.
\end{proof}

注意到, 因为在按列~\(1\) 展开行列式的公式里,
列~\(1\) 与其他列的地位不一样,
用它证明一个关于二列的性质%
对列~\(1\) 与其他列成立是有些挑战的.
不过, 若我们用按行~\(1\) 展开行列式的公式,
则这是较简单的.

\begin{proof}[用按行~\(1\) 展开]
    \(P(1)\) 不证自明.

    不难验证 \(P(2)\) 是正确的.

    现在, 我们假定 \(P(m-1)\) 是正确的 (\(m \geq 3\)).
    我们要证 \(P(m)\) 也是正确的.

    任取一个 \(m\)~级阵 \(A\).
    任取一个低于 \(m\) 的正整数 \(j\).
    交换 \(A\)~的列~\(j\), \(j+1\), 得 \(B\).

    注意,
    \([A]_{i,j} = [B]_{i,j+1}\),
    \([A]_{i,j+1} = [B]_{i,j}\).
    再注意, \(k \neq j\), \(j+1\) 时,
    \([A]_{i,k} = [B]_{i,k}\).
    则 \(A(i|j) = B(i|j+1)\),
    \(A(i|j+1) = B(i|j)\).
    若 \(k \neq j\), \(j+1\),
    则 \(B(i|k)\) 可被认为是%
    交换 \(A(i|k)\) 的 (相邻的) 二列%
    所得到的 \(m-1\)~级阵.
    由假定,
    \(\det {(B(i|k))} = -\det {(A(i|k))}\).
    从而
    \begin{align*}
             & \det {(B)}
        \\
        = {} &
        \sum_{k = 1}^{m}
        {(-1)^{1 + k} [B]_{1,k} \det {(B(1|k))}}
        \\
        = {} &
        \hphantom{{} + {}}
        (-1)^{1 + j} [B]_{1,j} \det {(B(1|j))}
        + (-1)^{1 + j+1} [B]_{1,j+1} \det {(B(1|j+1))}
        \\
             & +
        \sum_{\substack{1 \leq k \leq m \\k \neq j, j+1}}
        {(-1)^{1 + k} [B]_{1,k} \det {(B(1|k))}}
        \\
        = {} &
        \hphantom{{} + {}}
        (-1)^{1 + j} [A]_{1,j+1} \det {(A(1|j+1))}
        + (-1)^{1 + j+1} [A]_{1,j} \det {(A(1|j))}
        \\
             & +
        \sum_{\substack{1 \leq k \leq m \\k \neq j, j+1}}
        {(-1)^{1 + k} [A]_{1,k} (-\det {(A(1|k))})}
        \\
        = {} &
        \hphantom{{} + {}}
        {-(-1)^{1 + j+1}} [A]_{1,j+1} \det {(A(1|j+1))}
        - (-1)^{1 + j} [A]_{1,j} \det {(A(1|j))}
        \\
             & -
        \sum_{\substack{1 \leq k \leq m \\k \neq j, j+1}}
        {(-1)^{1 + k} [A]_{1,k} \det {(A(1|k))}}
        \\
        = {} &
        {-\sum_{k = 1}^{m}
        {(-1)^{1 + k} [A]_{1,k} \det {(A(1|k))}}}
        \\
        = {} & {-\det {(A)}}.
    \end{align*}

    综上,
    交换方阵的\emph{相邻的}二列,
    则其行列式变号.

    由 ``有用的事实'',
    交换方阵的二列,
    则其行列式变号.

    所以, \(P(m)\) 是正确的.
    由数学归纳法原理, 待证命题成立.
\end{proof}

\section{反称性的应用}

反称性是有用的.

由行列式的定义, 不难证明,
对任何 \(n-1\)~个 \(n \times 1\)~阵
\(b_2\), \(\dots\), \(b_n\),
任何二个 \(n \times 1\)~阵 \(x\), \(y\),
任何二个数 \(s\), \(t\),
有
\begin{align*}
    \det
    {[sx + ty, b_2, \dots, b_n]}
    = s
    \det {[x, b_2, \dots, b_n]}
    +
    t
    \det {[y, b_2, \dots, b_n]}.
\end{align*}
利用反称性, 当 \(j > 1\) 时,
\begin{align*}
         & \det
    {[a_1, \dots, a_{j-1}, sx + ty, a_{j+1}, \dots, a_n]}
    \\
    = {} & {-\det
            {[sx + ty, \dots, a_{j-1}, a_1, a_{j+1}, \dots, a_n]}}
    \\
    = {} & {-
            (s
            \det {[x, \dots, a_{j-1}, a_1, a_{j+1}, \dots, a_n]}
            +
            t
            \det {[y, \dots, a_{j-1}, a_1, a_{j+1}, \dots, a_n]}
            )
        }
    \\
    = {} &
    s(-\det {[x, \dots, a_{j-1}, a_1, a_{j+1}, \dots, a_n]})
    + t(-\det {[y, \dots, a_{j-1}, a_1, a_{j+1}, \dots, a_n]})
    \\
    = {} &
    s
    \det {[a_1, \dots, a_{j-1}, x, a_{j+1}, \dots, a_n]}
    +
    t
    \det {[a_1, \dots, a_{j-1}, y, a_{j+1}, \dots, a_n]}.
\end{align*}

我们也可由反称性推出交错性.
设方阵 \(A\)~的列~\(p\), \(q\) 是相同的 (\(p < q\)).
我们交换 \(A\)~的列 \(p\), \(q\), 得阵 \(B\).
根据反称性, \(\det {(B)} = -\det {(A)}\).
不过, 因为 \(A\)~的列~\(p\), \(q\) 是相同的,
故 \(B = A\).
所以, \(\det {(A)} = -\det {(A)}\).
由此可知 \(\det {(A)} = 0\).

我们可由反称性得到%
按 (任何) 一列展开行列式的公式.
设 \(A\) 是一个 \(n\)~级阵 (\(n \geq 2\)).
设 \(j \neq 1\).
交换 \(A\) 的列~\(j-1\), \(j\), 得阵~\(B_1\).
交换 \(B_1\) 的列~\(j-2\), \(j-1\), 得阵~\(B_2\).
\(\dots \dots\)
交换 \(B_{j-2}\) 的列~\(1\), \(2\), 得阵~\(B_{j-1}\).
此时, 可以发现,
\(B_{j-1}\)~的列~\(1\) 即为 \(A\)~的列~\(j\)
(即 \([B_{j-1}]_{i,1} = [A]_{i,j}\)),
且 \(B_{j-1}(i|1) = A(i|j)\).
我们作了 \(j-1\)~次 (相邻的) 列的交换,
故
\begin{align*}
    \det {(A)}
    = {} & (-1)^{j-1} \det {(B_{j-1})}
    \\
    = {} & (-1)^{j-1} \sum_{i = 1}^{n}
    {(-1)^{i+1} [B]_{i,1} \det {(B_{j-1}(i|1))}}
    \\
    = {} & \sum_{i = 1}^{n}
    {(-1)^{j-1} (-1)^{i+1} [A]_{i,1} \det {(A(i|j))}}
    \\
    = {} & \sum_{i = 1}^{n}
    {(-1)^{i+j} [A]_{i,j} \det {(A(i|j))}}.
\end{align*}

\section{用行列式的性质确定行列式}

本节, 我想讨论如何用行列式的性质确定行列式.

我们知道, 多线性与交错性可 ``基本确定'' 行列式:

\begin{theorem}
    设定义在全体 \(n\)~级阵上的函数 \(f\) 适合:

    (1)
    (多线性)
    对任何不超过 \(n\) 的正整数 \(j\),
    任何 \(n-1\)~个 \(n \times 1\)~阵
    \(a_1\), \(\dots\), \(a_{j-1}\),
    \(a_{j+1}\), \(\dots\), \(a_n\),
    任何二个 \(n \times 1\)~阵 \(x\), \(y\),
    任何二个数 \(s\), \(t\),
    有
    \begin{align*}
             & f
            {([a_1, \dots, a_{j-1}, sx + ty,
                        a_{j+1}, \dots, a_n])}
        \\
        = {} &
        s
        f {([a_1, \dots, a_{j-1}, x, a_{j+1}, \dots, a_n])}
        +
        t
        f {([a_1, \dots, a_{j-1}, y, a_{j+1}, \dots, a_n])}.
    \end{align*}

    (2)
    (交错性)
    若 \(n\)~级阵 \(A\) 有二列完全相同,
    则 \(f(A) = 0\).

    那么, 对任何 \(n\)~级阵 \(A\),
    \(f(A) = f(I) \det {(A)}\).

    特别地, 若 \(f(I) = 1\) (规范性),
    则 \(f\) 就是行列式.
\end{theorem}

现在, 我要展示一些变体.

\begin{theorem}
    设定义在全体 \(n\)~级阵上的函数 \(f\) 适合:

    (1)
    对任何 \(n-1\)~个 \(n \times 1\)~阵
    \(a_2\), \(a_3\), \(\dots\), \(a_n\),
    任何二个 \(n \times 1\)~阵 \(x\), \(y\),
    任何二个数 \(s\), \(t\),
    有
    \begin{align*}
        f {([sx + ty, a_2, \dots, a_n])}
        =
        s
        f {([x, a_2, \dots, a_n])}
        +
        t
        f {([y, a_2, \dots, a_n])}.
    \end{align*}

    (2)
    (反称性)
    设 \(A\) 是 \(n\)~级阵,
    设交换 \(A\)~的列~\(p\) 与列~\(q\) 后得到的阵为 \(B\)
    (\(p < q\)).
    则 \(f(B) = -f(A)\).

    那么, 对任何 \(n\)~级阵 \(A\),
    \(f(A) = f(I) \det {(A)}\).

    特别地, 若 \(f(I) = 1\) (规范性),
    则 \(f\) 就是行列式.
\end{theorem}

\begin{proof}
    见上节的讨论.
    我们可由 (1) 与反称性,
    得到多线性;
    我们可由反称性,
    得到交错性.
\end{proof}

我们也可代反称性以 ``相邻反称性'':
``设 \(A\) 是 \(n\)~级阵,
设交换 \(A\)~的列~\(p\) 与列~\(p+1\) 后得到的阵为 \(B\)
(\(p < n\)).
则 \(f(B) = -f(A)\).''
毕竟, 相邻反称性可推出反称性.

我们也可代交错性以 ``相邻交错性'':
``若 \(n\)~级阵 \(A\) 有相邻的二列完全相同,
则 \(f(A) = 0\).''
毕竟, 多线性与相邻交错性可推出相邻反称性,
相邻反称性可推出反称性,
且相邻交错性与反称性可推出交错性
(见 ``(关于列的) 交错性'' 的讨论).

接着, 我要展现一个 ``大不一样的'' 变体.
不过, 我要先定义一种行为:

\begin{restatable}[倍加]{definition}{DefinitionMultiplyAndAdd}
    设 \(A\) 是一个 \(m \times n\)~阵.
    设 \(p\), \(q\) 是二个不超过 \(n\)~的正整数,
    \emph{且 \(p \neq q\).}
    设 \(s\) 是一个数.
    作 \(m \times n\)~阵 \(B\), 其中
    \begin{align*}
        [B]_{i,j}
        = \begin{cases}
              [A]_{i,j},              & j \neq q; \\
              [A]_{i,q} + s[A]_{i,p}, & j = q.
          \end{cases}
    \end{align*}
    (通俗地,
    我们加 \(A\)~的列~\(p\) 的 \(s\)~倍于列~\(q\),
    不改变其他的列,
    得阵~\(B\).)
    我们说, 变 \(A\) 为 \(B\) 的行为是一次 (列的) \emph{倍加}.
\end{restatable}

注意到, \(s\) 可以取 \(0\),
而这相当于 \(B = A\).
所以, 我们认为, ``什么都不变'' 也是一次倍加.

利用多线性与交错性, 我们有

\begin{theorem}
    设定义在全体 \(n\)~级阵上的函数 \(f\)
    多线性与交错性.
    则 \(f\) 适合 ``倍加不变性'':
    \begin{quotation}
        设 \(A\) 是一个 \(n\)~级阵.
        设 \(p\), \(q\) 是二个不超过 \(n\)~的正整数,
        \emph{且 \(p \neq q\).}
        设 \(s\) 是一个数.
        作 \(n\)~级阵 \(B\), 其中
        \begin{align*}
            [B]_{i,j}
            = \begin{cases}
                  [A]_{i,j},              & j \neq q; \\
                  [A]_{i,q} + s[A]_{i,p}, & j = q.
              \end{cases}
        \end{align*}
        则 \(f(B) = f(A)\).
    \end{quotation}
\end{theorem}

\begin{proof}
    设 \(A = [a_1, a_2, \dots, a_n]\).

    先设 \(p < q\).
    为方便说话, 我们写
    \begin{align*}
        g(u, v)
        = f {([a_1, \dots, a_{p-1}, u, a_{p+1}, \dots,
                        a_{q-1}, v, a_{q+1}, \dots, a_n])}.
    \end{align*}
    于是, \(g(a_p, a_q)\) 就是 \(f(A)\),
    而 \(g(a_p, a_q + sa_p)\) 就是 \(f(B)\).
    利用多线性与交错性,
    \begin{align*}
        f(B)
        = {} & g(a_p, a_q + sa_p)         \\
        = {} & g(a_p, a_q) + sg(a_p, a_p) \\
        = {} & g(a_p, a_q) + s0           \\
        = {} & g(a_p, a_q)                \\
        = {} & f(A).
    \end{align*}

    再设 \(p > q\).
    为方便说话, 我们写
    \begin{align*}
        h(u, v)
        = f {([a_1, \dots, a_{q-1}, u, a_{q+1}, \dots,
                        a_{p-1}, v, a_{p+1}, \dots, a_n])}.
    \end{align*}
    于是, \(h(a_q, a_p)\) 就是 \(f(A)\),
    而 \(h(a_q + sa_p, a_p)\) 就是 \(f(B)\).
    利用多线性与交错性,
    \begin{align*}
        f(B)
        = {} & h(a_q + sa_p, a_p)         \\
        = {} & h(a_q, a_p) + sh(a_p, a_p) \\
        = {} & h(a_q, a_p) + s0           \\
        = {} & h(a_q, a_p)                \\
        = {} & f(A).
        \qedhere
    \end{align*}
\end{proof}

现在, 我要引出本节的主要结论.

\begin{theorem}
    设定义在全体 \(n\)~级阵上的函数 \(f\) 适合:

    (1)
    倍加不变性.

    (2)
    (多齐性)
    对任何不超过 \(n\) 的正整数 \(j\),
    任何 \(n-1\)~个 \(n \times 1\)~阵
    \(a_1\), \(\dots\), \(a_{j-1}\),
    \(a_{j+1}\), \(\dots\), \(a_n\),
    任何 \(n \times 1\)~阵 \(x\),
    任何数 \(s\),
    有
    \begin{align*}
        f {([a_1, \dots, a_{j-1}, sx, a_{j+1}, \dots, a_n])}
        =
        s
        f {([a_1, \dots, a_{j-1}, x, a_{j+1}, \dots, a_n])}.
    \end{align*}

    那么, 对任何 \(n\)~级阵 \(A\),
    \(f(A) = f(I) \det {(A)}\).

    特别地, 若 \(f(I) = 1\) (规范性),
    则 \(f\) 就是行列式.
\end{theorem}

此事是重要的, 故我会给二个证明.

\vspace{2ex}

不难看出, 倍加不变性与多齐性可推出交错性.
具体地, 设 \(A\) 的列~\(p\), \(q\) 相同,
且 \(p \neq q\).
加 \(A\) 的列~\(q\) 的 \(-1\)~倍于列~\(p\),
得阵~\(B\).
那么, \(B\) 的列~\(p\) 的元全为零.
由多齐性, \(f(B) = 0\).
由倍加不变性, \(f(A) = f(B) = 0\).

倍加不变性与多齐性还可推出反称性.
设 \(p < q\).
记
\begin{align*}
    g(u, v)
    = f {([a_1, \dots, a_{p-1}, u, a_{p+1}, \dots,
                    a_{q-1}, v, a_{q+1}, \dots, a_n])}.
\end{align*}
则
\begin{align*}
    g(a_q, a_p)
    = {} & g(a_q + 1a_p, a_p)                   \\
    = {} & g(a_p + a_q, a_p)                    \\
    = {} & g(a_p + a_q, a_p + (-1) (a_p + a_q)) \\
    = {} & g(a_p + a_q, -a_q)                   \\
    = {} & g(a_p + a_q + 1 (-a_q), -a_q)        \\
    = {} & g(a_p, (-1) a_q)                     \\
    = {} & (-1) g(a_p, a_q)                     \\
    = {} & {-g(a_p, a_q)}.
\end{align*}

假如, 我们能推出,
对任何 \(n-1\)~个 \(n \times 1\)~阵
\(a_2\), \(a_3\), \(\dots\), \(a_n\),
任何二个 \(n \times 1\)~阵 \(x\), \(y\),
任何二个数 \(s\), \(t\),
有
\begin{align*}
    f {([sx + ty, a_2, \dots, a_n])}
    =
    s
    f {([x, a_2, \dots, a_n])}
    +
    t
    f {([y, a_2, \dots, a_n])},
\end{align*}
那么, 利用反称性, 我们即得多线性.

在第一章, 节~\malneprasekcio{27} 里, 有如下结论:

\begin{theorem}
    设 \(A\) 是 \(m \times n\)~阵,
    且 \(A \neq 0\).
    设 \(A\) 有一个行列式非零的 \(r\)~级子阵
    \begin{align*}
        A_r = A\binom{i_1,\dots,i_r}{j_1,\dots,j_r}
    \end{align*}
    (其中
    \(1 \leq i_1 < \dots < i_r \leq m\),
    \(1 \leq j_1 < \dots < j_r \leq n\)),
    但 \(A\) 没有行列式非零的 \(r+1\)~级子阵.
    设 \(A\)~的%
    行~\(1\), \(2\), \(\dots\), \(m\)
    为 \(a_1\), \(a_2\), \(\dots\), \(a_m\).
    那么, 对任何不超过 \(m\) 的正整数 \(p\),
    一定存在 \(r\)~个数
    \(d_{p,1}\), \(d_{p,2}\), \(\dots\), \(d_{p,r}\),
    使
    \begin{align*}
        a_p
        = {} &
        d_{p,1} a_{i_1}
        + d_{p,2} a_{i_2}
        + \dots
        + d_{p,r} a_{i_r}
        \\
        = {} &
        \sum_{s = 1}^{r} {d_{p,s} a_{i_s}}.
    \end{align*}
\end{theorem}

利用类似的方法, 或利用转置, 我们可证

\begin{theorem}
    设 \(A\) 是 \(m \times n\)~阵,
    且 \(A \neq 0\).
    设 \(A\) 有一个行列式非零的 \(r\)~级子阵
    \begin{align*}
        A_r = A\binom{i_1,\dots,i_r}{j_1,\dots,j_r}
    \end{align*}
    (其中
    \(1 \leq i_1 < \dots < i_r \leq m\),
    \(1 \leq j_1 < \dots < j_r \leq n\)),
    但 \(A\) 没有行列式非零的 \(r+1\)~级子阵.
    设 \(A\)~的%
    列~\(1\), \(2\), \(\dots\), \(n\)
    为 \(a_1\), \(a_2\), \(\dots\), \(a_n\).
    那么, 对任何不超过 \(n\) 的正整数 \(q\),
    一定存在 \(r\)~个数
    \(d_{1,q}\), \(d_{2,q}\), \(\dots\), \(d_{r,q}\),
    使
    \begin{align*}
        a_q
        = {} &
        d_{1,q} a_{j_1}
        + d_{2,q} a_{j_2}
        + \dots
        + d_{r,q} a_{j_r}
        \\
        = {} &
        \sum_{s = 1}^{r} {d_{s,q} a_{j_s}}.
    \end{align*}
\end{theorem}

利用此事, 我们即可证明,
对任何 \(n-1\)~个 \(n \times 1\)~阵
\(a_2\), \(a_3\), \(\dots\), \(a_n\),
任何二个 \(n \times 1\)~阵 \(x\), \(y\),
任何二个数 \(s\), \(t\),
有
\begin{align*}
    f {([sx + ty, a_2, \dots, a_n])}
    =
    s
    f {([x, a_2, \dots, a_n])}
    +
    t
    f {([y, a_2, \dots, a_n])}.
\end{align*}
由此, 我们可证多线性.

\begin{proof}
    作 \(n \times (n-1)\)~阵
    \(B = [a_2, a_3, \dots, a_n]\);
    也就是说, \(B\) 的列~\(j-1\) 是 \(a_j\).

    若 \(B = 0\),
    由多齐性,
    \begin{align*}
        0 = f {([sx + ty, a_2, \dots, a_n])}
        = f {([x, a_2, \dots, a_n])}
        = f {([y, a_2, \dots, a_n])}.
    \end{align*}

    下设 \(B \neq 0\).
    那么, 存在一个低于 \(n\) 的正整数 \(r\),
    使 \(B\) 有一个行列式非零的 \(r\)~级子阵
    \begin{align*}
        B_r = B\binom{i_1,\dots,i_r}{j_1-1,\dots,j_r-1}
    \end{align*}
    (其中
    \(1 \leq i_1 < \dots < i_r \leq n\),
    \(2 \leq j_1 < \dots < j_r \leq n\)),
    但 \(B\) 没有行列式非零的 \(r+1\)~级子阵.

    若 \(r < n-1\), 那么,
    必存在\emph{不等于}
    \(j_1\), \(\dots\), \(j_r\) 的正整数 \(q\),
    与 \(r\)~个数
    \(d_{1,q}\), \(d_{2,q}\), \(\dots\), \(d_{r,q}\),
    使
    \begin{align*}
        a_q
        = {} &
        d_{1,q} a_{j_1}
        + d_{2,q} a_{j_2}
        + \dots
        + d_{r,q} a_{j_r}.
    \end{align*}
    任取 \(n \times 1\) 阵 \(z\).
    记
    \begin{align*}
        g(u)
        = f ([z, a_2, \dots, a_{q-1}, u, a_{q+1}, \dots, a_n]).
    \end{align*}
    利用倍加不变性,
    \begin{align*}
        g(a_q)
        = {} & g(a_q + (-d_{1,q})a_{j_1}) \\
        = {} & \dots \dots \dots \dots    \\
        = {} & g(a_q + (-d_{1,q})a_{j_1}
        + \dots + (-d_{r,q})a_{j_r})      \\
        = {} & g(a_q - (d_{1,q} a_{j_1}
        + \dots + d_{r,q} a_{j_r}))       \\
        = {} & g(0).
    \end{align*}
    利用多齐性, \(g(0) = 0\).
    故 \(g(a_q) = 0\).
    所以,
    \begin{align*}
        0 = f {([sx + ty, a_2, \dots, a_n])}
        = f {([x, a_2, \dots, a_n])}
        = f {([y, a_2, \dots, a_n])}.
    \end{align*}

    下设 \(r = n-1\).

    设从 \(1\), \(2\), \(\dots\), \(n\)
    去除 \(i_1\), \(i_2\), \(\dots\), \(i_{n-1}\) 后,
    还剩一个数 \(i_n\).
    设 \(a_1\) 是 \(n\)~级单位阵的列~\(i_n\).
    作 \(n\)~级阵
    \(A = [a_1, a_2, \dots, a_n]\).
    则 \(A(i_n | 1) = B_r\).
    不难算出,
    \(\det {(A)} = (-1)^{i_n + 1} \det {(B_r)} \neq 0\).

    作 \(n \times (n+2)\)~阵
    \(C = [a_1, a_2, \dots, a_n, x, y]\).
    于是, \(C\) 有一个行列式非零的 \(n\)~级子阵 \(A\),
    但 \(C\) 没有行列式非零的 \(n+1\)~级子阵.
    所以,
    一定存在 \(n\)~个数
    \(d_1\), \(d_2\), \(\dots\), \(d_n\),
    使
    \(x = d_1 a_1 + d_2 a_2 + \dots + d_n a_n\),
    也一定存在 \(n\)~个数
    \(d'_1\), \(d'_2\), \(\dots\), \(d'_n\),
    使
    \(y = d'_1 a_1 + d'_2 a_2 + \dots + d'_n a_n\).
    所以,
    \(sx + ty
    = (sd_1 + td'_1)a_1
    + (sd_2 + td'_2) a_2
    + \dots
    + (sd_n + td'_n) a_n\).

    记 \(h(z) = f ([z, a_2, \dots, a_n])\).
    则
    \begin{align*}
        f ([x, a_2, \dots, a_n])
        = {} & h(x)
        \\
        = {} & h(x - d_2 a_2)
        \\
        = {} & \dots \dots \dots \dots
        \\
        = {} & h(x - d_2 a_2 - \dots - d_n a_n)
        \\
        = {} & h(d_1 a_1)
        \\
        = {} & d_1 h(a_1).
    \end{align*}
    同理,
    \begin{align*}
         & f ([y, a_2, \dots, a_n]) = d'_1 h(a_1),
        \\
         & f ([sx + ty, a_2, \dots, a_n]) = (sd_1 + td'_1) h(a_1).
    \end{align*}
    比较, 得
    \begin{align*}
             &
        f {([sx + ty, a_2, \dots, a_n])}
        \\
        = {} &
        s f {([x, a_2, \dots, a_n])}
        +
        t f {([y, a_2, \dots, a_n])}.
        \qedhere
    \end{align*}
\end{proof}

接下来, 我展现另一个证明.
此证明或许更有意思.

\begin{restatable}{theorem}{TheoremMultiplyAndAdd}
    设 \(A\) 是一个 \(m \times n\)~阵.
    利用若干次 (列的) 倍加,
    我们可变 \(A\) 为一个 \(m \times n\)~阵 \(B\),
    使当 \(i < j\) 时,
    \([B]_{i,j} = 0\).
\end{restatable}

\begin{proof}
    我们用数学归纳法证明此事.
    具体地, 设 \(P(n)\) 为命题
    \begin{quotation}
        对\emph{任何}正整数 \(m\),
        对\emph{任何} \(m \times n\)~阵,
        利用若干次倍加 (指 ``列的倍加'', 下同),
        我们可变 \(A\) 为一个 \(m \times n\)~阵 \(B\),
        使当 \(i < j\) 时,
        \([B]_{i,j} = 0\).
    \end{quotation}
    则, 我们的目标是:
    对任何正整数 \(n\), \(P(n)\) 是正确的.

    \(P(1)\) 显然是对的.

    假定 \(P(n-1)\) 是对的.
    我们由此证 \(P(n)\) 也是对的.

    任取正整数 \(m\).
    任取一个 \(m \times n\)~阵 \(A\).

    我们先说明, 利用若干次倍加,
    我们可变 \(A\) 为一个 \(m \times n\)~阵 \(C\),
    使当 \(1 < j\) 时, \([C]_{1,j} = 0\).

    若 \(A\) 的行~\(1\) 的元全是零,
    我们 ``什么都不变'',
    取 \(C\) 为 \(A\).

    若 \([A]_{1,1} \neq 0\),
    我们可加 \(A\)~的列~\(1\)~的
    \(-[A]_{1,2} / [A]_{1,1}\)~倍于列~\(2\),
    得阵 \(A_2\).
    那么, \([A_2]_{1,2} = 0\),
    且 \([A_2]_{1,j} = [A]_{1,1}\)
    (\(j \neq 2\)).
    接着, 我们可加 \(A_2\)~的列~\(1\)~的
    \(-[A_2]_{1,3} / [A_2]_{1,1}\)~倍于列~\(3\),
    得阵 \(A_3\).
    那么, \([A_3]_{1,k} = 0\)
    (\(k = 2\), \(3\)),
    且 \([A_3]_{1,j} = [A_2]_{1,j}\)
    % (\(j \neq 2\), \(3\)).
    (\(j \neq 3\)).
    \(\dots \dots\)
    接着, 我们可加 \(A_{n-1}\)~的列~\(1\)~的
    \(-[A_{n-1}]_{1,n} / [A_{n-1}]_{1,1}\)~倍于列~\(n\),
    得阵 \(A_n\).
    那么, \([A_n]_{1,k} = 0\)
    (\(k = 2\), \(3\), \(\dots\), \(n\)).
    我们取 \(C\) 为 \(A_n\).

    若 \([A]_{1,1} = 0\),
    但有某 \([A]_{1,j} \neq 0\)
    (\(j \neq 1\)),
    我们加 \(A\)~的列~\(j\)~的 \(1\)~倍于列~\(1\),
    得阵 \(D\).
    那么, \([D]_{1,1} \neq 0\),
    这就转化问题为前面讨论过的情形.

    综上, 作若干次倍加,
    我们可变 \(A\) 为一个 \(m \times n\)~阵 \(C\),
    使当 \(1 < j\) 时, \([C]_{1,j} = 0\).

    考虑 \(C\) 的右下角的 \((m-1) \times (n-1)\)~子阵
    \(C(1|1)\).
    由假定, 作若干次倍加, 我们可变 \(C(1|1)\) 为一个
    \((m-1) \times (n-1)\)~阵 \(G\),
    使当 \(i < j\) 时, \([G]_{i,j} = 0\).

    注意到, 既然当 \(1 < j\) 时, \([C]_{1,j} = 0\),
    那么, 无论如何对 \(C\) 的不是列~\(1\) 的列作倍加,
    所得的阵的 \((1, j)\)-元一定是零.
    那么, 作若干次倍加后,
    我们可变 \(C\) 为一个 \(m \times n\)~阵 \(B\),
    使当 \(i\) 或 \(j\) 为 \(1\) 时,
    \([B]_{i,j} = [C]_{i,j}\),
    且 \(i\) 与 \(j\) 不为 \(1\) 时,
    \([B]_{i,j} = [G]_{i-1,j-1}\).
    所以, 当 \(i < j\) 时, \([B]_{i,j} = 0\).

    所以, \(P(n)\) 是正确的.
    由数学归纳法原理, 待证命题成立.
\end{proof}

\begin{theorem}
    设定义在全体 \(n\)~级阵上的函数 \(f\)
    适合倍加不变性与多齐性.
    设 \(A\) 是一个 \(n\)~级阵,
    且当 \(i < j\) 时, \([A]_{i,j} = 0\).
    则 \(f(A) = f(I_n) \det {(A)}\).
\end{theorem}

\begin{proof}
    我们用数学归纳法证明此事.
    具体地, 设 \(P(n)\) 为命题
    \begin{quotation}
        设定义在全体 \(n\)~级阵上的函数 \(f\)
        适合倍加不变性与多齐性.
        设 \(A\) 是一个 \(n\)~级阵,
        且当 \(i < j\) 时, \([A]_{i,j} = 0\).
        则 \(f(A) = f(I_n) \det {(A)}\).
    \end{quotation}
    则, 我们的目标是:
    对任何正整数 \(n\), \(P(n)\) 是正确的.

    \(P(1)\) 显然是对的.

    假定 \(P(n-1)\) 是对的.
    我们由此证 \(P(n)\) 也是对的.

    任取一个 \(n\)~级阵 \(A\),
    且当 \(i < j\) 时, \([A]_{i,j} = 0\).
    所以, \(A\) 形如
    \begin{align*}
        \begin{bmatrix}
            [A]_{1,1}   & 0           & \cdots & 0             & 0         \\
            [A]_{2,1}   & [A]_{2,2}   & \cdots & 0             & 0         \\
            \vdots      & \vdots      & \ddots & \vdots        & \vdots    \\
            [A]_{n-1,1} & [A]_{n-1,2} & \cdots & [A]_{n-1,n-1} & 0         \\
            [A]_{n,1}   & [A]_{n,2}   & \cdots & [A]_{n,n-1}   & [A]_{n,n} \\
        \end{bmatrix}
    \end{align*}
    那么, 由多齐性,
    \begin{align*}
        f(A) = [A]_{n,n}
        f\left(
        \begin{bmatrix}
                [A]_{1,1}   & 0           & \cdots & 0             & 0      \\
                [A]_{2,1}   & [A]_{2,2}   & \cdots & 0             & 0      \\
                \vdots      & \vdots      & \ddots & \vdots        & \vdots \\
                [A]_{n-1,1} & [A]_{n-1,2} & \cdots & [A]_{n-1,n-1} & 0      \\
                [A]_{n,1}   & [A]_{n,2}   & \cdots & [A]_{n,n-1}   & 1      \\
            \end{bmatrix}
        \right).
    \end{align*}
    利用 \(n-1\)~次倍加不变性,
    \begin{align*}
             &
        f\left(
        \begin{bmatrix}
                [A]_{1,1}   & 0           & \cdots & 0             & 0      \\
                [A]_{2,1}   & [A]_{2,2}   & \cdots & 0             & 0      \\
                \vdots      & \vdots      & \ddots & \vdots        & \vdots \\
                [A]_{n-1,1} & [A]_{n-1,2} & \cdots & [A]_{n-1,n-1} & 0      \\
                [A]_{n,1}   & [A]_{n,2}   & \cdots & [A]_{n,n-1}   & 1      \\
            \end{bmatrix}
        \right)
        \\
        = {} &
        f\left(
        \begin{bmatrix}
                [A]_{1,1}   & 0           & \cdots & 0             & 0      \\
                [A]_{2,1}   & [A]_{2,2}   & \cdots & 0             & 0      \\
                \vdots      & \vdots      & \ddots & \vdots        & \vdots \\
                [A]_{n-1,1} & [A]_{n-1,2} & \cdots & [A]_{n-1,n-1} & 0      \\
                0           & 0           & \cdots & 0             & 1      \\
            \end{bmatrix}
        \right),
    \end{align*}
    故
    \begin{align*}
        f(A) =
        f\left(
        \begin{bmatrix}
            [A]_{1,1}   & 0           & \cdots & 0             & 0      \\
            [A]_{2,1}   & [A]_{2,2}   & \cdots & 0             & 0      \\
            \vdots      & \vdots      & \ddots & \vdots        & \vdots \\
            [A]_{n-1,1} & [A]_{n-1,2} & \cdots & [A]_{n-1,n-1} & 0      \\
            0           & 0           & \cdots & 0             & 1      \\
        \end{bmatrix}
        \right)
        [A]_{n,n}.
    \end{align*}

    考虑定义在全体 \(n-1\)~级阵上的函数
    \begin{align*}
        g(X)
        =
        f\left(
        \begin{bmatrix}
            [X]_{1,1}   & [X]_{1,2}   & \cdots & [X]_{1,n-1}   & 0      \\
            [X]_{2,1}   & [X]_{2,2}   & \cdots & [X]_{2,n-1}   & 0      \\
            \vdots      & \vdots      & {}     & \vdots        & \vdots \\
            [X]_{n-1,1} & [X]_{n-1,2} & \cdots & [X]_{n-1,n-1} & 0      \\
            0           & 0           & \cdots & 0             & 1      \\
        \end{bmatrix}
        \right).
    \end{align*}
    不难验证, \(g\) 适合倍加不变性与多齐性.
    注意到, 若 \(i < j\),
    则 \(A(n|n)\) 的 \((i, j)\)-元为零.
    故, 由假定,
    \begin{align*}
        g(A(n|n)) = g(I_{n-1}) \det {(A(n|n))}
        = f(I_n) \det {(A(n|n))}.
    \end{align*}
    从而
    \begin{align*}
             &
        f\left(
        \begin{bmatrix}
                [A]_{1,1}   & 0           & \cdots & 0             & 0      \\
                [A]_{2,1}   & [A]_{2,2}   & \cdots & 0             & 0      \\
                \vdots      & \vdots      & \ddots & \vdots        & \vdots \\
                [A]_{n-1,1} & [A]_{n-1,2} & \cdots & [A]_{n-1,n-1} & 0      \\
                0           & 0           & \cdots & 0             & 1      \\
            \end{bmatrix}
        \right)
        \\
        = {} &
        g(A(n|n)) = f(I_n) \det {(A(n|n))}.
    \end{align*}
    则
    \begin{align*}
        f(A) = f(I_n) \det {(A(n|n))}\,[A]_{n,n}
        = f(I_n) \det {(A)}.
    \end{align*}

    所以, \(P(n)\) 是正确的.
    由数学归纳法原理, 待证命题成立.
\end{proof}

有了这些准备, 我们即可证明本节的主要结论.

\begin{proof}
    设定义在全体 \(n\)~级阵上的函数 \(f\)
    适合倍加不变性与多齐性.

    任取一个 \(n\)~级阵 \(A\).
    利用若干次倍加,
    我们可变 \(A\) 为一个 \(n\)~级阵 \(B\),
    使当 \(i < j\) 时, \([B]_{i,j} = 0\).
    因为倍加不变性, \(f(B) = f(A)\).
    由上个定理, \(f(B) = f(I) \det {(B)}\).
    故 \(f(A) = f(I) \det {(B)} = f(I) \det {(A)}\).
    (反过来, 不难验证, 若我们定义
    \(f(A) = f(I) \det {(A)}\),
    则 \(f\) 适合倍加不变性与多齐性.)
\end{proof}

\section{类行列式}

在上节, 我们知道, 若定义在全体 \(n\) 级阵上的函数 \(f\)
适合倍加不变性与多齐性,
则 \(f(A) = f(I) \det {(A)}\).
我们用了二种方法证明此事:
一种方法证明 \(f\) 适合多线性, 并使用已知的确定行列式的定理;
另一种方法不使用多线性, 也不使用已知的确定行列式的定理.
后一种方法可被推广, 得到确定 ``类行列式'' 的定理.
不过, 我们先定义一类函数.

\begin{definition}
    设 \(m\) 是定义在数上的函数.
    若 \(m(1) = 1\),
    且对任何数 \(s\), \(t\), 有 \(m(st) = m(s)\, m(t)\),
    则 \(m\) 是\emph{保乘的}.
\end{definition}

显然, 恒等函数 \(m(t) = t\) 是保乘的.
若 \(k\) 是正整数,
由次方的性质, 可以验证,
\(m(t) = t^k\) 也是保乘的.
此外, 绝对值函数 \(m(t) = |t|\) 也是保乘的
(因为 \(|1| = 1\),
且对任何数 \(a\), \(b\), 有 \(|ab| = |a|\,|b|\)).

\begin{theorem}
    设定义在全体 \(n\)~级阵上的函数 \(f\)
    适合:

    (1)
    倍加不变性.

    (2)
    (``多齐性的变体'')
    对任何不超过 \(n\) 的正整数 \(j\),
    任何 \(n-1\)~个 \(n \times 1\)~阵
    \(a_1\), \(\dots\), \(a_{j-1}\),
    \(a_{j+1}\), \(\dots\), \(a_n\),
    任何 \(n \times 1\)~阵 \(x\),
    任何数 \(s\),
    有
    \begin{align*}
        f {([a_1, \dots, a_{j-1}, sx, a_{j+1}, \dots, a_n])}
        =
        m(s)
        f {([a_1, \dots, a_{j-1}, x, a_{j+1}, \dots, a_n])},
    \end{align*}
    其中, 定义在数上的函数 \(m\) 是保乘的:
    \(m(1) = 1\),
    且对任何数 \(s\), \(t\), 有 \(m(st) = m(s)\, m(t)\).

    那么, 对任何 \(n\)~级阵 \(A\),
    \(f(A) = f(I)\, m(\det {(A)})\).
\end{theorem}

为证明此事, 我们先证如下命题.

\begin{theorem}
    设定义在全体 \(n\)~级阵上的函数 \(f\)
    适合倍加不变性与 ``多齐性的变体''.
    设 \(A\) 是一个 \(n\)~级阵,
    且当 \(i < j\) 时, \([A]_{i,j} = 0\).
    则 \(f(A) = f(I_n)\, m(\det {(A)})\).
\end{theorem}

\begin{proof}
    我们用数学归纳法证明此事.
    具体地, 设 \(P(n)\) 为命题
    \begin{quotation}
        设定义在全体 \(n\)~级阵上的函数 \(f\)
        适合倍加不变性与 ``多齐性的变体''.
        设 \(A\) 是一个 \(n\)~级阵,
        且当 \(i < j\) 时, \([A]_{i,j} = 0\).
        则 \(f(A) = f(I_n)\, m(\det {(A)})\).
    \end{quotation}
    则, 我们的目标是:
    对任何正整数 \(n\), \(P(n)\) 是正确的.

    \(P(1)\) 显然是对的.

    假定 \(P(n-1)\) 是对的.
    我们由此证 \(P(n)\) 也是对的.

    任取一个 \(n\)~级阵 \(A\),
    且当 \(i < j\) 时, \([A]_{i,j} = 0\).
    所以, \(A\) 形如
    \begin{align*}
        \begin{bmatrix}
            [A]_{1,1}   & 0           & \cdots & 0             & 0         \\
            [A]_{2,1}   & [A]_{2,2}   & \cdots & 0             & 0         \\
            \vdots      & \vdots      & \ddots & \vdots        & \vdots    \\
            [A]_{n-1,1} & [A]_{n-1,2} & \cdots & [A]_{n-1,n-1} & 0         \\
            [A]_{n,1}   & [A]_{n,2}   & \cdots & [A]_{n,n-1}   & [A]_{n,n} \\
        \end{bmatrix}
    \end{align*}
    那么, 由 ``多齐性的变体'',
    \begin{align*}
        f(A) = m([A]_{n,n})
        f\left(
        \begin{bmatrix}
                [A]_{1,1}   & 0           & \cdots & 0             & 0      \\
                [A]_{2,1}   & [A]_{2,2}   & \cdots & 0             & 0      \\
                \vdots      & \vdots      & \ddots & \vdots        & \vdots \\
                [A]_{n-1,1} & [A]_{n-1,2} & \cdots & [A]_{n-1,n-1} & 0      \\
                [A]_{n,1}   & [A]_{n,2}   & \cdots & [A]_{n,n-1}   & 1      \\
            \end{bmatrix}
        \right).
    \end{align*}
    利用 \(n-1\)~次倍加不变性,
    \begin{align*}
             &
        f\left(
        \begin{bmatrix}
                [A]_{1,1}   & 0           & \cdots & 0             & 0      \\
                [A]_{2,1}   & [A]_{2,2}   & \cdots & 0             & 0      \\
                \vdots      & \vdots      & \ddots & \vdots        & \vdots \\
                [A]_{n-1,1} & [A]_{n-1,2} & \cdots & [A]_{n-1,n-1} & 0      \\
                [A]_{n,1}   & [A]_{n,2}   & \cdots & [A]_{n,n-1}   & 1      \\
            \end{bmatrix}
        \right)
        \\
        = {} &
        f\left(
        \begin{bmatrix}
                [A]_{1,1}   & 0           & \cdots & 0             & 0      \\
                [A]_{2,1}   & [A]_{2,2}   & \cdots & 0             & 0      \\
                \vdots      & \vdots      & \ddots & \vdots        & \vdots \\
                [A]_{n-1,1} & [A]_{n-1,2} & \cdots & [A]_{n-1,n-1} & 0      \\
                0           & 0           & \cdots & 0             & 1      \\
            \end{bmatrix}
        \right),
    \end{align*}
    故
    \begin{align*}
        f(A) =
        f\left(
        \begin{bmatrix}
                [A]_{1,1}   & 0           & \cdots & 0             & 0      \\
                [A]_{2,1}   & [A]_{2,2}   & \cdots & 0             & 0      \\
                \vdots      & \vdots      & \ddots & \vdots        & \vdots \\
                [A]_{n-1,1} & [A]_{n-1,2} & \cdots & [A]_{n-1,n-1} & 0      \\
                0           & 0           & \cdots & 0             & 1      \\
            \end{bmatrix}
        \right)
        m([A]_{n,n}).
    \end{align*}

    考虑定义在全体 \(n-1\)~级阵上的函数
    \begin{align*}
        g(X)
        =
        f\left(
        \begin{bmatrix}
            [X]_{1,1}   & [X]_{1,2}   & \cdots & [X]_{1,n-1}   & 0      \\
            [X]_{2,1}   & [X]_{2,2}   & \cdots & [X]_{2,n-1}   & 0      \\
            \vdots      & \vdots      & {}     & \vdots        & \vdots \\
            [X]_{n-1,1} & [X]_{n-1,2} & \cdots & [X]_{n-1,n-1} & 0      \\
            0           & 0           & \cdots & 0             & 1      \\
        \end{bmatrix}
        \right).
    \end{align*}
    不难验证, \(g\) 适合倍加不变性与 ``多齐性的变体''.
    注意到, 若 \(i < j\),
    则 \(A(n|n)\) 的 \((i, j)\)-元为零.
    故, 由假定,
    \begin{align*}
        g(A(n|n)) = g(I_{n-1})\, m(\det {(A(n|n))})
        = f(I_n)\, m(\det {(A(n|n))}).
    \end{align*}
    从而
    \begin{align*}
             &
        f\left(
        \begin{bmatrix}
                [A]_{1,1}   & 0           & \cdots & 0             & 0      \\
                [A]_{2,1}   & [A]_{2,2}   & \cdots & 0             & 0      \\
                \vdots      & \vdots      & \ddots & \vdots        & \vdots \\
                [A]_{n-1,1} & [A]_{n-1,2} & \cdots & [A]_{n-1,n-1} & 0      \\
                0           & 0           & \cdots & 0             & 1      \\
            \end{bmatrix}
        \right)
        \\
        = {} &
        g(A(n|n)) = f(I_n)\, m(\det {(A(n|n))}).
    \end{align*}
    则
    \begin{align*}
        f(A)
        = {} & f(I_n)\, m(\det {(A(n|n))}) \,m([A]_{n,n}) \\
        = {} & f(I_n)\, m(\det {(A(n|n))} \,[A]_{n,n})    \\
        = {} & f(I_n)\, m(\det {(A)}).
    \end{align*}

    所以, \(P(n)\) 是正确的.
    由数学归纳法原理, 待证命题成立.
\end{proof}

有了这些准备, 我们即可证明本节的主要结论.

\begin{proof}
    设定义在全体 \(n\)~级阵上的函数 \(f\)
    适合倍加不变性与 ``多齐性的变体''.

    任取一个 \(n\)~级阵 \(A\).
    利用若干次倍加,
    我们可变 \(A\) 为一个 \(n\)~级阵 \(B\),
    使当 \(i < j\) 时, \([B]_{i,j} = 0\).
    因为倍加不变性, \(f(B) = f(A)\).
    由上个定理, \(f(B) = f(I)\, m(\det {(B)})\).
    故 \(f(A) = f(I)\, m(\det {(B)})
    = f(I)\, m(\det {(A)})\).
    (反过来, 不难验证, 若我们定义
    \(f(A) = f(I)\, m(\det {(A)})\),
    则 \(f\) 适合倍加不变性与 ``多齐性的变体''.)
\end{proof}

特别地, 取 \(m\) 为绝对值函数, 我们有

\begin{theorem}
    设定义在全体 \(n\)~级阵上的函数 \(f\)
    适合:

    (1)
    倍加不变性.

    (2)
    对任何不超过 \(n\) 的正整数 \(j\),
    任何 \(n-1\)~个 \(n \times 1\)~阵
    \(a_1\), \(\dots\), \(a_{j-1}\),
    \(a_{j+1}\), \(\dots\), \(a_n\),
    任何 \(n \times 1\)~阵 \(x\),
    任何数 \(s\),
    有
    \begin{align*}
        f {([a_1, \dots, a_{j-1}, sx, a_{j+1}, \dots, a_n])}
        =
        |s|
        f {([a_1, \dots, a_{j-1}, x, a_{j+1}, \dots, a_n])}.
    \end{align*}

    那么, 对任何 \(n\)~级阵 \(A\),
    \(f(A) = f(I)\, |{\det {(A)}}|\).
\end{theorem}

设您在研究某数学问题.
设您发现, 您研究的一个量可被认为是定义在 \(n\)~级阵上的函数
(\(n\) 是某个正整数),
且适合倍加不变性与 ``多齐性的变体''.
那么, 由本节的定理, 它是一个跟行列式有关的量.
这不是偶然的; 这是必然的.
这是好的, 我想.

% 几何地, \(2\)~级阵的行列式的绝对值与平行 \(4\)~边形的面积有关,
% 且 \(3\)~级阵的行列式的绝对值与平行 \(6\)~面体的体积有关.


% Stories about nonzero and positive determinants
\section{绝对值的性质 (1)}

设 \(a\) 是实数.
则 \(a\) 的绝对值
\begin{align*}
    |a| = \begin{cases}
              a,  & a \geq 0; \\
              -a, & a < 0.
          \end{cases}
\end{align*}

如下命题是正确的.

(1)
\(|0| = 0\);
若实数 \(a\) 适合 \(|a| = 0\), 则 \(a = 0\).

因为 \(0 \geq 0\),
故 \(|0| = 0\).

若 \(a > 0\), 则 \(|a| = a > 0\);
若 \(a < 0\), 则 \(|a| = -a > 0\).
于是, 若 \(|a| = 0\), 则 \(a\) 不能是正数,
也不能是负数, 故 \(a = 0\).

(2)
对任何实数 \(a\), 必 \(|a| \geq 0\).

我们已知, 若 \(a > 0\) 或 \(a < 0\), 则 \(|a| > 0\).
另外, \(|0| = 0\).

(3)
对任何实数 \(a\), 必 \(|a| \geq a \geq -|a|\).

若 \(a \geq 0\), 则 \(|a| = a\).
故 \(|a| = a \geq -a = -|a|\);
若 \(a < 0\), 则 \(|a| = -a\).
故 \(|a| = -a > a = -|a|\).

(4)
设 \(a\), \(b\) 是\emph{非负实数}.
若 \(a \geq b\), 则 \(a^2 \geq b^2\);
反过来, 若 \(a^2 \geq b^2\), 则 \(a \geq b\).

若 \(a > b \geq 0\), 则 \(a a > b b\), 故 \(a^2 > b^2\);
若 \(a = b \geq 0\), 则 \(a a = b b\), 故 \(a^2 = b^2\);
若 \(0 \leq a < b\), 则 \(a a < b b\), 故 \(a^2 < b^2\).
所以, 若 \(a \geq b\), 则 \(a^2 \geq b^2\);
反过来, 若 \(a^2 \geq b^2\), 则因 \(a < b\)
无法推出 \(a^2 \geq b^2\), 故必 \(a \geq b\).

(5)
对任何实数 \(a\), 必 \(|a|^2 = a^2\).
所以, 对任何实数 \(a\), 必 \(\sqrt{a^2} = |a|\).

若 \(a \geq 0\), 则 \(|a|^2 = a^2\);
若 \(a < 0\), 则 \(|a|^2 = (-a)^2 = a^2\).
因为\emph{非负实数} \(|a|\) 适合 \(|a|^2 = a^2\),
故由 \(\surd\) 的定义, \(\sqrt{a^2} = |a|\).

(6)
对任何实数 \(a\), \(b\), 必 \(|ab| = |a|\, |b|\).

若 \(a \geq 0\), \(b \geq 0\),
则 \(ab \geq 0\),
故 \(|ab| = ab = |a|\, |b|\);
若 \(a \geq 0\), \(b < 0\),
则 \(ab \leq 0\),
故 \(|ab| = -ab = a(-b) =|a|\, |b|\);
若 \(a < 0\), \(b \geq 0\),
则 \(ab \leq 0\),
故 \(|ab| = -ab = (-a)b = |a|\, |b|\);
若 \(a < 0\), \(b < 0\),
则 \(ab \geq 0\),
故 \(|ab| = ab = (-a)(-b) = |a|\, |b|\).

(7)
对任何实数 \(a\), \(b\), 必 \(|{a + b}| \leq |a| + |b|\).

若 \(a + b \geq 0\), 则 \(|{a + b}| = a + b \leq |a| + |b|\);
若 \(a + b < 0\), 则 \(|{a + b}| = -(a + b)
= (-a) + (-b) \leq |{-a}| + |{-b}| = |a| + |b|\).

类似地, 若 \(a_1\), \(\dots\), \(a_n\) 是实数, 则
\begin{align*}
    |{a_1 + \dots + a_n}| \leq |a_1| + \dots + |a_n|.
\end{align*}
可用数学归纳法证它.

(8)
对任何实数 \(a\), \(b\), 必 \(|{a - b}| \geq |a| - |b|\).

因为 \(|{a - b}| + |b| \geq |{(a - b) + b}| = |a|\).

\vspace{2ex}

这些事实会是有用的.

\section{绝对值的性质 (2)}

设 \(z = a + \mathrm{i} b\) 是复数
(其中, \(\mathrm{i}\) 是虚数单位,
\(a\), \(b\) 是实数, 下同).
则 \(z\) 的绝对值
\begin{align*}
    |z| = \sqrt{a^2 + b^2}.
\end{align*}

如下命题是正确的.

(1)
\(|0| = 0\);
若复数 \(z\) 适合 \(|z| = 0\), 则 \(z = 0\).

首先, \(|0| = |{0 + \mathrm{i} 0}| = \sqrt{0^2 + 0^2} = 0\).

若 \(z = a + \mathrm{i} b\) 适合 \(|z| = 0\),
则 \(\sqrt{a^2 + b^2} = 0\).
则 \(a^2 + b^2 = 0\).
则 \(a = b = 0\).
则 \(z = 0\).

(2)
对任何复数 \(z\), 必 \(|z| \geq 0\).

由 \(\surd\) 的定义, 这是显然的.

(3)
对任何复数 \(z\), \(w\), 必 \(|zw| = |z|\, |w|\).

设 \(z = a + \mathrm{i} b\), 且 \(w = c + \mathrm{i} d\).
则
\begin{align*}
    |zw|^2
    = {} & |{(ac - bd) + \mathrm{i} (ad + bc)}|^2 \\
    = {} & (ac - bd)^2 + (ad + bc)^2              \\
    = {} & a^2 c^2 + b^2 d^2 + a^2 d^2 + b^2 c^2  \\
    = {} & (a^2 + b^2) (c^2 + d^2)                \\
    = {} & |z|^2 |w|^2                            \\
    = {} & (|z|\, |w|)^2.
\end{align*}
因为 \(|zw|\) 与 \(|z|\, |w|\) 是非负实数,
故 \(|zw| = |z|\, |w|\).

顺便, 注意到, 对任何实数 \(a\), \(b\), \(c\), \(d\),
有
\begin{align*}
    (a^2 + b^2) (c^2 + d^2)
    = (ac - bd)^2 + (ad + bc)^2
    \geq (ac - bd)^2.
\end{align*}

(4)
对任何复数 \(z\), \(w\), 必 \(|{z + w}| \leq |z| + |w|\).

设 \(z = a + \mathrm{i} b\), 且 \(w = c + \mathrm{i} d\).
则
\begin{align*}
    |{z + w}|^2
    = {}    & |{(a + c) + \mathrm{i} (b + d)}|^2      \\
    = {}    & (a + c)^2 + (b + d)^2                   \\
    = {}    & a^2 + b^2 + c^2 + d^2 + 2(ac + bd)      \\
    = {}    & |z|^2 + |w|^2 + 2(ac + bd)              \\
    \leq {} & |z|^2 + |w|^2 + 2|{ac + bd}|            \\
    = {}    & |z|^2 + |w|^2 + 2 \sqrt{(ac + bd)^2}    \\
    = {}    & |z|^2 + |w|^2 + 2 \sqrt{(ac - (-b)d)^2} \\
    \leq {} & |z|^2 + |w|^2
    + 2 \sqrt{(a^2 + (-b)^2) (c^2 + d^2)}             \\
    \leq {} & |z|^2 + |w|^2 + 2 \sqrt{(|z|\,|w|)^2}   \\
    \leq {} & |z|^2 + |w|^2 + 2 \sqrt{|z|^2 |w|^2}    \\
    = {}    & |z|^2 + |w|^2 + 2 |z|\, |w|             \\
    = {}    & (|z| + |w|)^2.
\end{align*}
因为 \(|{z + w}|\) 与 \(|z| + |w|\) 是非负实数,
故 \(|{z + w}| \leq |z| + |w|\).

类似地, 若 \(z_1\), \(\dots\), \(z_n\) 是复数, 则
\begin{align*}
    |{z_1 + \dots + z_n}| \leq |z_1| + \dots + |z_n|.
\end{align*}
可用数学归纳法证它.

(5)
对任何复数 \(z\), \(w\), 必 \(|{z - w}| \geq |z| - |w|\).

因为 \(|{z - w}| + |w| \geq |{(z - w) + w}| = |z|\).

\vspace{2ex}

这些事实会是有用的.

\section{绝对值的性质 (3)}

设 \(z\), \(w\) 是数.
设 \(z\) 的绝对值是 \(|z|\).
则:

(1)
\(|0| = 0\);
若数 \(z\) 适合 \(|z| = 0\), 则 \(z = 0\).

(2)
\(|z| \geq 0\).

(3)
\(|zw| = |z|\, |w|\).

(4)
\(|{z + w}| \leq |z| + |w|\).

(5)
\(|{z - w}| \geq |z| - |w|\).

\vspace{2ex}

这些事实会是有用的.

\section{关于实数的大小的几个事实}

设 \(y_1\), \(y_2\), \(\dots\), \(y_n\) 是若干个实数.
我们总可从大到小地排它们,
故

\begin{theorem}
    设 \(y_1\), \(y_2\), \(\dots\), \(y_n\) 是若干个实数.
    则存在不超过 \(n\) 的正整数 \(i\),
    使对任何不超过 \(n\) 的正整数 \(k\),
    必 \(y_i \geq y_k\).
\end{theorem}

\begin{theorem}
    设 \(y_1\), \(y_2\), \(\dots\), \(y_n\) 是若干个实数
    (\(n \geq 2\)).

    (1)
    存在不超过 \(n\), 且不等的正整数 \(i\), \(j\),
    使对任何不超过 \(n\), 且不等于 \(i\) 的正整数 \(k\),
    必 \(y_i \geq y_j \geq y_k\).

    (2)
    对任何不超过 \(n\), 且不等于 \(i\) 的正整数 \(k\),
    必 \(y_j \geq y_k\).

    (3)
    对任何不超过 \(n\) 的正整数 \(k\),
    必 \(y_i \geq y_k\).
\end{theorem}

设 \(y_1\), \(y_2\), \(\dots\), \(y_n\) 是若干个\emph{非负}实数,
且不全是零.
从而有某不超过 \(n\) 的正整数 \(p\), 使 \(y_p > 0\).
再设某不超过 \(n\) 的正整数 \(i\) 适合:
对任何不超过 \(n\) 的正整数 \(k\),
必 \(y_i \geq y_k\).
则 \(y_i \geq y_p > 0\).
所以

\begin{theorem}
    设 \(y_1\), \(y_2\), \(\dots\), \(y_n\) 是若干个\emph{非负}实数,
    且\emph{不全是零}.
    则存在不超过 \(n\) 的正整数 \(i\),
    使对任何不超过 \(n\) 的正整数 \(k\),
    必 \(y_i \geq y_k\),
    且 \(y_i > 0\).
\end{theorem}

\begin{theorem}
    设 \(y_1\), \(y_2\), \(\dots\), \(y_n\) 是若干个\emph{非负}实数,
    且\emph{不全是零}
    (\(n \geq 2\)).

    (1)
    存在不超过 \(n\), 且不等的正整数 \(i\), \(j\),
    使对任何不超过 \(n\), 且不等于 \(i\) 的正整数 \(k\),
    必 \(y_i \geq y_j \geq y_k\),
    且 \(y_i > 0\).

    (2)
    对任何不超过 \(n\), 且不等于 \(i\) 的正整数 \(k\),
    必 \(y_j \geq y_k\).

    (3)
    对任何不超过 \(n\) 的正整数 \(k\),
    必 \(y_i \geq y_k\).

    (4)
    有一个特别的情形值得一提.
    设 \(y_j = 0\).
    由 (2) 知,
    对任何不超过 \(n\), 且不等于 \(i\) 的正整数 \(k\),
    必 \(0 = y_j \geq y_k \geq 0\).
    故对任何不超过 \(n\), 且不等于 \(i\) 的正整数 \(k\),
    必 \(y_k = 0\).
\end{theorem}

\section{行列式为零的阵的性质}

本节, 我们讨论行列式为零的阵的几个性质.

\begin{theorem}
    设 \(A\) 是一个 \(n\)~级阵.
    设 \(\det {(A)} = 0\).
    则存在不超过 \(n\) 的正整数 \(i\), 使
    \begin{equation}
        |[A]_{i,i}| \leq
        \sum_{\substack{1 \leq u \leq n \\
            u \neq i}} {|[A]_{i,u}|}.
        \label{eq:NonzeroDet1}
    \end{equation}
\end{theorem}

\begin{proof}
    因为 \(\det {(A)} = 0\),
    故有非零的 \(n \times 1\)~阵 \(x\),
    使 \(Ax = 0\)
    (见第一章, 节~\malneprasekcio{23}).

    考虑不全为零的非负实数
    \(|[x]_{1,1}|\), \(|[x]_{2,1}|\), \(\dots\), \(|[x]_{n,1}|\).
    则有不超过 \(n\) 的正整数 \(i\),
    使对任何不超过 \(n\) 的正整数 \(u\),
    有 \(|[x]_{i,1}| \geq |[x]_{u,1}|\),
    且 \(|[x]_{i,1}| > 0\).
    则
    \begin{align*}
        0
        = {} &
        [0]_{i,1}
        \\
        = {} &
        [A x]_{i,1}
        \\
        = {} &
        \sum_{1 \leq u \leq n}
        {[A]_{i,u} [x]_{u,1}}
        \\
        = {} &
        \sum_{\substack{1 \leq u \leq n \\ u \neq i}}
        {[A]_{i,u} [x]_{u,1}}
        + [A]_{i,i} [x]_{i,1}.
    \end{align*}
    则
    \begin{align*}
        |[A]_{i,i}|\,|[x]_{i,1}|
        = {}    &
        |{[A]_{i,i} [x]_{i,1}}|
        \\
        = {}    &
        |{-[A]_{i,i} [x]_{i,1}}|
        \\
        = {}    &
        \Bigg|
        \sum_{\substack{1 \leq u \leq n \\ u \neq i}}
        {[A]_{i,u} [x]_{u,1}}
        \Bigg|
        \\
        \leq {} &
        \sum_{\substack{1 \leq u \leq n \\ u \neq i}}
        {|{[A]_{i,u} [x]_{u,1}}|}
        \\
        = {}    &
        \sum_{\substack{1 \leq u \leq n \\ u \neq i}}
        {|[A]_{i,u}|\, |[x]_{u,1}|}
        \\
        \leq {} &
        \sum_{\substack{1 \leq u \leq n \\ u \neq i}}
        {|[A]_{i,u}|\, |[x]_{i,1}|}
        \\
        = {}    &
        \Bigg(
        \sum_{\substack{1 \leq u \leq n \\ u \neq i}}
        {|[A]_{i,u}|}
        \Bigg) |[x]_{i,1}|.
    \end{align*}
    因为 \(|[x]_{i,1}| > 0\),
    故式~\eqref{eq:NonzeroDet1} 是对的.
\end{proof}

不过, 此事反过来不一定是对的.

\begin{example}
    设 \(A =
    \begin{bmatrix}
        0  & 1 \\
        -1 & 0 \\
    \end{bmatrix}\)
    是一个 \(2\)~级阵.
    取 \(i = 1\) 或 \(i = 2\), 即有
    \begin{align*}
        |[A]_{i,1}|
        = 0
        \leq 1
        = \sum_{\substack{1 \leq u \leq 2 \\u \neq i}} {|[A]_{i,u}|}.
    \end{align*}
    可是, \(\det {(A)} = 1 \neq 0\).
\end{example}

\begin{theorem}
    设 \(A\) 是一个 \(n\)~级阵 (\(n \geq 2\)).
    设 \(\det {(A)} = 0\).
    则存在不超过 \(n\), 且不等的正整数 \(j\), \(k\), 使
    \begin{equation}
        |[A]_{j,j}|\,|[A]_{k,k}| \leq
        \Bigg(
        \sum_{\substack{1 \leq p \leq n \\
                p \neq j}} {|[A]_{j,p}|}
        \Bigg)
        \Bigg(
        \sum_{\substack{1 \leq q \leq n \\
                q \neq k}} {|[A]_{k,q}|}
        \Bigg).
        \label{eq:NonzeroDet2}
    \end{equation}
\end{theorem}

\begin{proof}
    因为 \(\det {(A)} = 0\),
    故有非零的 \(n \times 1\)~阵 \(x\),
    使 \(Ax = 0\)
    (见第一章, 节~\malneprasekcio{23}).

    考虑不全为零的非负实数
    \(|[x]_{1,1}|\), \(|[x]_{2,1}|\), \(\dots\), \(|[x]_{n,1}|\).
    则有不超过 \(n\), 且不等的正整数 \(j\), \(k\),
    使对任何不超过 \(n\), 且不等于 \(j\) 的正整数 \(p\),
    有 \(|[x]_{j,1}| \geq |[x]_{k,1}| \geq |[x]_{p,1}|\),
    且 \(|[x]_{j,1}| > 0\).

    注意到, 对任何不超过 \(n\) 的正整数 \(u\), 有
    \begin{align*}
        0
        = {} &
        [0]_{u,1}
        \\
        = {} &
        [A x]_{u,1}
        \\
        = {} &
        \sum_{1 \leq v \leq n}
        {[A]_{u,v} [x]_{v,1}}
        \\
        = {} &
        \sum_{\substack{1 \leq v \leq n \\ v \neq u}}
        {[A]_{u,v} [x]_{v,1}}
        + [A]_{u,u} [x]_{u,1}.
    \end{align*}
    故
    \begin{align*}
        -[A]_{u,u} [x]_{u,1}
        = \sum_{\substack{1 \leq v \leq n \\ v \neq u}}
        {[A]_{u,v} [x]_{v,1}}.
    \end{align*}
    则
    \begin{align*}
        |[A]_{u,u}|\, |[x]_{u,1}|
        = {}    &
        |{[A]_{u,u} [x]_{u,1}}|
        \\
        = {}    &
        |{-[A]_{u,u} [x]_{u,1}}|
        \\
        = {}    &
        \Bigg|
        \sum_{\substack{1 \leq v \leq n \\ v \neq u}}
        {[A]_{u,v} [x]_{v,1}}
        \Bigg|
        \\
        \leq {} &
        \sum_{\substack{1 \leq v \leq n \\ v \neq u}}
        {|{[A]_{u,v} [x]_{v,1}}|}
        \\
        = {}    &
        \sum_{\substack{1 \leq v \leq n \\ v \neq u}}
        {|{[A]_{u,v}|\, |[x]_{v,1}}|}.
    \end{align*}
    则
    (取 \(u = j\),
    并注意到对任何不超过 \(n\), 且不等于 \(j\) 的正整数 \(p\),
    有 \(|[x]_{p,1}| \leq |[x]_{k,1}|\))
    \begin{align*}
        |[A]_{j,j}|\, |[x]_{j,1}|
        \leq {} &
        \sum_{\substack{1 \leq p \leq n \\ p \neq j}}
        {|{[A]_{j,p}|\, |[x]_{p,1}}|}
        \\
        \leq {} &
        \sum_{\substack{1 \leq p \leq n \\ p \neq j}}
        {|{[A]_{j,p}|\, |[x]_{k,1}}|}
        \\
        = {}    &
        \Bigg(
        \sum_{\substack{1 \leq p \leq n \\ p \neq j}}
        {|{[A]_{j,p}|}
        \Bigg)
        |[x]_{k,1}}|,
    \end{align*}
    且
    (取 \(u = k\),
    并注意到对任何不超过 \(n\) 的正整数 \(q\),
    有 \(|[x]_{q,1}| \leq |[x]_{j,1}|\))
    \begin{align*}
        |[A]_{k,k}|\, |[x]_{k,1}|
        \leq {} &
        \sum_{\substack{1 \leq q \leq n \\ q \neq k}}
        {|{[A]_{k,q}|\, |[x]_{q,1}}|}
        \\
        \leq {} &
        \sum_{\substack{1 \leq q \leq n \\ q \neq k}}
        {|{[A]_{k,q}|\, |[x]_{j,1}}|}
        \\
        = {}    &
        \Bigg(
        \sum_{\substack{1 \leq q \leq n \\ q \neq k}}
        {|{[A]_{k,q}|}
        \Bigg)
        |[x]_{j,1}}|.
    \end{align*}
    故
    \begin{align*}
                &
        (|[A]_{j,j}|\, |[A]_{k,k}|)\,
        (|[x]_{j,1}|\, |[x]_{k,1}|)
        \\
        \leq {} &
        \Bigg(
        \sum_{\substack{1 \leq p \leq n \\ p \neq j}}
        {|{[A]_{j,p}|}}
        \Bigg)
        \Bigg(
        \sum_{\substack{1 \leq q \leq n \\ q \neq k}}
        {|{[A]_{k,q}|}}
        \Bigg)
        (|[x]_{j,1}|\, |[x]_{k,1}|).
    \end{align*}
    因为 \(|[x]_{j,1}| > 0\),
    故
    \begin{align*}
        % &
        (|[A]_{j,j}|\, |[A]_{k,k}|)\,
        |[x]_{k,1}|
        % \\
        % \leq {} &
        \leq
        \Bigg(
        \sum_{\substack{1 \leq p \leq n \\ p \neq j}}
        {|{[A]_{j,p}|}}
        \Bigg)
        \Bigg(
        \sum_{\substack{1 \leq q \leq n \\ q \neq k}}
        {|{[A]_{k,q}|}}
        \Bigg)
        |[x]_{k,1}|.
    \end{align*}

    若 \(|[x]_{k,1}| > 0\),
    则式~\eqref{eq:NonzeroDet2} 当然是对的.
    若 \(|[x]_{k,1}| = 0\),
    则对任何不超过 \(n\), 且不等于 \(j\) 的正整数 \(p\),
    有 \(0 = |[x]_{k,1}| \geq |[x]_{p,1}| \geq 0\).
    故 \(|[x]_{p,1}| = 0\).
    故 \([x]_{p,1} = 0\).
    则
    \begin{align*}
        -[A]_{j,j} [x]_{j,1}
        = \sum_{\substack{1 \leq p \leq n \\ p \neq j}}
        {[A]_{j,p} [x]_{p,1}} = 0.
    \end{align*}
    因为 \(|[x]_{j,1}| > 0\),
    故 \([x]_{j,1} \neq 0\).
    从而 \([A]_{j,j} = 0\).
    则式~\eqref{eq:NonzeroDet2} 的左侧是零.
\end{proof}

此事反过来也不一定是对的:
考虑上个例的 \(A\) 即可.

我们说, 若存在不超过 \(n\), 且不等的正整数 \(j\), \(k\),
使式~\eqref{eq:NonzeroDet2} 是对的,
则存在不超过 \(n\) 的正整数 \(i\),
使式~\eqref{eq:NonzeroDet1} 是对的.
用反证法, 不难看出, 若式~\eqref{eq:NonzeroDet2} 是对的,
则
\begin{align*}
    |[A]_{j,j}| >
    \sum_{\substack{1 \leq p \leq n \\
        p \neq j}} {|[A]_{j,p}|}
\end{align*}
与
\begin{align*}
    |[A]_{k,k}| >
    \sum_{\substack{1 \leq q \leq n \\
        q \neq k}} {|[A]_{k,q}|}
\end{align*}
不能全是对的.
故
\begin{align*}
    |[A]_{j,j}| \leq
    \sum_{\substack{1 \leq p \leq n \\
        p \neq j}} {|[A]_{j,p}|}
\end{align*}
与
\begin{align*}
    |[A]_{k,k}| \leq
    \sum_{\substack{1 \leq q \leq n \\
        q \neq k}} {|[A]_{k,q}|}
\end{align*}
的至少一个是对的.
取 \(i\) 为 \(j\) 或 \(k\) 即可.

\vspace{2ex}

或许, 您会想:
\begin{quotation}
    设 \(A\) 是一个 \(n\)~级阵 (\(n \geq 3\)).
    设 \(\det {(A)} = 0\).
    则存在不超过 \(n\),
    且互不相同的正整数 \(j\), \(k\), \(\ell\), 使
    \begin{align*}
                &
        |[A]_{j,j}|\,|[A]_{k,k}|\, |[A]_{\ell,\ell}|
        \\
        \leq {} &
        \Bigg(
        \sum_{\substack{1 \leq p \leq n \\
                p \neq j}} {|[A]_{j,p}|}
        \Bigg)
        \Bigg(
        \sum_{\substack{1 \leq q \leq n \\
                q \neq k}} {|[A]_{k,q}|}
        \Bigg)
        \Bigg(
        \sum_{\substack{1 \leq r \leq n \\
                r \neq \ell}} {|[A]_{\ell,r}|}
        \Bigg).
    \end{align*}
\end{quotation}
不过, 这不是对的.

\begin{example}
    设 \(A =
    \begin{bmatrix}
        1 & 1 & 1  \\
        1 & 1 & 1  \\
        1 & 1 & 10 \\
    \end{bmatrix}\)
    是一个 \(3\)~级阵.
    不难算出, \(\det {(A)} = 0\).
    可是
    (此处 \(j\), \(k\), \(\ell\) 分别是 \(1\), \(2\), \(3\))
    \begin{align*}
             &
        |[A]_{j,j}|\,|[A]_{k,k}|\, |[A]_{\ell,\ell}|
        \\
        = {} &
        1 \cdot 1 \cdot 10
        \\
        > {} &
        2 \cdot 2 \cdot 2
        \\
        = {} &
        \Bigg(
        \sum_{\substack{1 \leq p \leq 3 \\
                p \neq j}} {|[A]_{j,p}|}
        \Bigg)
        \Bigg(
        \sum_{\substack{1 \leq q \leq 3 \\
                q \neq k}} {|[A]_{k,q}|}
        \Bigg)
        \Bigg(
        \sum_{\substack{1 \leq r \leq 3 \\
                r \neq \ell}} {|[A]_{\ell,r}|}
        \Bigg).
    \end{align*}
    \(j\), \(k\), \(\ell\) 取其他的数时,
    也有类似的结果.
\end{example}

最后, 注意到, 一个阵与其转置的行列式相等:
若 \(\det {(A)} = 0\),
则 \(\det {(A^{\mathrm{T}})} = \det {(A)} = 0\).
应用前二个定理于 \(A^{\mathrm{T}}\), 立得

\begin{theorem}
    设 \(A\) 是一个 \(n\)~级阵.
    设 \(\det {(A)} = 0\).
    则存在不超过 \(n\) 的正整数 \(i\), 使
    \begin{align*}
        |[A]_{i,i}| \leq
        \sum_{\substack{1 \leq u \leq n \\
            u \neq i}} {|[A]_{u,i}|}.
    \end{align*}
\end{theorem}

\begin{theorem}
    设 \(A\) 是一个 \(n\)~级阵 (\(n \geq 2\)).
    设 \(\det {(A)} = 0\).
    则存在不超过 \(n\), 且不等的正整数 \(j\), \(k\), 使
    \begin{align*}
        |[A]_{j,j}|\,|[A]_{k,k}| \leq
        \Bigg(
        \sum_{\substack{1 \leq p \leq n \\
                p \neq j}} {|[A]_{p,j}|}
        \Bigg)
        \Bigg(
        \sum_{\substack{1 \leq q \leq n \\
                q \neq k}} {|[A]_{q,k}|}
        \Bigg).
    \end{align*}
\end{theorem}

\section{判断阵的行列式不是零的方法}

我们知道, 若一个阵的行列式是零,
则有形如式~\eqref{eq:NonzeroDet1}, \eqref{eq:NonzeroDet2} 的不等式成立.
利用反证法, 我们立得判断阵的行列式不是零的方法:

\begin{theorem}
    设 \(A\) 是一个 \(n\)~级阵.
    设对任何不超过 \(n\) 的正整数 \(i\), 必
    \begin{align*}
        |[A]_{i,i}| >
        \sum_{\substack{1 \leq u \leq n \\
            u \neq i}} {|[A]_{i,u}|}.
    \end{align*}
    则 \(\det {(A)} \neq 0\).
\end{theorem}

\begin{theorem}
    设 \(A\) 是一个 \(n\)~级阵 (\(n \geq 2\)).
    设对任何不超过 \(n\), 且不等的正整数 \(j\), \(k\), 必
    \begin{align*}
        |[A]_{j,j}|\,|[A]_{k,k}| >
        \Bigg(
        \sum_{\substack{1 \leq p \leq n \\
                p \neq j}} {|[A]_{j,p}|}
        \Bigg)
        \Bigg(
        \sum_{\substack{1 \leq q \leq n \\
                q \neq k}} {|[A]_{k,q}|}
        \Bigg).
    \end{align*}
    则 \(\det {(A)} \neq 0\).
\end{theorem}

当然, 如下命题也是正确的 (利用转置):

\begin{theorem}
    设 \(A\) 是一个 \(n\)~级阵.
    设对任何不超过 \(n\) 的正整数 \(i\), 必
    \begin{align*}
        |[A]_{i,i}| >
        \sum_{\substack{1 \leq u \leq n \\
            u \neq i}} {|[A]_{u,i}|}.
    \end{align*}
    则 \(\det {(A)} \neq 0\).
\end{theorem}

\begin{theorem}
    设 \(A\) 是一个 \(n\)~级阵 (\(n \geq 2\)).
    设对任何不超过 \(n\), 且不等的正整数 \(j\), \(k\), 必
    \begin{align*}
        |[A]_{j,j}|\,|[A]_{k,k}| >
        \Bigg(
        \sum_{\substack{1 \leq p \leq n \\
                p \neq j}} {|[A]_{p,j}|}
        \Bigg)
        \Bigg(
        \sum_{\substack{1 \leq q \leq n \\
                q \neq k}} {|[A]_{q,k}|}
        \Bigg).
    \end{align*}
    则 \(\det {(A)} \neq 0\).
\end{theorem}

这些定理的一个应用或许是,
对于一类阵,
即使我们不计算它的行列式,
我们也可判断它的行列式不是零.
这是好的.

\begin{example}\label{emp:NonzeroDet1}
    设 \(A =
    \begin{bmatrix}
        9 & 6 & 2 \\
        4 & 8 & 3 \\
        5 & 1 & 7 \\
    \end{bmatrix}\)
    是一个 \(3\)~级阵.
    不难算出, \(|9| > |6| + |2|\),
    \(|8| > |4| + |3|\),
    且 \(|7| > |5| + |1|\).
    于是, \(\det {(A)} \neq 0\).
    (其实, 不难算出, \(\det {(A)} = 327.\))
\end{example}

有时, 一个方阵 \(A\) 可能不适合定理的条件, 故我们无法直接地用定理.
不过, 既然我们研究是否 \(\det {(A)} \neq 0\),
我们可找二个跟 \(A\) 同尺寸的阵 \(B\), \(C\),
使 \(BAC\) 适合定理的条件.
则 \(\det {(BAC)} \neq 0\).
因为 \(0 \neq \det {(BAC)} = \det {(B)} \det {(A)} \det {(C)}\),
必 \(\det {(A)} \neq 0\).

\begin{example}\label{emp:NonzeroDet2}
    设 \(A =
    \begin{bmatrix}
        20 & 11 & 1 \\
        4  & 15 & 1 \\
        13 & 15 & 3 \\
    \end{bmatrix}\)
    是一个 \(3\)~级阵.
    不难验算, 我们无法直接地用前 4~个定理的任何一个%
    判断 \(\det {(A)}\) 是否非零.

    取
    \begin{align*}
        B = \begin{bmatrix}
                1 & 0 & 0 \\
                0 & 2 & 0 \\
                0 & 0 & 1 \\
            \end{bmatrix},
        \quad
        C = \begin{bmatrix}
                1 & 0 & 0  \\
                0 & 1 & 0  \\
                0 & 0 & 10 \\
            \end{bmatrix}.
    \end{align*}
    则
    \begin{align*}
        BAC = \begin{bmatrix}
                  20 & 11 & 10 \\
                  8  & 30 & 20 \\
                  13 & 15 & 30 \\
              \end{bmatrix}.
    \end{align*}
    不难算出
    \begin{align*}
         & R_1 = \sum_{\substack{1 \leq u \leq 3 \\u \neq 1}}
        {|[BAC]_{1,u}|} = |11| + |10| = 21,      \\
         & R_2 = \sum_{\substack{1 \leq u \leq 3 \\u \neq 2}}
        {|[BAC]_{2,u}|} = |8| + |20| = 28,       \\
         & R_3 = \sum_{\substack{1 \leq u \leq 3 \\u \neq 3}}
        {|[BAC]_{3,u}|} = |13| + |15| = 28,
    \end{align*}
    且
    \begin{align*}
         & |[BAC]_{1,1}|\,|[BAC]_{2,2}|
        = 600 > 588 = R_1 R_2,          \\
         & |[BAC]_{1,1}|\,|[BAC]_{3,3}|
        = 600 > 588 = R_1 R_3,          \\
         & |[BAC]_{2,2}|\,|[BAC]_{3,3}|
        = 30^2 > 28^2 = R_2 R_3.
    \end{align*}
    (注意到, 既然我们已算出,
    每个 \(|[BAC]_{j,j}|\,|[BAC]_{k,k}|\)
    都大于 \(R_j R_k\)
    (\(j < k\)),
    由乘法的交换律,
    我们不必再判断%
    每个 \(|[BAC]_{j,j}|\,|[BAC]_{k,k}|\)
    是否都大于 \(R_j R_k\)
    (\(j > k\)).)
    故 \(\det {(BAC)} \neq 0\).
    则 \(\det {(A)} \neq 0\).
    (其实, 不难算出, \(\det {(A)} = 476.\))
\end{example}

\begin{example}\label{emp:NonzeroDet3}
    设 \(n \geq 2\).
    设 \(n\)~级阵 \(A\) 适合
    \begin{align*}
        [A]_{i,j} =
        \begin{cases}
            2,
             & 1 \leq i = j \leq n;         \\
            \text{\(1\) 或 \(-1\)},
             & 1 \leq i = j - 1 \leq n - 1; \\
            \text{\(1\) 或 \(-1\)},
             & 2 \leq i = j + 1 \leq n;     \\
            0,
             & \text{其他}.
        \end{cases}
    \end{align*}
    形象地, 当 \(n = 4\) 时, \(A\) 形如
    \begin{align*}
        \begin{bmatrix}
            2     & \pm 1 & 0     & 0     \\
            \pm 1 & 2     & \pm 1 & 0     \\
            0     & \pm 1 & 2     & \pm 1 \\
            0     & 0     & \pm 1 & 2     \\
        \end{bmatrix}.
    \end{align*}
    (这里, 正负号可自由地组合.)

    不难算出, 对 \(i = 1\) 或 \(i = n\), 有
    \begin{align*}
        |[A]_{i,i}| = 2 > 1
        = \sum_{\substack{1 \leq u \leq n \\ u \neq i}}
        {|[A]_{i,u}|};
    \end{align*}
    对 \(1 < i < n\), 有
    \begin{align*}
        |[A]_{i,i}| = 2 = 2
        = \sum_{\substack{1 \leq u \leq n \\ u \neq i}}
        {|[A]_{i,u}|}.
    \end{align*}
    于是, 对 \(n = 2\), 我们可用第~1~个定理,
    得到 \(\det {(A)} \neq 0\);
    对 \(n = 3\), 我们可用第~2~个定理,
    得到 \(\det {(A)} \neq 0\);
    对 \(n \geq 4\), 这 4~个定理无法直接地被使用.

    我们试找一个 \(n\)~级阵 \(C\),
    使 \(AC\) 适合某个定理的条件,
    从而有 \(\det {(A)} \neq 0\).
    无妨设
    \begin{align*}
        C =
        \begin{bmatrix}
            c_1    & 0      & \cdots & 0      \\
            0      & c_2    & \cdots & 0      \\
            \vdots & \vdots & \ddots & \vdots \\
            0      & 0      & \cdots & c_n    \\
        \end{bmatrix},
    \end{align*}
    其中 \(c_1\), \(c_2\), \(\dots\), \(c_n\) 是待定的非零数.
    具体地,
    \begin{align*}
        [C]_{i,j}
        = \begin{cases}
              c_i, & 1 \leq i = j \leq n; \\
              0,   & \text{其他}.
          \end{cases}
    \end{align*}
    考虑到, 在后面的计算中, \(c_i\) 会被经常地取绝对值.
    既然如此, 为方便, 我们不如要求 \(c_i > 0\).
    不难算出, \([AC]_{i,j} = c_j [A]_{i,j}\);
    特别地, \(|[AC]_{i,i}| = 2c_i\).
    记
    \begin{align*}
        R_i = \sum_{\substack{1 \leq u \leq n \\ u \neq i}}
        {|[AC]_{i,u}|}.
    \end{align*}
    则
    \begin{align*}
        R_i =
        \begin{cases}
            c_2,               & i = 1;     \\
            c_{i-1} + c_{i+1}, & 1 < i < n; \\
            c_{n-1},           & i = n.
        \end{cases}
    \end{align*}
    我们希望, 找一组正数 \(c_1\), \(c_2\), \(\dots\), \(c_n\),
    使 \(|[AC]_{i,i}| > R_i\), 即
    \begin{align*}
        2c_1     & > c_2,           \\
        2c_2     & > c_1 + c_3,     \\
        2c_3     & > c_2 + c_4,     \\
        \dots \dots \dots
                 &
        \dots \dots \dots \dots,    \\
        2c_{n-1} & > c_{n-2} + c_n, \\
        2c_n     & > c_{n-1}.
    \end{align*}
    这些不等式相当于
    \begin{align*}
        c_1               & > c_2 - c_1,     \\
        c_2 - c_1         & > c_3 - c_2,     \\
        c_3 - c_2         & > c_4 - c_3,     \\
        \dots \dots \dots \dots
                          &
        \dots \dots \dots \dots \dots,       \\
        c_{n-1} - c_{n-2} & > c_n - c_{n-1}, \\
        c_n - c_{n-1}     & > -c_n.
    \end{align*}
    我们可试取
    \begin{align*}
        c_1               &
        = \frac{1}{1 \cdot 2}
        = 1 - \frac{1}{2},                   \\
        c_2 - c_1         &
        = \frac{1}{2 \cdot 3}
        = \frac{1}{2} - \frac{1}{3},         \\
        c_3 - c_2         &
        = \frac{1}{3 \cdot 4}
        = \frac{1}{3} - \frac{1}{4},         \\
        \dots \dots \dots \dots
                          &
        \dots \dots \dots \dots \dots \dots, \\
        c_{n-1} - c_{n-2} &
        = \frac{1}{(n-1)n}
        = \frac{1}{n-1} - \frac{1}{n},       \\
        c_n - c_{n-1}     &
        = \frac{1}{n(n+1)}
        = \frac{1}{n} - \frac{1}{n+1}.
    \end{align*}
    即
    \begin{align*}
        c_k = 1 - \frac{1}{k+1} = \frac{k}{k+1},
        \quad
        \text{\(k = 1\), \(2\), \(\dots\), \(n\)}.
    \end{align*}
    不难验证, \(c_k\) 是正数, 且
    \begin{align*}
        |[AC]_{1,1}| - R_1
         & = 2c_1 - c_2
        = 1 - \frac{2}{3} > 0,                 \\
        |[AC]_{n,n}| - R_n
         & = 2c_n - c_{n-1}
        = \frac{n-1}{n+1} + \frac{1}{n} > 0,   \\
        |[AC]_{i,i}| - R_i
         & = (c_i - c_{i-1}) - (c_{i+1} - c_i)
        = \frac{1}{i(i+1)} - \frac{1}{(i+1)(i+2)} > 0.
    \end{align*}
    故 \(AC\) 适合第~1~个定理的条件.
    则 \(\det {(AC)} \neq 0\).
    故 \(\det {(A)} \neq 0\).
\end{example}

\section{判断实阵的行列式大于零的方法}

最后, 我介绍判断实阵的行列式大于零的方法.

\begin{theorem}
    设 \(A\) 是一个 \(n\)~级\emph{实阵}.
    设对任何不超过 \(n\) 的正整数 \(i\), 必
    \begin{align*}
        [A]_{i,i} >
        \sum_{\substack{1 \leq u \leq n \\
            u \neq i}} {|[A]_{i,u}|}.
    \end{align*}
    则 \(\det {(A)} > 0\).
\end{theorem}

% 若我们巧用微积分, 此事是简单的.

% \begin{proof}
%     既然 \([A]_{i,i}\) 大于一个非负数,
%     则 \(0 < [A]_{i,i} = |[A]|_{i,i}|\).

%     作 \(n \times n\)~实阵 \(A_t\) 如下:
%     \begin{align*}
%         [A_t]_{i,j} =
%         \begin{cases}
%             [A]_{i,i},   & 1 \leq i = j \leq n; \\
%             t [A]_{i,j}, & \text{其他}.
%         \end{cases}
%     \end{align*}
%     作函数
%     \begin{align*}
%         \text{\(f\):} \quad
%         [0, 1] & \to \mathbb{R},       \\
%         t      & \mapsto \det {(A_t)}.
%     \end{align*}
%     对任何 \(0 \leq t \leq 1\),
%     与任何不超过 \(n\) 的正整数 \(i\),
%     \begin{align*}
%         |[A_t]_{i,i}|
%         = {}    &
%         |[A]_{i,i}|
%         \\
%         = {}    &
%         [A]_{i,i}
%         \\
%         > {}    &
%         \sum_{\substack{1 \leq u \leq n   \\
%             u \neq i}} {|[A]_{i,u}|}
%         \\
%         \geq {} &
%         t \sum_{\substack{1 \leq u \leq n \\
%             u \neq i}} {|[A]_{i,u}|}.
%         \\
%         = {}    &
%         \sum_{\substack{1 \leq u \leq n   \\
%             u \neq i}} {t |[A]_{i,u}|}.
%         \\
%         = {}    &
%         \sum_{\substack{1 \leq u \leq n   \\
%             u \neq i}} {|t|\, |[A]_{i,u}|}.
%         \\
%         = {}    &
%         \sum_{\substack{1 \leq u \leq n   \\
%             u \neq i}} {|[A_t]_{i,u}|}.
%     \end{align*}
%     故 \(f(t) = \det {(A_t)} \neq 0\),
%     对任何 \(0 \leq t \leq 1\).

%     因为 \(\det {(A_t)}\) 是关于 \(t\) 的整式,
%     故 \(f\) 是连续函数.
%     注意到, \(f(0) = [A]_{1,1} [A]_{2,2} \dots [A]_{n,n} > 0\);
%     又注意到, \(f(1) = \det {(A)}\).

%     我们用反证法说明, \(f(1) > 0\).
%     反设 \(f(1) \leq 0\).
%     若 \(f(1) = 0\), 这是矛盾.
%     若 \(f(1) < 0\), 则因 \(f(1) > 0\),
%     必有某 \(0 < s < 1\) 使 \(f(s) = 0\);
%     这又是矛盾.

%     故 \(f(1) > 0\).
% \end{proof}

% 不过, 若我们不用微积分, 此事是较复杂的.

我想给二个证明.
第~1~个证明的计算较多, 但它是较直接的.

\begin{proof}[法~1]
    作命题 \(P(n)\):
    \begin{quotation}
        对任何适合如下条件的 \(n\)~级实阵 \(A\),
        必 \(\det {(A)} > 0\):
        \begin{quotation}
            对任何不超过 \(n\)~的正整数 \(i\),
            必
            \begin{align*}
                [A]_{i,i} >
                \sum_{\substack{1 \leq u \leq n \\
                    u \neq i}} {|[A]_{i,u}|}.
            \end{align*}
        \end{quotation}
    \end{quotation}
    我们用数学归纳法证明,
    对任何正整数 \(n\),
    \(P(n)\) 是对的.

    \(P(1)\) 是对的; 这是显然的.

    \(P(2)\) 是对的.
    任取 \(2\)~级实阵
    \begin{align*}
        A = \begin{bmatrix}
                a & b \\
                c & d \\
            \end{bmatrix},
    \end{align*}
    其中 \(a > |b|\), 且 \(d > |c|\).
    则
    \begin{align*}
        \det {(A)} = ad - bc > |b|\,|c| - bc
        = |bc| - bc \geq 0.
    \end{align*}

    设 \(P(n-1)\) 是对的.
    我们要由此证 \(P(n)\) 是对的.

    设 \(A\) 是一个 \(n\)~级实阵,
    且对任何不超过 \(n\) 的正整数 \(i\), 必
    \begin{align*}
        [A]_{i,i} >
        \sum_{\substack{1 \leq u \leq n \\
            u \neq i}} {|[A]_{i,u}|}.
    \end{align*}
    因为 \([A]_{i,i}\) 大于一个非负数,
    故 \(0 < [A]_{i,i} = |[A]_{i,i}|\).
    作 \(n\)~级实阵 \(B_1 = A\).
    则 \([B_1]_{i,j} = [A]_{i,j}\),
    且 \(\det {(B_1)} = \det {(A)}\).

    我们加 \(B_1\) 的行~\(1\) 的 \(-[A]_{2,1} / [A]_{1,1}\)~倍%
    到 \(B_1\) 的行~\(2\), 得 \(n\)~级实阵 \(B_2\).
    则
    \begin{align*}
        [B_2]_{i,j}  & = [B_1]_{i,j},
                     & \quad i < 2;                        \\
        [B_2]_{2,1}  & = 0;                                \\
        [B_2]_{2,j}  & =
        [A]_{2,j} - \frac{[A]_{2,1}}{[A]_{1,1}} [A]_{1,j}; \\
        \det {(B_2)} & =
        \det {(B_1)} = \det {(A)}.
    \end{align*}

    我们加 \(B_2\) 的行~\(1\) 的 \(-[A]_{3,1} / [A]_{1,1}\)~倍%
    到 \(B_2\) 的行~\(3\), 得 \(n\)~级实阵 \(B_3\).
    则
    \begin{align*}
        [B_3]_{i,j}  & = [B_2]_{i,j},
                     & \quad i < 3;                        \\
        [B_3]_{3,1}  & = 0;                                \\
        [B_3]_{3,j}  & =
        [A]_{3,j} - \frac{[A]_{3,1}}{[A]_{1,1}} [A]_{1,j}; \\
        \det {(B_3)} & =
        \det {(B_2)} = \det {(A)}.
    \end{align*}

    \(\dots \dots\)

    我们加 \(B_{n-1}\) 的行~\(1\) 的 \(-[A]_{n,1} / [A]_{1,1}\)~倍%
    到 \(B_{n-1}\) 的行~\(n\), 得 \(n\)~级实阵 \(B_n\).
    则
    \begin{align*}
        [B_n]_{i,j}  & = [B_{n-1}]_{i,j},
                     & \quad i < n;                        \\
        [B_n]_{n,1}  & = 0;                                \\
        [B_n]_{n,j}  & =
        [A]_{n,j} - \frac{[A]_{n,1}}{[A]_{1,1}} [A]_{1,j}; \\
        \det {(B_n)} & =
        \det {(B_{n-1})} = \det {(A)}.
    \end{align*}

    综上, 我们作出了一个 \(n\)~级实阵 \(B_n\) 适合如下条件:
    \begin{align*}
        [B_n]_{1,j}  & = [A]_{1,j};                 \\
        [B_n]_{i,1}  & = 0,          & \quad i > 1; \\
        [B_n]_{i,j}  & =
        [A]_{i,j} - \frac{[A]_{i,1}}{[A]_{1,1}} [A]_{1,j},
                     & \quad i > 1;                 \\
        \det {(B_n)} & = \det {(A)}.
    \end{align*}
    作 \(n-1\)~级实阵 \(C = B_n (1|1)\).
    则
    \begin{align*}
        [C]_{i,j}  & = [B_n]_{i+1,j+1}
        = [A]_{i+1,j+1} - \frac{[A]_{i+1,1}}{[A]_{1,1}} [A]_{1,j+1};
        \\
        \det {(A)} & = \det {(B_n)} = [B_n]_{1,1} \det {(B_n (1|1))}
            = [A]_{1,1} \det {(C)}.
    \end{align*}

    注意到, 若 \(j \neq k\), 则
    \begin{align*}
        [A]_{k,k} >
        \sum_{\substack{1 \leq u \leq n \\
            u \neq k}} {|[A]_{k,u}|}
        \geq |[A]_{k,j}|,
    \end{align*}
    故
    \begin{align*}
        [C]_{i,i}
        = {}    &
        [A]_{i+1,i+1} - \frac{[A]_{i+1,1}}{[A]_{1,1}} [A]_{1,i+1}
        \\
        = {}    &
        \frac{[A]_{i+1,i+1} [A]_{1,1} - [A]_{1,i+1} [A]_{i+1,1}}
        {[A]_{1,1}}
        \\
        > {}    &
        \frac{|[A]_{1,i+1}|\, |[A]_{i+1,1}| - [A]_{1,i+1} [A]_{i+1,1}}
        {[A]_{1,1}}
        \\
        = {}    &
        \frac{|{[A]_{1,i+1} [A]_{i+1,1}}| - [A]_{1,i+1} [A]_{i+1,1}}
        {[A]_{1,1}}
        \\
        \geq {} &
        0.
    \end{align*}
    故 \(0 < [C]_{i,i} = |[C]_{i,i}|\).

    注意到
    \begin{align*}
        |[C]_{i,i}|
        = {}    &
        \Bigg|
        [A]_{i+1,i+1} - \frac{[A]_{i+1,1}}{[A]_{1,1}} [A]_{1,i+1}
        \Bigg|
        \\
        \geq {} &
        |[A]_{i+1,i+1}| -
        \Bigg|
        \frac{[A]_{i+1,1}}{[A]_{1,1}} [A]_{1,i+1}
        \Bigg|
        \\
        = {}    &
        |[A]_{i+1,i+1}| -
        \frac{|[A]_{i+1,1}|}{|[A]_{1,1}|} |[A]_{1,i+1}|,
    \end{align*}
    且
    \begin{align*}
                &
        \sum_{\substack{1 \leq v \leq n-1  \\ v \neq i}}
        {|[C]_{i,v}|}
        \\
        = {}    &
        \sum_{\substack{2 \leq \ell \leq n \\ \ell \neq i+1}}
        {|[C]_{i,\ell-1}|}
        \\
        = {}    &
        \sum_{\substack{2 \leq \ell \leq n \\ \ell \neq i+1}}
        {\Bigg|
        [A]_{i+1,\ell} - \frac{[A]_{i+1,1}}{[A]_{1,1}} [A]_{1,\ell}
        \Bigg|}
        \\
        \leq {} &
        \sum_{\substack{2 \leq \ell \leq n \\ \ell \neq i+1}}
        {\Bigg(
            |[A]_{i+1,\ell}| +
            \Bigg|
            \frac{[A]_{i+1,1}}{[A]_{1,1}} [A]_{1,\ell}
            \Bigg|
            \Bigg)}
        \\
        = {}    &
        \sum_{\substack{2 \leq \ell \leq n \\ \ell \neq i+1}}
        {\Bigg(
            |[A]_{i+1,\ell}| +
            \frac{|[A]_{i+1,1}|}{|[A]_{1,1}|} |[A]_{1,\ell}|
            \Bigg)}
        \\
        = {}    &
        \sum_{\substack{2 \leq \ell \leq n \\ \ell \neq i+1}}
        {
        |[A]_{i+1,\ell}|
        }
        +
        \frac{|[A]_{i+1,1}|}{|[A]_{1,1}|}
        \sum_{\substack{2 \leq \ell \leq n \\ \ell \neq i+1}}
        {
        |[A]_{1,\ell}|
        }
        \\
        = {}    &
        \sum_{\substack{1 \leq \ell \leq n \\ \ell \neq i+1}}
        {
        |[A]_{i+1,\ell}|
        }
        - |[A]_{i+1,1}|
        +
        \frac{|[A]_{i+1,1}|}{|[A]_{1,1}|}
        \sum_{2 \leq \ell \leq n}
        {
        |[A]_{1,\ell}|
        }
        \\
        {}      &
        \hphantom{=}
        -
        \frac{|[A]_{i+1,1}|}{|[A]_{1,1}|}
        |[A]_{1,i+1}|
        \\
        = {}    &
        \sum_{\substack{1 \leq \ell \leq n \\ \ell \neq i+1}}
        {
        |[A]_{i+1,\ell}|
        }
        - \frac{|[A]_{i+1,1}|}{|[A]_{1,1}|} |[A]_{1,1}|
        +
        \frac{|[A]_{i+1,1}|}{|[A]_{1,1}|}
        \sum_{2 \leq \ell \leq n}
        {
        |[A]_{1,\ell}|
        }
        \\
        {}      &
        \hphantom{=}
        -
        \frac{|[A]_{i+1,1}|}{|[A]_{1,1}|}
        |[A]_{1,i+1}|
        \\
        = {}    &
        \sum_{\substack{1 \leq \ell \leq n \\ \ell \neq i+1}}
        {
        |[A]_{i+1,\ell}|
        }
        - \frac{|[A]_{i+1,1}|}{|[A]_{1,1}|}
        \Bigg(
        |[A]_{1,1}|
        -
        \sum_{2 \leq \ell \leq n}
        {
            |[A]_{1,\ell}|
        }
        \Bigg)
        \\
        {}      &
        \hphantom{=}
        -
        \frac{|[A]_{i+1,1}|}{|[A]_{1,1}|}
        |[A]_{1,i+1}|
        \\
        \leq {} &
        \sum_{\substack{1 \leq \ell \leq n \\ \ell \neq i+1}}
        {
        |[A]_{i+1,\ell}|
        }
        -
        \frac{|[A]_{i+1,1}|}{|[A]_{1,1}|}
        |[A]_{1,i+1}|.
    \end{align*}
    故
    \begin{align*}
        |[C]_{i,i}| -
        \sum_{\substack{1 \leq v \leq n-1  \\ v \neq i}}
        {|[C]_{i,v}|}
        \geq
        |[A]_{i+1,i+1}| -
        \sum_{\substack{1 \leq \ell \leq n \\ \ell \neq i+1}}
        {
        |[A]_{i+1,\ell}|
        }
        > 0.
    \end{align*}

    综上, 由假定, \(\det {(C)} > 0\).

    既然 \([A]_{1,1} > 0\),
    且 \(\det {(C)} > 0\),
    故 \(\det {(A)} = [A]_{1,1} \det {(C)} > 0\).

    所以, \(P(n)\) 是正确的.
    由数学归纳法原理, 待证命题成立.
\end{proof}

第~2~个证明的计算较少.
不过, 我们要注意一件小事.

\begin{theorem}
    设 \(m\), \(b\) 是实数.
    设 \(b > 0\), 且对任何不超过 \(1\) 的正数 \(t\),
    有 \(mt + b \neq 0\).
    则 \(m + b > 0\).
\end{theorem}

\begin{proof}
    用反证法.
    反设 \(m + b \leq 0\).
    则 \(0 < b \leq -m\).
    设 \(s = b/(-m)\).
    则 \(0 < s \leq 1\).
    可是, \(ms + b = 0\).
    这是矛盾.
\end{proof}

\begin{proof}[法~2]
    作命题 \(P(n)\):
    \begin{quotation}
        对任何适合如下条件的 \(n\)~级实阵 \(A\),
        必 \(\det {(A)} > 0\):
        \begin{quotation}
            对任何不超过 \(n\)~的正整数 \(i\),
            必
            \begin{align*}
                [A]_{i,i} >
                \sum_{\substack{1 \leq u \leq n \\
                    u \neq i}} {|[A]_{i,u}|}.
            \end{align*}
        \end{quotation}
    \end{quotation}
    我们用数学归纳法证明,
    对任何正整数 \(n\),
    \(P(n)\) 是对的.

    \(P(1)\) 是对的; 这是显然的.

    设 \(P(n-1)\) 是对的.
    我们要由此证 \(P(n)\) 是对的.

    设 \(A\) 是一个 \(n\)~级实阵,
    且对任何不超过 \(n\) 的正整数 \(i\), 必
    \begin{align*}
        [A]_{i,i} >
        \sum_{\substack{1 \leq u \leq n \\
            u \neq i}} {|[A]_{i,u}|}.
    \end{align*}
    因为 \([A]_{i,i}\) 大于一个非负数,
    故 \(0 < [A]_{i,i} = |[A]_{i,i}|\).

    作 \(n \times n\)~实阵 \(A_t\) 如下:
    \begin{align*}
        [A_t]_{i,j} =
        \begin{cases}
            [A]_{i,j},     & i \neq 1;  \\
            [A]_{1,1},     & i = j = 1; \\
            t\, [A]_{1,j}, & \text{其他}.
        \end{cases}
    \end{align*}
    我们考虑 \(A_t\)~的行列式.
    按行~\(1\) 展开, 有
    \begin{align*}
        \det {(A_t)}
        = {} &
        \sum_{1 \leq j \leq n}
        {(-1)^{1 + j} [A_t]_{1,j} \det {(A_t (1 | j))}}
        \\
        = {} &
        \sum_{1 \leq j \leq n}
        {(-1)^{1 + j} [A_t]_{1,j} \det {(A (1 | j))}}
        \\
        = {} &
        \sum_{\substack{1 \leq j \leq n \\j \neq 1}}
        {(-1)^{1 + j} [A_t]_{1,j} \det {(A (1 | j))}}
        +
        (-1)^{1 + 1} [A_t]_{1,1} \det {(A (1 | 1))}
        \\
        = {} &
        \sum_{\substack{1 \leq j \leq n \\j \neq 1}}
        {(-1)^{1 + j} t\, [A]_{1,j} \det {(A (1 | j))}}
        +
        [A]_{1,1} \det {(A (1 | 1))}
        \\
        = {} &
        t
        \sum_{\substack{1 \leq j \leq n \\j \neq 1}}
        {(-1)^{1 + j} [A]_{1,j} \det {(A (1 | j))}}
        +
        [A]_{1,1} \det {(A (1 | 1))}.
    \end{align*}
    记
    \begin{align*}
        m & =
        \sum_{\substack{1 \leq j \leq n \\j \neq 1}}
        {(-1)^{1 + j} [A]_{1,j} \det {(A (1 | j))}},
        \\
        b & =
        [A]_{1,1} \det {(A (1 | 1))}.
    \end{align*}
    则 \(m\), \(b\) 是实数, 且
    \(\det {(A_t)} = mt + b\).

    作 \(n-1\)~级实阵 \(C = A(1 | 1)\),
    则 \([C]_{w,y} = [A]_{w+1,y+1}\).
    且对 \(w < n\),
    \begin{align*}
        [C]_{w,w}
        = {}    &
        [A]_{w+1,w+1}
        \\
        > {}    &
        \sum_{\substack{1 \leq u \leq n   \\u \neq w+1}}
        {|[A]_{w+1,u}|}
        \\
        \geq {} &
        \sum_{\substack{1 \leq u \leq n   \\u \neq 1, w+1}}
        {|[A]_{w+1,u}|}
        \\
        = {}    &
        \sum_{\substack{1 \leq y \leq n-1 \\y \neq w}}
        {|[A]_{w+1,y+1}|}
        \\
        = {}    &
        \sum_{\substack{1 \leq y \leq n-1 \\y \neq w}}
        {|[C]_{w,y}|}.
    \end{align*}
    所以, 由假定, \(\det {(A(1|1))} = \det {(C)} > 0\).
    因为 \([A]_{1,1} > 0\), 我们有 \(b > 0\).

    设 \(t\) 是不超过 \(1\)~的正数.
    则 \(0 \leq |t| = t \leq 1\).
    任取不超过 \(n\) 的正整数 \(i\).
    若 \(i \neq 1\), 则
    \begin{align*}
        |[A_t]_{i,i}|
        = {} &
        |[A]_{i,i}|
        \\
        = {} &
        [A]_{i,i}
        \\
        > {} &
        \sum_{\substack{1 \leq u \leq n \\u \neq i}}
        {|[A]_{i,u}|}
        \\
        = {} &
        \sum_{\substack{1 \leq u \leq n \\u \neq i}}
        {|[A_t]_{i,u}|}.
    \end{align*}
    若 \(i = 1\), 则
    \begin{align*}
        |[A_t]_{i,i}|
        = {}    &
        |[A]_{i,i}|
        \\
        = {}    &
        [A]_{i,i}
        \\
        > {}    &
        \sum_{\substack{1 \leq u \leq n     \\u \neq i}}
        {|[A]_{i,u}|}
        \\
        \geq {} &
        |t| \sum_{\substack{1 \leq u \leq n \\u \neq i}}
        {|[A]_{i,u}|}
        \\
        = {}    &
        \sum_{\substack{1 \leq u \leq n     \\u \neq i}}
        {|t|\, |[A]_{i,u}|}
        \\
        = {}    &
        \sum_{\substack{1 \leq u \leq n     \\u \neq i}}
        {|t\, [A]_{i,u}|}
        \\
        = {}    &
        \sum_{\substack{1 \leq u \leq n     \\u \neq i}}
        {|[A_t]_{i,u}|}.
    \end{align*}
    则 \(\det {(A_t)} \neq 0\).
    从而,
    对任何不超过 \(1\) 的正数 \(t\),
    有 \(mt + b \neq 0\).

    既然 \(b > 0\),
    且对任何不超过 \(1\) 的正数 \(t\),
    有 \(mt + b \neq 0\),
    我们能推出 \(m + b > 0\).
    故 \(\det {(A)} = \det {(A_1)} = m + b > 0\).

    所以, \(P(n)\) 是正确的.
    由数学归纳法原理, 待证命题成立.
\end{proof}

当然, 如下命题也是正确的 (利用转置):

\begin{theorem}
    设 \(A\) 是一个 \(n\)~级\emph{实阵}.
    设对任何不超过 \(n\) 的正整数 \(i\), 必
    \begin{align*}
        [A]_{i,i} >
        \sum_{\substack{1 \leq u \leq n \\
            u \neq i}} {|[A]_{u,i}|}.
    \end{align*}
    则 \(\det {(A)} > 0\).
\end{theorem}

\begingroup

我们回看上节的例.

\renewcommand\thmcontinues[1]{续}

\begin{example}[continues=emp:NonzeroDet1]
    设 \(A =
    \begin{bmatrix}
        9 & 6 & 2 \\
        4 & 8 & 3 \\
        5 & 1 & 7 \\
    \end{bmatrix}\)
    是一个 \(3\)~级实阵.
    不难算出, \(9 > |6| + |2|\),
    \(8 > |4| + |3|\),
    且 \(7 > |5| + |1|\).
    于是, \(\det {(A)} > 0\).
    (其实, 不难算出, \(\det {(A)} = 327.\))
\end{example}

有时, 一个实方阵 \(A\) 可能不适合定理的条件, 故我们无法直接地用定理.
不过, 既然我们研究是否 \(\det {(A)} > 0\),
我们可找二个跟 \(A\) 同尺寸的实阵 \(B\), \(C\),
使 \(\det {(B)} \det {(C)} > 0\),
且 \(BAC\) 适合定理的条件.
则 \(\det {(BAC)} > 0\).
因为 \(0 < \det {(BAC)} = \det {(B)} \det {(A)} \det {(C)}\),
且 \(\det {(B)} \det {(C)} > 0\),
故 \(\det {(A)} > 0\).

\begin{example}[continues=emp:NonzeroDet3]
    设 \(n \geq 2\).
    设 \(n\)~级实阵 \(A\) 适合
    \begin{align*}
        [A]_{i,j} =
        \begin{cases}
            2,
             & 1 \leq i = j \leq n;         \\
            \text{\(1\) 或 \(-1\)},
             & 1 \leq i = j - 1 \leq n - 1; \\
            \text{\(1\) 或 \(-1\)},
             & 2 \leq i = j + 1 \leq n;     \\
            0,
             & \text{其他}.
        \end{cases}
    \end{align*}
    我们无法直接地用定理说明 \(\det {(A)} > 0\).

    记 \(c_k = 1 - 1/(k+1) = k/(k+1) > 0\),
    \(k = 1\), \(2\), \(\dots\), \(n\).
    作 \(n\)~级实阵
    \begin{align*}
        C =
        \begin{bmatrix}
            c_1    & 0      & \cdots & 0      \\
            0      & c_2    & \cdots & 0      \\
            \vdots & \vdots & \ddots & \vdots \\
            0      & 0      & \cdots & c_n    \\
        \end{bmatrix};
    \end{align*}
    具体地,
    \begin{align*}
        [C]_{i,j}
        = \begin{cases}
              c_i, & 1 \leq i = j \leq n; \\
              0,   & \text{其他}.
          \end{cases}
    \end{align*}
    不难算出, \(\det {(C)} > 0\)
    (其实, \(\det {(C)} = 1/(n+1)\)).

    现在, 我们证明, \(\det {(AC)} > 0\).

    首先, \(AC\) 当然是一个实阵.
    不难算出, \([AC]_{i,j} = c_j\, [A]_{i,j}\);
    特别地, \([AC]_{i,i} = 2c_i\).
    记
    \begin{align*}
        R_i = \sum_{\substack{1 \leq u \leq n \\ u \neq i}}
        {|[AC]_{i,u}|}.
    \end{align*}
    则
    \begin{align*}
        R_i =
        \begin{dcases}
            c_2,               & i = 1;     \\
            c_{i-1} + c_{i+1}, & 1 < i < n; \\
            c_{n-1},           & i = n.
        \end{dcases}
    \end{align*}
    不难验证,
    \begin{align*}
        [AC]_{1,1} - R_1
         & = 2c_1 - c_2
        = 1 - \frac{2}{3} > 0,                 \\
        [AC]_{n,n} - R_n
         & = 2c_n - c_{n-1}
        = \frac{n-1}{n+1} + \frac{1}{n} > 0,   \\
        [AC]_{i,i} - R_i
         & = (c_i - c_{i-1}) - (c_{i+1} - c_i)
        = \frac{1}{i(i+1)} - \frac{1}{(i+1)(i+2)} > 0.
    \end{align*}
    于是, \(\det {(AC)} > 0\).

    既然 \(0 < \det {(AC)} = \det {(A)} \det {(C)}\),
    且 \(\det {(C)} > 0\),
    故 \(\det {(A)} > 0\).
\end{example}

我们还有如下定理.

\begin{theorem}
    设 \(A\) 是一个 \(n\)~级\emph{实阵} (\(n \geq 2\)).
    设对任何不超过 \(n\) 的正整数 \(i\), 必
    \([A]_{i,i} > 0\).
    设对任何不超过 \(n\), 且不等的正整数 \(j\), \(k\), 必
    \begin{align*}
        [A]_{j,j}\,[A]_{k,k} >
        \Bigg(
        \sum_{\substack{1 \leq p \leq n \\
                p \neq j}} {|[A]_{j,p}|}
        \Bigg)
        \Bigg(
        \sum_{\substack{1 \leq q \leq n \\
                q \neq k}} {|[A]_{k,q}|}
        \Bigg).
    \end{align*}
    则 \(\det {(A)} > 0\).
\end{theorem}

\begin{proof}
    作命题 \(P(n)\):
    \begin{quotation}
        对任何适合如下条件的 \(n\)~级实阵 \(A\),
        必 \(\det {(A)} > 0\):
        \begin{quotation}
            (1)
            对任何不超过 \(n\)~的正整数 \(i\),
            必 \([A]_{i,i} > 0\);

            (2)
            对任何不超过 \(n\), 且不等的正整数 \(j\), \(k\), 必
            \begin{align*}
                [A]_{j,j}\,[A]_{k,k} >
                \Bigg(
                \sum_{\substack{1 \leq p \leq n \\
                        p \neq j}} {|[A]_{j,p}|}
                \Bigg)
                \Bigg(
                \sum_{\substack{1 \leq q \leq n \\
                        q \neq k}} {|[A]_{k,q}|}
                \Bigg).
            \end{align*}
        \end{quotation}
    \end{quotation}
    我们用数学归纳法证明,
    对任何高于 \(1\)~的正整数 \(n\),
    \(P(n)\) 是对的.

    \(P(2)\) 是对的.
    任取 \(2\)~级实阵
    \begin{align*}
        A = \begin{bmatrix}
                a & b \\
                c & d \\
            \end{bmatrix},
    \end{align*}
    其中 \(a d > |b|\,|c|\).
    则
    \begin{align*}
        \det {(A)} = ad - bc > |b|\,|c| - bc
        = |bc| - bc \geq 0.
    \end{align*}

    设 \(P(n-1)\) 是对的.
    我们要由此证 \(P(n)\) 是对的.

    设 \(A\) 是一个 \(n\)~级实阵.
    设对任何不超过 \(n\) 的正整数 \(i\), 必
    \([A]_{i,i} > 0\).
    设对任何不超过 \(n\), 且不等的正整数 \(j\), \(k\), 必
    \begin{align*}
        [A]_{j,j}\,[A]_{k,k} >
        \Bigg(
        \sum_{\substack{1 \leq p \leq n \\
                p \neq j}} {|[A]_{j,p}|}
        \Bigg)
        \Bigg(
        \sum_{\substack{1 \leq q \leq n \\
                q \neq k}} {|[A]_{k,q}|}
        \Bigg).
    \end{align*}

    作 \(n \times n\)~实阵 \(A_t\) 如下:
    \begin{align*}
        [A_t]_{i,j} =
        \begin{cases}
            [A]_{i,j},     & i \neq 1;  \\
            [A]_{1,1},     & i = j = 1; \\
            t\, [A]_{1,j}, & \text{其他}.
        \end{cases}
    \end{align*}
    我们考虑 \(A_t\)~的行列式.
    按行~\(1\) 展开, 有
    \begin{align*}
        \det {(A_t)}
        = {} &
        \sum_{1 \leq j \leq n}
        {(-1)^{1 + j} [A_t]_{1,j} \det {(A_t (1 | j))}}
        \\
        = {} &
        \sum_{1 \leq j \leq n}
        {(-1)^{1 + j} [A_t]_{1,j} \det {(A (1 | j))}}
        \\
        = {} &
        \sum_{\substack{1 \leq j \leq n \\j \neq 1}}
        {(-1)^{1 + j} [A_t]_{1,j} \det {(A (1 | j))}}
        +
        (-1)^{1 + 1} [A_t]_{1,1} \det {(A (1 | 1))}
        \\
        = {} &
        \sum_{\substack{1 \leq j \leq n \\j \neq 1}}
        {(-1)^{1 + j} t\, [A]_{1,j} \det {(A (1 | j))}}
        +
        [A]_{1,1} \det {(A (1 | 1))}
        \\
        = {} &
        t
        \sum_{\substack{1 \leq j \leq n \\j \neq 1}}
        {(-1)^{1 + j} [A]_{1,j} \det {(A (1 | j))}}
        +
        [A]_{1,1} \det {(A (1 | 1))}.
    \end{align*}
    记
    \begin{align*}
        m & =
        \sum_{\substack{1 \leq j \leq n \\j \neq 1}}
        {(-1)^{1 + j} [A]_{1,j} \det {(A (1 | j))}},
        \\
        b & =
        [A]_{1,1} \det {(A (1 | 1))}.
    \end{align*}
    则 \(m\), \(b\) 是实数, 且
    \(\det {(A_t)} = mt + b\).

    作 \(n-1\)~级实阵 \(C = A(1 | 1)\).
    则 \([C]_{w,y} = [A]_{w+1,y+1}\),
    且对 \(v\), \(w < n\),
    且 \(v \neq w\),
    \begin{align*}
        [C]_{v,v}\, [C]_{w,w}
        = {}    &
        [A]_{v+1,v+1}\, [A]_{w+1,w+1}
        \\
        > {}    &
        \Bigg(
        \sum_{\substack{1 \leq u \leq n   \\u \neq v+1}}
        {|[A]_{v+1,u}|}
        \Bigg)
        \Bigg(
        \sum_{\substack{1 \leq u \leq n   \\u \neq w+1}}
        {|[A]_{w+1,u}|}
        \Bigg)
        \\
        \geq {} &
        \Bigg(
        \sum_{\substack{1 \leq u \leq n   \\u \neq 1, v+1}}
        {|[A]_{v+1,u}|}
        \Bigg)
        \Bigg(
        \sum_{\substack{1 \leq u \leq n   \\u \neq 1, w+1}}
        {|[A]_{w+1,u}|}
        \Bigg)
        \\
        = {}    &
        \Bigg(
        \sum_{\substack{1 \leq y \leq n-1 \\y \neq v}}
        {|[A]_{v+1,y+1}|}
        \Bigg)
        \Bigg(
        \sum_{\substack{1 \leq y \leq n-1 \\y \neq w}}
        {|[A]_{w+1,y+1}|}
        \Bigg)
        \\
        = {}    &
        \Bigg(
        \sum_{\substack{1 \leq y \leq n-1 \\y \neq v}}
        {|[C]_{v,y}|}
        \Bigg)
        \Bigg(
        \sum_{\substack{1 \leq y \leq n-1 \\y \neq w}}
        {|[C]_{w,y}|}
        \Bigg).
    \end{align*}
    所以, 由假定, \(\det {(A(1|1))} = \det {(C)} > 0\).
    因为 \([A]_{1,1} > 0\), 我们有 \(b > 0\).

    设 \(t\) 是不超过 \(1\)~的正数.
    则 \(0 \leq |t| = t \leq 1\).
    任取不超过 \(n\) 的正整数 \(j\), \(k\),
    且 \(j \neq k\).
    若 \(j \neq 1\) 且 \(k \neq 1\), 则
    \begin{align*}
        |[A_t]_{j,j}|\,|[A_t]_{k,k}|
        = {} &
        |[A]_{j,j}|\,|[A]_{k,k}|
        \\
        = {} &
        [A]_{j,j}\,[A]_{k,k}
        \\
        > {} &
        \Bigg(
        \sum_{\substack{1 \leq p \leq n \\
                p \neq j}} {|[A]_{j,p}|}
        \Bigg)
        \Bigg(
        \sum_{\substack{1 \leq q \leq n \\
                q \neq k}} {|[A]_{k,q}|}
        \Bigg)
        \\
        = {} &
        \Bigg(
        \sum_{\substack{1 \leq p \leq n \\
                p \neq j}} {|[A_t]_{j,p}|}
        \Bigg)
        \Bigg(
        \sum_{\substack{1 \leq q \leq n \\
                q \neq k}} {|[A_t]_{k,q}|}
        \Bigg).
    \end{align*}
    若 \(j = 1\) 且 \(k \neq j\), 则
    \begin{align*}
        |[A_t]_{j,j}|\,|[A_t]_{k,k}|
        = {}    &
        |[A]_{j,j}|\,|[A]_{k,k}|
        \\
        = {}    &
        [A]_{j,j}\,[A]_{k,k}
        \\
        > {}    &
        \Bigg(
        \sum_{\substack{1 \leq p \leq n \\
                p \neq j}} {|[A]_{j,p}|}
        \Bigg)
        \Bigg(
        \sum_{\substack{1 \leq q \leq n \\
                q \neq k}} {|[A]_{k,q}|}
        \Bigg)
        \\
        \geq {} &
        |t|\,\Bigg(
        \sum_{\substack{1 \leq p \leq n \\
                p \neq j}} {|[A]_{j,p}|}
        \Bigg)
        \Bigg(
        \sum_{\substack{1 \leq q \leq n \\
                q \neq k}} {|[A]_{k,q}|}
        \Bigg)
        \\
        = {}    &
        \Bigg(
        \sum_{\substack{1 \leq p \leq n \\
                p \neq j}} {|t|\,|[A]_{j,p}|}
        \Bigg)
        \Bigg(
        \sum_{\substack{1 \leq q \leq n \\
                q \neq k}} {|[A]_{k,q}|}
        \Bigg)
        \\
        = {}    &
        \Bigg(
        \sum_{\substack{1 \leq p \leq n \\
                p \neq j}} {|t\, [A]_{j,p}|}
        \Bigg)
        \Bigg(
        \sum_{\substack{1 \leq q \leq n \\
                q \neq k}} {|[A]_{k,q}|}
        \Bigg)
        \\
        = {}    &
        \Bigg(
        \sum_{\substack{1 \leq p \leq n \\
                p \neq j}} {|[A_t]_{j,p}|}
        \Bigg)
        \Bigg(
        \sum_{\substack{1 \leq q \leq n \\
                q \neq k}} {|[A_t]_{k,q}|}
        \Bigg).
    \end{align*}
    若 \(k = 1\) 且 \(j \neq k\),
    类似地, 我们也有
    \begin{align*}
        |[A_t]_{j,j}|\,|[A_t]_{k,k}|
        > {} &
        \Bigg(
        \sum_{\substack{1 \leq p \leq n \\
                p \neq j}} {|[A_t]_{j,p}|}
        \Bigg)
        \Bigg(
        \sum_{\substack{1 \leq q \leq n \\
                q \neq k}} {|[A_t]_{k,q}|}
        \Bigg).
    \end{align*}
    则 \(\det {(A_t)} \neq 0\).
    从而,
    对任何不超过 \(1\) 的正数 \(t\),
    有 \(mt + b \neq 0\).

    既然 \(b > 0\),
    且对任何不超过 \(1\) 的正数 \(t\),
    有 \(mt + b \neq 0\),
    我们能推出 \(m + b > 0\).
    故 \(\det {(A)} = \det {(A_1)} = m + b > 0\).

    所以, \(P(n)\) 是正确的.
    由数学归纳法原理, 待证命题成立.
\end{proof}

% the old proof, which works but is not so good
% \begin{proof}
%     设 \(A\) 是一个 \(n\)~级\emph{实阵} (\(n \geq 2\)).
%     设对任何不超过 \(n\) 的正整数 \(i\), 必
%     \([A]_{i,i} > 0\).
%     设对任何不超过 \(n\), 且不等的正整数 \(j\), \(k\), 必
%     \begin{align*}
%         [A]_{j,j}\,[A]_{k,k} >
%         \Bigg(
%         \sum_{\substack{1 \leq p \leq n \\
%                 p \neq j}} {|[A]_{j,p}|}
%         \Bigg)
%         \Bigg(
%         \sum_{\substack{1 \leq q \leq n \\
%                 q \neq k}} {|[A]_{k,q}|}
%         \Bigg).
%     \end{align*}
%     我们知道, 一定有
%     \(n-1\)~个互不相同的, 且不超过 \(n\) 的正整数
%     \(i_1\), \(i_2\), \(\dots\), \(i_{n-1}\),
%     使对任何不超过 \(n-1\) 的正整数 \(w\),
%     有
%     \begin{align*}
%         [A]_{i_w,i_w} >
%         \sum_{\substack{1 \leq u \leq n \\
%             u \neq i_w}} {|[A]_{i_w,u}|}.
%     \end{align*}
%     (无妨设 \(i_1\), \(i_2\), \(\dots\), \(i_{n-1}\)
%     是从小到大的.)
%     (可用反证法证明此事:
%     反设有不等的 j, k 不适合此不等式,
%     则可推出矛盾.)
%     从 \(1\), \(2\), \(\dots\), \(n\)
%     去除 \(i_1\), \(\dots\), \(i_{n-1}\) 后,
%     还剩一个数.
%     我们叫它 \(i_n\).
%     作 \(n \times n\)~实阵 \(A_t\) 如下:
%     \begin{align*}
%         [A_t]_{i,j} =
%         \begin{cases}
%             [A]_{i,j},       & i \neq i_n;  \\
%             [A]_{i_n,i_n},   & i = j = i_n; \\
%             t\, [A]_{i_n,j}, & \text{其他}.
%         \end{cases}
%     \end{align*}
%     我们考虑 \(A_t\)~的行列式.
%     按行~\(i_n\) 展开, 有
%     \begin{align*}
%         \det {(A_t)}
%         = {} &
%         \sum_{1 \leq j \leq n}
%         {(-1)^{i_n + j} [A_t]_{i_n,j} \det {(A_t (i_n | j))}}
%         \\
%         = {} &
%         \sum_{1 \leq j \leq n}
%         {(-1)^{i_n + j} [A_t]_{i_n,j} \det {(A (i_n | j))}}
%         \\
%         = {} &
%         \sum_{\substack{1 \leq j \leq n \\j \neq i_n}}
%         {(-1)^{i_n + j} [A_t]_{i_n,j} \det {(A (i_n | j))}}
%         +
%         (-1)^{i_n + i_n} [A_t]_{i_n,i_n} \det {(A (i_n | i_n))}
%         \\
%         = {} &
%         \sum_{\substack{1 \leq j \leq n \\j \neq i_n}}
%         {(-1)^{i_n + j} t\, [A]_{i_n,j} \det {(A (i_n | j))}}
%         +
%         [A]_{i_n,i_n} \det {(A (i_n | i_n))}
%         \\
%         = {} &
%         t
%         \sum_{\substack{1 \leq j \leq n \\j \neq i_n}}
%         {(-1)^{i_n + j} [A]_{i_n,j} \det {(A (i_n | j))}}
%         +
%         [A]_{i_n,i_n} \det {(A (i_n | i_n))}.
%     \end{align*}
%     记
%     \begin{align*}
%         m & =
%         \sum_{\substack{1 \leq j \leq n \\j \neq i_n}}
%         {(-1)^{i_n + j} [A]_{i_n,j} \det {(A (i_n | j))}},
%         \\
%         b & =
%         [A]_{i_n,i_n} \det {(A (i_n | i_n))}.
%     \end{align*}
%     则 \(m\), \(b\) 是实数, 且
%     \(\det {(A_t)} = mt + b\).

%     作 \(n-1\)~级实阵 \(C = A(i_n | i_n)\).
%     则 \([C]_{w,y} = [A]_{i_w,i_y}\)
%     (因为已设 \(i_1 < i_2 < \dots < i_{n-1}\)),
%     且对 \(w < n\),
%     \begin{align*}
%         [C]_{w,w}
%         = {}    &
%         [A]_{i_w,i_w}
%         \\
%         > {}    &
%         \sum_{\substack{1 \leq u \leq n   \\u \neq i_w}}
%         {|[A]_{i_w,u}|}
%         \\
%         \geq {} &
%         \sum_{\substack{1 \leq u \leq n   \\u \neq i_w, i_n}}
%         {|[A]_{i_w,u}|}
%         \\
%         = {}    &
%         \sum_{\substack{1 \leq y \leq n-1 \\y \neq w}}
%         {|[A]_{i_w,i_y}|}
%         \\
%         = {}    &
%         \sum_{\substack{1 \leq y \leq n-1 \\y \neq w}}
%         {|[C]_{w,y}|}.
%     \end{align*}
%     故 \(\det {(A(i_n | i_n))} = \det {(C)} > 0\),
%     由本节的第~1~个定理.
%     则 \(b > 0\).

%     设 \(t\) 是不超过 \(1\)~的正数.
%     则 \(0 \leq |t| = t \leq 1\).
%     任取不超过 \(n\), 且不等的正整数 \(j\), \(k\).
%     若 \(j\), \(k\) 的任何一个都不是 \(i_n\), 则
%     \begin{align*}
%         [A_t]_{j,j}\,[A_t]_{k,k}
%         = {} &
%         [A]_{j,j}\,[A]_{k,k}
%         \\
%         > {} &
%         \Bigg(
%         \sum_{\substack{1 \leq p \leq n \\
%                 p \neq j}} {|[A]_{j,p}|}
%         \Bigg)
%         \Bigg(
%         \sum_{\substack{1 \leq q \leq n \\
%                 q \neq k}} {|[A]_{k,q}|}
%         \Bigg)
%         \\
%         = {} &
%         \Bigg(
%         \sum_{\substack{1 \leq p \leq n \\
%                 p \neq j}} {|[A_t]_{j,p}|}
%         \Bigg)
%         \Bigg(
%         \sum_{\substack{1 \leq q \leq n \\
%                 q \neq k}} {|[A_t]_{k,q}|}
%         \Bigg).
%     \end{align*}
%     若 \(j = i_n\), 且 \(k \neq j\), 则
%     \begin{align*}
%         [A_t]_{j,j}\,[A_t]_{k,k}
%         = {}    &
%         [A]_{j,j}\,[A]_{k,k}
%         \\
%         > {}    &
%         \Bigg(
%         \sum_{\substack{1 \leq p \leq n \\
%                 p \neq j}} {|[A]_{j,p}|}
%         \Bigg)
%         \Bigg(
%         \sum_{\substack{1 \leq q \leq n \\
%                 q \neq k}} {|[A]_{k,q}|}
%         \Bigg)
%         \\
%         \geq {} &
%         |t|\,\Bigg(
%         \sum_{\substack{1 \leq p \leq n \\
%                 p \neq j}} {|[A]_{j,p}|}
%         \Bigg)
%         \Bigg(
%         \sum_{\substack{1 \leq q \leq n \\
%                 q \neq k}} {|[A]_{k,q}|}
%         \Bigg)
%         \\
%         = {}    &
%         \Bigg(
%         \sum_{\substack{1 \leq p \leq n \\
%                 p \neq j}} {|t|\,|[A]_{j,p}|}
%         \Bigg)
%         \Bigg(
%         \sum_{\substack{1 \leq q \leq n \\
%                 q \neq k}} {|[A]_{k,q}|}
%         \Bigg)
%         \\
%         = {}    &
%         \Bigg(
%         \sum_{\substack{1 \leq p \leq n \\
%                 p \neq j}} {|[A_t]_{j,p}|}
%         \Bigg)
%         \Bigg(
%         \sum_{\substack{1 \leq q \leq n \\
%                 q \neq k}} {|[A_t]_{k,q}|}
%         \Bigg).
%     \end{align*}
%     若 \(k = i_n\), 且 \(j \neq k\),
%     类似地, 我们也有
%     \begin{align*}
%         [A_t]_{j,j}\,[A_t]_{k,k}
%         > {} &
%         \Bigg(
%         \sum_{\substack{1 \leq p \leq n \\
%                 p \neq j}} {|[A_t]_{j,p}|}
%         \Bigg)
%         \Bigg(
%         \sum_{\substack{1 \leq q \leq n \\
%                 q \neq k}} {|[A_t]_{k,q}|}
%         \Bigg).
%     \end{align*}
%     注意到 \(0 < [A]_{i,i} = [A_t]_{i,i}\),
%     故 \(|[A_t]_{i,i}| = [A_t]_{i,i}\).
%     则 \(\det {(A_t)} \neq 0\);
%     也就是说,
%     对任何不超过 \(1\) 的正数 \(t\),
%     有 \(mt + b \neq 0\).

%     既然 \(b > 0\),
%     且对任何不超过 \(1\) 的正数 \(t\),
%     有 \(mt + b \neq 0\),
%     我们能推出 \(m + b > 0\).
%     故 \(\det {(A)} = \det {(A_1)} = m + b > 0\).
% \end{proof}

当然, 如下命题也是正确的 (利用转置):

\begin{theorem}
    设 \(A\) 是一个 \(n\)~级\emph{实阵} (\(n \geq 2\)).
    设对任何不超过 \(n\) 的正整数 \(i\), 必
    \([A]_{i,i} > 0\).
    设对任何不超过 \(n\), 且不等的正整数 \(j\), \(k\), 必
    \begin{align*}
        [A]_{j,j}\,[A]_{k,k} >
        \Bigg(
        \sum_{\substack{1 \leq p \leq n \\
                p \neq j}} {|[A]_{p,j}|}
        \Bigg)
        \Bigg(
        \sum_{\substack{1 \leq q \leq n \\
                q \neq k}} {|[A]_{q,k}|}
        \Bigg).
    \end{align*}
    则 \(\det {(A)} > 0\).
\end{theorem}

\begin{example}[continues=emp:NonzeroDet2]
    设 \(A =
    \begin{bmatrix}
        20 & 11 & 1 \\
        4  & 15 & 1 \\
        13 & 15 & 3 \\
    \end{bmatrix}\)
    是一个 \(3\)~级实阵.

    取实阵
    \begin{align*}
        B = \begin{bmatrix}
                1 & 0 & 0 \\
                0 & 2 & 0 \\
                0 & 0 & 1 \\
            \end{bmatrix},
        \quad
        C = \begin{bmatrix}
                1 & 0 & 0  \\
                0 & 1 & 0  \\
                0 & 0 & 10 \\
            \end{bmatrix}.
    \end{align*}
    不难算出, \(\det {(B)} > 0\),
    且 \(\det {(C)} > 0\).

    不难算出
    \begin{align*}
        BAC = \begin{bmatrix}
                  20 & 11 & 10 \\
                  8  & 30 & 20 \\
                  13 & 15 & 30 \\
              \end{bmatrix}.
    \end{align*}
    由此可见, \(BAC\) 是实阵, 且 \([BAC]_{i,i} > 0\).

    不难算出
    \begin{align*}
         & R_1 = \sum_{\substack{1 \leq u \leq 3 \\u \neq 1}}
        {|[BAC]_{1,u}|} = |11| + |10| = 21,      \\
         & R_2 = \sum_{\substack{1 \leq u \leq 3 \\u \neq 2}}
        {|[BAC]_{2,u}|} = |8| + |20| = 28,       \\
         & R_3 = \sum_{\substack{1 \leq u \leq 3 \\u \neq 3}}
        {|[BAC]_{3,u}|} = |13| + |15| = 28,
    \end{align*}
    且
    \begin{align*}
         & [BAC]_{1,1}\,[BAC]_{2,2}
        = 600 > 588 = R_1 R_2,      \\
         & [BAC]_{1,1}\,[BAC]_{3,3}
        = 600 > 588 = R_1 R_3,      \\
         & [BAC]_{2,2}\,[BAC]_{3,3}
        = 30^2 > 28^2 = R_2 R_3.
    \end{align*}
    (注意到, 既然我们已算出,
    每个 \([BAC]_{j,j}\,[BAC]_{k,k}\)
    都大于 \(R_j R_k\)
    (\(j < k\)),
    由乘法的交换律,
    我们不必再判断%
    每个 \([BAC]_{j,j}\,[BAC]_{k,k}\)
    是否都大于 \(R_j R_k\)
    (\(j > k\)).)
    故 \(\det {(BAC)} > 0\).

    既然 \(0 < \det {(BAC)} = \det {(B)} \det {(A)} \det {(C)}\),
    且 \(\det {(B)} > 0\), \(\det {(C)} > 0\),
    故 \(\det {(A)} > 0\).
    (其实, 不难算出, \(\det {(A)} = 476.\))
\end{example}

\endgroup


% Stories about classical adjoints (adjugates)
\section{消去律}

本节, 我们讨论一些运算的消去律.

\begin{theorem}
    设 \(x\), \(y\), \(z\) 是数.
    若 \(x + y = x + z\), 或 \(y + x = z + x\),
    则 \(y = z\).
\end{theorem}

我们知道:

(1)
存在数 \(0\), 使对任何数 \(a\), 必 \(0 + a = a = a + 0\);

(2)
对任何数 \(a\), 必存在数 \(-a\), 使 \((-a) + a = 0 = a + (-a)\).

那么, 若 \(x + y = x + z\),
则
\begin{align*}
    (-x) + (x + y) = (-x) + (x + z).
\end{align*}
由结合律, 有
\begin{align*}
    ((-x) + x) + y = ((-x) + x) + z,
\end{align*}
即
\begin{align*}
    0 + y = 0 + z.
\end{align*}
故 \(y = z\).

类似地, 我们可证,
若 \(y + x = z + x\), 则 \(y = z\).

类似地:

(3)
存在 \(m \times n\)~阵 \(0\),
使对任何 \(m \times n\)~阵 \(X\),
必 \(0 + X = X = X + 0\);

(4)
对任何 \(m \times n\)~阵 \(X\),
必存在 \(m \times n\)~阵 \(-X\),
使 \((-X) + X = 0 = X + (-X)\).

于是, 我们也有,

\begin{theorem}
    设 \(A\), \(B\), \(C\) 都是 \(m \times n\)~阵.
    若 \(A + B = A + C\), 或 \(B + A = C + A\),
    则 \(B = C\).
\end{theorem}

又设 \(x\), \(y\), \(z\) 是数.
那么, 若 \(xy = xz\), 或 \(yx = zx\),
还有 \(y = z\) 吗?
不一定.
毕竟, 对任何数 \(a\), 必 \(0a = 0 = a0\).
于是, 虽然 \(0 \cdot 1 = 0 \cdot 2\),
但 \(1 \neq 2\).

可是, 若我们还要求 \(x \neq 0\), 则 \(y = z\) 是对的.
毕竟:

(5)
存在数 \(1\), 使对任何数 \(a\), 必 \(1a = a = a1\);

(6)
对任何非零的数 \(a\), 必存在数 \(a^{-1}\), 使 \(a^{-1} a = 1 = a a^{-1}\).

设 \(x \neq 0\), 且 \(xy = xz\).
则 \(x^{-1} (xy) = x^{-1} (xz)\).
由结合律, 有 \((x^{-1} x) y = (x^{-1} x) z\).
则 \(1y = 1z\).
则 \(y = z\).

类似地, 可证,
若 \(x \neq 0\), 且 \(yx = zx\), 则 \(y = z\).
总之,

\begin{theorem}
    设 \(x\), \(y\), \(z\) 是数, 且 \(x \neq 0\).
    若 \(xy = xz\), 或 \(yx = zx\), 则 \(y = z\).
\end{theorem}

阵的积也有类似的性质.
不过, 由于阵的积是较复杂的, 相关的事实的论证也是较不简单的.

先看一个较简单的事实.

\begin{theorem}
    设 \(A\), \(B\) 是 \(m \times n\)~阵.
    设 \(x \neq 0\) 是一个数.
    若 \(xA = xB\), 则 \(A = B\).
\end{theorem}

\begin{proof}
    任取正整数 \(i \leq m\) 与 \(j \leq n\).
    则
    \begin{align*}
        x [A]_{i,j} = [xA]_{i,j} = [xB]_{i,j} = x [B]_{i,j}.
    \end{align*}
    由数的积的性质, 既然 \(x \neq 0\),
    则必 \([A]_{i,j} = [B]_{i,j}\).
\end{proof}

以下是本节的主要结论.

\begin{theorem}
    设 \(A\) 是一个 \(n\)~级阵, 且 \(\det {(A)} \neq 0\).

    设 \(n \times m\)~阵 \(B\), \(C\) 适合 \(AB = AC\).
    则 \(B = C\).

    设 \(m \times n\)~阵 \(F\), \(G\) 适合 \(FA = GA\).
    则 \(F = G\).
\end{theorem}

\begin{proof}
    我证第~1~个; 我留第~2~个为您的习题.

    设 \(A\) 是一个 \(n\)~级阵, 且 \(\det {(A)} \neq 0\).
    又设 \(n \times m\)~阵 \(B\), \(C\) 适合 \(AB = AC\).
    则
    \begin{align*}
        \operatorname{adj} {(A)}\, (AB)
        = \operatorname{adj} {(A)}\, (AC).
    \end{align*}
    由结合律,
    \begin{align*}
        (\operatorname{adj} {(A)}\, A) B
        = (\operatorname{adj} {(A)}\, A) C.
    \end{align*}
    由古伴的性质,
    \begin{align*}
        (\det {(A)}\, I) B = (\det {(A)}\, I) C,
    \end{align*}
    即
    \begin{align*}
        \det {(A)}\, (I B) = \det {(A)}\, (I C),
    \end{align*}
    即
    \begin{align*}
        \det {(A)}\, B = \det {(A)}\, C.
    \end{align*}
    因为 \(\det {(A)} \neq 0\),
    我们有 \(B = C\).
\end{proof}

以上结论也可被推广.
不过, 还是以上结论更常用, 更常见.

\begin{theorem}
    设 \(A\) 是一个 \(s \times n\)~阵.
    设 \(A\) 有一个行列式非零的 \(n\)~级子阵
    \begin{align*}
        A\binom{i_1,i_2,\dots,i_n}{1,2,\dots,n},
    \end{align*}
    其中 \(1 \leq i_1 < \dots < i_n \leq s\).
    设 \(n \times m\)~阵 \(B\), \(C\) 适合 \(AB = AC\).
    则 \(B = C\).

    设 \(H\) 是一个 \(n \times t\)~阵.
    设 \(H\) 有一个行列式非零的 \(n\)~级子阵
    \begin{align*}
        H\binom{1,2,\dots,n}{j_1,j_2,\dots,j_n},
    \end{align*}
    其中 \(1 \leq j_1 < \dots < j_n \leq t\).
    设 \(m \times n\)~阵 \(F\), \(G\) 适合 \(FH = GH\).
    则 \(F = G\).
\end{theorem}

\begin{proof}
    我证第~1~个; 我留第~2~个为您的习题.

    记
    \begin{align*}
        L = A\binom{i_1,i_2,\dots,i_n}{1,2,\dots,n}.
    \end{align*}
    注意到, \([L]_{p,v} = [A]_{i_p,v}\).

    从 \(1\), \(2\), \(\dots\), \(s\)
    去除 \(i_1\), \(i_2\), \(\dots\), \(i_n\)
    后, 还剩 \(s - n\)~个数.
    我们从小到大地叫这 \(s - n\)~个数为
    \(i_{n+1}\), \(\dots\), \(i_s\).

    作 \(n \times s\)~阵 \(X\) 如下:
    \begin{align*}
        [X]_{u,i_p}
        = \begin{cases}
              [\operatorname{adj} {(L)}]_{u,p},
                 & p \leq n; \\
              0, & p > n.
          \end{cases}
    \end{align*}
    则
    \begin{align*}
        [XA]_{u,v}
        = {} &
        \sum_{p = 1}^{s} {[X]_{u,p} [A]_{p,v}}
        \\
        = {} &
        \sum_{p = 1}^{s} {[X]_{u,i_p} [A]_{i_p,v}}
        \\
        = {} &
        \sum_{p = 1}^{n} {[X]_{u,i_p} [A]_{i_p,v}}
        + \sum_{p = n+1}^{s} {[X]_{u,i_p} [A]_{i_p,v}}
        \\
        = {} &
        \sum_{p = 1}^{n}
        {[\operatorname{adj} {(L)}]_{u,p} [L]_{p,v}}
        + \sum_{p = n+1}^{s} {0\, [A]_{i_p,v}}
        \\
        = {} &
        [\operatorname{adj} {(L)}\, L]_{u,v} + 0
        \\
        = {} &
        [\det {(L)}\,I_n]_{u,v}.
    \end{align*}
    故 \(X A = \det {(L)}\, I_n\).

    由 \(AB = AC\), 知
    \begin{align*}
        X (AB) = X (AC),
    \end{align*}
    即
    \begin{align*}
        (X A) B = (X A) C,
    \end{align*}
    即
    \begin{align*}
        (\det {(L)}\,I_n) B = (\det {(L)}\,I_n) C,
    \end{align*}
    即
    \begin{align*}
        \det {(L)}\,(I_n B) = \det {(L)}\,(I_n C),
    \end{align*}
    即
    \begin{align*}
        \det {(L)}\,B = \det {(L)}\,C.
    \end{align*}
    因为 \(\det {(L)} \neq 0\),
    故 \(B = C\).
\end{proof}

\section{古伴的性质 (1)}

本节, 我们认识古伴的二个性质.

一个 \(n\)~级阵 \(A\) 的古伴 \(\operatorname{adj} {(A)}\)
是一个 \(n\)~级阵,
其 \((i, j)\)-元
\begin{align*}
    [\operatorname{adj} {(A)}]_{i,j} = (-1)^{j+i} \det {(A(j|i))}.
\end{align*}
对任何方阵 \(A\), 有
\begin{align*}
    \operatorname{adj} {(A)}\, A
    = \det {(A)}\, I
    = A \operatorname{adj} {(A)}.
\end{align*}

\begin{theorem}
    设 \(A\) 是 \(n\)~级阵.
    则 \(\operatorname{adj} {(A^{\mathrm{T}})}
    = (\operatorname{adj} {(A)})^{\mathrm{T}}\).
\end{theorem}

\begin{proof}
    注意到
    \(A^{\mathrm{T}} (j|i) = (A (i|j))^{\mathrm{T}}\),
    故
    \begin{align*}
        [\operatorname{adj} {(A^{\mathrm{T}})}]_{i,j}
        = {} &
        (-1)^{j+i} \det {(A^{\mathrm{T}} (j|i))}
        \\
        = {} &
        (-1)^{j+i} \det {((A (i|j))^{\mathrm{T}})}
        \\
        = {} &
        (-1)^{i+j} \det {(A (i|j))}
        \\
        = {} &
        [\operatorname{adj} {(A)}]_{j,i}
        \\
        = {} &
        [(\operatorname{adj} {(A)})^{\mathrm{T}}]_{i,j}.
        \qedhere
    \end{align*}
\end{proof}

\begin{theorem}
    设 \(A\) 是 \(n\)~级阵 (\(n \geq 2\)).
    则
    \(\det {(\operatorname{adj} {(A)})} = (\det {(A)})^{n-1}\).
\end{theorem}

\begin{proof}
    不难看出, 若 \(B\) 是 \(n\)~级阵, \(k\) 是数, 则
    \(\det {(kB)} = k^n \det {(B)}\).
    于是
    \begin{align*}
        \det {(A)}\, (\det {(A)})^{n-1}
        = {} &
        (\det {(A)})^n
        \\
        = {} &
        (\det {(A)})^n \det {(I)}
        \\
        = {} &
        \det {(\det {(A)} I)}
        \\
        = {} &
        \det {(A \operatorname{adj} {(A)})}
        \\
        = {} &
        \det {(A)} \det {(\operatorname{adj} {(A)})}.
    \end{align*}
    若 \(\det {(A)} \neq 0\),
    我们可在等式的二侧消去它,
    即得结论.

    % 若 \(\det {(A)} = 0\),
    % 我们用别的方法说明 \(\det {(\operatorname{adj} {(A)})} = 0\).

    若 \(A = 0\), 则 \(\operatorname{adj} {(A)} = 0\),
    故 \(\det {(\operatorname{adj} {(A)})} = 0\).

    若 \(A \neq 0\), 且 \(\det {(A)} = 0\),
    我们用反证法说明
    \(\det {(\operatorname{adj} {(A)})} = 0\).
    反设
    \(\det {(\operatorname{adj} {(A)})} \neq 0\).
    则
    \begin{align*}
        \operatorname{adj} {(A)}\, A
        = \det {(A)}\, I
        = 0
        = \operatorname{adj} {(A)}\, 0.
    \end{align*}
    因为
    \(\det {(\operatorname{adj} {(A)})} \neq 0\),
    由消去律, 有 \(A = 0\).
    这是矛盾.
\end{proof}

\section{古伴的性质 (2)}

本节, 我们讨论古伴的子阵的行列式.

% \(A\) 的古伴来自 \(A\) 的子阵.
% 自然地, 我们会思考
% \(A\) 的古伴的子阵与 \(A\) 的子阵的关系.

\begin{theorem}
    设 \(A\) 是 \(n\)~级阵.
    设 \(k\) 是不超过 \(n\) 的正整数.
    设正整数 \(i_1\), \(i_2\), \(\dots\), \(i_k\)
    不超过 \(n\), 且是从小到大的.
    设正整数 \(j_1\), \(j_2\), \(\dots\), \(j_k\)
    不超过 \(n\), 且是从小到大的.
    则
    \begin{equation}
        \begin{aligned}
                 & \det {\left(
                (\operatorname{adj} {(A)})
                \binom{i_1,\dots,i_k}{j_1,\dots,j_k}
                \right)}
            \\
            = {} &
            (\det {(A)})^{k-1}\,
            (-1)^{j_1 + \dots + j_k + i_1 + \dots + i_k}
            \det {(A({j_1,\dots,j_k}|{i_1,\dots,i_k}))}.
        \end{aligned}
        \label{eq:C2401}
    \end{equation}
\end{theorem}

此事对 \(k = 1\) 是正确的.
此时, 等式的左侧即为 \(\operatorname{adj} {(A)}\)
的 \((i_1, j_1)\)-元,
而等式的右侧是
\(1 \, (-1)^{j_1 + i_1} \det {(A(j_1|i_1))}\).
这正是古伴的定义.

以下设 \(k > 1\).

从 \(1\), \(2\), \(\dots\), \(n\)
去除 \(i_1\), \(\dots\), \(i_k\) 后,
还剩 \(n - k\) 个数.
我们从小到大地叫这 \(n - k\) 个数为
\(i_{k+1}\), \(\dots\), \(i_n\).
类似地, 从 \(1\), \(2\), \(\dots\), \(n\)
去除 \(j_1\), \(\dots\), \(j_k\) 后,
还剩 \(n - k\) 个数.
我们从小到大地叫这 \(n - k\) 个数为
\(j_{k+1}\), \(\dots\), \(j_n\).
作 \(n\)~级阵 \(B\) 如下:
\begin{align*}
    [B]_{i,j_q}
    = \begin{cases}
          [\operatorname{adj} {(A)}]_{i,j_q},
           & q \leq k; \\
          [I_n]_{i,i_q},
           & q > k.
      \end{cases}
\end{align*}
形象地, \(B\) 的列 \(j_1\), \(\dots\), \(j_k\)
分别与 \(\operatorname{adj} {(A)}\) 的列 \(j_1\), \(\dots\), \(j_k\)
相等,
但 \(B\) 的列 \(j_{k+1}\), \(\dots\), \(j_n\)
分别是 \(n\)~级单位阵 \(I_n\) 的列 \(i_{k+1}\), \(\dots\), \(i_n\).
由此可见,
\begin{align*}
    B\binom{i_1,\dots,i_k}{j_1,\dots,j_k}
    = B({i_{k+1},\dots,i_n}|{j_{k+1},\dots,j_n})
    = (\operatorname{adj} {(A)})
    \binom{i_1,\dots,i_k}{j_1,\dots,j_k},
\end{align*}
且
\begin{align*}
    B({i_1,\dots,i_k}|{j_1,\dots,j_k})
    = B\binom{i_{k+1},\dots,i_n}{j_{k+1},\dots,j_n}
    = I_{n-k}.
\end{align*}
% 其中 \(I_{n-k}\) 指 \(n-k\)~级单位阵.

按列 \(j_{k+1}\) 展开 \(\det {(B)}\), 有
\begin{align*}
    \det {(B)} = (-1)^{i_{k+1}+j_{k+1}} \, 1
    \det {(B(i_{k+1}|j_{j+1}))}.
\end{align*}
注意到 \(B\) 的列 \(j_{k+2}\) 即为
\(B(i_{k+1}|j_{k+1})\) 的列 \(j_{k+2}-1\).
按列 \(j_{k+2}-1\) 展开 \(\det {(B(i_{k+1}|j_{k+1}))}\),
有
\begin{align*}
    \det {(B(i_{k+1}|j_{k+1}))} = (-1)^{i_{k+2}-1+j_{k+2}-1} \, 1
    \det {(B({i_{k+1},i_{k+2}}|{j_{k+1},j_{k+2}}))}.
\end{align*}
故
\begin{align*}
    \det {(B)} = (-1)^{i_{k+1}+i_{k+2}+j_{k+1}+j_{k+2}}
    \det {(B({i_{k+1},i_{k+2}}|{j_{k+1},j_{k+2}}))}.
\end{align*}
\(\dots \dots\)
最后, 我们有
\begin{align*}
    \det {(B)}
    = (-1)^{i_{k+1}+\dots+i_n+j_{k+1}+\dots+j_n}
    \det {(B({i_{k+1},\dots,i_n}|{j_{k+1},\dots,j_n}))}.
\end{align*}

我们考虑 \(AB\) 的行列式.
一方面,
\begin{align*}
    \det {(AB)}
    = {} & \det {(A)} \det {(B)}
    \\
    = {} & (-1)^{i_{k+1}+\dots+i_n+j_{k+1}+\dots+j_n}
    \det {(A)}
    \det {\left(
        (\operatorname{adj} {(A)})
        \binom{i_1,\dots,i_k}{j_1,\dots,j_k}
        \right)}.
\end{align*}
另一方面, 当 \(q \leq k\) 时,
\begin{align*}
    [AB]_{i,j_q}
    = {} &
    \sum_{u=1}^{n} {[A]_{i,u} [B]_{u,j_q}}
    \\
    = {} &
    \sum_{u=1}^{n} {[A]_{i,u} [\operatorname{adj} {(A)}]_{u,j_q}}
    \\
    = {} &
    [A \operatorname{adj} {(A)}]_{i,j_q}
    \\
    = {} &
    [\det {(A)}\,I_n]_{i,j_q};
\end{align*}
当 \(q > k\) 时,
\begin{align*}
    [AB]_{i,j_q}
    = {} &
    \sum_{u=1}^{n} {[A]_{i,u} [B]_{u,j_q}}
    \\
    = {} &
    \sum_{u=1}^{n} {[A]_{i,u} [I_n]_{u,i_q}}
    \\
    = {} &
    [A I_n]_{i,i_q}
    \\
    = {} &
    [A]_{i,i_q}.
\end{align*}
形象地, \(AB\) 的列 \(j_1\), \(\dots\), \(j_k\)
分别与 \(\det {(A)}\,I_n\) 的列 \(j_1\), \(\dots\), \(j_k\)
相等,
但 \(AB\) 的列 \(j_{k+1}\), \(\dots\), \(j_n\)
分别是 \(A\) 的列 \(i_{k+1}\), \(\dots\), \(i_n\).
由此可见,
\begin{align*}
    (AB)\binom{j_1,\dots,j_k}{j_1,\dots,j_k}
    = (AB)({j_{k+1},\dots,j_n}|{j_{k+1},\dots,j_n})
    = \det {(A)}\, I_k,
\end{align*}
且
\begin{align*}
    (AB)({j_1,\dots,j_k}|{j_1,\dots,j_k})
    = (AB)\binom{j_{k+1},\dots,j_n}{j_{k+1},\dots,j_n}
    = A({j_1,\dots,j_k}|{i_1,\dots,i_k}).
\end{align*}

按列 \(j_1\) 展开 \(\det {(AB)}\), 有
\begin{align*}
    \det {(AB)} = (-1)^{j_1+j_1} \det {(A)}
    \det {((AB)(j_1|j_1))}.
\end{align*}
注意到 \(AB\) 的列 \(j_2\) 即为
\((AB)(j_1|j_1)\) 的列 \(j_2-1\).
按列 \(j_2-1\) 展开 \(\det {((AB)(j_1|j_1))}\),
有
\begin{align*}
    \det {((AB)(j_1|j_1))}
    = (-1)^{j_2-1+j_2-1} \det {(A)}
    \det {((AB)({j_1,j_2}|{j_1,j_2}))}.
\end{align*}
故
\begin{align*}
    \det {(AB)} = (\det {(A)})^2
    \det {((AB)({i_1,i_2}|{j_1,j_2}))}.
\end{align*}
\(\dots \dots\)
最后, 我们有
\begin{align*}
    \det {(AB)}
    = (\det {(A)})^k
    \det {((AB)({j_1,\dots,j_k}|{j_1,\dots,j_k}))}.
\end{align*}

比较二次计算的结果, 我们应有
\begin{align*}
         &
    (-1)^{i_{k+1}+\dots+i_n+j_{k+1}+\dots+j_n}
    \det {(A)}
    \det {\left(
        (\operatorname{adj} {(A)})
        \binom{i_1,\dots,i_k}{j_1,\dots,j_k}
        \right)}
    \\
    = {} &
    (\det {(A)})^k
    \det {(A({j_1,\dots,j_k}|{i_1,\dots,i_k}))}.
\end{align*}
注意到
\begin{align*}
         &
    i_{k+1} + \dots + i_n + j_{k+1} + \dots + j_n
    \\
    = {} &
    ((1 + 2 + \dots + n) - (i_1 + \dots + i_k))
    + ((1 + 2 + \dots + n) - (j_1 + \dots + j_k))
    \\
    = {} &
    2(1 + 2 + \dots + n) - (j_1 + \dots + j_k + i_1 + \dots + i_k),
\end{align*}
故
\begin{align*}
         &
    \det {(A)}\,
    \det {\left(
        (\operatorname{adj} {(A)})
        \binom{i_1,\dots,i_k}{j_1,\dots,j_k}
        \right)}
    \\
    = {} &
    \det {(A)}\, (\det {(A)})^{k-1}\,
    (-1)^{j_1+\dots+j_k+i_1+\dots+i_k}
    \det {(A({j_1,\dots,j_k}|{i_1,\dots,i_k}))}.
\end{align*}

若 \(\det {(A)} \neq 0\),
我们可在等式的二侧消去它, 得式~\eqref{eq:C2401}.
不过, 若 \(\det {(A)} = 0\),
会发生什么?

\begin{theorem}
    设 \(A\) 是一个 \(n\)~级阵.
    设 \(\det {(A)} = 0\).
    则存在不超过 \(n\) 的正整数 \(s\),
    使对任何不超过 \(n\) 的正整数 \(j\),
    存在数 \(m_j\),
    使对任何不超过 \(n\) 的正整数 \(i\),
    有
    \([\operatorname{adj} {(A)}]_{i,j}
    = m_j [\operatorname{adj} {(A)}]_{i,s}\);
    通俗地,
    存在不超过 \(n\) 的正整数 \(s\),
    使 \(\operatorname{adj} {(A)}\) 的任何一列
    都是 \(\operatorname{adj} {(A)}\) 的列~\(s\) 的数乘.
\end{theorem}

有了此事, 不难看出, 若 \(\det {(A)} = 0\),
且 \(k > 1\), 则
\begin{align*}
    \det {\left(
        (\operatorname{adj} {(A)})
        \binom{i_1,\dots,i_k}{j_1,\dots,j_k}
        \right)} = 0,
\end{align*}
故式~\eqref{eq:C2401} 仍是正确的.

那么, 若我们证明了此事, 则我们也就证明了式~\eqref{eq:C2401}.

\begin{proof}
    若 \(\operatorname{adj} {(A)} = 0\),
    则 \(\operatorname{adj} {(A)}\) 的每一列都是零.
    从而 \(\operatorname{adj} {(A)}\)
    的每一列都是
    \(\operatorname{adj} {(A)}\) 的列~\(1\) 的数乘.

    下设 \(\operatorname{adj} {(A)} \neq 0\).
    则有不超过 \(n\) 的正整数 \(s\), \(t\),
    使 \([\operatorname{adj} {(A)}]_{t,s} \neq 0\).
    则 \((-1)^{s+t} \det {(A(s|t))} \neq 0\).
    作 \(n-1\)~级阵 \(C = A(s|t)\).
    则 \(\det {(C)} \neq 0\).

    从 \(1\), \(2\), \(\dots\), \(n\)
    去除 \(t\) 后, 还剩 \(n - 1\) 个数.
    我们从小到大地叫这 \(n - 1\) 个数为
    \(i_1\), \(\dots\), \(i_{n-1}\).
    类似地, 从 \(1\), \(2\), \(\dots\), \(n\)
    去除 \(s\) 后, 还剩 \(n - 1\) 个数.
    我们从小到大地叫这 \(n - 1\) 个数为
    \(\ell_1\), \(\dots\), \(\ell_{n-1}\).
    则 \([C]_{p,q} = [A]_{\ell_p,i_q}\).
    再记 \(i_n = t\), 且 \(\ell_n = s\).

    设 \(j\) 是不超过 \(n\) 的正整数.
    作 \((n-1) \times 1\)~阵 \(y_j\) 如下:
    \begin{align*}
        [y_j]_{u,1} = [\operatorname{adj} {(A)}]_{i_u,j};
    \end{align*}
    通俗地, 若 \(D_j\) 是
    \(\operatorname{adj} {(A)}\) 的列~\(j\),
    则 \(y_j\) 是去除 \(D_j\) 的行~\(t\) 后的那个
    \((n-1) \times 1\)~阵.
    则对 \(p < n\),
    \begin{align*}
        [C y_j]_{p,1}
        = {} &
        \sum_{u=1}^{n-1} {[C]_{p,u} [y_j]_{u,1}}
        \\
        = {} &
        \sum_{u=1}^{n-1}
        {[A]_{\ell_p,i_u} [\operatorname{adj} {(A)}]_{i_u,j}}
        \\
        = {} &
        \sum_{u=1}^{n}
        {[A]_{\ell_p,i_u} [\operatorname{adj} {(A)}]_{i_u,j}}
        - [A]_{\ell_p,i_n} [\operatorname{adj} {(A)}]_{i_n,j}
        \\
        = {} &
        \sum_{u=1}^{n}
        {[A]_{\ell_p,u} [\operatorname{adj} {(A)}]_{u,j}}
        - [A]_{\ell_p,t} [\operatorname{adj} {(A)}]_{t,j}
        \\
        = {} &
        [A \operatorname{adj} {(A)}]_{\ell_p,j}
        - [A]_{\ell_p,t} [\operatorname{adj} {(A)}]_{t,j}
        \\
        = {} &
        [\det {(A)}\, I_n]_{\ell_p,j}
        - [A]_{\ell_p,t} [\operatorname{adj} {(A)}]_{t,j}
        \\
        = {} &
        {- [\operatorname{adj} {(A)}]_{t,j} [A]_{\ell_p,t}}.
    \end{align*}

    作 \((n-1) \times 1\)~阵 \(z\) 如下:
    \begin{align*}
        [z]_{u,1} = [A]_{\ell_u,t};
    \end{align*}
    通俗地, 若 \(w\) 是 \(A\) 的列~\(t\),
    则 \(z\) 是去除 \(w\) 的行~\(s\) 后的那个
    \((n-1) \times 1\)~阵.

    由前面的计算,
    \(C y_j = -[\operatorname{adj} {(A)}]_{t,j} z\),
    且
    \(C y_s = -[\operatorname{adj} {(A)}]_{t,s} z\).
    记
    \begin{align*}
        m_j =
        \frac{-[\operatorname{adj} {(A)}]_{t,j}}
        {-[\operatorname{adj} {(A)}]_{t,s}}.
    \end{align*}
    则
    \begin{align*}
        C (m_j y_s)
        = {} & m_j (C y_s)
        = m_j (-[\operatorname{adj} {(A)}]_{t,s} z)
        = (m_j (-[\operatorname{adj} {(A)}]_{t,s})) z
        \\
        = {} &
        {-[\operatorname{adj} {(A)}]_{t,j} z}
        \\
        = {} &
        C y_j.
    \end{align*}
    % % 于是, \(X = y_j\) 与 \(X = m_j y_s\) 都是线性方程组
    % % \(CX = -[\operatorname{adj} {(A)}]_{t,j} z\)
    % % 的解.
    % % 因为 \(\det {(C)} \neq 0\),
    % % 故 \(y_j = m_j y_s\).
    % 则
    % \begin{align*}
    %     \operatorname{adj} {(C)}\, (C (m_j y_s))
    %     = \operatorname{adj} {(C)}\, (C y_j),
    % \end{align*}
    % 即
    % \begin{align*}
    %     (\operatorname{adj} {(C)}\, C) (m_j y_s)
    %     = (\operatorname{adj} {(C)}\, C) (y_j),
    % \end{align*}
    % 即
    % \begin{align*}
    %     (\det {(C)}\, I_{n-1}) (m_j y_s)
    %     = (\det {(C)}\, I_{n-1}) (y_j),
    % \end{align*}
    % 即
    % \begin{align*}
    %     \det {(C)}\, (I_{n-1} (m_j y_s))
    %     = \det {(C)}\, (I_{n-1} y_j),
    % \end{align*}
    % 即
    % \begin{align*}
    %     \det {(C)}\, (m_j y_s)
    %     = \det {(C)}\, y_j.
    % \end{align*}
    % 因为数 \(\det {(C)} \neq 0\),
    因为 \(\det {(C)} \neq 0\),
    故 \(m_j y_s = y_j\).
    故 \(m_j [\operatorname{adj} {(A)}]_{i_u,s}
        = [\operatorname{adj} {(A)}]_{i_u,j}\),
    对 \(u < n\).
    则 \(m_j [\operatorname{adj} {(A)}]_{i,s}
        = [\operatorname{adj} {(A)}]_{i,j}\),
    对 \(i \leq n\).
\end{proof}

% 最后, 为方便, 我写下式~\eqref{eq:C2401} 的一个特别的情形.

% 设 \(A\) 是 \(n\)~级阵 (\(n \geq 3\)).

% 设 \(1 \leq u < v \leq n\).
% 取 \(k = 2\),
% \(i_1 = j_1 = u\),
% \(i_2 = j_2 = v\),
% 有
% \begin{align*}
%     \det {\left(
%         (\operatorname{adj} {(A)})
%         \binom{u,v}{u,v}
%         \right)}
%     = (\det {(A)})^{2-1}\,
%     (-1)^{u+v+u+v}
%     \det {(A({u,v}|{u,v}))}.
% \end{align*}
% 不难算出
% \begin{align*}
%     \det {(A(u|u))} \det {(A(v|v))}
%     - \det {(A(u|v))} \det {(A(v|u))}
%     = \det {(A)} \det {(A({u,v}|{u,v}))}.
% \end{align*}

% 若 \(1 \leq v < u \leq n\),
% 则, 类似地,
% \begin{align*}
%     \det {(A(v|v))} \det {(A(u|u))}
%     - \det {(A(v|u))} \det {(A(u|v))}
%     = \det {(A)} \det {(A({v,u}|{v,u}))},
% \end{align*}
% 也就是
% \begin{align*}
%     \det {(A(u|u))} \det {(A(v|v))}
%     - \det {(A(u|v))} \det {(A(v|u))}
%     = \det {(A)} \det {(A({u,v}|{u,v}))}.
% \end{align*}

% 总结这些结果, 有

% \begin{theorem}
%     设 \(A\) 是 \(n\)~级阵 (\(n \geq 3\)).
%     设 \(u\), \(v\) 是不超过 \(n\) 的正整数, 且互不相同.
%     则
%     \begin{align*}
%         \det {(A(u|u))} \det {(A(v|v))}
%         = \det {(A(u|v))} \det {(A(v|u))}
%         + \det {(A)} \det {(A({u,v}|{u,v}))}.
%     \end{align*}
%     特别地, 若 \(\det {(A)} = 0\), 则
%     \begin{align*}
%         \det {(A(u|u))} \det {(A(v|v))}
%         = \det {(A(u|v))} \det {(A(v|u))}.
%     \end{align*}
% \end{theorem}

\section{古伴的性质 (3)}

本节, 我们再认识古伴的二个性质.

\begin{theorem}
    设 \(A\) 是 \(n\)~级阵 (\(n \geq 2\)).
    则
    \(\operatorname{adj} {(\operatorname{adj} {(A)})}
    = (\det {(A)})^{n-2}\,A\).
\end{theorem}

\begin{proof}
    设 \(n = 2\).
    不难算出, \(2\)~级阵
    \(A =
    \begin{bmatrix}
        a & b \\
        c & d \\
    \end{bmatrix}
    \)
    的古伴是
    \(
    \begin{bmatrix}
        d  & -b \\
        -c & a  \\
    \end{bmatrix}
    \).
    则
    \(
    \begin{bmatrix}
        d  & -b \\
        -c & a  \\
    \end{bmatrix}
    \)
    的古伴是
    \(
    \begin{bmatrix}
        a     & -(-b) \\
        -(-c) & d     \\
    \end{bmatrix}
    \),
    即 \(A\).
    注意到,
    \((\det {(A)})^{2-2} A = 1 A = A\).

    下设 \(n > 2\).
    一方面,
    \begin{align*}
        \operatorname{adj} {(A)}
        \operatorname{adj} {(\operatorname{adj} {(A)})}
        = \det {(\operatorname{adj} {(A)})}\, I_n
        = (\det {(A)})^{n-1}\, I_n;
    \end{align*}
    另一方面,
    \begin{align*}
        \operatorname{adj} {(A)}\,
        ((\det {(A)})^{n-2}\, A)
        = {} &
        (\det {(A)})^{n-2}\,
        (\operatorname{adj} {(A)} A)
        \\
        = {} &
        (\det {(A)})^{n-2}\, (\det {(A)}\,I_n)
        \\
        = {} &
        (\det {(A)})^{n-1}\, I_n.
    \end{align*}
    故
    \begin{align*}
        \operatorname{adj} {(A)}
        \operatorname{adj} {(\operatorname{adj} {(A)})}
        =
        \operatorname{adj} {(A)}\,
        ((\det {(A)})^{n-2} A).
    \end{align*}
    若 \(\det {(A)} \neq 0\),
    则 \(\det {(\operatorname{adj} {(A)})}
    = (\det {(A)})^{n-1} \neq 0\).
    由消去律,
    我们可在等式的二侧消去 \(\operatorname{adj} {(A)}\),
    即得结论.

    若 \(\det {(A)} = 0\),
    由上节的结论, \(\operatorname{adj} {(A)}\)~的%
    每一个 \(n-1\)~级子阵的行列式都是 \(0\).
    故 \(\operatorname{adj} {(\operatorname{adj} {(A)})}
    = 0 = 0A = (\det {(A)})^{n-2}\,A\).
\end{proof}

\begin{theorem}
    设 \(A\), \(B\) 是 \(n\)~级阵.
    则
    \begin{align*}
        \operatorname{adj} {(BA)}
        = \operatorname{adj} {(A)} \operatorname{adj} {(B)}.
    \end{align*}
\end{theorem}

\begin{proof}
    注意到
    \begin{align*}
        (BA) \operatorname{adj} {(BA)}
        = {} &
        \det {(BA)}\, I
        \\
        = {} &
        (\det {(B)} \det {(A)})\, I
        \\
        = {} &
        (\det {(A)} \det {(B)})\, I
        \\
        = {} &
        \det {(A)}\, (\det {(B)}\, I)
        \\
        = {} &
        \det {(A)}\, (B \operatorname{adj} {(B)})
        \\
        = {} &
        B (\det {(A)} \operatorname{adj} {(B)})
        \\
        = {} &
        B (\det {(A)}\, (I \operatorname{adj} {(B)}))
        \\
        = {} &
        B ((\det {(A)}\, I) \operatorname{adj} {(B)})
        \\
        = {} &
        B ((A \operatorname{adj} {(A)})\, \operatorname{adj} {(B)})
        \\
        = {} &
        B (A\, (\operatorname{adj} {(A)} \operatorname{adj} {(B)}))
        \\
        = {} &
        (B A)\, (\operatorname{adj} {(A)} \operatorname{adj} {(B)}),
    \end{align*}
    若 \(\det {(BA)} \neq 0\),
    由消去律, 即得结论.

    若 \(\det {(BA)} = 0\), 我们这么作.
    由阵的积的定义,
    \begin{align*}
        [\operatorname{adj} {(A)} \operatorname{adj} {(B)}]_{i,j}
        = {} &
        \sum_{u = 1}^{n}
        {[\operatorname{adj} {(A)}]_{i,u}
            [\operatorname{adj} {(B)}]_{u,j}}
        \\
        = {} &
        \sum_{u = 1}^{n}
        {(-1)^{u+i} \det {(A(u|i))}\,
        (-1)^{j+u} \det {(B(j|u))}}
        \\
        = {} &
        \sum_{u = 1}^{n}
        {(-1)^{u+i} (-1)^{j+u}
        \det {(A(u|i))} \det {(B(j|u))}}
        \\
        = {} &
        \sum_{u = 1}^{n}
        {(-1)^{j+i} \det {(B(j|u))} \det {(A(u|i))}}
        \\
        = {} &
        (-1)^{j+i}
        \sum_{u = 1}^{n}
        {\det {(B(j|u))} \det {(A(u|i))}}
        \\
        = {} &
        (-1)^{j+i}
        \det {((BA)(j|i))}
        \\
        = {} &
        [\operatorname{adj} {(BA)}]_{i,j}.
    \end{align*}
    这里, 我们用了 Binet--Cauchy 公式的推广.
\end{proof}

% 为方便, 我记下一个简单的推广.
% 设 \(A\), \(B\), \(C\) 是 \(n\)~级阵.
% 设 \(\det {(CBA)} \neq 0\).
% 因为 \(\det {(CBA)} = \det {(C)} \det {(BA)}\),
% 故 \(\det {(BA)} \neq 0\).
% 故
% \begin{align*}
%     \operatorname{adj} {(BA)}
%     = \operatorname{adj} {(A)} \operatorname{adj} {(B)}.
% \end{align*}
% 则
% \begin{align*}
%     \operatorname{adj} {(CBA)}
%     = {} &
%     \operatorname{adj} {(C(BA))}
%     \\
%     = {} &
%     \operatorname{adj} {(BA)}
%     \operatorname{adj} {(C)}
%     \\
%     = {} &
%     (\operatorname{adj} {(A)} \operatorname{adj} {(B)})
%     \operatorname{adj} {(C)}
%     \\
%     = {} &
%     \operatorname{adj} {(A)}
%     \operatorname{adj} {(B)}
%     \operatorname{adj} {(C)}.
% \end{align*}

\section{阵的积, 分块地写}

我们知道, 若 \(B\) 是 \(s \times m\)~阵,
且 \(a_1\), \(a_2\), \(\dots\), \(a_n\)
是 \(m \times 1\) 阵,
则
\begin{align*}
    B\,[a_1, a_2, \dots, a_n] = [Ba_1, Ba_2, \dots, Ba_n].
\end{align*}
形象地, 这是分块地写阵的积.
证明 \(\det {(BA)} = \det {(B)} \det {(A)}\)
(\(A\), \(B\) 是同级的方阵) 时,
我们用了此事, 使推理更简单.

我要介绍另一种分块地写阵的积的方式.

\begin{theorem}
    设
    \(A_{1,1}\), \(A_{1,2}\), \(A_{2,1}\), \(A_{2,2}\),
    \(B_{1,1}\), \(B_{1,2}\), \(B_{2,1}\), \(B_{2,2}\)
    是阵.
    设 \(A_{1,1}\), \(A_{1,2}\) 都有 \(r_1\) 行;
    设 \(A_{2,1}\), \(A_{2,2}\) 都有 \(r_2\) 行;
    设 \(A_{1,1}\), \(A_{2,1}\) 都有 \(u_1\) 列;
    设 \(A_{1,2}\), \(A_{2,2}\) 都有 \(u_2\) 列.
    设 \(B_{1,1}\), \(B_{1,2}\) 都有 \(u_1\) 行;
    设 \(B_{2,1}\), \(B_{2,2}\) 都有 \(u_2\) 行;
    设 \(B_{1,1}\), \(B_{2,1}\) 都有 \(c_1\) 列;
    设 \(B_{1,2}\), \(B_{2,2}\) 都有 \(c_2\) 列.
    作 \((r_1 + r_2) \times (u_1 + u_2)\)~阵
    \begin{align*}
        A = \begin{bmatrix}
                A_{1,1} & A_{1,2} \\
                A_{2,1} & A_{2,2} \\
            \end{bmatrix};
    \end{align*}
    具体地,
    \begin{align*}
        [A]_{i,j}
        = \begin{cases}
              [A_{1,1}]_{i,j},
               & \text{\(i \leq r_1\), \(j \leq u_1\)}; \\
              [A_{1,2}]_{i,j-u_1},
               & \text{\(i \leq r_1\), \(j > u_1\)};    \\
              [A_{2,1}]_{i-r_1,j},
               & \text{\(i > r_1\), \(j \leq u_1\)};    \\
              [A_{2,2}]_{i-r_1,j-u_1},
               & \text{\(i > r_1\), \(j > u_1\)}.
          \end{cases}
    \end{align*}
    作 \((u_1 + u_2) \times (c_1 + c_2)\)~阵
    \begin{align*}
        B = \begin{bmatrix}
                B_{1,1} & B_{1,2} \\
                B_{2,1} & B_{2,2} \\
            \end{bmatrix};
    \end{align*}
    具体地,
    \begin{align*}
        [B]_{i,j}
        = \begin{cases}
              [B_{1,1}]_{i,j},
               & \text{\(i \leq u_1\), \(j \leq c_1\)}; \\
              [B_{1,2}]_{i,j-c_1},
               & \text{\(i \leq u_1\), \(j > c_1\)};    \\
              [B_{2,1}]_{i-u_1,j},
               & \text{\(i > u_1\), \(j \leq c_1\)};    \\
              [B_{2,2}]_{i-u_1,j-c_1},
               & \text{\(i > u_1\), \(j > c_1\)}.
          \end{cases}
    \end{align*}
    因为 \(A\) 的列数等于 \(B\) 的行数,
    故 \(AB\) 有意义.

    记 \(C_{i,j} = A_{i,1} B_{1,j} + A_{i,2} B_{2,j}\),
    \(i\), \(j = 1\), \(2\).
    不难看出, \(C_{i,j}\) 是 \(r_i \times c_j\)~阵.
    作 \((r_1+r_2) \times (c_1+c_2)\)~阵
    \begin{align*}
        C = \begin{bmatrix}
                C_{1,1} & C_{1,2} \\
                C_{2,1} & C_{2,2} \\
            \end{bmatrix};
    \end{align*}
    具体地,
    \begin{align*}
        [C]_{i,j}
        = \begin{cases}
              [C_{1,1}]_{i,j},
               & \text{\(i \leq r_1\), \(j \leq c_1\)}; \\
              [C_{1,2}]_{i,j-c_1},
               & \text{\(i \leq r_1\), \(j > c_1\)};    \\
              [C_{2,1}]_{i-r_1,j},
               & \text{\(i > r_1\), \(j \leq c_1\)};    \\
              [C_{2,2}]_{i-r_1,j-c_1},
               & \text{\(i > r_1\), \(j > c_1\)}.
          \end{cases}
    \end{align*}
    则 \(AB = C\).

    形象地,
    \begin{align*}
        \begin{bmatrix}
            A_{1,1} & A_{1,2} \\
            A_{2,1} & A_{2,2} \\
        \end{bmatrix}
        \begin{bmatrix}
            B_{1,1} & B_{1,2} \\
            B_{2,1} & B_{2,2} \\
        \end{bmatrix}
        =
        \begin{bmatrix}
            A_{1,1} B_{1,1} + A_{1,2} B_{2,1}
             & A_{1,1} B_{1,2} + A_{1,2} B_{2,2} \\
            A_{2,1} B_{1,1} + A_{2,2} B_{2,1}
             & A_{2,1} B_{1,2} + A_{2,2} B_{2,2} \\
        \end{bmatrix}.
    \end{align*}
    于是, 形式地,
    分块地写阵的积与直接用定义写阵的积没有区别.

    % 特别地,
    % 若取 \(A_{1,2}\), \(A_{2,1}\), \(B_{1,2}\), \(B_{2,1}\)
    % 为零阵,
    % 有
    % \begin{align*}
    %     \begin{bmatrix}
    %         A_{1,1} & 0       \\
    %         0       & A_{2,2} \\
    %     \end{bmatrix}
    %     \begin{bmatrix}
    %         B_{1,1} & 0       \\
    %         0       & B_{2,2} \\
    %     \end{bmatrix}
    %     =
    %     \begin{bmatrix}
    %         A_{1,1} B_{1,1} & 0               \\
    %         0               & A_{2,2} B_{2,2} \\
    %     \end{bmatrix}.
    % \end{align*}
\end{theorem}

\begin{proof}
    注意到
    \begin{align*}
        [AB]_{i,j}
        = {} &
        \sum_{k=1}^{u_1+u_2}
        {[A]_{i,k} [B]_{k,j}}
        \\
        = {} &
        \sum_{k=1}^{u_1}
        {[A]_{i,k} [B]_{k,j}}
        +
        \sum_{k=u_1+1}^{u_1+u_2}
        {[A]_{i,k} [B]_{k,j}}.
    \end{align*}

    若 \(j \leq c_1\), 则
    \begin{align*}
        [AB]_{i,j}
        = {} &
        \sum_{k=1}^{u_1}
        {[A]_{i,k} [B]_{k,j}}
        +
        \sum_{k=u_1+1}^{u_1+u_2}
        {[A]_{i,k} [B]_{k,j}}
        \\
        = {} &
        \sum_{k=1}^{u_1}
        {[A]_{i,k} [B_{1,1}]_{k,j}}
        +
        \sum_{k=u_1+1}^{u_1+u_2}
        {[A]_{i,k} [B_{2,1}]_{k-u_1,j}}.
    \end{align*}
    当 \(i \leq r_1\) 时,
    \begin{align*}
        [AB]_{i,j}
        = {} &
        \sum_{k=1}^{u_1}
        {[A]_{i,k} [B_{1,1}]_{k,j}}
        +
        \sum_{k=u_1+1}^{u_1+u_2}
        {[A]_{i,k} [B_{2,1}]_{k-u_1,j}}
        \\
        = {} &
        \sum_{k=1}^{u_1}
        {[A_{1,1}]_{i,k} [B_{1,1}]_{k,j}}
        +
        \sum_{k=u_1+1}^{u_1+u_2}
        {[A_{1,2}]_{i,k-u_1} [B_{2,1}]_{k-u_1,j}}
        \\
        = {} &
        [A_{1,1} B_{1,1}]_{i,j}
        + [A_{1,2} B_{2,1}]_{i,j}
        \\
        = {} &
        [A_{1,1} B_{1,1} + A_{1,2} B_{2,1}]_{i,j}
        \\
        = {} &
        [C_{1,1}]_{i,j}
        \\
        = {} &
        [C]_{i,j}.
    \end{align*}
    当 \(i > r_1\) 时,
    \begin{align*}
        [AB]_{i,j}
        = {} &
        \sum_{k=1}^{u_1}
        {[A]_{i,k} [B_{1,1}]_{k,j}}
        +
        \sum_{k=u_1+1}^{u_1+u_2}
        {[A]_{i,k} [B_{2,1}]_{k-u_1,j}}
        \\
        = {} &
        \sum_{k=1}^{u_1}
        {[A_{2,1}]_{i-r_1,k} [B_{1,1}]_{k,j}}
        +
        \sum_{k=u_1+1}^{u_1+u_2}
        {[A_{2,2}]_{i-r_1,k-u_1} [B_{2,1}]_{k-u_1,j}}
        \\
        = {} &
        [A_{2,1} B_{1,1}]_{i-r_1,j}
        + [A_{2,2} B_{2,1}]_{i-r_1,j}
        \\
        = {} &
        [A_{2,1} B_{1,1} + A_{2,2} B_{2,1}]_{i-r_1,j}
        \\
        = {} &
        [C_{2,1}]_{i-r_1,j}
        \\
        = {} &
        [C]_{i,j}.
    \end{align*}

    若 \(j > c_1\), 则
    \begin{align*}
        [AB]_{i,j}
        = {} &
        \sum_{k=1}^{u_1}
        {[A]_{i,k} [B]_{k,j}}
        +
        \sum_{k=u_1+1}^{u_1+u_2}
        {[A]_{i,k} [B]_{k,j}}
        \\
        = {} &
        \sum_{k=1}^{u_1}
        {[A]_{i,k} [B_{1,2}]_{k,j-c_1}}
        +
        \sum_{k=u_1+1}^{u_1+u_2}
        {[A]_{i,k} [B_{2,2}]_{k-u_1,j-c_1}}.
    \end{align*}
    当 \(i \leq r_1\) 时,
    \begin{align*}
        [AB]_{i,j}
        = {} &
        \sum_{k=1}^{u_1}
        {[A]_{i,k} [B_{1,2}]_{k,j-c_1}}
        +
        \sum_{k=u_1+1}^{u_1+u_2}
        {[A]_{i,k} [B_{2,2}]_{k-u_1,j-c_1}}
        \\
        = {} &
        \sum_{k=1}^{u_1}
        {[A_{1,1}]_{i,k} [B_{1,2}]_{k,j-c_1}}
        +
        \sum_{k=u_1+1}^{u_1+u_2}
        {[A_{1,2}]_{i,k-u_1} [B_{2,2}]_{k-u_1,j-c_1}}
        \\
        = {} &
        [A_{1,1} B_{1,2}]_{i,j-c_1}
        + [A_{1,2} B_{2,2}]_{i,j-c_1}
        \\
        = {} &
        [A_{1,1} B_{1,2} + A_{1,2} B_{2,2}]_{i,j-c_1}
        \\
        = {} &
        [C_{1,2}]_{i,j-c_1}
        \\
        = {} &
        [C]_{i,j}.
    \end{align*}
    当 \(i > r_1\) 时,
    \begin{align*}
        [AB]_{i,j}
        = {} &
        \sum_{k=1}^{u_1}
        {[A]_{i,k} [B_{1,2}]_{k,j-c_1}}
        +
        \sum_{k=u_1+1}^{u_1+u_2}
        {[A]_{i,k} [B_{2,2}]_{k-u_1,j-c_1}}
        \\
        = {} &
        \sum_{k=1}^{u_1}
        {[A_{2,1}]_{i-r_1,k} [B_{1,2}]_{k,j-c_1}}
        +
        \sum_{k=u_1+1}^{u_1+u_2}
        {[A_{2,2}]_{i-r_1,k-u_1} [B_{2,2}]_{k-u_1,j-c_1}}
        \\
        = {} &
        [A_{2,1} B_{1,2}]_{i-r_1,j-c_1}
        + [A_{2,2} B_{2,2}]_{i-r_1,j-c_1}
        \\
        = {} &
        [A_{2,1} B_{1,2} + A_{2,2} B_{2,2}]_{i-r_1,j-c_1}
        \\
        = {} &
        [C_{2,2}]_{i-r_1,j-c_1}
        \\
        = {} &
        [C]_{i,j}.
        \qedhere
    \end{align*}
\end{proof}


% Stories about pfaffians
\section{转置的性质}

本节, 我们讨论转置的一些性质.

设 \(A\) 是一个 \(m \times n\)~阵.
则 \(A\)~的转置 \(A^{\mathrm{T}}\) 是一个 \(n \times m\)~阵,
且对 \(i \leq n\) 与 \(j \leq m\),
有 \([A^{\mathrm{T}}]_{i,j} = [A]_{j,i}\).

我们已知, 若 \(A\) 是一个 \(m \times n\)~阵,
则 \((A^{\mathrm{T}})^{\mathrm{T}} = A\).
我们还知道, 若 \(A\) 是一个 \(n\)~级阵,
则 \(\det {(A)} = \det {(A^{\mathrm{T}})}\).

转置当然还有一些性质;
我只是还没提到它们.

\begin{theorem}
    设 \(A\), \(B\) 是 \(m \times n\) 阵.
    设 \(C\) 是 \(n \times s\) 阵.
    设 \(k\) 是数.
    则:

    (1)
    \((A + B)^{\mathrm{T}} = A^{\mathrm{T}} + B^{\mathrm{T}}\);

    (2)
    \((k A)^{\mathrm{T}} = k A^{\mathrm{T}}\);

    (3)
    \((A C)^{\mathrm{T}} = C^{\mathrm{T}} A^{\mathrm{T}}\).
\end{theorem}

\begin{proof}
    您验证, 等式 (1) 与 (2) 的二侧的阵的尺寸是一样的;
    我验证, 等式 (3) 的二侧的阵的尺寸是一样的.

    (1)
    对 \(i \leq n\) 与 \(j \leq m\),
    \begin{align*}
        [(A + B)^{\mathrm{T}}]_{i,j}
        = {} &
        [A + B]_{j,i}
        \\
        = {} &
        [A]_{j,i} + [B]_{j,i}
        \\
        = {} &
        [A^{\mathrm{T}}]_{i,j} + [B^{\mathrm{T}}]_{i,j}
        \\
        = {} &
        [A^{\mathrm{T}} + B^{\mathrm{T}}]_{i,j}.
    \end{align*}

    (2)
    对 \(i \leq n\) 与 \(j \leq m\),
    \begin{align*}
        [(k A)^{\mathrm{T}}]_{i,j}
        = {} &
        [k A]_{j,i}
        \\
        = {} &
        k [A]_{j,i}
        \\
        = {} &
        k [A^{\mathrm{T}}]_{i,j}
        \\
        = {} &
        [k A^{\mathrm{T}}]_{i,j}.
    \end{align*}

    (3)
    \(A\) 是 \(m \times n\)~的, \(C\) 是 \(n \times s\)~的,
    故 \(AC\) 是 \(m \times s\)~的,
    故 \((AC)^{\mathrm{T}}\) 是 \(s \times m\)~的.
    另一方面, \(C^{\mathrm{T}}\) 是 \(s \times n\)~的,
    \(A^{\mathrm{T}}\) 是 \(n \times m\)~的,
    故 \(C^{\mathrm{T}} A^{\mathrm{T}}\) 是 \(s \times m\)~的.
    对 \(i \leq s\) 与 \(j \leq m\),
    \begin{align*}
        [(AC)^{\mathrm{T}}]_{i,j}
        = {} &
        [AC]_{j,i}
        \\
        = {} &
        \sum_{p=1}^{n} {[A]_{j,p} [C]_{p,i}}
        \\
        = {} &
        \sum_{p=1}^{n}
        {[A^{\mathrm{T}}]_{p,j} [C^{\mathrm{T}}]_{i,p}}
        \\
        = {} &
        \sum_{p=1}^{n}
        {[C^{\mathrm{T}}]_{i,p} [A^{\mathrm{T}}]_{p,j}}
        \\
        = {} &
        [C^{\mathrm{T}} A^{\mathrm{T}}]_{i,j}.
        \qedhere
    \end{align*}
\end{proof}

为方便, 我记下一个简单的推广.
设 \(A\), \(B\), \(C\) 分别是
\(m \times s\), \(s \times t\), \(t \times n\)~阵.
则
\begin{align*}
    (ABC)^{\mathrm{T}}
    = {} &
    ((AB) C)^{\mathrm{T}}
    \\
    = {} &
    C^{\mathrm{T}}
    (AB)^{\mathrm{T}}
    \\
    = {} &
    C^{\mathrm{T}}
    (B^{\mathrm{T}} A^{\mathrm{T}})
    \\
    = {} &
    C^{\mathrm{T}}
    B^{\mathrm{T}}
    A^{\mathrm{T}}.
\end{align*}
一般地, 我们有

\begin{theorem}
    设 \(A_1\), \(A_2\), \(\dots\), \(A_n\)
    分别是 \(m_0 \times m_1\), \(m_1 \times m_2\), \(\dots\),
    \(m_{n-1} \times m_n\)~阵
    (也就是说, \(A_k\) 的列数等于 \(A_{k+1}\) 的行数,
    \(k = 1\), \(2\), \(\dots\), \(n-1\)).
    则
    \begin{align*}
        (A_1 A_2 \dots A_n)^{\mathrm{T}}
        = A_n^{\mathrm{T}} \dots A_2^{\mathrm{T}} A_1^{\mathrm{T}}.
    \end{align*}
\end{theorem}

\begin{proof}
    请允许我留此事为您的习题.
    您用数学归纳法即可.
\end{proof}

\section{辛阵}

设 \(2m\) 是一个不低于 \(2\) 的偶数.
作 \(2m\)~级阵
\begin{align*}
    K_m =
    \begin{bmatrix}
        0    & I_m \\
        -I_m & 0   \\
    \end{bmatrix},
\end{align*}
其中 \(0\) 是 \(m\)~级零阵;
具体地,
\begin{align*}
    [K_m]_{i,j} =
    \begin{cases}
        [0]_{i,j},      & \text{\(i \leq m\), \(j \leq m\)}; \\
        [I_m]_{i,j-m},  & \text{\(i \leq m\), \(j > m\)};    \\
        [-I_m]_{i-m,j}, & \text{\(i > m\), \(j \leq m\)};    \\
        [0]_{i-m,j-m},  & \text{\(i > m\), \(j > m\)}.
    \end{cases}
\end{align*}
更具体地,
\begin{align*}
    [K_m]_{i,j} =
    \begin{cases}
        1,  & 1 \leq i = j - m \leq m; \\
        -1, & 1 \leq i - m = j \leq m; \\
        0,  & \text{其他}.
    \end{cases}
\end{align*}
不难算出, \(K_m^{\mathrm{T}} = -K_m\),
且 \(K_m K_m = -I_{2m}\).

按列~\(1\) 展开 \(\det {(K_m)}\), 有
\begin{align*}
    \det {(K_m)}
    = (-1)^{m+1} (-1) \det {(K_m ({m+1}|1))}
    = (-1)^m \det {
        \begin{bmatrix}
            0        & I_m \\
            -I_{m-1} & 0   \\
        \end{bmatrix}
    }.
\end{align*}
再按行~\(1\) 展开 \(\det {(K_m ({m+1}|1))}\), 有
\begin{align*}
    \det {(K_m ({m+1}|1))}
    = {} &
    (-1)^{1+m-1}\, 1 \det {(K_m ({m+1,1}|{1,m+1}))}
    \\
    = {} &
    (-1)^m \det {(K_{m-1})}.
\end{align*}
则 \(\det {(K_m)} = \det {(K_{m-1})}\),
当 \(m > 1\).
不难算出 \(\det {(K_1)} = 1\).
则 \(\det {(K_m)} = 1\), 对 \(m \geq 1\).
% 则 \(\operatorname{adj} {(K_m)} K_m = \det {(K_m)} I_{2m} = I_{2m}\).
% 又因为 \((-K_m) K_m = -(K_m K_m) = -(-I_{2m}) = I_{2m}\),
% 故 \(\operatorname{adj} {(K_m)} = -K_m\).

现在, 我们可定义辛阵 (simplektika matrico) 了.

\begin{definition}
    设 \(A\) 是一个 \(2m\)~级阵.
    若 \(A^{\mathrm{T}} K_m A = K_m\),
    则 \(A\) 是一个\emph{辛阵}.
\end{definition}

我们可写出辛阵的一些性质.

不难看出, \(2m\)~级单位阵 \(I_{2m}\) 是一个辛阵.
由 \(K_m\) 的性质, 不难验证,
\(K_m\) 也是一个辛阵.

设 \(A\), \(B\) 是 \(2m\)~级辛阵.
则
\begin{align*}
    (AB)^{\mathrm{T}} K_m (AB)
    = {} &
    (B^{\mathrm{T}} A^{\mathrm{T}})
    (K_m A) B
    \\
    = {} &
    B^{\mathrm{T}}
    (A^{\mathrm{T}} K_m A)
    B
    \\
    = {} &
    B^{\mathrm{T}} K_m B
    \\
    = {} &
    K_m.
\end{align*}

因为
\begin{align*}
    \det {(K_m)}
    = \det {(A^{\mathrm{T}} K_m A)}
    = \det {(A^{\mathrm{T}})} \det {(K_m)} \det {(A)}
    = \det {(K_m)}\, (\det {(A)})^2,
\end{align*}
且 \(\det {(K_m)} = 1\),
故 \((\det {(A)})^2 = 1\).

若数 \(t\) 适合 \(t^2 = 1\),
则
\begin{align*}
    (t A)^{\mathrm{T}} K_m (t A)
    = (t A^{\mathrm{T}}) K_m (t A)
    = t^2 (A^{\mathrm{T}} K_m A)
    = 1 K_m = K_m.
\end{align*}

注意到
\(\det {(K_m A)} = \det {(K_m)} \det {(A)} \neq 0\),
且
\begin{align*}
    ((A^{\mathrm{T}})^{\mathrm{T}} K_m A^{\mathrm{T}})
    (K_m A)
    = {} &
    (A K_m A^{\mathrm{T}})
    (K_m A)
    \\
    = {} &
    (A K_m) (A^{\mathrm{T}} K_m A)
    \\
    = {} &
    (A K_m) K_m
    \\
    = {} &
    A (K_m K_m)
    \\
    = {} &
    A (-I_{2m})
    \\
    = {} &
    {(-I_{2m})} A
    \\
    = {} &
    (K_m K_m) A
    \\
    = {} &
    K_m (K_m A),
\end{align*}
故
\((A^{\mathrm{T}})^{\mathrm{T}} K_m A^{\mathrm{T}} = K_m\).

最后, 注意到
\begin{align*}
    (\operatorname{adj} {(A)})^{\mathrm{T}}\,
    K_m
    \operatorname{adj} {(A)}
    = {} &
    (\operatorname{adj} {(A)})^{\mathrm{T}}\,
    (A^{\mathrm{T}} K_m A)
    \operatorname{adj} {(A)}
    \\
    = {} &
    ((\operatorname{adj} {(A)})^{\mathrm{T}}\, A^{\mathrm{T}})\,
    K_m\,
    (A \operatorname{adj} {(A)})
    \\
    = {} &
    (A \operatorname{adj} {(A)})^{\mathrm{T}}\,
    K_m\,
    (A \operatorname{adj} {(A)})
    \\
    = {} &
    (\det {(A)}\, I_{2m})^{\mathrm{T}}\,
    K_m\,
    (\det {(A)}\, I_{2m})
    \\
    = {} &
    (\det {(A)}\, I_{2m}^{\mathrm{T}})\, K_m\,
    (\det {(A)}\, I_{2m})
    \\
    = {} &
    (\det {(A)})^2\, (I_{2m}^{\mathrm{T}} K_m I_{2m})
    \\
    = {} &
    K_m.
\end{align*}

总结这些结果, 我们有

\begin{theorem}
    (1)
    \(I_{2m}\) 与 \(K_m\) 是 \(2m\)~级辛阵.

    (2)
    \(2m\)~级辛阵 \(A\) 的行列式的平方为 \(1\).

    (3)
    设 \(A\), \(B\) 是 \(2m\)~级辛阵.
    设数 \(t\) 适合 \(t^2 = 1\).
    则 \(AB\), \(tA\),
    \(A^{\mathrm{T}}\),
    \(\operatorname{adj} {(A)}\)
    也是辛阵.
\end{theorem}

本书是关于行列式的.
于是, 一个自然的问题是, 辛阵的行列式是多少.
我们知道, 辛阵的行列式的平方是 \(1\),
故辛阵的行列式是 \(1\) 或 \(-1\).
不过, 不平凡地, 辛阵的行列式\emph{一定}是 \(1\).
我们现有的知识无法解决此事.
我会在后面的几节, 介绍更多的知识, 以解决它.

虽然如此, 我们还是能解决 \(2m = 2\) 的情形.
设
\begin{align*}
    A = \begin{bmatrix}
            a & b \\
            c & d \\
        \end{bmatrix}
\end{align*}
是一个 \(2\)~级辛阵.
由 \(A^{\mathrm{T}} K_1 A = K_1\), 知
\begin{align*}
    \begin{bmatrix}
        a & c \\
        b & d \\
    \end{bmatrix}
    \begin{bmatrix}
        0  & 1 \\
        -1 & 0 \\
    \end{bmatrix}
    \begin{bmatrix}
        a & b \\
        c & d \\
    \end{bmatrix}
    =
    \begin{bmatrix}
        0  & 1 \\
        -1 & 0 \\
    \end{bmatrix},
\end{align*}
即
\begin{align*}
    \begin{bmatrix}
        0          & ad - bc \\
        -(ad - bc) & 0       \\
    \end{bmatrix}
    =
    \begin{bmatrix}
        0  & 1 \\
        -1 & 0 \\
    \end{bmatrix}.
\end{align*}
故
\(\det {(A)} = ad - bc = 1\).

\section{反称阵}

从本节开始, 我们讨论反称阵与其性质.

\begin{definition}
    设 \(A\) 是一个 \(n\)~级阵.
    若 \([A]_{i,i} = 0\),
    且
    \([A]_{i,j} + [A]_{j,i} = 0\),
    则 \(A\) 是一个\emph{反称阵}.
\end{definition}

\begin{example}
    不难看出, \(1\)~级反称阵即为 \([0]\),
    \(2\)~级反称阵形如
    \begin{align*}
        \begin{bmatrix}
            0  & a \\
            -a & 0 \\
        \end{bmatrix},
    \end{align*}
    \(3\)~级反称阵形如
    \begin{align*}
        \begin{bmatrix}
            0  & a  & b \\
            -a & 0  & c \\
            -b & -c & 0 \\
        \end{bmatrix},
    \end{align*}
    \(4\)~级反称阵形如
    \begin{align*}
        \begin{bmatrix}
            0  & a  & b  & c \\
            -a & 0  & d  & e \\
            -b & -d & 0  & f \\
            -c & -e & -f & 0 \\
        \end{bmatrix}.
    \end{align*}
\end{example}

\begin{example}
    不难验证, \(2m\)~级阵
    \(
    K_m =
    \begin{bmatrix}
        0    & I_m \\
        -I_m & 0   \\
    \end{bmatrix}
    \)
    是反称阵.
\end{example}

不难看出,
\([A]_{i,j} + [A]_{j,i} = 0\)
相当于 \(A^{\mathrm{T}} = -A\):
注意到 \([A^{\mathrm{T}}]_{i,j} = [A]_{j,i}\)
与 \([-A]_{i,j} = -[A]_{i,j}\).
有时, 这更方便应用.

\begin{theorem}
    设 \(A\), \(B\) 是 \(n\)~级反称阵.
    设 \(k\) 是数.
    设 \(X\) 是 \(n \times m\)~阵.

    (1)
    \(n\)~级阵 \(0\) 是反称阵.

    (2)
    \(A + B\) 是反称阵.

    (3)
    \(kA\) 是反称阵;
    特别地, \((-1)A = -A = A^{\mathrm{T}}\) 也是反称阵.

    (4)
    \(X^{\mathrm{T}} A X\) 是反称阵.
\end{theorem}

\begin{proof}
    (1)
    \(0^{\mathrm{T}} = 0 = -0\),
    且 \([0]_{i,i} = 0\).

    (2)
    由转置的性质,
    \((A + B)^{\mathrm{T}} = A^{\mathrm{T}} + B^{\mathrm{T}}
    = (-A) + (-B) = -(A + B)\).
    再注意到
    \begin{align*}
        [A + B]_{i,i} = [A]_{i,i} + [B]_{i,i} = 0 + 0 = 0.
    \end{align*}

    (3)
    由转置的性质,
    \((kA)^{\mathrm{T}} = kA^{\mathrm{T}} = k(-A) = -(kA)\).
    再注意到
    \begin{align*}
        [kA]_{i,i} = k [A]_{i,i} = k 0 = 0.
    \end{align*}

    (4)
    \(X\) 是 \(n \times m\)~的, 则 \(AX\) 是 \(n \times m\) 的;
    \(X^{\mathrm{T}}\) 是 \(m \times n\)~的,
    则 \(X^{\mathrm{T}} A X\) 是 \(m \times m\)~的.
    由转置的性质,
    \begin{align*}
        (X^{\mathrm{T}} A X)^{\mathrm{T}}
        = X^{\mathrm{T}} A^{\mathrm{T}} (X^{\mathrm{T}})^{\mathrm{T}}
        = X^{\mathrm{T}} (-A) X
        = -(X^{\mathrm{T}} A X).
    \end{align*}
    由阵的积的定义,
    \begin{align*}
             &
        [X^{\mathrm{T}} A X]_{i,i}
        \\
        = {} &
        [X^{\mathrm{T}} (A X)]_{i,i}
        \\
        = {} &
        \sum_{\ell = 1}^{n}
        {[X^{\mathrm{T}}]_{i,\ell} [A X]_{\ell,i}}
        \\
        = {} &
        \sum_{\ell = 1}^{n}
        {[X]_{\ell,i} [A X]_{\ell,i}}
        \\
        = {} &
        \sum_{\ell = 1}^{n}
        {
        [X]_{\ell,i}
        \left(\sum_{k = 1}^{n} [A]_{\ell,k} [X]_{k,i}\right)
        }
        \\
        = {} &
        \sum_{\ell = 1}^{n}
        \sum_{k = 1}^{n}
        {[X]_{\ell,i} [A]_{\ell,k} [X]_{k,i}}
        \\
        = {} &
        \sum_{\ell, k = 1}^{n}
        {[X]_{\ell,i} [X]_{k,i} [A]_{\ell,k}}
        \\
        = {} & \hphantom{{} + {}}
        \sum_{1 \leq \ell = k \leq n}
        {[X]_{\ell,i} [X]_{k,i} [A]_{\ell,k}}
        +
        \sum_{1 \leq \ell < k \leq n}
        {[X]_{\ell,i} [X]_{k,i} [A]_{\ell,k}}
        \\
             & +
        \sum_{1 \leq k < \ell \leq n}
        {[X]_{\ell,i} [X]_{k,i} [A]_{\ell,k}}
        \\
        = {} & \hphantom{{} + {}}
        \sum_{1 \leq \ell = k \leq n}
        {[X]_{\ell,i} [X]_{k,i}\, 0}
        +
        \sum_{1 \leq \ell < k \leq n}
        {[X]_{\ell,i} [X]_{k,i} [A]_{\ell,k}}
        \\
             & +
        \sum_{1 \leq \ell < k \leq n}
        {[X]_{k,i} [X]_{\ell,i} [A]_{k,\ell}}
        \\
        = {} &
        0 +
        \sum_{1 \leq \ell < k \leq n}
        {[X]_{\ell,i} [X]_{k,i} ([A]_{\ell,k} + [A]_{k,\ell})}
        \\
        = {} &
        \sum_{1 \leq \ell < k \leq n}
        {[X]_{\ell,i} [X]_{k,i}\, 0}
        \\
        = {} &
        0.
    \end{align*}
    于是, \(X^{\mathrm{T}} A X\) 是一个 \(m\)~级反称阵.
\end{proof}

\section{\texorpdfstring{奇数级反称阵的行列式为 \(0\)}
  {奇数级反称阵的行列式为 0}}

我们讨论反称阵的行列式.

我们先计算小级反称阵的行列式.

\begin{example}
    设 \(A\) 是 \(1\)~级反称阵.
    则 \(A = [0]\).
    故 \(\det {(A)} = 0\).
\end{example}

\begin{example}
    设 \(A\) 是 \(2\)~级反称阵.
    则 \(A\) 形如
    \begin{align*}
        \begin{bmatrix}
            0  & a \\
            -a & 0 \\
        \end{bmatrix}.
    \end{align*}
    则 \(\det {(A)} = a^2\),
    即 \(\det {(A)} = [A]_{1,2}^2\).
\end{example}

\begin{example}
    设 \(A\) 是 \(3\)~级反称阵.
    则 \(A\) 形如
    \begin{align*}
        \begin{bmatrix}
            0  & a  & b \\
            -a & 0  & c \\
            -b & -c & 0 \\
        \end{bmatrix}.
    \end{align*}
    则
    \begin{align*}
        \det {(A)}
        = {} &
        0
        \det {\begin{bmatrix}
                      0  & c \\
                      -c & 0 \\
                  \end{bmatrix}}
        - (-a)
        \det {\begin{bmatrix}
                      a  & b \\
                      -c & 0 \\
                  \end{bmatrix}}
        + (-b)
        \det {\begin{bmatrix}
                      a & b \\
                      0 & c \\
                  \end{bmatrix}}
        \\
        = {} &
        a(bc) + (-b)(ac)
        \\
        = {} & 0.
    \end{align*}
\end{example}

\begin{example}
    设 \(A\) 是 \(4\)~级反称阵.
    则 \(A\) 形如
    \begin{align*}
        \begin{bmatrix}
            0  & a  & b  & c \\
            -a & 0  & d  & e \\
            -b & -d & 0  & f \\
            -c & -e & -f & 0 \\
        \end{bmatrix}.
    \end{align*}
    则
    \begin{align*}
             & \det {(A)}
        \\
        = {} &
        \hphantom{{} + {}}
        0
        \det {\begin{bmatrix}
                      0  & d  & e \\
                      -d & 0  & f \\
                      -e & -f & 0 \\
                  \end{bmatrix}}
        - (-a)
        \det {\begin{bmatrix}
                      a  & b  & c \\
                      -d & 0  & f \\
                      -e & -f & 0 \\
                  \end{bmatrix}}
        \\
             &
        + (-b)
        \det {\begin{bmatrix}
                      a  & b  & c \\
                      0  & d  & e \\
                      -e & -f & 0 \\
                  \end{bmatrix}}
        - (-c)
        \det {\begin{bmatrix}
                      a  & b & c \\
                      0  & d & e \\
                      -d & 0 & f \\
                  \end{bmatrix}}
        \\
        = {} &
        a (dfc - ebf + aff)
        - b (-ebe + afe + edc)
        + c (adf - dbe + ddc)
        \\
        = {} &
        af (af - be + cd)
        - be (af - be + cd)
        + cd (af - be + cd)
        \\
        = {} & (af - be + cd)^2,
    \end{align*}
    即
    \begin{align*}
        \det {(A)}
        = ([A]_{1,2} [A]_{3,4} - [A]_{1,3} [A]_{2,4}
        + [A]_{1,4} [A]_{2,3})^2.
    \end{align*}
\end{example}

我们发现, \(1\)~级反称阵与 \(3\)~级反称阵%
的行列式为 \(0\).
一般地, 我们有

\begin{theorem}
    设 \(A\) 是一个奇数级反称阵.
    则 \(\det {(A)} = 0\).
\end{theorem}

我的论证会用如下几个事实.

(1)
设 \(A\) 是一个 \(n\)~级阵.
则
\begin{align*}
    \det {(A)}
    = \sum_{i = 1}^{n}
    {(-1)^{i+1} [A]_{i,1} \det {(A(i|1))}}.
\end{align*}
这就是按列~\(1\) 展开行列式.

(2)
设 \(A\) 是一个 \(n\)~级阵.
则
\begin{align*}
    \det {(A)}
    = \sum_{j = 1}^{n}
    {(-1)^{1+j} [A]_{1,j} \det {(A(1|j))}}.
\end{align*}
这就是按行~\(1\) 展开行列式.

(3)
设 \(A\) 是一个 \(n\)~级阵 (\(n \geq 3\)).
同时用 (1) (2),
有
\begin{align*}
         & \det {(A)}
    \\
    = {} &
    [A]_{1,1} \det {(A(1|1))}
    +
    \sum_{j = 2}^{n} {
    (-1)^{1+j} [A]_{1,j} \det {(A(1|j))}
    }
    \\
    = {} &
    [A]_{1,1} \det {(A(1|1))}
    +
    \sum_{j = 2}^{n} {
    (-1)^{1+j} [A]_{1,j}
    \sum_{i = 2}^{n} {
    (-1)^{i-1+1} [A]_{i,1} \det {(A(1,i|j,1))}
    }
    }
    \\
    = {} &
    [A]_{1,1} \det {(A(1|1))}
    +
    \sum_{j = 2}^{n} {
    \sum_{i = 2}^{n} {
    (-1)^{1+j} [A]_{1,j}
    (-1)^{i-1+1} [A]_{i,1} \det {(A(1,i|j,1))}
    }
    }
    \\
    = {} &
    [A]_{1,1} \det {(A(1|1))}
    +
    \sum_{i, j = 2}^{n} {
    (-1)^{i+j-1}
        [A]_{i,1} [A]_{1,j} \det {(A(1,i|1,j))}
    }.
\end{align*}
特别地, 若 \(A\) 还是一个反称阵, 则
\begin{align*}
    \det {(A)}
    = \sum_{i, j = 2}^{n} {
    (-1)^{i+j}
        [A]_{1,i} [A]_{1,j} \det {(A(1,i|1,j))}
    }.
\end{align*}

(4)
设 \(A\) 是一个 \(n\)~级阵.
则 \(\det {(A^{\mathrm{T}})} = \det {(A)}\).
这就是行列式与转置的关系.

(5)
设 \(A\) 是一个 \(n\)~级阵.
设 \(u\) 是一个数.
则 \(\det {(uA)} = u^n \det {(A)}\).
% 这是我讲过的例题.
特别地,
\(\det {(-A)} = (-1)^n \det {(A)}\).

(6)
设 \(A\) 是一个 \(n\)~级反称阵 (\(n \geq 3\)).
设 \(r\) 是不超过 \(n\) 的正整数.
设 \(s\), \(t\) 是二个不超过 \(n\) 的正整数,
且 \(s \neq r\), \(t \neq r\).
则
\(A(r,t|r,s) = -(A(r,s|r,t))^{\mathrm{T}}\).
并且, \(A(r,s|r,s)\)
是一个 \(n-2\)~级反称阵.

(7)
每一个适合条件
``\(i\), \(j\) 都是不低于 \(p\), 且不低于 \(q\) 的整数''
的有序对 \((i, j)\) \emph{恰}适合以下三个条件之一:
(a)
\(p \leq i = j \leq q\);
(b)
\(p \leq i < j \leq q\);
(c)
\(p \leq j < i \leq q\).

(8)
任取一个正奇数 \(n\), 一定存在一个正整数 \(k\)
使 \(n = 2k - 1\).

\vspace{2ex}

介绍完这几件事后, 我总算可以证明定理了.

\begin{proof}
    记命题 \(P(k)\) 为
    \begin{quotation}
        每一个 \(2k - 1\)~级反称阵的行列式都是 \(0\).
    \end{quotation}
    我们用数学归纳法证明,
    对任何正整数~\(k\),
    \(P(k)\) 是正确的.

    不难验证 \(P(1)\) 是正确的.

    现在, 我们假定 \(P(k-1)\) 是正确的 (\(k \geq 2\)).
    我们要证 \(P(k)\) 也是正确的.
    记 \(n = 2k-1\).
    设 \(A\) 是一个 \(n\)~级反称阵.
    那么
    \begin{align*}
             & \det {(A)}     \\
        = {} &
        \sum_{i, j = 2}^{n} {
        (-1)^{i+j}
            [A]_{1,i} [A]_{1,j} \det {(A(1,i|1,j))}
        }
        \\
        = {} &
        \hphantom{{} + {}}
        \sum_{2 \leq i = j \leq n} {
            (-1)^{i+j}
                [A]_{1,i} [A]_{1,j} \det {(A(1,i|1,j))}
        }
        \\
             & +
        \sum_{2 \leq i < j \leq n} {
            (-1)^{i+j}
                [A]_{1,i} [A]_{1,j} \det {(A(1,i|1,j))}
        }
        \\
             & +
        \sum_{2 \leq j < i \leq n} {
            (-1)^{i+j}
                [A]_{1,i} [A]_{1,j} \det {(A(1,i|1,j))}
        }
        \\
        = {} &
        \hphantom{{} + {}}
        \sum_{2 \leq i \leq n} {
            (-1)^{i+i}
                [A]_{1,i} [A]_{1,i} \det {(A(1,i|1,i))}
        }
        \\
             & +
        \sum_{2 \leq i < j \leq n} {
            (-1)^{i+j}
                [A]_{1,i} [A]_{1,j} \det {(A(1,i|1,j))}
        }
        \\
             & +
        \sum_{2 \leq i < j \leq n} {
            (-1)^{j+i}
                [A]_{1,j} [A]_{1,i} \det {(A(1,j|1,i))}
        }
        \\
        = {} &
        \hphantom{{} + {}}
        \sum_{2 \leq i \leq n} {
            [A]_{1,i}^2 \det {(A(1,i|1,i))}
        }
        \\
             & +
        \sum_{2 \leq i < j \leq n} {
            (-1)^{i+j}
                [A]_{1,i} [A]_{1,j} \det {(A(1,i|1,j))}
        }
        \\
             & +
        \sum_{2 \leq i < j \leq n} {
            (-1)^{i+j}
                [A]_{1,i} [A]_{1,j}
            \det {(-(A(1,i|1,j))^{\mathrm{T}})}
        }
        \\
        = {} &
        \hphantom{{} + {}}
        \sum_{2 \leq i \leq n} {
            [A]_{1,i}^2 \det {(A(1,i|1,i))}
        }
        \\
             & +
        \sum_{2 \leq i < j \leq n} {
            (-1)^{i+j}
                [A]_{1,i} [A]_{1,j} \det {(A(1,i|1,j))}
        }
        \\
             & +
        \sum_{2 \leq i < j \leq n} {
            (-1)^{i+j}
                [A]_{1,i} [A]_{1,j}
            (-1)^{n-2} \det {((A(1,i|1,j))^{\mathrm{T}})}
        }
        \\
        = {} &
        \hphantom{{} + {}}
        \sum_{2 \leq i \leq n} {
            [A]_{1,i}^2 \det {(A(1,i|1,i))}
        }
        \\
             & +
        \sum_{2 \leq i < j \leq n} {
            (-1)^{i+j}
                [A]_{1,i} [A]_{1,j} \det {(A(1,i|1,j))}
        }
        \\
             & +
        \sum_{2 \leq i < j \leq n} {
            (-1)^{i+j}
                [A]_{1,i} [A]_{1,j}
            (-1)^{n} \det {(A(1,i|1,j))}
        }
        \\
        = {} &
        \hphantom{{} + {}}
        \sum_{2 \leq i \leq n} {
            [A]_{1,i}^2 \det {(A(1,i|1,i))}
        }
        \\
             & + (1 + (-1)^n)
        \sum_{2 \leq i < j \leq n} {
            (-1)^{i+j}
                [A]_{1,i} [A]_{1,j} \det {(A(1,i|1,j))}
        }.
    \end{align*}
    注意到 \(A(1,i|1,i)\) 是
    \(n - 2\)~级,
    即 \(2(k-1) - 1\)~级反称阵;
    由假定,
    其行列式为 \(0\).
    再注意到 \(n = 2k - 1\),
    故 \(1 + (-1)^n = 0\).
    所以, \(\det {(A)} = 0\).

    所以, \(P(k)\) 是正确的.
    由数学归纳法原理, 待证命题成立.
\end{proof}

我们还不了解偶数级反称阵的行列式.
不过, 我们之后会了解它的.

\section{阵的积与倍加}

前面, 在研究行列式的性质时, 我们引入了一种叫 ``倍加'' 的行为:

\DefinitionMultiplyAndAdd*

我们说, 有列的倍加, 也有行的倍加.
具体地, 加 \(A\)~的一行的倍于另一行,
且不改变其他的行, 是一次行的倍加.

回想起, 行列式有倍加不变性.
具体地, 若我们加方阵 \(A\)~的一列 (行) 的倍于另一列 (行),
且不改变其他的列 (行), 得方阵 \(B\),
则 \(\det {(B)} = \det {(A)}\).

本节, 我们讨论阵的积与倍加的关系.

设 \(A\) 是 \(m \times n\) 阵.
设 \(A\) 的%
列~\(1\), \(2\), \(\dots\), \(n\) 分别是
\(a_1\), \(a_2\), \(\dots\), \(a_n\).
则 \(A = [a_1, a_2, \dots, a_n]\).
再设 \(n\)~级单位阵 \(I_n\) 的%
列~\(1\), \(2\), \(\dots\), \(n\) 分别是
\(e_1\), \(e_2\), \(\dots\), \(e_n\).
则 \(I_n = [e_1, e_2, \dots, e_n]\).
由 \(A I_n = A\), 知
\begin{align*}
    [Ae_1, Ae_2, \dots, Ae_n]
    = A[e_1, e_2, \dots, e_n]
    = [a_1, a_2, \dots, a_n].
\end{align*}
故 \(Ae_j = a_j\),
对任何不超过 \(n\) 的正整数 \(j\).

设 \(p \neq q\).
我们加 \(A\) 的列~\(p\) 的 \(s\)~倍于列~\(q\),
不改变其他的列, 得 \(n\)~级阵 \(B\).
设 \(B\) 的%
列~\(1\), \(2\), \(\dots\), \(n\) 分别是
\(b_1\), \(b_2\), \(\dots\), \(b_n\).
则 \(B = [b_1, b_2, \dots, b_n]\).
则 \(j \neq q\) 时, 有 \(b_j = a_j\),
且 \(b_q = a_q + s a_p\).
则 \(j \neq q\) 时,
\begin{align*}
    B e_j = b_j = a_j = A e_j,
\end{align*}
且
\begin{align*}
    B e_q = b_q = a_q + s a_p = A e_q + s (A e_p) = A (e_q + s e_p).
\end{align*}
作 \(n\)~级阵
\(E(n; p, q; s) = [f_1, f_2, \dots, f_n]\),
其中
\begin{align*}
    f_j
    = \begin{cases}
          e_j,         & j \neq q; \\
          e_q + s e_p, & j = q.
      \end{cases}
\end{align*}
则 \(B e_j = A f_j\),
对任何不超过 \(n\) 的正整数 \(j\).
则
\begin{align*}
    B = BI_n = [B e_1, B e_2, \dots, B e_n]
    = [A f_1, A f_2, \dots, A f_n]
    = A E(n; p, q; s).
\end{align*}
不难写出
\begin{align*}
    [E(n; p, q; s)]_{i,j}
    = \begin{cases}
          s,     & \text{\(i = p\), 且 \(j = q\)}; \\
          [I_n], & \text{其他}.
      \end{cases}
\end{align*}
于是, 我们有

\begin{theorem}
    设 \(A\) 是 \(m \times n\)~阵.
    设 \(p\), \(q\) 是不超过 \(n\)~的正整数,
    且 \(p \neq q\).
    设 \(s\) 是数.
    作 \(n\)~级阵 \(E(n; p, q; s)\) 如下:
    \begin{align*}
        [E(n; p, q; s)]_{i,j}
        = \begin{cases}
              s,     & \text{\(i = p\), 且 \(j = q\)}; \\
              [I_n], & \text{其他}.
          \end{cases}
    \end{align*}
    (通俗地,
    加 \(I_n\) 的列~\(p\) 的 \(s\)~倍于列~\(q\),
    不改变其他的列, 得 \(n\)~级阵 \(E(n; p, q; s)\).)
    那么,
    加 \(A\) 的列~\(p\) 的 \(s\)~倍于列~\(q\),
    不改变其他的列, 得 \(m \times n\)~阵 \(A E(n; p, q; s)\).
\end{theorem}

\begin{proof}
    前面的说明就是一个证明.
    当然, 不考虑前面的说明, 我们也可直接用阵的积的定义验证它.
    毕竟, 我们的目标是
    \begin{equation*}
        [A E(n; p, q; s)]_{i,j}
        = \begin{cases}
            [A]_{i,j},              & j \neq q; \\
            [A]_{i,q} + s[A]_{i,p}, & j = q.
        \end{cases}
        \qedhere
    \end{equation*}
\end{proof}

列的倍加可用阵的积实现.
行的倍加当然也可用阵的积实现.
具体地,

\begin{theorem}
    设 \(A\) 是 \(m \times n\)~阵.
    设 \(p\), \(q\) 是不超过 \(m\)~的正整数,
    且 \(p \neq q\).
    设 \(s\) 是数.
    作 \(m\)~级阵 \(E(m; q, p; s)\) 如下:
    \begin{align*}
        [E(m; q, p; s)]_{i,j}
        = \begin{cases}
              s,     & \text{\(i = q\), 且 \(j = p\)}; \\
              [I_m], & \text{其他}.
          \end{cases}
    \end{align*}
    (通俗地,
    加 \(I_m\) 的行~\(p\) 的 \(s\)~倍于行~\(q\),
    不改变其他的行, 得 \(m\)~级阵 \(E(m; q, p; s)\).)
    那么,
    加 \(A\) 的列~\(p\) 的 \(s\)~倍于列~\(q\),
    不改变其他的列, 得 \(m \times n\)~阵 \(E(m; q, p; s) A\).
\end{theorem}

\begin{proof}
    请允许我留此事为您的习题.
    当然, 我想给您一个提示:
    证明此事的要点是验证
    \begin{equation*}
        [E(m; q, p; s) A]_{i,j}
        = \begin{cases}
            [A]_{i,j},              & i \neq q; \\
            [A]_{q,j} + s[A]_{p,j}, & i = q.
        \end{cases}
        \qedhere
    \end{equation*}
\end{proof}

我再说几件事.
设 \(p\), \(q\) 是不超过 \(n\) 的正整数,
且 \(p \neq q\).

不难看出, \(E(n; p, q; s)\) 的行列式为 \(1\):
毕竟, 这是对单位阵作一次倍加后得到的方阵,
且单位阵的行列式为 \(1\).

不难验证, \(E(n; p, q; s)\) 的转置是
\(E(n; q, p; s)\);
对比这二个阵的元即可.

最后, 我说, 用阵的积表示倍加是好的.
我们知道, 阵的积适合一些运算律;
于是, 我们或可用阵的运算律发现倍加的某些规律.
反过来, 我们也或可用倍加发现阵的一些性质.

\section{反称阵与倍加}

本节, 我们用倍加研究反称阵.

先看一个简单的结果.

\begin{theorem}
    设 \(A\) 是 \(n\)~级反称阵.
    设 \(p\), \(q\) 是不超过 \(n\)~的正整数,
    且 \(p \neq q\).
    设 \(s\) 是数.

    (1)
    加 \(A\) 的列~\(p\) 的 \(s\)~倍于列~\(q\),
    不改变其他的列, 得 \(n\)~级阵 \(B\).
    加 \(B\) 的行~\(p\) 的 \(s\)~倍于行~\(q\),
    不改变其他的行, 得 \(n\)~级阵 \(C\).
    则 \(C\) 是反称阵.
    % \begin{align*}
    %     [C]_{i,j} =
    %     \begin{cases}
    %         [A]_{q,j} + s [A]_{p,j},
    %          & \text{\(i = q\), 且 \(j \neq q\)}; \\
    %         [A]_{i,q} + s [A]_{i,p},
    %          & \text{\(j = q\), 且 \(i \neq q\)}; \\
    %         [A]_{i,j},
    %          & \text{其他}.
    %     \end{cases}
    % \end{align*}
    通俗地,
    对反称阵先作一次列的倍加,
    再作一次对应的行的倍加,
    则二次倍加后的阵仍是反称阵.

    (2)
    在 (1) 中, 先行后列不影响结果.
    具体地, 设
    加 \(A\) 的行~\(p\) 的 \(s\)~倍于行~\(q\),
    不改变其他的行, 得 \(n\)~级阵 \(F\).
    加 \(F\) 的列~\(p\) 的 \(s\)~倍于列~\(q\),
    不改变其他的列, 得 \(n\)~级阵 \(G\).
    则 \(G = C\).
\end{theorem}

\begin{proof}
    (1)
    我用二种方法证明此事.

    法~1:
    设加 \(A\) 的列~\(p\) 的 \(s\)~倍于列~\(q\),
    不改变其他的列, 得 \(n\)~级阵 \(B\).
    则
    \begin{align*}
        [B]_{i,j}
        = \begin{cases}
              [A]_{i,j},              & j \neq q; \\
              [A]_{i,q} + s[A]_{i,p}, & j = q.
          \end{cases}
    \end{align*}
    设加 \(B\) 的行~\(p\) 的 \(s\)~倍于行~\(q\),
    不改变其他的行, 得 \(n\)~级阵 \(C\).
    则
    \begin{align*}
        [C]_{i,j}
        = \begin{cases}
              [B]_{i,j},              & i \neq q; \\
              [B]_{q,j} + s[B]_{p,j}, & i = q.
          \end{cases}
    \end{align*}
    当 \(i = q\), 且 \(j \neq q\) 时,
    \begin{align*}
        [C]_{i,j}
            = [B]_{q,j} + s[B]_{p,j}
            = [A]_{q,j} + s[A]_{p,j};
    \end{align*}
    当 \(j = q\), 且 \(i \neq q\) 时,
    \begin{align*}
        [C]_{i,j}
            = [B]_{i,q}
            = [A]_{i,q} + s[A]_{i,p};
    \end{align*}
    当 \(i = q = j\) 时,
    \begin{align*}
        [C]_{i,j}
        = {} &
        [B]_{q,q} + s[B]_{p,q}
        \\
        = {} &
        ([A]_{q,q} + s[A]_{q,p}) + s([A]_{p,q} + s[A]_{p,p})
        \\
        = {} &
        [A]_{q,q} + s([A]_{q,p} + s[A]_{p,q}) + s^2 [A]_{p,p}
        \\
        = {} &
        0 + s 0 + s^2 0
        \\
        = {} &
        0
        \\
        = {} &
        [A]_{i,j};
    \end{align*}
    当 \(i \neq q\), 且 \(j \neq q\) 时,
    \begin{align*}
        [C]_{i,j} = [B]_{i,j} = [A]_{i,j}.
    \end{align*}
    由此, 不难验证, \(C\) 是一个反称阵.

    法~2:
    设 \(u\), \(v\) 是不超过 \(n\) 的正整数, 且 \(u \neq v\).
    作 \(n\)~级阵 \(E(n; u, v; s)\) 如下:
    \begin{align*}
        [E(n; u, v; s)]_{i,j}
        = \begin{cases}
              s,     & \text{\(i = u\), 且 \(j = v\)}; \\
              [I_n], & \text{其他}.
          \end{cases}
    \end{align*}
    于是, 加 \(A\) 的列~\(p\) 的 \(s\)~倍于列~\(q\),
    不改变其他的列, 得 \(n\)~级阵 \(B = A E(n; p, q; s)\).
    进一步地,
    加 \(B\) 的行~\(p\) 的 \(s\)~倍于行~\(q\),
    不改变其他的行, 得 \(n\)~级阵 \(C = E(n; q, p; s) B\).
    则
    \begin{align*}
        C = E(n; q, p; s) (A E(n; p, q; s))
        = (E(n; p, q; s))^{\mathrm{T}} A E(n; p, q; s).
    \end{align*}
    故 \(C\) 是一个反称阵.

    注意到, 法~2 利用了已有的阵的运算律:
    法~1 是较直接的; 法~2 是较聪明的.

    (2)
    不难写出,
    \begin{align*}
        F = E(n; q, p; s) A = (E(n; p, q; s))^{\mathrm{T}} A,
    \end{align*}
    且
    \begin{align*}
        G = F E(n; p, q; s)
        = ((E(n; p, q; s))^{\mathrm{T}} A) E(n; p, q; s).
    \end{align*}
    因为结合律, 我们有
    \begin{equation*}
        G
        = (E(n; p, q; s))^{\mathrm{T}} (A E(n; p, q; s))
        = C.
        \qedhere
    \end{equation*}
\end{proof}

利用此事, 我们可证明如下重要的定理.

\begin{theorem}
    设 \(A\) 是 \(n\)~级反称阵.
    利用若干次列的倍加, 与对应的行的倍加,
    我们可变 \(A\) 为形如
    \begin{equation}
        \begin{bmatrix}
            0      & b_1    & 0      & 0      & \cdots & 0      & 0      \\
            -b_1   & 0      & 0      & 0      & \cdots & 0      & 0      \\
            0      & 0      & 0      & b_2    & \cdots & 0      & 0      \\
            0      & 0      & -b_2   & 0      & \cdots & 0      & 0      \\
            \vdots & \vdots & \vdots & \vdots & \ddots & \vdots & \vdots \\
            0      & 0      & 0      & 0      & \cdots & 0      & b_m    \\
            0      & 0      & 0      & 0      & \cdots & -b_m   & 0      \\
        \end{bmatrix}
        \label{eq:C3201}
    \end{equation}
    (若 \(n\) 是偶数), 或
    \begin{equation}
        \begin{bmatrix}
            0      & b_1    & 0      & 0      & \cdots & 0      & 0      & 0      \\
            -b_1   & 0      & 0      & 0      & \cdots & 0      & 0      & 0      \\
            0      & 0      & 0      & b_2    & \cdots & 0      & 0      & 0      \\
            0      & 0      & -b_2   & 0      & \cdots & 0      & 0      & 0      \\
            \vdots & \vdots & \vdots & \vdots & \ddots & \vdots & \vdots & \vdots \\
            0      & 0      & 0      & 0      & \cdots & 0      & b_k    & 0      \\
            0      & 0      & 0      & 0      & \cdots & -b_k   & 0      & 0      \\
            0      & 0      & 0      & 0      & \cdots & 0      & 0      & 0      \\
        \end{bmatrix}
        \label{eq:C3202}
    \end{equation}
    (若 \(n\) 是奇数) 的反称阵.

    我们约定, 作倍加时, 我们先列后行, 交替地作.
    具体地, 我们先作一次列的倍加
    (比如, 加列~\(p\) 的 \(s\)~倍于列~\(q\),
    其中 \(p \neq q\)),
    然后立即作一次对应的行的倍加
    (加行~\(p\) 的 \(s\)~倍于行~\(q\)).
    然后再作一次列的, 且再作一次对应的行的 (若还有)
    \(\dots \dots\).

    当然, 若 \(n = 2\), 则 \(A\) 已形如
    \begin{align*}
        \begin{bmatrix}
            0  & a \\
            -a & 0 \\
        \end{bmatrix}
    \end{align*}
    (取式~\eqref{eq:C3201} 的 \(m\) 为 \(1\));
    若 \(n = 1\), 则 \(A\) 已形如
    \([0]\)
    (取式~\eqref{eq:C3202} 的 \(k\) 为 \(0\)).
\end{theorem}

\begin{proof}
    作命题 \(P(n)\):
    对任何 \(n\)~级反称阵 \(A\),
    存在若干次列的倍加, 与对应的行的倍加,
    变 \(A\) 为一个%
    形如式~\eqref{eq:C3201} (若 \(n\) 是偶数)
    或式~\eqref{eq:C3202} (若 \(n\) 是奇数)
    的反称阵.
    再作命题 \(Q(n)\):
    \(P(n-1)\) 与 \(P(n)\) 是对的.
    我们用数学归纳法证明:
    对任何高于 \(1\) 的整数 \(n\), \(Q(n)\) 是对的.

    \(Q(2)\) 是对的,
    因为 \(P(1)\) 与 \(P(2)\) 是对的;
    注意到,
    我们可加一列 (行) 的 \(0\)~倍于另一列 (行),
    从而 ``什么也不变''.

    我们设 \(Q(n-1)\) 是对的 (\(n \geq 3\)).
    则 \(P(n-2)\) 与 \(P(n-1)\) 是对的.
    我们要由此证明 \(Q(n)\) 是对的,
    即 \(P(n-1)\) 与 \(P(n)\) 是对的.
    \(P(n-1)\) 当然是对的, 由假定.
    所以, 我们由假定
    ``\(P(n-2)\) 与 \(P(n-1)\) 是对的''
    证 \(P(n)\) 是对的,
    即得 \(Q(n)\) 是对的.

    任取一个 \(n\)~级反称阵 \(A\).
    我们先说明:
    存在若干次列的倍加, 与对应的行的倍加,
    变 \(A\) 为一个反称阵 \(C\),
    其中
    \([C]_{1,j} = [C]_{2,j} = 0\),
    \([C]_{j,1} = [C]_{j,2} = 0\),
    对任何高于 \(2\), 且不超过 \(n\) 的正整数 \(j\).

    若对任何高于 \(2\), 且不超过 \(n\) 的正整数 \(j\),
    已有
    \([A]_{1,j} = [A]_{2,j} = 0\),
    \([A]_{j,1} = [A]_{j,2} = 0\),
    我们 ``什么也不变'', 取 \(C = A\) 即可.

    若 \([A]_{1,2} \neq 0\),
    则 \([A]_{2,1}\) 当然也不是零.
    我们加 \(A\) 的%
    列~\(2\) 的 \(-[A]_{1,3}/[A]_{1,2}\) 倍于列~\(3\),
    且加行~\(2\) 的 \(-[A]_{3,1}/[A]_{2,1} = -[A]_{1,3}/[A]_{1,2}\)
    倍于行~\(3\), 得阵 \(F_3\).
    则 \(F_3\) 是一个反称阵,
    \([F_3]_{1,3} = 0\),
    \([F_3]_{3,1} = 0\),
    且 \([F_3]_{1,j} = [A]_{1,j}\),
    \([F_3]_{j,1} = [A]_{j,1}\)
    (\(j \neq 3\)).
    接着, 我们加 \(F_3\) 的%
    列~\(2\) 的 \(-[F_3]_{1,4}/[F_3]_{1,2}\) 倍于列~\(4\),
    且加行~\(2\) 的 \(-[F_3]_{4,1}/[F_3]_{2,1}
    = -[F_3]_{1,4}/[F_3]_{1,2}\)
    倍于行~\(4\), 得阵 \(F_4\).
    则 \(F_4\) 是一个反称阵,
    \([F_4]_{1,3} = [F_4]_{1,4} = 0\),
    \([F_4]_{3,1} = [F_4]_{4,1} = 0\),
    且 \([F_4]_{1,j} = [F_3]_{1,j}\),
    \([F_4]_{j,1} = [F_3]_{j,1}\)
    (\(j \neq 4\)).
    \(\dots \dots\)
    接着, 我们加 \(F_{n-1}\) 的%
    列~\(2\) 的 \(-[F_{n-1}]_{1,n}/[F_{n-1}]_{1,2}\) 倍于列~\(n\),
    且加行~\(2\) 的 \(-[F_{n-1}]_{n,1}/[F_{n-1}]_{2,1}
    = -[F_{n-1}]_{1,n}/[F_{n-1}]_{1,2}\)
    倍于行~\(n\), 得阵 \(F_n\).
    则 \(F_n\) 是一个反称阵,
    \([F_n]_{1,j} = 0\),
    \([F_n]_{j,1} = 0\)
    (\(j = 3\), \(4\), \(\dots\)),
    且 \([F_n]_{1,2} = -[F_n]_{2,1} = [A]_{1,2} \neq 0\).
    然后, 我们加 \(F_n\) 的%
    列~\(1\) 的 \(-[F_n]_{2,3}/[F_n]_{2,1}\) 倍于列~\(3\),
    且加行~\(1\) 的 \(-[F_n]_{3,2}/[F_n]_{1,2} = -[F_n]_{2,3}/[F_n]_{2,1}\)
    倍于行~\(3\), 得阵 \(G_3\).
    则 \(G_3\) 是一个反称阵,
    \([G_3]_{2,3} = 0\),
    \([G_3]_{3,2} = 0\),
    且 \([G_3]_{2,j} = [F_n]_{2,j}\),
    \([G_3]_{j,2} = [F_n]_{j,2}\)
    (\(j \neq 3\)).
    接着, 我们加 \(G_3\) 的%
    列~\(1\) 的 \(-[G_3]_{2,4}/[G_3]_{2,1}\) 倍于列~\(4\),
    且加行~\(1\) 的 \(-[G_3]_{4,2}/[G_3]_{1,2} = -[G_3]_{2,4}/[G_3]_{2,1}\)
    倍于行~\(4\), 得阵 \(G_4\).
    则 \(G_4\) 是一个反称阵,
    \([G_4]_{2,3} = [G_4]_{2,4} = 0\),
    \([G_4]_{3,2} = [G_4]_{4,2} = 0\),
    且 \([G_4]_{2,j} = [F_n]_{2,j}\),
    \([G_4]_{j,2} = [F_n]_{j,2}\)
    (\(j \neq 4\)).
    \(\dots \dots\)
    最后, 我们加 \(G_{n-1}\) 的%
    列~\(1\) 的 \(-[G_{n-1}]_{2,n}/[G_{n-1}]_{2,1}\) 倍于列~\(n\),
    且加行~\(1\) 的 \(-[G_{n-1}]_{n,2}/[G_{n-1}]_{1,2}
    = -[G_{n-1}]_{2,n}/[G_{n-1}]_{2,1}\)
    倍于行~\(n\), 得阵 \(G_n\).
    则 \(G_n\) 是一个反称阵,
    \([G_n]_{1,j} = 0\),
    \([G_n]_{j,1} = 0\),
    \([G_n]_{2,j} = 0\),
    \([G_n]_{j,2} = 0\),
    (\(j = 3\), \(4\), \(\dots\)).
    取 \(C\) 为 \(G_n\) 即可.

    若 \([A]_{1,2} = -[A]_{2,1} = 0\),
    但有某个 \([A]_{\ell,j} \neq 0\)
    (\(\ell = 1\) 或 \(2\),
    且 \(j > 2\)),
    我们可加 \(A\) 的列~\(j\) 于列~\(3 - \ell\),
    且加行~\(j\) 于行~\(3 - \ell\)
    (注意到, 当 \(\ell = 1\) 或 \(2\) 时,
    \(3 - \ell = 2\) 或 \(1\), 分别地),
    得阵 \(D\).
    则 \([D]_{\ell,3 - \ell} \neq 0\),
    这就转化问题为前面讨论过的情形.

    综上, 作若干次列的倍加, 与对应的行的倍加,
    我们可变 \(A\) 为一个反称阵 \(C\),
    其中
    \([C]_{1,j} = [C]_{2,j} = 0\),
    \([C]_{j,1} = [C]_{j,2} = 0\),
    对任何高于 \(2\), 且不超过 \(n\) 的正整数 \(j\).

    考虑 \(C\) 的右下角的 \(n-2\)~级子阵 \(C({1,2}|{1,2})\).
    不难看出, 它是一个 \(n-2\)~级反称阵.
    由假定,
    作若干次列的倍加, 与对应的行的倍加,
    我们可变 \(C({1,2}|{1,2})\) 为一个反称阵 \(H\),
    其中 \(H\) 形如式~\eqref{eq:C3201} (若 \(n-2\) 是偶数)
    或式~\eqref{eq:C3202} (若 \(n-2\) 是奇数).

    注意到, 既然当 \(2 < j\) 时,
    \([C]_{1,j} = [C]_{2,j} = 0\),
    \([C]_{j,1} = [C]_{j,2} = 0\),
    那么,
    无论如何对 \(C\) 的不是列~\(1\) 或列~\(2\) 的列作倍加,
    也无论如何对 \(C\) 的不是行~\(1\) 或行~\(2\) 的行作倍加,
    所得的阵的
    \((1, j)\)-元, \((2, j)\)-元, \((j, 1)\)-元, \((j, 2)\)-元,
    一定是零.
    所以, 作若干次列的倍加, 与对应的行的倍加后,
    我们可变 \(C\) 为一个 \(n\)~级反称阵 \(B\),
    使
    \begin{align*}
        [B]_{i,j}
        = \begin{cases}
              [H]_{i-2,j-2}, & \text{\(i > 2\), 且 \(j > 2\)}; \\
              [A]_{1,2},     & \text{\(i = 1\), 且 \(j = 2\)}; \\
              [A]_{2,1},     & \text{\(i = 2\), 且 \(j = 1\)}; \\
              0,             & \text{其他}.
          \end{cases}
    \end{align*}
    于是, \(B\) 是%
    形如式~\eqref{eq:C3201} (若 \(n\) 是偶数)
    或式~\eqref{eq:C3202} (若 \(n\) 是奇数)
    的反称阵.

    所以, \(P(n)\) 是正确的.
    则 \(Q(n)\) 是正确的.
    由数学归纳法原理, 待证命题成立.
\end{proof}

由倍加与阵的积的关系, 我们有:

\begin{theorem}
    设 \(A\) 是 \(n\)~级反称阵.
    则存在若干个形如 \(E(n; p, q; s)\)
    (\(s\) 是一个数;
    \(p\), \(q\) 是不超过 \(n\) 的正整数,
    \(p \neq q\))
    的阵
    \(E_1\), \(E_2\), \(\dots\), \(E_w\),
    使 \(V^{\mathrm{T}} AV\) 是%
    形如式~\eqref{eq:C3201} (若 \(n\) 是偶数)
    或式~\eqref{eq:C3202} (若 \(n\) 是奇数)
    的反称阵,
    其中 \(V = E_1 E_2 \dots E_w\).
\end{theorem}

\begin{proof}
    由上个定理, 存在若干个形如 \(E(n; p, q; s)\)
    (\(s\) 是一个数;
    \(p\), \(q\) 是不超过 \(n\) 的正整数,
    \(p \neq q\))
    的阵
    \(E_1\), \(E_2\), \(\dots\), \(E_w\),
    使
    \begin{align*}
        E_w^{\mathrm{T}}
        (E_{w-1}^{\mathrm{T}}
        \dots
        (E_2^{\mathrm{T}}
        (E_1^{\mathrm{T}}
        A
        E_1)
        E_2)
        \dots
        E_{w-1})
        E_w
    \end{align*}
    是%
    形如式~\eqref{eq:C3201} (若 \(n\) 是偶数)
    或式~\eqref{eq:C3202} (若 \(n\) 是奇数)
    的反称阵.
    由结合律, 上式相当于
    \begin{align*}
        (E_w^{\mathrm{T}}
        E_{w-1}^{\mathrm{T}}
        \dots
        E_2^{\mathrm{T}}
        E_1^{\mathrm{T}})
        A
        (E_1
        E_2
        \dots
        E_{w-1}
        E_w).
    \end{align*}
    记 \(V = E_1 E_2 \dots E_{w-1} E_w\).
    由转置的性质,
    \(V^{\mathrm{T}}
    = E_w^{\mathrm{T}}
    E_{w-1}^{\mathrm{T}}
    \dots
    E_2^{\mathrm{T}}
    E_1^{\mathrm{T}}\).
    故上式相当于 \(V^{\mathrm{T}} A V\).
\end{proof}

我们计算如式~\eqref{eq:C3201} 所示的反称阵的行列式.
记
\begin{align*}
    B_m =
    \begin{bmatrix}
        0      & b_1    & 0      & 0      & \cdots & 0      & 0      \\
        -b_1   & 0      & 0      & 0      & \cdots & 0      & 0      \\
        0      & 0      & 0      & b_2    & \cdots & 0      & 0      \\
        0      & 0      & -b_2   & 0      & \cdots & 0      & 0      \\
        \vdots & \vdots & \vdots & \vdots & \ddots & \vdots & \vdots \\
        0      & 0      & 0      & 0      & \cdots & 0      & b_m    \\
        0      & 0      & 0      & 0      & \cdots & -b_m   & 0      \\
    \end{bmatrix}.
\end{align*}
按列~\(2m\) 展开, 有
\begin{align*}
    \det {(B_m)}
    = {} &
    (-1)^{2m-1+2m} [B_m]_{2m-1,2m} \det {(B_m ({2m-1}|{2m}))}
    \\
    = {} &
    {-b_m} \det {
        \begin{bmatrix}
            B_{m-1} & 0    \\
            0       & -b_m \\
        \end{bmatrix}
    }.
\end{align*}
再按行~\(2m-1\) 展开 \(\det {(B_m ({2m-1}|{2m}))}\), 有
\begin{align*}
    \det {(B_m ({2m-1}|{2m}))}
    = {} & (-1)^{2m-1+2m-1} (-b_m)
    \det {(B_{m-1})}                \\
    = {} & {-b_m} \det {(B_{m-1})}.
\end{align*}
则 \(\det {(B_m)} = \det {(B_{m-1})}\, b_m^2\),
当 \(m > 1\).
不难算出 \(\det {(B_1)} = b_1^2\).
则
\begin{align*}
    \det {(B_m)} = b_1^2 b_2^2 \dots b_m^2
    = (b_1 b_2 \dots b_m)^2.
\end{align*}

还有一件事值得一提.
注意到, 倍加不改变行列式.
由此, 我们立得, 奇数级反称阵的行列式为 \(0\):
毕竟, 利用倍加, 我们可变一个奇数级反称阵为%
一个至少有一列的元全为零的阵.

\section{Pfaffian}

我们说, 行列式是方阵的一个重要的属性.
反称阵是方阵, 故行列式也是反称阵的一个重要的属性.
本节, 我们学习反称阵的另一个属性.
它也是重要的, 且与行列式有关.

\begin{definition}[pfaffian]
    设 \(A\) 是 \(n\)~级\emph{反称阵}.
    定义 \(A\)~的 \emph{pfaffian} 为
    \begin{align*}
        \operatorname{pf} {(A)} =
        \begin{dcases}
            0,
             & n = 1;    \\
            [A]_{1,2},
             & n = 2;    \\
            \sum_{j = 2}^{n}
            {(-1)^{j} [A]_{1,j}
            \operatorname{pf} {(A({1,j}|{1,j}))}},
             & n \geq 3.
        \end{dcases}
    \end{align*}
\end{definition}

% 据说, 这是英国数学家 Arthur Cayley
% 用德国数学家 Johann Friedrich Pfaff 的姓%
% 命名的概念.

注意, 我们只对反称阵定义 pfaffian.

我们计算不高于 \(4\)~级的反称阵的 pfaffian.

\begin{example}
    设 \(A\) 是 \(1\)~级反称阵.
    则 \(\operatorname{pf} {(A)} = 0\).

    回想起, \(\det {(A)}\) 也是零.
\end{example}

\begin{example}
    设 \(A\) 是 \(2\)~级反称阵.
    则 \(\operatorname{pf} {(A)} = [A]_{1,2}\).

    回想起, \(\det {(A)} = [A]_{1,2}^2\);
    于是, \(\det {(A)} = (\operatorname{pf} {(A)})^2\).
\end{example}

\begin{example}
    设 \(A\) 是 \(3\)~级反称阵.
    则
    \begin{align*}
        \operatorname{pf} {(A)}
        = {} &
        (-1)^2 [A]_{1,2} \operatorname{pf} {(A({1,2}|{1,2}))}
        +
        (-1)^3 [A]_{1,3} \operatorname{pf} {(A({1,3}|{1,3}))}
        \\
        = {} &
        [A]_{1,2} 0 - [A]_{1,3} 0
        \\
        = {} &
        0.
    \end{align*}

    回想起, \(\det {(A)}\) 也是零.
\end{example}

\begin{example}
    设 \(A\) 是 \(4\)~级反称阵.
    则
    \begin{align*}
        \operatorname{pf} {(A)}
        = {} &
        \hphantom{{} + {}}
        (-1)^2 [A]_{1,2} \operatorname{pf} {(A({1,2}|{1,2}))}
        +
        (-1)^3 [A]_{1,3} \operatorname{pf} {(A({1,3}|{1,3}))}
        \\
             &
        +
        (-1)^4 [A]_{1,4} \operatorname{pf} {(A({1,4}|{1,4}))}
        \\
        = {} &
        [A]_{1,2} [A]_{3,4}
        - [A]_{1,3} [A]_{2,4}
        + [A]_{1,4} [A]_{2,3}.
    \end{align*}

    回想起, \(\det {(A)} =
    ([A]_{1,2} [A]_{3,4}
    - [A]_{1,3} [A]_{2,4}
    + [A]_{1,4} [A]_{2,3})^2\);
    于是, \(\det {(A)} = (\operatorname{pf} {(A)})^2\).
\end{example}

看来, 对不高于 \(4\)~级的反称阵 \(A\),
我们有 \(\det {(A)} = (\operatorname{pf} {(A)})^2\).
我们会说明, 这对任何级的反称阵 \(A\) 都是对的.
为此, 我们会证明 pfaffian 的一些性质,
再利用这些性质解决此事.

最后, 我们再看一些较特别的阵的 pfaffian 吧.

\begin{example}
    作 \(2m\)~级反称阵
    \begin{align*}
        B =
        \begin{bmatrix}
            0      & b_1    & 0      & 0      & \cdots & 0      & 0      \\
            -b_1   & 0      & 0      & 0      & \cdots & 0      & 0      \\
            0      & 0      & 0      & b_2    & \cdots & 0      & 0      \\
            0      & 0      & -b_2   & 0      & \cdots & 0      & 0      \\
            \vdots & \vdots & \vdots & \vdots & \ddots & \vdots & \vdots \\
            0      & 0      & 0      & 0      & \cdots & 0      & b_m    \\
            0      & 0      & 0      & 0      & \cdots & -b_m   & 0      \\
        \end{bmatrix}.
    \end{align*}
    我们计算 \(\operatorname{pf} (B)\).
    由 pfaffian 的定义,
    \begin{align*}
        \operatorname{pf} {(B)}
        = (-1)^2 [B]_{1,2} \operatorname{pf} {(B({1,2}|{1,2}))}
        = b_1 \operatorname{pf} {(B({1,2}|{1,2}))}
    \end{align*}
    (注意到, 对任何 \(j > 2\), 有 \([B]_{1,j} = 0\)).
    不难看出,
    \begin{align*}
        B({1,2}|{1,2}) =
        \begin{bmatrix}
            0      & b_2    & \cdots & 0      & 0      \\
            -b_2   & 0      & \cdots & 0      & 0      \\
            \vdots & \vdots & \ddots & \vdots & \vdots \\
            0      & 0      & \cdots & 0      & b_m    \\
            0      & 0      & \cdots & -b_m   & 0      \\
        \end{bmatrix}.
    \end{align*}
    则, 类似地,
    \begin{align*}
        \operatorname{pf} {(B({1,2}|{1,2}))}
        = b_2 \operatorname{pf} {(B({1,2,3,4}|{1,2,3,4}))}.
    \end{align*}
    故
    \begin{align*}
        \operatorname{pf} {(B)}
        = b_1 b_2 \operatorname{pf} {(B({1,2,3,4}|{1,2,3,4}))}.
    \end{align*}
    \(\dots \dots\)
    最后, 我们算出
    \begin{align*}
        \operatorname{pf} {(B)}
        = b_1 b_2 \dots b_{m-1}
        \operatorname{pf} {(
        B({1,2,\dots,2m-2}|{1,2,\dots,2m-2})
        )}
        = b_1 b_2 \dots b_{m-1} b_m.
    \end{align*}
    回想起, \(\det {(B)} = (b_1 b_2 \dots b_m)^2\);
    于是, \(\det {(B)} = (\operatorname{pf} {(B)})^2\).
\end{example}

\begin{example}
    设 \(A\) 是 \(n\)~级反称阵.
    设 \(D\) 是 \(t\)~级反称阵.
    作 \(n + t\)~级阵
    \begin{align*}
        M =
        \begin{bmatrix}
            A & 0 \\
            0 & D \\
        \end{bmatrix}.
    \end{align*}
    不难验证, \(M\) 也是一个反称阵.
    我们用数学归纳法证明, \(\operatorname{pf} {(M)}
    = \operatorname{pf} {(A)} \operatorname{pf} {(D)}\).

    作命题 \(P(n)\):
    对任何 \(n\)~级反称阵 \(A\),
    \begin{align*}
        \operatorname{pf} {
            \begin{bmatrix}
                A & 0 \\
                0 & D \\
            \end{bmatrix}
        } = \operatorname{pf} {(A)} \operatorname{pf} {(D)}.
    \end{align*}
    再作命题 \(Q(n)\):
    \(P(n-1)\) 与 \(P(n)\) 是对的.
    我们用数学归纳法证明:
    对任何高于 \(1\) 的整数 \(n\), \(Q(n)\) 是对的.

    首先, \(P(1)\) 是对的, 因为 \(n = 1\)~时,
    \begin{align*}
        \begin{bmatrix}
            A & 0 \\
            0 & D \\
        \end{bmatrix}
    \end{align*}
    的行~\(1\) 的元全是 \(0\),
    故由 pfaffian 的定义, 此阵的 pfaffian 是 \(0\).

    然后, \(P(2)\) 是对的.
    记 \(J =
    \begin{bmatrix}
        A & 0 \\
        0 & D \\
    \end{bmatrix}
    \),
    其中 \(A\) 是 \(2\)~级反称阵.
    由 pfaffian 的定义,
    \begin{align*}
        \operatorname{pf} {(J)}
        = (-1)^{2} [J]_{1,2} \operatorname{pf} {(J({1,2}|{1,2}))}
        = [A]_{1,2} \operatorname{pf} {(D)}
        = \operatorname{pf} {(A)} \operatorname{pf} {(D)}.
    \end{align*}

    由此可见, \(Q(2)\) 是对的.

    我们设 \(Q(n-1)\) 是对的 (\(n \geq 3\)).
    则 \(P(n-2)\) 与 \(P(n-1)\) 是对的.
    我们要由此证明 \(Q(n)\) 是对的,
    即 \(P(n-1)\) 与 \(P(n)\) 是对的.
    \(P(n-1)\) 当然是对的, 由假定.
    所以, 我们由假定
    ``\(P(n-2)\) 与 \(P(n-1)\) 是对的''
    证 \(P(n)\) 是对的,
    即得 \(Q(n)\) 是对的.

    任取一个 \(n\)~级反称阵 \(A\).
    记 \(M =
    \begin{bmatrix}
        A & 0 \\
        0 & D \\
    \end{bmatrix}
    \).
    则
    \begin{align*}
             &
        \operatorname{pf} {(M)}
        \\
        = {} &
        \sum_{2 \leq j \leq n+t}
        {(-1)^{j} [M]_{1,j} \operatorname{pf} {(M({1,j}|{1,j}))}}
        \\
        = {} &
        \sum_{2 \leq j \leq n}
        {(-1)^{j} [M]_{1,j} \operatorname{pf} {(M({1,j}|{1,j}))}}
        +
        \sum_{n+1 \leq j \leq n+t}
        {(-1)^{j} [M]_{1,j} \operatorname{pf} {(M({1,j}|{1,j}))}}
        \\
        = {} &
        \sum_{2 \leq j \leq n}
        {(-1)^{j} [A]_{1,j} \operatorname{pf} {(M({1,j}|{1,j}))}}
        +
        \sum_{n+1 \leq j \leq n+t}
        {(-1)^{j} 0 \operatorname{pf} {(M({1,j}|{1,j}))}}
        \\
        = {} &
        \sum_{2 \leq j \leq n}
        {(-1)^{j} [A]_{1,j} \operatorname{pf} {(M({1,j}|{1,j}))}}.
    \end{align*}
    注意到, \(2 \leq j \leq n\) 时,
    \begin{align*}
        M({1,j}|{1,j})
        = \begin{bmatrix}
              A({1,j}|{1,j}) & 0 \\
              0              & D \\
          \end{bmatrix}.
    \end{align*}
    由假定, \(\operatorname{pf} {(M({1,j}|{1,j}))}
    = \operatorname{pf} {(A({1,j}|{1,j}))}
    \operatorname{pf} {(D)}\).
    从而
    \begin{align*}
        \operatorname{pf} {(M)}
        = {} &
        \sum_{2 \leq j \leq n}
        {(-1)^{j} [A]_{1,j} \operatorname{pf} {(M({1,j}|{1,j}))}}
        \\
        = {} &
        \sum_{2 \leq j \leq n}
        {(-1)^{j} [A]_{1,j}
        \operatorname{pf} {(A({1,j}|{1,j}))} \operatorname{pf} {(D)}
        }
        \\
        = {} &
        \left(
        \sum_{2 \leq j \leq n}
        {(-1)^{j} [A]_{1,j}
        \operatorname{pf} {(A({1,j}|{1,j}))}}
        \right) \operatorname{pf} {(D)}
        \\
        = {} &
        \operatorname{pf} {(A)} \operatorname{pf} {(D)}.
    \end{align*}

    所以, \(P(n)\) 是正确的.
    则 \(Q(n)\) 是正确的.
    由数学归纳法原理, 待证命题成立.

    回想起, 当 \(A\) 是 \(n\)~级阵,
    \(D\) 是 \(t\)~级阵,
    \(C\) 是 \(n \times t\)~阵时,
    \begin{align*}
        \det {
            \begin{bmatrix}
                A & C \\
                0 & D \\
            \end{bmatrix}
        }
        = \det {(A)} \det {(D)}.
    \end{align*}
    那么, 当 \(A\), \(D\) 是反称阵,
    且 \(C = 0\) 时,
    \(\det {(M)}\)
    当然也是 \(\det {(A)} \det {(D)}\).
    不过, 用现有的知识,
    我们还不知道这个 \(M\) 的行列式与 pfaffian 的关系.
\end{example}

在看最后一个例前, 我们先计算一类阵的行列式.
设 \(A\), \(D\) 分别是 \(n\)~即阵与 \(t\)~级阵.
设 \(C\) 是 \(n \times t\)~阵.
作 \(n + t\)~级阵
\begin{align*}
    P = \begin{bmatrix}
            C & A \\
            D & 0 \\
        \end{bmatrix}.
\end{align*}
我们计算 \(\det {(P)}\).
为此, 我们可用反称性.
具体地, 我们交换 \(P\) 的列~\(t+1\) 与列~\(t\),
得阵~\(P_t\).
则 \(\det {(P_t)} = (-1) \det {(P)}\).
再交换 \(P_t\) 的列~\(t\) 与列~\(t-1\),
得阵~\(P_{t-1}\).
则 \(\det {(P_{t-1})} = (-1) \det {(P_t)}
= (-1)^2 \det {(P)}\).
\(\dots \dots\)
再交换 \(P_2\) 的列~\(2\) 与列~\(1\),
得阵~\(P_1\).
则 \(\det {(P_1)} = (-1) \det {(P_2)}
= (-1)^t \det {(P)}\).
注意到, 我们作了 \(t\) 次相邻的列的交换,
使 \(P\) 的列~\(t+1\) 到列~\(1\) 的位置,
且其他的列的相对位置不变.
类似地, 我们再分别对 \(P_1\)~的列~\(t+2\), \(\dots\), \(t+n\)
也相邻地向左移 \(t\)~次,
即可使 \(P\) 的列~\(t+1\), \(\dots\), \(t+n\)
分别到列~\(1\), \(\dots\), \(n\) 的位置,
变 \(P\) 为
\begin{align*}
    M = \begin{bmatrix}
            A & C \\
            0 & D \\
        \end{bmatrix}.
\end{align*}
我们一共作了 \(t + (n-1)t = nt\)~次列的交换.
故
\begin{align*}
    \det {
        \begin{bmatrix}
            C & A \\
            D & 0 \\
        \end{bmatrix}
    }
    = (-1)^{nt} \det {
        \begin{bmatrix}
            A & C \\
            0 & D \\
        \end{bmatrix}
    }
    = (-1)^{nt} \det {(A)} \det {(D)}.
\end{align*}
特别地, 若我们取 \(D = -A^{\mathrm{T}}\),
且 \(C = 0\),
则
\begin{align*}
    \det {
        \begin{bmatrix}
            0               & A \\
            -A^{\mathrm{T}} & 0 \\
        \end{bmatrix}
    }
    = {} &
    (-1)^{n^2} \det {(-A^{\mathrm{T}})} \det {(A)}
    \\
    = {} &
    (-1)^{n^2} (-1)^n \det {(A^{\mathrm{T}})} \det {(A)}
    \\
    = {} &
    (-1)^{n(n+1)} (\det {(A)})^2
    \\
    = {} &
    (\det {(A)})^2.
\end{align*}
(注意到, \(n(n+1)\) 一定是一个偶数.)

\begin{example}
    设 \(A\) 是一个 \(m\)~级阵.
    为方便, 我们记
    \begin{align*}
        S(A) = \begin{bmatrix}
                   0               & A \\
                   -A^{\mathrm{T}} & 0 \\
               \end{bmatrix}.
    \end{align*}
    不难看出, \(S(A)\) 是一个 \(2m\)~级反称阵.
    我们用数学归纳法证明,
    \(\operatorname{pf} {(A)} = (-1)^{m(m-1)/2} \det {(A)}\).
    这也说明, pfaffian 跟行列式有一定的联系.

    作命题 \(P(m)\):
    对任何 \(m\)~级阵 \(A\),
    \(\operatorname{pf} {(A)} = (-1)^{m(m-1)/2} \det {(A)}\).
    我们用数学归纳法证明:
    对任何正整数 \(m\), \(P(m)\) 是对的.

    \(P(1)\) 是对的.
    设 \(A = [a]\).
    则 \(S(A) = \begin{bmatrix}
        0  & a \\
        -a & 0 \\
    \end{bmatrix}\).
    由定义, \(\det {(A)} = a\),
    且
    \(\operatorname{pf} {(S(A))} = a = (-1)^{1(1-1)/2} \det {(A)}\).

    我们设 \(P(m-1)\) 是对的.
    我们要由此证明 \(P(m)\) 是对的.

    任取 \(m\)~级阵 \(A\).
    则
    \begin{align*}
        \operatorname{pf} {(S(A))}
        = {} &
        \sum_{j=2}^{2m}
        {(-1)^j [S(A)]_{1,j} \operatorname{pf} {((S(A))({1,j}|{1,j}))}}
        \\
        = {} &
        \hphantom{{} + {}}
        \sum_{2 \leq j \leq m}
        {(-1)^j [S(A)]_{1,j} \operatorname{pf} {((S(A))({1,j}|{1,j}))}}
        \\
             & +
        \sum_{m+1 \leq j \leq 2m}
        {(-1)^j [S(A)]_{1,j} \operatorname{pf} {((S(A))({1,j}|{1,j}))}}
        \\
        = {} &
        \hphantom{{} + {}}
        \sum_{2 \leq j \leq m}
        {(-1)^j 0 \operatorname{pf} {((S(A))({1,j}|{1,j}))}}
        \\
             & +
        \sum_{1 \leq k \leq m}
        {(-1)^{m+k} [S(A)]_{1,m+k}
        \operatorname{pf} {((S(A))({1,m+k}|{1,m+k}))}}
        \\
        = {} &
        \sum_{1 \leq k \leq m}
        {(-1)^{m+k} [A]_{1,k}
            \operatorname{pf} {(S(A(1|k)))}}
        \\
        = {} &
        \sum_{1 \leq k \leq m}
        {(-1)^{m-1} (-1)^{1+k} [A]_{1,k}
            \operatorname{pf} {(S(A(1|k)))}}
        \\
        = {} &
        \sum_{1 \leq k \leq m}
        {(-1)^{1+k} [A]_{1,k}
            (-1)^{m-1} \operatorname{pf} {(S(A(1|k)))}}.
    \end{align*}
    注意到, \(A(1|k)\) 是 \(m-1\)~级阵.
    由假定,
    \begin{align*}
        \operatorname{pf} {(S(A(1|k)))}
        = (-1)^{(m-1)(m-2)/2} \det {(A(1|k))}.
    \end{align*}
    则
    \begin{align*}
        \operatorname{pf} {(S(A))}
        = {} &
        \sum_{1 \leq k \leq m}
        {(-1)^{1+k} [A]_{1,k}
            (-1)^{m-1} \operatorname{pf} {(S(A(1|k)))}}
        \\
        = {} &
        \sum_{1 \leq k \leq m}
        {(-1)^{1+k} [A]_{1,k}
            (-1)^{m-1} (-1)^{(m-1)(m-2)/2} \det {(A(1|k))}}
        \\
        = {} &
        \sum_{1 \leq k \leq m}
        {(-1)^{1+k} [A]_{1,k}
            (-1)^{m(m-1)/2} \det {(A(1|k))}}
        \\
        = {} &
        (-1)^{m(m-1)/2}
        \sum_{1 \leq k \leq m}
        {(-1)^{1+k} [A]_{1,k}
            \det {(A(1|k))}}
        \\
        = {} &
        (-1)^{m(m-1)/2} \det {(A)}.
    \end{align*}

    所以, \(P(m)\) 是正确的.
    由数学归纳法原理, 待证命题成立.

    最后, 不难看出,
    \(\det {(S(A))} = (\det {(A)})^2
    = (\operatorname{pf} {(S(A))})^2\).
\end{example}

我们作一个小结.

\begin{theorem}
    (1)
    设 \(A\), \(D\) 是 \(n\)~级与 \(t\)~级反称阵.
    则
    \begin{align*}
        \operatorname{pf} {
            \begin{bmatrix}
                A & 0 \\
                0 & D
            \end{bmatrix}
        }
        = \operatorname{pf} {(A)} \operatorname{pf} {(D)}.
    \end{align*}

    (2)
    设 \(A\) 是 \(m\)~级阵.
    则
    \begin{align*}
        \operatorname{pf} {
            \begin{bmatrix}
                0               & A \\
                -A^{\mathrm{T}} & 0 \\
            \end{bmatrix}
        } = (-1)^{m(m-1)/2} \det {(A)}.
    \end{align*}
    特别地, 取 \(A = I_m\), 即知
    \(K_m = \begin{bmatrix}
        0    & I_m \\
        -I_m & 0   \\
    \end{bmatrix}\)
    的 pfaffian 是 \((-1)^{m(m-1)/2}\).

    (3)
    \begin{align*}
        \operatorname{pf} {
            \begin{bmatrix}
                0      & b_1    & 0      & 0      & \cdots & 0      & 0      \\
                -b_1   & 0      & 0      & 0      & \cdots & 0      & 0      \\
                0      & 0      & 0      & b_2    & \cdots & 0      & 0      \\
                0      & 0      & -b_2   & 0      & \cdots & 0      & 0      \\
                \vdots & \vdots & \vdots & \vdots & \ddots & \vdots & \vdots \\
                0      & 0      & 0      & 0      & \cdots & 0      & b_m    \\
                0      & 0      & 0      & 0      & \cdots & -b_m   & 0      \\
            \end{bmatrix}
        } = b_1 b_2 \dots b_m.
    \end{align*}
\end{theorem}

\section{Pfaffian 的性质}

本节, 我们讨论 pfaffian 的一些性质.

回想起, \(1\)~级反称阵与 \(3\)~级反称阵的 pfaffian 为 \(0\).
一般地,

\begin{theorem}
    奇数级反称阵的 pfaffian 为 \(0\).
\end{theorem}

\begin{proof}
    作命题 \(P(k)\):
    对任何 \(2k-1\)~级反称阵 \(A\),
    必 \(\operatorname{pf} {(A)} = 0\).
    我们用数学归纳法证明:
    对任何正整数 \(k\), \(P(k)\) 是对的.

    \(P(1)\) 不证自明.

    我们设 \(P(k-1)\) 是对的.
    我们要由此证明 \(P(k)\) 是对的.

    任取 \(2k-1\)~级反称阵 \(A\).
    则
    \begin{align*}
        \operatorname{pf} {(A)}
        = {} &
        \sum_{j = 2}^{2k-1}
        {(-1)^{j} [A]_{1,j}
        \operatorname{pf} {(A({1,j}|{1,j}))}}
        \\
        = {} &
        \sum_{j = 2}^{2k-1}
        {(-1)^{j} [A]_{1,j}\, 0}
        \\
        = {} &
        0.
    \end{align*}

    所以, \(P(k)\) 是正确的.
    由数学归纳法原理, 待证命题成立.
\end{proof}

在讨论较复杂的性质前, 我们先看一个简单的.

\begin{theorem}
    设 \(m\) 是整数.
    设 \(A\) 是 \(2m\)~级反称阵.
    设 \(x\) 是数.
    则 \(\operatorname{pf} {(xA)}
    = x^m \operatorname{pf} {(A)}\).
    特别地, \(A^{\mathrm{T}} = (-1)A\) 的 pfaffian
    是 \((-1)^m \operatorname{pf} {(A)}\).
\end{theorem}

\begin{proof}
    作命题 \(P(m)\):
    对任何 \(2m\)~级反称阵 \(A\),
    必 \(\operatorname{pf} {(xA)}
    = x^m \operatorname{pf} {(A)}\).
    我们用数学归纳法证明:
    对任何正整数 \(m\), \(P(m)\) 是对的.

    \(P(1)\) 是对的:
    \begin{align*}
        \operatorname{pf} {
            \begin{bmatrix}
                0   & xa \\
                -xa & 0  \\
            \end{bmatrix}
        }
        = xa
        = x^1 \operatorname{pf} {
            \begin{bmatrix}
                0  & a \\
                -a & 0 \\
            \end{bmatrix}
        }.
    \end{align*}

    我们设 \(P(m-1)\) 是对的.
    我们要由此证明 \(P(m)\) 是对的.

    任取 \(2m\)~级反称阵 \(A\).
    则
    \begin{align*}
        \operatorname{pf} {(xA)}
        = {} &
        \sum_{j = 2}^{2m}
        {(-1)^{j} [xA]_{1,j}
        \operatorname{pf} {((xA)({1,j}|{1,j}))}}
        \\
        = {} &
        \sum_{j = 2}^{2m}
        {(-1)^{j} x\, [A]_{1,j}
        \operatorname{pf} {(x\,A({1,j}|{1,j}))}}
        \\
        = {} &
        \sum_{j = 2}^{2m}
        {(-1)^{j} x\, [A]_{1,j}
        x^{m-1} \operatorname{pf} {(A({1,j}|{1,j}))}}
        \\
        = {} &
        x\,x^{m-1}
        \sum_{j = 2}^{2m}
        {(-1)^{j} [A]_{1,j}
        \operatorname{pf} {(A({1,j}|{1,j}))}}
        \\
        = {} &
        x^m \operatorname{pf} {(A)}.
    \end{align*}

    所以, \(P(m)\) 是正确的.
    由数学归纳法原理, 待证命题成立.
\end{proof}

\begin{theorem}
    设 \(A\), \(B\), \(C\) 是三个 \(n\)~级反称阵.
    设 \(q\) 是不超过 \(n\) 的正整数.
    设 \(s\), \(t\) 是数.
    设 \([A]_{i,j} = [B]_{i,j} = [C]_{i,j}\),
    对任何不等于 \(q\), 且不超过 \(n\) 的正整数 \(i\), \(j\);
    设 \([C]_{i,j} = s[A]_{i,j} + t[B]_{i,j}\),
    若 \(i = q\) 或 \(j = q\).
    则
    \begin{align*}
        \operatorname{pf} {(C)}
        = s \operatorname{pf} {(A)}
        + t \operatorname{pf} {(B)}.
    \end{align*}
\end{theorem}

此性质或许跟行列式的多线性有些像, 但不完全一样.

\begin{proof}
    作命题 \(P(n)\):
    \begin{quotation}
        对任何数 \(s\), \(t\),
        对任何不超过 \(n\) 的正整数 \(q\),
        对任何适合如下条件的三个 \(n\)~级反称阵 \(A\), \(B\), \(C\),
        必有
        \(
        \operatorname{pf} {(C)}
        = s \operatorname{pf} {(A)}
        + t \operatorname{pf} {(B)}
        \):

        (1)
        \([A]_{i,j} = [B]_{i,j} = [C]_{i,j}\),
        对任何不等于 \(q\), 且不超过 \(n\) 的正整数 \(i\), \(j\);

        (2)
        \([C]_{i,j} = s[A]_{i,j} + t[B]_{i,j}\),
        若 \(i = q\) 或 \(j = q\).
    \end{quotation}
    再作命题 \(Q(n)\):
    \(P(n-1)\) 与 \(P(n)\) 是对的.
    我们用数学归纳法证明:
    对任何高于 \(1\) 的整数 \(n\), \(Q(n)\) 是对的.

    \(P(1)\) 不证自明.

    \(P(2)\) 是简单的.
    无论 \(q = 1\) 或 \(q = 2\),
    我们都有 \([C]_{1,2} = s[A]_{1,2} + t[B]_{1,2}\).
    则 \(\operatorname{pf} {(C)}
    = s \operatorname{pf} {(A)}
    + t \operatorname{pf} {(B)}\).

    综上, \(Q(2)\) 是对的.

    我们设 \(Q(n-1)\) 是对的 (\(n \geq 3\)).
    则 \(P(n-2)\) 与 \(P(n-1)\) 是对的.
    我们要由此证明 \(Q(n)\) 是对的,
    即 \(P(n-1)\) 与 \(P(n)\) 是对的.
    \(P(n-1)\) 当然是对的, 由假定.
    所以, 我们由假定
    ``\(P(n-2)\) 与 \(P(n-1)\) 是对的''
    证 \(P(n)\) 是对的,
    即得 \(Q(n)\) 是对的.

    设 \(A\), \(B\), \(C\) 是三个 \(n\)~级反称阵.
    设 \(q\) 是不超过 \(n\) 的正整数.
    设 \(s\), \(t\) 是数.
    设 \([A]_{i,j} = [B]_{i,j} = [C]_{i,j}\),
    对任何不等于 \(q\), 且不超过 \(n\) 的正整数 \(i\), \(j\);
    设 \([C]_{i,j} = s[A]_{i,j} + t[B]_{i,j}\),
    若 \(i = q\) 或 \(j = q\).

    若 \(q = 1\), 则
    \([C]_{1,j} = s[A]_{1,j} + t[B]_{1,j}\),
    且 \(C({1,j}|{1,j}) = A({1,j}|{1,j}) = B({1,j}|{1,j})\),
    对 \(j > 1\).
    则
    \begin{align*}
             &
        \operatorname{pf} {(C)}
        \\
        = {} &
        \sum_{j=2}^{n}
        {(-1)^j [C]_{1,j}
        \operatorname{pf} {(C({1,j}|{1,j}))}}
        \\
        = {} &
        \sum_{j=2}^{n}
        {(-1)^j (s[A]_{1,j} + t[B]_{1,j})
        \operatorname{pf} {(C({1,j}|{1,j}))}}
        \\
        = {} &
        s \sum_{j=2}^{n}
        {(-1)^j [A]_{1,j}
        \operatorname{pf} {(C({1,j}|{1,j}))}}
        +
        t \sum_{j=2}^{n}
        {(-1)^j [B]_{1,j}
        \operatorname{pf} {(C({1,j}|{1,j}))}}
        \\
        = {} &
        s \sum_{j=2}^{n}
        {(-1)^j [A]_{1,j}
        \operatorname{pf} {(A({1,j}|{1,j}))}}
        +
        t \sum_{j=2}^{n}
        {(-1)^j [B]_{1,j}
        \operatorname{pf} {(B({1,j}|{1,j}))}}
        \\
        = {} &
        s \operatorname{pf} {(A)}
        + t \operatorname{pf} {(B)}.
    \end{align*}

    设 \(q \neq 1\).
    那么, 当 \(j \neq q\) 时,
    \([A]_{1,j} = [B]_{1,j} = [C]_{1,j}\),
    且 \(n-2\)~级反称阵
    \(A({1,j}|{1,j})\),
    \(B({1,j}|{1,j})\),
    \(C({1,j}|{1,j})\)
    适合:

    (1\ensuremath{'})
    \([A({1,j}|{1,j})]_{u,v}
    = [B({1,j}|{1,j})]_{u,v}
    = [C({1,j}|{1,j})]_{u,v}\),
    对任何不等于 \(q' = q - \rho(q, 1) - \rho(q, j)
    = q - 1 - \rho(q, j)\),
    且不超过 \(n-2\) 的正整数 \(u\), \(v\);

    (2\ensuremath{'})
    \([C({1,j}|{1,j})]_{u,v}
    = s [A({1,j}|{1,j})]_{u,v}
    + t [B({1,j}|{1,j})]_{u,v}\),
    若 \(u = q'\) 或 \(v = q'\).

    于是, 由假定, 当 \(j \neq q\) 时,
    \begin{align*}
        \operatorname{pf} {(C({1,j}|{1,j}))}
        = s \operatorname{pf} {(A({1,j}|{1,j}))}
        + t \operatorname{pf} {(B({1,j}|{1,j}))}.
    \end{align*}

    另一方面, 当 \(j = q\) 时,
    \([C]_{1,j} = s[A]_{1,j} + t[B]_{1,j}\),
    且 \(C({1,j}|{1,j}) = A({1,j}|{1,j}) = B({1,j}|{1,j})\).

    综上, 我们有
    \begin{align*}
             &
        \operatorname{pf} {(C)}
        \\
        = {} &
        \sum_{j=2}^{n}
        {(-1)^j [C]_{1,j}
        \operatorname{pf} {(C({1,j}|{1,j}))}}
        \\
        = {} &
        (-1)^q [C]_{1,q}
        \operatorname{pf} {(C({1,q}|{1,q}))}
        +
        \sum_{\substack{2 \leq j \leq n     \\j \neq q}}
        {(-1)^j [C]_{1,j}
        \operatorname{pf} {(C({1,j}|{1,j}))}}
        \\
        = {} &
        \hphantom{{} + {}}
        (-1)^q (s[A]_{1,j} + t[B]_{1,j})
        \operatorname{pf} {(C({1,q}|{1,q}))}
        \\
             &
        +
        \sum_{\substack{2 \leq j \leq n     \\j \neq q}}
        {(-1)^j [C]_{1,j}
        (s \operatorname{pf} {(A({1,j}|{1,j}))}
        + t \operatorname{pf} {(B({1,j}|{1,j}))})
        }
        \\
        = {} &
        \hphantom{{} + {}}
        s (-1)^q [A]_{1,j}
        \operatorname{pf} {(C({1,q}|{1,q}))}
        + t (-1)^q [B]_{1,j}
        \operatorname{pf} {(C({1,q}|{1,q}))}
        \\
             &
        +
        s \sum_{\substack{2 \leq j \leq n   \\j \neq q}}
        {(-1)^j [C]_{1,j}
        \operatorname{pf} {(A({1,j}|{1,j}))}
        }
        + t \sum_{\substack{2 \leq j \leq n \\j \neq q}}
        {(-1)^j [C]_{1,j}
        \operatorname{pf} {(B({1,j}|{1,j}))}
        }
        \\
        = {} &
        \hphantom{{} + {}}
        s (-1)^q [A]_{1,j}
        \operatorname{pf} {(A({1,q}|{1,q}))}
        + t (-1)^q [B]_{1,j}
        \operatorname{pf} {(B({1,q}|{1,q}))}
        \\
             &
        +
        s \sum_{\substack{2 \leq j \leq n   \\j \neq q}}
        {(-1)^j [A]_{1,j}
        \operatorname{pf} {(A({1,j}|{1,j}))}
        }
        + t \sum_{\substack{2 \leq j \leq n \\j \neq q}}
        {(-1)^j [B]_{1,j}
        \operatorname{pf} {(B({1,j}|{1,j}))}
        }
        \\
        = {} &
        s \sum_{j=2}^{n}
        {(-1)^j [A]_{1,j}
        \operatorname{pf} {(A({1,j}|{1,j}))}
        }
        +
        t \sum_{j=2}^{n}
        {(-1)^j [B]_{1,j}
        \operatorname{pf} {(B({1,j}|{1,j}))}
        }
        \\
        = {} &
        s \operatorname{pf} {(A)}
        + t \operatorname{pf} {(B)}.
    \end{align*}

    所以, \(P(n)\) 是正确的.
    则 \(Q(n)\) 是正确的.
    由数学归纳法原理, 待证命题成立.
\end{proof}

特别地, 若我们取 \(A = B\), 且 \(t = 0\),
我们有

\begin{restatable}{theorem}{TheoremPfaffianMulitply}
    设 \(A\), \(C\) 是二个 \(n\)~级反称阵.
    设 \(q\) 是不超过 \(n\) 的正整数.
    设 \(s\) 是数.
    设 \([A]_{i,j} = [C]_{i,j}\),
    对任何不等于 \(q\), 且不超过 \(n\) 的正整数 \(i\), \(j\);
    设 \([C]_{i,j} = s[A]_{i,j}\),
    若 \(i = q\) 或 \(j = q\).
    则
    \(\operatorname{pf} {(C)}
    = s \operatorname{pf} {(A)}\).
\end{restatable}

\begin{theorem}
    设 \(A\) 是 \(n\)~级反称阵.
    设 \(p\), \(q\) 是不超过 \(n\) 的正整数, 且 \(p \neq q\).
    设交换 \(A\) 的列~\(p\), \(q\), 不改变其他的列, 得阵~\(B\).
    设交换 \(B\) 的行~\(p\), \(q\), 不改变其他的行, 得阵~\(C\).
    则 \(C\) 是反称阵, 且
    \(
    \operatorname{pf} {(C)} = -\operatorname{pf} {(A)}
    \).
\end{theorem}

\begin{proof}
    我们先说明 \(C\) 是反称阵.
    不难写出
    \begin{align*}
        [B]_{i,j} =
        \begin{cases}
            [A]_{i,q}, & j = p;     \\
            [A]_{i,p}, & j = q;     \\
            [A]_{i,j}, & \text{其他}.
        \end{cases}
    \end{align*}
    也不难写出
    \begin{align*}
        [C]_{i,j} =
        \begin{cases}
            [B]_{q,j}, & i = p;     \\
            [B]_{p,j}, & i = q;     \\
            [B]_{i,j}, & \text{其他}.
        \end{cases}
    \end{align*}
    当 \(i \neq p\), \(i \neq q\),
    \(j \neq p\), \(j \neq q\) 时,
    \begin{align*}
        [C]_{i,j} = [B]_{i,j} = [A]_{i,j};
    \end{align*}
    当 \(i = p\), \(j \neq p\), \(j \neq q\) 时,
    \begin{align*}
        [C]_{i,j} = [B]_{q,j} = [A]_{q,j};
    \end{align*}
    当 \(i = q\), \(j \neq p\), \(j \neq q\) 时,
    \begin{align*}
        [C]_{i,j} = [B]_{p,j} = [A]_{p,j};
    \end{align*}
    当 \(j = p\), \(i \neq p\), \(i \neq q\) 时,
    \begin{align*}
        [C]_{i,j} = [B]_{i,p} = [A]_{i,q};
    \end{align*}
    当 \(j = q\), \(i \neq p\), \(i \neq q\) 时,
    \begin{align*}
        [C]_{i,j} = [B]_{i,q} = [A]_{i,p};
    \end{align*}
    当 \(i = p\), \(j = p\) 时,
    \begin{align*}
        [C]_{i,j} = [B]_{q,p} = [A]_{q,q};
    \end{align*}
    当 \(i = p\), \(j = q\) 时,
    \begin{align*}
        [C]_{i,j} = [B]_{q,q} = [A]_{q,p};
    \end{align*}
    当 \(i = q\), \(j = p\) 时,
    \begin{align*}
        [C]_{i,j} = [B]_{p,p} = [A]_{p,q};
    \end{align*}
    当 \(i = q\), \(j = q\) 时,
    \begin{align*}
        [C]_{i,j} = [B]_{p,q} = [A]_{p,p}.
    \end{align*}
    由此, 不难验证, \(C\) 是反称阵.

    我们再说明
    \(
    \operatorname{pf} {(C)} = -\operatorname{pf} {(A)}
    \).
    以下, 我们无妨设 \(p < q\).

    作命题 \(P(n)\):
    对任何 \(n\)~级反称阵 \(A\),
    对任何 \(1 \leq p < q \leq n\),
    记 \(B\) 是交换 \(A\) 的列~\(p\), \(q\), 且不改变其他的列得到的阵,
    记 \(C\) 是交换 \(B\) 的行~\(p\), \(q\), 且不改变其他的行得到的阵,
    则 \(
    \operatorname{pf} {(C)} = -\operatorname{pf} {(A)}
    \).

    再作命题 \(Q(n)\):
    \(P(n-1)\) 与 \(P(n)\) 是对的.
    我们用数学归纳法证明:
    对任何高于 \(1\) 的整数 \(n\), \(Q(n)\) 是对的.

    \(P(1)\) 不证自明.
    既然没有二列或二行可交换,
    我们认为, 它是平凡地对的.

    \(P(2)\) 是简单的.
    既然 \(n = 2\), 且 \(1 \leq p < q \leq n\),
    则 \(p = 1\), \(q = 2\).
    不难写出, 若 \(A =
    \begin{bmatrix}
        0  & a \\
        -a & 0 \\
    \end{bmatrix}
    \),
    则交换列~\(1\), \(2\)
    与行~\(1\), \(2\) 后,
    我们有 \(C =
    \begin{bmatrix}
        0 & -a \\
        a & 0  \\
    \end{bmatrix}
    \).
    则
    \begin{align*}
        \operatorname{pf} {(C)}
        = -a
        = -\operatorname{pf} {(A)}.
    \end{align*}

    综上, \(Q(2)\) 是对的.

    我们设 \(Q(n-1)\) 是对的 (\(n \geq 3\)).
    则 \(P(n-2)\) 与 \(P(n-1)\) 是对的.
    我们要由此证明 \(Q(n)\) 是对的,
    即 \(P(n-1)\) 与 \(P(n)\) 是对的.
    \(P(n-1)\) 当然是对的, 由假定.
    所以, 我们由假定
    ``\(P(n-2)\) 与 \(P(n-1)\) 是对的''
    证 \(P(n)\) 是对的,
    即得 \(Q(n)\) 是对的.

    设 \(A\) 是 \(n\)~级反称阵.
    设 \(1 \leq p < q \leq n\).
    设交换 \(A\) 的列~\(p\), \(q\), 不改变其他的列, 得阵~\(B\).
    设交换 \(B\) 的行~\(p\), \(q\), 不改变其他的行, 得阵~\(C\).
    我们分类讨论.

    (1)
    \(p = 1\), \(q = 2\).
    当 \(d_1 > 2\), \(d_2 > 2\), 且 \(d_2 \neq d_1\) 时,
    \([A]_{2,d_2}\) 是 \(A({1,d_1}|{1,d_1})\)
    的 \((1, d_2 - 1 - \rho (d_2, d_1))\) 元.
    则
    \begin{align*}
             &
        \operatorname{pf} {(A)}
        \\
        = {} &
        \sum_{2 \leq d_1 \leq n}
        {
        (-1)^{d_1} [A]_{1,d_1}
        \operatorname{pf} {(A({1,d_1}|{1,d_1}))}
        }
        \\
        = {} &
        (-1)^{2} [A]_{1,2}
        \operatorname{pf} {(A({1,2}|{1,2}))}
        +
        \sum_{3 \leq d_1 \leq n}
        {
        (-1)^{d_1} [A]_{1,d_1}
        \operatorname{pf} {(A({1,d_1}|{1,d_1}))}
        }
        \\
        = {} &
        \hphantom{{} + {}}
        [A]_{1,2}
        \operatorname{pf} {(A({1,2}|{1,2}))}
        \\
             &
        +
        \sum_{3 \leq d_1 \leq n}
        {
            (-1)^{d_1} [A]_{1,d_1}
        \sum_{\substack{3 \leq d_2 \leq n      \\d_2 \neq d_1}}
            {
                (-1)^{d_2 - 1 - \rho(d_2, d_1)}
            }
        }
        \\
             &
        \qquad \qquad \qquad
        \qquad \qquad \qquad
        \cdot
        [A]_{2,d_2}
        \operatorname{pf} {(A({1,2,d_1,d_2}|{1,2,d_1,d_2}))}
        \\
        = {} &
        \hphantom{{} + {}}
        [A]_{1,2}
        \operatorname{pf} {(A({1,2}|{1,2}))}
        \\
             &
        +
        \sum_{3 \leq d_1 \leq n}
        {
        \sum_{\substack{3 \leq d_2 \leq n      \\d_2 \neq d_1}}
            {
                (-1)^{d_1}
                    [A]_{1,d_1}
                (-1)^{d_2 - 1 - \rho(d_2, d_1)}
            }
        }
        \\
             &
        \qquad \qquad \qquad
        \qquad \qquad \qquad
        \cdot
        [A]_{2,d_2}
        \operatorname{pf} {(A({1,2,d_1,d_2}|{1,2,d_1,d_2}))}
        \\
        = {} &
        \hphantom{{} + {}}
        [A]_{1,2}
        \operatorname{pf} {(A({1,2}|{1,2}))}
        \\
             &
        +
        \sum_{\substack{3 \leq d_1, d_2 \leq n \\d_2 \neq d_1}}
        {
        (-1)^{d_1 + d_2 - 1 - \rho(d_2, d_1)}
            [A]_{1,d_1} [A]_{2,d_2}
        \operatorname{pf} {(A({1,2,d_1,d_2}|{1,2,d_1,d_2}))}
        }
        \\
        = {} &
        \hphantom{{} + {}}
        [A]_{1,2}
        \operatorname{pf} {(A({1,2}|{1,2}))}
        \\
             &
        +
        \sum_{3 \leq d_1 < d_2 \leq n}
        {
        (-1)^{d_1 + d_2 - 1 - \rho(d_2, d_1)}
            [A]_{1,d_1} [A]_{2,d_2}
        \operatorname{pf} {(A({1,2,d_1,d_2}|{1,2,d_1,d_2}))}
        }
        \\
             &
        +
        \sum_{3 \leq d_2 < d_1 \leq n}
        {
        (-1)^{d_1 + d_2 - 1 - \rho(d_2, d_1)}
            [A]_{1,d_1} [A]_{2,d_2}
        \operatorname{pf} {(A({1,2,d_1,d_2}|{1,2,d_1,d_2}))}
        }
        \\
        = {} &
        \hphantom{{} + {}}
        [A]_{1,2}
        \operatorname{pf} {(A({1,2}|{1,2}))}
        \\
             &
        +
        \sum_{3 \leq j < k \leq n}
        {
        (-1)^{j + k - 1 - 1}
            [A]_{1,j} [A]_{2,k}
        \operatorname{pf} {(A({1,2,j,k}|{1,2,j,k}))}
        }
        \\
             &
        +
        \sum_{3 \leq j < k \leq n}
        {
        (-1)^{k + j - 1}
            [A]_{1,k} [A]_{2,j}
        \operatorname{pf} {(A({1,2,k,j}|{1,2,k,j}))}
        }
        \\
        = {} &
        \hphantom{{} + {}}
        [A]_{1,2}
        \operatorname{pf} {(A({1,2}|{1,2}))}
        \\
             &
        +
        \sum_{3 \leq j < k \leq n}
        {
        (-1)^{j + k}
            [A]_{1,j} [A]_{2,k}
        \operatorname{pf} {(A({1,2,j,k}|{1,2,j,k}))}
        }
        \\
             &
        +
        \sum_{3 \leq j < k \leq n}
        {
        (-1)^{j + k}
        (-1) [A]_{1,k} [A]_{2,j}
        \operatorname{pf} {(A({1,2,j,k}|{1,2,j,k}))}
        }
        \\
        = {} &
        \hphantom{{} + {}}
        [A]_{1,2}
        \operatorname{pf} {(A({1,2}|{1,2}))}
        \\
             &
        +
        \sum_{3 \leq j < k \leq n}
        {
        (-1)^{j + k}\,
        ([A]_{1,j} [A]_{2,k} - [A]_{1,k} [A]_{2,j})
        \operatorname{pf} {(A({1,2,j,k}|{1,2,j,k}))}
        }
        \\
        = {} &
        \hphantom{{} + {}}
        [A]_{1,2}
        \operatorname{pf} {(A({1,2}|{1,2}))}
        \\
             &
        +
        \sum_{3 \leq j < k \leq n}
        {
        (-1)^{j + k}
        \det {
            \begin{bmatrix}
                [A]_{1,j} & [A]_{1,k} \\
                [A]_{2,j} & [A]_{2,k} \\
            \end{bmatrix}
        }
        \operatorname{pf} {(A({1,2,j,k}|{1,2,j,k}))}
        }.
    \end{align*}
    类似地,
    \begin{align*}
        %  &
        \operatorname{pf} {(C)}
        % \\
        = {} &
        \hphantom{{} + {}}
        [C]_{1,2}
        \operatorname{pf} {(C({1,2}|{1,2}))}
        \\
             &
        +
        \sum_{3 \leq j < k \leq n}
        {
        (-1)^{j + k}
        \det {
            \begin{bmatrix}
                [C]_{1,j} & [C]_{1,k} \\
                [C]_{2,j} & [C]_{2,k} \\
            \end{bmatrix}
        }
        \operatorname{pf} {(C({1,2,j,k}|{1,2,j,k}))}
        }.
    \end{align*}
    注意到, \([C]_{1,2} = -[A]_{1,2}\),
    且 \(A({1,2}|{1,2}) = C({1,2}|{1,2})\);
    再注意到, \(3 \leq j < k \leq n\) 时,
    \([C]_{1,j} = [A]_{2,j}\),
    \([C]_{1,k} = [A]_{2,k}\),
    \([C]_{2,j} = [A]_{1,j}\),
    \([C]_{2,k} = [A]_{1,k}\),
    且 \(A({1,2,j,k}|{1,2,j,k}) = C({1,2,j,k}|{1,2,j,k})\).
    故
    \begin{align*}
        %  &
        \operatorname{pf} {(C)}
        % \\
        = {} &
        \hphantom{{} + {}}
        {-[A]_{1,2}}
        \operatorname{pf} {(A({1,2}|{1,2}))}
        \\
             &
        +
        \sum_{3 \leq j < k \leq n}
        {
        (-1)^{j + k}
        \det {
            \begin{bmatrix}
                [A]_{2,j} & [A]_{2,k} \\
                [A]_{1,j} & [A]_{1,k} \\
            \end{bmatrix}
        }
        \operatorname{pf} {(A({1,2,j,k}|{1,2,j,k}))}
        }.
    \end{align*}
    由此可见, \(\operatorname{pf} {(C)}
    = -\operatorname{pf} {(A)}\).

    (2)
    \(2 \leq p = q - 1 \leq n - 1\).
    则
    \begin{align*}
             &
        \operatorname{pf} {(A)}
        \\
        = {} &
        \sum_{2 \leq j \leq n}
        {
        (-1)^{j} [A]_{1,j}
        \operatorname{pf} {(A({1,j}|{1,j}))}
        }
        \\
        = {} &
        \hphantom{{} + {}}
        (-1)^{p} [A]_{1,p}
        \operatorname{pf} {(A({1,p}|{1,p}))}
        +
        (-1)^{p+1} [A]_{1,p+1}
        \operatorname{pf} {(A({1,p+1}|{1,p+1}))}
        \\
             &
        +
        \sum_{\substack{2 \leq j \leq n \\ j \neq p, p+1}}
        {
        (-1)^{j} [A]_{1,j}
        \operatorname{pf} {(A({1,j}|{1,j}))}
        }.
    \end{align*}
    类似地,
    \begin{align*}
             &
        \operatorname{pf} {(C)}
        \\
        = {} &
        \hphantom{{} + {}}
        (-1)^{p} [C]_{1,p}
        \operatorname{pf} {(C({1,p}|{1,p}))}
        +
        (-1)^{p+1} [C]_{1,p+1}
        \operatorname{pf} {(C({1,p+1}|{1,p+1}))}
        \\
             &
        +
        \sum_{\substack{2 \leq j \leq n \\ j \neq p, p+1}}
        {
        (-1)^{j} [C]_{1,j}
        \operatorname{pf} {(C({1,j}|{1,j}))}
        }.
    \end{align*}
    注意到,
    \([C]_{1,p} = [A]_{1,p+1}\),
    \([C]_{1,p+1} = [A]_{1,p}\),
    且
    \begin{align*}
         & C({1,p}|{1,p}) = A({1,p+1}|{1,p+1}), \\
         & C({1,p+1}|{1,p+1}) = A({1,p}|{1,p});
    \end{align*}
    再注意到, \(j > 2\), 且 \(j \neq p\), \(j \neq p+1\) 时,
    \([C]_{1,j} = [A]_{1,j}\),
    且 \(C({1,j}|{1,j})\)
    可被认为是交换
    \(A({1,j}|{1,j})\) 的列~\(p'\), \(p'+1\)
    与行~\(p'\), \(p'+1\) 得到的反称阵
    (其中 \(p' = p - \rho (p, 1) - \rho(p, j)
    = p - 1 - \rho(p, j)\)).
    由假定,
    \(
    \operatorname{pf} {(C({1,j}|{1,j}))}
    = (-1) \operatorname{pf} {(A({1,j}|{1,j}))}
    \).
    故
    \begin{align*}
             &
        \operatorname{pf} {(C)}
        \\
        = {} &
        \hphantom{{} + {}}
        (-1)^{p} [A]_{1,p+1}
        \operatorname{pf} {(A({1,p+1}|{1,p+1}))}
        +
        (-1)^{p+1} [A]_{1,p}
        \operatorname{pf} {(A({1,p}|{1,p}))}
        \\
             &
        +
        \sum_{\substack{2 \leq j \leq n \\ j \neq p, p+1}}
        {
        (-1)^{j} [A]_{1,j}
        (-1) \operatorname{pf} {(A({1,j}|{1,j}))}
        }.
    \end{align*}
    由此可见, \(\operatorname{pf} {(C)}
    = -\operatorname{pf} {(A)}\).

    (3)
    其他的情形.
    不难看出, 我们已证,
    若列~\(p\), \(q\) (行~\(p\), \(q\)) 是相邻的
    (即 \(p = q - 1\);
    注意, 我们已约定 \(p < q\)),
    则 \(\operatorname{pf} {(C)}
    = -\operatorname{pf} {(A)}\).
    那么, 其他的情形自然是%
    当列~\(p\), \(q\) (行~\(p\), \(q\)) 不是相邻的时%
    的情形.

    我们交换 \(A\)~的列~\(p\), \(p+1\) 与行~\(p\), \(p+1\),
    得反称阵~\(C_1\).
    则 \(\operatorname{pf} {(C_1)} = -\operatorname{pf} {(A)}
    = (-1)^{1} \operatorname{pf} {(A)}\).
    我们交换 \(C_1\)~的列~\(p+1\), \(p+2\) 与行~\(p+1\), \(p+2\),
    得反称阵~\(C_2\).
    则 \(\operatorname{pf} {(C_2)} = -\operatorname{pf} {(C_1)}
    = (-1)^{2} \operatorname{pf} {(A)}\).
    \(\dots \dots\)
    我们交换 \(C_{q-p-1}\)~的列~\(q-1\), \(q\) 与行~\(q-1\), \(q\),
    得反称阵~\(C_{q-p}\).
    则 \(\operatorname{pf} {(C_{q-p})} = -\operatorname{pf} {(C_{q-p-1})}
    = (-1)^{q-p} \operatorname{pf} {(A)}\).

    记 \(G_0 = C_{q-p}\).
    则 \(\operatorname{pf} {(G_0)} = (-1)^{q-p} \operatorname{pf} {(A)}\).
    我们交换 \(G_0\)~的列~\(q-2\), \(q-1\) 与行~\(q-2\), \(q-1\),
    得反称阵~\(G_1\).
    则 \(\operatorname{pf} {(G_1)} = -\operatorname{pf} {(G_0)}
    = (-1)^{q-p+1} \operatorname{pf} {(A)}\).
    我们交换 \(G_1\)~的列~\(q-3\), \(q-2\) 与行~\(q-3\), \(q-2\),
    得反称阵~\(G_2\).
    则 \(\operatorname{pf} {(G_2)} = -\operatorname{pf} {(G_1)}
    = (-1)^{q-p+2} \operatorname{pf} {(A)}\).
    \(\dots \dots\)
    我们交换 \(G_{q-p-2}\)~的列~\(p\), \(p+1\) 与行~\(p\), \(p+1\),
    得反称阵~\(G_{q-p-1}\).
    则 \(\operatorname{pf} {(G_{q-p-1})} = -\operatorname{pf} {(G_{q-p-2})}
    = (-1)^{q-p+(q-p-1)} \operatorname{pf} {(A)}\).
    不难看出, \(C = G_{q-p-1}\).
    所以,
    \(\operatorname{pf} {(C)}
    = (-1)^{2(q-p)-1} \operatorname{pf} {(A)}
    = -\operatorname{pf} {(A)}\).

    所以, \(P(n)\) 是正确的.
    则 \(Q(n)\) 是正确的.
    由数学归纳法原理, 待证命题成立.
\end{proof}

利用此事, 我们可写出 pfaffian 的定义的变体.

\begin{theorem}
    设 \(A\) 是 \(n\)~级反称阵 (\(n > 2\)).
    设 \(i\) 是不超过 \(n\) 的正整数.
    则
    \begin{align*}
        \operatorname{pf} {(A)}
        =
        \sum_{\substack{1 \leq j \leq n \\j \neq i}}
        {
        (-1)^{i-1+j+\rho(i,j)}\, [A]_{i,j}
        \operatorname{pf} {(A({i,j}|{i,j}))}}.
    \end{align*}
\end{theorem}

\begin{proof}
    设 \(i = 1\).
    则 \(i - 1 + j + \rho(i, j) = j\), 若 \(j > 1\).
    所以, 这是对的.

    设 \(i > 1\).
    交换 \(A\) 的列~\(i-1\), \(i\) 与行~\(i-1\), \(i\),
    得反称阵~\(C_1\).
    则 \(\operatorname{pf} {(C_1)}
    = -\operatorname{pf} {(A)}
    = (-1)^1 \operatorname{pf} {(A)}\).
    交换 \(C_1\) 的列~\(i-2\), \(i-1\) 与行~\(i-2\), \(i-1\),
    得反称阵~\(C_2\).
    则 \(\operatorname{pf} {(C_2)}
    = -\operatorname{pf} {(C_1)}
    = (-1)^2 \operatorname{pf} {(A)}\).
    \(\dots \dots\)
    交换 \(C_{i-2}\) 的列~\(1\), \(2\) 与行~\(1\), \(2\),
    得反称阵~\(C_{i-1}\).
    则 \(\operatorname{pf} {(C_{i-1})}
    = -\operatorname{pf} {(C_{i-2})}
    = (-1)^{i-1} \operatorname{pf} {(A)}\).
    记 \(C = C_{i-1}\).
    则 \(\operatorname{pf} {(A)}
    = (-1)^{i-1} \operatorname{pf} {(C)}\).
    我们有如下发现.

    (1)
    \(A(i|i) = C(1|1)\).

    (2)
    当 \(j < i\) 时,
    \([A]_{i,j} = [C]_{1,j+1}\);
    当 \(j > i\) 时,
    \([A]_{i,j} = [C]_{1,j}\).
    于是,
    当 \(2 \leq j \leq i\) 时,
    \([C]_{1,j} = [A]_{i,j-1}\);
    当 \(j > i\) 时,
    \([C]_{1,j} = [A]_{i,j}\).

    (3)
    当 \(j < i\) 时,
    \(A({i,j}|{i,j}) = C({1,j+1}|{1,j+1})\);
    当 \(j > i\) 时,
    \(A({i,j}|{i,j}) = C({1,j}|{1,j})\).
    于是,
    当 \(2 \leq j \leq i\) 时,
    \(C({1,j}|{1,j}) = A({i,j-1}|{i,j-1})\);
    当 \(j > i\) 时,
    \(C({1,j}|{1,j}) = A({i,j}|{i,j})\).

    则
    \begin{align*}
             &
        \operatorname{pf} {(C)}
        \\
        = {} &
        \sum_{j = 2}^{n}
        {(-1)^{j} [C]_{1,j}
        \operatorname{pf} {(C({1,j}|{1,j}))}}
        \\
        = {} &
        \sum_{2 \leq j \leq i}
        {(-1)^{j} [C]_{1,j}
        \operatorname{pf} {(C({1,j}|{1,j}))}}
        +
        \sum_{i < j \leq n}
        {(-1)^{j} [C]_{1,j}
        \operatorname{pf} {(C({1,j}|{1,j}))}}
        \\
        = {} &
        \sum_{2 \leq j \leq i}
        {(-1)^{j} [A]_{i,j-1}
        \operatorname{pf} {(A({i,j-1}|{i,j-1}))}}
        +
        \sum_{i < j \leq n}
        {(-1)^{j} [A]_{1,j}
        \operatorname{pf} {(A({i,j}|{i,j}))}}
        \\
        = {} &
        \sum_{1 \leq j < i}
        {(-1)^{j+1} [A]_{i,j}
        \operatorname{pf} {(A({i,j}|{i,j}))}}
        +
        \sum_{i < j \leq n}
        {(-1)^{j} [A]_{1,j}
        \operatorname{pf} {(A({i,j}|{i,j}))}}
        \\
        = {} &
        \hphantom{{} + {}}
        \sum_{1 \leq j < i}
        {(-1)^{j+\rho(i,j)}\, [A]_{i,j}
        \operatorname{pf} {(A({i,j}|{i,j}))}}
        \\
             &
        +
        \sum_{i < j \leq n}
        {(-1)^{j+\rho(i,j)}\, [A]_{1,j}
        \operatorname{pf} {(A({i,j}|{i,j}))}}
        \\
        = {} &
        \sum_{\substack{1 \leq j \leq n \\j \neq i}}
        {(-1)^{j+\rho(i,j)}\, [A]_{i,j}
        \operatorname{pf} {(A({i,j}|{i,j}))}}.
    \end{align*}
    故
    \begin{align*}
        \operatorname{pf} {(A)}
        = {} &
        (-1)^{i-1}
        \operatorname{pf} {(C)}
        \\
        = {} &
        (-1)^{i-1}
        \sum_{\substack{1 \leq j \leq n \\j \neq i}}
        {(-1)^{j+\rho(i,j)}\, [A]_{i,j}
        \operatorname{pf} {(A({i,j}|{i,j}))}}
        \\
        = {} &
        \sum_{\substack{1 \leq j \leq n \\j \neq i}}
        {
        (-1)^{i-1}
        (-1)^{j+\rho(i,j)}\, [A]_{i,j}
        \operatorname{pf} {(A({i,j}|{i,j}))}}
        \\
        = {} &
        \sum_{\substack{1 \leq j \leq n \\j \neq i}}
        {
        (-1)^{i-1+j+\rho(i,j)}\, [A]_{i,j}
        \operatorname{pf} {(A({i,j}|{i,j}))}}.
        \qedhere
    \end{align*}
\end{proof}

\begin{theorem}
    设 \(A\) 是 \(n\)~级反称阵.
    设 \(p\), \(q\) 是不超过 \(n\) 的正整数, 且 \(p \neq q\).
    设 \(A\) 的列~\(p\), \(q\) 相等
    (于是, 显然, \(A\) 的行~\(p\), \(q\) 也相等).
    则 \(\operatorname{pf} {(A)} = 0\).
\end{theorem}

\begin{proof}
    我们先说明:
    \begin{quotation}
        设 \(A\) 是 \(n\)~级反称阵.
        设 \(p\), \(q\) 是不超过 \(n\) 的正整数, 且 \(p \neq q\).
        设 \(A\) 的列~\(p\), \(q\) 相等
        (于是, 显然, \(A\) 的行~\(p\), \(q\) 也相等).
        再设 \(u\), \(v\) 是不超过 \(n\) 的正整数, 且 \(u \neq v\).
        设交换 \(A\) 的列~\(u\), \(v\), 不改变其他的列, 得阵~\(B\).
        设交换 \(B\) 的行~\(u\), \(v\), 不改变其他的行,
        得反称阵~\(C\).
        则 \(C\) 仍有二列相等 (当然, 也有二行相等).
    \end{quotation}

    首先, 我们可具体地写下 \(C\) 的元:
    \begin{align*}
        [C]_{i,j} =
        \begin{cases}
            [A]_{i,j},
             &
            \text{\(i \neq u\), \(i \neq v\), \(j \neq u\), \(j \neq v\)};
            \\
            [A]_{v,j},
             &
            \text{\(i = u\), \(j \neq u\), \(j \neq v\)};
            \\
            [A]_{u,j},
             &
            \text{\(i = v\), \(j \neq u\), \(j \neq v\)};
            \\
            [A]_{i,v},
             &
            \text{\(j = u\), \(i \neq u\), \(i \neq v\)};
            \\
            [A]_{i,u},
             &
            \text{\(j = v\), \(i \neq u\), \(i \neq v\)};
            \\
            [A]_{v,v},
             &
            \text{\(i = j = u\)};
            \\
            [A]_{u,u},
             &
            \text{\(i = j = v\)};
            \\
            [A]_{v,u},
             &
            \text{\(i = u\), \(j = v\)};
            \\
            [A]_{u,v},
             &
            \text{\(i = v\), \(j = u\)}.
        \end{cases}
    \end{align*}
    然后, 我们可分类讨论.
    无妨设 \(p < q\), 且 \(u < v\).
    则:

    若 \(p \neq u\), \(p \neq v\),
    且 \(q \neq u\), \(q \neq v\),
    则我们可验证, \(C\) 的列~\(p\), \(q\) 相等;

    若 \(p = u\), 且 \(q = v\),
    则我们可验证, \(C\) 的列~\(p\), \(q\) 相等;

    若 \(p = u\), \(q \neq v\),
    则我们可验证, \(C\) 的列~\(q\), \(v\) 相等;

    若 \(p = v\),
    则我们可验证, \(C\) 的列~\(u\), \(q\) 相等;

    若 \(q = u\),
    则我们可验证, \(C\) 的列~\(p\), \(v\) 相等;

    若 \(q = v\), 且 \(p \neq u\),
    则我们可验证, \(C\) 的列~\(u\), \(p\) 相等.

    \vspace{2ex}

    好的.
    现在, 我们证原命题.

    无妨设 \(p < q\).

    先设 \(p = 1\), \(q = 2\).
    则
    \begin{align*}
        %  &
        \operatorname{pf} {(A)}
        % \\
        = {} &
        \hphantom{{} + {}}
        [A]_{1,2}
        \operatorname{pf} {(A({1,2}|{1,2}))}
        \\
             &
        +
        \sum_{3 \leq j < k \leq n}
        {
        (-1)^{j + k}
        \det {
            \begin{bmatrix}
                [A]_{1,j} & [A]_{1,k} \\
                [A]_{2,j} & [A]_{2,k} \\
            \end{bmatrix}
        }
        \operatorname{pf} {(A({1,2,j,k}|{1,2,j,k}))}
        }.
    \end{align*}
    因为 \(A\) 的列~\(1\), \(2\) 相等,
    故 \(A\) 的行~\(1\), \(2\) 也相等
    (注意, \(A\) 是反称阵),
    且 \([A]_{1,2} = [A]_{1,1} = 0\).
    则
    \begin{align*}
        %  &
        \operatorname{pf} {(A)}
        % \\
        = {} &
        \hphantom{{} + {}}
        0
        \operatorname{pf} {(A({1,2}|{1,2}))}
        \\
             &
        +
        \sum_{3 \leq j < k \leq n}
        {
        (-1)^{j + k}
        \,
        0
        \operatorname{pf} {(A({1,2,j,k}|{1,2,j,k}))}
        }
        \\
        = {} &
        0.
    \end{align*}

    设 \(p = 1\), 但 \(q > 2\).
    交换 \(A\) 的列~\(2\), \(q\), 不改变其他的列, 得阵~\(B_1\).
    交换 \(B_1\) 的行~\(2\), \(q\), 不改变其他的行,
    得反称阵~\(C_1\).
    则 \(C_1\) 的列~\(1\), \(2\) 相等,
    故 \(\operatorname{pf} {(C_1)} = 0\).
    另一方面, \(\operatorname{pf} {(C_1)}
    = -\operatorname{pf} {(A)}\),
    故 \(\operatorname{pf} {(A)} = 0\).

    设 \(p = 2\).
    交换 \(A\) 的列~\(1\), \(q\), 不改变其他的列, 得阵~\(B_2\).
    交换 \(B_2\) 的行~\(1\), \(q\), 不改变其他的行,
    得反称阵~\(C_2\).
    则 \(C_2\) 的列~\(1\), \(2\) 相等,
    故 \(\operatorname{pf} {(C_2)} = 0\).
    另一方面, \(\operatorname{pf} {(C_2)}
    = -\operatorname{pf} {(A)}\),
    故 \(\operatorname{pf} {(A)} = 0\).

    最后, 设 \(2 < p\).
    交换 \(A\) 的列~\(2\), \(p\), 不改变其他的列, 得阵~\(B_3\).
    交换 \(B_3\) 的行~\(2\), \(p\), 不改变其他的行,
    得反称阵~\(C_3\).
    则 \(C_3\) 的列~\(2\), \(q\) 相等,
    故 \(\operatorname{pf} {(C_3)} = 0\)
    (我们已证此情形).
    另一方面, \(\operatorname{pf} {(C_3)}
    = -\operatorname{pf} {(A)}\),
    故 \(\operatorname{pf} {(A)} = 0\).
\end{proof}

有了这些有较长的论证的性质, 我们可以证明%
倍加与反称阵的 pfaffian 的关系.
这是重要的.

\begin{restatable}{theorem}{TheoremPfaffianMulitplyAndAdd}
    设 \(A\) 是 \(n\)~级反称阵.
    加 \(A\) 的列~\(p\) 的 \(s\)~倍于列~\(q\)
    (\(p \neq q\)),
    不改变其他的列, 得 \(n\)~级阵 \(B\).
    加 \(B\) 的行~\(p\) 的 \(s\)~倍于行~\(q\),
    不改变其他的行, 得 \(n\)~级阵 \(C\).
    则 \(\operatorname{pf} {(C)} = \operatorname{pf} {(A)}\).
\end{restatable}

\begin{proof}
    不难写出 \(C\)~的元:
    \begin{align*}
        [C]_{i,j} =
        \begin{cases}
            [A]_{q,j} + s[A]_{p,j},
             & \text{\(i = q\), 且 \(j \neq q\)}; \\
            [A]_{i,q} + s[A]_{i,p},
             & \text{\(j = q\), 且 \(i \neq q\)}; \\
            [A]_{i,j},
             & \text{其他}.
        \end{cases}
    \end{align*}
    作 \(n\)~级阵 \(G\) 如下:
    \begin{align*}
        [G]_{i,j} =
        \begin{cases}
            [A]_{p,j}, & \text{\(i = q\), 且 \(j \neq q\)}; \\
            [A]_{i,p}, & \text{\(j = q\), 且 \(i \neq q\)}; \\
            [A]_{i,j}, & \text{其他}.
        \end{cases}
    \end{align*}
    则 \(G\) 是反称阵, \(G\) 的列~\(p\), \(q\) 相等,
    且:

    (1)
    \([A]_{i,j} = [G]_{i,j} = [C]_{i,j}\),
    对任何不等于 \(q\), 且不超过 \(n\) 的正整数 \(i\), \(j\);

    (2)
    \([C]_{i,j} = 1[A]_{i,j} + s[G]_{i,j}\),
    若 \(i = q\) 或 \(j = q\).

    所以,
    \begin{align*}
        \operatorname{pf} {(C)}
        = {} &
        1 \operatorname{pf} {(A)} + s \operatorname{pf} {(G)}
        \\
        = {} &
        \operatorname{pf} {(A)} + s \, 0
        \\
        = {} &
        \operatorname{pf} {(A)}.
        \qedhere
    \end{align*}
\end{proof}

现在, 我们可以证明,
反称阵的行列式与 pfaffian 有如下关系.

\begin{restatable}{theorem}{TheoremPfaffianSquareDet}
    设 \(A\) 是 \(n\)~级反称阵.
    则 \(\det {(A)}
    = (\operatorname{pf} {(A)})^2\).
\end{restatable}

\begin{proof}
    设 \(n\) 是奇数.
    则 \(\det {(A)} = 0\),
    且 \(\operatorname{pf} {(A)} = 0\).

    再设 \(n = 2m\) 是偶数.
    则利用若干次列的倍加, 与对应的行的倍加
    (``先列后行, 交替地作''),
    我们可变 \(A\) 为
    \begin{align*}
        B =
        \begin{bmatrix}
            0      & b_1    & 0      & 0      & \cdots & 0      & 0      \\
            -b_1   & 0      & 0      & 0      & \cdots & 0      & 0      \\
            0      & 0      & 0      & b_2    & \cdots & 0      & 0      \\
            0      & 0      & -b_2   & 0      & \cdots & 0      & 0      \\
            \vdots & \vdots & \vdots & \vdots & \ddots & \vdots & \vdots \\
            0      & 0      & 0      & 0      & \cdots & 0      & b_m    \\
            0      & 0      & 0      & 0      & \cdots & -b_m   & 0      \\
        \end{bmatrix}.
    \end{align*}
    则 \(\det {(B)} = \det {(A)}\),
    且 \(\operatorname{pf} {(B)} = \operatorname{pf} {(A)}\).
    最后, 注意到 \(\det {(B)}
    = (\operatorname{pf} {(B)})^2\),
    故 \(\det {(A)}
    = (\operatorname{pf} {(A)})^2\).
\end{proof}

我以一个应用结束本节.

设 \(S\) 是 ``整'', 或 ``有理'', 或 ``实'', 或 ``复'' 中的一个.
我们约定,
当 \(S\) 是 ``整'' 时, \(S\)-数是 ``整数'';
当 \(S\) 是 ``有理'' 时, \(S\)-数是 ``有理数'';
当 \(S\) 是 ``实'' 时, \(S\)-数是 ``实数'';
当 \(S\) 是 ``复'' 时, \(S\)-数是 ``复数''.
于是, 二个 \(S\)-数的和、差、积还是 \(S\)-数.

若一个阵的元全是 \(S\)-数,
我们说, 此阵是一个 \(S\)-阵.

现在, 我们设 \(A\) 是一个 \(S\)-反称阵
(\(A\) 是反称阵, 且 \(A\) 是 \(S\)-阵).
因为 pfaffian 的定义是由一些数的和、差、积作成的,
且不含除法,
故 \(\operatorname{pf} {(A)}\) 是一个 \(S\)-数.
则 \(\det {(A)}\) 是一个 \(S\)-数的平方.
具体地 (且较有趣地),
若 \(A\) 是一个整反称阵, 则 \(\det {(A)}\) 是一个整数的平方
(完全平方数);
若 \(A\) 是一个实反称阵, 则 \(\det {(A)}\) 是一个实数的平方
(非负实数).

\section{阵的积与倍加 (续)}

本节, 我们进一步地讨论阵的积与 (列的) 倍加的关系.

回想起, 我们有如下结论.

\TheoremMultiplyAndAdd*

那么, 特别地, 取 \(A\) 为方阵, 有

\begin{theorem}
    设 \(A\) 是 \(n\)~级阵.
    利用若干次列的倍加,
    我们可变 \(A\) 为 \(n\)~级阵 \(B\),
    使当 \(i < j\) 时,
    \([B]_{i,j} = 0\).
\end{theorem}

形象地, 对任何 \(n\)~级阵
\begin{align*}
    A =
    \begin{bmatrix}
        [A]_{1,1}   & [A]_{1,2}   & \cdots & [A]_{1,n-1}   & [A]_{1,n}   \\
        [A]_{2,1}   & [A]_{2,2}   & \cdots & [A]_{2,n-1}   & [A]_{2,n}   \\
        \vdots      & \vdots      & {}     & \vdots        & \vdots      \\
        [A]_{n-1,1} & [A]_{n-1,2} & \cdots & [A]_{n-1,n-1} & [A]_{n-1,n} \\
        [A]_{n,1}   & [A]_{n,2}   & \cdots & [A]_{n,n-1}   & [A]_{n,n}   \\
    \end{bmatrix},
\end{align*}
我们总可作列的倍加, 变 \(A\) 为
\begin{align*}
    B =
    \begin{bmatrix}
        [B]_{1,1}   & 0           & \cdots & 0             & 0         \\
        [B]_{2,1}   & [B]_{2,2}   & \cdots & 0             & 0         \\
        \vdots      & \vdots      & \ddots & \vdots        & \vdots    \\
        [B]_{n-1,1} & [B]_{n-1,2} & \cdots & [B]_{n-1,n-1} & 0         \\
        [B]_{n,1}   & [B]_{n,2}   & \cdots & [B]_{n,n-1}   & [B]_{n,n} \\
    \end{bmatrix}.
\end{align*}

因为 \(A\), \(B\) 是方阵,
故我们可说 \(A\), \(B\) 的行列式.
现在, 我说, 若 \(\det {(A)} \neq 0\),
则我们可用列的倍加, 进一步地变 \(B\) 为
\begin{align*}
    D =
    \begin{bmatrix}
        [B]_{1,1} & 0         & \cdots & 0             & 0         \\
        0         & [B]_{2,2} & \cdots & 0             & 0         \\
        \vdots    & \vdots    & \ddots & \vdots        & \vdots    \\
        0         & 0         & \cdots & [B]_{n-1,n-1} & 0         \\
        0         & 0         & \cdots & 0             & [B]_{n,n} \\
    \end{bmatrix};
\end{align*}
具体地, \(D\) 是一个 \(n\)~级阵,
且
\begin{align*}
    [D]_{i,j} =
    \begin{cases}
        [B]_{i,i}, & i = j;     \\
        0,         & \text{其他}.
    \end{cases}
\end{align*}

我们知道, 倍加不改变行列式.
于是, \(\det {(B)} = \det {(A)} \neq 0\).
另一方面, \(\det {(B)} = [B]_{1,1} [B]_{2,2} \dots [B]_{n,n}\),
故 \([B]_{1,1}\), \([B]_{2,2}\), \(\dots\), \([B]_{n,n}\)
都不是零.
那么, 我们加 \(B\) 的%
列~\(n\) 的 \(-[B]_{n,n-1}/[B]_{n,n}\)~倍于列~\(n-1\),
列~\(n\) 的 \(-[B]_{n,n-2}/[B]_{n,n}\)~倍于列~\(n-2\),
\(\dots \dots\),
列~\(n\) 的 \(-[B]_{n,1}/[B]_{n,n}\)~倍于列~\(1\),
得
\begin{align*}
    \begin{bmatrix}
        [B]_{1,1}   & 0           & \cdots & 0             & 0             & 0         \\
        [B]_{2,1}   & [B]_{2,2}   & \cdots & 0             & 0             & 0         \\
        \vdots      & \vdots      & \ddots & \vdots        & \vdots        & \vdots    \\
        [B]_{n-2,1} & [B]_{n-2,2} & \cdots & [B]_{n-2,n-2} & 0             & 0         \\
        [B]_{n-1,1} & [B]_{n-1,2} & \cdots & [B]_{n-1,n-2} & [B]_{n-1,n-1} & 0         \\
        0           & 0           & \cdots & 0             & 0             & [B]_{n,n} \\
    \end{bmatrix}.
\end{align*}
我们再加%
列~\(n-1\) 的 \(-[B]_{n-1,n-3}/[B]_{n-1,n-1}\)~倍于列~\(n-2\),
列~\(n-1\) 的 \(-[B]_{n-1,n-3}/[B]_{n-1,n-1}\)~倍于列~\(n-3\),
\(\dots \dots\),
列~\(n-1\) 的 \(-[B]_{n-1,1}/[B]_{n-1,n-1}\)~倍于列~\(1\),
得
\begin{align*}
    \begin{bmatrix}
        [B]_{1,1}   & 0           & \cdots & 0             & 0             & 0         \\
        [B]_{2,1}   & [B]_{2,2}   & \cdots & 0             & 0             & 0         \\
        \vdots      & \vdots      & \ddots & \vdots        & \vdots        & \vdots    \\
        [B]_{n-2,1} & [B]_{n-2,2} & \cdots & [B]_{n-2,n-2} & 0             & 0         \\
        0           & 0           & \cdots & 0             & [B]_{n-1,n-1} & 0         \\
        0           & 0           & \cdots & 0             & 0             & [B]_{n,n} \\
    \end{bmatrix}.
\end{align*}
\(\dots \dots\)
最后, 我们得
\begin{align*}
    \begin{bmatrix}
        [B]_{1,1} & 0         & \cdots & 0             & 0         \\
        0         & [B]_{2,2} & \cdots & 0             & 0         \\
        \vdots    & \vdots    & \ddots & \vdots        & \vdots    \\
        0         & 0         & \cdots & [B]_{n-1,n-1} & 0         \\
        0         & 0         & \cdots & 0             & [B]_{n,n} \\
    \end{bmatrix}.
\end{align*}

综上, 我们有

\begin{theorem}
    设 \(A\) 是 \(n\)~级阵,
    且 \(\det {(A)} \neq 0\).
    利用若干次列的倍加,
    我们可变 \(A\) 为 \(n\)~级阵 \(D\),
    使当 \(i \neq j\) 时,
    \([D]_{i,j} = 0\).
\end{theorem}

进一步地, 我说, 我们还可变 \(D\) 为
\begin{align*}
    M =
    \begin{bmatrix}
        \det {(A)} & 0      & \cdots & 0      & 0      \\
        0          & 1      & \cdots & 0      & 0      \\
        \vdots     & \vdots & \ddots & \vdots & \vdots \\
        0          & 0      & \cdots & 1      & 0      \\
        0          & 0      & \cdots & 0      & 1      \\
    \end{bmatrix};
\end{align*}
具体地, \(M\) 是一个 \(n\)~级阵,
且
\begin{align*}
    [M]_{i,j} =
    \begin{cases}
        \det {(A)}, & i = j = 1; \\
        1,          & i = j > 1; \\
        0,          & \text{其他}.
    \end{cases}
\end{align*}

设 \(D\) 形如
\begin{align*}
    \begin{bmatrix}
        [B]_{1,1} & 0         & \cdots & 0             & 0             & 0         \\
        0         & [B]_{2,2} & \cdots & 0             & 0             & 0         \\
        \vdots    & \vdots    & \ddots & \vdots        & \vdots        & \vdots    \\
        0         & 0         & \cdots & [B]_{n-2,n-2} & 0             & 0         \\
        0         & 0         & \cdots & 0             & [B]_{n-1,n-1} & 0         \\
        0         & 0         & \cdots & 0             & 0             & [B]_{n,n} \\
    \end{bmatrix},
\end{align*}
其中 \([B]_{1,1}\), \([B]_{2,2}\), \(\dots\), \([B]_{n,n}\)
都不是零.

加列~\(n\) 的 \((1 - [B]_{n,n})/[B]_{n,n}\)~倍于列~\(n-1\), 有
\begin{align*}
    \begin{bmatrix}
        [B]_{1,1} & 0         & \cdots & 0             & 0             & 0         \\
        0         & [B]_{2,2} & \cdots & 0             & 0             & 0         \\
        \vdots    & \vdots    & \ddots & \vdots        & \vdots        & \vdots    \\
        0         & 0         & \cdots & [B]_{n-2,n-2} & 0             & 0         \\
        0         & 0         & \cdots & 0             & [B]_{n-1,n-1} & 0         \\
        0         & 0         & \cdots & 0             & 1-[B]_{n,n}   & [B]_{n,n} \\
    \end{bmatrix}.
\end{align*}
加列~\(n-1\) 于列~\(n\), 有
\begin{align*}
    \begin{bmatrix}
        [B]_{1,1} & 0         & \cdots & 0             & 0             & 0             \\
        0         & [B]_{2,2} & \cdots & 0             & 0             & 0             \\
        \vdots    & \vdots    & \ddots & \vdots        & \vdots        & \vdots        \\
        0         & 0         & \cdots & [B]_{n-2,n-2} & 0             & 0             \\
        0         & 0         & \cdots & 0             & [B]_{n-1,n-1} & [B]_{n-1,n-1} \\
        0         & 0         & \cdots & 0             & 1-[B]_{n,n}   & 1             \\
    \end{bmatrix}.
\end{align*}
加列~\(n\) 的 \([B]_{n,n}-1\)~倍于列~\(n-1\), 有
\begin{align*}
    \begin{bmatrix}
        [B]_{1,1} & 0         & \cdots & 0             & 0                       & 0             \\
        0         & [B]_{2,2} & \cdots & 0             & 0                       & 0             \\
        \vdots    & \vdots    & \ddots & \vdots        & \vdots                  & \vdots        \\
        0         & 0         & \cdots & [B]_{n-2,n-2} & 0                       & 0             \\
        0         & 0         & \cdots & 0             & [B]_{n-1,n-1} [B]_{n,n} & [B]_{n-1,n-1} \\
        0         & 0         & \cdots & 0             & 0                       & 1             \\
    \end{bmatrix}.
\end{align*}
加列~\(n-1\) 的 \(-1/[B]_{n,n}\)~倍于列~\(n\), 有
\begin{align*}
    \begin{bmatrix}
        [B]_{1,1} & 0         & \cdots & 0             & 0                       & 0      \\
        0         & [B]_{2,2} & \cdots & 0             & 0                       & 0      \\
        \vdots    & \vdots    & \ddots & \vdots        & \vdots                  & \vdots \\
        0         & 0         & \cdots & [B]_{n-2,n-2} & 0                       & 0      \\
        0         & 0         & \cdots & 0             & [B]_{n-1,n-1} [B]_{n,n} & 0      \\
        0         & 0         & \cdots & 0             & 0                       & 1      \\
    \end{bmatrix}.
\end{align*}
为方便, 记 \(d_2 = [B]_{n-1,n-1} [B]_{n,n}\).
加列~\(n-1\) 的 \((1 - d_2)/d_2\)~倍于列~\(n-2\), 有
\begin{align*}
    \begin{bmatrix}
        [B]_{1,1} & 0         & \cdots & 0             & 0      & 0      \\
        0         & [B]_{2,2} & \cdots & 0             & 0      & 0      \\
        \vdots    & \vdots    & \ddots & \vdots        & \vdots & \vdots \\
        0         & 0         & \cdots & [B]_{n-2,n-2} & 0      & 0      \\
        0         & 0         & \cdots & 1 - d_2       & d_2    & 0      \\
        0         & 0         & \cdots & 0             & 0      & 1      \\
    \end{bmatrix}.
\end{align*}
加列~\(n-2\) 于列~\(n-1\), 有
\begin{align*}
    \begin{bmatrix}
        [B]_{1,1} & 0         & \cdots & 0             & 0             & 0      \\
        0         & [B]_{2,2} & \cdots & 0             & 0             & 0      \\
        \vdots    & \vdots    & \ddots & \vdots        & \vdots        & \vdots \\
        0         & 0         & \cdots & [B]_{n-2,n-2} & [B]_{n-2,n-2} & 0      \\
        0         & 0         & \cdots & 1 - d_2       & 1             & 0      \\
        0         & 0         & \cdots & 0             & 0             & 1      \\
    \end{bmatrix}.
\end{align*}
加列~\(n-1\) 的 \(d_2 - 1\)~倍于列~\(n-2\), 有
\begin{align*}
    \begin{bmatrix}
        [B]_{1,1} & 0         & \cdots & 0                 & 0             & 0      \\
        0         & [B]_{2,2} & \cdots & 0                 & 0             & 0      \\
        \vdots    & \vdots    & \ddots & \vdots            & \vdots        & \vdots \\
        0         & 0         & \cdots & [B]_{n-2,n-2} d_2 & [B]_{n-2,n-2} & 0      \\
        0         & 0         & \cdots & 0                 & 1             & 0      \\
        0         & 0         & \cdots & 0                 & 0             & 1      \\
    \end{bmatrix}.
\end{align*}
加列~\(n-2\) 的 \(-1/d_2\)~倍于列~\(n-1\), 有
\begin{align*}
    \begin{bmatrix}
        [B]_{1,1} & 0         & \cdots & 0                                     & 0      & 0      \\
        0         & [B]_{2,2} & \cdots & 0                                     & 0      & 0      \\
        \vdots    & \vdots    & \ddots & \vdots                                & \vdots & \vdots \\
        0         & 0         & \cdots & [B]_{n-2,n-2} [B]_{n-1,n-1} [B]_{n,n} & 0      & 0      \\
        0         & 0         & \cdots & 0                                     & 1      & 0      \\
        0         & 0         & \cdots & 0                                     & 0      & 1      \\
    \end{bmatrix}.
\end{align*}
\(\dots \dots\)
最后, 我们得
\begin{align*}
    \begin{bmatrix}
        [B]_{1,1} [B]_{2,2} \dots [B]_{n,n} & 0      & \cdots & 0      & 0      \\
        0                                   & 1      & \cdots & 0      & 0      \\
        \vdots                              & \vdots & \ddots & \vdots & \vdots \\
        0                                   & 0      & \cdots & 1      & 0      \\
        0                                   & 0      & \cdots & 0      & 1      \\
    \end{bmatrix};
\end{align*}
再注意到
\(\det {(A)} = [B]_{1,1} [B]_{2,2} \dots [B]_{n,n}\).

综上, 我们有

\begin{theorem}
    设 \(A\) 是 \(n\)~级阵,
    且 \(\det {(A)} \neq 0\).
    利用若干次列的倍加,
    我们可变 \(A\) 为 \(n\)~级阵 \(M\),
    使
    \begin{align*}
        [M]_{i,j} =
        \begin{cases}
            \det {(A)}, & i = j = 1; \\
            1,          & i = j > 1; \\
            0,          & \text{其他}.
        \end{cases}
    \end{align*}
\end{theorem}

我们知道, 我们可用阵的积表示倍加.
于是

\begin{theorem}
    设 \(A\) 是 \(n\)~级阵,
    且 \(\det {(A)} \neq 0\).
    则存在若干个形如 \(E(n; p, q; s)\)
    (\(s\) 是一个数;
    \(p\), \(q\) 是不超过 \(n\) 的正整数,
    \(p \neq q\))
    的阵
    \(E_1\), \(E_2\), \(\dots\), \(E_w\),
    使
    \begin{align*}
        [A E_1 E_2 \dots E_w]_{i,j} =
        \begin{cases}
            \det {(A)}, & i = j = 1; \\
            1,          & i = j > 1; \\
            0,          & \text{其他}.
        \end{cases}
    \end{align*}
\end{theorem}

回想起, \(E(n; p, q; s)\)
(\(s\) 是一个数;
\(p\), \(q\) 是不超过 \(n\) 的正整数,
\(p \neq q\))
是适合如下条件的 \(n\)~级阵:
\begin{align*}
    [E(n; p, q; s)]_{i,j}
    = \begin{cases}
          s,     & \text{\(i = p\), 且 \(j = q\)}; \\
          [I_n], & \text{其他}.
      \end{cases}
\end{align*}

\section{Pfaffian 的性质 (续)}

前面, 我们得到了 pfaffian 的不少性质.
以下三条是重要的:

\TheoremPfaffianMulitply*

\TheoremPfaffianMulitplyAndAdd*

\TheoremPfaffianSquareDet*

本节, 我们证明 pfaffian 的一个新的重要的性质.
为此, 我们先改写第~2~条性质.

\begin{theorem}
    设 \(A\) 是 \(n\)~级反称阵.
    设 \(s\) 是数.
    设 \(p\), \(q\) 是不超过 \(n\) 的正整数,
    且 \(p \neq q\).
    则
    \begin{align*}
        \operatorname{pf}{( (E(n; p, q; s))^{\mathrm{T}}
            A E(n; p, q; s) )}
        = \operatorname{pf} {(A)} \det {(E(n; p, q; s))}.
    \end{align*}
\end{theorem}

\begin{proof}
    加 \(A\) 的列~\(p\) 的 \(s\)~倍于列~\(q\),
    不改变其他的列, 得 \(n\)~级阵 \(B\).
    加 \(B\) 的行~\(p\) 的 \(s\)~倍于行~\(q\),
    不改变其他的行, 得 \(n\)~级阵 \(C\).
    则 \(\operatorname{pf} {(C)} = \operatorname{pf} {(A)}\).

    注意到 \(C = (E(n; p, q; s))^{\mathrm{T}}
    A E(n; p, q; s)\),
    故
    \begin{align*}
        \operatorname{pf}{( (E(n; p, q; s))^{\mathrm{T}}
            A E(n; p, q; s) )} = \operatorname{pf} {(A)}.
    \end{align*}
    再注意到 \(\det {(E(n; p, q; s))} = 1\),
    故
    \begin{equation*}
        %  &
        \operatorname{pf}{( (E(n; p, q; s))^{\mathrm{T}}
            A E(n; p, q; s) )}
        % \\
        % =
        % {} &
        % \operatorname{pf} {(A)}\, 1
        % \\
        =
        % {} &
        \operatorname{pf} {(A)}
        \det {(E(n; p, q; s))}.
        \qedhere
    \end{equation*}
\end{proof}

好的.
现在, 我们可以证明本节的主要结论了.

\begin{theorem}
    设 \(A\) 是 \(n\)~级反称阵.
    设 \(P\) 是 \(n\)~级阵.
    则
    \begin{align*}
        \operatorname{pf} {(P^{\mathrm{T}} A P)}
        = \operatorname{pf} {(A)} \det {(P)}.
    \end{align*}
\end{theorem}

\begin{proof}
    若 \(\det {(P)} = 0\),
    则 \(\det {(P^{\mathrm{T}} A P)}
    = \det {(P^{\mathrm{T}} A)} \det {(P)}
    = 0\).
    则 \(P^{\mathrm{T}} A P\) 的 pfaffian 为 \(0\).
    所以, 若 \(\det {(P)} = 0\),
    则命题是对的.

    若 \(\det {(P)} \neq 0\),
    则存在若干个形如 \(E(n; p, q; s)\)
    (\(s\) 是一个数;
    \(p\), \(q\) 是不超过 \(n\) 的正整数,
    \(p \neq q\))
    的阵
    \(E_1\), \(E_2\), \(\dots\), \(E_w\),
    使 \(M = P E_1 E_2 \dots E_w\)
    适合
    \begin{align*}
        [M]_{i,j} =
        \begin{cases}
            \det {(P)}, & i = j = 1; \\
            1,          & i = j > 1; \\
            0,          & \text{其他}.
        \end{cases}
    \end{align*}
    则
    \begin{align*}
        \operatorname{pf} {(P^{\mathrm{T}} A P)}
        = {} &
        \operatorname{pf} {(P^{\mathrm{T}} A P)}
        \, 1
        \\
        = {} &
        \operatorname{pf} {(P^{\mathrm{T}} A P)}
        \det {(E_1)}
        \det {(E_2)}
        \dots
        \det {(E_{w-1})}
        \det {(E_w)}
        \\
        = {} &
        \operatorname{pf}
        {(E_1^{\mathrm{T}} (P^{\mathrm{T}} A P) E_1)}
        \det {(E_2)}
        \dots
        \det {(E_{w-1})}
        \det {(E_w)}
        \\
        = {} &
        \dots \dots \dots \dots
        \dots \dots \dots \dots
        \dots \dots \dots \dots
        \dots \dots \dots \dots
        \\
        = {} &
        \operatorname{pf}
        {(
        E_{w-1}^{\mathrm{T}}
        \dots
        (E_2^{\mathrm{T}}
        (E_1^{\mathrm{T}}
        (P^{\mathrm{T}} A P)
        E_1)
        E_2)
        \dots
        E_{w-1}
        )}
        \det {(E_w)}
        \\
        = {} &
        \operatorname{pf}
        {(
        E_w^{\mathrm{T}}
        (E_{w-1}^{\mathrm{T}}
        \dots
        (E_2^{\mathrm{T}}
        (E_1^{\mathrm{T}}
        (P^{\mathrm{T}} A P)
        E_1)
        E_2)
        \dots
        E_{w-1})
        E_w
        )}
        \\
        = {} &
        \operatorname{pf}
        {(
        (
        E_w^{\mathrm{T}}
        E_{w-1}^{\mathrm{T}}
        \dots
        E_2^{\mathrm{T}}
        E_1^{\mathrm{T}}
        P^{\mathrm{T}}
        )
        A
        (
        P
        E_1
        E_2
        \dots
        E_{w-1}
        E_w
        )
        )}
        \\
        = {} &
        \operatorname{pf}
        {(
            (P E_1 E_2 \dots E_{w-1} E_w)^{\mathrm{T}}
            A
            (P E_1 E_2 \dots E_{w-1} E_w)
            )}
        \\
        = {} &
        \operatorname{pf}
        {(
            M^{\mathrm{T}} A M
            )}.
    \end{align*}
    我们计算 \(M^{\mathrm{T}} A M\):
    \begin{align*}
        %  &
        [M^{\mathrm{T}} A M]_{i,j}
        % \\
        = {} &
        \sum_{k = 1}^{n}
        {
        [M^{\mathrm{T}}]_{i,k} [A M]_{k,j}
        }
        \\
        = {} &
        \sum_{k = 1}^{n}
        {
        [M]_{k,i} [A M]_{k,j}
        }
        \\
        = {} &
        [M]_{i,i} [A M]_{i,j}
        +
        \sum_{\substack{1 \leq k \leq n    \\k \neq i}}
        {
            [M]_{k,i} [A M]_{k,j}
        }
        \\
        = {} &
        [M]_{i,i} [A M]_{i,j}
        +
        \sum_{\substack{1 \leq k \leq n    \\k \neq i}}
        {
            0\, [A M]_{k,j}
        }
        \\
        = {} &
        [M]_{i,i}
            [A M]_{i,j}
        \\
        = {} &
        [M]_{i,i}
        \sum_{\ell = 1}^{n}
        {[A]_{i,\ell} [M]_{\ell,j}}
        \\
        = {} &
        [M]_{i,i}
        \Bigg(
        [A]_{i,j} [M]_{j,j}
        +
        \sum_{\substack{1 \leq \ell \leq n \\ \ell \neq j}}
        {[A]_{i,\ell} [M]_{\ell,j}}
        \Bigg)
        \\
        = {} &
        [M]_{i,i}
        \Bigg(
        [A]_{i,j} [M]_{j,j}
        +
        \sum_{\substack{1 \leq \ell \leq n \\ \ell \neq j}}
        {[A]_{i,\ell} \, 0}
        \Bigg)
        \\
        = {} &
        [M]_{i,i}
            [A]_{i,j}
            [M]_{j,j}.
    \end{align*}
    当 \(i \neq 1\), \(j \neq 1\) 时,
    \begin{align*}
        [M]_{i,i}
        [A]_{i,j}
        [M]_{j,j}
        = 1\, [A]_{i,j}\, 1
            = [A]_{i,j};
    \end{align*}
    当 \(i = 1\), \(j \neq 1\) 时,
    \begin{align*}
        [M]_{i,i}
        [A]_{i,j}
        [M]_{j,j}
        = \det {(P)}\, [A]_{i,j}\, 1
        = \det {(P)}\, [A]_{i,j};
    \end{align*}
    当 \(i \neq 1\), \(j = 1\) 时,
    \begin{align*}
        [M]_{i,i}
        [A]_{i,j}
        [M]_{j,j}
        = 1\, [A]_{i,j} \det {(P)}
        = \det {(P)}\, [A]_{i,j};
    \end{align*}
    当 \(i = 1\), \(j = 1\) 时,
    \begin{align*}
        [M]_{i,i}
        [A]_{i,j}
        [M]_{j,j}
        = \det {(P)}\, 0 \det {(P)}
        = 0
        = \det {(P)}\, [A]_{i,j}.
    \end{align*}
    则
    \begin{equation*}
        \operatorname{pf} {(
            P^{\mathrm{T}} A P
            )}
        =
        \operatorname{pf} {(
            M^{\mathrm{T}} A M
            )}
        =
        \det {(P)} \operatorname{pf} {(A)}
        =
        \operatorname{pf} {(A)} \det {(P)}.
        \qedhere
    \end{equation*}
\end{proof}

% 最后, 注意到 \((P^{\mathrm{T}})^{\mathrm{T}} = P\),
% 且 \(\det {(P^{\mathrm{T}})} = \det {(P)}\),
% 我们有

% \begin{theorem}
%     设 \(A\) 是 \(n\)~级反称阵.
%     设 \(P\) 是 \(n\)~级阵.
%     则
%     \begin{align*}
%         \operatorname{pf} {(P A P^{\mathrm{T}})}
%         = \operatorname{pf} {(A)} \det {(P)}.
%     \end{align*}
% \end{theorem}

\section{\texorpdfstring{辛阵的行列式为 \(1\)}
  {辛阵的行列式为 1}}

现在, 我们可以证明, 辛阵的行列式为 \(1\).

回想起, 说一个 \(2m\)~级阵 \(A\) 是辛阵,
若 \(A^{\mathrm{T}} K_m A = K_m\),
其中 \(K_m\) 是形如
\begin{align*}
    \begin{bmatrix}
        0    & I_m \\
        -I_m & 0   \\
    \end{bmatrix}
\end{align*}
的 \(2m\)~级反称阵.

% 不难算出, \(\operatorname{pf} {(K_m)} = (-1)^{m(m-1)/2}\).
% 则 \(\operatorname{pf} {(K_m)} \operatorname{pf} {(K_m)} = 1\).
不难算出, \(\det {(K_m)} = 1\).
则 \(\operatorname{pf} {(K_m)} \operatorname{pf} {(K_m)}
= \det {(K_m)} = 1\).
由 \(\operatorname{pf} {(K_m)}
= \operatorname{pf} {(A^{\mathrm{T}} K_m A)}
= \operatorname{pf} {(K_m)} \det {(A)}\),
知 \(\operatorname{pf} {(K_m)} \operatorname{pf} {(K_m)}
= \operatorname{pf} {(K_m)} \operatorname{pf} {(K_m)} \det {(A)}\),
即 \(1 = 1 \det {(A)} = \det {(A)}\).

当然, 还有不少别的证明辛阵的行列式为 \(1\) 的方法,
但我就不在这儿说它们了.


\section{杂例}

\begin{example}
    设 \(A\) 是 \(n\)~级阵.
    设 \(n\)~级阵 \(D\) 适合
    \begin{align*}
        [D]_{i,j}
        = \begin{cases}
              d_i, & i = j;     \\
              0,   & \text{其他}.
          \end{cases}
    \end{align*}
    则
    \begin{align*}
             &
        \det {(A + D)}
        \\
        = {} &
        \hphantom{{} + {}}
        \det {(D)}
        \\
             &
        + \sum_{k = 1}^{n-1}
        {
        \sum_{1 \leq j_1 < j_2 < \dots < j_k \leq n}
        \det {\left(
            A\binom{j_1,\dots,j_k}{j_1,\dots,j_k}
            \right)}
        \det {(D({j_1,\dots,j_k}|{j_1,\dots,j_k}))}
        }
        \\
             &
        + \det {(A)}.
    \end{align*}

    设 \(A\) 的列~\(1\), \(2\), \(\dots\), \(n\)
    分别是 \(b_1^{(1)}\), \(b_2^{(1)}\), \(\dots\), \(b_n^{(1)}\).
    设 \(D\) 的列~\(1\), \(2\), \(\dots\), \(n\)
    分别是 \(b_1^{(0)}\), \(b_2^{(0)}\), \(\dots\), \(b_n^{(0)}\).
    则 \(A + D\) 的列~\(j\) 是
    \begin{align*}
        b_j^{(1)} + b_j^{(0)}
        = b_j^{(0)} + b_j^{(1)}
        = \sum_{c_j = 0}^{1} {b_j^{(c_j)}}.
    \end{align*}
    则, 由多线性,
    \begin{align*}
             &
        \det {(A + D)}
        \\
        = {} &
        \det {\left[
        \sum_{c_1 = 0}^{1} b_1^{(c_1)},
        \sum_{c_2 = 0}^{1} b_2^{(c_2)},
        \dots,
        \sum_{c_n = 0}^{1} b_n^{(c_n)}
        \right]}
        \\
        = {} &
        \sum_{c_1 = 0}^{1}
        \det {\left[
        b_1^{(c_1)},
        \sum_{c_2 = 0}^{1} b_2^{(c_2)},
        \dots,
        \sum_{c_n = 0}^{1} b_n^{(c_n)}
        \right]}
        \\
        = {} &
        \sum_{c_1 = 0}^{1}
        \sum_{c_2 = 0}^{1}
        \det {\left[
        b_1^{(c_1)},
        b_2^{(c_2)},
        \dots,
        \sum_{c_n = 0}^{1} b_n^{(c_n)}
        \right]}
        \\
        = {} &
        \dots \dots \dots \dots
        \dots \dots \dots \dots
        \dots \dots \dots \dots
        \\
        = {} &
        \sum_{c_1 = 0}^{1}
        \sum_{c_2 = 0}^{1}
        \dots
        \sum_{c_n = 0}^{1}
        \det {[b_1^{(c_1)}, b_2^{(c_2)}, \dots, b_n^{(c_n)}]}
        \\
        = {} &
        \sum_{0 \leq c_1, c_2, \dots, c_n \leq 1}
        \det {[b_1^{(c_1)}, b_2^{(c_2)}, \dots, b_n^{(c_n)}]}.
    \end{align*}
    由加法的结合律与交换律,
    我们可按任何方式, 任何次序求这些
    \begin{align*}
        \det {[b_1^{(c_1)}, b_2^{(c_2)}, \dots, b_n^{(c_n)}]}
    \end{align*}
    的和.
    特别地, 我们可按 \(c_1 + c_2 + \dots + c_n\) 分%
    这些数为若干组,
    求出一组的元的和 (即一组的 ``组和''),
    再求这些组的 ``组和'' 的和.
    因为 \(0 \leq c_j \leq 1\),
    故 \(0 \leq c_1 + c_2 + \dots + c_n \leq n\).
    于是, 我们可分这些数为 \(n+1\)~组:
    适合 \(c_1 + c_2 + \dots + c_n = 0\) 的项在一组;
    适合 \(c_1 + c_2 + \dots + c_n = 1\) 的项在一组;
    \(\dots \dots\);
    适合 \(c_1 + c_2 + \dots + c_n = n\) 的项在一组.
    不难看出,
    每一项一定在某一组里,
    且每一项不能在二个不同的组里.
    则
    \begin{align*}
             &
        \det {(A + D)}
        \\
        = {} &
        \sum_{0 \leq c_1, c_2, \dots, c_n \leq 1}
        \det {[b_1^{(c_1)}, b_2^{(c_2)}, \dots, b_n^{(c_n)}]}
        \\
        = {} &
        \sum_{k = 0}^{n}\,
        \sum_{\substack{0 \leq c_1, c_2, \dots, c_n \leq 1 \\
                c_1 + c_2 + \dots + c_n = k}}
        \det {[b_1^{(c_1)}, b_2^{(c_2)}, \dots, b_n^{(c_n)}]}
        \\
        = {} &
        \hphantom{{} + {}}
        \sum_{\substack{0 \leq c_1, c_2, \dots, c_n \leq 1 \\
                c_1 + c_2 + \dots + c_n = 0}}
        \det {[b_1^{(c_1)}, b_2^{(c_2)}, \dots, b_n^{(c_n)}]}
        \\
             &
        +
        \sum_{k = 1}^{n-1}\,
        \sum_{\substack{0 \leq c_1, c_2, \dots, c_n \leq 1 \\
                c_1 + c_2 + \dots + c_n = k}}
        \det {[b_1^{(c_1)}, b_2^{(c_2)}, \dots, b_n^{(c_n)}]}
        \\
             &
        +
        \sum_{\substack{0 \leq c_1, c_2, \dots, c_n \leq 1 \\
                c_1 + c_2 + \dots + c_n = n}}
        \det {[b_1^{(c_1)}, b_2^{(c_2)}, \dots, b_n^{(c_n)}]}.
    \end{align*}

    不难看出, \(c_1 + c_2 + \dots + c_n = 0\) 时,
    \(c_1 = c_2 = \dots = c_n = 0\).
    则
    \begin{align*}
        \sum_{0 \leq c_1, c_2, \dots, c_n \leq 1}
        \det {[b_1^{(c_1)}, b_2^{(c_2)}, \dots, b_n^{(c_n)}]}
        =
        \det {[b_1^{(0)}, b_2^{(0)}, \dots, b_n^{(0)}]}
        =
        \det {(D)}.
    \end{align*}

    不难看出, \(c_1 + c_2 + \dots + c_n = n\) 时,
    \(c_1 = c_2 = \dots = c_n = 1\).
    则
    \begin{align*}
        \sum_{0 \leq c_1, c_2, \dots, c_n \leq 1}
        \det {[b_1^{(c_1)}, b_2^{(c_2)}, \dots, b_n^{(c_n)}]}
        =
        \det {[b_1^{(1)}, b_2^{(1)}, \dots, b_n^{(1)}]}
        =
        \det {(A)}.
    \end{align*}

    设正整数 \(k < n\).
    由 \(c_1 + c_2 + \dots + c_n = k\),
    知在 \(c_1\), \(c_2\), \(\dots\), \(c_n\)
    中, 有 \(k\)~个 \(1\) 与 \(n-k\)~个 \(0\).
    设 \(c_{j_1} = c_{j_2} = \dots = c_{j_k} = 1\)
    (其中 \(1 \leq j_1 < j_2 < \dots < j_k \leq n\)),
    且 \(c_{j_{k+1}} = \dots = c_{j_n} = 0\)
    (其中 \(1 \leq j_{k+1} < \dots < j_n \leq n\)).
    记 \(n\)~级阵
    \begin{align*}
        B_{c_1,\dots,c_n}
        = [b_1^{(c_1)}, b_2^{(c_2)}, \dots, b_n^{(c_n)}].
    \end{align*}
    注意到, 当 \(\ell > k\),
    且 \(i \neq j_\ell\) 时,
    \([B_{c_1,\dots,c_n}]_{i,j_\ell} = 0\),
    且 \([B_{c_1,\dots,c_n}]_{j_\ell,j_\ell} = d_{j_\ell}\),
    按列~\(j_{k+1}\) 展开 \(\det {(B_{c_1,\dots,c_n})}\),
    有
    \begin{align*}
        \det {(B_{c_1,\dots,c_n})}
        = {} & (-1)^{j_{k+1}+j_{k+1}}
        [B_{c_1,\dots,c_n}]_{j_\ell,j_\ell}
        \det {(B_{c_1,\dots,c_n}(j_{k+1}|j_{k+1}))}
        \\
        = {} &
        d_{j_{k+1}} \det {(B_{c_1,\dots,c_n}(j_{k+1}|j_{k+1}))}.
    \end{align*}
    按列~\(j_{k+2}-1\) 展开
    \(\det {(B_{c_1,\dots,c_n}(j_{k+1}|j_{k+1}))}\),
    有
    \begin{align*}
             &
        \det {(B_{c_1,\dots,c_n}(j_{k+1}|j_{k+1}))}
        \\
        = {} & (-1)^{j_{k+2}-1+j_{k+2}-1}
        [B_{c_1,\dots,c_n}]_{j_\ell,j_\ell}
        \det {(B_{c_1,\dots,c_n}
        ({j_{k+1},j_{k+2}}|{j_{k+1},j_{k+2}}))}
        \\
        = {} &
        d_{j_{k+2}}
        \det {(B_{c_1,\dots,c_n}
        ({j_{k+1},j_{k+2}}|{j_{k+1},j_{k+2}}))}.
    \end{align*}
    故
    \begin{align*}
        \det {(B_{c_1,\dots,c_n})}
        = d_{j_{k+1}} d_{j_{k+2}}
        \det {(B_{c_1,\dots,c_n}
        ({j_{k+1},j_{k+2}}|{j_{k+1},j_{k+2}}))}.
    \end{align*}
    \(\dots \dots\)
    最后, 我们算出
    \begin{align*}
             &
        \det {(B_{c_1,\dots,c_n})}
        \\
        = {} &
        d_{j_{k+1}} d_{j_{k+2}} \dots d_{j_n}
        \det {(B_{c_1,\dots,c_n}
        ({j_{k+1},\dots,j_n}|{j_{k+1},\dots,j_n}))}
        \\
        = {} &
        \det {(B_{c_1,\dots,c_n}
        ({j_{k+1},\dots,j_n}|{j_{k+1},\dots,j_n}))}
        \,
        d_{j_{k+1}} d_{j_{k+2}} \dots d_{j_n}
        \\
        = {} &
        \det {\left(
            A\binom{j_1,\dots,j_k}{j_1,\dots,j_k}
            \right)}
        \det {(D({j_1,\dots,j_k}|{j_1,\dots,j_k}))}.
    \end{align*}
    于是
    \begin{align*}
             &
        \sum_{\substack{0 \leq c_1, c_2, \dots, c_n \leq 1 \\
                c_1 + c_2 + \dots + c_n = k}}
        \det {[b_1^{(c_1)}, b_2^{(c_2)}, \dots, b_n^{(c_n)}]}
        \\
        = {} &
        \sum_{1 \leq j_1 < j_2 < \dots < j_k \leq n}
        \det {\left(
            A\binom{j_1,\dots,j_k}{j_1,\dots,j_k}
            \right)}
        \det {(D({j_1,\dots,j_k}|{j_1,\dots,j_k}))}.
    \end{align*}

    综上,
    \begin{align*}
             &
        \det {(A + D)}
        \\
        = {} &
        \hphantom{{} + {}}
        \sum_{\substack{0 \leq c_1, c_2, \dots, c_n \leq 1 \\
                c_1 + c_2 + \dots + c_n = 0}}
        \det {[b_1^{(c_1)}, b_2^{(c_2)}, \dots, b_n^{(c_n)}]}
        \\
             &
        +
        \sum_{k = 1}^{n-1}\,
        \sum_{\substack{0 \leq c_1, c_2, \dots, c_n \leq 1 \\
                c_1 + c_2 + \dots + c_n = k}}
        \det {[b_1^{(c_1)}, b_2^{(c_2)}, \dots, b_n^{(c_n)}]}
        \\
             &
        +
        \sum_{\substack{0 \leq c_1, c_2, \dots, c_n \leq 1 \\
                c_1 + c_2 + \dots + c_n = n}}
        \det {[b_1^{(c_1)}, b_2^{(c_2)}, \dots, b_n^{(c_n)}]}
        \\
        = {} &
        \hphantom{{} + {}}
        \det {(D)}
        \\
             &
        + \sum_{k = 1}^{n-1}
        {
        \sum_{1 \leq j_1 < j_2 < \dots < j_k \leq n}
        \det {\left(
            A\binom{j_1,\dots,j_k}{j_1,\dots,j_k}
            \right)}
        \det {(D({j_1,\dots,j_k}|{j_1,\dots,j_k}))}
        }
        \\
             &
        + \det {(A)}.
    \end{align*}
\end{example}

\begin{example}
    设 \(A\) 是 \(n\)~级阵.
    设 \(x\) 是数.
    则
    \begin{align*}
        %  &
        \det {(xI_n + A)}
        % \\
        =
        % {} &
        x^n
        + \sum_{k = 1}^{n}
        {x^{n-k}
        \sum_{1 \leq j_1 < \dots < j_k \leq n}
        \det {\left(
            A\binom{j_1,\dots,j_k}{j_1,\dots,j_k}
            \right)}
        }.
    \end{align*}

    注意到
    \begin{align*}
        [xI_n]_{i,j}
        = \begin{cases}
              x, & i = j;     \\
              0, & \text{其他}.
          \end{cases}
    \end{align*}
    故, 由上个例,
    \begin{align*}
             &
        \det {(xI_n + A)}
        \\
        = {} &
        \hphantom{{} + {}}
        \det {(xI_n)}
        \\
             &
        + \sum_{k = 1}^{n-1}
        {
        \sum_{1 \leq j_1 < \dots < j_k \leq n}
        \det {\left(
            A\binom{j_1,\dots,j_k}{j_1,\dots,j_k}
            \right)}
        \det {((xI_n)({j_1,\dots,j_k}|{j_1,\dots,j_k}))}
        }
        \\
             &
        + \det {(A)}
        \\
        = {} &
        \hphantom{{} + {}}
        x^n
        % \\
        %      &
        + \sum_{k = 1}^{n-1}
        {
        \sum_{1 \leq j_1 < \dots < j_k \leq n}
        \det {\left(
            A\binom{j_1,\dots,j_k}{j_1,\dots,j_k}
            \right)}
        \,
        x^{n-k}
        }
        \\
             &
        + \det {(A)}
        \\
        = {} &
        \hphantom{{} + {}}
        x^n
        % \\
        %      &
        + \sum_{k = 1}^{n-1}
        {
        x^{n-k}
        \sum_{1 \leq j_1 < \dots < j_k \leq n}
        \det {\left(
            A\binom{j_1,\dots,j_k}{j_1,\dots,j_k}
            \right)}
        }
        \\
             &
        + x^{n-n}
        \sum_{1 \leq j_1 < \dots < j_n \leq n}
        \det {\left(
            A\binom{j_1,\dots,j_n}{j_1,\dots,j_n}
            \right)}
        \\
        = {} &
        x^n
        + \sum_{k = 1}^{n}
        {x^{n-k}
        \sum_{1 \leq j_1 < \dots < j_k \leq n}
        \det {\left(
            A\binom{j_1,\dots,j_k}{j_1,\dots,j_k}
            \right)}
        }.
    \end{align*}
\end{example}

\begin{example}
    设正整数 \(m\), \(n\) 适合 \(m \geq n\).
    设 \(A\), \(B\) 分别是 \(m \times n\) 与 \(n \times m\) 阵.
    设正整数 \(k \leq n\).
    由 Binet--Cauchy 公式的推广,
    \begin{align*}
             &
        \sum_{1 \leq j_1 < \dots < j_k \leq m}
        \det {\left(
            (AB)\binom{j_1,\dots,j_k}{j_1,\dots,j_k}
            \right)}
        \\
        = {} &
        \sum_{1 \leq j_1 < \dots < j_k \leq m}
        \sum_{1 \leq i_1 < \dots < i_k \leq n}
        \det {\left(
            A\binom{j_1,\dots,j_k}{i_1,\dots,i_k}
            \right)}
        \det {\left(
            B\binom{i_1,\dots,i_k}{j_1,\dots,j_k}
            \right)}
        \\
        = {} &
        \sum_{1 \leq j_1 < \dots < j_k \leq m}
        \sum_{1 \leq i_1 < \dots < i_k \leq n}
        \det {\left(
            B\binom{i_1,\dots,i_k}{j_1,\dots,j_k}
            \right)}
        \det {\left(
            A\binom{j_1,\dots,j_k}{i_1,\dots,i_k}
            \right)}
        \\
        = {} &
        \sum_{1 \leq i_1 < \dots < i_k \leq n}
        \sum_{1 \leq j_1 < \dots < j_k \leq m}
        \det {\left(
            B\binom{i_1,\dots,i_k}{j_1,\dots,j_k}
            \right)}
        \det {\left(
            A\binom{j_1,\dots,j_k}{i_1,\dots,i_k}
            \right)}
        \\
        = {} &
        \sum_{1 \leq i_1 < \dots < i_k \leq n}
        \det {\left(
            (BA)\binom{i_1,\dots,i_k}{i_1,\dots,i_k}
            \right)}.
    \end{align*}
\end{example}

\begin{example}
    设正整数 \(m\), \(n\) 适合 \(m \geq n\).
    设 \(A\), \(B\) 分别是 \(m \times n\) 与 \(n \times m\) 阵.
    设 \(x\) 是数.
    则
    \begin{align*}
             &
        \det {(xI_m + AB)}
        \\
        = {} &
        x^m
        + \sum_{k = 1}^{m}
        {x^{m-k}
        \sum_{1 \leq j_1 < \dots < j_k \leq m}
        \det {\left(
            (AB)\binom{j_1,\dots,j_k}{j_1,\dots,j_k}
            \right)}
        }
        \\
        = {} &
        \hphantom{{} + {}}
        x^m
        + \sum_{k = 1}^{n}
        {x^{m-k}
        \sum_{1 \leq j_1 < \dots < j_k \leq m}
        \det {\left(
            (AB)\binom{j_1,\dots,j_k}{j_1,\dots,j_k}
            \right)}
        }
        \\
             &
        + \sum_{k = n+1}^{m}
        {x^{m-k}
        \sum_{1 \leq j_1 < \dots < j_k \leq m}
        \det {\left(
            (AB)\binom{j_1,\dots,j_k}{j_1,\dots,j_k}
            \right)}
        }
        \\
        = {} &
        \hphantom{{} + {}}
        x^m
        + \sum_{k = 1}^{n}
        {x^{m-k}
        \sum_{1 \leq j_1 < \dots < j_k \leq m}
        \det {\left(
            (AB)\binom{j_1,\dots,j_k}{j_1,\dots,j_k}
            \right)}
        }
        \\
             &
        + \sum_{k = n+1}^{m}
        {x^{m-k}
        \sum_{1 \leq j_1 < \dots < j_k \leq m}
        0
        }
        \\
        = {} &
        x^m
        + \sum_{k = 1}^{n}
        {x^{m-k}
        \sum_{1 \leq j_1 < \dots < j_k \leq m}
        \det {\left(
            (AB)\binom{j_1,\dots,j_k}{j_1,\dots,j_k}
            \right)}
        }
        \\
        = {} &
        x^{m-n} x^n
        + \sum_{k = 1}^{n}
        {x^{m-n} x^{n-k}
        \sum_{1 \leq i_1 < \dots < i_k \leq n}
        \det {\left(
            (BA)\binom{i_1,\dots,i_k}{i_1,\dots,i_k}
            \right)}
        }
        \\
        = {} &
        x^{m-n} x^n
        + x^{m-n} \sum_{k = 1}^{n}
        {x^{n-k}
        \sum_{1 \leq i_1 < \dots < i_k \leq n}
        \det {\left(
            (BA)\binom{i_1,\dots,i_k}{i_1,\dots,i_k}
            \right)}
        }
        \\
        = {} &
        x^{m-n}
        \left(
        x^n
        + \sum_{k = 1}^{n}
        {x^{n-k}
            \sum_{1 \leq i_1 < \dots < i_k \leq n}
            \det {\left(
                (BA)\binom{i_1,\dots,i_k}{i_1,\dots,i_k}
                \right)}
        }
        \right)
        \\
        = {} &
        x^{m-n} \det {(xI_n + BA)}.
    \end{align*}
    特别地, 代 \(x\) 以数 \(1\), 有
    \begin{align*}
        \det {(I_m + AB)} = \det {(I_n + BA)}.
    \end{align*}
\end{example}

\begin{example}
    设 \(a_1\), \(b_1\), \(a_2\), \(b_2\),
    \(\dots\), \(a_m\), \(b_m\) 是 \(2m\)~个数.
    作 \(m\)~级阵 \(C\) 如下:
    \begin{align*}
        [C]_{i,j} =
        \begin{cases}
            1 + a_i b_i, & i = j;     \\
            a_i b_j,     & \text{其他}.
        \end{cases}
    \end{align*}
    我们计算 \(\det {(C)}\).

    设 \(A = [a_1, a_2, \dots, a_m]^{\mathrm{T}}\),
    \(B = [b_1, b_2, \dots, b_m]\).
    则
    \begin{align*}
        [AB]_{i,j} = [A]_{i,1} [B]_{1,j} = a_i b_j.
    \end{align*}
    由此, 不难看出, \(C = I_m + AB\).
    故
    \begin{align*}
        \det {(C)}
        = {} & \det {(I_m + AB)}                        \\
        = {} & \det {(I_1 + BA)}                        \\
        = {} & [I_1 + BA]_{1,1}                         \\
        = {} & [I_1]_{1,1} + [BA]_{1,1}                 \\
        = {} & 1 + b_1 a_1 + b_2 a_2 + \dots + b_m a_m.
    \end{align*}

    我们当然也可用别的方法.
    作 \(m+1\)~级阵 \(G\) 如下:
    \begin{align*}
        [G]_{i,j} =
        \begin{cases}
            1,         & i = j = m+1; \\
            -a_i,      & i < j = m+1; \\
            0,         & m+1 = i > j; \\
            [C]_{i,j}, & \text{其他}.
        \end{cases}
    \end{align*}
    形象地,
    \begin{align*}
        G =
        \begin{bmatrix}
            1 + a_1 b_1 & a_1 b_2     & \cdots & a_1 b_{m-1}         & a_1 b_m     & -a_1     \\
            a_2 b_1     & 1 + a_2 b_2 & \cdots & a_2 b_{m-1}         & a_2 b_m     & -a_2     \\
            \vdots      & \vdots      & {}     & \vdots              & \vdots      & \vdots   \\
            a_{m-1} b_1 & a_{m-1} b_2 & \cdots & 1 + a_{m-1} b_{m-1} & a_{m-1} b_m & -a_{m-1} \\
            a_m b_1     & a_m b_2     & \cdots & a_m b_{m-1}         & 1 + a_m b_m & -a_m     \\
            0           & 0           & \cdots & 0                   & 0           & 1        \\
        \end{bmatrix}.
    \end{align*}
    一方面, 按行~\(m+1\) 展开, 有
    \begin{align*}
        \det {(G)}
        = (-1)^{m+1+m+1}\,1\,\det {(G(1|1))}
        = \det {(C)}.
    \end{align*}
    另一方面, 我们%
    加列~\(m+1\) 的 \(b_1\)~倍于列~\(1\),
    加列~\(m+1\) 的 \(b_2\)~倍于列~\(2\),
    \(\dots \dots\),
    加列~\(m+1\) 的 \(b_m\)~倍于列~\(m\),
    得 \(m+1\)~级阵
    \begin{align*}
        H =
        \begin{bmatrix}
            1      & 0      & \cdots & 0       & 0      & -a_1     \\
            0      & 1      & \cdots & 0       & 0      & -a_2     \\
            \vdots & \vdots & {}     & \vdots  & \vdots & \vdots   \\
            0      & 0      & \cdots & 1       & 0      & -a_{m-1} \\
            0      & 0      & \cdots & 0       & 1      & -a_m     \\
            b_1    & b_2    & \cdots & b_{m-1} & b_m    & 1        \\
        \end{bmatrix}.
    \end{align*}
    则 \(\det {(H)} = \det {(G)}\).
    故 \(\det {(C)} = \det {(H)}\).

    我们%
    加列~\(1\) 的 \(a_1\)~倍于列~\(m+1\),
    加列~\(1\) 的 \(a_2\)~倍于列~\(m+1\),
    \(\dots \dots\),
    加列~\(1\) 的 \(a_m\)~倍于列~\(m+1\),
    得 \(m+1\)~级阵
    \begin{align*}
        J =
        \begin{bmatrix}
            1      & 0      & \cdots & 0       & 0      & 0                                       \\
            0      & 1      & \cdots & 0       & 0      & 0                                       \\
            \vdots & \vdots & {}     & \vdots  & \vdots & \vdots                                  \\
            0      & 0      & \cdots & 1       & 0      & 0                                       \\
            0      & 0      & \cdots & 0       & 1      & 0                                       \\
            b_1    & b_2    & \cdots & b_{m-1} & b_m    & 1 + b_1 a_1 + b_2 a_2 + \dots + b_m a_m \\
        \end{bmatrix}.
    \end{align*}
    则 \(\det {(J)} = \det {(H)}\).
    故
    \begin{align*}
        \det {(C)} = \det {(J)}
        = 1 + b_1 a_1 + b_2 a_2 + \dots + b_m a_m.
    \end{align*}
\end{example}

\begin{example}
    设 \(n\), \(m\) 是非负整数,
    且 \(m + n \geq 1\).
    设
    \begin{align*}
        f(x) &
        = \sum_{k = 0}^{n} {a_k x^{n-k}}
        = a_0 x^n + a_1 x^{n-1} + \dots + a_n,
        \\
        g(x) &
        = \sum_{k = 0}^{m} {b_k x^{m-k}}
        = b_0 x^m + b_1 x^{m-1} + \dots + b_m.
    \end{align*}
    为方便, 我们约定:
    若 \(k < 0\) 或 \(k > n\), 则 \(a_k = 0\);
    若 \(k < 0\) 或 \(k > m\), 则 \(b_k = 0\).
    作 \(m + n\)~级阵 \(R\) 如下:
    \begin{align*}
        [R]_{i,j}
        = \begin{cases}
              a_{i-j},     & j \leq m; \\
              b_{i-(j-m)}, & j > m.
          \end{cases}
    \end{align*}
    形象地, 比如, 若 \(n = 2\), \(m = 5\), 则
    \begin{align*}
        R
        =
        \begin{bmatrix}
            a_{0} & a_{-1} & a_{-2} & a_{-3} & a_{-4} & b_{0} & b_{-1} \\
            a_{1} & a_{0}  & a_{-1} & a_{-2} & a_{-3} & b_{1} & b_{0}  \\
            a_{2} & a_{1}  & a_{0}  & a_{-1} & a_{-2} & b_{2} & b_{1}  \\
            a_{3} & a_{2}  & a_{1}  & a_{0}  & a_{-1} & b_{3} & b_{2}  \\
            a_{4} & a_{3}  & a_{2}  & a_{1}  & a_{0}  & b_{4} & b_{3}  \\
            a_{5} & a_{4}  & a_{3}  & a_{2}  & a_{1}  & b_{5} & b_{4}  \\
            a_{6} & a_{5}  & a_{4}  & a_{3}  & a_{2}  & b_{6} & b_{5}  \\
        \end{bmatrix}
        =
        \begin{bmatrix}
            a_0 & 0   & 0   & 0   & 0   & b_0 & 0   \\
            a_1 & a_0 & 0   & 0   & 0   & b_1 & b_0 \\
            a_2 & a_1 & a_0 & 0   & 0   & b_2 & b_1 \\
            0   & a_2 & a_1 & a_0 & 0   & b_3 & b_2 \\
            0   & 0   & a_2 & a_1 & a_0 & b_4 & b_3 \\
            0   & 0   & 0   & a_2 & a_1 & b_5 & b_4 \\
            0   & 0   & 0   & 0   & a_2 & 0   & b_5 \\
        \end{bmatrix}.
    \end{align*}

    (1)
    设 \(e\) 是 \(m + n\)~级单位阵的列~\(m + n\).
    我们说, 存在
    % 我们说, 存在非零的
    \((m + n) \times 1\)~阵 \(p\) 使
    \(Rp = \det {(R)}\, e\).
    % 若 \(\det {(R)} = 0\),
    % 则 \(\det {(R)}\, e = 0\).
    % 我们知道, 存在非零的 \((m + n) \times 1\)~阵 \(p\) 使
    % \(Rp = 0 = \det {(R)}\, e\).
    % 若 \(\det {(R)} \neq 0\),
    % 我们可%

    取 \(p\) 为 \(R\) 的古伴
    \(\operatorname{adj} {(R)}\) 的列~\(m + n\).
    则
    \begin{align*}
        [Rp]_{i,1}
        = {} &
        \sum_{\ell = 1}^{m + n}
        {[R]_{i,\ell} [p]_{\ell,1}}
        \\
        = {} &
        \sum_{\ell = 1}^{m + n}
        {[R]_{i,\ell} [\operatorname{adj} {(R)}]_{\ell,m+n}}
        \\
        = {} &
        [R \operatorname{adj} {(R)}]_{i,m+n}
        \\
        = {} &
        [\det {(R)}\, I_{m+n}]_{i,m+n}
        \\
        = {} &
        \det {(R)}\, [I_{m+n}]_{i,m+n}
        \\
        = {} &
        \det {(R)}\, [e]_{i,1}
        \\
        = {} &
        [\det {(R)}\, e]_{i,1}.
    \end{align*}
    (顺便一提, 若 \(\det {(R)} \neq 0\), 显然有 \(p \neq 0\);
    若 \(\det {(R)} = 0\),
    由第一章, 节~\malneprasekcio{23} 的知识,
    存在非零的 \((m + n) \times 1\)~阵 \(q\)
    使 \(Rq = 0 = \det {(R)}\, e\).)

    % (2)
    % 设
    % \begin{align*}
    %     c_k =
    %     \begin{cases}
    %         [p]_{k+1,1}, & 0 \leq k \leq m - 1; \\
    %         0,           & \text{其他}.
    %     \end{cases}
    % \end{align*}
    % 再设
    % \begin{align*}
    %     d_k =
    %     \begin{cases}
    %         [p]_{k+m+1,1}, & 0 \leq k \leq n - 1; \\
    %         0,             & \text{其他}.
    %     \end{cases}
    % \end{align*}
    % 不难看出
    % \begin{align*}
    %     [p]_{i,1} =
    %     \begin{cases}
    %         c_{i-1},   & i \leq m; \\
    %         d_{i-m-1}, & i > m.
    %     \end{cases}
    % \end{align*}
    % 记
    % \begin{align*}
    %     u(x) &
    %     = \sum_{k = 0}^{m-1} {c_k x^{m-1-k}}
    %     = c_0 x^{m-1} + c_1 x^{m-2} + \dots + c_{m-1},
    %     \\
    %     v(x) &
    %     = \sum_{k = 0}^{n-1} {d_k x^{n-1-k}}
    %     = d_0 x^{n-1} + d_1 x^{n-2} + \dots + d_{n-1}.
    % \end{align*}
    % % 显然, \(u(x)\) 与 \(v(x)\) 不全是零.
    % 我们说, \(f(x)\, u(x) + g(x)\, v(x) = \det {(R)}\).

    % 首先, 不难算出
    % \begin{align*}
    %     f(x)\, u(x) &
    %     = \sum_{k = 0}^{n+m-1}
    %     {\left(
    %     \sum_{\ell = 0}^{k} {a_{k-\ell} c_{\ell}}
    %     \right)
    %     x^{n+m-1-k}},
    %     \\
    %     g(x)\, v(x) &
    %     = \sum_{k = 0}^{m+n-1}
    %     {\left(
    %     \sum_{\ell = 0}^{k} {b_{k-\ell} d_{\ell}}
    %     \right)
    %     x^{m+n-1-k}}.
    % \end{align*}
    % 故
    % \begin{align*}
    %     f(x)\, u(x) + g(x)\, v(x)
    %     = \sum_{k = 0}^{m+n-1}
    %     {\left(
    %     \sum_{\ell = 0}^{k} {a_{k-\ell} c_{\ell}}
    %     +
    %     \sum_{\ell = 0}^{k} {b_{k-\ell} d_{\ell}}
    %     \right)
    %     x^{m+n-1-k}}.
    % \end{align*}
    % 注意到, 当 \(0 \leq k < m + n\) 时,
    % \begin{align*}
    %     [Rp]_{k+1,1}
    %     = {} &
    %     \sum_{h = 1}^{m + n}
    %     {[R]_{k+1,h} [p]_{h,1}}
    %     \\
    %     = {} &
    %     \sum_{h = 1}^{m}
    %     {[R]_{k+1,h} [p]_{h,1}}
    %     +
    %     \sum_{h = m + 1}^{m + n}
    %     {[R]_{k+1,h} [p]_{h,1}}
    %     \\
    %     = {} &
    %     \sum_{h = 1}^{m}
    %     {a_{k+1-h} c_{h-1}}
    %     +
    %     \sum_{h = m + 1}^{m + n}
    %     {b_{k+1-(h-m)} d_{h-m-1}}
    %     \\
    %     = {} &
    %     \sum_{h = 1}^{m}
    %     {a_{k-(h-1)} c_{h-1}}
    %     +
    %     \sum_{h = m + 1}^{m + n}
    %     {b_{k-(h-m-1)} d_{h-m-1}}
    %     \\
    %     = {} &
    %     \sum_{\ell = 0}^{m-1}
    %     {a_{k-\ell} c_{\ell}}
    %     +
    %     \sum_{\ell = 0}^{n-1}
    %     {b_{k-\ell} d_{\ell}}.
    % \end{align*}
    % 若 \(k \leq m - 1\), 则 \(\ell > k\) 时, 必 \(a_{k-\ell} = 0\);
    % 若 \(k > m - 1\), 则 \(\ell > m - 1\) 时, 必 \(c_\ell = 0\).
    % 所以
    % \begin{align*}
    %     \sum_{\ell = 0}^{m-1}
    %     {a_{k-\ell} c_{\ell}}
    %     =
    %     \sum_{\ell = 0}^{k}
    %     {a_{k-\ell} c_{\ell}}.
    % \end{align*}
    % 类似地,
    % \begin{align*}
    %     \sum_{\ell = 0}^{n-1}
    %     {b_{k-\ell} d_{\ell}}
    %     =
    %     \sum_{\ell = 0}^{k}
    %     {b_{k-\ell} d_{\ell}}.
    % \end{align*}
    % 故
    % \begin{align*}
    %     %  &
    %     f(x)\, u(x) + g(x)\, v(x)
    %     % \\
    %     = {} &
    %     \sum_{k = 0}^{m+n-1}
    %     {[Rp]_{k+1,1}\,
    %     x^{m+n-1-k}}
    %     \\
    %     = {} &
    %     \sum_{k = 0}^{m+n-1}
    %     {[\det {(R)}\, e]_{k+1,1}\,
    %     x^{m+n-1-k}}
    %     \\
    %     = {} &
    %     \sum_{k = 0}^{m+n-1}
    %     {\det {(R)}\, [e]_{k+1,1}\,
    %     x^{m+n-1-k}}
    %     \\
    %     = {} &
    %     \det {(R)}\, [e]_{m+n-1+1,1}\,
    %     x^{m+n-1-(m+n-1)}
    %     \\
    %     = {} &
    %     \det {(R)}.
    % \end{align*}

    % (3)
    % 设数 \(a\) 适合 \(f(a) = 0 = g(a)\).
    % 由 (2) 知, 必
    % \(0 = f(a)\, u(a) + g(a)\, v(a) = \det {(R)}\)
    % (我们代 \(x\) 以数 \(a\)).
    % 这有时是有用的.

    (2)
    作 \(1 \times (m + n)\)~阵
    \(X = [x^{m+n-1}, x^{m+n-2}, \dots, 1]\);
    具体地,
    \([X]_{1,j} = x^{m+n-j}\).
    则 \(XR\) 是 \(1 \times (m + n)\)~阵,
    且
    \begin{align*}
        [XR]_{1,j}
        =
        \sum_{\ell = 1}^{m+n}
        {[X]_{1,\ell} [R]_{\ell,j}}
        =
        \sum_{1 \leq \ell \leq m+n}
        {[R]_{\ell,j} x^{m+n-\ell}}.
    \end{align*}
    若 \(j \leq m\), 则 \([R]_{\ell,j} = a_{\ell-j}\), 且
    \begin{align*}
        [XR]_{1,j}
        = {} &
        \sum_{1 \leq \ell \leq m+n}
        {a_{\ell-j} x^{m+n-\ell}}
        \\
        = {} &
        \sum_{1-j \leq \ell-j \leq m+n-j}
        {a_{\ell-j} x^{n-(\ell-j)+(m-j)}}
        \\
        = {} &
        \sum_{1-j \leq h \leq n+(m-j)}
        {a_h x^{n-h}\, x^{m-j}}
        \\
        = {} &
        \sum_{0 \leq h \leq n}
        {a_h x^{n-h}\, x^{m-j}}
        +
        \sum_{\substack{
        1-j \leq h < 0 \\
                \text{或}\, n < h \leq n+(m-j)
            }}
        {a_h x^{n-h}\, x^{m-j}}
        \\
        = {} &
        \Bigg( \sum_{0 \leq h \leq n}
        {a_h x^{n-h}} \Bigg) x^{m-j}
        +
        \sum_{\substack{
        1-j \leq h < 0 \\
                \text{或}\, n < h \leq n+(m-j)
            }}
        {0\, x^{n-h}\, x^{m-j}}
        \\
        = {} &
        f(x)\, x^{m-j}.
    \end{align*}
    若 \(j > m\), 则 \([R]_{\ell,j} = b_{\ell-(j-m)}\).
    为方便, 记 \(k = j - m\).
    则
    \begin{align*}
        [XR]_{1,j}
        = {} &
        \sum_{1 \leq \ell \leq m+n}
        {b_{\ell-k} x^{m+n-\ell}}
        \\
        = {} &
        \sum_{1-k \leq \ell-k \leq m+n-k}
        {b_{\ell-k} x^{m-(\ell-k)+(n-k)}}
        \\
        = {} &
        \sum_{1-k \leq h \leq m+(n-k)}
        {b_h x^{m-h}\, x^{n-k}}
        \\
        = {} &
        \sum_{0 \leq h \leq m}
        {b_h x^{m-h}\, x^{n-k}}
        +
        \sum_{\substack{
        1-k \leq h < 0 \\
                \text{或}\, m < h \leq m+(n-k)
            }}
        {b_h x^{m-h}\, x^{n-k}}
        \\
        = {} &
        \Bigg( \sum_{0 \leq h \leq m}
        {b_h x^{m-h}} \Bigg) x^{n-(j-m)}
        +
        \sum_{\substack{
        1-k \leq h < 0 \\
                \text{或}\, m < h \leq m+(n-k)
            }}
        {0\, x^{m-h}\, x^{n-k}}
        \\
        = {} &
        g(x)\, x^{n-(j-m)}.
    \end{align*}

    % 综上, \(1 \times (m+n)\)~阵 \(XR\) 的元适合:
    % \begin{align*}
    %     [XR]_{1,j} =
    %     \begin{cases}
    %         f(x)\, x^{m-j},     & j \leq m; \\
    %         g(x)\, x^{n-(j-m)}, & j > m.
    %     \end{cases}
    % \end{align*}

    (3)
    既然 \(XR\) 是 \(1 \times (m+n)\)~阵,
    则 \((XR)p\) 是 \(1\)~级阵.
    记 \(r = [(XR)p]_{1,1}\).
    则
    \begin{align*}
        r
        = {} &
        \sum_{j = 1}^{m+n}
        {[XR]_{1,j} [p]_{j,1}}
        \\
        = {} &
        \sum_{j = 1}^{m}
        {[XR]_{1,j} [p]_{j,1}}
        +
        \sum_{j = m+1}^{m+n}
        {[XR]_{1,j} [p]_{j,1}}
        \\
        = {} &
        \sum_{j = 1}^{m}
        {f(x)\, x^{m-j}\, [p]_{j,1}}
        +
        \sum_{j = m+1}^{m+n}
        {g(x)\, x^{n-(j-m)}\, [p]_{j,1}}
        \\
        = {} &
        f(x)
        \sum_{j = 1}^{m}
        {x^{m-j}\, [p]_{j,1}}
        +
        g(x)
        \sum_{j = m+1}^{m+n}
        {x^{n-(j-m)}\, [p]_{j,1}}
        \\
        = {} &
        f(x)
        \sum_{k = 0}^{m-1}
        {[p]_{k+1,1} x^{m-1-k}}
        +
        g(x)
        \sum_{k = 0}^{n-1}
        {[p]_{m+1+k,1} x^{n-1-k}}.
    \end{align*}
    记
    \begin{align*}
        u(x)
        = \sum_{k = 0}^{m-1} {[p]_{k+1,1} x^{m-1-k}}
        = {} &
        [p]_{1,1} x^{m-1} + [p]_{2,1} x^{m-2}
        + \dots + [p]_{m,1},
        \\
        v(x)
        = \sum_{k = 0}^{n-1} {[p]_{m+1+k,1} x^{n-1-k}}
        = {} &
        [p]_{m+1,1} x^{n-1} + [p]_{m+2,1} x^{n-2}
        + \dots + [p]_{m+n,1}.
    \end{align*}
    则
    \(f(x)\, u(x) + g(x)\, v(x) = r\).
    另一方面,
    \begin{align*}
        r
        = {} &
        [(XR)p]_{1,1}
        % \\
            =
            % {} &
            [X(Rp)]_{1,1}
        \\
        = {} &
        [X(\det {(R)}\, e)]_{1,1}
        \\
        = {} &
        \sum_{j = 1}^{m + n}
        {[X]_{1,j} [\det {(R)}\, e]_{j,1}}
        \\
        = {} &
        [X]_{1,m+n} [\det {(R)}\, e]_{m+n,1}
        \\
        = {} &
        1 \det {(R)}
        =
        \det {(R)}.
    \end{align*}
    故
    \(f(x)\, u(x) + g(x)\, v(x) = \det {(R)}\).

    (4)
    设数 \(c\) 适合 \(f(c) = 0 = g(c)\).
    由 (3) 知, 必
    \(0 = f(c)\, u(c) + g(c)\, v(c) = \det {(R)}\)
    (我们代 \(x\) 以数 \(c\)).
    这有时是有用的.
\end{example}

\begin{example}
    解方程组
    \begin{equation}
        \begin{cases}
            11x^2 - 2xy - 44y^2 - x + 26y - 32 = 0, \\
            4x^2 - 18xy + 49y^2 - 4x + 9y - 118 = 0.
        \end{cases}
        \label{eq:C3801}
    \end{equation}

    我们可用上例的结果解方程组~\eqref{eq:C3801}.
    为此, 我们记
    \begin{align*}
        f_y (x) & = 11 x^2 + (-2 y - 1) x + (-44 y^2 + 26 y - 32), \\
        g_y (x) & = 4 x^2 + (-18 y - 4) x + (49 y^2 + 9 y - 118).
    \end{align*}
    设 \(x = a\), \(y = b\) 是%
    方程组~\eqref{eq:C3801} 的一个解.
    则
    \begin{align*}
         &
        \begin{aligned}
            0
            = {} &
            11a^2 - 2ab - 44b^2 - a + 26b - 32
            \\
            = {} &
            11 a^2 + (-2 b - 1) a + (-44 b^2 + 26 b - 32),
        \end{aligned}
        \\
         &
        \begin{aligned}
            0
            = {} &
            4a^2 - 18ab + 49b^2 - 4a + 9b - 118
            \\
            = {} &
            4 a^2 + (-18 b - 4) a + (49 b^2 + 9 b - 118),
        \end{aligned}
    \end{align*}
    即 \(f_b (a) = 0 = g_b (a)\).
    故
    \begin{align*}
        \begin{bmatrix}
            11            & 0             & 4            & 0            \\
            -2b-1         & 11            & -18b-4       & 4            \\
            -44b^2+26b-32 & -2b-1         & 49b^2+9b-118 & -18b-4       \\
            0             & -44b^2+26b-32 & 0            & 49b^2+9b-118 \\
        \end{bmatrix}
    \end{align*}
    的行列式是 \(0\).
    另一方面, 可以算出, 此阵的行列式是
    \(342\,125\, (b^2 - 1)\, (b^2 - 4)\).
    于是, 若 \(x = a\), \(y = b\) 是%
    方程组~\eqref{eq:C3801} 的一个解,
    则 \(b = 1\) 或 \(b = -1\)
    或 \(b = 2\) 或 \(b = -2\).
    则 (代 \(b\) 以 \(1\))
    \begin{equation}
        \begin{cases}
            11a^2 - 2a - 44 - a + 26 - 32 = 0, \\
            4a^2 - 18a + 49 - 4a + 9 - 118 = 0;
        \end{cases}
        \label{eq:C3802}
    \end{equation}
    或 (代 \(b\) 以 \(-1\))
    \begin{equation}
        \begin{cases}
            11a^2 + 2a - 44 - a - 26 - 32 = 0, \\
            4a^2 + 18a + 49 - 4a - 9 - 118 = 0;
        \end{cases}
        \label{eq:C3803}
    \end{equation}
    或 (代 \(b\) 以 \(2\))
    \begin{equation}
        \begin{cases}
            11a^2 - 4a - 176 - a + 52 - 32 = 0, \\
            4a^2 - 36a + 196 - 4a + 18 - 118 = 0;
        \end{cases}
        \label{eq:C3804}
    \end{equation}
    或 (代 \(b\) 以 \(-2\))
    \begin{equation}
        \begin{cases}
            11a^2 + 4a - 176 - a - 52 - 32 = 0, \\
            4a^2 + 36a + 196 - 4a - 18 - 118 = 0.
        \end{cases}
        \label{eq:C3805}
    \end{equation}
    方程组~\eqref{eq:C3802} 的解是 \(a = -2\);
    方程组~\eqref{eq:C3803} 的解是 \(a = 3\);
    方程组~\eqref{eq:C3804} 的解是 \(a = 4\);
    方程组~\eqref{eq:C3805} 的解是 \(a = -5\).
    于是, 若 \(x = a\), \(y = b\) 是%
    方程组~\eqref{eq:C3801} 的一个解,
    必:
    \begin{align*}
         & a = -2, \quad b = 1;  \\
        \text{或} \quad
         & a = 3, \quad b = -1;  \\
        \text{或} \quad
         & a = 4, \quad b = 2;   \\
        \text{或} \quad
         & a = -5, \quad b = -2.
    \end{align*}
    最后, 我们验证这些是否是%
    方程组~\eqref{eq:C3801} 的解.
    可以验证, 它们都是解.
    于是,
    方程组~\eqref{eq:C3801} 的解是:
    \begin{align*}
         & x = -2, \quad y = 1;  \\
        \text{或} \quad
         & x = 3, \quad y = -1;  \\
        \text{或} \quad
         & x = 4, \quad y = 2;   \\
        \text{或} \quad
         & x = -5, \quad y = -2.
    \end{align*}
\end{example}


\section{总结}

用归纳定义定义行列式的教材%
似乎要比用组合定义定义行列式的教材少.
具体地, 为数学学生准备的教材
(即 ``高等代数'')
较少用归纳定义,
而为非数学学生准备的教材
(即 ``线性代数'')
倒是有不少用归纳定义的,
但仍不算 ``多''.

\vspace{2ex}

按多列 (行) 展开行列式是教学的一个难点.
毕竟, 我们较少用它算行列式.
我认为,
这是一个更理论的 (而不是应用的) 定理.
我想, 我要么不讲它, 要么通俗地讲它
(不要求熟悉).
所以, 我加了一个具体的例,
并希望此例可助学生理解按多列展开行列式.
证明的方法还是数学归纳法.
我用到了按 (任何) 一列展开行列式的公式.
所以, 理论地,
这个论证可被任何一本教材用
(毕竟, 一般地, 讲按多列展开行列式前,
都会讲按一列展开行列式).
不过, 我没看到这样的教材
(包括一本用归纳定义定义行列式,
但用完全展开的公式证明此事%
的高等代数教材).

\vspace{2ex}

我似乎讲完了我想讲的.
希望我的书可助您真地入门.
再见.
