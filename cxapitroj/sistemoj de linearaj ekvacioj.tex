% The following text
% was intended to be directly inserted in "determinantoj.tex".
% However, because it is too long,
% and it is not really on determinants,
% I decide to create a new file to contain it.

\section{线性方程组}

在接下来的若干节里, 我想用行列式讨论%
线性方程组的解.
这是行列式的一个应用.
或许, 您可以在这些讨论里体会到,
``行列式是一个\emph{工具}''
``行列式是方阵的一个\emph{属性}''
的意思.

本节, 我们学习一些基本的概念.

设 \(a_1\), \(a_2\), \(\dots\), \(a_n\) 是 \(n\)~个常数,
\(c\) 是常数,
\(x_1\), \(x_2\), \(\dots\), \(x_n\) 是 \(n\)~个未知数.
我们说, 形如
\begin{align*}
    a_1 x_1 + a_2 x_2 + \dots + a_n x_n + c
\end{align*}
的式是一个 \emph{\(n\)~元 \({\leq} 1\)~次式}.
如果 \(a_1\), \(a_2\), \(\dots\), \(a_n\)
中有一个数不为零,
我们说, 这个式是一个 \emph{\(n\)~元 \(1\)~次式}.
如果 \(a_1\), \(a_2\), \(\dots\), \(a_n\)
全为零,
那么, 这个式就是一个常数.

形如
\begin{align*}
    a_1 x_1 + a_2 x_2 + \dots + a_n x_n + c = 0
\end{align*}
的方程是一个 \emph{\(n\)~元 \({\leq} 1\)~次方程}.
不过, 习惯地, 我们移常数项 \(c\) 到等式的右侧.
也就是说, 形如
\begin{align*}
    a_1 x_1 + a_2 x_2 + \dots + a_n x_n = b
\end{align*}
的方程也是一个 \(n\)~元 \({\leq} 1\)~次方程,
其中 \(b\) 就是常数 \(-c\).
若我们不想强调未知数之数~\(n\),
我们说, 这是一个 \emph{\({\leq} 1\)~次方程};
我们也可说, 这是一个\emph{线性方程}.

\emph{由 \(m\)~个 \(n\)~元 \({\leq} 1\)~次方程作成的方程组},
是形如
\begin{align*}
    \begin{cases}
        a_{1,1} x_1 + a_{1,2} x_2 + \dots + a_{1,n} x_n = b_1, \\
        a_{2,1} x_1 + a_{2,2} x_2 + \dots + a_{2,n} x_n = b_2, \\
        \dots \dots \dots \dots
        \dots \dots \dots \dots
        \dots \dots \dots \dots
        ,                                                      \\
        a_{m,1} x_1 + a_{m,2} x_2 + \dots + a_{m,n} x_n = b_m  \\
    \end{cases}
\end{align*}
的方程组,
其中
\(a_{1,1}\), \(a_{1,2}\), \(\dots\), \(a_{1,n}\),
\(a_{2,1}\), \(a_{2,2}\), \(\dots\), \(a_{2,n}\),
\(\dots\),
\(a_{m,1}\), \(a_{m,2}\), \(\dots\), \(a_{m,n}\),
\(b_1\), \(b_2\), \(\dots\), \(b_m\)
是事先指定的 \(mn + m\) 个数,
而 \(x_1\), \(x_2\), \(\dots\), \(x_n\)
都是未知数.
若我们不想强调方程数~\(m\),
我们可说, 这是一个 \emph{\(n\)~元 \({\leq} 1\)~次方程组};
若我们不想强调未知数之数~\(n\),
我们可说, 这是一个%
\emph{由 \(m\)~个 \({\leq} 1\)~次方程作成的方程组}.
既然 \({\leq} 1\)~次方程的另一个名字是线性方程,
我们也可说, 这是一个\emph{线性方程组}.

若数 \(c_1\), \(c_2\), \(\dots\), \(c_n\) 适合
\begin{align*}
    a_{1,1} c_1 + a_{1,2} c_2 + \dots + a_{1,n} c_n = b_1, \\
    a_{2,1} c_1 + a_{2,2} c_2 + \dots + a_{2,n} c_n = b_2, \\
    \dots \dots \dots \dots
    \dots \dots \dots \dots
    \dots \dots \dots \dots
    ,                                                      \\
    a_{m,1} c_1 + a_{m,2} c_2 + \dots + a_{m,n} c_n = b_m,
\end{align*}
我们说, \((c_1, c_2, \dots, c_n)\)
是此方程组的一个\emph{解}.
有时, 我们也说, 形如
\(x_1 = c_1\),
\(x_2 = c_2\),
\(\dots\),
\(x_n = c_n\)
的 \(n\)~个等式 (的联合)
是此方程组的一个解.

若 \(c_1 = c_2 = \dots = c_n = 0\),
且 \((c_1, c_2, \dots, c_n)\) 是此方程组的一个解,
我们说, 这是此方程组的\emph{零解}.
若 \(c_1\), \(c_2\), \(\dots\), \(c_n\) 不全为零,
且 \((c_1, c_2, \dots, c_n)\) 是此方程组的一个解,
我们说, 这是此方程组的一个\emph{非零解}.

\begin{example}
    假定有若干只鸡与若干只兔被关在某处.
    假定每一只鸡有 \(1\)~个头与 \(2\)~只腿;
    假定每一只兔有 \(1\)~个头与 \(4\)~只腿.
    假定, 我们知道,
    这些鸡与兔一共有 \(35\)~个头与 \(94\)~只腿.
    我们能由此算出鸡与兔的数目吗?

    我们代数地思考此事.
    设有 \(x_1\)~只鸡, 有 \(x_2\)~只兔.
    那么,
    这些鸡与兔%
    一共有 \(x_1 + x_2\)~个头,
    与 \(2x_1 + 4x_2\)~只腿.
    注意到, 这二个式都是 \(2\)~元 \({\leq} 1\)~次式.
    我们可列出%
    由 \(2\)~个 \(2\)~元 \({\leq} 1\)~次方程作成的方程组
    \begin{align*}
        \begin{cases}
            x_1 + x_2 = 35, \\
            2x_1 + 4x_2 = 94.
        \end{cases}
    \end{align*}
    接下来的问题, 就是解这个方程组.
    不过,
    % 不要急;
    此例的目的是使您熟悉概念.
    我想之后讲如何解这个方程组.

    可以验证, \((23, 12)\) 是此方程组的一个解:
    因为 \(23 + 12 = 35\),
    且 \(2 \cdot 23 + 4 \cdot 12 = 94\).
    因为 \(23\), \(12\) 里有一个数不是零,
    故 \((23, 12)\) 是此方程组的一个非零解.
    我们也可说,
    \(x_1 = 23\),
    \(x_2 = 12\)
    是此方程组的一个 (非零) 解.

    不过, \((12, 23)\) 不是此方程组的一个解:
    虽然 \(12 + 23 = 35\),
    但是 \(2 \cdot 12 + 4 \cdot 23 = 116 \neq 94\).
\end{example}

\begin{example}
    考虑%
    由 \(2\)~个 \(3\)~元 \({\leq} 1\)~次方程作成的方程组
    \begin{align*}
        \begin{cases}
            x_1 + 2x_2 + 3x_3 = 0, \\
            4x_1 + 5x_2 + 6x_3 = 0.
        \end{cases}
    \end{align*}
    可以验证, 对每一个数~\(k\),
    \((k, -2k, k)\) 是此方程组的一个解:
    \begin{align*}
        k + 2(-2k) + 3k = k - 4k + 3k = 0, \\
        4k + 5(-2k) + 6k = 4k - 10k + 6k = 0.
    \end{align*}
    当 \(k = 0\) 时, 这是零解;
    当 \(k \neq 0\) 时, 因为
    \(k\), \(-2k\), \(k\) 里有一个数 \(k\) 不是零,
    故这是一个非零解.

    我们也可说,
    \(x_1 = k\),
    \(x_2 = -2k\),
    \(x_3 = k\)
    是此方程组的一个解.
\end{example}

利用阵的积, 我们可简单地写一个线性方程组.
设
\begin{align*}
    A =
    \begin{bmatrix}
        a_{1,1} & a_{1,2} & \cdots & a_{1,n} \\
        a_{2,1} & a_{2,2} & \cdots & a_{2,n} \\
        \vdots  & \vdots  & {}     & \vdots  \\
        a_{m,1} & a_{m,2} & \cdots & a_{m,n} \\
    \end{bmatrix},
    \quad
    X =
    \begin{bmatrix}
        x_1 \\ x_2 \\ \vdots \\ x_n
    \end{bmatrix},
    \quad
    B =
    \begin{bmatrix}
        b_1 \\ b_2 \\ \vdots \\ b_m
    \end{bmatrix}.
\end{align*}
则, 对不超过 \(m\) 的正整数~\(i\),
\begin{align*}
    [B]_{i,1}
    = {} &
    b_i
    \\
    = {} &
    a_{i,1} x_1 + a_{i,2} x_2 + \dots + a_{i,n} x_n
    \\
    = {} &
    [A]_{i,1} [X]_{1,1} + [A]_{i,2} [X]_{2,1}
    + \dots + [A]_{i,n} [X]_{n,1}
    \\
    = {} &
    [AX]_{i,1}.
\end{align*}
注意到 \(AX\) 的尺寸也是 \(m \times 1\), 故
\begin{align*}
    AX = B.
\end{align*}
所以, 我们也说形如 \(AX = B\) 的阵等式
(\(X\) 的元是未知数, \(X\) 是恰有 \(1\)~列的阵)
是一个线性方程组.
相应地, 若 \(n \times 1\)~阵 \(C\)
(其元跟 \(X\)~的元相比, 自然都是已知数)
适合 \(AC = B\),
我们说, \(C\) 是此方程组的一个解;
若 \(C = 0\)
(也就是说, \(C\) 的每一个元都是零)
是此方程组的一个解,
我们说, \(C\) 是此方程组的零解;
若不等于零的 \(C\)
(也就是说, \(C\) 有一个不为零的元)
是此方程组的一个解,
我们说, \(C\) 是此方程组的一个非零解.

\begin{example}
    我们可改写
    \begin{align*}
        \begin{cases}
            x_1 + x_2 = 35, \\
            2x_1 + 4x_2 = 94.
        \end{cases}
    \end{align*}
    为
    \begin{align*}
        \begin{bmatrix}
            1 & 1 \\
            2 & 4 \\
        \end{bmatrix}
        \begin{bmatrix}
            x_1 \\
            x_2 \\
        \end{bmatrix}
        =
        \begin{bmatrix}
            35 \\
            94 \\
        \end{bmatrix}.
    \end{align*}
    可以验证,
    \(
    C = \begin{bmatrix}
        23 \\
        12 \\
    \end{bmatrix}
    \)
    是此方程组的一个非零解:
    \(C \neq 0\), 且
    \begin{align*}
        \begin{bmatrix}
            1 & 1 \\
            2 & 4 \\
        \end{bmatrix}
        \begin{bmatrix}
            23 \\
            12 \\
        \end{bmatrix}
        =
        \begin{bmatrix}
            1 \cdot 23 + 1 \cdot 12 \\
            2 \cdot 23 + 4 \cdot 12 \\
        \end{bmatrix}
        =
        \begin{bmatrix}
            35 \\
            94 \\
        \end{bmatrix}.
    \end{align*}

    不过,
    \(
    D = \begin{bmatrix}
        12 \\
        23 \\
    \end{bmatrix}
    \)
    不是此方程组的一个解:
    \begin{align*}
        \begin{bmatrix}
            1 & 1 \\
            2 & 4 \\
        \end{bmatrix}
        \begin{bmatrix}
            12 \\
            23 \\
        \end{bmatrix}
        =
        \begin{bmatrix}
            1 \cdot 12 + 1 \cdot 23 \\
            2 \cdot 12 + 4 \cdot 23 \\
        \end{bmatrix}
        =
        \begin{bmatrix}
            35  \\
            116 \\
        \end{bmatrix}
        \neq
        \begin{bmatrix}
            35 \\
            94 \\
        \end{bmatrix}.
    \end{align*}
\end{example}

\begin{example}
    我们可改写
    \begin{align*}
        \begin{cases}
            x_1 + 2x_2 + 3x_3 = 0, \\
            4x_1 + 5x_2 + 6x_3 = 0.
        \end{cases}
    \end{align*}
    为
    \begin{align*}
        \begin{bmatrix}
            1 & 2 & 3 \\
            4 & 5 & 6 \\
        \end{bmatrix}
        \begin{bmatrix}
            x_1 \\
            x_2 \\
            x_3 \\
        \end{bmatrix}
        =
        \begin{bmatrix}
            0 \\
            0 \\
        \end{bmatrix}.
    \end{align*}
    可以验证, 对每一个数~\(k\),
    \(
    C = k\begin{bmatrix}
        1  \\
        -2 \\
        1  \\
    \end{bmatrix}
    \)
    是此方程组的一个解:
    \begin{align*}
        \begin{bmatrix}
            1 & 2 & 3 \\
            4 & 5 & 6 \\
        \end{bmatrix}
        \left(
        k\begin{bmatrix}
                 1  \\
                 -2 \\
                 1  \\
             \end{bmatrix}
        \right)
        = {} &
        k
        \left(
        \begin{bmatrix}
                1 & 2 & 3 \\
                4 & 5 & 6 \\
            \end{bmatrix}
        \begin{bmatrix}
                1  \\
                -2 \\
                1  \\
            \end{bmatrix}
        \right)
        \\
        = {} &
        k
        \begin{bmatrix}
            1 \cdot 1 + 2 \cdot (-2) + 3 \cdot 1 \\
            4 \cdot 1 + 5 \cdot (-2) + 6 \cdot 1 \\
        \end{bmatrix}
        \\
        = {} &
        k
        \begin{bmatrix}
            0 \\
            0 \\
        \end{bmatrix}
        =
        \begin{bmatrix}
            0 \\
            0 \\
        \end{bmatrix}.
    \end{align*}
    当 \(k = 0\) 时, 它是零解;
    当 \(k \neq 0\) 时, 它是一个非零解.
\end{example}

最后, 有一件小事值得一提.
前面, 我们写一个%
由 \(m\)~个 \(n\)~元 \({\leq} 1\)~次方程作成的方程组%
为阵等式.
反过来,
设 \(A\) 是 \(m \times n\)~阵,
\(B\) 是 \(m \times 1\)~阵,
\(X\) 是未知的 \(n \times 1\)~阵.
那么, 我们也可还原形如 \(AX = B\) 的阵等式为%
由 \(m\)~个 \(n\)~元 \({\leq} 1\)~次方程作成的方程组.
比如, 我们可写
\begin{align*}
    \begin{bmatrix}
        2 & 3  & -5 & 7  & 0   \\
        0 & 1  & 11 & 13 & -17 \\
        4 & -6 & 8  & -9 & 12  \\
    \end{bmatrix}
    \begin{bmatrix}
        x_1 \\
        x_2 \\
        x_3 \\
        x_4 \\
        x_5 \\
    \end{bmatrix}
    =
    \begin{bmatrix}
        7 \\
        8 \\
        9 \\
    \end{bmatrix}
\end{align*}
为
\begin{align*}
    \begin{cases}
        2x_1 + 3x_2 - 5x_3 + 7x_4 = 7,   \\
        x_2 + 11x_3 + 13x_4 - 17x_5 = 8, \\
        4x_1 - 6x_2 + 8x_3 - 9x_4 + 12x_5 = 9.
    \end{cases}
\end{align*}
(注意到, 我们未写系数为 \(0\) 的项.)

\section{\texorpdfstring{由 \(n\)~个 \(n\)~元
      \({\leq} 1\)~次方程作成的方程组 (1)}%
  {由 n 个 n 元 ≤1 次方程作成的方程组 (1)}}

在上节, 我们学习了线性方程组的一些基本的概念.
并且, 我们知道, 可用阵等式简单地写一个线性方程组
(或者说, 形如 \(AX = B\) 的阵等式就是一个线性方程组).

从本节起, 我们讨论%
由 \(n\)~个 \(n\)~元 \({\leq} 1\)~次方程作成的方程组.
用阵等式表示这种方程组,
就是 \(AX = B\),
其中 \(A\) 是 \(n\)~级阵,
\(B\) 是 \(n \times 1\)~阵,
\(X\) 是未知的 \(n \times 1\)~阵.
\(A\) 是方阵, 故它有行列式.
\(AX = B\) 的解跟 \(A\)~的行列式是否有关系?
此事的回答是 ``是''.
本节, 我们讨论一种特别的情形.

\begin{theorem}[Cramer 公式, 1]
    设 \(A\) 是 \(n\)~级阵 (\(n \geq 1\)).
    设 \(B\) 是 \(n \times 1\)~阵.
    设 \(X\) 是未知的 \(n \times 1\)~阵.
    若 \(\det {(A)} \neq 0\),
    则线性方程组 \(AX = B\) 有唯一的解
    \begin{align*}
        X = \frac{1}{\det {(A)}} \operatorname{adj} {(A)}\,B.
    \end{align*}
\end{theorem}

在论证此事前, 我想用一个简单的例助您理解它.

\begin{example}
    设 \(a\), \(b\) 是常数.
    一个 \(1\)~元 \({\leq} 1\)~次方程
    \(ax = b\)
    当然也是一个线性方程组:
    这是方程数为 \(1\) 时的特别情形.
    我们在中学就知道, 当 \(a \neq 0\) 时,
    \(ax = b\) 有唯一的解 \(x = \frac{1}{a}\, b\).
    一方面, \(x = \frac{1}{a}\, b\) 是一个解:
    \begin{align*}
        a \Big(\frac{1}{a}\, b \Big)
        = \Big(a\, \frac{1}{a} \Big) b
        = 1b
        = b.
    \end{align*}
    另一方面, 若数~\(y\) 也适合 \(ay = b\), 则
    \begin{align*}
        y
        = 1y
        = \Big(\frac{1}{a}\, a\Big)y
        = \frac{1}{a} (ay)
        = \frac{1}{a}\, b.
    \end{align*}

    我们说, 此事是 Cramer 公式的一个特例.
    首先, 我们可写 \(ax = b\)
    为阵等式 \([a]\, [x] = [b]\).
    (注意到, \([a]\), \([x]\), \([b]\) 都是 \(1\)~级阵.)
    Cramer 公式说, 若 \(\det {[a]} \neq 0\),
    则 \([a]\, [x] = [b]\)
    有唯一的解
    (注意, \(1\)~级阵的古伴是 \([1]\))
    \begin{align*}
        [x] = \frac{1}{\det {[a]}}
        \operatorname{adj} {([a])}\,[b]
        = \frac{1}{a}\, [1]\,[b]
        = \frac{1}{a}\, [b]
        = \Big[ \frac{1}{a}\, b \Big].
    \end{align*}
    这跟我们已知的结论是一样的.
\end{example}

\begin{proof}
    我们先验证
    \(C = \frac{1}{\det {(A)}} \operatorname{adj} {(A)}\,B\)
    适合 \(AC = B\):
    \begin{align*}
        AC
        = {} & A\Big( \frac{1}{\det {(A)}}
        \operatorname{adj} {(A)}\,B \Big)
        \\
        = {} & \frac{1}{\det {(A)}} (A\operatorname{adj} {(A)}\,B)
        \\
        = {} & \frac{1}{\det {(A)}} (\det {(A)}\,I_n B)
        \\
        = {} & B.
    \end{align*}

    我们再证 \(AX = B\) 至多有一个解.
    设 \(n \times 1\)~阵 \(D\) 也适合 \(AD = B\).
    则
    \begin{align*}
        D
        = {} & I_n D
        \\
        = {} & \Big( \frac{1}{\det {(A)}}
        \operatorname{adj} {(A)}\,A \Big) D
        \\
        = {} & \frac{1}{\det {(A)}}
        \operatorname{adj} {(A)}\, (AD)
        \\
        = {} & \frac{1}{\det {(A)}}
        \operatorname{adj} {(A)}\,B
        \\
        = {} & C.
        \qedhere
    \end{align*}
\end{proof}

\begin{example}
    考虑线性方程组 \(AX = B\),
    其中
    \begin{align*}
        A = \begin{bmatrix}
                1 & 1 \\
                2 & 4 \\
            \end{bmatrix},
        \quad
        X =
        \begin{bmatrix}
            x_1 \\
            x_2 \\
        \end{bmatrix},
        \quad
        B =
        \begin{bmatrix}
            35 \\
            94 \\
        \end{bmatrix}.
    \end{align*}
    不难算出
    \begin{align*}
        \det {(A)} = 1 \cdot 4 - 2 \cdot 1 = 2 \neq 0,
    \end{align*}
    所以, 此方程组有唯一的解
    \begin{align*}
        X
        = {} &
        \frac{1}{\det {(A)}} \operatorname{adj} {(A)}\,B
        \\
        = {} &
        \frac{1}{2}
        \begin{bmatrix}
            4  & -1 \\
            -2 & 1  \\
        \end{bmatrix}
        \begin{bmatrix}
            35 \\
            94 \\
        \end{bmatrix}
        =
        \frac{1}{2}
        \begin{bmatrix}
            4 \cdot 35 - 1 \cdot 94  \\
            -2 \cdot 35 + 1 \cdot 94 \\
        \end{bmatrix}
        \\
        = {} &
        \frac{1}{2}
        \begin{bmatrix}
            46 \\
            24 \\
        \end{bmatrix}
        =
        \begin{bmatrix}
            23 \\
            12 \\
        \end{bmatrix}.
    \end{align*}
\end{example}

我们可进一步地改写 Cramer 公式.
我们先具体地算出
\(\operatorname{adj} {(A)}\,B\)~的每一个元:
\begin{align*}
    [\operatorname{adj} {(A)}\,B]_{i,1}
    = {} &
    \sum_{\ell = 1}^{n}
    {[\operatorname{adj} {(A)}]_{i,\ell} [B]_{\ell,1}}
    \\
    = {} &
    \sum_{\ell = 1}^{n}
    {(-1)^{\ell+i} \det {(A(\ell|i))}\, [B]_{\ell,1}}.
\end{align*}
设 \(A \{i, B\}\) 是%
以 \(B\) 代阵~\(A\) 的列~\(i\) 后所得的阵,
即
\begin{align*}
    [A \{i, B\}]_{\ell,k}
    = \begin{cases}
          [A]_{\ell,k}, & k \neq i; \\
          [B]_{\ell,1}, & k = i.
      \end{cases}
\end{align*}
由此可见,
\(A(\ell|i) = (A \{i, B\})(\ell|i)\)
(\(\ell = 1\), \(2\), \(\dots\), \(n\)).
所以
\begin{align*}
         &
    \sum_{\ell = 1}^{n}
    {(-1)^{\ell+i} \det {(A(\ell|i))}\, [B]_{\ell,1}}
    \\
    = {} &
    \sum_{\ell = 1}^{n}
    {(-1)^{\ell+i} \det {((A \{i, B\})(\ell|i))}\,
    [A \{i, B\}]_{\ell,1}}
    \\
    = {} &
    \det {(A \{i, B\})}.
\end{align*}
从而
\begin{align*}
    [X]_{i,1}
    = {} &
    \Big[ \frac{1}{\det {(A)}}
        \operatorname{adj} {(A)}\,B \Big]_{i,1}
    \\
    = {} &
    \frac{1}{\det {(A)}}
    [ \operatorname{adj} {(A)}\,B ]_{i,1}
    \\
    = {} &
    \frac{1}{\det {(A)}}
    \det {(A \{i, B\})}
    \\
    = {} &
    \frac{\det {(A \{i, B\})}}{\det {(A)}}.
\end{align*}
由此, 我们得到 Cramer 公式的另一个形式:

\begin{theorem}[Cramer 公式, 2]
    设线性方程组
    \begin{align*}
        \begin{cases}
            a_{1,1} x_1 + a_{1,2} x_2 + \dots
            + a_{1,n} x_n = b_1, \\
            a_{2,1} x_1 + a_{2,2} x_2 + \dots
            + a_{2,n} x_n = b_2, \\
            \dots \dots \dots \dots
            \dots \dots \dots \dots
            \dots \dots \dots \dots,
            \\
            a_{n,1} x_1 + a_{n,2} x_2 + \dots
            + a_{n,n} x_n = b_n. \\
        \end{cases}
    \end{align*}
    (注意, 方程数等于未知数之数.)
    记
    \begin{align*}
        A =
        \begin{bmatrix}
            a_{1,1} & a_{1,2} & \cdots & a_{1,n} \\
            a_{2,1} & a_{2,2} & \cdots & a_{2,n} \\
            \vdots  & \vdots  & {}     & \vdots  \\
            a_{n,1} & a_{n,2} & \cdots & a_{n,n} \\
        \end{bmatrix},
        \quad
        B =
        \begin{bmatrix}
            b_1 \\ b_2 \\ \vdots \\ b_n
        \end{bmatrix}.
    \end{align*}
    若 \(\det {(A)} \neq 0\),
    则此线性方程组有唯一的解
    \begin{align*}
        x_i = \frac{\det {(A \{i, B\})}}{\det {(A)}},
        \quad
        \text{\(i = 1\), \(2\), \(\dots\), \(n\)},
    \end{align*}
    其中 \(A \{i, B\}\) 是%
    以 \(B\) 代阵~\(A\) 的列~\(i\) 后所得的阵.
\end{theorem}

\begin{example}
    考虑由 \(2\)~个 \(2\)~元 \({\leq} 1\)~次方程作成的方程组
    \begin{align*}
        \begin{cases}
            a_{1,1} x_1 + a_{1,2} x_2 = b_1, \\
            a_{2,1} x_1 + a_{2,2} x_2 = b_2. \\
        \end{cases}
    \end{align*}
    根据 Cramer 公式, 若
    \(d = \det {
        \begin{bmatrix}
            a_{1,1} & a_{1,2} \\
            a_{2,1} & a_{2,2} \\
        \end{bmatrix}
    }
    = a_{1,1} a_{2,2} - a_{2,1} a_{1,2} \neq 0\),
    则此方程组有唯一的解
    \begin{align*}
        x_1
        = \frac{\det {
                \begin{bmatrix}
                    b_{1} & a_{1,2} \\
                    b_{2} & a_{2,2} \\
                \end{bmatrix}
            }}{d}
        = \frac{b_1 a_{2,2} - b_2 a_{1,2}}{d},
        \\
        x_2
        = \frac{\det {
                \begin{bmatrix}
                    a_{1,1} & b_{1} \\
                    a_{2,1} & b_{2} \\
                \end{bmatrix}
            }}{d}
        = \frac{a_{1,1} b_2 - a_{2,1} b_1}{d}.
    \end{align*}
\end{example}

\begin{example}
    考虑线性方程组
    \begin{align*}
        \begin{cases}
            x_1 + x_2 = 35, \\
            2x_1 + 4x_2 = 94.
        \end{cases}
    \end{align*}
    因为 \(d = 1 \cdot 4 - 2 \cdot 1 = 2 \neq 0\),
    故, 由上个例,
    此方程组有唯一的解
    \begin{align*}
        x_1 = \frac{35 \cdot 4 - 94 \cdot 1}{2} = 23,
        \\
        x_2 = \frac{1 \cdot 94 - 2 \cdot 35}{2} = 12.
    \end{align*}
\end{example}

一般地, Cramer 公式, 只是一个理论的公式.
这是因为, 一般地, 计算行列式是较复杂的事
(或许, 您还记得,
\(3\)~级阵的行列式的较具体的公式含 \(6\)~项,
而 \(4\)~级阵的行列式的较具体的公式含 \(24\)~项).
所以, 我们一般用别的方法
(如代入消元法、加减消元法等)
解线性方程组.

还是以
\begin{align*}
    \begin{cases}
        x_1 + x_2 = 35, \\
        2x_1 + 4x_2 = 94
    \end{cases}
\end{align*}
为例.
由方程~\(1\), 有 \(x_1 = 35 - x_2\).
代入其到方程~\(2\), 有
\(2(35 - x_2) + 4x_2 = 94\),
即 \(2x_2 = 24\).
由此可知 \(x_2 = 12\).
代入其到 \(x_1 = 35 - x_2\),
有 \(x_1 = 23\).
最后, 经验证, \((23, 12)\) 的确是此方程组的一个解.

\KunAsteriskoEnEnhavtabelo

\section{\texorpdfstring{由 \(n\)~个 \(n\)~元
      \({\leq} 1\)~次方程作成的方程组 (2)}%
  {由 n 个 n 元 ≤1 次方程作成的方程组 (2)}}

\maldevigalegajxo

在上节, 我们\emph{定量地}研究了%
由 \(n\)~个 \(n\)~元 \({\leq} 1\)~次方程作成的方程组%
的解.
具体地, 当 \(n\)~级阵 \(A\) 的行列式不为零时,
我们用行列式写出了 \(AX = B\) 的唯一的解,
其中
\(B\) 是 \(n \times 1\)~阵,
\(X\) 是未知的 \(n \times 1\)~阵.
但是, 若 \(\det {(A)} = 0\),
则 Cramer 公式不可用
(因为分母不可为零).
其实, 当 \(\det {(A)} = 0\) 时,
\(AX = B\) 是否有解是一个较复杂的问题.

\begin{example}
    设
    \begin{align*}
        A = \begin{bmatrix}
                1  & 1  \\
                -1 & -1 \\
            \end{bmatrix},
        \quad
        X =
        \begin{bmatrix}
            x_1 \\
            x_2 \\
        \end{bmatrix},
        \quad
        B =
        \begin{bmatrix}
            1 \\
            b \\
        \end{bmatrix},
    \end{align*}
    其中 \(b\) 是常数.
    不难算出, \(\det {(A)} = 0\).
    当 \(b\) 取某些数时,
    我们讨论 \(AX = B\) 的解.

    当 \(b = -1\) 时, 此方程组有解:
    \begin{align*}
        A
        \begin{bmatrix}
            1 \\
            0 \\
        \end{bmatrix}
        =
        \begin{bmatrix}
            1 \cdot 1 + 1 \cdot 0     \\
            -1 \cdot 1 + (-1) \cdot 0 \\
        \end{bmatrix}
        =
        B.
    \end{align*}
    甚至, \(AX = B\) 有无限多个解.
    设 \(t\) 是常数.
    则
    \begin{align*}
        A
        \begin{bmatrix}
            1 - t \\
            t     \\
        \end{bmatrix}
        =
        \begin{bmatrix}
            1 \cdot (1 - t) + 1 \cdot t     \\
            -1 \cdot (1 - t) + (-1) \cdot t \\
        \end{bmatrix}
        =
        B.
    \end{align*}
    取不同的 \(t\),
    就有不一样的
    \(
    \begin{bmatrix}
        1 - t \\
        t     \\
    \end{bmatrix}
    \).
    我们知道, 有无限多个数.
    所以, \(AX = B\) 有无限多个解.

    当 \(b = 0\) 时, 此方程组无解.
    用反证法.
    反设
    \(
    A \begin{bmatrix}
        y_1 \\
        y_2 \\
    \end{bmatrix}
    = B
    \),
    即
    \begin{align*}
        y_1 + y_2 = 1, \\
        -y_1 - y_2 = 0.
    \end{align*}
    从而
    \begin{align*}
        1 + 0
        = {} & (y_1 + y_2) + (-y_1 - y_2)     \\
        = {} & (y_1 - y_1) + (y_2 - y_2) = 0.
    \end{align*}
    这是矛盾.
\end{example}

从本节开始, 我们\emph{定性地}研究%
由 \(n\)~个 \(n\)~元 \({\leq} 1\)~次方程作成的方程组%
的解.
具体地, 我们主要讨论解的性质,
而不是解的公式.

根据 Cramer 公式, 我们不难得到如下事实:

\begin{theorem}
    设 \(A\) 是 \(n\)~级阵.
    设 \(\det {(A)} \neq 0\).
    那么, 对任何的 \(n \times 1\)~阵 \(B\),
    存在一个 \(n \times 1\)~阵 \(C\),
    使 \(AC = B\).
    (换句话说,
    若存在某 \(n \times 1\)~阵 \(B\),
    使对任何的 \(n \times 1\)~阵 \(C\),
    必 \(AC \neq B\),
    则 \(\det {(A)} = 0\).)
\end{theorem}

重要地, 此事反过来也对.

\begin{theorem}
    设 \(A\) 是 \(n\)~级阵.
    设对任何的 \(n \times 1\)~阵 \(B\),
    存在一个 \(n \times 1\)~阵 \(C\),
    使 \(AC = B\).
    则 \(\det {(A)} \neq 0\).
    (换句话说,
    若 \(\det {(A)} = 0\),
    则存在某 \(n \times 1\)~阵 \(B\),
    使对任何的 \(n \times 1\)~阵 \(C\),
    必 \(AC \neq B\).)
\end{theorem}

\begin{proof}
    设 \(n\)~级单位阵~\(I\) 的%
    列~\(1\), \(2\), \(\dots\), \(n\)
    分别是
    \(e_1\), \(e_2\), \(\dots\), \(e_n\).
    那么, 每一个 \(e_i\) 都是 \(n \times 1\)~阵.
    根据假定, 存在一个 \(n \times 1\)~阵 \(f_i\)
    使 \(Af_i = e_i\).
    作 \(n\)~级阵
    \(F = [f_1, f_2, \dots, f_n]\).
    则
    \begin{align*}
        AF
        = {} & A\, [f_1, f_2, \dots, f_n] \\
        = {} & [Af_1, Af_2, \dots, Af_n]  \\
        = {} & [e_1, e_2, \dots, e_n]     \\
        = {} & I.
    \end{align*}
    因为二个同级的方阵的积的行列式%
    等于这二个阵的行列式的积,
    \begin{align*}
        1 = \det {(I)} = \det {(A)} \det {(F)}.
    \end{align*}
    从而 \(\det {(A)} \neq 0\).
\end{proof}

\section{\texorpdfstring{由 \(n\)~个 \(n\)~元
      \({\leq} 1\)~次方程作成的方程组 (3)}%
  {由 n 个 n 元 ≤1 次方程作成的方程组 (3)}}

\maldevigalegajxo

我们进一步地\emph{定性地}研究%
由 \(n\)~个 \(n\)~元 \({\leq} 1\)~次方程作成的方程组%
的解.

前面, 我们知道,
若 \(n\)~级阵 \(A\) 的行列式为零,
则存在某 \(n \times 1\)~阵 \(B\),
使线性方程组 \(AX = B\) 无解.
当然, 对某些 \(B\),
\(AX = B\) 还是有解的:
取 \(B = 0\),
则 \(AX = B\) 至少有一个解 \(X = 0\)
(\(0\) 是元全为零的 \(n \times 1\)~阵).

本节, 我们研究, 若 \(AX = B\) 有解,
则它是否有唯一的解.

我们先看一个较简单的事实.

\begin{theorem}
    设 \(A\) 是 \(n\)~级阵.
    设 \(B\) 是 \(n \times 1\)~阵.
    设 \(n \times 1\)~阵 \(C\) 适合 \(AC = B\).

    (1)
    若 \(AX = 0\) 只有零解, 则 \(AX = B\) 的解是唯一的;

    (2)
    若存在非零的 \(n \times 1\)~阵 \(D\)
    使 \(AD = 0\),
    则 \(AX = B\) 有无限多个解.
\end{theorem}

\begin{proof}
    (1)
    设 \(n \times 1\)~阵 \(Y\) 适合 \(AY = B\).
    则
    \begin{align*}
        A(Y - C) = AY - AC = B - B = 0.
    \end{align*}
    故 \(Y - C\) 是 \(AX = 0\) 的一个解.
    因为 \(AX = 0\) 只有零解,
    故 \(Y - C = 0\).
    从而 \(Y = C\).

    (2)
    要证 \(AX = B\) 有无限多个解,
    只要证:
    任取一个正整数 \(m\),
    我们一定能找到 \(AX = B\) 的
    \(m\) 个\emph{互不相同的}解.

    作 \(m\)~个 \(n \times 1\)~阵
    \(C_1\), \(C_2\), \(\dots\), \(C_m\),
    其中 \(C_j = C + jD\).
    首先, 每一个 \(C_j\) 都是 \(AX = B\) 的解:
    \begin{align*}
        AC_j = A(C + jD) = AC + A(jD) = B + j(AD) = B + j0 = B.
    \end{align*}
    然后, 我们证,
    若 \(p\), \(q\) 是二个不相等的%
    且不超过 \(m\) 的正整数,
    则 \(C_p \neq C_q\).
    用反证法.
    反设 \(C_p = C_q\),
    则
    \begin{align*}
        0 = C_p - C_q = (C + pD) - (C + qD) = (p - q)D.
    \end{align*}
    因为 \(p \neq q\), 故 \(p - q \neq 0\).
    从而
    \begin{align*}
        0 = \frac{1}{p-q}\, 0
        = \frac{1}{p-q}\, ((p-q)D)
        = \Big( \frac{1}{p-q}\, (p-q) \Big) D
        = 1D
        = D.
    \end{align*}
    这是矛盾.
    所以, \(C_p \neq C_q\).
\end{proof}

由此可见, 若 \(AX = 0\) 有非零解,
则 \(AX = B\) 有解时,
其解不但不唯一, 还有无限多个.
若 \(AX = 0\) 只有零解,
则 \(AX = B\) 有解时,
其解是唯一的.

根据 Cramer 公式, 我们不难得到如下事实:

\begin{theorem}
    设 \(A\) 是 \(n\)~级阵.
    设 \(AX = 0\) 有非零解.
    则 \(\det {(A)} = 0\).
    (换句话说,
    若 \(\det {(A)} \neq 0\),
    则 \(AX = 0\) 只有零解.)
\end{theorem}

重要地, 此事反过来也对.

\begin{theorem}
    设 \(A\) 是 \(n\)~级阵.
    设 \(\det {(A)} = 0\).
    则 \(AX = 0\) 有非零解.
    (换句话说,
    若 \(AX = 0\) 只有零解,
    则 \(\det {(A)} \neq 0\).)
\end{theorem}

为说话方便, 我要引入一个小概念.
说阵~\(Z\) 是 \(A\) 的一个 \emph{\(p\)~级子阵},
就是说, \(Z\) 是 \(A\) 的一个子阵
(见本章, 节~\sekcio{5}),
且 \(Z\) 是一个 \(p\)~级阵.

在论证此事前, 我想用三个例%
助您理解此事为什么是正确的.

\begin{example}
    设 \(A\) 是一个 \(5\)~级阵,
    且 \(A = 0\).
    那么, 我们任取一个非零的 \(5 \times 1\)~阵 \(S\),
    必有 \(AS = 0\).
\end{example}

\begin{example}
    设 \(A\) 是一个 \(5\)~级阵,
    且 \(\det {(A)} = 0\),
    但 \(A \neq 0\).

    我们知道,
    \(A \operatorname{adj} {(A)} = \det {(A)}\, I\),
    其中 \(\operatorname{adj} {(A)}\)
    是 \(A\) 的古伴,
    \(I\) 是 \(5\)~级单位阵.
    因为 \(\det {(A)} = 0\),
    故 \(A \operatorname{adj} {(A)} = 0\).

    若 \(\operatorname{adj} {(A)} \neq 0\),
    则存在 \(k\), \(v\), 使
    \([\operatorname{adj} {(A)}]_{k,v} \neq 0\).
    我们设 \(\operatorname{adj} {(A)}\) 的%
    列~\(v\) 是 \(Y\).
    那么, \([Y]_{k,1}
        = [\operatorname{adj} {(A)}]_{k,v} \neq 0\),
    从而 \(Y \neq 0\).
    我们证明 \(AY = 0\):
    \begin{align*}
        [AY]_{i,1}
        = {} &
        \sum_{\ell = 1}^{5}
        {[A]_{i,\ell} [Y]_{\ell,1}}
        \\
        = {} &
        \sum_{\ell = 1}^{5}
        {[A]_{i,\ell} [\operatorname{adj} {(A)}]_{\ell,v}}
        \\
        = {} &
        [A \operatorname{adj} {(A)}]_{i,v}
        \\
        = {} &
        [0]_{i,v}
        \\
        = {} &
        0.
    \end{align*}
\end{example}

\begin{example}
    仍设 \(A\) 是一个 \(5\)~级阵,
    且 \(\det {(A)} = 0\),
    但 \(A \neq 0\).

    在上个例里, 我们假定
    \(\operatorname{adj} {(A)} \neq 0\).
    可是, 它也有可能为 \(0\), 即使 \(A \neq 0\)
    (比如, 可以验证
    \begin{align*}
        M = \begin{bmatrix}
                1  & 1 & -1 & -1 & 1 \\
                1  & 2 & 0  & 0  & 2 \\
                2  & 1 & -3 & -3 & 1 \\
                -1 & 0 & 2  & 2  & 0 \\
                -1 & 1 & 3  & 3  & 1 \\
            \end{bmatrix}
    \end{align*}
    的古伴就是 \(0\)).

    现在, 我们假定
    \(\operatorname{adj} {(A)} = 0\).
    此时, 我们不能利用
    \(\operatorname{adj} {(A)}\)
    写出 \(AX = 0\) 的非零解.
    我们应如何作?

    若 \(\operatorname{adj} {(A)} = 0\),
    则 \(A\) 的每一个 \(4\)~级子阵的行列式都是零.
    从而, 一定存在一个%
    低于 \(4\) 的正整数 \(r\)
    适合性质:
    \emph{\(A\) 有一个 \(r\)~级子阵, 其行列式非零,
        且 \(A\) 的每一个 \(r+1\)~级子阵的行列式都是零.}
    (我们已知 \(A\) 的每一个 \(4\)~级子阵的行列式都是零.
    我们考虑 \(A\) 的 \(3\)~级子阵.
    若有一个 \(3\)~级子阵的行列式不是零,
    我们就取 \(r = 3\).
    若不然, \(A\) 的每一个 \(3\)~级子阵的行列式都是零.
    我们考虑 \(A\) 的 \(2\)~级子阵.
    若有一个 \(2\)~级子阵的行列式不是零,
    我们就取 \(r = 2\).
    若不然, \(A\) 的每一个 \(2\)~级子阵的行列式都是零.
    我们考虑 \(A\) 的 \(1\)~级子阵.
    因为 \(A \neq 0\),
    故 \(A\) 一定有 \(1\)~级子阵的行列式不是零.
    此时, 取 \(r = 1\) 即可.)

    我以 \(r = 2\) 为例%
    演示找 \(AX = 0\) 的非零解的方法
    (可以验证,
    前面的 \(M\) 的行~\(1\), \(2\)
    与列~\(1\), \(2\) 作成的
    \(2\)~级子阵的行列式不是零,
    但 \(M\) 的每一个 \(3\)~级子阵的行列式都是零);
    \(r = 3\) 或 \(r = 1\) 时的情形是类似的.

    我们设
    \(\displaystyle
    A_2 =
    A\binom{1,2}{1,2}
    \)
    的行列式不为零,
    且 \(A\) 的每一个 \(3\)~级子阵的行列式都是零.
    再设
    \(\displaystyle
    E =
    A\binom{1,2,3}{1,2,3}
    \).
    因为 \(E\) 是 \(A\) 的一个 \(3\)~级子阵,
    故 \(\det {(E)} = 0\).
    从而
    \(E \operatorname{adj} {(E)} = 0\).
    注意到
    \([\operatorname{adj} {(E)}]_{3,3}
    = (-1)^{3+3} \det {(A_2)} \neq 0\).
    设 \(\operatorname{adj} {(E)}\)
    的列~\(3\) 为 \(T\).
    那么 \(T \neq 0\).
    并且, 我们可用完全类似的方法, 证明
    \(ET = 0\).

    接下来, 我们设法由此得到一个 \(AX = 0\)
    的非零解.
    我们作一个 \(5 \times 1\)~阵 \(W\), 其中
    \begin{align*}
        [W]_{\ell,1}
        = \begin{cases}
              [\operatorname{adj} {(E)}]_{\ell,3},
                 & \ell \leq 3; \\
              0, & \ell > 3.
          \end{cases}
    \end{align*}
    (通俗地, 我们在 \(T\) 的最后一个元后加 \(2\)~个 \(0\),
    变 \(3 \times 1\)~阵 \(T\)
    为一个 \(5 \times 1\)~阵 \(W\).)
    因为 \([W]_{3,1} \neq 0\),
    故 \(W \neq 0\).
    我们证明 \(AW = 0\).

    取不超过 \(5\) 的正整数 \(q\).
    则
    \begin{align*}
        [AW]_{q,1}
        = {} &
        \sum_{\ell = 1}^{5}
        {[A]_{q,\ell} [W]_{\ell,1}}
        \\
        = {} &
        \sum_{\ell = 1}^{3}
        {[A]_{q,\ell} [W]_{\ell,1}}
        +
        \sum_{\ell = 4}^{5}
        {[A]_{q,\ell} [W]_{\ell,1}}
        \\
        = {} &
        \sum_{\ell = 1}^{3}
        {[A]_{q,\ell} [\operatorname{adj} {(E)}]_{\ell,3}}
        +
        \sum_{\ell = 4}^{5}
        {[A]_{q,\ell}\, 0}
        \\
        = {} &
        \sum_{\ell = 1}^{3}
        {[A]_{q,\ell} [\operatorname{adj} {(E)}]_{\ell,3}}.
    \end{align*}
    若 \(q \leq 3\), 则
    \begin{align*}
        [AW]_{q,1}
        = {} &
        \sum_{\ell=1}^{3}
        {[A]_{q,\ell}
            [\operatorname{adj} {(E)}]_{\ell,3}}
        \\
        = {} &
        \sum_{p=1}^{3}
        {[E]_{q,\ell}
            [\operatorname{adj} {(E)}]_{\ell,3}}
        \\
        = {} &
        [E \operatorname{adj} {(E)}]_{q,3}
        \\
        = {} &
        [0]_{q,3}
        \\
        = {} & 0.
    \end{align*}
    设 \(q > 3\).
    则
    \begin{align*}
        [AW]_{q,1}
        = {} &
        \sum_{\ell=1}^{3}
        {[A]_{q,\ell}
            [\operatorname{adj} {(E)}]_{\ell,3}}
        \\
        = {} &
        \sum_{\ell=1}^{3}
        {[A]_{q,\ell}
        (-1)^{3+\ell} \det {(E(3|\ell))}}.
    \end{align*}
    作一个 \(3\)~级阵 \(C_q\), 其中
    \begin{align*}
        [C_q]_{h,\ell}
        = \begin{cases}
              [E]_{h,\ell},
               & h \neq 3; \\
              [A]_{q,\ell},
               & h = 3.
          \end{cases}
    \end{align*}
    于是, \(C_q(3|\ell) = E(3|\ell)\).
    从而
    \begin{align*}
        [AW]_{q,1}
        = {} &
        \sum_{\ell=1}^{3}
        {[A]_{q,\ell}
        (-1)^{3+\ell} \det {(E(3|\ell))}}
        \\
        = {} &
        \sum_{\ell=1}^{3}
        {[C_q]_{3,\ell}
        (-1)^{3+\ell} \det {(C_q (3|\ell))}}
        \\
        = {} &
        \det {(C_q)}.
    \end{align*}
    注意到
    \(
    \displaystyle
    C_q = A\binom{1,2,q}{1,2,3}
    \)
    是 \(A\)~的一个 \(3\)~级子阵,
    故
    \begin{align*}
        [AW]_{q,1} = \det {(C_q)} = 0.
    \end{align*}

    综上, \(AW = 0\), 且 \(W \neq 0\).
\end{example}

下面, 我给出此事的论证.
证明分三种情形:
(a)
\(A = 0\);
(b)
\(A \neq 0\), 且
\(\operatorname{adj} {(A)} \neq 0\);
(c)
\(A \neq 0\), 且
\(\operatorname{adj} {(A)} = 0\).
情形 (a) 最简单,
我们随便找一个非零的 \(n \times 1\)~阵即可.
情形 (b) 也不难,
我们取 \(A\) 的古伴的一个非零的列即可.
情形 (c) 是最复杂的.
在上个例里, 我们假定%
行列式非零的子阵%
是由 \(A\) 的前 \(r\)~行与前 \(r\)~列作成的,
且每一个 \(r+1\)~级子阵的行列式都是零
(此处 \(r = 2\)).
(通俗地, 我们假定行列式非零的子阵在左上角.)
不过, 对一般的阵来说,
行列式非零的子阵不一定在左上角,
这是论证的最大的挑战.

\begin{proof}
    设 \(A\) 是 \(n\)~级阵,
    且 \(\det {(A)} = 0\).

    若 \(A = 0\),
    我们任取一个非零的 \(n \times 1\)~阵 \(S\).
    则 \(AS = 0\).

    以下, 我们假定 \(A \neq 0\);
    也就是说, \(A\) 有一个元不是零
    (同时, 这也相当于 \(n > 1\):
    回想 \(1\)~级阵的行列式是什么).

    我们知道,
    \(A \operatorname{adj} {(A)} = \det {(A)}\, I\),
    其中 \(\operatorname{adj} {(A)}\)
    是 \(A\) 的古伴,
    \(I\) 是 \(n\)~级单位阵.
    因为 \(\det {(A)} = 0\),
    故 \(A \operatorname{adj} {(A)} = 0\).

    若 \(\operatorname{adj} {(A)} \neq 0\),
    则存在 \(k\), \(v\), 使
    \([\operatorname{adj} {(A)}]_{k,v} \neq 0\).
    我们设 \(\operatorname{adj} {(A)}\) 的%
    列~\(v\) 是 \(Y\).
    那么, \([Y]_{k,1}
        = [\operatorname{adj} {(A)}]_{k,v} \neq 0\),
    从而 \(Y \neq 0\).
    我们证明 \(AY = 0\):
    \begin{align*}
        [AY]_{i,1}
        = {} &
        \sum_{\ell = 1}^{n}
        {[A]_{i,\ell} [Y]_{\ell,1}}
        \\
        = {} &
        \sum_{\ell = 1}^{n}
        {[A]_{i,\ell} [\operatorname{adj} {(A)}]_{\ell,v}}
        \\
        = {} &
        [A \operatorname{adj} {(A)}]_{i,v}
        \\
        = {} &
        [0]_{i,v}
        \\
        = {} &
        0.
    \end{align*}

    若 \(\operatorname{adj} {(A)} = 0\),
    则 \(A\) 的每一个 \(n-1\)~级子阵的行列式都是零
    (同时, 这也相当于 \(n > 2\):
    回想 \(2\)~级阵的古伴是什么;
    再回想前面作过的假定 \(A \neq 0\)).
    从而, 一定存在一个%
    低于 \(n-1\) 的正整数 \(r\)
    适合性质:
    \emph{\(A\) 有一个 \(r\)~级子阵, 其行列式非零,
        且 \(A\) 的每一个 \(r+1\)~级子阵的行列式都是零.}
    (我们已知 \(A\) 的每一个 \(n-1\)~级子阵的行列式都是零.
    我们考虑 \(A\) 的 \(n-2\)~级子阵.
    若有一个 \(n-2\)~级子阵的行列式不是零,
    我们就取 \(r = n-2\).
    若不然, \(A\) 的每一个 \(n-2\)~级子阵的行列式都是零.
    我们考虑 \(A\) 的 \(n-3\)~级子阵.
    若有一个 \(n-3\)~级子阵的行列式不是零,
    我们就取 \(r = n-3\).
    \(\dots \dots\)
    若不然, \(A\) 的每一个 \(2\)~级子阵的行列式都是零.
    我们考虑 \(A\) 的 \(1\)~级子阵.
    因为 \(A \neq 0\),
    故 \(A\) 一定有 \(1\)~级子阵的行列式不是零.
    此时, 取 \(r = 1\) 即可.)

    我们设 \(A\)~的 \(r\)~级子阵
    \(
    \displaystyle
    A_r = A\binom{i_1,\dots,i_r}{j_1,\dots,j_r}
    \)
    的行列式非零,
    其中 \(i_1 < i_2 < \dots < i_r\),
    且 \(j_1 < j_2 < \dots < j_r\).
    我们从 \(1\), \(2\), \(\dots\), \(n\) 中
    去除 \(r\)~个整数
    \(i_1\), \(i_2\), \(\dots\), \(i_r\);
    此时, 还剩下 \(n-r\)~个整数,
    我们从中选一个为 \(i_{r+1}\).
    类似地,
    我们再从 \(1\), \(2\), \(\dots\), \(n\) 中
    去除 \(r\)~个整数
    \(j_1\), \(j_2\), \(\dots\), \(j_r\);
    此时, 还剩下 \(n-r\)~个整数,
    我们从中选一个为 \(j_{r+1}\).
    作 \(r+1\)~级阵
    \(
    \displaystyle
    E =
    A\binom{i_1,\dots,i_r,i_{r+1}}
    {j_1,\dots,j_r,j_{r+1}}
    \).
    我们设 \(i_p\) 在
    \(i_1\), \(\dots\), \(i_r\), \(i_{r+1}\) 中%
    是第~\(f(i_p)\) 小的数.
    再设 \(j_p\) 在
    \(j_1\), \(\dots\), \(j_r\), \(j_{r+1}\) 中%
    是第~\(g(j_p)\) 小的数.
    因为 \(E\) 是 \(A\) 的一个 \(r+1\)~级子阵,
    故 \(\det {(E)} = 0\).
    从而
    \(E \operatorname{adj} {(E)} = 0\).
    注意到
    \([\operatorname{adj} {(E)}]_{g(j_{r+1}), f(i_{r+1})}
    = (-1)^{f(i_{r+1})+g(j_{r+1})} \det {(A_r)} \neq 0\).
    设 \(\operatorname{adj} {(E)}\)
    的列~\(f(i_{r+1})\) 为 \(T\).
    那么 \(T \neq 0\).
    并且, 我们可用完全类似的方法, 证明
    \(ET = 0\).

    接下来, 我们设法由此得到一个 \(AX = 0\)
    的非零解.
    我们作一个 \(n \times 1\)~阵 \(W\), 其中
    \begin{align*}
        [W]_{\ell,1}
        = \begin{cases}
              [\operatorname{adj} {(E)}]_{g(j_p),f(i_{r+1})},
                 & \text{\(\ell = j_p\),
              \(p = 1\), \(\dots\), \(r\), \(r+1\)}; \\
              0, & \text{其他}.
          \end{cases}
    \end{align*}
    (通俗地, 我们在适当的位置写零,
    变 \((r+1) \times 1\)~阵 \(T\)
    为一个 \(n \times 1\)~阵 \(W\).)
    因为 \([W]_{j_{r+1},1} \neq 0\),
    故 \(W \neq 0\).
    我们证明 \(AW = 0\).

    取不超过 \(n\) 的正整数 \(q\).
    则
    \begin{align*}
        [AW]_{q,1}
        = {} &
        \sum_{\ell = 1}^{n}
        {[A]_{q,\ell} [W]_{\ell,1}}
        \\
        = {} &
        \sum_{\substack{
        1 \leq \ell \leq n \\
        \ell = j_p         \\
                1 \leq p \leq r+1
            }}
        {[A]_{q,\ell} [W]_{\ell,1}}
        +
        \sum_{\substack{
        1 \leq \ell \leq n \\
                \text{其他}
            }}
        {[A]_{q,\ell} [W]_{\ell,1}}
        \\
        = {} &
        \sum_{\substack{
        1 \leq \ell \leq n \\
        \ell = j_p         \\
                1 \leq p \leq r+1
            }}
        {[A]_{q,j_p}
                [\operatorname{adj} {(E)}]_{g(j_p),f(i_{r+1})}}
        +
        \sum_{\substack{
        1 \leq \ell \leq n \\
                \text{其他}
            }}
        {[A]_{q,\ell}\, 0}
        \\
        = {} &
        \sum_{p=1}^{r+1}
        {[A]_{q,j_p}
            [\operatorname{adj} {(E)}]_{g(j_p),f(i_{r+1})}}.
    \end{align*}
    若 \(q\) 等于某个 \(i_t\), 则
    \begin{align*}
        [AW]_{i_t,1}
        = {} &
        \sum_{p=1}^{r+1}
        {[A]_{i_t,j_p}
            [\operatorname{adj} {(E)}]_{g(j_p),f(i_{r+1})}}
        \\
        = {} &
        \sum_{p=1}^{r+1}
        {[E]_{f(i_t),g(j_p)}
            [\operatorname{adj} {(E)}]_{g(j_p),f(i_{r+1})}}
        \\
        = {} &
        [E \operatorname{adj} {(E)}]_{f(i_t),f(i_{r+1})}
        \\
        = {} &
        [0]_{f(i_t),f(i_{r+1})}
        \\
        = {} & 0.
    \end{align*}
    若 \(q\) 不等于
    \(i_1\), \(\dots\), \(i_r\), \(i_{r+1}\)
    中的任何一个,
    则
    \begin{align*}
        [AW]_{q,1}
        = {} &
        \sum_{p=1}^{r+1}
        {[A]_{q,j_p}
            [\operatorname{adj} {(E)}]_{g(j_p),f(i_{r+1})}}
        \\
        = {} &
        \sum_{p=1}^{r+1}
        {[A]_{q,j_p}
        (-1)^{f(i_{r+1})+g(j_p)}
        \det {(E(f(i_{r+1})|g(j_p)))}}.
    \end{align*}
    作一个 \(r+1\)~级阵 \(C_q\), 其中
    \begin{align*}
        [C_q]_{h,g(j_p)}
        = \begin{cases}
              [E]_{h,g(j_p)},
               & h \neq f(i_{r+1}); \\
              [A]_{q,j_p},
               & h = f(i_{r+1}).
          \end{cases}
    \end{align*}
    于是, \(C_q(f(i_{r+1})|g(j_p))
    = E(f(i_{r+1})|g(j_p))\).
    从而
    \begin{align*}
        [AW]_{q,1}
        = {} &
        \sum_{p=1}^{r+1}
        {[A]_{q,j_p}
        (-1)^{f(i_{r+1})+g(j_p)}
        \det {(E(f(i_{r+1})|g(j_p)))}}
        \\
        = {} &
        \sum_{p=1}^{r+1}
        {[C_q]_{f(i_{r+1}),g(j_p)}
        (-1)^{f(i_{r+1})+g(j_p)}
        \det {(C_q (f(i_{r+1})|g(j_p)))}}
        \\
        = {} &
        \det {(C_q)}.
    \end{align*}

    再作一个 \(r+1\)~级阵
    \(
    \displaystyle
    D_q =
    A\binom{i_1,\dots,i_r,q}
    {j_1,\dots,j_r,j_{r+1}}
    \).
    不难看出, 适当地交换 \(C_q\) 的行的次序,
    即可变 \(C_q\) 为 \(D_q\).
    根据反称性,
    \(\det {(C_q)} = \pm \det {(D_q)}\).
    (具体地, 设 \(q\) 在
    \(i_1\), \(\dots\), \(i_r\), \(q\) 中%
    是第~\(u\) 小的数.
    那么, 当 \(f(i_{r+1}) = u\) 时, \(C_q = D_q\).
    当 \(f(i_{r+1}) < u\) 时,
    我们交换行~\(f(i_{r+1})\) 与 \(f(i_{r+1})+1\),
    再交换行~\(f(i_{r+1})+1\) 与 \(f(i_{r+1})+2\),
    \(\dots \dots\),
    再交换行~\(u-1\) 与 \(u\),
    作 \(u - f(i_{r+1})\)~次相邻行的交换,
    即可变 \(C_q\) 为 \(D_q\).
    当 \(f(i_{r+1}) > u\) 时,
    我们交换行~\(f(i_{r+1})\) 与 \(f(i_{r+1})-1\),
    再交换行~\(f(i_{r+1})-1\) 与 \(f(i_{r+1})-2\),
    \(\dots \dots\),
    再交换行~\(u+1\) 与 \(u\),
    作 \(f(i_{r+1}) - u\)~次相邻行的交换,
    即可变 \(C_q\) 为 \(D_q\).)
    因为 \(D_q\) 是 \(A\) 的一个 \(r+1\)~级子阵,
    故其行列式为零.
    从而
    \begin{align*}
        [AW]_{q,1}
        = \det {(C_q)}
        = \pm \det {(D_q)}
        = 0.
    \end{align*}

    综上, 我们找到了一个 \(n \times 1\)~阵 \(W\)
    使 \(AW = 0\), 且 \(W \neq 0\).
\end{proof}

\section{\texorpdfstring{由 \(n\)~个 \(n\)~元
      \({\leq} 1\)~次方程作成的方程组 (4)}%
  {由 n 个 n 元 ≤1 次方程作成的方程组 (4)}}

\maldevigalegajxo

本节没有新的知识.

综合前几节的讨论, 我们有如下定理.

\begin{theorem}
    设 \(A\) 是 \(n\)~级阵,
    \(B\) 是 \(n \times 1\)~阵,
    \(X\) 是未知的 \(n \times 1\)~阵.

    (1)
    当 \(\det {(A)} \neq 0\) 时,
    \(AX = B\) 有唯一的解;

    (2)
    当 \(\det {(A)} = 0\) 时,
    \(AX = B\) 要么无解,
    要么有无限多个解.
\end{theorem}

这就是%
由 \(n\)~个 \(n\)~元 \({\leq} 1\)~次方程作成的方程组%
的解的定性的理论 (青春版).
不过,
% 不过:
% (a)
当 \(\det {(A)} = 0\) 时,
如何进一步地\emph{判断} \(AX = B\) 是否有解?
这是此理论无法解答的问题.
% (b)
% 此理论能否被推广到一般的%
% 由 \(m\)~个 \(n\)~元 \({\leq} 1\)~次方程作成的方程组?
% (c)
% 若 \(AX = B\) 有解,
% 但 \(\det {(A)} = 0\),
% 能否给出解的公式?
% 这是定量的理论.

若 \(\det {(A)} \neq 0\),
则, 定量地,
我们可用 Cramer 公式写出
\(AX = B\) 的 (唯一的) 解.
不过, 若 \(AX = B\),
但 \(\det {(A)} = 0\),
我们如何写出它的 (所有的) 解?

我会在接下来的几节里,
讨论一般的%
由 \(m\)~个 \(n\)~元 \({\leq} 1\)~次方程作成的方程组,
并解决这些问题.

\section{\texorpdfstring{由 \(m\)~个 \(n\)~元
      \({\leq} 1\)~次方程作成的方程组 (1)}%
  {由 m 个 n 元 ≤1 次方程作成的方程组 (1)}}

\maldevigalegajxo

在接下来的若干节里, 我想用行列式讨论%
由 \(m\)~个 \(n\)~元 \({\leq} 1\)~次方程作成的方程组%
的解.
我希望您注意到, 方程数 \(m\) 不一定等于未知数之数 \(n\).

讨论%
由 \(n\)~个 \(n\)~元 \({\leq} 1\)~次方程作成的方程组%
的解时,
我们先对一种特别情形作了定量的讨论
(Cramer 公式),
再对较一般的情形作了定性的讨论;
换句话说, 我们当时是先定量, 再定性.
不过, 这次,
讨论%
由 \(m\)~个 \(n\)~元 \({\leq} 1\)~次方程作成的方程组%
时, 我们先定性, 再定量.
这么作, 会较方便一些.

或许, 您记得,
若一个 \(n\)~级阵 \(A\) 的行列式为零,
则 \(AX = 0\) 有非零解.
或许, 您还记得,
\(AX = 0\) 是否有非零解%
跟 \(AX = B\) (有解时) 是否有唯一的解%
有关.
对一般的%
由 \(m\)~个 \(n\)~元 \({\leq} 1\)~次方程作成的方程组%
来说, 我们也有类似的结论.

本节, 我们讨论, 线性方程组有解时, 解是否唯一.

我们先看一个较简单的事实.

\begin{theorem}
    设 \(A\) 是 \(m \times n\)~阵.
    设 \(B\) 是 \(m \times 1\)~阵.
    设 \(n \times 1\)~阵 \(C\) 适合 \(AC = B\).

    (1)
    若 \(AX = 0\) 只有零解, 则 \(AX = B\) 的解是唯一的;

    (2)
    若存在非零的 \(n \times 1\)~阵 \(D\)
    使 \(AD = 0\),
    则 \(AX = B\) 有无限多个解.
\end{theorem}

\begin{proof}
    (1)
    设 \(n \times 1\)~阵 \(Y\) 适合 \(AY = B\).
    则
    \begin{align*}
        A(Y - C) = AY - AC = B - B = 0.
    \end{align*}
    故 \(Y - C\) 是 \(AX = 0\) 的一个解.
    因为 \(AX = 0\) 只有零解,
    故 \(Y - C = 0\).
    从而 \(Y = C\).

    (2)
    要证 \(AX = B\) 有无限多个解,
    只要证:
    任取一个正整数 \(\ell\),
    我们一定能找到 \(AX = B\) 的
    \(\ell\) 个\emph{互不相同的}解.

    作 \(\ell\)~个 \(n \times 1\)~阵
    \(C_1\), \(C_2\), \(\dots\), \(C_\ell\),
    其中 \(C_j = C + jD\).
    首先, 每一个 \(C_j\) 都是 \(AX = B\) 的解:
    \begin{align*}
        AC_j = A(C + jD) = AC + A(jD) = B + j(AD) = B + j0 = B.
    \end{align*}
    然后, 我们证,
    若 \(p\), \(q\) 是二个不相等的%
    且不超过 \(\ell\) 的正整数,
    则 \(C_p \neq C_q\).
    用反证法.
    反设 \(C_p = C_q\),
    则
    \begin{align*}
        0 = C_p - C_q = (C + pD) - (C + qD) = (p - q)D.
    \end{align*}
    因为 \(p \neq q\), 故 \(p - q \neq 0\).
    从而
    \begin{align*}
        0 = \frac{1}{p-q}\, 0
        = \frac{1}{p-q}\, ((p-q)D)
        = \Big( \frac{1}{p-q}\, (p-q) \Big) D
        = 1D
        = D.
    \end{align*}
    这是矛盾.
    所以, \(C_p \neq C_q\).
\end{proof}

由此可见, 若 \(AX = 0\) 有非零解,
则 \(AX = B\) 有解时,
其解不但不唯一, 还有无限多个.
若 \(AX = 0\) 只有零解,
则 \(AX = B\) 有解时,
其解是唯一的.

接着, 我们讨论, \(AX = 0\) 是否有非零解.
不过, 为了使讨论简单一些, 我们要作一些准备.

\begin{theorem}
    设 \(A\) 是 \(m \times n\)~阵.
    若 \(A\) 没有行列式非零的 \(r+1\)~级子阵,
    则对任何高于 \(r\)~的整数 \(s\),
    \(A\) 没有行列式非零的 \(s\)~级子阵.
\end{theorem}

\begin{proof}
    我们用数学归纳法证明此事.
    具体地, 设命题 \(P(s)\) 为:
    \begin{quotation}
        \(A\) 没有行列式非零的 \(s\)~级子阵.
    \end{quotation}
    则我们的目标是:
    对任何高于 \(r\)~的整数 \(s\),
    \(P(s)\) 是正确的.

    \(P(r+1)\) 是正确的.

    假定 \(P(t)\) 是正确的.
    我们由此证, \(P(t+1)\) 也是正确的.
    若 \(A\) 不存在 \(t+1\)~级子阵,
    此事自然是正确的.
    若 \(A\) 存在 \(t+1\)~级子阵,
    % 利用按一列 (或一行) 展开行列式,
    利用定义, 按列~\(1\) 展开其行列式,
    可知,
    此子阵的行列式%
    是 \(A\)~的 \(t+1\)~个 \(t\)~级子阵的行列式的倍的和.
    由此可知,
    \(A\) 没有行列式非零的 \(t+1\)~级子阵.

    所以, \(P(t+1)\) 是正确的.
    由数学归纳法原理, 待证命题成立.
\end{proof}

\begin{theorem}
    设 \(A\) 是 \(m \times n\)~阵.
    存在一个唯一的非负整数 \(r\),
    使:

    (1)
    \(A\) 有一个行列式非零的 \(r\)~级子阵;

    (2)
    \(A\) 没有行列式非零的 \(r+1\)~级子阵.
\end{theorem}

\begin{proof}
    唯一性是较简单的.
    设还有非负整数 \(s\) 适合条件:

    (1\ensuremath{'})
    \(A\) 有一个行列式非零的 \(s\)~级子阵;

    (2\ensuremath{'})
    \(A\) 没有行列式非零的 \(s+1\)~级子阵.

    反设 \(s > r\).
    既然 \(A\) 没有行列式非零的 \(r+1\)~级子阵,
    故 \(A\) 没有行列式非零的 \(s\)~级子阵.
    这跟 (1\ensuremath{'}) 矛盾.
    所以, \(s \leq r\).
    类似地, 我们可证 \(r \leq s\).
    从而 \(s = r\).

    下面, 我们说明存在性.
    我们设 \(t\) 是 \(m\) 与 \(n\) 中的较小者.
    % 毕竟, 若此 \(r\) 存在, 则它既不超过 \(m\), 也不超过 \(n\).
    若 \(A\) 有行列式非零的 \(t\)~级子阵,
    我们可取 \(r = t\);
    否则, 我们代 \(t\) 以 \(t - 1\).
    若 \(A\) 有行列式非零的 \(t - 1\)~级子阵,
    我们可取 \(r = t - 1\);
    否则, 我们代 \(t - 1\) 以 \(t - 2\).
    \(\dots \dots\)
    反复地作下去,
    我们一定能找到此 \(r\).
    (注意到, 我们约定 ``\(0\)~级阵'' 的行列式为 \(1\),
    所以这个过程一定能结束.)
\end{proof}

\begin{theorem}[零的作用]
    设 \(A\) 是 \(m \times n\)~阵.
    设 \(B\) 是 \(m \times 1\)~阵.
    设 \(k\) 是一个不超过 \(m\)~的正整数.
    作一个 \((m + 1) \times n\)~阵 \(\underbar{A}\),
    其中
    \begin{align*}
        [\underbar{A}]_{i,j}
        = \begin{cases}
              [A]_{i,j},   & i \leq k;  \\
              0,           & i = k + 1; \\
              [A]_{i-1,j}, & i > k + 1.
          \end{cases}
    \end{align*}
    (通俗地, \(\underbar{A}\)
    是在 \(A\)~的行~\(k\) 下写零行作成的阵.)
    再作一个 \((m + 1) \times 1\)~阵 \(\underbar{B}\),
    其中
    \begin{align*}
        [\underbar{B}]_{i,1}
        = \begin{cases}
              [B]_{i,1},   & i \leq k;  \\
              0,           & i = k + 1; \\
              [B]_{i-1,1}, & i > k + 1.
          \end{cases}
    \end{align*}
    (通俗地, \(\underbar{B}\)
    是在 \(B\)~的行~\(k\) 下写零作成的阵.)

    (1)
    若 \(n \times 1\)~阵 \(C\) 适合
    \(AC = B\),
    则 \(\underbar{A}C = \underbar{B}\).

    (2)
    若 \(n \times 1\)~阵 \(C\) 适合
    \(\underbar{A}C = \underbar{B}\),
    则 \(AC = B\).

    (3)
    若 \(A\) 有一个行列式非零的 \(r\)~级子阵,
    但没有行列式非零的 \(r+1\)~级子阵,
    则 \(\underbar{A}\) 有一个行列式非零的 \(r\)~级子阵,
    但没有行列式非零的 \(r+1\)~级子阵.

    (4)
    若 \(\underbar{A}\) 有一个行列式非零的 \(r\)~级子阵,
    但没有行列式非零的 \(r+1\)~级子阵,
    则 \(A\) 有一个行列式非零的 \(r\)~级子阵,
    但没有行列式非零的 \(r+1\)~级子阵.
\end{theorem}

\begin{proof}
    (1)
    设 \(AC = B\).
    我们要证 \(\underbar{A}C = \underbar{B}\).

    首先, \(\underbar{A}C\) 与 \(\underbar{B}\)
    的尺寸都是 \((m+1) \times 1\).

    若 \(i \leq k\), 则
    \begin{align*}
        [\underbar{A}C]_{i,1}
        = {} &
        \sum_{\ell = 1}^{n}
        {[\underbar{A}]_{i,\ell} [C]_{\ell,1}}
        \\
        = {} &
        \sum_{\ell = 1}^{n}
        {[A]_{i,\ell} [C]_{\ell,1}}
        \\
        = {} & [AC]_{i,1}
        \\
        = {} & [B]_{i,1}
        % \\
        % = {} &
            =
            [\underbar{B}]_{i,1}.
    \end{align*}
    若 \(i = k + 1\), 则
    \begin{align*}
        [\underbar{A}C]_{i,1}
        = {} &
        \sum_{\ell = 1}^{n}
        {[\underbar{A}]_{i,\ell} [C]_{\ell,1}}
        \\
        = {} &
        \sum_{\ell = 1}^{n}
        {0\,[C]_{\ell,1}}
        \\
        = {} & 0
        % \\
        % = {} &
        =
        [\underbar{B}]_{i,1}.
    \end{align*}
    若 \(i > k + 1\), 则
    \begin{align*}
        [\underbar{A}C]_{i,1}
        = {} &
        \sum_{\ell = 1}^{n}
        {[\underbar{A}]_{i,\ell} [C]_{\ell,1}}
        \\
        = {} &
        \sum_{\ell = 1}^{n}
        {[A]_{i-1,\ell} [C]_{\ell,1}}
        \\
        = {} & [AC]_{i-1,1}
        \\
        = {} & [B]_{i-1,1}
        % \\
        % = {} &
            =
            [\underbar{B}]_{i,1}.
    \end{align*}
    由此可见, \(\underbar{A}C = \underbar{B}\).

    (2)
    设 \(\underbar{A}C = \underbar{B}\).
    我们要证 \(AC = B\).

    首先, \(AC\) 与 \(B\)
    的尺寸都是 \(m \times 1\).

    其次, 注意到
    \begin{align*}
        [A]_{i,j}
        = \begin{cases}
              [\underbar{A}]_{i,j},   & i \leq k; \\
              [\underbar{A}]_{i+1,j}, & i > k.
          \end{cases}
    \end{align*}
    类似地,
    \begin{align*}
        [B]_{i,1}
        = \begin{cases}
              [\underbar{B}]_{i,1},   & i \leq k; \\
              [\underbar{B}]_{i+1,1}, & i > k.
          \end{cases}
    \end{align*}

    若 \(i \leq k\), 则
    \begin{align*}
        [AC]_{i,1}
        = {} &
        \sum_{\ell = 1}^{n}
        {[A]_{i,\ell} [C]_{\ell,1}}
        \\
        = {} &
        \sum_{\ell = 1}^{n}
        {[\underbar{A}]_{i,\ell} [C]_{\ell,1}}
        \\
        = {} & [\underbar{A}C]_{i,1}
        \\
        = {} & [\underbar{B}]_{i,1}
        % \\
        % = {} &
            =
            [B]_{i,1}.
    \end{align*}
    若 \(i > k\), 则
    \begin{align*}
        [AC]_{i,1}
        = {} &
        \sum_{\ell = 1}^{n}
        {[A]_{i,\ell} [C]_{\ell,1}}
        \\
        = {} &
        \sum_{\ell = 1}^{n}
        {[\underbar{A}]_{i+1,\ell} [C]_{\ell,1}}
        \\
        = {} & [\underbar{A}C]_{i+1,1}
        \\
        = {} & [\underbar{B}]_{i+1,1}
        % \\
        % = {} &
            =
            [B]_{i,1}.
    \end{align*}
    由此可见, \(AC = B\).

    (3)
    注意到, \(A\)~的子阵都是 \(\underbar{A}\)~的子阵.
    所以, 若 \(A\) 有一个行列式非零的 \(r\)~级子阵,
    则 \(\underbar{A}\) 也有一个行列式非零的 \(r\)~级子阵.
    对 \(\underbar{A}\)~的每一个子阵来说,
    要么 \(\underbar{A}\) 的行~\(k+1\) 被选中,
    要么 \(\underbar{A}\) 的行~\(k+1\) 不被选中.
    所以, 那些 \(\underbar{A}\) 的行~\(k+1\) 不被选中的
    \(r+1\)~级子阵 (若存在)
    一定是 \(A\)~的 \(r+1\)~级子阵 (若存在),
    故其行列式为零.
    而那些 \(\underbar{A}\) 的行~\(k+1\) 被选中的
    \(r+1\)~级子阵 (若存在)
    有一行的元全是零,
    故其行列式为零.
    从而, \(\underbar{A}\) 也没有行列式非零的 \(r+1\)~级子阵.

    (4)
    对 \(\underbar{A}\)~的每一个子阵来说,
    要么 \(\underbar{A}\) 的行~\(k+1\) 被选中,
    要么 \(\underbar{A}\) 的行~\(k+1\) 不被选中.
    设 \(\underbar{A}\) 有一个行列式非零的 \(r\)~级子阵.
    那么, 这个子阵一定不含元全为零的行.
    特别地, 对此子阵而言,
    \(\underbar{A}\) 的行~\(k+1\) 一定不被选中.
    所以, 这也是 \(A\)~的一个 \(r\)~级子阵.
    所以, \(A\) 也有一个行列式非零的 \(r\)~级子阵.
    注意到, \(A\)~的子阵都是 \(\underbar{A}\)~的子阵.
    既然 \(\underbar{A}\) 没有行列式非零的 \(r+1\)~级子阵,
    那么 \(A\) 也没有行列式非零的 \(r+1\)~级子阵.
\end{proof}

这是本节的首个重要的结论.

\begin{theorem}
    设 \(A\) 是 \(m \times n\)~阵.
    设 \(A\) 有一个行列式非零的 \(r\)~级子阵,
    但没有行列式非零的 \(r+1\)~级子阵.
    设 \(r < n\).
    则 \(AX = 0\) 有非零解.
\end{theorem}

\begin{proof}
    设 \(A\) 是 \(m \times n\)~阵.

    我们无妨设 \(m \geq n\).
    若 \(m < n\), 我们作一个 \(n\)~级阵 \(\underbar{A}\),
    其中
    \begin{align*}
        [\underbar{A}]_{i,j}
        = \begin{cases}
              [A]_{i,j}, & i \leq m; \\
              0,         & i > m.
          \end{cases}
    \end{align*}
    (通俗地, \(\underbar{A}\)
    是在 \(A\)~的行~\(m\) 下写 \(n-m\)~个零行作成的阵.)
    由此, 不难得到:
    (a)
    若 \(n \times 1\)~阵 \(C\) 适合 \(\underbar{A}C = 0\),
    则 \(AC = 0\);
    (b)
    \(\underbar{A}\) 有一个行列式非零的 \(r\)~级子阵,
    但没有行列式非零的 \(r+1\)~级子阵.
    所以, 若我们找到了 \(\underbar{A}X = 0\)
    的非零解,
    则我们也找到了 \(AX = 0\) 的非零解.
    以下, 我们设 \(m \geq n\).

    若 \(A = 0\),
    我们任取一个非零的 \(n \times 1\)~阵 \(S\).
    则 \(AS = 0\).

    以下, 我们假定 \(A \neq 0\);
    也就是说, \(A\) 有一个元不是零
    (同时, 这也相当于 \(n > 1\):
    回想 \(1\)~级阵的行列式是什么).
    从而 \(r \geq 1\).

    我们设 \(A\)~的 \(r\)~级子阵
    \(
    \displaystyle
    A_r = A\binom{i_1,\dots,i_r}{j_1,\dots,j_r}
    \)
    的行列式非零,
    其中 \(i_1 < i_2 < \dots < i_r\),
    且 \(j_1 < j_2 < \dots < j_r\).
    我们从 \(1\), \(2\), \(\dots\), \(m\) 中
    去除 \(r\)~个整数
    \(i_1\), \(i_2\), \(\dots\), \(i_r\);
    此时, 还剩下 \(m-r\)~个整数,
    我们从中选一个为 \(i_{r+1}\).
    类似地,
    我们再从 \(1\), \(2\), \(\dots\), \(n\) 中
    去除 \(r\)~个整数
    \(j_1\), \(j_2\), \(\dots\), \(j_r\);
    此时, 还剩下 \(n-r\)~个整数,
    我们从中选一个为 \(j_{r+1}\).
    作 \(r+1\)~级阵
    \(
    \displaystyle
    E =
    A\binom{i_1,\dots,i_r,i_{r+1}}
    {j_1,\dots,j_r,j_{r+1}}
    \).
    我们设 \(i_p\) 在
    \(i_1\), \(\dots\), \(i_r\), \(i_{r+1}\) 中%
    是第~\(f(i_p)\) 小的数.
    再设 \(j_p\) 在
    \(j_1\), \(\dots\), \(j_r\), \(j_{r+1}\) 中%
    是第~\(g(j_p)\) 小的数.
    因为 \(E\) 是 \(A\) 的一个 \(r+1\)~级子阵,
    故 \(\det {(E)} = 0\).
    从而
    \(E \operatorname{adj} {(E)} = 0\).
    注意到
    \([\operatorname{adj} {(E)}]_{g(j_{r+1}), f(i_{r+1})}
    = (-1)^{f(i_{r+1})+g(j_{r+1})} \det {(A_r)} \neq 0\).

    接下来, 我们设法由此得到一个 \(AX = 0\)
    的非零解.
    我们作一个 \(n \times 1\)~阵 \(W\), 其中
    \begin{align*}
        [W]_{\ell,1}
        = \begin{cases}
              [\operatorname{adj} {(E)}]_{g(j_p),f(i_{r+1})},
                 & \text{\(\ell = j_p\),
              \(p = 1\), \(\dots\), \(r\), \(r+1\)}; \\
              0, & \text{其他}.
          \end{cases}
    \end{align*}
    (通俗地, 我们在适当的位置写零,
    变 \(\operatorname{adj} {(E)}\)
    的列~\(f(i_{r+1})\)
    为一个 \(n \times 1\)~阵 \(W\).)
    因为 \([W]_{j_{r+1},1} \neq 0\),
    故 \(W \neq 0\).
    我们证明 \(AW = 0\).

    取不超过 \(m\) 的正整数 \(q\).
    则
    \begin{align*}
        [AW]_{q,1}
        = {} &
        \sum_{\ell = 1}^{n}
        {[A]_{q,\ell} [W]_{\ell,1}}
        \\
        = {} &
        \sum_{\substack{
        1 \leq \ell \leq n \\
        \ell = j_p         \\
                1 \leq p \leq r+1
            }}
        {[A]_{q,\ell} [W]_{\ell,1}}
        +
        \sum_{\substack{
        1 \leq \ell \leq n \\
                \text{其他}
            }}
        {[A]_{q,\ell} [W]_{\ell,1}}
        \\
        = {} &
        \sum_{\substack{
        1 \leq \ell \leq n \\
        \ell = j_p         \\
                1 \leq p \leq r+1
            }}
        {[A]_{q,j_p}
                [\operatorname{adj} {(E)}]_{g(j_p),f(i_{r+1})}}
        +
        \sum_{\substack{
        1 \leq \ell \leq n \\
                \text{其他}
            }}
        {[A]_{q,\ell}\, 0}
        \\
        = {} &
        \sum_{p=1}^{r+1}
        {[A]_{q,j_p}
            [\operatorname{adj} {(E)}]_{g(j_p),f(i_{r+1})}}.
    \end{align*}
    若 \(q\) 等于某个 \(i_t\), 则
    \begin{align*}
        [AW]_{i_t,1}
        = {} &
        \sum_{p=1}^{r+1}
        {[A]_{i_t,j_p}
            [\operatorname{adj} {(E)}]_{g(j_p),f(i_{r+1})}}
        \\
        = {} &
        \sum_{p=1}^{r+1}
        {[E]_{f(i_t),g(j_p)}
            [\operatorname{adj} {(E)}]_{g(j_p),f(i_{r+1})}}
        \\
        = {} &
        [E \operatorname{adj} {(E)}]_{f(i_t),f(i_{r+1})}
        \\
        = {} &
        [0]_{f(i_t),f(i_{r+1})}
        \\
        = {} & 0.
    \end{align*}
    若 \(q\) 不等于
    \(i_1\), \(\dots\), \(i_r\), \(i_{r+1}\)
    中的任何一个,
    则
    \begin{align*}
        [AW]_{q,1}
        = {} &
        \sum_{p=1}^{r+1}
        {[A]_{q,j_p}
            [\operatorname{adj} {(E)}]_{g(j_p),f(i_{r+1})}}
        \\
        = {} &
        \sum_{p=1}^{r+1}
        {[A]_{q,j_p}
        (-1)^{f(i_{r+1})+g(j_p)}
        \det {(E(f(i_{r+1})|g(j_p)))}}.
    \end{align*}
    作一个 \(r+1\)~级阵 \(C_q\), 其中
    \begin{align*}
        [C_q]_{h,g(j_p)}
        = \begin{cases}
              [E]_{h,g(j_p)},
               & h \neq f(i_{r+1}); \\
              [A]_{q,j_p},
               & h = f(i_{r+1}).
          \end{cases}
    \end{align*}
    于是, \(C_q(f(i_{r+1})|g(j_p))
    = E(f(i_{r+1})|g(j_p))\).
    从而
    \begin{align*}
        [AW]_{q,1}
        = {} &
        \sum_{p=1}^{r+1}
        {[A]_{q,j_p}
        (-1)^{f(i_{r+1})+g(j_p)}
        \det {(E(f(i_{r+1})|g(j_p)))}}
        \\
        = {} &
        \sum_{p=1}^{r+1}
        {[C_q]_{f(i_{r+1}),g(j_p)}
        (-1)^{f(i_{r+1})+g(j_p)}
        \det {(C_q (f(i_{r+1})|g(j_p)))}}
        \\
        = {} &
        \det {(C_q)}.
    \end{align*}

    再作一个 \(r+1\)~级阵
    \(
    \displaystyle
    D_q =
    A\binom{i_1,\dots,i_r,q}
    {j_1,\dots,j_r,j_{r+1}}
    \).
    不难看出, 适当地交换 \(C_q\) 的行的次序,
    即可变 \(C_q\) 为 \(D_q\).
    根据反称性,
    \(\det {(C_q)} = \pm \det {(D_q)}\).
    (具体地, 设 \(q\) 在
    \(i_1\), \(\dots\), \(i_r\), \(q\) 中%
    是第~\(u\) 小的数.
    那么, 当 \(f(i_{r+1}) = u\) 时, \(C_q = D_q\).
    当 \(f(i_{r+1}) < u\) 时,
    我们交换行~\(f(i_{r+1})\) 与 \(f(i_{r+1})+1\),
    再交换行~\(f(i_{r+1})+1\) 与 \(f(i_{r+1})+2\),
    \(\dots \dots\),
    再交换行~\(u-1\) 与 \(u\),
    作 \(u - f(i_{r+1})\)~次相邻行的交换,
    即可变 \(C_q\) 为 \(D_q\).
    当 \(f(i_{r+1}) > u\) 时,
    我们交换行~\(f(i_{r+1})\) 与 \(f(i_{r+1})-1\),
    再交换行~\(f(i_{r+1})-1\) 与 \(f(i_{r+1})-2\),
    \(\dots \dots\),
    再交换行~\(u+1\) 与 \(u\),
    作 \(f(i_{r+1}) - u\)~次相邻行的交换,
    即可变 \(C_q\) 为 \(D_q\).)
    因为 \(D_q\) 是 \(A\) 的一个 \(r+1\)~级子阵,
    故其行列式为零.
    从而
    \begin{align*}
        [AW]_{q,1}
        = \det {(C_q)}
        = \pm \det {(D_q)}
        = 0.
    \end{align*}

    综上, 我们找到了一个 \(n \times 1\)~阵 \(W\)
    使 \(AW = 0\), 且 \(W \neq 0\).
\end{proof}

这是本节的另一个重要的结论.

\begin{theorem}
    设 \(A\) 是 \(m \times n\)~阵.
    设 \(A\) 有一个行列式非零的 \(n\)~级子阵
    (此时, \(A\) 当然没有行列式非零的 \(n+1\)~级子阵).
    则 \(AX = 0\) 只有零解.
\end{theorem}

\begin{proof}
    设 \(n \times 1\)~阵 \(C\) 适合 \(AC = 0\).
    设 \(A\)~的 \(n\)~级子阵
    \begin{align*}
        E = A\binom{i_1,i_2,\dots,i_n}{1,2,\dots,n}
    \end{align*}
    的行列式非零,
    其中
    \(1 \leq i_1 < i_2 < \dots < i_n \leq m\).
    因为 \(C\) 适合
    \begin{align*}
        [A]_{1,1} [C]_{1,1} + [A]_{1,2} [C]_{2,1}
        + \dots + [A]_{1,n} [C]_{n,1} = 0, \\
        [A]_{2,1} [C]_{1,1} + [A]_{2,2} [C]_{2,1}
        + \dots + [A]_{2,n} [C]_{n,1} = 0, \\
        \dots \dots \dots
        \dots \dots \dots \dots
        \dots \dots \dots \dots
        \dots \dots \dots \dots
        \dots \dots \dots \dots,
        \\
        [A]_{m,1} [C]_{1,1} + [A]_{m,2} [C]_{2,1}
        + \dots + [A]_{m,n} [C]_{n,1} = 0,
    \end{align*}
    故 \(C\) 当然也适合
    \begin{align*}
        [A]_{i_1,1} [C]_{1,1} + [A]_{i_1,2} [C]_{2,1}
        + \dots + [A]_{i_1,n} [C]_{n,1} = 0, \\
        [A]_{i_2,1} [C]_{1,1} + [A]_{i_2,2} [C]_{2,1}
        + \dots + [A]_{i_2,n} [C]_{n,1} = 0, \\
        \dots \dots \dots
        \dots \dots \dots \dots
        \dots \dots \dots \dots
        \dots \dots \dots \dots
        \dots \dots \dots \dots,
        \\
        [A]_{i_n,1} [C]_{1,1} + [A]_{i_n,2} [C]_{2,1}
        + \dots + [A]_{i_n,n} [C]_{n,1} = 0,
    \end{align*}
    即 \(EC = 0\).
    因为 \(\det {(E)} \neq 0\),
    故, 由 Cramer 公式,
    有 \(C = 0\).
\end{proof}

现在, 我们作一个小结.

\begin{theorem}
    设 \(A\) 是 \(m \times n\)~阵.
    设 \(A\) 有一个行列式非零的 \(r\)~级子阵,
    但没有行列式非零的 \(r+1\)~级子阵.

    (1)
    若 \(r < n\),
    则 \(AX = 0\) 有非零解;

    (2)
    若 \(r = n\),
    则 \(AX = 0\) 只有零解.
\end{theorem}

\begin{theorem}
    设 \(A\) 是 \(m \times n\)~阵.
    设 \(B\) 是 \(m \times 1\)~阵.
    设 \(n \times 1\)~阵 \(C\) 适合 \(AC = B\);
    也就是说, \(AX = B\) 有解.
    设 \(A\) 有一个行列式非零的 \(r\)~级子阵,
    但没有行列式非零的 \(r+1\)~级子阵.

    (1)
    若 \(r < n\),
    则 \(AX = B\) 有无限多个解;

    (2)
    若 \(r = n\),
    则 \(AX = B\) 有唯一的解.
\end{theorem}

\section{\texorpdfstring{由 \(m\)~个 \(n\)~元
      \({\leq} 1\)~次方程作成的方程组 (2)}%
  {由 m 个 n 元 ≤1 次方程作成的方程组 (2)}}

\maldevigalegajxo

前面, 我们讨论了
``\(AX = B\) 有解时, 解是否唯一''
问题.
此事的回答跟 \(A\)~的子阵的行列式有关.
设 \(A\) 有 \(n\)~列.
当 \(A\) 有一个行列式非零的 \(n\)~级子阵时,
此方程组有唯一的解;
当 \(A\) 没有行列式非零的 \(n\)~级子阵时,
其解不但不唯一, 还有无限多个.

接着, 自然地, 我们想讨论线性方程组何时有解.
或许, 一个好的想法是:
我们先讨论 \(AX = B\) 有解时,
\(A\), \(B\) 应适合什么条件;
然后, 反过来,
我们再讨论适合这些条件的 \(A\), \(B\)
是否能使 \(AX = B\) 有解.

我们会用此思想研究此事.
不过, 在此之前, 请允许我介绍一件有用的小事.

\begin{theorem}
    设 \(A\) 是 \(n\)~级阵.
    设 \(p\), \(q\) 是二个不超过 \(n\)~的正整数,
    且 \(p \neq q\).
    设 \(x\) 是一个数.
    作 \(n\)~级阵 \(E\), 其中
    \begin{align*}
        [E]_{i,j}
        = \begin{cases}
              [A]_{i,j},              & j \neq q; \\
              [A]_{i,q} + x[A]_{i,p}, & j = q.
          \end{cases}
    \end{align*}
    (通俗地,
    我们加 \(A\)~的列~\(p\) 的 \(x\)~倍于列~\(q\),
    不改变其他的列,
    得阵~\(E\).)
    则 \(\det {(E)} = \det {(A)}\).

    类似地,
    若我们加 \(A\)~的行~\(p\) 的 \(x\)~倍于行~\(q\),
    不改变其他的行,
    得阵~\(F\),
    则我们也有 \(\det {(F)} = \det {(A)}\).
\end{theorem}

\begin{proof}
    我证明关于列的事;
    我们可用类似的方法证明关于行的事
    (或者, 利用行列式与转置的关系).

    设 \(A = [a_1, a_2, \dots, a_n]\).

    先设 \(p < q\).
    为方便说话, 我们写
    \begin{align*}
        f(u, v)
        = \det {[a_1, \dots, a_{p-1}, u, a_{p+1}, \dots,
                    a_{q-1}, v, a_{q+1}, \dots, a_n]}.
    \end{align*}
    于是, \(f(a_p, a_q)\) 就是 \(\det {(A)}\),
    而 \(f(a_p, a_q + xa_p)\) 就是 \(\det {(E)}\).
    利用多线性与交错性,
    \begin{align*}
        \det {(E)}
        = {} & f(a_p, a_q + xa_p)         \\
        = {} & f(a_p, a_q) + xf(a_p, a_p) \\
        = {} & f(a_p, a_q) + x0           \\
        = {} & f(a_p, a_q)                \\
        = {} & \det {(A)}.
    \end{align*}

    再设 \(p > q\).
    为方便说话, 我们写
    \begin{align*}
        g(u, v)
        = \det {[a_1, \dots, a_{q-1}, u, a_{q+1}, \dots,
                    a_{p-1}, v, a_{p+1}, \dots, a_n]}.
    \end{align*}
    于是, \(g(a_q, a_p)\) 就是 \(\det {(A)}\),
    而 \(g(a_q + xa_p, a_p)\) 就是 \(\det {(E)}\).
    利用多线性与交错性,
    \begin{align*}
        \det {(E)}
        = {} & g(a_q + xa_p, a_p)         \\
        = {} & g(a_q, a_p) + xg(a_p, a_p) \\
        = {} & g(a_q, a_p) + x0           \\
        = {} & g(a_q, a_p)                \\
        = {} & \det {(A)}.
        \qedhere
    \end{align*}
\end{proof}

现在, 我们可以研究,
若 \(AX = B\) 有解,
则 \(A\), \(B\) 应适合什么条件.

\begin{restatable}[]{theorem}{TheoremNecessityForConsistency}
    设 \(A\) 是 \(m \times n\)~阵.
    设 \(B\) 是 \(m \times 1\)~阵.
    设存在 \(n \times 1\)~阵 \(C\)
    适合 \(AC = B\).
    作一个 \(m \times (n+1)\)~阵 \(G\),
    其中
    \begin{align*}
        [G]_{i,j}
        = \begin{cases}
              [A]_{i,j}, & j \leq n;  \\
              [B]_{i,1}, & j = n + 1.
          \end{cases}
    \end{align*}
    (通俗地, 在 \(A\)~的最后一列的右侧加入一列 \(B\),
    得到尺寸较大的阵 \(G\).)
    则一定存在一个非负整数 \(r\), 使
    \(A\) 有一个行列式非零的 \(r\)~级子阵
    (从而 \(G\) 也有一个行列式非零的 \(r\)~级子阵),
    但 \(G\) 没有行列式非零的 \(r+1\)~级子阵
    (从而 \(A\) 也没有行列式非零的 \(r+1\)~级子阵).
\end{restatable}

\begin{proof}
    我们知道, 存在一个非负整数 \(r\),
    使 \(A\) 有一个行列式非零的 \(r\)~级子阵,
    但 \(A\) 没有行列式非零的 \(r+1\)~级子阵.
    我们证明,
    \(G\) 也没有行列式非零的 \(r+1\)~级子阵.

    若 \(G\) 根本没有 \(r+1\)~级子阵,
    则 \(G\) 当然也没有行列式非零的 \(r+1\)~级子阵.

    若 \(G\) 有 \(r+1\)~级子阵,
    那么,
    对 \(G\)~的每一个 \(r+1\)~级子阵来说,
    要么 \(G\) 的列~\(n+1\) 被选中,
    要么 \(G\) 的列~\(n+1\) 不被选中.

    若 \(G\) 的列~\(n+1\) 不被选中,
    那么, 这个子阵就是 \(A\)~的 \(r+1\)~级子阵,
    故其行列式为零.

    若 \(G\) 的列~\(n+1\) 被选中,
    我们说明,
    这个子阵的行列式是
    \(A\)~的一些 \(r+1\)~级子阵的行列式的倍的和,
    故其行列式仍为零.
    因为 \(AC = B\), 故,
    对任何不超过 \(m\) 的正整数 \(i\),
    \begin{align*}
        [B]_{i,1}
            = [A]_{i,1} [C]_{1,1} + [A]_{i,2} [C]_{2,1}
        + \dots + [A]_{i,n} [C]_{n,1},
    \end{align*}
    即
    \begin{align*}
        [G]_{i,n+1}
        = [G]_{i,1} [C]_{1,1} + [G]_{i,2} [C]_{2,1}
        + \dots + [G]_{i,n} [C]_{n,1}.
    \end{align*}
    设这个子阵为
    \begin{align*}
        D = G\binom{i_1,\dots,i_r,i_{r+1}}{j_1,\dots,j_r,n+1},
    \end{align*}
    其中
    \(1 \leq i_1 < \dots < i_r < i_{r+1} \leq m\),
    \(1 \leq j_1 < \dots < j_r \leq n\).
    我们加 \(D\) 的列~\(1\) 的 \(-[C]_{j_1,1}\)~倍%
    于列~\(r+1\),
    得阵~\(D_1\),
    则 \(\det {(D_1)} = \det {(D)}\).
    我们加 \(D_1\) 的列~\(2\) 的 \(-[C]_{j_2,1}\)~倍%
    于列~\(r+1\),
    得阵~\(D_2\),
    则 \(\det {(D_2)} = \det {(D_1)} = \det {(D)}\).
    \(\dots \dots\)
    我们加 \(D_{r-1}\) 的列~\(r\) 的 \(-[C]_{j_r,1}\)~倍%
    于列~\(r+1\),
    得阵~\(D_r\),
    则 \(\det {(D_r)} = \det {(D_{r-1})} = \det {(D)}\).
    注意到, \(D_r\)~的列
    \(1\), \(2\), \(\dots\), \(r\)
    分别跟 \(D\)~的列
    \(1\), \(2\), \(\dots\), \(r\)
    相等;
    不过,
    \begin{align*}
        [D_r]_{s,r+1}
        = {} &
        [G]_{i_s,n+1}
        - \sum_{\ell = 1}^{r}
        {[G]_{i_s,j_\ell} [C]_{j_\ell,1}}
        \\
        = {} &
        \sum_{1 \leq t \leq n}
        {[G]_{i_s,t} [C]_{t,1}}
        -
        \sum_{\substack{1 \leq t \leq n \\
                \text{\(t\) 等于某个 \(j_\ell\)}}}
        {[G]_{i_s,t} [C]_{t,1}}
        \\
        = {} &
        \sum_{\substack{1 \leq t \leq n \\
                \text{\(t\) 不等于任何 \(j_\ell\)}}}
        {[G]_{i_s,t} [C]_{t,1}}.
    \end{align*}
    为方便说话, 我们记 \((r+1) \times n\)~阵
    \begin{align*}
        A\binom{i_1,\dots,i_r,i_{r+1}}{1,2,\dots,n}
    \end{align*}
    的列~\(1\), \(2\), \(\dots\), \(n\) 为
    \(h_1\), \(h_2\), \(\dots\), \(h_n\).
    于是, \(D_r\)~的列~\(1\), \(2\), \(\dots\), \(r\) 就是
    \(h_{j_1}\), \(h_{j_2}\), \(\dots\), \(h_{j_r}\),
    而 \(D_r\)~的列~\(r+1\) 是
    \begin{align*}
        \sum_{\substack{1 \leq t \leq n \\
                \text{\(t\) 不等于任何 \(j_\ell\)}}}
        {[C]_{t,1} h_t}.
    \end{align*}
    从而, 利用多线性,
    \begin{align*}
        \det {(D)}
        = {} &
        \det {(D_r)}
        \\
        = {} &
        \det {
            \left[
                h_{j_1}, \dots, h_{j_r},
        \sum_{\substack{1 \leq t \leq n \\
                        \text{\(t\) 不等于任何 \(j_\ell\)}}}
                {[C]_{t,1} h_t}
                \right]
        }
        \\
        = {} &
        \sum_{\substack{1 \leq t \leq n \\
                \text{\(t\) 不等于任何 \(j_\ell\)}}}
        {
            [C]_{t,1} \det {[h_{j_1}, \dots, h_{j_r}, h_t]}
        }.
    \end{align*}
    适当地交换
    \([h_{j_1}, \dots, h_{j_r}, h_t]\)
    的列的次序,
    利用反称性, 有
    \begin{align*}
        \det {(D)}
        = {} &
        \sum_{\substack{1 \leq t \leq n \\
                \text{\(t\) 不等于任何 \(j_\ell\)}}}
        {
            [C]_{t,1} \det {[h_{j_1}, \dots, h_{j_r}, h_t]}
        }
        \\
        = {} &
        \sum_{\substack{1 \leq t \leq n \\
                \text{\(t\) 不等于任何 \(j_\ell\)}}}
        {
            (\pm)\, [C]_{t,1}
            \det {\left(
                A\binom{i_1,\dots,i_r,i_{r+1}}{j_1,\dots,j_r,t}
                \right)}
        }
        \\
        = {} &
        \sum_{\substack{1 \leq t \leq n \\
                \text{\(t\) 不等于任何 \(j_\ell\)}}}
        {
            (\pm)\, [C]_{t,1} \,0
        }
        \\
        = {} &
        0.
        \qedhere
    \end{align*}
\end{proof}

\section{\texorpdfstring{由 \(m\)~个 \(n\)~元
      \({\leq} 1\)~次方程作成的方程组 (3)}%
  {由 m 个 n 元 ≤1 次方程作成的方程组 (3)}}

\maldevigalegajxo

前面, 我们讨论了
\(AX = B\) 有解时,
\(A\), \(B\) 应适合的条件:

\TheoremNecessityForConsistency*

那么, 反过来,
若 \(A\) 有一个行列式非零的 \(r\)~级子阵,
但 \(G\) 没有行列式非零的 \(r+1\)~级子阵,
则 \(AX = B\) 是否有解?
% 幸运地,
此事的回答是 ``是''.
不过, 为了论证此事,
我们要作一些准备.

\begin{theorem}
    设 \(A\) 是 \(m \times n\)~阵,
    且 \(A \neq 0\).
    设 \(A\) 有一个行列式非零的 \(r\)~级子阵
    \begin{align*}
        A_r = A\binom{i_1,\dots,i_r}{j_1,\dots,j_r}
    \end{align*}
    (其中
    \(1 \leq i_1 < \dots < i_r \leq m\),
    \(1 \leq j_1 < \dots < j_r \leq n\)),
    但 \(A\) 没有行列式非零的 \(r+1\)~级子阵.
    那么, 对任何不超过 \(m\) 的正整数 \(p\),
    一定存在 \(r\)~个数
    \(d_{p,1}\), \(d_{p,2}\), \(\dots\), \(d_{p,r}\),
    使对任何不超过 \(n\) 的正整数 \(q\),
    \begin{align*}
        [A]_{p,q}
        = {} &
        [A]_{i_1,q} d_{p,1}
        + [A]_{i_2,q} d_{p,2}
        + \dots
        + [A]_{i_r,q} d_{p,r}
        \\
        = {} &
        \sum_{s = 1}^{r} {[A]_{i_s,q} d_{p,s}}.
    \end{align*}
\end{theorem}

我们也可如此说前面的结论:

\begin{theorem}
    设 \(A\) 是 \(m \times n\)~阵,
    且 \(A \neq 0\).
    设 \(A\) 有一个行列式非零的 \(r\)~级子阵
    \begin{align*}
        A_r = A\binom{i_1,\dots,i_r}{j_1,\dots,j_r}
    \end{align*}
    (其中
    \(1 \leq i_1 < \dots < i_r \leq m\),
    \(1 \leq j_1 < \dots < j_r \leq n\)),
    但 \(A\) 没有行列式非零的 \(r+1\)~级子阵.
    设 \(A\)~的%
    行~\(1\), \(2\), \(\dots\), \(m\)
    为 \(a_1\), \(a_2\), \(\dots\), \(a_m\).
    那么, 对任何不超过 \(m\) 的正整数 \(p\),
    一定存在 \(r\)~个数
    \(d_{p,1}\), \(d_{p,2}\), \(\dots\), \(d_{p,r}\),
    使
    \begin{align*}
        a_p
        = {} &
        d_{p,1} a_{i_1}
        + d_{p,2} a_{i_2}
        + \dots
        + d_{p,r} a_{i_r}
        \\
        = {} &
        \sum_{s = 1}^{r} {d_{p,s} a_{i_s}}.
    \end{align*}
\end{theorem}

通俗地, 这个定理说,
任给一个非零阵 \(A\),
我们总能找出它的某 \(r\)~行
\(a_{i_1}\), \(a_{i_2}\), \(\dots\), \(a_{i_r}\),
使 \(A\)~的每一行都可被写为%
这 \(r\)~行的数乘的和,
且由这 \(r\)~行作成的子阵一定有一个%
行列式非零的 \(r\)~级子阵.

\begin{example}
    设
    \begin{align*}
        A =
        \begin{bmatrix}
            1 & 0 & 0 & 0 & 0 & -1 \\
            0 & 1 & 0 & 0 & 0 & 1  \\
            0 & 0 & 1 & 0 & 0 & 0  \\
            0 & 1 & 1 & 0 & 0 & 1  \\
        \end{bmatrix}.
    \end{align*}
    不难看出, \(A\) 有一个 \(3\)~级子阵, 其行列式非零:
    \begin{align*}
        \det {\left(
            A\binom{1,2,3}{1,2,3}
            \right)} = 1.
    \end{align*}
    不过, \(A\) 没有行列式非零的 \(4\)~级子阵.
    我们考虑 \(A\) 的列~\(4\), \(5\).
    任取 \(A\)~的一个 \(4\)~级子阵.
    若 \(A\) 的列~\(4\) 或列~\(5\) 被选中,
    那么其行列式显然为零.
    若 \(A\) 的列~\(4\) 与列~\(5\) 都不被选中,
    则这个子阵一定是
    \begin{align*}
        A\binom{1,2,3,4}{1,2,3,6}
        = \begin{bmatrix}
              1 & 0 & 0 & -1 \\
              0 & 1 & 0 & 1  \\
              0 & 0 & 1 & 0  \\
              0 & 1 & 1 & 1  \\
          \end{bmatrix}.
    \end{align*}
    不难算出, 它的行列式为零.

    我们说, \(A\) 的每一行一定可被写为
    \(A\) 的前 \(3\)~行的数乘的和.
    设 \(A\)~的%
    行~\(1\), \(2\), \(3\), \(4\)
    为 \(a_1\), \(a_2\), \(a_3\), \(a_4\).
    则
    \begin{align*}
        a_1 = 1a_1 + 0a_2 + 0a_3, \\
        a_2 = 0a_1 + 1a_2 + 0a_3, \\
        a_3 = 0a_1 + 0a_2 + 1a_3, \\
        a_4 = 0a_1 + 1a_2 + 1a_3.
    \end{align*}
    并且, 由这 \(3\)~行作成的子阵一定有一个%
    行列式非零的 \(3\)~级子阵:
    \(3\)~级单位阵就是一个.

    顺便一提,
    \(A\) 的每一行当然可被写为
    \(A\) 的前 \(4\)~行的数乘的和:
    \begin{align*}
        a_1 = 1a_1 + 0a_2 + 0a_3 + 0a_4, \\
        a_2 = 0a_1 + 1a_2 + 0a_3 + 0a_4, \\
        a_3 = 0a_1 + 0a_2 + 1a_3 + 0a_4, \\
        a_4 = 0a_1 + 0a_2 + 0a_3 + 1a_4.
    \end{align*}
    可是, 由这 \(4\)~行作成的子阵没有%
    行列式非零的 \(4\)~级子阵.
\end{example}

\begin{proof}
    任取不超过 \(m\) 的正整数 \(p\).

    若 \(p\) 等于某个 \(i_s\)
    (\(s = 1\), \(2\), \(\dots\), \(r\)),
    我们取
    \begin{align*}
        d_{p,v}
        = \begin{cases}
              1, & v = s;    \\
              0, & v \neq s.
          \end{cases}
    \end{align*}
    于是, 对任何不超过 \(n\) 的正整数 \(q\),
    \begin{align*}
        [A]_{i_1,q} d_{p,1}
        + [A]_{i_2,q} d_{p,2}
        + \dots
        + [A]_{i_r,q} d_{p,r}
            = [A]_{i_s,q} 1
            = [A]_{p,q}.
    \end{align*}

    下设 \(p\) 不等于任何 \(i_s\).

    考虑%
    由 \(r\)~个 \(r\)~元 \({\leq} 1\)~次方程作成的方程组
    \begin{align*}
        \begin{cases}
            [A]_{i_1,j_1} x_1 + [A]_{i_2,j_1} x_2
            + \dots + [A]_{i_r,j_1} x_r = [A]_{p,j_1},
            \\
            [A]_{i_1,j_2} x_1 + [A]_{i_2,j_2} x_2
            + \dots + [A]_{i_r,j_2} x_r = [A]_{p,j_2},
            \\
            \dots
            \dots \dots \dots \dots
            \dots \dots \dots \dots
            \dots \dots \dots \dots
            \dots \dots \dots \dots,
            \\
            [A]_{i_1,j_r} x_1 + [A]_{i_2,j_r} x_2
            + \dots + [A]_{i_r,j_r} x_r = [A]_{p,j_r},
        \end{cases}
    \end{align*}
    即
    \begin{align*}
        A_r^{\mathrm{T}}
        \begin{bmatrix}
            x_1    \\
            x_2    \\
            \vdots \\
            x_r    \\
        \end{bmatrix}
        =
        \begin{bmatrix}
            [A]_{p,j_1} \\
            [A]_{p,j_2} \\
            \vdots      \\
            [A]_{p,j_r} \\
        \end{bmatrix}.
    \end{align*}
    因为
    \(\det {(A_r^{\mathrm{T}})}
    = \det {(A_r)} \neq 0\),
    故, 由 Cramer 公式,
    存在 \(r\)~个数
    \(d_{p,1}\), \(d_{p,2}\), \(\dots\), \(d_{p,r}\),
    使
    \begin{align*}
        [A]_{i_1,j_1} d_{p,1} + [A]_{i_2,j_1} d_{p,2}
        + \dots + [A]_{i_r,j_1} d_{p,r} = [A]_{p,j_1},
        \\
        [A]_{i_1,j_2} d_{p,1} + [A]_{i_2,j_2} d_{p,2}
        + \dots + [A]_{i_r,j_2} d_{p,r} = [A]_{p,j_2},
        \\
        \dots \dots \dots
        \dots \dots \dots \dots
        \dots \dots \dots \dots
        \dots \dots \dots \dots
        \dots \dots \dots \dots,
        \\
        [A]_{i_1,j_r} d_{p,1} + [A]_{i_2,j_r} d_{p,2}
        + \dots + [A]_{i_r,j_r} d_{p,r} = [A]_{p,j_r}.
    \end{align*}
    我们由此证明, 对任何不超过 \(n\) 的正整数 \(q\),
    \begin{align*}
        [A]_{p,q}
            =
            [A]_{i_1,q} d_{p,1}
        + [A]_{i_2,q} d_{p,2}
        + \dots
        + [A]_{i_r,q} d_{p,r}.
    \end{align*}

    若 \(q\) 等于某个 \(j_\ell\), 显然.
    下设 \(q\) 不等于任何 \(j_\ell\).

    考虑 \(A\) 的 \(r+1\)~级子阵
    \begin{align*}
        E =
        A\binom{i_1,\dots,i_r,p}{j_1,\dots,j_r,q}.
    \end{align*}
    我们用 ``算二次 (dufoje)'' 思想,
    证我们想要的等式.

    一方面, 我们知道, 既然
    \(A\) 没有行列式非零的 \(r+1\)~级子阵,
    故 \(\det {(E)} = 0\).

    另一方面, 我们也可适当地作一些辅助阵,
    它们的行列式等于 \(E\) 的行列式;
    从而, 计算这些辅助阵的行列式,
    也就相当于计算 \(E\) 的行列式.
    为方便, 我们记
    \(i_{r+1} = p\),
    \(j_{r+1} = q\).
    设 \(i_s\) 是在
    \(i_1\), \(\dots\), \(i_r\), \(i_{r+1}\)
    中第~\(f(i_s)\)~小的数;
    设 \(j_\ell\) 是在
    \(j_1\), \(\dots\), \(j_r\), \(j_{r+1}\)
    中第~\(g(j_\ell)\)~小的数.
    我们加 \(E\) 的行~\(f(i_1)\) 的 \(-d_{p,1}\)~倍%
    于行~\(f(p)\),
    得阵~\(E_1\),
    则 \(\det {(E_1)} = \det {(E)}\).
    我们加 \(E_1\) 的行~\(f(i_2)\) 的 \(-d_{p,2}\)~倍%
    于行~\(f(p)\),
    得阵~\(E_2\),
    则 \(\det {(E_2)} = \det {(E_1)} = \det {(E)}\).
    \(\dots \dots\)
    我们加 \(E_{r-1}\) 的行~\(f(i_r)\) 的 \(-d_{p,r}\)~倍%
    于行~\(f(p)\),
    得阵~\(E_r\),
    则 \(\det {(E_r)} = \det {(E_{r-1})} = \det {(E)}\).
    注意到, \(E_r\)~的行
    \(f(i_1)\), \(f(i_2)\), \(\dots\), \(f(i_r)\)
    分别跟 \(E\)~的行
    \(f(i_1)\), \(f(i_2)\), \(\dots\), \(f(i_r)\)
    相等;
    不过,
    \begin{align*}
        [E_r]_{f(p),g(v)}
        = [A]_{p,v} -
        \sum_{s = 1}^{r}
        {[A]_{i_s,v} d_{p,s}},
    \end{align*}
    其中 \(v = j_1\), \(\dots\), \(j_r\), \(q\).
    所以, 当
    % \(t \neq r+1\)
    \(g(v) \neq g(q)\)
    时,
    % \([E_r]_{f(p),g(j_t)} = 0\).
    \([E_r]_{f(p),g(v)} = 0\).
    我们按行~\(f(p)\) 展开 \(E_r\) 的行列式,
    有
    \begin{align*}
        \det {(E_r)}
        = {} &
        (-1)^{f(p)+g(q)} [E_r]_{f(p),g(q)}
        \det {(E_r (f(p)|g(q)))}
        \\
        = {} &
        (-1)^{f(p)+g(q)} \det {(A_r)}
        \left(
        [A]_{p,q} -
        \sum_{s = 1}^{r}
        {[A]_{i_s,q} d_{p,s}}
        \right).
    \end{align*}

    回想, \(0 = \det {(E)}\),
    且 \(\det {(E)} = \det {(E_r)}\).
    比较二次计算的结果, 我们应有
    \begin{align*}
        0 =
        (-1)^{f(p)+g(q)} \det {(A_r)}
        \left(
        [A]_{p,q} -
        \sum_{s = 1}^{r}
        {[A]_{i_s,q} d_{p,s}}
        \right).
    \end{align*}
    注意到,
    \((-1)^{f(p)+g(q)} \det {(A_r)} \neq 0\),
    故
    \begin{equation*}
        [A]_{p,q} -
        \sum_{s = 1}^{r}
        {[A]_{i_s,q} d_{p,s}} = 0.
        \qedhere
    \end{equation*}
\end{proof}

现在, 我们可以证明本节的重要结论了.

\begin{theorem}
    设 \(A\) 是 \(m \times n\)~阵.
    设 \(B\) 是 \(m \times 1\)~阵.
    作一个 \(m \times (n+1)\)~阵 \(G\),
    其中
    \begin{align*}
        [G]_{i,j}
        = \begin{cases}
              [A]_{i,j}, & j \leq n;  \\
              [B]_{i,1}, & j = n + 1.
          \end{cases}
    \end{align*}
    (通俗地, 在 \(A\)~的最后一列的右侧加入一列 \(B\),
    得到尺寸较大的阵 \(G\).)
    设
    \(A\) 有一个行列式非零的 \(r\)~级子阵
    (从而 \(G\) 也有一个行列式非零的 \(r\)~级子阵),
    但 \(G\) 没有行列式非零的 \(r+1\)~级子阵
    (从而 \(A\) 也没有行列式非零的 \(r+1\)~级子阵).
    则存在一个 \(n \times 1\)~阵 \(C\),
    使 \(AC = B\).
\end{theorem}

\begin{proof}
    先设 \(A = 0\).
    那么, \(A\) 有一个行列式非零的 \(0\)~级子阵,
    但 \(G\) 没有行列式非零的 \(1\)~级子阵.
    所以 \(G = 0\).
    所以 \(B = 0\).
    那么, 任何一个 \(n \times 1\)~阵 \(C\)
    都适合 \(AC = B\).

    下设 \(A \neq 0\).
    设 \(A\) 的 \(r\)~级子阵
    \begin{align*}
        A_r = A\binom{i_1,\dots,i_r}{j_1,\dots,j_r}
    \end{align*}
    (其中
    \(1 \leq i_1 < \dots < i_r \leq m\),
    \(1 \leq j_1 < \dots < j_r \leq n\))
    的行列式非零.
    \(A_r\) 当然也是 \(G\)~的子阵, 且
    \begin{align*}
        A\binom{i_1,\dots,i_r}{j_1,\dots,j_r}
        =
        G\binom{i_1,\dots,i_r}{j_1,\dots,j_r}.
    \end{align*}
    所以, 对任何不超过 \(m\) 的正整数 \(p\),
    一定存在 \(r\)~个数
    \(d_{p,1}\), \(d_{p,2}\), \(\dots\), \(d_{p,r}\),
    使对任何不超过 \(n+1\) 的正整数 \(q\),
    \begin{align*}
        [G]_{p,q}
        = {} &
        [G]_{i_1,q} d_{p,1}
        + [G]_{i_2,q} d_{p,2}
        + \dots
        + [G]_{i_r,q} d_{p,r}
        \\
        = {} &
        \sum_{s = 1}^{r} {[G]_{i_s,q} d_{p,s}}.
    \end{align*}

    考虑%
    由 \(r\)~个 \(n\)~元 \({\leq} 1\)~次方程作成的方程组
    \begin{align*}
        \begin{cases}
            [A]_{i_1,1} x_1 + [A]_{i_1,2} x_2 + \dots
            + [A]_{i_1,n} x_n = [B]_{i_1,1},
            \\
            [A]_{i_2,1} x_1 + [A]_{i_2,2} x_2 + \dots
            + [A]_{i_2,n} x_n = [B]_{i_2,1},
            \\
            \dots
            \dots \dots \dots \dots
            \dots \dots \dots \dots
            \dots \dots \dots \dots
            \dots \dots \dots \dots,
            \\
            [A]_{i_r,1} x_1 + [A]_{i_r,2} x_2 + \dots
            + [A]_{i_r,n} x_n = [B]_{i_r,1}.
        \end{cases}
    \end{align*}
    我们说, 这个方程组一定有解.

    若 \(r = n\), 则这是一个%
    由 \(n\)~个 \(n\)~元 \({\leq} 1\)~次方程作成的方程组.
    因为 \(\det {(A_r)} \neq 0\),
    故, 由 Cramer 公式,
    此方程组有一个 (唯一的) 解.

    若 \(r < n\),
    从 \(1\), \(2\), \(\dots\), \(n\)
    去除 \(j_1\), \(j_2\), \(\dots\), \(j_r\)
    后, 还剩 \(n - r\)~个数.
    我们从小到大地叫这 \(n - r\)~个数为
    \(j_{r+1}\), \(\dots\), \(j_n\).
    我们可改写此方程组为
    \begin{align*}
        \sum_{\ell = 1}^{r}
        {[A]_{p,j_\ell} x_{j_\ell}}
            = [B]_{p,1}
        - \sum_{r < \ell \leq n}
        {[A]_{p,j_\ell} x_{j_\ell}},
    \end{align*}
    其中 \(p = i_1\), \(i_2\), \(\dots\), \(i_r\),
    下同.

    设
    \(f_{j_{r+1}}\), \(\dots\), \(f_{j_r}\)
    为任何的 \(n - r\) 个数.
    我们考虑%
    由 \(r\)~个 \(r\)~元 \({\leq} 1\)~次方程作成的方程组
    \begin{align*}
        \sum_{\ell = 1}^{r}
        {[A]_{p,j_\ell} y_\ell}
        = [B]_{p,1}
        - \sum_{r < \ell \leq n}
        {[A]_{p,j_\ell} f_{j_\ell}}.
    \end{align*}
    利用阵等式, 我们可写
    \begin{align*}
        A_r
        \begin{bmatrix}
            y_1    \\
            y_2    \\
            \vdots \\
            y_r    \\
        \end{bmatrix}
        =
        \begin{bmatrix}
            z_{i_1} \\
            z_{i_2} \\
            \vdots  \\
            z_{i_r} \\
        \end{bmatrix},
    \end{align*}
    其中
    \begin{align*}
        z_p = [B]_{p,1}
        - \sum_{r < \ell \leq n}
        {[A]_{p,j_\ell} f_{j_\ell}}.
    \end{align*}
    因为 \(\det {(A_r)} \neq 0\),
    故, 由 Cramer 公式,
    存在 \(r\) 个数
    \(f_{j_1}\), \(f_{j_2}\), \(\dots\), \(f_{j_r}\),
    使
    \begin{align*}
        \sum_{\ell = 1}^{r}
        {[A]_{p,j_\ell} f_{j_\ell}}
            = [B]_{p,1}
        - \sum_{r < \ell \leq n}
        {[A]_{p,j_\ell} f_{j_\ell}}.
    \end{align*}
    从而
    \begin{align*}
        \sum_{\ell = 1}^{n}
        {[A]_{p,j_\ell} f_{j_\ell}}
            = [B]_{p,1}.
    \end{align*}

    设 \(n\) 个数
    \(c_1\), \(c_2\), \(\dots\), \(c_n\)
    适合
    \begin{align*}
        [A]_{i_1,1} c_1 + [A]_{i_1,2} c_2 + \dots
        + [A]_{i_1,n} c_n = [B]_{i_1,1},
        \\
        [A]_{i_2,1} c_1 + [A]_{i_2,2} c_2 + \dots
        + [A]_{i_2,n} c_n = [B]_{i_2,1},
        \\
        \dots
        \dots \dots \dots \dots
        \dots \dots \dots \dots
        \dots \dots \dots \dots
        \dots \dots \dots \dots,
        \\
        [A]_{i_r,1} c_1 + [A]_{i_r,2} c_2 + \dots
        + [A]_{i_r,n} c_n = [B]_{i_r,1}.
    \end{align*}
    作 \(n \times 1\)~阵 \(C\),
    其中 \([C]_{i,1} = c_i\).
    我们证明, \(AC = B\).

    若 \(i\) 等于某 \(i_s\), 则
    \begin{align*}
        [AC]_{i,1}
        = {} &
        \sum_{\ell = 1}^{n}
        {[A]_{i,\ell} [C]_{\ell,1}}
        \\
        = {} &
        \sum_{\ell = 1}^{n}
        {[A]_{i,\ell} c_\ell}
        \\
        = {} & [B]_{i,1}.
    \end{align*}
    若 \(i\) 不等于任何一个 \(i_s\), 则
    \begin{align*}
        [AC]_{i,1}
        = {} &
        \sum_{\ell = 1}^{n}
        {[A]_{i,\ell} [C]_{\ell,1}}
        \\
        = {} &
        \sum_{\ell = 1}^{n}
        {[G]_{i,\ell} [C]_{\ell,1}}
        \\
        = {} &
        \sum_{\ell = 1}^{n}
        {
        \left( \sum_{s = 1}^{r}
        {
            [G]_{i_s,\ell} d_{i,s}
        } \right) [C]_{\ell,1}
        }
        \\
        = {} &
        \sum_{\ell = 1}^{n}
        {
        \sum_{s = 1}^{r}
        {
        [G]_{i_s,\ell} d_{i,s} [C]_{\ell,1}
        }
        }
        \\
        = {} &
        \sum_{s = 1}^{r}
        {
        \sum_{\ell = 1}^{n}
        {
        [G]_{i_s,\ell} d_{i,s} [C]_{\ell,1}
        }
        }
        \\
        = {} &
        \sum_{s = 1}^{r}
        {
        \sum_{\ell = 1}^{n}
        {
        [A]_{i_s,\ell} d_{i,s} c_\ell
        }
        }
        \\
        = {} &
        \sum_{s = 1}^{r}
        {
        \sum_{\ell = 1}^{n}
        {
        [A]_{i_s,\ell} c_\ell d_{i,s}
        }
        }
        \\
        = {} &
        \sum_{s = 1}^{r}
        {
        \left(\sum_{\ell = 1}^{n}
        {
            [A]_{i_s,\ell} c_\ell
        }\right)
        d_{i,s}
        }
        \\
        = {} &
        \sum_{s = 1}^{r}
        {[B]_{i_s,1} d_{i,s}}
        \\
        = {} &
        \sum_{s = 1}^{r}
        {[G]_{i_s,n+1} d_{i,s}}
        \\
        = {} &
        [G]_{i,n+1}
        \\
        = {} &
        [B]_{i,1}.
        \qedhere
    \end{align*}
\end{proof}

结合前几节的讨论, 我们可以得到判断
\(AX = B\) 是否有解,
与有解时其解是否唯一的方法
(定性的理论, 完整版):

\begin{restatable}[]{theorem}%
    {TheoremQualitativeTheoryOfLinearSystemOfEquations}
    设 \(A\) 是 \(m \times n\)~阵.
    设 \(B\) 是 \(m \times 1\)~阵.
    作一个 \(m \times (n+1)\)~阵 \(G\),
    其中
    \begin{align*}
        [G]_{i,j}
        = \begin{cases}
              [A]_{i,j}, & j \leq n;  \\
              [B]_{i,1}, & j = n + 1.
          \end{cases}
    \end{align*}
    (通俗地, 在 \(A\)~的最后一列的右侧加入一列 \(B\),
    得到尺寸较大的阵 \(G\).)
    设
    \(A\) 有一个行列式非零的 \(r\)~级子阵,
    但没有行列式非零的 \(r+1\)~级子阵.

    (1)
    若 \(G\)~也没有行列式非零的 \(r+1\)~级子阵,
    则 \(AX = B\) 有解.
    进一步地, 若 \(r < n\),
    则 \(AX = B\) 有无限多个解;
    若 \(r = n\),
    则 \(AX = B\) 有唯一的解.

    (2)
    若 \(G\)~有一个行列式非零的 \(r+1\)~级子阵,
    则 \(AX = B\) 无解.
\end{restatable}

\section{\texorpdfstring{由 \(m\)~个 \(n\)~元
      \({\leq} 1\)~次方程作成的方程组 (4)}%
  {由 m 个 n 元 ≤1 次方程作成的方程组 (4)}}

% 本节是 ``选学内容中的选学内容'';
% 换句话说, 您不学它, 不影响您理解别的选学内容.
% 毕竟, 本节的公式, 几乎无用,
% 因为, 若您想用它计算线性方程组的解,
% 您要计算好几个阵的行列式.
% 计算一个阵的行列式已经是一件较复杂的事了,
% 计算好几个阵的行列式自然就是一件更复杂的事.
% 不过, 本节自然是可以作为阅读材料的.

\maldevigalegajxo

前面, 我们得到了线性方程组的解的定性的理论:

\TheoremQualitativeTheoryOfLinearSystemOfEquations*

现在, 我们作定量的讨论:
当 \(AX = B\) 有解时,
我们试作出公式, 以表示它的解.

设 \(A\) 是一个 \(m \times n\)~阵.
设 \(B\) 是一个 \(m \times 1\)~阵.
设 \(AX = B\) 有解.

若 \(A = 0\),
则因 \(AX = B\) 有解,
必有 \(B = 0\).
此时, 显然, 每一个 \(n \times 1\)~阵都是解.
下设 \(A \neq 0\).
设 \(A\) 有一个行列式非零的 \(r\)~级子阵
\begin{align*}
    T = A\binom{i_1,\dots,i_r}{j_1,\dots,j_r}
\end{align*}
(其中
\(1 \leq i_1 < \dots < i_r \leq m\),
\(1 \leq j_1 < \dots < j_r \leq n\)),
但没有行列式非零的 \(r+1\)~级子阵.
为说话方便, 我们作一个 \(r \times n\)~阵
\begin{align*}
    U = A\binom{i_1,\dots,i_r}{1,\dots,n},
\end{align*}
与一个 \(r \times 1\)~阵
\begin{align*}
    V = B\binom{i_1,\dots,i_r}{1}
    =
    \begin{bmatrix}
        [B]_{i_1,1} \\
        [B]_{i_2,1} \\
        \vdots      \\
        [B]_{i_r,1} \\
    \end{bmatrix}.
\end{align*}
注意到, \(U\) 有一个行列式非零的 \(r\)~级子阵 \(T\).

我们回想上节的讨论.
为了解方程组 \(AX = B\), 即
\begin{align*}
    \begin{cases}
        [A]_{1,1} x_1 + [A]_{1,2} x_2 + \dots
        + [A]_{1,n} x_n = [B]_{1,1},
        \\
        [A]_{2,1} x_1 + [A]_{2,2} x_2 + \dots
        + [A]_{2,n} x_n = [B]_{2,1},
        \\
        \dots
        \dots \dots \dots \dots
        \dots \dots \dots \dots
        \dots \dots \dots \dots
        \dots \dots \dots \dots,
        \\
        [A]_{m,1} x_1 + [A]_{m,2} x_2 + \dots
        + [A]_{m,n} x_n = [B]_{m,1},
    \end{cases}
\end{align*}
我们考虑了由这 \(m\)~个方程的%
第~\(i_1\), \(i_2\), \(\dots\), \(i_r\)~个方程%
作成的方程组
\begin{align*}
    \begin{cases}
        [A]_{i_1,1} x_1 + [A]_{i_1,2} x_2 + \dots
        + [A]_{i_1,n} x_n = [B]_{i_1,1},
        \\
        [A]_{i_2,1} x_1 + [A]_{i_2,2} x_2 + \dots
        + [A]_{i_2,n} x_n = [B]_{i_2,1},
        \\
        \dots
        \dots \dots \dots \dots
        \dots \dots \dots \dots
        \dots \dots \dots \dots
        \dots \dots \dots \dots,
        \\
        [A]_{i_r,1} x_1 + [A]_{i_r,2} x_2 + \dots
        + [A]_{i_r,n} x_n = [B]_{i_r,1},
    \end{cases}
\end{align*}
即 \(UX = V\).
当时, 我们已经证明了,
\(UX = V\) 的解都是 \(AX = B\) 的解.
反过来, \(AX = B\) 的解显然都是 \(UX = V\) 的解,
因为后者的方程全部都是来自前者的.
这么看来, \(AX = B\) 跟 \(UX = V\) 有着相同的解.
所以, 研究 \(AX = B\) 的解的公式,
相当于研究 \(UX = V\) 的解的公式.

若 \(r = n\),
则我们直接用 Cramer 公式,
即可写出 \(UX = V\) 的唯一的解
(注意, 此时 \(U\) 是一个 \(r \times r\)~阵)
\begin{align*}
    X = \frac{1}{\det {(U)}} \operatorname{adj} {(U)}\,V.
\end{align*}
我们也可较直接地表达此解.
设 \(U\{k,V\}\) 是以 \(r \times 1\)~阵 \(V\)
代 \(U\) 的列~\(k\) 后得到的阵.
则
\begin{align*}
    x_k = \frac{\det {(U\{k,V\})}}{\det {(U)}}.
\end{align*}

下设 \(r < n\).
我们试写出方程组 \(UX = V\) 的所有的解.

\begin{theorem}
    设 \(V\) 是一个 \(r \times 1\)~阵.
    设 \(U\) 是一个 \(r \times n\)~阵,
    其中 \(r < n\),
    且 \(U\) 有一个行列式非零的 \(r\)~级子阵
    \begin{align*}
        T = U\binom{1,2,\dots,r}{j_1,j_2,\dots,j_r},
    \end{align*}
    其中 \(1 \leq j_1 < \dots < j_r \leq n\).
    从 \(1\), \(2\), \(\dots\), \(n\)
    去除 \(j_1\), \(j_2\), \(\dots\), \(j_r\)
    后, 还剩 \(n - r\)~个数.
    我们从小到大地叫这 \(n - r\)~个数为
    \(j_{r+1}\), \(\dots\), \(j_n\).
    再设 \(U\)~的%
    列~\(1\), \(2\), \(\dots\), \(n\)
    分别是 \(u_1\), \(u_2\), \(\dots\), \(u_n\).

    (1)
    设 \(c_{j_{r+1}}\), \(\dots\), \(c_{j_n}\)
    是 \(n-r\)~个常数.
    作 \(n \times 1\)~阵 \(C\), 其中
    \begin{align*}
        [C]_{j_k,1}
        = \begin{dcases}
              \frac{\det {(T\{k,V\})}}{\det {(T)}}
              - \sum_{r < \ell \leq n}
              {c_{j_\ell}
              \frac{\det {(T\{k,u_{j_\ell}\})}}{\det {(T)}}},
               & k \leq r; \\
              c_{j_k},
               & k > r.
          \end{dcases}
    \end{align*}
    其中 \(T\{k,Y\}\) 是以 \(r \times 1\)~阵 \(Y\)
    代 \(T\) 的列~\(k\) 后得到的阵.
    则 \(UC = V\).

    (2)
    若 \(n \times 1\)~阵 \(D\) 适合 \(UD = V\),
    则存在 \(n-r\)~个数
    \(c_{j_{r+1}}\), \(\dots\), \(c_{j_n}\),
    使
    \begin{align*}
        [D]_{j_k,1}
        = \begin{dcases}
              \frac{\det {(T\{k,V\})}}{\det {(T)}}
              - \sum_{r < \ell \leq n}
              {c_{j_\ell}
              \frac{\det {(T\{k,u_{j_\ell}\})}}{\det {(T)}}},
               & k \leq r; \\
              c_{j_k},
               & k > r.
          \end{dcases}
    \end{align*}
    换句话说, \(UX = V\) 的每一个解%
    都可被 (1) 中的公式表达.
\end{theorem}

\begin{proof}
    (1)
    我们可改写方程组 \(UX = V\) 为
    \begin{align*}
        \sum_{\ell = 1}^{r}
        {[U]_{p,j_\ell} x_{j_\ell}}
            = [V]_{p,1}
        - \sum_{r < \ell \leq n}
        {[U]_{p,j_\ell} x_{j_\ell}},
    \end{align*}
    其中 \(p = 1\), \(2\), \(\dots\), \(r\),
    下同.
    考虑%
    由 \(r\)~个 \(r\)~元 \({\leq} 1\)~次方程作成的方程组
    \begin{align*}
        \sum_{\ell = 1}^{r}
        {[U]_{p,j_\ell} y_\ell}
        = [V]_{p,1}
        - \sum_{r < \ell \leq n}
        {[U]_{p,j_\ell} c_{j_\ell}}.
    \end{align*}
    利用阵等式, 我们可写
    \begin{align*}
        T
        \begin{bmatrix}
            y_1    \\
            y_2    \\
            \vdots \\
            y_r    \\
        \end{bmatrix}
        =
        V -
        \sum_{r < \ell \leq n}
        {c_{j_\ell} u_{j_\ell}}.
    \end{align*}
    因为 \(\det {(T)} \neq 0\),
    故, 由 Cramer 公式,
    存在 \(r\) 个数
    \(c_{j_1}\), \(c_{j_2}\), \(\dots\), \(c_{j_r}\),
    使
    \begin{align*}
        \sum_{\ell = 1}^{r}
        {[U]_{p,j_\ell} c_{j_\ell}}
            = [V]_{p,1}
        - \sum_{r < \ell \leq n}
        {[U]_{p,j_\ell} c_{j_\ell}},
    \end{align*}
    其中
    \begin{align*}
        c_{j_k}
        = {} &
        \frac{1}{\det {(T)}}
        \det {
            \left(
            T\left\{
            k,
            V -
            \sum_{r < \ell \leq n}
            {c_{j_\ell} u_{j_\ell}}
            \right\}
            \right)
        }
        \\
        = {} &
        \frac{1}{\det {(T)}} \det {(T\{k,V\})}
        -
        \frac{1}{\det {(T)}}
        \sum_{r < \ell \leq n}
        {c_{j_\ell}
        \det {(T\{k,u_{j_\ell}\})}
        }
        \\
        = {} &
        \frac{\det {(T\{k,V\})}}{\det {(T)}}
        - \sum_{r < \ell \leq n}
        {c_{j_\ell}
        \frac{\det {(T\{k,u_{j_\ell}\})}}{\det {(T)}}}.
    \end{align*}
    从而
    \begin{align*}
        \sum_{\ell = 1}^{n}
        {[U]_{p,j_\ell} c_{j_\ell}}
            = [V]_{p,1}.
    \end{align*}
    注意到 \([C]_{j_k,1} = c_{j_k}\).
    所以, \(UC = V\).

    (2)
    设 \(D\) 适合 \(UD = V\).
    则
    \begin{align*}
        \sum_{\ell = 1}^{n}
        {[U]_{p,j_\ell} [D]_{j_\ell,1}}
            = [V]_{p,1}.
    \end{align*}
    从而
    \begin{align*}
        \sum_{\ell = 1}^{r}
        {[U]_{p,j_\ell} [D]_{j_\ell,1}}
            = [V]_{p,1}
        - \sum_{r < \ell \leq n}
        {[U]_{p,j_\ell} [D]_{j_\ell,1}}.
    \end{align*}

    取 \(c_{j_\ell} = [D]_{j_\ell,1}\),
    \(\ell > r\).
    考虑方程组
    \begin{align*}
        \sum_{\ell = 1}^{r}
        {[U]_{p,j_\ell} y_\ell}
        = [V]_{p,1}
        - \sum_{r < \ell \leq n}
        {[U]_{p,j_\ell} c_{j_\ell}}.
    \end{align*}
    由 (1), 我们知道
    \begin{align*}
        y_k =
        \frac{\det {(T\{k,V\})}}{\det {(T)}}
        - \sum_{r < \ell \leq n}
        {c_{j_\ell}
        \frac{\det {(T\{k,u_{j_\ell}\})}}{\det {(T)}}}
    \end{align*}
    (其中 \(k = 1\), \(2\), \(\dots\), \(r\), 下同)
    是一个解;
    另一方面,
    \(y_k = [D]_{j_k,1}\) 也是一个解.
    因为 \(\det {(T)} \neq 0\),
    故, 由 Cramer 公式,
    这二组解应是相同的,
    即
    \begin{equation*}
        [D]_{j_k,1} =
        \frac{\det {(T\{k,V\})}}{\det {(T)}}
        - \sum_{r < \ell \leq n}
        {c_{j_\ell}
        \frac{\det {(T\{k,u_{j_\ell}\})}}{\det {(T)}}}.
        \qedhere
    \end{equation*}
\end{proof}

现在, 我们作一个小结.

注意到, 当 \(r = n\) 时,
不可能有数 \(\ell\) 适合 \(r < \ell \leq n\).
既然没有数被加, 那么, 形如
\begin{align*}
    \sum_{r < \ell \leq n} {f(\ell)}
\end{align*}
的式是零.
用这种约定,
我们可统一地写%
在 \(r = n\) 与 \(r < n\) 这二种情形下%
解的公式.

\begin{theorem}
    设 \(V\) 是一个 \(r \times 1\)~阵.
    设 \(U\) 是一个 \(r \times n\)~阵,
    其中 \(r \leq n\),
    且 \(U\) 有一个行列式非零的 \(r\)~级子阵
    \begin{align*}
        T = U\binom{1,2,\dots,r}{j_1,j_2,\dots,j_r},
    \end{align*}
    其中 \(1 \leq j_1 < \dots < j_r \leq n\).
    从 \(1\), \(2\), \(\dots\), \(n\)
    去除 \(j_1\), \(j_2\), \(\dots\), \(j_r\)
    后, 还剩 \(n - r\)~个数.
    我们从小到大地叫这 \(n - r\)~个数为
    \(j_{r+1}\), \(\dots\), \(j_n\).
    再设 \(U\)~的%
    列~\(1\), \(2\), \(\dots\), \(n\)
    分别是 \(u_1\), \(u_2\), \(\dots\), \(u_n\).
    那么, \(UX = V\) 的解\emph{恰}为
    \begin{align*}
        x_{j_k}
        = \begin{dcases}
              \frac{\det {(T\{k,V\})}}{\det {(T)}}
              - \sum_{r < \ell \leq n}
              {c_{j_\ell}
              \frac{\det {(T\{k,u_{j_\ell}\})}}{\det {(T)}}},
               & k \leq r; \\
              c_{j_k},
               & k > r,
          \end{dcases}
    \end{align*}
    其中 \(c_{j_{r+1}}\), \(\dots\), \(c_{j_n}\)
    是任何的 \(n-r\) 个数.
\end{theorem}

\section{\texorpdfstring{由 \(m\)~个 \(n\)~元
      \({\leq} 1\)~次方程作成的方程组 (5)}%
  {由 m 个 n 元 ≤1 次方程作成的方程组 (5)}}

\maldevigalegajxo

线性方程组的故事要结束了.
毕竟, 我们已解决线性方程组的三个较重要的问题:

(1)
一个线性方程组何时有解?

(2)
一个线性方程组有解时, 其解是否唯一?

(3)
一个线性方程组有解时, 我们如何找到它的全部的解?

我们用行列式回答了这三个问题,
作了一个较完整的线性方程组的理论.
所以, 理论地, 给了一个线性方程组,
我们可以计算一些行列式以确定它是否有解,
且有解时其解为何.

不过,
计算一个方阵的行列式并不是什么简单的事情:
\(3\)~级阵的行列式的较具体的公式含 \(6\)~项,
而 \(4\)~级阵的行列式的较具体的公式含 \(24\)~项.
所以, 像 Cramer 公式那样,
我们又作了一个 ``较理论的'' 理论.

但是, 此理论仍然是重要的.
或许, 您还记得,
任给一个阵~\(A\),
必存在唯一的非负整数 \(r\),
使 \(A\)~有一个行列式非零的 \(r\)~级子阵,
但 \(A\)~没有行列式非零的 \(r+1\)~级子阵.
聪明的数学家意识到,
此 \(r\) 十分地重要,
故专门为它起了一个名字, ``\emph{秩}''.
然后, 他们研究了秩,
发现:
(a)
可以施适当的变换于阵~\(A\),
得到一个新阵~\(E\),
且 \(A\) 与 \(E\) 有相同的秩;
(b)
``适当的变换'' 跟计算方阵的行列式比,
有着较少的计算量
(具体地, 加、减、乘、除的次数);
(c)
\(E\)~的秩不难计算,
甚至可被直接看出.
于是, 联合这个发现与我们的理论,
就是一个更好的线性方程组的理论.
若您对此事感兴趣,
您可以见线性代数 (或高等代数) 教材.
% 此事虽好,
% 但可能不适合在一门主讲行列式的课程里被具体地讨论.

我就说这么多吧.

\SenAsteriskoEnEnhavtabelo
