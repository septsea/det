\section{我要如何定义行列式?}

行列式是什么?
% 粗糙地,
我认为,
就像迹那样
(注: \(n\)~级阵 \(A\)~的\emph{迹}是
\([A]_{1,1} + [A]_{2,2} + \dots + [A]_{n,n}\)),
行列式也只是方阵一个属性而已.
不过, 这个看法, 多少有些不全面;
毕竟, 这可能会使人认为,
``行列式就是个 `新定义运算' 而已''.
行列式是有用的;
至少, 不说线性代数 (或高等代数),
它在微积分与几何里, 也是有用的.

出于多种原因, 我决定,
写一本行列式的入门教材.
既然是\emph{入门}教材,
它自然要简单一些.
我能想到至少三种 (差别较大的) 定义方式:

\begin{definition}[归纳定义]
    设 \(A\) 是 \(n\)~级阵 (\(n \geq 1\)).
    定义 \(A\) 的行列式
    \begin{align*}
        \det {(A)}
        =
        \begin{dcases}
            [A]_{1,1},
             & n = 1;    \\
            \sum_{i = 1}^{n}
            {(-1)^{i+1} [A]_{i,1} \det {(A(i|1))}},
             & n \geq 2.
        \end{dcases}
    \end{align*}
\end{definition}

\begin{definition}[组合定义]
    设 \(A\) 是 \(n\)~级阵 (\(n \geq 1\)).
    定义 \(A\) 的行列式
    \begin{align*}
        \det {(A)}
        = {} &
        \sum_{\substack{
        1 \leq i_1, i_2, \dots, i_n \leq n \\
                i_1, i_2, \dots, i_n\,\text{互不相同}
            }}
        {s(i_1, i_2, \dots, i_n)\,
            [A]_{i_1,1} [A]_{i_2,2} \dots [A]_{i_n,n}},
    \end{align*}
    其中
    \begin{align*}
        s(i_1, i_2, \dots, i_n)
        = \prod_{1 \leq p < q \leq n}
        {\operatorname{sgn} {(i_q - i_p)}}
    \end{align*}
    是文字列
    (或者, ``排列'',
    因为这里的 \(i_1\), \(i_2\), \(\dots\), \(i_n\)
    是互不相同的)
    \(i_1\), \(i_2\), \(\dots\), \(i_n\) 的符号.
\end{definition}

\begin{definition}[公理定义]
    设 \(f\) 是定义在全体 \(n\)~级阵上的函数.
    若 \(f\) 适合如下三条, 则说
    \(f\) 是 (\(n\)~级阵的) \emph{一个}行列式函数
    (自然地, 若 \(A\) 是 \(n\)~级阵,
    则 \(f(A)\) 是 \(A\) 的\emph{一个}行列式):

    (1)
    (规范性)
    若 \(I\) 是 \(n\)~级单位阵,
    则 \(f(I) = 1\).

    (2)
    (多线性)
    对任何不超过 \(n\) 的正整数 \(j\),
    任何 \(n-1\)~个 \(n \times 1\)~阵
    \(a_1\), \(\dots\), \(a_{j-1}\),
    \(a_{j+1}\), \(\dots\), \(a_n\),
    任何二个 \(n \times 1\)~阵 \(x\), \(y\),
    任何二个数 \(s\), \(t\),
    有
    \begin{align*}
             & f
            {([a_1, \dots, a_{j-1}, sx + ty,
                        a_{j+1}, \dots, a_n])}
        \\
        = {} &
        s
        f {([a_1, \dots, a_{j-1}, x, a_{j+1}, \dots, a_n])}
        +
        t
        f {([a_1, \dots, a_{j-1}, y, a_{j+1}, \dots, a_n])}.
    \end{align*}

    (3)
    (交错性)
    若 \(n\)~级阵 \(A\) 有二列完全相同,
    则 \(f {(A)} = 0\).
\end{definition}

值得注意的是, 这里的定义是关于列的.
我们知道, 一个阵的行列式等于其转置的行列式,
故阵的行与阵的列在行列式里的地位是一样的,
进而我们也可用关于行的版本定义行列式.
% (我就不单独列出了).
这, 我认为, 只是个人的喜好而已.
毕竟, 行或列不是本质的.
规范性、多线性、交错性是本质.

粗略地, 我想到的三种定义,
对初学行列式的人,
都是有一定挑战的,
因为它们都涉及了 ``非高中数学内容''.
归纳定义不好, 因为学生不一定熟悉数学归纳法.
当我是高中生时, 数学归纳法至少是必学的;
可是, 过了几年, 新教材里的数学归纳法是选学内容,
且新高考也不再考它.
组合定义不好, 因为学生不一定熟悉
(比数学归纳法抽象的) 排列或置换.
并且,
% 尴尬地,
在一些线性代数 (或高等代数) 教材里,
% 排列或置换的理论似乎只为行列式服务.
排列或置换的理论似乎只为行列式所用.
公理定义不好, 因为它有些抽象.
这要一定的准备.
据说, 老的中学数学有
``\(2\)~级行列式'' ``\(3\)~级行列式''
(即, \(2\)~级阵的行列式与 \(3\)~级阵的行列式);
可是, 我是高中生时, (必学的) 教材没有了行列式;
(必学的) 新教材自然也没有行列式.
我认为, 这么讲行列式, 会使更多的初学者不解
(若学生没有什么准备知识).

虽然这三个定义都对初学者有一定的挑战,
我还是选了归纳定义;
毕竟, 我想, 数学归纳法应是%
每一个 (学数学的) 人都了解的 (基础的) 原理.

于是, 我开始写本书的第一章.

\section{我要讲阵吗?}

理论地, 我可不用阵讲行列式.
具体地, 我可以这么定义行列式.

\begin{definition}
    我们叫下面用二条竖线围起来的%
    由 \(n\)~行 \(n\)~列元%
    作成的式为一个 \emph{\(n\)~级行列式}:
    \begin{equation}
        D =
        \begin{vmatrix}
            a_{1,1} & a_{1,2} & \cdots & a_{1,n} \\
            a_{2,1} & a_{2,2} & \cdots & a_{2,n} \\
            \vdots  & \vdots  & {}     & \vdots  \\
            a_{n,1} & a_{n,2} & \cdots & a_{n,n} \\
        \end{vmatrix}.
        \label{eq:C0201}
    \end{equation}
    它由 \(n\)~行 \(n\)~列, 共 \(n^2\)~个元作成.
    我们叫行~\(i\) 的 \(n\)~个元
    \(a_{i,1}\), \(a_{i,2}\), \(\dots\), \(a_{i,n}\)
    为行列式~\(D\) 的行~\(i\),
    叫列~\(j\) 的 \(n\)~个元
    \(a_{1,j}\), \(a_{2,j}\), \(\dots\), \(a_{n,j}\)
    为行列式~\(D\) 的列~\(j\).
    我们叫行~\(i\), 列~\(j\) 交点上的元 \(a_{i,j}\)
    为行列式~\(D\) 的 \((i,j)\)-元.

    我们定义 \(a_{i,j}\)~的\emph{余子式} \(M_{i,j}\)
    为由行列式~\(D\) 中%
    去除行~\(i\) 列~\(j\) 后%
    剩下的 \(n-1\)~行与 \(n-1\)~列元%
    作成的行列式:
    \begin{align*}
        M_{i,j} =
        \begin{vmatrix}
            a_{1,1}     & \cdots & a_{1,j-1}   &
            a_{1,j+1}   & \cdots & a_{1,n}       \\
            \vdots      & {}     & \vdots      &
            \vdots      & {}     & \vdots        \\
            a_{i-1,1}   & \cdots & a_{i-1,j-1} &
            a_{i-1,j+1} & \cdots & a_{i-1,n}     \\
            a_{i+1,1}   & \cdots & a_{i+1,j-1} &
            a_{i+1,j+1} & \cdots & a_{i+1,n}     \\
            \vdots      & {}     & \vdots      &
            \vdots      & {}     & \vdots        \\
            a_{n,1}     & \cdots & a_{n,j-1}   &
            a_{n,j+1}   & \cdots & a_{n,n}       \\
        \end{vmatrix}.
    \end{align*}

    当 \(n = 1\) 时, 定义式~\eqref{eq:C0201} 为
    \begin{equation}
        D = a_{1,1}.
        \label{eq:C0202}
    \end{equation}
    当 \(n \geq 2\) 时, 归纳地定义式~\eqref{eq:C0201} 为
    \begin{equation}
        \begin{aligned}
            D
            = {} &
            a_{1,1} M_{1,1} - a_{2,1} M_{2,1} + \dots +
            (-1)^{n+1} a_{n,1} M_{n,1}
            \\
            = {} &
            \sum_{i = 1}^{n} {(-1)^{i+1} a_{i,1} M_{i,1}}.
        \end{aligned}
        \label{eq:C0203}
    \end{equation}
\end{definition}

可以看到, 在这个定义里,
``行列式'' 至少有二种意思:
一是形如式~\eqref{eq:C0201} 的方形数表的式,
二是由式~\eqref{eq:C0202}, \eqref{eq:C0203}
定义的一个数.
既然式~\eqref{eq:C0201} 只是一个式,
又因为二个式相等是指它们的结果相等,
故, 有着不一样的元的二个 \(n\)~级行列式可能相等.

我如何定义行列式?
我先定义阵 (矩形数表),
再定义方阵 (方形数表) 的行列式%
是施一定的规则于方阵得到的数.
(具体地, 您看第一章, 节~\sekcio{5}, \sekcio{6} 即知.)

由此可见, 这个定义跟第一章的定义,
在思想上, 是有一些区别的.
我用的定义视行列式为方阵的一个属性,
而这个定义,
% 单纯地,
是一个
``形如式~\eqref{eq:C0201} 的方形数表的式'',
或
``由式~\eqref{eq:C0202}, \eqref{eq:C0203} 确定的数''.

历史地, 行列式比阵早出现.
所以, 这种不涉及阵的定义, 是较古典的.
逻辑地, 它没有什么问题.
不过, 教学地, 有的学生区分行列式与阵是有些挑战的.
毕竟,
% 二者相貌类似,
二者长得差不多,
且阵 (的运算) 与行列式对%
线性代数 (或高等代数) 的初学者来说%
都是有一定挑战的.

我想到的解决此问题的方法就是先讲阵
(至少, 一定要先讲阵的记号),
再讲行列式.
这样, 初学者更能体会,
行列式是方阵的一个属性.
至少, 我不想在入门课给学生较多挑战.

类似地, 历史地, 对数比指数早出现.
可是, 我们在高中学数学时,
也并没有先讲对数, 再讲指数.
相反, 教材用指数讲对数.

\vspace{2ex}

最后, 我以一个较形象的例结束本节.

我在前面说过,
\(n\)~级阵 \(A\)~的迹是
\([A]_{1,1} + [A]_{2,2} + \dots + [A]_{n,n}\).
能否不用阵定义迹?
我想, 理论地, 当然可以.
我试作了一个定义.
您看看它如何吧.

\begin{definition}
    我们叫下面用括号 \(\{ \ \}\)
    (注意, 这只是我自己选的一个记号)
    围起来的%
    由 \(n\)~行 \(n\)~列元%
    作成的式为一个 \emph{\(n\)~级迹}:
    \begin{equation}
        T =
        \begin{Bmatrix}
            a_{1,1} & a_{1,2} & \cdots & a_{1,n} \\
            a_{2,1} & a_{2,2} & \cdots & a_{2,n} \\
            \vdots  & \vdots  & {}     & \vdots  \\
            a_{n,1} & a_{n,2} & \cdots & a_{n,n} \\
        \end{Bmatrix}.
        \label{eq:C0204}
    \end{equation}
    它由 \(n\)~行 \(n\)~列, 共 \(n^2\)~个元作成.
    我们叫行~\(i\) 的 \(n\)~个元
    \(a_{i,1}\), \(a_{i,2}\), \(\dots\), \(a_{i,n}\)
    为迹~\(T\) 的行~\(i\),
    叫列~\(j\) 的 \(n\)~个元
    \(a_{1,j}\), \(a_{2,j}\), \(\dots\), \(a_{n,j}\)
    为迹~\(T\) 的列~\(j\).
    我们叫行~\(i\), 列~\(j\) 交点上的元 \(a_{i,j}\)
    为迹~\(T\) 的 \((i,j)\)-元.

    我们定义式~\eqref{eq:C0204} 为
    \begin{align*}
        T = a_{1,1} + a_{2,2} + \dots + a_{n,n}.
    \end{align*}
\end{definition}

% 这是一个开放的问题;
% 或者, 用《周易》的话,
% ``仁者见之谓之仁, 智者见之谓之智''.
% 这是一个
% ``仁者见之谓之仁, 智者见之谓之智''
% 的问题;
这是一个个人喜好问题.
不同的人, 会有不同的想法.
