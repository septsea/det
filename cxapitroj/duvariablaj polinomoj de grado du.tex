\chapter{\texorpdfstring{\(2\)~元 \(2\)~次式}{2 元 2 次式}}

% 本章, 我们用行列式研究
本章, 我们研究
\(2\)~元 \(2\)~次式.

% \vfill
% \begin{ilustrajxo*}[h!]%
%     \includegraphics[height=12.5cm]{2}%
%     \centering%
% \end{ilustrajxo*}
% \vfill
\clearpage

\section{准备}

为方便, 定义
\begin{align*}
    \begin{vmatrix}
        a_{1,1} & a_{1,2} & a_{1,3} \\
        a_{2,1} & a_{2,2} & a_{2,3} \\
        a_{3,1} & a_{3,2} & a_{3,3} \\
    \end{vmatrix}
    % = {} &
    % \hphantom{{} + {}}
    % a_{1,1} a_{2,2} a_{3,3}
    % + a_{2,1} a_{3,2} a_{1,3}
    % + a_{3,1} a_{1,2} a_{2,3}
    % \\
    %      &
    % - a_{1,1} a_{3,2} a_{2,3}
    % - a_{2,1} a_{1,2} a_{3,3}
    % - a_{3,1} a_{2,2} a_{1,3}
    % \\
    = {} &
    \hphantom{{} + {}}
    a_{1,1} a_{2,2} a_{3,3}
    + a_{1,2} a_{2,3} a_{3,1}
    + a_{1,3} a_{2,1} a_{3,2}
    \\
         &
    - a_{1,1} a_{2,3} a_{3,2}
    - a_{1,2} a_{2,1} a_{3,3}
    - a_{1,3} a_{2,2} a_{3,1}.
\end{align*}
可用此法助记此式.
此式含 \(6\)~项,
每一项都是不同行不同列的 \(3\)~个数的积.
% 由左上 \(a_{1,1}\) 画一条实对角线至右下 \(a_{3,3}\),
% 再由右上 \(a_{1,3}\) 画一条虚对角线至左下 \(a_{3,1}\).
% 实对角线及跟实对角线平行的线上的 \(3\)~个数的积
% \(a_{1,1} a_{2,2} a_{3,3}\),
% \(a_{1,2} a_{2,3} a_{3,1}\),
% \(a_{1,3} a_{2,1} a_{3,2}\)
% 前带符号 \(1\);
% 虚对角线及跟虚对角线平行的线上的 \(3\)~个数的积
% \(a_{1,1} a_{2,3} a_{3,2}\),
% \(a_{1,2} a_{2,1} a_{3,3}\),
% \(a_{1,3} a_{2,2} a_{3,1}\)
% 前带符号 \(-1\).
我们在原 \(3\)~行 \(3\)~列的数表的右侧复写它,
作成一个 \(3\)~行 \(6\)~列的数表.
由左上至右下的对角线 (实线) 上的数的积的和%
减由右上至左下的对角线 (虚线) 上的数的积的和%
即为得数.
% https://tex.stackexchange.com/questions/32978/typesetting-a-matrix-with-crossing-arrows-on-it/#32981
\begin{tikzbildo*}[h!]
    \centering
    \begin{tikzpicture}[>=stealth]
        \matrix [%
            matrix of math nodes,
            column sep=1em,
            row sep=1em
        ] (s3h) {%
            a_{1,1} & a_{1,2} & a_{1,3} &
            a_{1,1} & a_{1,2} & a_{1,3}   \\
            a_{2,1} & a_{2,2} & a_{2,3} &
            a_{2,1} & a_{2,2} & a_{2,3}   \\
            a_{3,1} & a_{3,2} & a_{3,3} &
            a_{3,1} & a_{3,2} & a_{3,3}   \\
        };

        \path
        ($(s3h-1-1.north west)-(0.5em,0)$) edge ($(s3h-3-1.south west)-(0.5em,0)$)
        ($(s3h-1-3.north east)+(0.5em,0)$) edge ($(s3h-3-3.south east)+(0.5em,0)$)
        (s3h-1-1)                          edge            (s3h-2-2)
        (s3h-2-2)                          edge[->]        (s3h-3-3)
        (s3h-1-2)                          edge            (s3h-2-3)
        (s3h-2-3)                          edge[->]        (s3h-3-4)
        (s3h-1-3)                          edge            (s3h-2-4)
        (s3h-2-4)                          edge[->]        (s3h-3-5)
        % (s3h-3-1)                          edge[dashed]    (s3h-2-2)
        % (s3h-2-2)                          edge[->,dashed] (s3h-1-3)
        % (s3h-3-2)                          edge[dashed]    (s3h-2-3)
        % (s3h-2-3)                          edge[->,dashed] (s3h-1-4)
        % (s3h-3-3)                          edge[dashed]    (s3h-2-4)
        % (s3h-2-4)                          edge[->,dashed] (s3h-1-5);
        (s3h-1-4)                          edge[dashed]    (s3h-2-3)
        (s3h-2-3)                          edge[->,dashed] (s3h-3-2)
        (s3h-1-5)                          edge[dashed]    (s3h-2-4)
        (s3h-2-4)                          edge[->,dashed] (s3h-3-3)
        (s3h-1-6)                          edge[dashed]    (s3h-2-5)
        (s3h-2-5)                          edge[->,dashed] (s3h-3-4);

        \foreach \c in {1,2,3}
            {\node[anchor=south] at (s3h-1-\c.north) {\(+\)};}
        \foreach \c in {4,5,6}
            {\node[anchor=south] at (s3h-1-\c.north) {\(-\)};}
        % \foreach \c in {1,2,3}
        %     {\node[anchor=north] at (s3h-3-\c.south) {\(-\)};}
    \end{tikzpicture}
\end{tikzbildo*}

% % https://www.zhihu.com/question/525613625
% \begin{tikzbildo*}[h!]
%     \centering
%     \begin{tikzpicture}
%         \matrix [%
%             matrix of math nodes,
%             left delimiter=|,
%             right delimiter=|,
%             column sep=1.5em,
%             row sep=1.5em,
%             inner sep=0.3ex
%         ] (b3) {%
%             a_{1,1} & a_{1,2} & a_{1,3} \\
%             a_{2,1} & a_{2,2} & a_{2,3} \\
%             a_{3,1} & a_{3,2} & a_{3,3} \\
%         };
%         %\begin{scope}[draw=gray!70, rounded corners,
%         \begin{scope}[rounded corners,
%                 x = {($ (b3-2-2) - (b3-1-1) $)},%
%                 y = {($ (b3-2-1) - (b3-1-2) $)}%
%             ]
%             \draw
%             (b3-1-1) -- (b3-2-2) -- (b3-3-3)
%             (b3-1-2) -- (b3-2-3) -- ++(1,0)
%             -- ++(0,1.5) -- (b3-3-1)
%             (b3-1-3) -- ++(2,0) -- ++(0,1.5)
%             -- (b3-3-2) -- (b3-2-1);
%             \draw[densely dashed]
%             (b3-1-3) -- (b3-2-2) -- (b3-3-1)
%             (b3-2-3) -- (b3-3-2) -- ++(0,1.5)
%             -- ++(-1.5,0) -- (b3-1-1)
%             (b3-3-3) -- ++(0,1.5) -- ++(-1.5,0)
%             -- (b3-2-1) -- (b3-1-2);
%         \end{scope}
%     \end{tikzpicture}
% \end{tikzbildo*}

本章的 ``数'' 都是复数.
这是因为, 在我们的讨论里, 我们要 ``开根'',
而不是所有的实数都能 (在实数里) 开根
(比如, 一定不存在实数 \(x\) 使 \(x^2 = -1\)).
但是, 我们总是可以在复数里开根.
具体地,

\begin{theorem}
    设 \(z\) 是复数.
    一定存在复数 \(w\) 使 \(w^2 = z\).
\end{theorem}

% 在解高次方程时,
有时,
我们还要 ``开立方根''.
我们在中学里知道, 每一个实数都能 (在实数里) 开立方根.
不过, 进一步地, 我们也有

\begin{theorem}
    设 \(z\) 是复数.
    一定存在复数 \(v\) 使 \(v^3 = z\).
\end{theorem}

在此, 我就不证明这二个定理了.
就像在中学时, 您接受
``对任何的非负实数 \(y\),
必有非负实数 \(x\) 使 \(x^2 = y\)''
与
``对任何的实数 \(y\),
必有实数 \(u\) 使 \(u^3 = y\)''
那样,
您也接受它们就好.
我们要讨论的内容并不依赖%
它们如何被论证.

\vspace{2ex}

下面, 我展现一些我们要用到的恒等式.

\begin{theorem}
    设 \(a\), \(b\), \(x\) 是复数,
    且 \(a \neq 0\).
    则
    \begin{align*}
        ax^2 + 2bx
        = \frac{(ax + b)^2 - b^2}{a}.
    \end{align*}
    特别, 取 \(a = 1\),
    知
    \begin{align*}
        x^2 + 2bx = (x + b)^2 - b^2.
    \end{align*}
\end{theorem}

\begin{proof}
    \begin{align*}
        \frac{(ax + b)^2 - b^2}{a}
        = {} & \frac{(ax + b + b)(ax + b - b)}{a} \\
        = {} & \frac{(ax + 2b)ax}{a}              \\
        = {} & (ax + 2b)x                         \\
        = {} & ax^2 + 2bx.
        \qedhere
    \end{align*}
\end{proof}

\begin{theorem}
    设 \(a\), \(b\) 是复数.
    则
    \begin{align*}
        a^3 \pm b^3 = (a \pm b)(a^2 \mp ab + b^2).
    \end{align*}
\end{theorem}

\begin{proof}
    \begin{align*}
        a^3 - b^3
        = {} & a^3 - ab^2 + ab^2 - b^3 \\
        = {} & a(a^2 - b^2) + (a-b)b^2 \\
        = {} & (a-b)a(a+b) + (a-b)b^2  \\
        = {} & (a-b)(a^2 + ab + b^2).
    \end{align*}
    最后, 换 \(b\) 为 \(-b\),
    即得另一个公式.
\end{proof}

\begin{theorem}
    设 \(\omega = (-1 + \mathrm{i} \sqrt{3})/2\).
    则 \(\omega^2 + \omega = -1\),
    且 \(\omega^3 = 1\).
\end{theorem}

\begin{proof}
    注意到 \(2\omega + 1 = \mathrm{i} \sqrt{3}\).
    平方, 可知
    \begin{align*}
        4\omega^2 + 4\omega + 1 = -3.
    \end{align*}
    整理, 即知.

    最后, 注意到 \(1 = 1^2 = 1^3\),
    \(\omega = \omega \cdot 1\),
    故
    \begin{equation*}
        \omega^3
        = 1 + (\omega^3 - 1^3)
        = 1 + (\omega - 1)(\omega^2 + \omega \cdot 1 + 1^2)
        = 1.
        \qedhere
    \end{equation*}
\end{proof}

\begin{theorem}
    设 \(a\), \(b\) 是复数.
    则
    \begin{align*}
        a^3 \pm 3a^2 b + 3ab^2 \pm b^3 = (a \pm b)^3.
    \end{align*}
\end{theorem}

\begin{proof}
    \begin{align*}
             & a^3 + 3a^2 b + 3ab^2 + b^3             \\
        = {} & a^3 + b^3 + 3a^2 b + 3ab^2             \\
        = {} & (a + b)(a^2 - ab + b^2) + (a + b)(3ab) \\
        = {} & (a + b)(a^2 + 2ab + b^2)               \\
        = {} & (a + b)^3.
    \end{align*}
    最后, 换 \(b\) 为 \(-b\),
    即得另一个公式.
\end{proof}

\begin{theorem}
    设 \(a\), \(b\), \(c\) 是复数.
    设 \(\omega = (-1 + \mathrm{i} \sqrt{3})/2\).
    则
    \begin{align*}
        a^3 + b^3 + c^3 - 3abc
        = {} & (a + b + c)(a^2 + b^2 + c^2 - ab - bc - ca) \\
        = {} & (a + b + c)(a + \omega b + \omega^2 c)
        (a + \omega^2 b + \omega c).
    \end{align*}
\end{theorem}

\begin{proof}
    \begin{align*}
             & a^3 + b^3 + c^3 - 3abc                       \\
        = {} & a^3 + ((b + c)^3 - 3b^2 c - 3bc^2) - 3abc    \\
        = {} & a^3 + (b + c)^3 - (a + b + c)(3bc)           \\
        = {} & (a + (b + c))(a^2 - a(b + c) + (b + c)^2)
        - (a + b + c)(3bc)                                  \\
        = {} & (a + b + c)(a^2 + b^2 + c^2 - ab - bc - ca).
    \end{align*}
    注意到
    \begin{align*}
             & 4(a^2 + b^2 + c^2 - ab - bc - ca)          \\
        = {} & 4(a^2  - a(b + c) + (b^2 - bc + c^2))      \\
        = {} & (2a)^2 - 2 \cdot 2a \cdot (b + c)
        + 4(b^2 - bc + c^2)                               \\
        = {} & (2a - b - c)^2 - (b + c)^2
        + (4b^2 - 4bc + 4c^2)                             \\
        = {} & (2a - b - c)^2 + 3(b^2 - 2bc + c^2)        \\
        = {} & (2a - b - c)^2 - (2\omega + 1)^2 (b - c)^2 \\
        = {} & (2a - b - c + (2\omega + 1)(b - c))
        (2a - b - c - (2\omega + 1)(b - c))               \\
        = {} & (2a + 2\omega b - 2(\omega + 1) c)
        (2a - 2(\omega + 1) b + 2\omega c)                \\
        = {} & 4(a + \omega b + \omega^2 c)
        (a + \omega^2 b + \omega c),
    \end{align*}
    故
    \begin{align*}
             & a^3 + b^3 + c^3 - 3abc                      \\
        = {} & (a + b + c)(a^2 + b^2 + c^2 - ab - bc - ca) \\
        = {} & (a + b + c)(a + \omega b + \omega^2 c)
        (a + \omega^2 b + \omega c).
        \qedhere
    \end{align*}
\end{proof}

\begin{theorem}
    设 \(x\), \(b\), \(c\) 是复数.
    设 \(\omega = (-1 + \mathrm{i} \sqrt{3})/2\).
    则
    \begin{align*}
        x^3 - 3bcx - (b^3 + c^3)
        = {} & (x - (b + c))(x - (\omega b + \omega^2 c))
        (x - (\omega^2 b + \omega c)).
    \end{align*}
\end{theorem}

\begin{proof}
    代上个定理的 \(a\), \(b\), \(c\)
    以 \(x\), \(-b\), \(-c\).
\end{proof}

上个定理允许我们解缺 \(2\)~次项的 \(1\)~元 \(3\)~次方程.

具体地, 设 \(Ax^3 + Cx + D = 0\) 是%
缺 \(2\)~次项的 \(1\)~元 \(3\)~次方程
(\(A \neq 0\)).
以 \(A\) 除方程的二侧, 有
\begin{align*}
    x^3 + C'x + D' = 0,
\end{align*}
其中 \(C' = C/A\), \(D' = D/A\).
若 \(C' = 0\), 这就是开立方根问题.
下设 \(C' \neq 0\).

考虑方程组
\begin{align*}
    \begin{cases}
        -3uv = C', \\
        -(u^3 + v^3) = D'.
    \end{cases}
\end{align*}
若我们能找到一组解 \(u = u_0\), \(v = v_0\),
取 \(b\), \(c\) 为 \(u_0\), \(v_0\).
则
\begin{align*}
    x^3 + C'x + D'
    = {} & x^3 - 3bcx - (b^3 + c^3)                   \\
    = {} & (x - (b + c))(x - (\omega b + \omega^2 c))
    (x - (\omega^2 b + \omega c)).
\end{align*}
这样, 我们就找到了 \(x^3 + C'x + D' = 0\) 的解.

现在, 我们只要解方程组
\begin{align*}
    \begin{cases}
        -3uv = C', \\
        -(u^3 + v^3) = D'.
    \end{cases}
\end{align*}
我们先考虑, 若 \(u = u_0\), \(v = v_0\)
是一组解, 它们应适合什么性质.
既然 \(C' \neq 0\),
且 \(-3u_0 v_0 = C'\),
故 \(v_0 = -\frac{C'}{3u_0}\).
从而
\begin{align*}
    -u_0^3 + \frac{(C')^3}{27u_0^3} = D'.
\end{align*}
也就是说, \(u_0^3\) 是 \(1\)~元 \(2\)~次方程
\begin{align*}
    z^2 + D'z - \frac{(C')^3}{27} = 0
\end{align*}
的解.

我们熟知, \(1\)~元 \(2\)~次方程总是有一个解.
故, 我们设复数 \(z_0\) 适合
\begin{align*}
    z_0^2 + D'z_0 - \frac{(C')^3}{27} = 0.
\end{align*}
% 开立方根, 可知, 存在一个复数 \(u_0\)
% 使 \(u_0^3 = z_0\).
% 再记 \(v_0 = -\frac{C'}{3u_0}\).
% 不难验证, 这么选取的 \(u_0\), \(v_0\)
% 适合条件.
开立方根, 可知, 存在一个复数 \(b\)
使 \(b^3 = z_0\).
再记 \(c = -\frac{C'}{3b}\).
不难验证, 这么取的 \(b\), \(c\)
适合
\(-3bc = C'\),
且 \(-(b^3 + c^3) = D'\).

\begin{theorem}
    作适当的换元,
    每一个 \(1\)~元 \(3\)~次方程都可被化为%
    缺 \(2\)~次项的方程.
\end{theorem}

\begin{proof}
    取 \(1\)~元 \(3\)~次方程 \(Ax^3 + Bx^2 + Cx + D = 0\).
    代 \(x\) 以 \(y - \frac{B}{3A}\), 有
    \begin{equation*}
        Ay^3 + \left(C - \frac{B^2}{3A} \right)y
        + D + \frac{2B^3}{27A^2} - \frac{BC}{3A} = 0.
        \qedhere
    \end{equation*}
\end{proof}

所以, 理论地, 我们可解任何一个 \(1\)~元 \(3\)~次方程.
不过, 这种解方程的方法有些复杂,
故我们一般不用它.

\section{\texorpdfstring{\(2\)~元 \(2\)~次式}{2 元 2 次式}}

设 \(A\), \(B\), \(C\), \(D\), \(E\), \(F\) 为复数.
形如
\begin{align*}
    f(x, y) = Ax^2 + 2Bxy + Cy^2 + 2Dx + 2Ey + F
\end{align*}
的式是一个 \(2\)~元 \({\leq} 2\)~次式.
若 \(A\), \(B\), \(C\) 至少有一个不是零,
就说它是一个 \(2\)~元 \(2\)~次式.
若 \(A\), \(B\), \(C\) 全是零,
就说它是一个 \(2\)~元 \({\leq} 1\)~次式.
若 \(A\), \(B\), \(C\) 全是零,
但 \(D\), \(E\) 至少有一个不是零,
就说它是一个 \(2\)~元 \(1\)~次式;
若 \(A\), \(B\), \(C\), \(D\), \(E\) 都是零,
那么它就是一个常数.

\begin{theorem}
    设 \(A\), \(B\), \(C\), \(D\), \(E\), \(F\) 为复数.
    设 \(f(x, y) = Ax^2 + 2Bxy + Cy^2 + 2Dx + 2Ey + F\).
    那么, 当 \(A\), \(B\), \(C\), \(D\), \(E\), \(F\)
    全为零时,
    对任何的复数对 \((z, w)\),
    \(f(z, w) = 0\).
    反过来, 若对任何的复数对 \((z, w)\),
    \(f(z, w) = 0\),
    则 \(A\), \(B\), \(C\), \(D\), \(E\), \(F\)
    全为零.
\end{theorem}

\begin{proof}
    前半部分是显然的, 因为零跟任何数的积都是零.
    我们看后半部分.

    假定, 对任何的复数对 \((z, w)\), \(f(z, w) = 0\).
    因为 \(f(0, 0) = 0\), 故 \(F = 0\).
    因为 \(f(1, 0) = A + 2D + F = A + 2D = 0\),
    且 \(f(-1, 0) = A - 2D + F = A - 2D = 0\),
    故 \(A = D = 0\).
    因为 \(f(0, 1) = C + 2E + F = C + 2E = 0\),
    且 \(f(0, -1) = C - 2E + F = C - 2E = 0\),
    故 \(C = E = 0\).
    最后, 因为
    \(f(1, 1) = A + 2B + C + 2D + 2E + F = 2B = 0\),
    故 \(B = 0\).
\end{proof}

\begin{theorem}
    设 \(A_1\), \(A_2\),
    \(B_1\), \(B_2\),
    \(C_1\), \(C_2\),
    \(D_1\), \(D_2\),
    \(E_1\), \(E_2\),
    \(F_1\), \(F_2\)
    都是复数.
    设
    \(f_i (x, y) = A_i x^2 + 2B_i xy + C_i y^2
    + 2D_i x + 2E_i y + F_i\)
    (\(i = 1\), \(2\)).
    那么,
    ``对任何复数对 \((z, w)\), 必 \(f_1 (z, w) = f_2 (z, w)\)''
    相当于
    ``\(A_1 = A_2\), \(B_1 = B_2\), \(C_1 = C_2\),
    \(D_1 = D_2\), \(E_1 = E_2\), \(F_1 = F_2\)''.
\end{theorem}

\begin{proof}
    设
    \begin{align*}
        g(x, y)
        = {} & f_1 (x, y) - f_2 (x, y)
        \\
        = {} & \hphantom{{} + {}}
        (A_1 - A_2) x^2 + 2(B_1 - B_2) xy + (C_1 - C_2) y^2
        \\
             &
        + 2(D_1 - D_2) x + 2(E_1 - E_2) y + (F_1 - F_2).
    \end{align*}
    施上个定理于 \(g(x, y)\) 即可.
\end{proof}

\begin{definition}
    设 \(f(x, y) = Ax^2 + 2Bxy + Cy^2 + 2Dx + 2Ey + F\).
    定义 \(f(x, y)\) 的判别式
    \begin{align*}
        \Delta
        = {} &
        \begin{vmatrix}
            A & B & D \\
            B & C & E \\
            D & E & F \\
        \end{vmatrix}                            \\
        % = {} & ACF - A E^2 - B^2 F + BED + DBE - DCD \\
        = {} & ACF + BED + DBE - AEE - BBF - DCD  \\
        = {} & ACF + 2BED - (AE^2 + CD^2 + FB^2).
    \end{align*}
\end{definition}

\(2\)~元 \({\leq} 2\)~次式 \(f(x, y)\)~的判别式是否为零%
与%
是否可写 \(f(x, y)\) 为%
二个 \(2\)~元 \({\leq} 1\)~次式的积%
有关.

\begin{theorem}
    设 \(f(x, y) = (a_1 x + a_2 y + a_3) (a_4 x + a_5 y + a_6)\).
    则 \(f(x, y)\) 的判别式为零.
\end{theorem}

\begin{proof}
    展开 \(f(x, y)\), 有
    \begin{align*}
        f(x,y) = A x^2 + 2B xy + C y^2 + 2D x + 2E y + F,
    \end{align*}
    其中
    \begin{align*}
         & A = a_1 a_4, \quad C = a_2 a_5, \quad F = a_3 a_6, \\
         & 2B = a_1 a_5 + a_2 a_4,                            \\
         & 2D = a_1 a_6 + a_3 a_4,                            \\
         & 2E = a_2 a_6 + a_3 a_5.
    \end{align*}
    从而
    \begin{align*}
        \Delta
        = {} & ACF + 2BED - (AE^2 + CD^2 + FB^2)                                                                                        \\
        = {} & \frac{1}{4} (4ACF + 2B \cdot 2E \cdot 2D) - \frac{1}{4} (A (2E)^2 + C (2D)^2 + F(2B)^2)                                  \\
        = {} & \frac{1}{4} (4 a_1 a_2 a_3 a_4 a_5 a_6 + a_2 a_3^2 a_5 a_4^2+a_2^2 a_3 a_6 a_4^2+a_1 a_3^2 a_5^2 a_4+a_1 a_2^2 a_6^2 a_4 \\
             & \quad \quad +2 a_1 a_2 a_3 a_5 a_6 a_4+a_1^2 a_2 a_5 a_6^2+a_1^2 a_3 a_5^2 a_6)                                          \\
             & \quad - \frac{1}{4} (a_1 a_3^2 a_4 a_5^2+2 a_1 a_2 a_3 a_4 a_6 a_5+a_1 a_2^2 a_4 a_6^2                                   \\
             & \quad \quad +a_2 a_3^2 a_5 a_4^2+2 a_1 a_2 a_3 a_5 a_6 a_4+a_1^2 a_2 a_5 a_6^2                                           \\
             & \quad \quad +a_2^2 a_3 a_6 a_4^2+2 a_1 a_2 a_3 a_5 a_6 a_4+a_1^2 a_3 a_5^2 a_6)                                          \\
        = {} & 0.
        \qedhere
    \end{align*}
\end{proof}

\begin{theorem}
    设 \(f(x, y) = Ax^2 + 2Bxy + Cy^2 + 2Dx + 2Ey + F\)
    的判别式为零.
    则存在复数
    \(a_1\), \(a_2\), \(a_3\), \(a_4\), \(a_5\), \(a_6\)
    使
    \begin{align*}
        f(x, y) = (a_1 x + a_2 y + a_3) (a_4 x + a_5 y + a_6).
    \end{align*}
\end{theorem}

\begin{proof}
    设
    \begin{align*}
        \Delta = ACF + 2BED - (AE^2 + CD^2 + FB^2) = 0.
    \end{align*}
    我们分类讨论.

    (1)
    \(A = B = C = 0\).
    则
    \begin{align*}
        f(x, y)
        = 2Dx + 2Ey + F
        = (0x + 0y + 1)(2Dx + 2Ey + F).
    \end{align*}

    (2)
    \(A = C = 0\), 但 \(B \neq 0\).
    则
    \begin{align*}
        f(x, y)
        = {} & 2Bxy + 2Dx + 2Ey + F                                                               \\
        = {} & \frac{2}{B}(Bx \cdot By + D \cdot Bx + E \cdot By) + F                             \\
        = {} & \frac{2}{B}(Bx \cdot By + D \cdot Bx + E \cdot By + DE - DE) + F                   \\
        = {} & \frac{2}{B}(Bx \cdot By + D \cdot Bx + E \cdot By + DE) - \frac{2}{B} \cdot DE + F \\
        = {} & \frac{2}{B}(Bx + E)(By + D) - \frac{2ED - FB}{B}                                   \\
        = {} & \left(1x + 0y + \frac{E}{B}\right)(0x + 2By + 2D) - \frac{2BED - FB^2}{B^2}.
    \end{align*}
    因为 \(\Delta = 0\), 且 \(A = C = 0\), 故
    \begin{align*}
        \Delta = 0 + 2BED - (0 + 0 + FB^2) = 2BED - FB^2 = 0.
    \end{align*}
    从而
    \begin{align*}
        f(x, y) = \left(1x + 0y + \frac{E}{B}\right)(0x + 2By + 2D).
    \end{align*}

    (3)
    \(A \neq 0\).
    则
    \begin{align*}
             & f(x, y)                                                                                      \\
        = {} & Ax^2 + 2(By + D)x + (Cy^2 + 2Ey + F)                                                         \\
        = {} & \frac{(Ax + By + D)^2 - (By + D)^2}{A}
        + (Cy^2 + 2Ey + F)                                                                                  \\
        = {} & \frac{1}{A} (Ax + By + D)^2 + \frac{1}{A}((AC - B^2)y^2 + 2(AE - BD)y) + \frac{AF - D^2}{A}.
    \end{align*}
    我们要进一步地分类讨论.

    (3.1)
    \(AC - B^2 = 0\).
    此时, \(C = \frac{B^2}{A}\).
    故
    \begin{align*}
        \Delta
        = {} & A \cdot \frac{B^2}{A} \cdot F + 2BED - AE^2
        - \frac{B^2 D^2}{A} - FB^2                         \\
        = {} & {-\frac{1}{A}}
        (A^2 E^2 - 2AE BD + B^2 D^2)                       \\
        = {} & {-\frac{1}{A}} (AE - BD)^2.
    \end{align*}
    因为 \(\Delta = 0\),
    故 \(AE - BD = 0\).
    从而
    \begin{align*}
        f(x, y) = \frac{(Ax + By + D)^2 - (D^2 - AF)}{A}.
    \end{align*}
    设复数 \(d\) 适合 \(d^2 = D^2 - AF\).
    则
    \begin{align*}
        f(x, y) = \frac{1}{A} (Ax + By + D + d)(Ax + By + D - d).
    \end{align*}
    也就是
    \begin{align*}
        f(x, y) = \left( 1x + \frac{B}{A} y + \frac{D+d}{A} \right) (Ax + By + (D - d)).
    \end{align*}

    (3.2)
    \(AC - B^2 \neq 0\).
    此时
    \begin{align*}
             & (AC - B^2) y^2 + 2(AE - BD)y
        \\
        = {} & \frac{1}{AC - B^2} ((AC - B^2)y + (AE - BD))^2 - \frac{(AE - BD)^2}{AC - B^2}.
    \end{align*}
    所以
    \begin{align*}
        f(x, y)
        = {} &
        \hphantom{{} + {}}
        \frac{(Ax + By + D)^2}{A} - \frac{((B^2 - AC)y + (BD - AE))^2}{A(B^2 - AC)}
        \\
             &
        + \frac{AF - D^2}{A} - \frac{(AE - BD)^2}{A(AC - B^2)}.
    \end{align*}
    也就是
    \begin{align*}
        f(x, y) = \frac{(Ax + By + D)^2}{A} - \frac{((B^2 - AC)y + (BD - AE))^2}{A(B^2 - AC)} + \frac{\Delta}{AC - B^2}.
    \end{align*}
    因为 \(\Delta = 0\), 故
    \begin{align*}
        f(x, y) = \frac{(Ax + By + D)^2}{A} - \frac{((B^2 - AC)y + (BD - AE))^2}{A(B^2 - AC)}.
    \end{align*}
    设复数 \(e\) 适合 \(e^2 = B^2 - AC\).
    设 \(f = \frac{BD-AE}{e}\).
    则
    \begin{align*}
        f(x, y)
        = {} & \frac{(Ax + By + D)^2}{A} - \frac{(e^2 y + ef)^2}{Ae^2}                   \\
        = {} & \frac{(Ax + By + D)^2}{A} - \frac{(ey + f)^2}{A}                          \\
        = {} & \frac{1}{A}(Ax + By + D + ey + f)(Ax + By + D - ey - f)                   \\
        = {} & \left( 1x + \frac{B+e}{A}y + \frac{D+f}{A} \right) (Ax + (B-e)y + (D-f)).
    \end{align*}

    (4)
    \(C \neq 0\).
    则
    \begin{align*}
        f(x, y) = Cy^2 + 2Byx + Ax^2 + 2Ey + 2Dx + F.
    \end{align*}
    考虑 \(2\)~元 \({\leq} 2\)~次式
    \begin{align*}
        g(x, y) = Cx^2 + 2Bxy + Ay^2 + 2Ex + 2Dy + F.
    \end{align*}
    换句话说, 对任何复数对 \((z, w)\),
    \begin{align*}
        g(z, w) = f(w, z).
    \end{align*}
    我们计算 \(g(x, y)\) 的判别式:
    \begin{align*}
        \Delta'
        = {} &
        \begin{vmatrix}
            C & B & E \\
            B & A & D \\
            E & D & F \\
        \end{vmatrix}                           \\
        = {} & CAF + 2BDE - (CD^2 + AE^2 + FB^2) \\
        = {} & ACF + 2BED - (AE^2 + CD^2 + FB^2) \\
        = {} & \Delta.
    \end{align*}
    从而, 由 (3),
    存在复数
    \(a_1\), \(a_2\), \(a_3\), \(a_4\), \(a_5\), \(a_6\)
    使
    \begin{align*}
        g(x, y) = (a_1 x + a_2 y + a_3) (a_4 x + a_5 y + a_6).
    \end{align*}
    所以
    \begin{equation*}
        f(x, y) = g(y, x) = (a_2 x + a_1 y + a_3) (a_5 x + a_4 y + a_6).
        \qedhere
    \end{equation*}
\end{proof}

综上, 我们得到本章的重要的定理:

\begin{theorem}
    设 \(f(x, y) = Ax^2 + 2Bxy + Cy^2 + 2Dx + 2Ey + F\).
    若 \(f(x, y)\) 的判别式为零,
    则我们必可写其为二个 (复系数的)
    \(2\)~元 \({\leq} 1\)~次式的积;
    反过来,
    若 \(f(x, y)\) 是二个 (复系数的)
    \(2\)~元 \({\leq} 1\)~次式的积,
    则其判别式必为零.
\end{theorem}

我们看二个例.

\begin{example}
    设
    \begin{align*}
        b(x, y) = 24x^2 + 20xy - 16y^2 - 14x - 26y - 2.
    \end{align*}
    \(A\), \(B\), \(C\), \(D\), \(E\), \(F\)
    分别是
    \(24\), \(10\), \(-16\), \(-7\), \(-13\), \(-2\).
    由此可知, \(b(x, y)\) 之判别式
    \begin{align*}
        \Delta
        = \begin{vmatrix}
              24 & 10  & -7  \\
              10 & -16 & -13 \\
              -7 & -13 & -2  \\
          \end{vmatrix}
        = -484.
    \end{align*}
    所以, 我们无法写其为二个
    (关于 \(x\), \(y\) 的)
    \(2\)~元 \({\leq} 1\)~次式的积.
\end{example}

\begin{example}
    设 \(f(x, y) = x^2 + y^2\).
    不难算出, \(f(x, y)\)~的判别式是 \(0\).
    所以, \(f(x, y)\) 可被写为%
    二个 (关于 \(x\), \(y\) 的)
    \(2\)~元 \({\leq} 1\)~次式的积.

    但是, 这二个 \({\leq} 1\)~次式的系数不能全是实数.

    反设存在\emph{实数}
    \(a_1\), \(a_2\), \(a_3\), \(a_4\), \(a_5\), \(a_6\) 使
    \begin{align*}
        x^2 + y^2 = (a_1 x + a_2 y + a_3) (a_4 x + a_5 y + a_6).
    \end{align*}
    代 \(x\), \(y\) 以 \(0\), \(0\), 知 \(0 = a_3 a_6\).
    因为乘法是可交换的, 故无妨设 \(a_3 = 0\).

    代 \(x\), \(y\) 以 \(1\), \(0\), 知
    \(1 = a_1 (a_4 + a_6)\);
    代 \(x\), \(y\) 以 \(-1\), \(0\), 知
    \(1 = -a_1 (-a_4 + a_6) = a_1 (a_4 - a_6)\).
    由此可知 \(a_4 + a_6 = a_4 - a_6\),
    故 \(a_6 = 0\).
    从而 \(a_1 a_4 = 1\).
    故 \(a_4 = a_1^{-1}\).

    代 \(x\), \(y\) 以 \(0\), \(1\), 知 \(a_2 a_5 = 1\).
    故 \(a_5 = a_2^{-1}\).
    从而
    \begin{align*}
        x^2 + y^2 = (a_1 x + a_2 y) (a_1^{-1} x + a_2^{-1} y).
    \end{align*}
    代 \(x\), \(y\) 以 \(1\), \(1\), 有
    \begin{align*}
        2 = (a_1 + a_2) \cdot \frac{a_1 + a_2}{a_1 a_2},
    \end{align*}
    即
    \begin{align*}
        a_1^2 + a_2^2 = 0.
    \end{align*}
    所以
    \begin{align*}
        0
        = (a_1^2 + a_2^2) \cdot a_5^2
        = (a_1 a_5)^2 + 1.
    \end{align*}
    因为 \(a_1\), \(a_5\) 是实数, 故 \(a_1 a_5\) 也是实数.
    但是, 我们知道,
    任何实数的平方加 \(1\) 不可能是 \(0\).
    这是矛盾.

    其实,
    \begin{align*}
        x^2 + y^2 = (x + \mathrm{i} y) (x - \mathrm{i} y).
    \end{align*}
\end{example}

\section{\texorpdfstring{\(2\)~元 \(2\)~次方程组}%
  {2 元 2 次方程组}}

本节, 我们讨论如何求解方程组
\begin{equation}
    \begin{cases}
        f_1 (x, y) = 0, \\
        f_2 (x, y) = 0,
    \end{cases}
    \label{eq:A0301}
\end{equation}
其中 \(f_i (x, y) = A_i x^2 + 2B_i xy + C_i y^2
+ 2D_i x + 2E_i y + F_i\)
(\(i = 1\), \(2\)).

我们看一个简单的例.

\begin{example}
    解方程组
    \begin{align*}
        \begin{cases}
            f_1 (x, y) = x^2 - y^2 - 4x + 4 = 0, \\
            f_2 (x, y) = x^2 + y^2 - 2x = 0.
        \end{cases}
    \end{align*}

    注意到
    \begin{align*}
        f_1 (x, y)
        = {} & x^2 - y^2 - 4x + 4      \\
        = {} & (x^2 - 4x + 4) - y^2    \\
        = {} & (x - 2)^2 - y^2         \\
        = {} & (x - 2 + y)(x - 2 - y).
    \end{align*}
    故
    \begin{align*}
        x = 2 - y
        \quad \text{或} \quad
        x = 2 + y.
    \end{align*}
    联立 \(x = 2 - y\) 与 \(f_2 (x, y) = 0\),
    可解出
    \begin{align*}
        (x, y) = (1, 1)
        \quad \text{或} \quad
        (x, y) = (2, 0).
    \end{align*}
    联立 \(x = 2 + y\) 与 \(f_2 (x, y) = 0\),
    可解出
    \begin{align*}
        (x, y) = (1, -1)
        \quad \text{或} \quad
        (x, y) = (2, 0).
    \end{align*}
    经验证, \((1, 1)\), \((1, -1)\), \((2, 0)\) 都是解.

    不难算出 \(f_1 (x, y)\) 的判别式
    \begin{align*}
        \Delta_1
        =
        \begin{vmatrix}
            1  & 0  & -2 \\
            0  & -1 & 0  \\
            -2 & 0  & 4  \\
        \end{vmatrix}
        = 0.
    \end{align*}
    所以, 理论地, \(f_1 (x, y)\) 确实可被%
    写为二个 \(2\)~元 \({\leq} 1\)~次式的积.
    不过, \(f_2 (x, y)\) 的判别式
    \begin{align*}
        \Delta_2
        =
        \begin{vmatrix}
            1  & 0 & -1 \\
            0  & 1 & 0  \\
            -1 & 0 & 0  \\
        \end{vmatrix}
        = -1.
    \end{align*}
    所以, \(f_2 (x, y)\) 不可被%
    写为二个 \(2\)~元 \({\leq} 1\)~次式的积.
\end{example}

由此可见, 若我们能写方程组~\eqref{eq:A0301}
的一个方程的左侧为二个 \(2\)~元 \({\leq} 1\)~次式的积,
则我们可解二次 (dufoje) \(1\)~元 \({\leq} 2\)~次方程,
以算出方程组的解.

不过, 不是所有的方程组都这么简单.

\begin{example}\label{emp:A0301}
    解方程组
    \begin{align*}
        \begin{cases}
            f_1 (x, y) = x^2 + 3y^2 + 2x - 11 = 0, \\
            f_2 (x, y) = 3x^2 - y^2 - 4x - 3 = 0.
        \end{cases}
    \end{align*}

    我们试用老方法解此方程组.
    不过, 试了试, 发现
    \(f_1 (x, y)\) 与 \(f_2 (x, y)\)
    好像都 ``不可被分解''.
    此时, 我们来计算判别式.
    \(f_1 (x, y)\), \(f_2 (x, y)\)~的判别式分别是
    \begin{align*}
         & \Delta_1
        =
        \begin{vmatrix}
            1 & 0 & 1   \\
            0 & 3 & 0   \\
            1 & 0 & -11 \\
        \end{vmatrix}
        = -36,
        \\
         & \Delta_2
        =
        \begin{vmatrix}
            3  & 0  & -2 \\
            0  & -1 & 0  \\
            -2 & 0  & -3 \\
        \end{vmatrix}
        = 13.
    \end{align*}
    看来, 我们没法像上个例那样简单地解此方程组.
\end{example}

不过, 我们并非无计可施.

\begin{theorem}
    设 \(k\) 是复数.
    则方程组~\eqref{eq:A0301}
    与方程组
    \begin{equation}
        \begin{cases}
            f_1 (x, y) + kf_2 (x, y) = 0, \\
            f_2 (x, y) = 0
        \end{cases}
        \label{eq:A0302}
    \end{equation}
    同解.
\end{theorem}

\begin{proof}
    任取方程组~\eqref{eq:A0301} 的一个解 \((s, t)\).
    于是, \(f_1 (s, t) = f_2 (s, t) = 0\).
    从而 \(f_1 (s, t) + kf_2 (s, t) = 0 + k0 = 0\).
    也就是说,
    方程组~\eqref{eq:A0301} 的每一个解%
    都是方程组~\eqref{eq:A0302} 的解.

    反过来, 任取方程组~\eqref{eq:A0302} 的一个解 \((u, v)\).
    于是,
    \(f_1 (u, v) + k f_2 (u, v) = f_2 (u, v) = 0\).
    从而 \(f_1 (u, v)
    = (f_1 (u, v) + k f_2 (u, v)) - kf_2 (u, v)
    = 0 - k0 = 0\).
    也就是说,
    方程组~\eqref{eq:A0302} 的每一个解%
    都是方程组~\eqref{eq:A0301} 的解.
\end{proof}

根据上个定理,
方程组~\eqref{eq:A0301} 与方程组~\eqref{eq:A0302}
同解.
那么, 能否取适当的 \(k\),
使方程组~\eqref{eq:A0302} 的%
首个方程的左侧可被写为%
二个 \(2\)~元 \({\leq} 1\)~次式的积?
我们计算 \(f_3 (x, y) = f_1 (x, y) + kf_2 (x, y)\)
的判别式:
\begin{align*}
    \Delta_3
    = {} & \begin{vmatrix}
               A_1 + kA_2 & B_1 + kB_2 & D_1 + kD_2 \\
               B_1 + kB_2 & C_1 + kC_2 & E_1 + kE_2 \\
               D_1 + kD_2 & E_1 + kE_2 & F_1 + kF_2 \\
           \end{vmatrix}       \\
    = {} & \Delta_1 + c_1 k + c_2 k^2 + \Delta_2 k^3,
\end{align*}
其中, \(\Delta_i\) 是 \(f_i (x, y)\)~的判别式
(\(i = 1\), \(2\)), 且
\begin{align*}
    % c_1
    % = {} & A_2 C_1 F_1+A_1 C_2 F_1+A_1 C_1 F_2-A_2 E_1^2-2 A_1 E_1 E_2 \\
    %      & {} +2 B_1 D_2 E_1+2 B_1 D_1 E_2+2 B_2 D_1 E_1-B_1^2 F_2     \\
    %      & {} -2 B_2 B_1 F_1-C_2 D_1^2-2 C_1 D_1 D_2,                  \\
    % c_2
    % = {} & A_2 C_2 F_1+A_2 C_1 F_2+A_1 C_2 F_2-A_1 E_2^2-2 A_2 E_1 E_2 \\
    %      & {} +2 B_2 D_2 E_1+2 B_2 D_1 E_2+2 B_1 D_2 E_2-B_2^2 F_1     \\
    %      & {} -2 B_1 B_2 F_2-C_1 D_2^2-2 C_2 D_1 D_2.
    c_1
    = {} &
    \begin{vmatrix}
        A_2 & B_1 & D_1 \\
        B_2 & C_1 & E_1 \\
        D_2 & E_1 & F_1 \\
    \end{vmatrix}
    +
    \begin{vmatrix}
        A_1 & B_2 & D_1 \\
        B_1 & C_2 & E_1 \\
        D_1 & E_2 & F_1 \\
    \end{vmatrix}
    +
    \begin{vmatrix}
        A_1 & B_1 & D_2 \\
        B_1 & C_1 & E_2 \\
        D_1 & E_1 & F_2 \\
    \end{vmatrix}, \\
    c_2
    = {} &
    \begin{vmatrix}
        A_1 & B_2 & D_2 \\
        B_1 & C_2 & E_2 \\
        D_1 & E_2 & F_2 \\
    \end{vmatrix}
    +
    \begin{vmatrix}
        A_2 & B_1 & D_2 \\
        B_2 & C_1 & E_2 \\
        D_2 & E_1 & F_2 \\
    \end{vmatrix}
    +
    \begin{vmatrix}
        A_2 & B_2 & D_1 \\
        B_2 & C_2 & E_1 \\
        D_2 & E_2 & F_1 \\
    \end{vmatrix}.
\end{align*}
\(\Delta_3\) 是一个 (关于 \(k\)~的) \(1\)~元 \({\leq} 3\)~次式,
故 \(\Delta_3 = 0\) 是一个 \(1\)~元 \({\leq} 3\)~次方程.
理论地, 这是可解的.
解出一个 \(k\).
从而, 我们可用 ``老方法''
解跟方程组~\eqref{eq:A0301} 同解的%
方程组~\eqref{eq:A0302},
进而得到方程组~\eqref{eq:A0301} 的解.

\begingroup
\renewcommand\thmcontinues[1]{%
    %\ifcsname hyperref\endcsname
    %\hyperref[#1]{continuing}
    %\else
    %continuing
    %\fi
    %from p.\,\pageref{#1}%
    续%
}
\begin{example}[continues=emp:A0301]
    我们解当时未能求解的方程组
    \begin{align*}
        \begin{cases}
            f_1 (x, y) = x^2 + 3y^2 + 2x - 11 = 0, \\
            f_2 (x, y) = 3x^2 - y^2 - 4x - 3 = 0.
        \end{cases}
    \end{align*}

    待定复数 \(k\).
    作出同解方程组
    \begin{align*}
        \begin{cases}
            f_3 (x, y) = 0, \\
            f_2 (x, y) = 0,
        \end{cases}
    \end{align*}
    其中
    \begin{align*}
        f_3 (x, y)
        = {} & f_1 (x, y) + kf_2 (x, y)                    \\
        = {} & (1+3k)x^2 + (3-k)y^2 + 2(1-2k)x + (-3k-11).
    \end{align*}
    计算 \(f_3 (x, y)\) 的判别式
    \begin{align*}
        \Delta_3
        = {} & \begin{vmatrix}
                   1+3k & 0   & 1-2k   \\
                   0    & 3-k & 0      \\
                   1-2k & 0   & -3k-11 \\
               \end{vmatrix}      \\
        = {} & (k-3)(13k^2 + 32k + 12).
    \end{align*}
    所以, 我们可取 \(k = 3\).
    则
    \begin{align*}
        f_3 (x, y) = 10x^2 - 10x - 20 = 10(x - 2)(x + 1).
    \end{align*}
    从而 \(x = 2\) 或 \(x = -1\).
    联立 \(x = 2\) 与 \(f_2 (x, y) = 0\),
    可解出
    \begin{align*}
        (x, y) = (2, 1)
        \quad \text{或} \quad
        (x, y) = (2, -1).
    \end{align*}
    联立 \(x = -1\) 与 \(f_2 (x, y) = 0\),
    可解出
    \begin{align*}
        (x, y) = (-1, 2)
        \quad \text{或} \quad
        (x, y) = (-1, -2).
    \end{align*}
    经验证,
    \((2, 1)\), \((2, -1)\), \((-1, 2)\), \((-1, -2)\)
    都是解.
\end{example}
\endgroup

\section{\texorpdfstring{\(1\)~元 \(4\)~次方程}%
  {1 元 4 次方程}}

我们知道, 理论地,
每一个 \(1\)~元 \(m\)~次方程都是可被 (根式) 求解的,
其中 \(m = 1\), \(2\), \(3\).
现在, 我展现一种解 \(1\)~元 \(4\)~次方程的方法.

任取一个 \(1\)~元 \(4\)~次方程
\(Ax^4 + 2Dx^3 + Fx^2 + 2Ex + C = 0\),
其中 \(A \neq 0\).
若 \(C = 0\), 则我们可写
\begin{align*}
    Ax^4 + 2Dx^3 + Fx^2 + 2Ex = x(Ax^3 + 2Dx^2 + Fx + 2E).
\end{align*}
括号里有一个 \(1\)~元 \(3\)~次式.
这, 理论地, 是可解的.
所以, 接下来, 我们假定 \(C \neq 0\).

假定 \(s\) 是
\(Ax^4 + 2Dx^3 + Fx^2 + 2Ex + C = 0\)
的一个解.
因为 \(C \neq 0\), 故 \(s \neq 0\).
从而
\begin{align*}
    As^2 + 2Ds + F + \frac{2E}{s} + \frac{C}{s^2} = 0.
\end{align*}
由此可见, \((s, 1/s)\) 是 \(2\)~元 \(2\)~次方程组
\begin{equation}
    \begin{cases}
        f_1 (x, y) = Ax^2 + Cy^2 + 2Dx + 2Ey + F = 0, \\
        f_2 (x, y) = 2xy - 2 = 0
    \end{cases}
    \label{eq:A0401}
\end{equation}
的一个解.

反过来, 设 \((u, v)\) 是这个方程组的一个解.
那么, \(v = 1/u\).
从而
\begin{align*}
    Au^2 + \frac{C}{u^2} + 2Du + \frac{2E}{u} + F = 0,
\end{align*}
即
\begin{align*}
    Au^4 + C + 2Du^3 + 2Eu + Fu^2 = 0.
\end{align*}
故 \(u\) 是
\(Ax^4 + 2Dx^3 + Fx^2 + 2Ex + C = 0\)
的一个解.

这么看来, 我们可变%
解 \(1\)~元 \(4\)~次方程的问题%
为%
解 \(2\)~元 \(2\)~次方程组的问题.

待定复数 \(k\).
作一个跟方程组~\eqref{eq:A0401} 同解的方程组
\begin{align*}
    \begin{cases}
        f_3 (x, y) = 0, \\
        f_2 (x, y) = 0,
    \end{cases}
\end{align*}
其中
\begin{align*}
    f_3 (x, y)
    = {} & f_1 (x, y) + kf_2 (x, y)                   \\
    = {} & Ax^2 + 2kxy + Cy^2 + 2Dx + 2Ey + (F - 2k).
\end{align*}
计算 \(f_3 (x, y)\) 的判别式
\begin{align*}
    \Delta_3
    = {} & \begin{vmatrix}
               A & k & D    \\
               k & C & E    \\
               D & E & F-2k \\
           \end{vmatrix}                                 \\
    = {} & 2k^3 - Fk^2 - 2(AC-DE)k + (ACF - AE^2 - CD^2).
\end{align*}
\(\Delta_3\) 是一个 (关于 \(k\)~的) \(1\)~元 \(3\)~次式,
故 \(\Delta_3 = 0\) 是一个 \(1\)~元 \(3\)~次方程.
理论地, 这是可解的.
解出一个 \(k\).
从而, 我们可用 ``老方法'' 解方程组,
进而得到 \(1\)~元 \(4\)~次方程的解.

\begin{example}
    解方程
    \begin{align*}
        x^4 - 6x^3 + 11x^2 - 6x - 24 = 0.
    \end{align*}

    注意到 \(0\)~次项 (\(-24\)) 非零,
    故此方程无零解.
    考虑解方程组
    \begin{align*}
        \begin{cases}
            f_1 (x, y) = x^2 - 6x + 11 - 6y - 24y^2 = 0, \\
            f_2 (x, y) = 2xy - 2 = 0.
        \end{cases}
    \end{align*}
    此方程组的解的首个分量即为原方程的解.

    待定复数 \(k\).
    作出同解方程组
    \begin{align*}
        \begin{cases}
            f_3 (x, y) = 0, \\
            f_2 (x, y) = 0,
        \end{cases}
    \end{align*}
    其中
    \begin{align*}
        f_3 (x, y)
        = {} & f_1 (x, y) + kf_2 (x, y)                  \\
        = {} & x^2 + 2kxy - 24y^2 - 6x - 6y + (11 - 2k).
    \end{align*}
    计算 \(f_3 (x, y)\) 的判别式
    \begin{align*}
        \Delta_3
        = {} & \begin{vmatrix}
                   1  & k   & -3    \\
                   k  & -24 & -3    \\
                   -3 & -3  & 11-2k \\
               \end{vmatrix}        \\
        = {} & (k-1) (2k^2 - 9k + 57).
    \end{align*}
    所以, 我们可取 \(k = 1\).
    则
    \begin{align*}
        f_3 (x, y)
        = {} & x^2 + 2xy - 24y^2 - 6x - 6y + 9          \\
        = {} & x^2 + 2x(y-3) - 24y^2 - 6y + 9           \\
        = {} & (x + y - 3)^2 - (y-3)^2 - 24y^2 - 6y + 9 \\
        = {} & (x + y - 3)^2 - (5y)^2                   \\
        = {} & (x - 4y - 3) (x + 6y - 3).
    \end{align*}
    所以
    \begin{align*}
        x - 4y - 3 = 0
        \quad \text{或} \quad
        x + 6y - 3 = 0.
    \end{align*}
    从而
    \begin{align*}
        x^2 - 4xy - 3x = 0
        \quad \text{或} \quad
        x^2 + 6xy - 3x = 0.
    \end{align*}
    因为 \(2xy - 2 = 0\),
    故
    \begin{align*}
        x^2 - 3x - 4 = 0
        \quad \text{或} \quad
        x^2 - 3x + 6 = 0.
    \end{align*}
    注意,
    我们的目的是求解 (原方程的) \(x\),
    而 \(y\) 只是辅助量,
    故我们用 \(y = 1/x\) 消去了 \(y\),
    进而直接求解 \(x\).
    我们不必同时解出 \(x\) 与 \(y\).

    \(1\)~元 \(2\)~次方程是不难求解的.
    不难算出
    \begin{align*}
        x = -1
        \quad \text{或} \quad
        x = 4
        \quad \text{或} \quad
        x = \frac{3 + \mathrm{i}\sqrt{15}}{2}
        \quad \text{或} \quad
        x = \frac{3 - \mathrm{i}\sqrt{15}}{2}.
    \end{align*}
    经验证, 它们都是原方程的解.
\end{example}
