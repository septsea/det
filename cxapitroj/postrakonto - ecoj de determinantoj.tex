\section{按一行展开行列式}

本节, 我想用定义 (按列~\(1\) 展开),
证明按一行展开行列式的公式.

\begin{theorem}
    设 \(A\) 是 \(n\)~级阵 (\(n \geq 1\)).
    设 \(i\) 为整数, 且 \(1 \leq i \leq n\).
    则
    \begin{align*}
        \det {(A)} = \sum_{j = 1}^{n}
        {(-1)^{i+j} [A]_{i,j} \det {(A(i|j))}}.
    \end{align*}
\end{theorem}

\begin{proof}
    我们用数学归纳法证明此事.
    具体地, 设 \(P(n)\) 为命题
    \begin{quotation}
        对任何 \(n\)~级阵 \(A\),
        对任何不超过 \(n\) 的正整数 \(i\),
        \begin{align*}
            \det {(A)} = \sum_{j = 1}^{n}
            {(-1)^{i+j} [A]_{i,j} \det {(A(i|j))}}.
        \end{align*}
    \end{quotation}
    则, 我们的目标是:
    对任何正整数 \(n\), \(P(n)\) 是正确的.

    \(P(1)\) 显然是正确的.

    可以验证, \(P(2)\) 也是正确的 (习题).

    现在, 我们假定 \(P(m-1)\) 是正确的.
    我们要证 \(P(m)\) 也是正确的.
    任取一个 \(m\)~级阵 \(A\).
    任取不超过 \(m\) 的正整数 \(i\).
    为方便, 我们记
    \(f = (-1)^{i+1} [A]_{i,1} \det {(A(i|1))}\).
    则 (第~2~个等号利用了假定,
    并注意到,
    \(A\)~的行~\(i\) (其中 \(k \neq i\)) 对应
    \(A(k|1)\)~的行~\(i - \rho(i,k)\),
    \(A\)~的列~\(j\) (其中 \(j \neq 1\)) 对应
    \(A(k|1)\)~的列~\(j - 1\))
    \begin{align*}
             & \det {(A)}
        \\
        = {} &
        f + \sum_{\substack{1 \leq k \leq m \\k \neq i}}
        {(-1)^{k+1} [A]_{k,1} \det {(A(k|1))}}
        \\
        = {} &
        f + \sum_{\substack{1 \leq k \leq m \\k \neq i}}
        {(-1)^{k+1} [A]_{k,1}
        \sum_{j = 2}^{m}
        {(-1)^{i - \rho(i,k) + j - 1}
            [A]_{i,j} \det {(A(k,i|1,j))}}}
        \\
        = {} &
        f + \sum_{\substack{1 \leq k \leq m \\k \neq i}}
        {\sum_{j = 2}^{m}
        {(-1)^{k+1} [A]_{k,1}
        (-1)^{i - \rho(i,k) + j - 1}
            [A]_{i,j} \det {(A(k,i|1,j))}}}
        \\
        = {} &
        f + \sum_{j = 2}^{m}
        {\sum_{\substack{1 \leq k \leq m    \\k \neq i}}
        {(-1)^{k+1} [A]_{k,1}
            (-1)^{i - \rho(i,k) + j - 1}
                [A]_{i,j} \det {(A(k,i|1,j))}}}
        \\
        = {} &
        f + \sum_{j = 2}^{m}
        {\sum_{\substack{1 \leq k \leq m    \\k \neq i}}
        {(-1)^{i+j} [A]_{i,j}
            (-1)^{k - \rho(k,i) + 1}
                [A]_{k,1} \det {(A(k,i|1,j))}}}
        \\
        = {} &
        f + \sum_{j = 2}^{m}
        {(-1)^{i+j} [A]_{i,j}
        \sum_{\substack{1 \leq k \leq m     \\k \neq i}}
        {(-1)^{k - \rho(k,i) + 1}
                [A]_{k,1} \det {(A(k,i|1,j))}}}
        \\
        = {} &
        f + \sum_{j = 2}^{m}
        {(-1)^{i+j} [A]_{i,j} \det {(A(i|j))}}.
    \end{align*}
    所以, \(P(m)\) 是正确的.
    由数学归纳法原理, 待证命题成立.
\end{proof}

\section{方阵与其转置的行列式相等}

本节, 我想证明: 一个方阵与其转置的行列式相等.

其实, 在第一章, 节~\sekcio{15},
我已用定义 (按列~\(1\) 展开)
\emph{与}按行~\(1\) 展开证明了它.
不过, 我想, 知道别的证明也好.

% 我先引用定理:

% \begin{theorem}
%     设 \(A\) 是 \(n\)~级阵 (\(n \geq 1\)).
%     则 \(A\)~的转置, \(A^{\mathrm{T}}\),
%     的行列式等于 \(A\)~的行列式.
% \end{theorem}

以下, 设 \(P(n)\) 为命题
\begin{quotation}
    对\emph{任何} \(n\)~级阵 \(A\),
    \begin{align*}
        \det {(A^{\mathrm{T}})} = \det {(A)}.
    \end{align*}
\end{quotation}
则, 我们的目标是:
对任何正整数 \(n\), \(P(n)\) 是正确的.

\begin{proof}[同时用按列~\(1\) 展开与按行~\(1\) 展开]
    见第一章, 节~\sekcio{15}.
\end{proof}

\begin{proof}[用按一列展开]
    \(P(1)\) 是正确的.
    毕竟, \(1\)~级阵的转置就是自己.

    现在, 我们假定 \(P(m-1)\) 是正确的.
    我们要证 \(P(m)\) 也是正确的.

    任取一个 \(m\)~级阵 \(A\).
    一方面, 我们知道,
    对任何数 \(x\), 与任何正整数 \(s\),
    必有
    \begin{align*}
        x = \frac{1}{s} \sum_{j = 1}^{s} {x}.
    \end{align*}
    另一方面, 我们已知,
    对每一个 \(s\) 级阵 \(B\),
    与每一个不超过 \(s\) 的正整数 \(j\),
    \begin{align*}
        \det {(B)}
        = \sum_{i = 1}^{s} {(-1)^{i+j} [B]_{i,j} \det {(B(i|j))}}.
    \end{align*}
    最后, 注意到, 既然
    \([A]_{i,j} = [A^{\mathrm{T}}]_{j,i}\),
    则, 不难验证,
    \(A^{\mathrm{T}}(i|j) = (A(j|i))^{\mathrm{T}}\).
    结合这三件事,
    与用过多次的加法的结合律与交换律,
    并利用假定 (第~4~个等号), 我们有
    \begin{align*}
             & \det {(A^{\mathrm{T}})}
        \\
        = {} &
        \frac{1}{m}
        \sum_{j=1}^{m}
        {\det {(A^{\mathrm{T}})}}
        \\
        = {} &
        \frac{1}{m}
        \sum_{j=1}^{m} {
        \sum_{i=1}^{m} {
        (-1)^{i+j} [A^{\mathrm{T}}]_{i,j}
        \det {(A^{\mathrm{T}}(i|j))}}}
        \\
        = {} &
        \frac{1}{m}
        \sum_{j=1}^{m} {
        \sum_{i=1}^{m} {
        (-1)^{i+j} [A]_{j,i}
        \det {((A(j|i))^{\mathrm{T}})}}}
        \\
        = {} &
        \frac{1}{m}
        \sum_{j=1}^{m} {
        \sum_{i=1}^{m} {
        (-1)^{i+j} [A]_{j,i}
        \det {(A(j|i))}}}
        \\
        = {} &
        \frac{1}{m}
        \sum_{j=1}^{m} {
        \sum_{i=1}^{m} {
        (-1)^{j+i} [A]_{j,i}
        \det {(A(j|i))}}}
        \\
        = {} &
        \frac{1}{m}
        \sum_{i=1}^{m} {
        \sum_{j=1}^{m} {
        (-1)^{j+i} [A]_{j,i}
        \det {(A(j|i))}}}
        \\
        = {} &
        \frac{1}{m}
        \sum_{i=1}^{m}
        {\det {(A)}}
        \\
        = {} & \det {(A)}.
    \end{align*}
    所以, \(P(m)\) 是正确的.
    由数学归纳法原理, 待证命题成立.
\end{proof}

\begin{proof}[用按列~\(1\) 展开]
    作辅助命题 \(Q(n)\):
    \begin{quotation}
        \(P(n-1)\) 与 \(P(n)\) 是正确的.
    \end{quotation}
    我们用数学归纳法证明:
    对任何高于 \(1\) 的整数 \(n\), \(Q(n)\) 是正确的.
    由此, 我们可知, 对任何正整数 \(n\),
    \(P(n)\) 是正确的.

    不难验证 \(Q(2)\) 是正确的.

    现在, 我们假定 \(Q(m-1)\) 是正确的.
    我们要证 \(Q(m)\) 也是正确的.
    既然 \(Q(m-1)\) 是正确的,
    则 \(P(m-2)\) 与 \(P(m-1)\) 是正确的.
    所以, 若我们能由此证明 \(P(m)\) 是正确的,
    则 \(Q(m)\) 是正确的.

    任取一个 \(m\)~级阵 \(A\).
    则
    \begin{align*}
             & \det {(A^{\mathrm{T}})}
        \\
        = {} &
        \sum_{i = 1}^{m} {
        (-1)^{i+1} [A^{\mathrm{T}}]_{i,1}
        \det {(A^{\mathrm{T}}(i|1))}
        }
        \\
        = {} &
        \hphantom{{} + {}}
        (-1)^{1+1} [A^{\mathrm{T}}]_{1,1}
        \det {(A^{\mathrm{T}}(1|1))}
        \\
             &
        + \sum_{i = 2}^{m} {
        (-1)^{i+1} [A^{\mathrm{T}}]_{i,1}
        \det {(A^{\mathrm{T}}(i|1))}
        }
        \\
        = {} &
        \hphantom{{} + {}}
        (-1)^{1+1} [A]_{1,1} \det {(A(1|1))}
        \tag*{(1)}
        \\
             &
        + \sum_{i = 2}^{m} {
        (-1)^{i+1} [A^{\mathrm{T}}]_{i,1}
        \det {(A(1|i))}
        }
        \tag*{(2)}
        \\
        = {} &
        \hphantom{{} + {}}
        (-1)^{1+1} [A]_{1,1} \det {(A(1|1))}
        \\
             &
        + \sum_{i = 2}^{m} {
        (-1)^{i+1} [A^{\mathrm{T}}]_{i,1}
        \sum_{k = 2}^{m} {
        (-1)^{k-1+1} [A]_{k,1} \det {(A({1,k}|{i,1}))}
        }
        }
        \tag*{(3)}
        \\
        = {} &
        \hphantom{{} + {}}
        (-1)^{1+1} [A]_{1,1} \det {(A(1|1))}
        \\
             &
        + \sum_{i = 2}^{m} {
        \sum_{k = 2}^{m} {
        (-1)^{i+1} [A^{\mathrm{T}}]_{i,1}
        (-1)^{k-1+1} [A]_{k,1} \det {(A({1,k}|{i,1}))}
        }
        }
        \\
        = {} &
        \hphantom{{} + {}}
        (-1)^{1+1} [A]_{1,1} \det {(A(1|1))}
        \\
             &
        + \sum_{k = 2}^{m} {
        \sum_{i = 2}^{m} {
        (-1)^{i+1} [A^{\mathrm{T}}]_{i,1}
        (-1)^{k-1+1} [A]_{k,1} \det {(A({1,k}|{i,1}))}
        }
        }
        \\
        = {} &
        \hphantom{{} + {}}
        (-1)^{1+1} [A]_{1,1} \det {(A(1|1))}
        \\
             &
        + \sum_{k = 2}^{m} {
        \sum_{i = 2}^{m} {
        (-1)^{k+1} [A]_{k,1}
        (-1)^{i-1+1} [A^{\mathrm{T}}]_{i,1}
        \det {(A({1,k}|{i,1}))}
        }
        }
        \\
        = {} &
        \hphantom{{} + {}}
        (-1)^{1+1} [A]_{1,1} \det {(A(1|1))}
        \\
             &
        + \sum_{k = 2}^{m} {
        (-1)^{k+1} [A]_{k,1}
        \sum_{i = 2}^{m} {
        (-1)^{i-1+1} [A^{\mathrm{T}}]_{i,1}
        \det {(A({1,k}|{i,1}))}
        }
        }
        \\
        = {} &
        \hphantom{{} + {}}
        (-1)^{1+1} [A]_{1,1} \det {(A(1|1))}
        \\
             &
        + \sum_{k = 2}^{m} {
        (-1)^{k+1} [A]_{k,1}
        \sum_{i = 2}^{m} {
        (-1)^{i-1+1} [A^{\mathrm{T}}]_{i,1}
        \det {(A^{\mathrm{T}} ({i,1}|{1,k}))}
        }
        }
        \tag*{(4)}
        \\
        = {} &
        \hphantom{{} + {}}
        (-1)^{1+1} [A]_{1,1} \det {(A(1|1))}
        \\
             &
        + \sum_{k = 2}^{m} {
        (-1)^{k+1} [A]_{k,1}
        \det {(A^{\mathrm{T}}(1|k))}
        }
        \tag*{(5)}
        \\
        = {} &
        \hphantom{{} + {}}
        (-1)^{1+1} [A]_{1,1} \det {(A(1|1))}
        \\
             &
        + \sum_{k = 2}^{m} {
        (-1)^{k+1} [A]_{k,1}
        \det {(A(k|1))}
        }
        \tag*{(6)}
        \\
        = {} &
        \sum_{k = 1}^{m} {
        (-1)^{k+1} [A]_{k,1}
        \det {(A(k|1))}
        }
        \\
        = {} &
        \det {(A)}.
    \end{align*}
    所以, \(P(m)\) 是正确的.
    所以, \(Q(m)\) 是正确的.
    由数学归纳法原理, 待证命题成立.

    我简单地作一些解释吧.

    (1) (2)
    都利用了假定 \(P(m-1)\).
    注意, \(A^{\mathrm{T}} (i|1)\)
    的转置是 \(A(1|i)\).

    (3)
    利用了按列~\(1\) 展开.
    注意, \([A]_{k,1}\) 是 \(A(1|i)\)~的
    \((k-1, 1)\)-元
    (\(i, k > 1\)).

    (4)
    利用了假定 \(P(m-2)\).
    注意, \(A({1,k}|{i,1})\)
    的转置是 \(A^{\mathrm{T}} ({i,1}|{1,k})\).

    (5)
    利用了按列~\(1\) 展开.
    注意, \([A^{\mathrm{T}}]_{i,1}\) 是
    \(A^{\mathrm{T}} (1|k)\)~的
    \((i-1, 1)\)-元
    (\(i, k > 1\)).

    (6)
    利用了假定 \(P(m-1)\).
    注意, \(A^{\mathrm{T}} (1|k)\)
    的转置是 \(A(k|1)\).
\end{proof}

\section{(关于列的) 多线性}

本节, 我想证明 (关于列的) 多线性.

其实, 在第一章, 节~\sekcio{13},
我已用按一列展开行列式的公式证明了它.
不过, 我想, 知道别的证明也好.

% 我先引用定理:

% \begin{theorem}
%     行列式 (关于列) 是多线性的.
%     具体地, 对任何不超过 \(n\) 的正整数 \(j\),
%     任何 \(n-1\)~个 \(n \times 1\)~阵
%     \(a_1\), \(\dots\), \(a_{j-1}\),
%     \(a_{j+1}\), \(\dots\), \(a_n\),
%     任何二个 \(n \times 1\)~阵 \(x\), \(y\),
%     任何二个数 \(s\), \(t\),
%     有
%     \begin{align*}
%              & \det
%         {[a_1, \dots, a_{j-1}, sx + ty, a_{j+1}, \dots, a_n]}
%         \\
%         = {} &
%         s
%         \det {[a_1, \dots, a_{j-1}, x, a_{j+1}, \dots, a_n]}
%         +
%         t
%         \det {[a_1, \dots, a_{j-1}, y, a_{j+1}, \dots, a_n]}.
%     \end{align*}
%     (若 \(j = 1\), 则 \(a_1\) 不出现;
%     若 \(j = n\), 则 \(a_n\) 不出现.
%     下同.)
% \end{theorem}

以下, 设 \(P(n)\) 为命题
\begin{quotation}
    对任何不超过 \(n\) 的正整数 \(j\),
    任何 \(n-1\)~个 \(n \times 1\)~阵
    \(a_1\), \(\dots\), \(a_{j-1}\),
    \(a_{j+1}\), \(\dots\), \(a_n\),
    任何二个 \(n \times 1\)~阵 \(x\), \(y\),
    任何二个数 \(s\), \(t\),
    有
    \begin{align*}
             & \det
        {[a_1, \dots, a_{j-1}, sx + ty, a_{j+1}, \dots, a_n]}
        \\
        = {} &
        s
        \det {[a_1, \dots, a_{j-1}, x, a_{j+1}, \dots, a_n]}
        +
        t
        \det {[a_1, \dots, a_{j-1}, y, a_{j+1}, \dots, a_n]}.
    \end{align*}
\end{quotation}
则, 我们的目标是:
对任何正整数 \(n\), \(P(n)\) 是正确的.

\begin{proof}[用按一列展开]
    见第一章, 节~\sekcio{13}.
\end{proof}

\begin{proof}[用按列~\(1\) 展开]
    不难验证 \(P(1)\) 是正确的.

    现在, 我们假定 \(P(m-1)\) 是正确的.
    我们要证 \(P(m)\) 也是正确的.

    任取不超过 \(m\) 的正整数 \(j\).
    任取 \(m-1\)~个 \(m \times 1\)~阵
    \(a_1\), \(\dots\), \(a_{j-1}\),
    \(a_{j+1}\), \(\dots\), \(a_m\).
    任取二个 \(m \times 1\)~阵 \(x\), \(y\),
    任取二个数 \(s\), \(t\).
    作三个 \(m\)~级阵 \(A\), \(B\), \(C\):
    \(A\), \(B\), \(C\)~的列~\(k\) 为 \(a_k\) (\(k \neq j\));
    \(A\)~的列~\(j\) 为 \(x\);
    \(B\)~的列~\(j\) 为 \(y\);
    \(C\)~的列~\(j\) 为 \(sx + ty\).
    那么,
    \begin{align*}
         & \det {(A)}
        = \det
        {[a_1, \dots, a_{j-1}, x, a_{j+1}, \dots, a_m]},
        \\
         & \det {(B)}
        = \det
        {[a_1, \dots, a_{j-1}, y, a_{j+1}, \dots, a_m]},
        \\
         & \det {(C)}
        = \det
        {[a_1, \dots, a_{j-1}, sx + ty, a_{j+1}, \dots, a_m]}.
    \end{align*}
    不难发现, 若 \(k \neq j\), 则
    \(
    [A]_{i,k} = [B]_{i,k} = [C]_{i,k},
    \)
    故
    \(
    A(i|j) = B(i|j) = C(i|j).
    \)

    设 \(j = 1\).
    那么
    \begin{align*}
             & \det {(C)}
        \\
        = {} &
        \sum_{i = 1}^{m} {
        (-1)^{i+j} [C]_{i,j} \det {(C(i|j))}
        }
        \\
        = {} &
        \sum_{i = 1}^{m} {
        (-1)^{i+j} (s[A]_{i,j} + t[B]_{i,j}) \det {(C(i|j))}
        }
        \\
        = {} &
        s\sum_{i = 1}^{m} {
        (-1)^{i+j} [A]_{i,j} \det {(C(i|j))}
        }
        +
        t\sum_{i = 1}^{m} {
        (-1)^{i+j} [B]_{i,j} \det {(C(i|j))}
        }
        \\
        = {} &
        s\sum_{i = 1}^{m} {
        (-1)^{i+j} [A]_{i,j} \det {(A(i|j))}
        }
        +
        t\sum_{i = 1}^{m} {
        (-1)^{i+j} [B]_{i,j} \det {(B(i|j))}
        }
        \\
        = {} &
        s \det{(A)} + t \det{(B)}.
    \end{align*}

    设 \(j > 1\).
    设 \(A(i|1)\), \(B(i|1)\), \(C(i|1)\)
    的列~\(j-1\) 分别是 \(p\), \(q\), \(r\).
    则 \(r = sp + tq\).
    由假定,
    \(\det {(C(i|1))} = s\det {(A(i|1))} + t\det {(B(i|1))}\).
    从而
    \begin{align*}
             & \det {(C)}
        \\
        = {} &
        \sum_{i = 1}^{m}
        {(-1)^{i+1} [C]_{i,1} \det {(C(i|1))}}
        \\
        = {} &
        \sum_{i = 1}^{m}
        {(-1)^{i+1} [C]_{i,1}
        (s\det {(A(i|1))} + t\det {(B(i|1))})}
        \\
        = {} &
        s \sum_{i = 1}^{m}
        {(-1)^{i+1} [C]_{i,1} \det {(A(i|1))}}
        +
        t \sum_{i = 1}^{m}
        {(-1)^{i+1} [C]_{i,1} \det {(B(i|1))}}
        \\
        = {} &
        s \sum_{i = 1}^{m}
        {(-1)^{i+1} [A]_{i,1} \det {(A(i|1))}}
        +
        t \sum_{i = 1}^{m}
        {(-1)^{i+1} [B]_{i,1} \det {(B(i|1))}}
        \\
        = {} &
        s \det {(A)} + t \det {(B)}.
    \end{align*}
    所以, \(P(m)\) 是正确的.
    由数学归纳法原理, 待证命题成立.
\end{proof}

\begin{proof}[用按行~\(1\) 展开]
    不难验证 \(P(1)\) 是正确的.

    现在, 我们假定 \(P(m-1)\) 是正确的.
    我们要证 \(P(m)\) 也是正确的.

    任取不超过 \(m\) 的正整数 \(j\).
    任取 \(m-1\)~个 \(m \times 1\)~阵
    \(a_1\), \(\dots\), \(a_{j-1}\),
    \(a_{j+1}\), \(\dots\), \(a_m\).
    任取二个 \(m \times 1\)~阵 \(x\), \(y\),
    任取二个数 \(s\), \(t\).
    作三个 \(m\)~级阵 \(A\), \(B\), \(C\):
    \(A\), \(B\), \(C\)~的列~\(k\) 为 \(a_k\) (\(k \neq j\));
    \(A\)~的列~\(j\) 为 \(x\);
    \(B\)~的列~\(j\) 为 \(y\);
    \(C\)~的列~\(j\) 为 \(sx + ty\).
    那么,
    \begin{align*}
         & \det {(A)}
        = \det
        {[a_1, \dots, a_{j-1}, x, a_{j+1}, \dots, a_m]},
        \\
         & \det {(B)}
        = \det
        {[a_1, \dots, a_{j-1}, y, a_{j+1}, \dots, a_m]},
        \\
         & \det {(C)}
        = \det
        {[a_1, \dots, a_{j-1}, sx + ty, a_{j+1}, \dots, a_m]}.
    \end{align*}

    一方面, 若 \(k = j\), 则
    \([C]_{1,k} = x[A]_{1,k} + y[B]_{1,k}\),
    且
    \(A(1|k) = B(1|k) = C(1|k)\).
    另一方面, 若 \(k \neq j\),
    则
    \([A]_{i,k} = [B]_{i,k} = [C]_{i,k}\).
    设 \(A(1|k)\), \(B(1|k)\), \(C(1|k)\)
    的列~\(j - \rho(j, k)\)
    分别是
    \(p\), \(q\), \(r\).
    则
    \(r = sp + tq\).
    由假定,
    \(\det {(C(1|k))} = s \det {(A(1|k))} + t \det {(B(1|k))}\).
    从而
    \begin{align*}
             & \det {(C)}
        \\
        = {} &
        \sum_{k = 1}^{m}
        {(-1)^{1 + k} [C]_{1,k} \det {(C(1|k))}}
        \\
        = {} &
        (-1)^{1 + j} [C]_{1,j} \det {(C(1|j))}
        +
        \sum_{\substack{1 \leq k \leq m     \\k \neq j}}
        {(-1)^{1 + k} [C]_{1,k} \det {(C(1|k))}}
        \\
        = {} &
        \hphantom{{} + {}}
        (-1)^{1 + j} (s[A]_{1,j} + t[B]_{1,j}) \det {(C(1|j))}
        \\
             &
        +
        \sum_{\substack{1 \leq k \leq m     \\k \neq j}}
        {(-1)^{1 + k} [C]_{1,k}
            (s \det {(A(1|k))} + t \det {(B(1|k))})}
        \\
        = {} &
        \hphantom{{} + {}}
        s (-1)^{1 + j} [A]_{1,j} \det {(C(1|j))}
        + t (-1)^{1 + j} [B]_{1,j} \det {(C(1|j))}
        \\
             &
        + s \sum_{\substack{1 \leq k \leq m \\k \neq j}}
        {(-1)^{1 + k} [C]_{1,k} \det {(A(1|k))}}
        \\
             &
        + t \sum_{\substack{1 \leq k \leq m \\k \neq j}}
        {(-1)^{1 + k} [C]_{1,k} \det {(B(1|k))}}
        \\
        = {} &
        \hphantom{{} + {}}
        s (-1)^{1 + j} [A]_{1,j} \det {(A(1|j))}
        + t (-1)^{1 + j} [B]_{1,j} \det {(B(1|j))}
        \\
             &
        + s \sum_{\substack{1 \leq k \leq m \\k \neq j}}
        {(-1)^{1 + k} [A]_{1,k} \det {(A(1|k))}}
        \\
             &
        + t \sum_{\substack{1 \leq k \leq m \\k \neq j}}
        {(-1)^{1 + k} [B]_{1,k} \det {(B(1|k))}}
        \\
        = {} &
        \hphantom{{} + {}}
        s (-1)^{1 + j} [A]_{1,j} \det {(A(1|j))}
        + s \sum_{\substack{1 \leq k \leq m \\k \neq j}}
        {(-1)^{1 + k} [A]_{1,k} \det {(A(1|k))}}
        \\
             &
        + t (-1)^{1 + j} [B]_{1,j} \det {(B(1|j))}
        + t \sum_{\substack{1 \leq k \leq m \\k \neq j}}
        {(-1)^{1 + k} [B]_{1,k} \det {(B(1|k))}}
        \\
        = {} &
        s \sum_{k = 1}^{m}
        {(-1)^{1 + k} [A]_{1,k} \det {(A(1|k))}}
        + t \sum_{k = 1}^{m}
        {(-1)^{1 + k} [B]_{1,k} \det {(B(1|k))}}
        \\
        = {} & s \det {(A)} + t \det {(B)}.
    \end{align*}
    所以, \(P(m)\) 是正确的.
    由数学归纳法原理, 待证命题成立.
\end{proof}

\section{按前二列展开行列式}

本节, 我想证明一个公式.
它在交错性的一个证明%
与反称性的一个证明里有用.

\begin{theorem}
    设 \(A\) 是 \(n\)~级阵 (\(n \geq 2\)).
    则
    \begin{align*}
        \det {(A)}
        = \sum_{1 \leq i < k \leq n}
        {\det {
                \begin{bmatrix}
                    [A]_{i,1} & [A]_{i,2} \\
                    [A]_{k,1} & [A]_{k,2} \\
                \end{bmatrix}
            }
            (-1)^{i+k+1+2}
            \det {(A(i,k|1,2))}}.
    \end{align*}
\end{theorem}

\begin{proof}
    注意到, 当 \(d_2 \neq d_1\) 时,
    \([A]_{d_2,2}\) 是
    \(A(d_1|1)\) 的
    \((d_2 - \rho(d_2, d_1), 1)\)-元.
    从而
    \begin{align*}
             & \det {(A)}
        \\
        = {} &
        \sum_{d_1 = 1}^{n} {(-1)^{d_1+1} [A]_{d_1,1}
        \det {(A(d_1|1))}}
        \\
        = {} &
        \sum_{d_1 = 1}^{n} {(-1)^{d_1+1} [A]_{d_1,1}
        \sum_{\substack{1 \leq d_2 \leq n     \\d_2 \neq d_1}}
        {(-1)^{d_2 - \rho(d_2,d_1) + 1} [A]_{d_2,2}
            \det {(A(d_1,d_2|1,2))}}}
        \\
        = {} &
        \sum_{d_1 = 1}^{n}
        {\sum_{\substack{1 \leq d_2 \leq n    \\d_2 \neq d_1}}
        {(-1)^{d_1+1} [A]_{d_1,1}
            (-1)^{d_2 - \rho(d_2,d_1) + 1} [A]_{d_2,2}
            \det {(A(d_1,d_2|1,2))}}}
        \\
        = {} &
        \sum_{\substack{1 \leq d_1,d_2 \leq n \\d_1 \neq d_2}}
        {(-1)^{d_1+1} [A]_{d_1,1}
            (-1)^{d_2 - \rho(d_2,d_1) + 1} [A]_{d_2,2}
            \det {(A(d_1,d_2|1,2))}}
        \\
        = {} &
        \sum_{\substack{1 \leq d_1,d_2 \leq n \\d_1 \neq d_2}}
        {(-1)^{\rho(d_1,d_2)}
                [A]_{d_1,1} [A]_{d_2,2}
            (-1)^{d_1+d_2+1+2}
            \det {(A(d_1,d_2|1,2))}}
        \\
        = {} &
        \hphantom{{} + {}}
        \sum_{\substack{1 \leq d_1,d_2 \leq n \\d_1 < d_2}}
        {(-1)^{\rho(d_1,d_2)}
                [A]_{d_1,1} [A]_{d_2,2}
            (-1)^{d_1+d_2+1+2}
            \det {(A(d_1,d_2|1,2))}}
        \\
             &
        +
        \sum_{\substack{1 \leq d_1,d_2 \leq n \\d_1 > d_2}}
        {(-1)^{\rho(d_1,d_2)}
                [A]_{d_1,1} [A]_{d_2,2}
            (-1)^{d_1+d_2+1+2}
            \det {(A(d_1,d_2|1,2))}}
        \\
        = {} &
        \hphantom{{} + {}}
        \sum_{\substack{1 \leq i,k \leq n     \\i < k}}
        {(-1)^{\rho(i,k)}
                [A]_{i,1} [A]_{k,2}
            (-1)^{i+k+1+2}
            \det {(A(i,k|1,2))}}
        \\
             &
        +
        \sum_{\substack{1 \leq k,i \leq n     \\k > i}}
        {(-1)^{\rho(k,i)}
                [A]_{k,1} [A]_{i,2}
            (-1)^{k+i+1+2}
            \det {(A(k,i|1,2))}}
        \\
        = {} &
        \sum_{1 \leq i < k \leq n}
        {([A]_{i,1} [A]_{k,2} - [A]_{k,1} [A]_{i,2})\,
        (-1)^{i+k+1+2}
        \det {(A(i,k|1,2))}}
        \\
        = {} &
        \sum_{1 \leq i < k \leq n}
        {\det {
                \begin{bmatrix}
                    [A]_{i,1} & [A]_{i,2} \\
                    [A]_{k,1} & [A]_{k,2} \\
                \end{bmatrix}
            }
            (-1)^{i+k+1+2}
            \det {(A(i,k|1,2))}}.
        \qedhere
    \end{align*}
\end{proof}

\section{(关于列的) 交错性}

本节, 我想证明 (关于列的) 交错性.

其实, 在第一章, 节~\sekcio{13},
我已用按一列展开行列式的公式证明了它.
不过, 我想, 知道别的证明也好.

% 我先引用定理:

% \begin{theorem}
%     行列式 (关于列) 是交错性的.
%     具体地,
%     若 \(n\)~级阵 \(A\) 有二列完全相同,
%     则 \(\det {(A)} = 0\).
% \end{theorem}

以下, 设 \(P(n)\) 为命题
\begin{quotation}
    对每一个有完全相同的二列的 \(n\)~级阵 \(A\),
    其行列式必为零.
\end{quotation}
则, 我们的目标是:
对任何正整数 \(n\), \(P(n)\) 是正确的.

\begin{proof}[用按一列展开]
    见第一章, 节~\sekcio{13}.
\end{proof}

在介绍其他的证明前, 我要介绍二个有用的事实.

\begin{theorem}
    设命题~I:
    若一个方阵有\emph{相邻的}二列相同, 则其行列式为零.
    设命题~II:
    交换方阵 \(A\) 的\emph{相邻的}二列, 得方阵 \(B\),
    则 \(\det {(B)} = -\det {(A)}\).
    则, 用行列式的 (关于列的) 多线性,
    我们可由 I 推出 II.
\end{theorem}

\begin{proof}
    设 \(A\) 是一个 \(n\)~级阵 (\(n \geq 2\)).
    设 \(A\) 的列 \(1\), \(2\), \(\dots\), \(n\)
    为 \(a_1\), \(a_2\), \(\dots\), \(a_n\).
    设 \(1 \leq j < n\).
    交换 \(A\) 的列 \(j\), \(j+1\), 得 \(B\).
    记
    \begin{align*}
        f_j (x, y)
        = \det {[\dots, a_{j-1}, x, y, a_{j+2}, \dots]}.
    \end{align*}
    那么,
    \(f_j (a_j, a_{j+1}) = \det {(A)}\),
    \(f_j (a_{j+1}, a_j) = \det {(B)}\).
    则
    \begin{align*}
        0
        = {} & f_j (a_j + a_{j+1}, a_j + a_{j+1})    \\
        = {} & f_j (a_j, a_j + a_{j+1})
        + f_j (a_{j+1}, a_j + a_{j+1})               \\
        = {} & (f_j (a_j, a_j) + f_j (a_j, a_{j+1}))
        + (f_j (a_{j+1}, a_j)
        + f_j (a_{j+1}, a_{j+1}))                    \\
        = {} & f_j (a_j, a_{j+1})
        + f_j (a_{j+1}, a_j)                         \\
        = {} & \det {(A)} + \det {(B)}.
        \qedhere
    \end{align*}
\end{proof}

\begin{restatable}[]{theorem}{TheoremSwapTwoAdjacentColumns}
    设命题~II:
    交换方阵 \(A\) 的\emph{相邻的}二列, 得方阵 \(B\),
    则 \(\det {(B)} = -\det {(A)}\).
    设命题~III:
    交换方阵 \(A\) 的二列, 得方阵 \(B\),
    则 \(\det {(B)} = -\det {(A)}\).
    则, 我们可由 II 推出 III.
\end{restatable}

\begin{proof}
    设 \(A\) 是 \(n\)~级阵,
    设交换 \(A\)~的列~\(p\), \(q\) 后得到的阵为 \(B\)
    (\(p < q\)).
    我们证明:
    \(\det {(B)} = -\det {(A)}\).

    我们交换 \(A\)~的列~\(p\), \(p+1\), 得阵~\(B_1\).
    则 \(\det {(B_1)} = -\det {(A)}
    (-1)^{1} \det {(A)}\).
    我们交换 \(B_1\)~的列~\(p+1\), \(p+2\), 得阵~\(B_2\).
    则 \(\det {(B_2)} = -\det {(B_1)}
    = (-1)^{2} \det {(A)}\).
    \(\dots \dots\)
    我们交换 \(B_{q-p-1}\)~的列~\(q-1\), \(q\), 得阵~\(B_{q-p}\).
    则 \(\det {(B_{q-p})} = -\det {(B_{q-p-1})}
    = (-1)^{q-p} \det {(A)}\).

    记 \(C_0 = B_{q-p}\).
    则 \(\det {(C_0)} = (-1)^{q-p} \det {(A)}\).
    我们交换 \(C_0\)~的列~\(q-2\), \(q-1\), 得阵~\(C_1\).
    则 \(\det {(C_1)} = -\det {(C_0)}
    = (-1)^{q-p+1} \det {(A)}\).
    我们交换 \(C_1\)~的列~\(q-3\), \(q-2\), 得阵~\(C_2\).
    则 \(\det {(C_2)} = -\det {(C_1)}
    = (-1)^{q-p+2} \det {(A)}\).
    \(\dots \dots\)
    我们交换 \(C_{q-p-2}\)~的列~\(p\), \(p+1\),
    得阵~\(C_{q-p-1}\).
    则 \(\det {(C_{q-p-1})} = -\det {(C_{q-p-2})}
    = (-1)^{q-p+(q-p-1)} \det {(A)}\).
    不难看出, \(B = C_{q-p-1}\).
    所以,
    \(\det {(B)}
    = (-1)^{2(q-p)-1} \det {(A)}
    = -\det {(A)}\).
\end{proof}

好的.
现在, 我介绍交错性的其他的证明.

\begin{proof}[同时用按列~\(1\) 展开与按前二列展开]
    \(P(1)\) 不证自明.

    不难验证 \(P(2)\) 是正确的.

    现在, 我们假定 \(P(m-1)\) 是正确的 (\(m \geq 3\)).
    我们要证 \(P(m)\) 也是正确的.

    任取一个 \(m\)~级阵 \(A\).

    设 \(A\) 的列~\(1\), \(2\) 相同.
    则
    \begin{align*}
        \det {(A)}
        = {} &
        \sum_{1 \leq i < k \leq m}
        {\det {
                \begin{bmatrix}
                    [A]_{i,1} & [A]_{i,2} \\
                    [A]_{k,1} & [A]_{k,2} \\
                \end{bmatrix}
            }
            (-1)^{i+k+1+2}
            \det {(A(i,k|1,2))}}
        \\
        = {} &
        \sum_{1 \leq i < k \leq m}
        {\det {
                \begin{bmatrix}
                    [A]_{i,1} & [A]_{i,1} \\
                    [A]_{k,1} & [A]_{k,1} \\
                \end{bmatrix}
            }
            (-1)^{i+k+1+2}
            \det {(A(i,k|1,2))}}
        \\
        = {} &
        \sum_{1 \leq i < k \leq m}
        {0\, (-1)^{i+k+1+2} \det {(A(i,k|1,2))}}
        \\
        = {} & 0.
    \end{align*}

    设 \(A\) 的列~\(j\), \(j+1\) 相同 (\(1 < j < m\)).
    则
    \begin{align*}
        \det {(A)}
        = \sum_{i = 1}^{m} {(-1)^{i + 1} [A]_{i,1}
        \det {(A(i|1))}}.
    \end{align*}
    注意, \(A(i|1)\) 的列~\(j-1\), \(j\) 相同.
    由假定, \(\det {(A(i|1))} = 0\).
    故 \(\det {(A)} = 0\).

    综上,
    若 \(A\) 的\emph{相邻的}二列相同,
    则其行列式为零.

    现假定 \(A\) 的列 \(p\), \(q\) 相同
    (\(1 \leq p < q \leq m\)).
    若 \(q - p = 1\), 则这是相邻的二列,
    故 \(\det {(A)} = 0\).
    若 \(q - p > 1\), 则交换列 \(p\), \(q-1\),
    得阵~\(B\).
    \(B\) 有相邻的二列相同, 故 \(\det {(B)} = 0\).
    从而, 由 ``有用的事实'',
    \(\det {(A)} = -\det {(B)} = 0\).

    所以, \(P(m)\) 是正确的.
    由数学归纳法原理, 待证命题成立.
\end{proof}

注意到, 因为在按列~\(1\) 展开行列式的公式里,
列~\(1\) 与其他列的地位不一样,
用它证明一个关于二列的性质%
对列~\(1\) 与其他列成立是有些挑战的.
不过, 若我们用按行~\(1\) 展开行列式的公式,
则这是较简单的.

\begin{proof}[用按行~\(1\) 展开]
    \(P(1)\) 不证自明.

    不难验证 \(P(2)\) 是正确的.

    现在, 我们假定 \(P(m-1)\) 是正确的 (\(m \geq 3\)).
    我们要证 \(P(m)\) 也是正确的.

    设 \(A\) 的列~\(j\), \(j+1\) 相同 (\(1 \leq j < m\)).

    注意, \([A]_{1,j} = [A]_{1,j+1}\),
    且 \(A(1|j) = A(1|j+1)\).
    再注意, 若 \(k \neq j\), \(j+1\),
    则 \(A(1|k)\) 有 (相邻的) 二列相同.
    由假定, \(\det {(A(1|k))} = 0\).
    从而
    \begin{align*}
             & \det {(A)}
        \\
        = {} &
        \sum_{k = 1}^{m}
        {(-1)^{1 + k} [A]_{1,k} \det {(A(1|k))}}
        \\
        = {} &
        \hphantom{{} + {}}
        (-1)^{1 + j} [A]_{1,j} \det {(A(1|j))}
        + (-1)^{1 + j+1} [A]_{1,j+1} \det {(A(1|j+1))}
        \\
             & +
        \sum_{\substack{1 \leq k \leq m \\k \neq j, j+1}}
        {(-1)^{1 + k} [A]_{1,k} \det {(A(1|k))}}
        \\
        = {} &
        \hphantom{{} + {}}
        (-1)^{1 + j} [A]_{1,j} \det {(A(1|j))}
        + (-1)^{1 + j+1} [A]_{1,j} \det {(A(1|j))}
        \\
             & +
        \sum_{\substack{1 \leq k \leq m \\k \neq j, j+1}}
        {(-1)^{1 + k} [A]_{1,k}\,0 }
        \\
        = {} &
        (-1)^{1 + j} [A]_{1,j} \det {(A(1|j))}
        - (-1)^{1 + j} [A]_{1,j} \det {(A(1|j))}
        \\
        = {} & 0.
    \end{align*}

    综上,
    若 \(A\) 的\emph{相邻的}二列相同,
    则其行列式为零.

    现假定 \(A\) 的列 \(p\), \(q\) 相同
    (\(1 \leq p < q \leq m\)).
    若 \(q - p = 1\), 则这是相邻的二列,
    故 \(\det {(A)} = 0\).
    若 \(q - p > 1\), 则交换列 \(p\), \(q-1\),
    得阵~\(B\).
    \(B\) 有相邻的二列相同, 故 \(\det {(B)} = 0\).
    从而, 由 ``有用的事实'',
    \(\det {(A)} = -\det {(B)} = 0\).

    所以, \(P(m)\) 是正确的.
    由数学归纳法原理, 待证命题成立.
\end{proof}

\section{(关于列的) 反称性}

本节, 我想证明 (关于列的) 反称性.

其实, 在第一章, 节~\sekcio{13},
我已用多线性与交错性证明了它.
不过, 我想, 知道别的证明也好.

% 我先引用定理:

% \begin{theorem}[反称性]
%     行列式 (关于列) 是反称性的.
%     具体地,
%     设 \(A\) 是 \(n\)~级阵,
%     设交换 \(A\)~的列~\(p\) 与列~\(q\) 后得到的阵为 \(B\)
%     (\(p < q\)).
%     则 \(\det {(B)} = -\det {(A)}\).
%     (通俗地, 交换方阵的二列, 则其行列式变号.)
% \end{theorem}

以下, 设 \(P(n)\) 为命题
\begin{quotation}
    对任何 \(n\)~级阵 \(A\),
    对任何不超过 \(n\) 且高于 \(1\) 的整数 \(q\),
    对任何低于 \(q\) 的正整数 \(p\),
    必 \(\det {(B)} = -\det {(A)}\),
    其中 \(B\) 是交换 \(A\)~的%
    列~\(p\), \(q\) 后得到的阵.
\end{quotation}
则, 我们的目标是:
对任何正整数 \(n\), \(P(n)\) 是正确的.

\begin{proof}[用按一列展开]
    \(P(1)\) 不证自明.

    不难验证 \(P(2)\) 是正确的.

    现在, 我们假定 \(P(m-1)\) 是正确的 (\(m \geq 3\)).
    我们要证 \(P(m)\) 也是正确的.

    任取一个 \(m\)~级阵 \(A\).
    任取一个不超过 \(m\) 且高于 \(1\) 的整数 \(q\).
    任取一个低于 \(q\) 的正整数 \(p\).
    交换 \(A\)~的列~\(p\), \(q\), 得 \(B\).
    在 \(1\), \(2\), \(\dots\), \(m\) 这 \(m\)~个数里,
    我们必定能找到一个数 \(j\),
    它既不等于 \(p\), 也不等于 \(q\).
    则
    \begin{align*}
        \det {(B)}
        = \sum_{i = 1}^{m}
        {(-1)^{i + j} [B]_{i,j} \det {(B(i|j))}}.
    \end{align*}
    注意, \([B]_{i,j} = [A]_{i,j}\).
    再注意, \(B(i|j)\) 可被认为是%
    交换 \(A(i|j)\) 的二列所得到的 \(m-1\)~级阵.
    由假定, \(\det {(B(i|j))} = -\det {(A(i|j))}\).
    从而
    \begin{align*}
        \det {(B)}
        = {} &
        \sum_{i = 1}^{m}
        {(-1)^{i + j} [A]_{i,j} (-\det {(A(i|j))})}
        \\
        = {} &
        {-\sum_{i = 1}^{m}
        {(-1)^{i + j} [A]_{i,j} \det {(A(i|j))}}}
        \\
        = {} &
        {-\det {(A)}}.
    \end{align*}

    所以, \(P(m)\) 是正确的.
    由数学归纳法原理, 待证命题成立.
\end{proof}

在介绍其他的证明前, 我要介绍一个有用的事实.

\TheoremSwapTwoAdjacentColumns*

\begin{proof}
    略.
\end{proof}

好的.
现在, 我介绍反称性的其他的证明.

\begin{proof}[同时用按列~\(1\) 展开与按前二列展开]
    \(P(1)\) 不证自明.

    不难验证 \(P(2)\) 是正确的.

    现在, 我们假定 \(P(m-1)\) 是正确的 (\(m \geq 3\)).
    我们要证 \(P(m)\) 也是正确的.

    任取一个 \(m\)~级阵 \(A\).
    任取一个低于 \(m\) 的正整数 \(j\).
    交换 \(A\)~的列~\(j\), \(j+1\), 得 \(B\).

    设 \(j = 1\).
    注意到, \(i \neq k\) 时,
    \(A(i,k|1,2) = B(i,k|1,2)\).
    则
    \begin{align*}
        \det {(B)}
        = {} &
        \sum_{1 \leq i < k \leq m}
        {\det {
                \begin{bmatrix}
                    [B]_{i,1} & [B]_{i,2} \\
                    [B]_{k,1} & [B]_{k,2} \\
                \end{bmatrix}
            }
            (-1)^{i+k+1+2}
            \det {(B(i,k|1,2))}}
        \\
        = {} &
        \sum_{1 \leq i < k \leq m}
        {\det {
                \begin{bmatrix}
                    [A]_{i,2} & [A]_{i,1} \\
                    [A]_{k,2} & [A]_{k,1} \\
                \end{bmatrix}
            }
            (-1)^{i+k+1+2}
            \det {(A(i,k|1,2))}}
        \\
        = {} &
        \sum_{1 \leq i < k \leq m}
        {\left(-\det {
                \begin{bmatrix}
                    [A]_{i,1} & [A]_{i,2} \\
                    [A]_{k,1} & [A]_{k,2} \\
                \end{bmatrix}
            }\right)
            (-1)^{i+k+1+2}
            \det {(A(i,k|1,2))}}
        \\
        = {} & {-\det {(A)}}.
    \end{align*}

    设 \(j > 1\).
    则
    \begin{align*}
        \det {(B)}
        = \sum_{i = 1}^{m}
        {(-1)^{i + 1} [B]_{i,1} \det {(B(i|1))}}.
    \end{align*}
    注意, \([B]_{i,1} = [A]_{i,1}\).
    再注意, \(B(i|1)\) 可被认为是%
    交换 \(A(i|1)\) 的二列所得到的 \(m-1\)~级阵.
    由假定, \(\det {(B(i|1))} = -\det {(A(i|1))}\).
    从而
    \begin{align*}
        \det {(B)}
        = {} &
        \sum_{i = 1}^{m}
        {(-1)^{i + 1} [A]_{i,1} (-\det {(A(i|1))})}
        \\
        = {} &
        {-\sum_{i = 1}^{m}
        {(-1)^{i + 1} [A]_{i,1} \det {(A(i|1))}}}
        \\
        = {} &
        {-\det {(A)}}.
    \end{align*}

    综上,
    交换方阵的\emph{相邻的}二列,
    则其行列式变号.

    由 ``有用的事实'',
    交换方阵的二列,
    则其行列式变号.

    所以, \(P(m)\) 是正确的.
    由数学归纳法原理, 待证命题成立.
\end{proof}

注意到, 因为在按列~\(1\) 展开行列式的公式里,
列~\(1\) 与其他列的地位不一样,
用它证明一个关于二列的性质%
对列~\(1\) 与其他列成立是有些挑战的.
不过, 若我们用按行~\(1\) 展开行列式的公式,
则这是较简单的.

\begin{proof}[用按行~\(1\) 展开]
    \(P(1)\) 不证自明.

    不难验证 \(P(2)\) 是正确的.

    现在, 我们假定 \(P(m-1)\) 是正确的 (\(m \geq 3\)).
    我们要证 \(P(m)\) 也是正确的.

    任取一个 \(m\)~级阵 \(A\).
    任取一个低于 \(m\) 的正整数 \(j\).
    交换 \(A\)~的列~\(j\), \(j+1\), 得 \(B\).

    注意,
    \([A]_{i,j} = [B]_{i,j+1}\),
    \([A]_{i,j+1} = [B]_{i,j}\).
    再注意, \(k \neq j\), \(j+1\) 时,
    \([A]_{i,k} = [B]_{i,k}\).
    则 \(A(i|j) = B(i|j+1)\),
    \(A(i|j+1) = B(i|j)\).
    若 \(k \neq j\), \(j+1\),
    则 \(B(i|k)\) 可被认为是%
    交换 \(A(i|k)\) 的 (相邻的) 二列%
    所得到的 \(m-1\)~级阵.
    由假定,
    \(\det {(B(i|k))} = -\det {(A(i|k))}\).
    从而
    \begin{align*}
             & \det {(B)}
        \\
        = {} &
        \sum_{k = 1}^{m}
        {(-1)^{1 + k} [B]_{1,k} \det {(B(1|k))}}
        \\
        = {} &
        \hphantom{{} + {}}
        (-1)^{1 + j} [B]_{1,j} \det {(B(1|j))}
        + (-1)^{1 + j+1} [B]_{1,j+1} \det {(B(1|j+1))}
        \\
             & +
        \sum_{\substack{1 \leq k \leq m \\k \neq j, j+1}}
        {(-1)^{1 + k} [B]_{1,k} \det {(B(1|k))}}
        \\
        = {} &
        \hphantom{{} + {}}
        (-1)^{1 + j} [A]_{1,j+1} \det {(A(1|j+1))}
        + (-1)^{1 + j+1} [A]_{1,j} \det {(A(1|j))}
        \\
             & +
        \sum_{\substack{1 \leq k \leq m \\k \neq j, j+1}}
        {(-1)^{1 + k} [A]_{1,k} (-\det {(A(1|k))})}
        \\
        = {} &
        \hphantom{{} + {}}
        {-(-1)^{1 + j+1}} [A]_{1,j+1} \det {(A(1|j+1))}
        - (-1)^{1 + j} [A]_{1,j} \det {(A(1|j))}
        \\
             & -
        \sum_{\substack{1 \leq k \leq m \\k \neq j, j+1}}
        {(-1)^{1 + k} [A]_{1,k} \det {(A(1|k))}}
        \\
        = {} &
        {-\sum_{k = 1}^{m}
        {(-1)^{1 + k} [A]_{1,k} \det {(A(1|k))}}}
        \\
        = {} & {-\det {(A)}}.
    \end{align*}

    综上,
    交换方阵的\emph{相邻的}二列,
    则其行列式变号.

    由 ``有用的事实'',
    交换方阵的二列,
    则其行列式变号.

    所以, \(P(m)\) 是正确的.
    由数学归纳法原理, 待证命题成立.
\end{proof}

\section{反称性的应用}

反称性是有用的.

由行列式的定义, 不难证明,
对任何 \(n-1\)~个 \(n \times 1\)~阵
\(b_2\), \(\dots\), \(b_n\),
任何二个 \(n \times 1\)~阵 \(x\), \(y\),
任何二个数 \(s\), \(t\),
有
\begin{align*}
    \det
    {[sx + ty, b_2, \dots, b_n]}
    = s
    \det {[x, b_2, \dots, b_n]}
    +
    t
    \det {[y, b_2, \dots, b_n]}.
\end{align*}
利用反称性, 当 \(j > 1\) 时,
\begin{align*}
         & \det
    {[a_1, \dots, a_{j-1}, sx + ty, a_{j+1}, \dots, a_n]}
    \\
    = {} & {-\det
            {[sx + ty, \dots, a_{j-1}, a_1, a_{j+1}, \dots, a_n]}}
    \\
    = {} & {-
            (s
            \det {[x, \dots, a_{j-1}, a_1, a_{j+1}, \dots, a_n]}
            +
            t
            \det {[y, \dots, a_{j-1}, a_1, a_{j+1}, \dots, a_n]}
            )
        }
    \\
    = {} &
    s(-\det {[x, \dots, a_{j-1}, a_1, a_{j+1}, \dots, a_n]})
    + t(-\det {[y, \dots, a_{j-1}, a_1, a_{j+1}, \dots, a_n]})
    \\
    = {} &
    s
    \det {[a_1, \dots, a_{j-1}, x, a_{j+1}, \dots, a_n]}
    +
    t
    \det {[a_1, \dots, a_{j-1}, y, a_{j+1}, \dots, a_n]}.
\end{align*}

我们也可由反称性推出交错性.
设方阵 \(A\)~的列~\(p\), \(q\) 是相同的 (\(p < q\)).
我们交换 \(A\)~的列 \(p\), \(q\), 得阵 \(B\).
根据反称性, \(\det {(B)} = -\det {(A)}\).
不过, 因为 \(A\)~的列~\(p\), \(q\) 是相同的,
故 \(B = A\).
所以, \(\det {(A)} = -\det {(A)}\).
由此可知 \(\det {(A)} = 0\).

我们可由反称性得到%
按 (任何) 一列展开行列式的公式.
设 \(A\) 是一个 \(n\)~级阵 (\(n \geq 2\)).
设 \(j \neq 1\).
交换 \(A\) 的列~\(j-1\), \(j\), 得阵~\(B_1\).
交换 \(B_1\) 的列~\(j-2\), \(j-1\), 得阵~\(B_2\).
\(\dots \dots\)
交换 \(B_{j-2}\) 的列~\(1\), \(2\), 得阵~\(B_{j-1}\).
此时, 可以发现,
\(B_{j-1}\)~的列~\(1\) 即为 \(A\)~的列~\(j\)
(即 \([B_{j-1}]_{i,1} = [A]_{i,j}\)),
且 \(B_{j-1}(i|1) = A(i|j)\).
我们作了 \(j-1\)~次 (相邻的) 列的交换,
故
\begin{align*}
    \det {(A)}
    = {} & (-1)^{j-1} \det {(B_{j-1})}
    \\
    = {} & (-1)^{j-1} \sum_{i = 1}^{n}
    {(-1)^{i+1} [B]_{i,1} \det {(B_{j-1}(i|1))}}
    \\
    = {} & \sum_{i = 1}^{n}
    {(-1)^{j-1} (-1)^{i+1} [A]_{i,1} \det {(A(i|j))}}
    \\
    = {} & \sum_{i = 1}^{n}
    {(-1)^{i+j} [A]_{i,j} \det {(A(i|j))}}.
\end{align*}

\section{用行列式的性质确定行列式}

本节, 我想讨论如何用行列式的性质确定行列式.

我们知道, 多线性与交错性可 ``基本确定'' 行列式:

\begin{theorem}
    设定义在全体 \(n\)~级阵上的函数 \(f\) 适合:

    (1)
    (多线性)
    对任何不超过 \(n\) 的正整数 \(j\),
    任何 \(n-1\)~个 \(n \times 1\)~阵
    \(a_1\), \(\dots\), \(a_{j-1}\),
    \(a_{j+1}\), \(\dots\), \(a_n\),
    任何二个 \(n \times 1\)~阵 \(x\), \(y\),
    任何二个数 \(s\), \(t\),
    有
    \begin{align*}
             & f
            {([a_1, \dots, a_{j-1}, sx + ty,
                        a_{j+1}, \dots, a_n])}
        \\
        = {} &
        s
        f {([a_1, \dots, a_{j-1}, x, a_{j+1}, \dots, a_n])}
        +
        t
        f {([a_1, \dots, a_{j-1}, y, a_{j+1}, \dots, a_n])}.
    \end{align*}

    (2)
    (交错性)
    若 \(n\)~级阵 \(A\) 有二列完全相同,
    则 \(f(A) = 0\).

    那么, 对任何 \(n\)~级阵 \(A\),
    \(f(A) = f(I) \det {(A)}\).

    特别地, 若 \(f(I) = 1\) (规范性),
    则 \(f\) 就是行列式.
\end{theorem}

现在, 我要展示一些变体.

\begin{theorem}
    设定义在全体 \(n\)~级阵上的函数 \(f\) 适合:

    (1)
    对任何 \(n-1\)~个 \(n \times 1\)~阵
    \(a_2\), \(a_3\), \(\dots\), \(a_n\),
    任何二个 \(n \times 1\)~阵 \(x\), \(y\),
    任何二个数 \(s\), \(t\),
    有
    \begin{align*}
        f {([sx + ty, a_2, \dots, a_n])}
        =
        s
        f {([x, a_2, \dots, a_n])}
        +
        t
        f {([y, a_2, \dots, a_n])}.
    \end{align*}

    (2)
    (反称性)
    设 \(A\) 是 \(n\)~级阵,
    设交换 \(A\)~的列~\(p\) 与列~\(q\) 后得到的阵为 \(B\)
    (\(p < q\)).
    则 \(f(B) = -f(A)\).

    那么, 对任何 \(n\)~级阵 \(A\),
    \(f(A) = f(I) \det {(A)}\).

    特别地, 若 \(f(I) = 1\) (规范性),
    则 \(f\) 就是行列式.
\end{theorem}

\begin{proof}
    见上节的讨论.
    我们可由 (1) 与反称性,
    得到多线性;
    我们可由反称性,
    得到交错性.
\end{proof}

我们也可代反称性以 ``相邻反称性'':
``设 \(A\) 是 \(n\)~级阵,
设交换 \(A\)~的列~\(p\) 与列~\(p+1\) 后得到的阵为 \(B\)
(\(p < n\)).
则 \(f(B) = -f(A)\).''
毕竟, 相邻反称性可推出反称性.

我们也可代交错性以 ``相邻交错性'':
``若 \(n\)~级阵 \(A\) 有相邻的二列完全相同,
则 \(f(A) = 0\).''
毕竟, 多线性与相邻交错性可推出相邻反称性,
相邻反称性可推出反称性,
且相邻交错性与反称性可推出交错性
(见 ``(关于列的) 交错性'' 的讨论).

接着, 我要展现一个 ``大不一样的'' 变体.
不过, 我要先定义一种行为:

\begin{restatable}[倍加]{definition}{DefinitionMultiplyAndAdd}
    设 \(A\) 是一个 \(m \times n\)~阵.
    设 \(p\), \(q\) 是二个不超过 \(n\)~的正整数,
    \emph{且 \(p \neq q\).}
    设 \(s\) 是一个数.
    作 \(m \times n\)~阵 \(B\), 其中
    \begin{align*}
        [B]_{i,j}
        = \begin{cases}
              [A]_{i,j},              & j \neq q; \\
              [A]_{i,q} + s[A]_{i,p}, & j = q.
          \end{cases}
    \end{align*}
    (通俗地,
    我们加 \(A\)~的列~\(p\) 的 \(s\)~倍于列~\(q\),
    不改变其他的列,
    得阵~\(B\).)
    我们说, 变 \(A\) 为 \(B\) 的行为是一次 (列的) \emph{倍加}.
\end{restatable}

注意到, \(s\) 可以取 \(0\),
而这相当于 \(B = A\).
所以, 我们认为, ``什么都不变'' 也是一次倍加.

利用多线性与交错性, 我们有

\begin{theorem}
    设定义在全体 \(n\)~级阵上的函数 \(f\)
    多线性与交错性.
    则 \(f\) 适合 ``倍加不变性'':
    \begin{quotation}
        设 \(A\) 是一个 \(n\)~级阵.
        设 \(p\), \(q\) 是二个不超过 \(n\)~的正整数,
        \emph{且 \(p \neq q\).}
        设 \(s\) 是一个数.
        作 \(n\)~级阵 \(B\), 其中
        \begin{align*}
            [B]_{i,j}
            = \begin{cases}
                  [A]_{i,j},              & j \neq q; \\
                  [A]_{i,q} + s[A]_{i,p}, & j = q.
              \end{cases}
        \end{align*}
        则 \(f(B) = f(A)\).
    \end{quotation}
\end{theorem}

\begin{proof}
    设 \(A = [a_1, a_2, \dots, a_n]\).

    先设 \(p < q\).
    为方便说话, 我们写
    \begin{align*}
        g(u, v)
        = f {([a_1, \dots, a_{p-1}, u, a_{p+1}, \dots,
                        a_{q-1}, v, a_{q+1}, \dots, a_n])}.
    \end{align*}
    于是, \(g(a_p, a_q)\) 就是 \(f(A)\),
    而 \(g(a_p, a_q + sa_p)\) 就是 \(f(B)\).
    利用多线性与交错性,
    \begin{align*}
        f(B)
        = {} & g(a_p, a_q + sa_p)         \\
        = {} & g(a_p, a_q) + sg(a_p, a_p) \\
        = {} & g(a_p, a_q) + s0           \\
        = {} & g(a_p, a_q)                \\
        = {} & f(A).
    \end{align*}

    再设 \(p > q\).
    为方便说话, 我们写
    \begin{align*}
        h(u, v)
        = f {([a_1, \dots, a_{q-1}, u, a_{q+1}, \dots,
                        a_{p-1}, v, a_{p+1}, \dots, a_n])}.
    \end{align*}
    于是, \(h(a_q, a_p)\) 就是 \(f(A)\),
    而 \(h(a_q + sa_p, a_p)\) 就是 \(f(B)\).
    利用多线性与交错性,
    \begin{align*}
        f(B)
        = {} & h(a_q + sa_p, a_p)         \\
        = {} & h(a_q, a_p) + sh(a_p, a_p) \\
        = {} & h(a_q, a_p) + s0           \\
        = {} & h(a_q, a_p)                \\
        = {} & f(A).
        \qedhere
    \end{align*}
\end{proof}

现在, 我要引出本节的主要结论.

\begin{theorem}
    设定义在全体 \(n\)~级阵上的函数 \(f\) 适合:

    (1)
    倍加不变性.

    (2)
    (多齐性)
    对任何不超过 \(n\) 的正整数 \(j\),
    任何 \(n-1\)~个 \(n \times 1\)~阵
    \(a_1\), \(\dots\), \(a_{j-1}\),
    \(a_{j+1}\), \(\dots\), \(a_n\),
    任何 \(n \times 1\)~阵 \(x\),
    任何数 \(s\),
    有
    \begin{align*}
        f {([a_1, \dots, a_{j-1}, sx, a_{j+1}, \dots, a_n])}
        =
        s
        f {([a_1, \dots, a_{j-1}, x, a_{j+1}, \dots, a_n])}.
    \end{align*}

    那么, 对任何 \(n\)~级阵 \(A\),
    \(f(A) = f(I) \det {(A)}\).

    特别地, 若 \(f(I) = 1\) (规范性),
    则 \(f\) 就是行列式.
\end{theorem}

此事是重要的, 故我会给二个证明.

\vspace{2ex}

不难看出, 倍加不变性与多齐性可推出交错性.
具体地, 设 \(A\) 的列~\(p\), \(q\) 相同,
且 \(p \neq q\).
加 \(A\) 的列~\(q\) 的 \(-1\)~倍于列~\(p\),
得阵~\(B\).
那么, \(B\) 的列~\(p\) 的元全为零.
由多齐性, \(f(B) = 0\).
由倍加不变性, \(f(A) = f(B) = 0\).

倍加不变性与多齐性还可推出反称性.
设 \(p < q\).
记
\begin{align*}
    g(u, v)
    = f {([a_1, \dots, a_{p-1}, u, a_{p+1}, \dots,
                    a_{q-1}, v, a_{q+1}, \dots, a_n])}.
\end{align*}
则
\begin{align*}
    g(a_q, a_p)
    = {} & g(a_q + 1a_p, a_p)                   \\
    = {} & g(a_p + a_q, a_p)                    \\
    = {} & g(a_p + a_q, a_p + (-1) (a_p + a_q)) \\
    = {} & g(a_p + a_q, -a_q)                   \\
    = {} & g(a_p + a_q + 1 (-a_q), -a_q)        \\
    = {} & g(a_p, (-1) a_q)                     \\
    = {} & (-1) g(a_p, a_q)                     \\
    = {} & {-g(a_p, a_q)}.
\end{align*}

假如, 我们能推出,
对任何 \(n-1\)~个 \(n \times 1\)~阵
\(a_2\), \(a_3\), \(\dots\), \(a_n\),
任何二个 \(n \times 1\)~阵 \(x\), \(y\),
任何二个数 \(s\), \(t\),
有
\begin{align*}
    f {([sx + ty, a_2, \dots, a_n])}
    =
    s
    f {([x, a_2, \dots, a_n])}
    +
    t
    f {([y, a_2, \dots, a_n])},
\end{align*}
那么, 利用反称性, 我们即得多线性.

在第一章, 节~\malneprasekcio{27} 里, 有如下结论:

\begin{theorem}
    设 \(A\) 是 \(m \times n\)~阵,
    且 \(A \neq 0\).
    设 \(A\) 有一个行列式非零的 \(r\)~级子阵
    \begin{align*}
        A_r = A\binom{i_1,\dots,i_r}{j_1,\dots,j_r}
    \end{align*}
    (其中
    \(1 \leq i_1 < \dots < i_r \leq m\),
    \(1 \leq j_1 < \dots < j_r \leq n\)),
    但 \(A\) 没有行列式非零的 \(r+1\)~级子阵.
    设 \(A\)~的%
    行~\(1\), \(2\), \(\dots\), \(m\)
    为 \(a_1\), \(a_2\), \(\dots\), \(a_m\).
    那么, 对任何不超过 \(m\) 的正整数 \(p\),
    一定存在 \(r\)~个数
    \(d_{p,1}\), \(d_{p,2}\), \(\dots\), \(d_{p,r}\),
    使
    \begin{align*}
        a_p
        = {} &
        d_{p,1} a_{i_1}
        + d_{p,2} a_{i_2}
        + \dots
        + d_{p,r} a_{i_r}
        \\
        = {} &
        \sum_{s = 1}^{r} {d_{p,s} a_{i_s}}.
    \end{align*}
\end{theorem}

利用类似的方法, 或利用转置, 我们可证

\begin{theorem}
    设 \(A\) 是 \(m \times n\)~阵,
    且 \(A \neq 0\).
    设 \(A\) 有一个行列式非零的 \(r\)~级子阵
    \begin{align*}
        A_r = A\binom{i_1,\dots,i_r}{j_1,\dots,j_r}
    \end{align*}
    (其中
    \(1 \leq i_1 < \dots < i_r \leq m\),
    \(1 \leq j_1 < \dots < j_r \leq n\)),
    但 \(A\) 没有行列式非零的 \(r+1\)~级子阵.
    设 \(A\)~的%
    列~\(1\), \(2\), \(\dots\), \(n\)
    为 \(a_1\), \(a_2\), \(\dots\), \(a_n\).
    那么, 对任何不超过 \(n\) 的正整数 \(q\),
    一定存在 \(r\)~个数
    \(d_{1,q}\), \(d_{2,q}\), \(\dots\), \(d_{r,q}\),
    使
    \begin{align*}
        a_q
        = {} &
        d_{1,q} a_{j_1}
        + d_{2,q} a_{j_2}
        + \dots
        + d_{r,q} a_{j_r}
        \\
        = {} &
        \sum_{s = 1}^{r} {d_{s,q} a_{j_s}}.
    \end{align*}
\end{theorem}

利用此事, 我们即可证明,
对任何 \(n-1\)~个 \(n \times 1\)~阵
\(a_2\), \(a_3\), \(\dots\), \(a_n\),
任何二个 \(n \times 1\)~阵 \(x\), \(y\),
任何二个数 \(s\), \(t\),
有
\begin{align*}
    f {([sx + ty, a_2, \dots, a_n])}
    =
    s
    f {([x, a_2, \dots, a_n])}
    +
    t
    f {([y, a_2, \dots, a_n])}.
\end{align*}
由此, 我们可证多线性.

\begin{proof}
    作 \(n \times (n-1)\)~阵
    \(B = [a_2, a_3, \dots, a_n]\);
    也就是说, \(B\) 的列~\(j-1\) 是 \(a_j\).

    若 \(B = 0\),
    由多齐性,
    \begin{align*}
        0 = f {([sx + ty, a_2, \dots, a_n])}
        = f {([x, a_2, \dots, a_n])}
        = f {([y, a_2, \dots, a_n])}.
    \end{align*}

    下设 \(B \neq 0\).
    那么, 存在一个低于 \(n\) 的正整数 \(r\),
    使 \(B\) 有一个行列式非零的 \(r\)~级子阵
    \begin{align*}
        B_r = B\binom{i_1,\dots,i_r}{j_1-1,\dots,j_r-1}
    \end{align*}
    (其中
    \(1 \leq i_1 < \dots < i_r \leq n\),
    \(2 \leq j_1 < \dots < j_r \leq n\)),
    但 \(B\) 没有行列式非零的 \(r+1\)~级子阵.

    若 \(r < n-1\), 那么,
    必存在\emph{不等于}
    \(j_1\), \(\dots\), \(j_r\) 的正整数 \(q\),
    与 \(r\)~个数
    \(d_{1,q}\), \(d_{2,q}\), \(\dots\), \(d_{r,q}\),
    使
    \begin{align*}
        a_q
        = {} &
        d_{1,q} a_{j_1}
        + d_{2,q} a_{j_2}
        + \dots
        + d_{r,q} a_{j_r}.
    \end{align*}
    任取 \(n \times 1\) 阵 \(z\).
    记
    \begin{align*}
        g(u)
        = f ([z, a_2, \dots, a_{q-1}, u, a_{q+1}, \dots, a_n]).
    \end{align*}
    利用倍加不变性,
    \begin{align*}
        g(a_q)
        = {} & g(a_q + (-d_{1,q})a_{j_1}) \\
        = {} & \dots \dots \dots \dots    \\
        = {} & g(a_q + (-d_{1,q})a_{j_1}
        + \dots + (-d_{r,q})a_{j_r})      \\
        = {} & g(a_q - (d_{1,q} a_{j_1}
        + \dots + d_{r,q} a_{j_r}))       \\
        = {} & g(0).
    \end{align*}
    利用多齐性, \(g(0) = 0\).
    故 \(g(a_q) = 0\).
    所以,
    \begin{align*}
        0 = f {([sx + ty, a_2, \dots, a_n])}
        = f {([x, a_2, \dots, a_n])}
        = f {([y, a_2, \dots, a_n])}.
    \end{align*}

    下设 \(r = n-1\).

    设从 \(1\), \(2\), \(\dots\), \(n\)
    去除 \(i_1\), \(i_2\), \(\dots\), \(i_{n-1}\) 后,
    还剩一个数 \(i_n\).
    设 \(a_1\) 是 \(n\)~级单位阵的列~\(i_n\).
    作 \(n\)~级阵
    \(A = [a_1, a_2, \dots, a_n]\).
    则 \(A(i_n | 1) = B_r\).
    不难算出,
    \(\det {(A)} = (-1)^{i_n + 1} \det {(B_r)} \neq 0\).

    作 \(n \times (n+2)\)~阵
    \(C = [a_1, a_2, \dots, a_n, x, y]\).
    于是, \(C\) 有一个行列式非零的 \(n\)~级子阵 \(A\),
    但 \(C\) 没有行列式非零的 \(n+1\)~级子阵.
    所以,
    一定存在 \(n\)~个数
    \(d_1\), \(d_2\), \(\dots\), \(d_n\),
    使
    \(x = d_1 a_1 + d_2 a_2 + \dots + d_n a_n\),
    也一定存在 \(n\)~个数
    \(d'_1\), \(d'_2\), \(\dots\), \(d'_n\),
    使
    \(y = d'_1 a_1 + d'_2 a_2 + \dots + d'_n a_n\).
    所以,
    \(sx + ty
    = (sd_1 + td'_1)a_1
    + (sd_2 + td'_2) a_2
    + \dots
    + (sd_n + td'_n) a_n\).

    记 \(h(z) = f ([z, a_2, \dots, a_n])\).
    则
    \begin{align*}
        f ([x, a_2, \dots, a_n])
        = {} & h(x)
        \\
        = {} & h(x - d_2 a_2)
        \\
        = {} & \dots \dots \dots \dots
        \\
        = {} & h(x - d_2 a_2 - \dots - d_n a_n)
        \\
        = {} & h(d_1 a_1)
        \\
        = {} & d_1 h(a_1).
    \end{align*}
    同理,
    \begin{align*}
         & f ([y, a_2, \dots, a_n]) = d'_1 h(a_1),
        \\
         & f ([sx + ty, a_2, \dots, a_n]) = (sd_1 + td'_1) h(a_1).
    \end{align*}
    比较, 得
    \begin{align*}
             &
        f {([sx + ty, a_2, \dots, a_n])}
        \\
        = {} &
        s f {([x, a_2, \dots, a_n])}
        +
        t f {([y, a_2, \dots, a_n])}.
        \qedhere
    \end{align*}
\end{proof}

接下来, 我展现另一个证明.
此证明或许更有意思.

\begin{restatable}{theorem}{TheoremMultiplyAndAdd}
    设 \(A\) 是一个 \(m \times n\)~阵.
    利用若干次 (列的) 倍加,
    我们可变 \(A\) 为一个 \(m \times n\)~阵 \(B\),
    使当 \(i < j\) 时,
    \([B]_{i,j} = 0\).
\end{restatable}

\begin{proof}
    我们用数学归纳法证明此事.
    具体地, 设 \(P(n)\) 为命题
    \begin{quotation}
        对\emph{任何}正整数 \(m\),
        对\emph{任何} \(m \times n\)~阵,
        利用若干次倍加 (指 ``列的倍加'', 下同),
        我们可变 \(A\) 为一个 \(m \times n\)~阵 \(B\),
        使当 \(i < j\) 时,
        \([B]_{i,j} = 0\).
    \end{quotation}
    则, 我们的目标是:
    对任何正整数 \(n\), \(P(n)\) 是正确的.

    \(P(1)\) 显然是对的.

    假定 \(P(n-1)\) 是对的.
    我们由此证 \(P(n)\) 也是对的.

    任取正整数 \(m\).
    任取一个 \(m \times n\)~阵 \(A\).

    我们先说明, 利用若干次倍加,
    我们可变 \(A\) 为一个 \(m \times n\)~阵 \(C\),
    使当 \(1 < j\) 时, \([C]_{1,j} = 0\).

    若 \(A\) 的行~\(1\) 的元全是零,
    我们 ``什么都不变'',
    取 \(C\) 为 \(A\).

    若 \([A]_{1,1} \neq 0\),
    我们可加 \(A\)~的列~\(1\)~的
    \(-[A]_{1,2} / [A]_{1,1}\)~倍于列~\(2\),
    得阵 \(A_2\).
    那么, \([A_2]_{1,2} = 0\),
    且 \([A_2]_{1,j} = [A]_{1,1}\)
    (\(j \neq 2\)).
    接着, 我们可加 \(A_2\)~的列~\(1\)~的
    \(-[A_2]_{1,3} / [A_2]_{1,1}\)~倍于列~\(3\),
    得阵 \(A_3\).
    那么, \([A_3]_{1,k} = 0\)
    (\(k = 2\), \(3\)),
    且 \([A_3]_{1,j} = [A_2]_{1,j}\)
    % (\(j \neq 2\), \(3\)).
    (\(j \neq 3\)).
    \(\dots \dots\)
    接着, 我们可加 \(A_{n-1}\)~的列~\(1\)~的
    \(-[A_{n-1}]_{1,n} / [A_{n-1}]_{1,1}\)~倍于列~\(n\),
    得阵 \(A_n\).
    那么, \([A_n]_{1,k} = 0\)
    (\(k = 2\), \(3\), \(\dots\), \(n\)).
    我们取 \(C\) 为 \(A_n\).

    若 \([A]_{1,1} = 0\),
    但有某 \([A]_{1,j} \neq 0\)
    (\(j \neq 1\)),
    我们加 \(A\)~的列~\(j\)~的 \(1\)~倍于列~\(1\),
    得阵 \(D\).
    那么, \([D]_{1,1} \neq 0\),
    这就转化问题为前面讨论过的情形.

    综上, 作若干次倍加,
    我们可变 \(A\) 为一个 \(m \times n\)~阵 \(C\),
    使当 \(1 < j\) 时, \([C]_{1,j} = 0\).

    考虑 \(C\) 的右下角的 \((m-1) \times (n-1)\)~子阵
    \(C(1|1)\).
    由假定, 作若干次倍加, 我们可变 \(C(1|1)\) 为一个
    \((m-1) \times (n-1)\)~阵 \(G\),
    使当 \(i < j\) 时, \([G]_{i,j} = 0\).

    注意到, 既然当 \(1 < j\) 时, \([C]_{1,j} = 0\),
    那么, 无论如何对 \(C\) 的不是列~\(1\) 的列作倍加,
    所得的阵的 \((1, j)\)-元一定是零.
    那么, 作若干次倍加后,
    我们可变 \(C\) 为一个 \(m \times n\)~阵 \(B\),
    使当 \(i\) 或 \(j\) 为 \(1\) 时,
    \([B]_{i,j} = [C]_{i,j}\),
    且 \(i\) 与 \(j\) 不为 \(1\) 时,
    \([B]_{i,j} = [G]_{i-1,j-1}\).
    所以, 当 \(i < j\) 时, \([B]_{i,j} = 0\).

    所以, \(P(n)\) 是正确的.
    由数学归纳法原理, 待证命题成立.
\end{proof}

\begin{theorem}
    设定义在全体 \(n\)~级阵上的函数 \(f\)
    适合倍加不变性与多齐性.
    设 \(A\) 是一个 \(n\)~级阵,
    且当 \(i < j\) 时, \([A]_{i,j} = 0\).
    则 \(f(A) = f(I_n) \det {(A)}\).
\end{theorem}

\begin{proof}
    我们用数学归纳法证明此事.
    具体地, 设 \(P(n)\) 为命题
    \begin{quotation}
        设定义在全体 \(n\)~级阵上的函数 \(f\)
        适合倍加不变性与多齐性.
        设 \(A\) 是一个 \(n\)~级阵,
        且当 \(i < j\) 时, \([A]_{i,j} = 0\).
        则 \(f(A) = f(I_n) \det {(A)}\).
    \end{quotation}
    则, 我们的目标是:
    对任何正整数 \(n\), \(P(n)\) 是正确的.

    \(P(1)\) 显然是对的.

    假定 \(P(n-1)\) 是对的.
    我们由此证 \(P(n)\) 也是对的.

    任取一个 \(n\)~级阵 \(A\),
    且当 \(i < j\) 时, \([A]_{i,j} = 0\).
    所以, \(A\) 形如
    \begin{align*}
        \begin{bmatrix}
            [A]_{1,1}   & 0           & \cdots & 0             & 0         \\
            [A]_{2,1}   & [A]_{2,2}   & \cdots & 0             & 0         \\
            \vdots      & \vdots      & \ddots & \vdots        & \vdots    \\
            [A]_{n-1,1} & [A]_{n-1,2} & \cdots & [A]_{n-1,n-1} & 0         \\
            [A]_{n,1}   & [A]_{n,2}   & \cdots & [A]_{n,n-1}   & [A]_{n,n} \\
        \end{bmatrix}
    \end{align*}
    那么, 由多齐性,
    \begin{align*}
        f(A) = [A]_{n,n}
        f\left(
        \begin{bmatrix}
                [A]_{1,1}   & 0           & \cdots & 0             & 0      \\
                [A]_{2,1}   & [A]_{2,2}   & \cdots & 0             & 0      \\
                \vdots      & \vdots      & \ddots & \vdots        & \vdots \\
                [A]_{n-1,1} & [A]_{n-1,2} & \cdots & [A]_{n-1,n-1} & 0      \\
                [A]_{n,1}   & [A]_{n,2}   & \cdots & [A]_{n,n-1}   & 1      \\
            \end{bmatrix}
        \right).
    \end{align*}
    利用 \(n-1\)~次倍加不变性,
    \begin{align*}
             &
        f\left(
        \begin{bmatrix}
                [A]_{1,1}   & 0           & \cdots & 0             & 0      \\
                [A]_{2,1}   & [A]_{2,2}   & \cdots & 0             & 0      \\
                \vdots      & \vdots      & \ddots & \vdots        & \vdots \\
                [A]_{n-1,1} & [A]_{n-1,2} & \cdots & [A]_{n-1,n-1} & 0      \\
                [A]_{n,1}   & [A]_{n,2}   & \cdots & [A]_{n,n-1}   & 1      \\
            \end{bmatrix}
        \right)
        \\
        = {} &
        f\left(
        \begin{bmatrix}
                [A]_{1,1}   & 0           & \cdots & 0             & 0      \\
                [A]_{2,1}   & [A]_{2,2}   & \cdots & 0             & 0      \\
                \vdots      & \vdots      & \ddots & \vdots        & \vdots \\
                [A]_{n-1,1} & [A]_{n-1,2} & \cdots & [A]_{n-1,n-1} & 0      \\
                0           & 0           & \cdots & 0             & 1      \\
            \end{bmatrix}
        \right),
    \end{align*}
    故
    \begin{align*}
        f(A) =
        f\left(
        \begin{bmatrix}
            [A]_{1,1}   & 0           & \cdots & 0             & 0      \\
            [A]_{2,1}   & [A]_{2,2}   & \cdots & 0             & 0      \\
            \vdots      & \vdots      & \ddots & \vdots        & \vdots \\
            [A]_{n-1,1} & [A]_{n-1,2} & \cdots & [A]_{n-1,n-1} & 0      \\
            0           & 0           & \cdots & 0             & 1      \\
        \end{bmatrix}
        \right)
        [A]_{n,n}.
    \end{align*}

    考虑定义在全体 \(n-1\)~级阵上的函数
    \begin{align*}
        g(X)
        =
        f\left(
        \begin{bmatrix}
            [X]_{1,1}   & [X]_{1,2}   & \cdots & [X]_{1,n-1}   & 0      \\
            [X]_{2,1}   & [X]_{2,2}   & \cdots & [X]_{2,n-1}   & 0      \\
            \vdots      & \vdots      & {}     & \vdots        & \vdots \\
            [X]_{n-1,1} & [X]_{n-1,2} & \cdots & [X]_{n-1,n-1} & 0      \\
            0           & 0           & \cdots & 0             & 1      \\
        \end{bmatrix}
        \right).
    \end{align*}
    不难验证, \(g\) 适合倍加不变性与多齐性.
    注意到, 若 \(i < j\),
    则 \(A(n|n)\) 的 \((i, j)\)-元为零.
    故, 由假定,
    \begin{align*}
        g(A(n|n)) = g(I_{n-1}) \det {(A(n|n))}
        = f(I_n) \det {(A(n|n))}.
    \end{align*}
    从而
    \begin{align*}
             &
        f\left(
        \begin{bmatrix}
                [A]_{1,1}   & 0           & \cdots & 0             & 0      \\
                [A]_{2,1}   & [A]_{2,2}   & \cdots & 0             & 0      \\
                \vdots      & \vdots      & \ddots & \vdots        & \vdots \\
                [A]_{n-1,1} & [A]_{n-1,2} & \cdots & [A]_{n-1,n-1} & 0      \\
                0           & 0           & \cdots & 0             & 1      \\
            \end{bmatrix}
        \right)
        \\
        = {} &
        g(A(n|n)) = f(I_n) \det {(A(n|n))}.
    \end{align*}
    则
    \begin{align*}
        f(A) = f(I_n) \det {(A(n|n))}\,[A]_{n,n}
        = f(I_n) \det {(A)}.
    \end{align*}

    所以, \(P(n)\) 是正确的.
    由数学归纳法原理, 待证命题成立.
\end{proof}

有了这些准备, 我们即可证明本节的主要结论.

\begin{proof}
    设定义在全体 \(n\)~级阵上的函数 \(f\)
    适合倍加不变性与多齐性.

    任取一个 \(n\)~级阵 \(A\).
    利用若干次倍加,
    我们可变 \(A\) 为一个 \(n\)~级阵 \(B\),
    使当 \(i < j\) 时, \([B]_{i,j} = 0\).
    因为倍加不变性, \(f(B) = f(A)\).
    由上个定理, \(f(B) = f(I) \det {(B)}\).
    故 \(f(A) = f(I) \det {(B)} = f(I) \det {(A)}\).
    (反过来, 不难验证, 若我们定义
    \(f(A) = f(I) \det {(A)}\),
    则 \(f\) 适合倍加不变性与多齐性.)
\end{proof}

\section{类行列式}

在上节, 我们知道, 若定义在全体 \(n\) 级阵上的函数 \(f\)
适合倍加不变性与多齐性,
则 \(f(A) = f(I) \det {(A)}\).
我们用了二种方法证明此事:
一种方法证明 \(f\) 适合多线性, 并使用已知的确定行列式的定理;
另一种方法不使用多线性, 也不使用已知的确定行列式的定理.
后一种方法可被推广, 得到确定 ``类行列式'' 的定理.
不过, 我们先定义一类函数.

\begin{definition}
    设 \(m\) 是定义在数上的函数.
    若 \(m(1) = 1\),
    且对任何数 \(s\), \(t\), 有 \(m(st) = m(s)\, m(t)\),
    则 \(m\) 是\emph{保乘的}.
\end{definition}

显然, 恒等函数 \(m(t) = t\) 是保乘的.
若 \(k\) 是正整数,
由次方的性质, 可以验证,
\(m(t) = t^k\) 也是保乘的.
此外, 绝对值函数 \(m(t) = |t|\) 也是保乘的
(因为 \(|1| = 1\),
且对任何数 \(a\), \(b\), 有 \(|ab| = |a|\,|b|\)).

\begin{theorem}
    设定义在全体 \(n\)~级阵上的函数 \(f\)
    适合:

    (1)
    倍加不变性.

    (2)
    (``多齐性的变体'')
    对任何不超过 \(n\) 的正整数 \(j\),
    任何 \(n-1\)~个 \(n \times 1\)~阵
    \(a_1\), \(\dots\), \(a_{j-1}\),
    \(a_{j+1}\), \(\dots\), \(a_n\),
    任何 \(n \times 1\)~阵 \(x\),
    任何数 \(s\),
    有
    \begin{align*}
        f {([a_1, \dots, a_{j-1}, sx, a_{j+1}, \dots, a_n])}
        =
        m(s)
        f {([a_1, \dots, a_{j-1}, x, a_{j+1}, \dots, a_n])},
    \end{align*}
    其中, 定义在数上的函数 \(m\) 是保乘的:
    \(m(1) = 1\),
    且对任何数 \(s\), \(t\), 有 \(m(st) = m(s)\, m(t)\).

    那么, 对任何 \(n\)~级阵 \(A\),
    \(f(A) = f(I)\, m(\det {(A)})\).
\end{theorem}

为证明此事, 我们先证如下命题.

\begin{theorem}
    设定义在全体 \(n\)~级阵上的函数 \(f\)
    适合倍加不变性与 ``多齐性的变体''.
    设 \(A\) 是一个 \(n\)~级阵,
    且当 \(i < j\) 时, \([A]_{i,j} = 0\).
    则 \(f(A) = f(I_n)\, m(\det {(A)})\).
\end{theorem}

\begin{proof}
    我们用数学归纳法证明此事.
    具体地, 设 \(P(n)\) 为命题
    \begin{quotation}
        设定义在全体 \(n\)~级阵上的函数 \(f\)
        适合倍加不变性与 ``多齐性的变体''.
        设 \(A\) 是一个 \(n\)~级阵,
        且当 \(i < j\) 时, \([A]_{i,j} = 0\).
        则 \(f(A) = f(I_n)\, m(\det {(A)})\).
    \end{quotation}
    则, 我们的目标是:
    对任何正整数 \(n\), \(P(n)\) 是正确的.

    \(P(1)\) 显然是对的.

    假定 \(P(n-1)\) 是对的.
    我们由此证 \(P(n)\) 也是对的.

    任取一个 \(n\)~级阵 \(A\),
    且当 \(i < j\) 时, \([A]_{i,j} = 0\).
    所以, \(A\) 形如
    \begin{align*}
        \begin{bmatrix}
            [A]_{1,1}   & 0           & \cdots & 0             & 0         \\
            [A]_{2,1}   & [A]_{2,2}   & \cdots & 0             & 0         \\
            \vdots      & \vdots      & \ddots & \vdots        & \vdots    \\
            [A]_{n-1,1} & [A]_{n-1,2} & \cdots & [A]_{n-1,n-1} & 0         \\
            [A]_{n,1}   & [A]_{n,2}   & \cdots & [A]_{n,n-1}   & [A]_{n,n} \\
        \end{bmatrix}
    \end{align*}
    那么, 由 ``多齐性的变体'',
    \begin{align*}
        f(A) = m([A]_{n,n})
        f\left(
        \begin{bmatrix}
                [A]_{1,1}   & 0           & \cdots & 0             & 0      \\
                [A]_{2,1}   & [A]_{2,2}   & \cdots & 0             & 0      \\
                \vdots      & \vdots      & \ddots & \vdots        & \vdots \\
                [A]_{n-1,1} & [A]_{n-1,2} & \cdots & [A]_{n-1,n-1} & 0      \\
                [A]_{n,1}   & [A]_{n,2}   & \cdots & [A]_{n,n-1}   & 1      \\
            \end{bmatrix}
        \right).
    \end{align*}
    利用 \(n-1\)~次倍加不变性,
    \begin{align*}
             &
        f\left(
        \begin{bmatrix}
                [A]_{1,1}   & 0           & \cdots & 0             & 0      \\
                [A]_{2,1}   & [A]_{2,2}   & \cdots & 0             & 0      \\
                \vdots      & \vdots      & \ddots & \vdots        & \vdots \\
                [A]_{n-1,1} & [A]_{n-1,2} & \cdots & [A]_{n-1,n-1} & 0      \\
                [A]_{n,1}   & [A]_{n,2}   & \cdots & [A]_{n,n-1}   & 1      \\
            \end{bmatrix}
        \right)
        \\
        = {} &
        f\left(
        \begin{bmatrix}
                [A]_{1,1}   & 0           & \cdots & 0             & 0      \\
                [A]_{2,1}   & [A]_{2,2}   & \cdots & 0             & 0      \\
                \vdots      & \vdots      & \ddots & \vdots        & \vdots \\
                [A]_{n-1,1} & [A]_{n-1,2} & \cdots & [A]_{n-1,n-1} & 0      \\
                0           & 0           & \cdots & 0             & 1      \\
            \end{bmatrix}
        \right),
    \end{align*}
    故
    \begin{align*}
        f(A) =
        f\left(
        \begin{bmatrix}
                [A]_{1,1}   & 0           & \cdots & 0             & 0      \\
                [A]_{2,1}   & [A]_{2,2}   & \cdots & 0             & 0      \\
                \vdots      & \vdots      & \ddots & \vdots        & \vdots \\
                [A]_{n-1,1} & [A]_{n-1,2} & \cdots & [A]_{n-1,n-1} & 0      \\
                0           & 0           & \cdots & 0             & 1      \\
            \end{bmatrix}
        \right)
        m([A]_{n,n}).
    \end{align*}

    考虑定义在全体 \(n-1\)~级阵上的函数
    \begin{align*}
        g(X)
        =
        f\left(
        \begin{bmatrix}
            [X]_{1,1}   & [X]_{1,2}   & \cdots & [X]_{1,n-1}   & 0      \\
            [X]_{2,1}   & [X]_{2,2}   & \cdots & [X]_{2,n-1}   & 0      \\
            \vdots      & \vdots      & {}     & \vdots        & \vdots \\
            [X]_{n-1,1} & [X]_{n-1,2} & \cdots & [X]_{n-1,n-1} & 0      \\
            0           & 0           & \cdots & 0             & 1      \\
        \end{bmatrix}
        \right).
    \end{align*}
    不难验证, \(g\) 适合倍加不变性与 ``多齐性的变体''.
    注意到, 若 \(i < j\),
    则 \(A(n|n)\) 的 \((i, j)\)-元为零.
    故, 由假定,
    \begin{align*}
        g(A(n|n)) = g(I_{n-1})\, m(\det {(A(n|n))})
        = f(I_n)\, m(\det {(A(n|n))}).
    \end{align*}
    从而
    \begin{align*}
             &
        f\left(
        \begin{bmatrix}
                [A]_{1,1}   & 0           & \cdots & 0             & 0      \\
                [A]_{2,1}   & [A]_{2,2}   & \cdots & 0             & 0      \\
                \vdots      & \vdots      & \ddots & \vdots        & \vdots \\
                [A]_{n-1,1} & [A]_{n-1,2} & \cdots & [A]_{n-1,n-1} & 0      \\
                0           & 0           & \cdots & 0             & 1      \\
            \end{bmatrix}
        \right)
        \\
        = {} &
        g(A(n|n)) = f(I_n)\, m(\det {(A(n|n))}).
    \end{align*}
    则
    \begin{align*}
        f(A)
        = {} & f(I_n)\, m(\det {(A(n|n))}) \,m([A]_{n,n}) \\
        = {} & f(I_n)\, m(\det {(A(n|n))} \,[A]_{n,n})    \\
        = {} & f(I_n)\, m(\det {(A)}).
    \end{align*}

    所以, \(P(n)\) 是正确的.
    由数学归纳法原理, 待证命题成立.
\end{proof}

有了这些准备, 我们即可证明本节的主要结论.

\begin{proof}
    设定义在全体 \(n\)~级阵上的函数 \(f\)
    适合倍加不变性与 ``多齐性的变体''.

    任取一个 \(n\)~级阵 \(A\).
    利用若干次倍加,
    我们可变 \(A\) 为一个 \(n\)~级阵 \(B\),
    使当 \(i < j\) 时, \([B]_{i,j} = 0\).
    因为倍加不变性, \(f(B) = f(A)\).
    由上个定理, \(f(B) = f(I)\, m(\det {(B)})\).
    故 \(f(A) = f(I)\, m(\det {(B)})
    = f(I)\, m(\det {(A)})\).
    (反过来, 不难验证, 若我们定义
    \(f(A) = f(I)\, m(\det {(A)})\),
    则 \(f\) 适合倍加不变性与 ``多齐性的变体''.)
\end{proof}

特别地, 取 \(m\) 为绝对值函数, 我们有

\begin{theorem}
    设定义在全体 \(n\)~级阵上的函数 \(f\)
    适合:

    (1)
    倍加不变性.

    (2)
    对任何不超过 \(n\) 的正整数 \(j\),
    任何 \(n-1\)~个 \(n \times 1\)~阵
    \(a_1\), \(\dots\), \(a_{j-1}\),
    \(a_{j+1}\), \(\dots\), \(a_n\),
    任何 \(n \times 1\)~阵 \(x\),
    任何数 \(s\),
    有
    \begin{align*}
        f {([a_1, \dots, a_{j-1}, sx, a_{j+1}, \dots, a_n])}
        =
        |s|
        f {([a_1, \dots, a_{j-1}, x, a_{j+1}, \dots, a_n])}.
    \end{align*}

    那么, 对任何 \(n\)~级阵 \(A\),
    \(f(A) = f(I)\, |{\det {(A)}}|\).
\end{theorem}

设您在研究某数学问题.
设您发现, 您研究的一个量可被认为是定义在 \(n\)~级阵上的函数
(\(n\) 是某个正整数),
且适合倍加不变性与 ``多齐性的变体''.
那么, 由本节的定理, 它是一个跟行列式有关的量.
这不是偶然的; 这是必然的.
这是好的, 我想.

% 几何地, \(2\)~级阵的行列式的绝对值与平行 \(4\)~边形的面积有关,
% 且 \(3\)~级阵的行列式的绝对值与平行 \(6\)~面体的体积有关.
