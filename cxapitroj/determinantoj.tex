\chapter{行列式}

行列式是一种有用的\emph{工具}.
本章主要介绍方阵的行列式的定义与基本的性质.

% \vfill
% \begin{ilustrajxo*}[h!]%
%     \includegraphics[height=12.5cm]{1}%
%     \centering%
% \end{ilustrajxo*}
% \vfill
\clearpage

\section{缺项定位}

在正式进入本节的内容前,
我要提到一种十分常用的简写.
设 \(a\), \(b\), \(c\), \(d\), \(\dots\) 是若干个文字.
那么, \(a = b = c\) 是
``\(a = b\) 且 \(b = c\)'' 之略.
同理, \(a = b = c = d\) 是
``\(a = b\) 且 \(b = c\) 且 \(c = d\)'' 之略.
类似地, 若 \(x\), \(y\), \(z\) 是实数,
则 \(x < y < z\) 是
``\(x < y\) 且 \(y < z\)'' 之略.
自然地, \(x < y \leq z\) 是
``\(x < y\) 且 \(y \leq z\)'' 之略.
此类简写在本章随处可见,
故您一定要知道它.

\vspace{2ex}

我们先看一个简单的问题.

\begin{example}
    考虑文字列~I:
    \(1\), \(2\), \(3\), \(4\), \(5\),
    \(6\), \(7\), \(8\), \(9\), \(10\).
    我们去除第~\(2\), \(5\), \(8\) 个文字
    (也就是说, 去除 \(2\), \(5\), \(8\)),
    不改变文字的前后次序,
    得到文字列~II:
    \(1\), \(3\), \(4\), \(6\), \(7\), \(9\), \(10\).
    试用公式表示%
    文字列~II 的文字~\(x\) 在文字列~II 的位置~\(f(x)\)
    (比如, 文字 \(7\) 的位置是 \(5\)).

    此事自然不难.
    分段地, 我们可写
    \begin{align*}
        f(x) =
        \begin{cases}
            1, & x = 1;  \\
            2, & x = 3;  \\
            3, & x = 4;  \\
            4, & x = 6;  \\
            5, & x = 7;  \\
            6, & x = 9;  \\
            7, & x = 10. \\
        \end{cases}
    \end{align*}
    不过, 分段或许较多.
    能否减少分段的数目?
    这是可以的.
    % 如果更少的分段是被追求的, 那么,
    比如,
    我们可写
    \begin{align*}
        f(x) =
        \begin{cases}
            x,     & x < 2;     \\
            x - 1, & 2 < x < 5; \\
            x - 2, & 5 < x < 8; \\
            x - 3, & 8 < x.     \\
        \end{cases}
    \end{align*}
    我认为这种写法好一些,
    因为, 这种写法, 更体现 ``本质'':
    \(x\) 前缺几项, 其位置就减几.

    如果我们想进一步地简化公式,
    那我们可作一个新记号.
    设 \(i\), \(j\) 为二个整数.
    定义
    \begin{align*}
        \rho(i, j)
        = \begin{cases}
              0, & i < j;    \\
              1, & i \geq j.
          \end{cases}
    \end{align*}
    注意到, 对\emph{不相等的}二个整数 \(i\), \(j\),
    \(\rho(i, j) = 0\) 相当于 \(i < j\),
    而 \(\rho(i, j) = 1\) 相当于 \(i > j\).
    利用这个 ``\(\rho\)-记号'', 我们可较方便地写出,
    文字列~II 的文字~\(x\) 在文字列~II 的位置
    \begin{align*}
        f(x) = x - (\rho(x, 2) + \rho(x, 5) + \rho(x, 8)).
    \end{align*}
    我们记
    \(m(x) = \rho(x, 2) + \rho(x, 5) + \rho(x, 8)\).
    不难看出,
    \(x < 2\) 时, \(m(x) = 0\);
    此时, \(x\) 前不缺项.
    \(2 < x < 5\) 时, \(m(x) = 1\);
    此时, \(x\) 前缺 \(1\)~项.
    \(5 < x < 8\) 时, \(m(x) = 2\);
    此时, \(x\) 前缺 \(2\)~项.
    \(8 < x\) 时, \(m(x) = 3\);
    此时, \(x\) 前缺 \(3\)~项.
    所以, 用 \(\rho\)-记号表示的 \(f(x)\)
    跟用分段表示的 \(f(x)\) 是一样的.

    还有一件事值得一提.
    因为数的加法适合结合律与交换律,
    故我们可写
    \begin{align*}
        f(x)
        = {} & x - (\rho(x, 2) + \rho(x, 5) + \rho(x, 8))  \\
        = {} & x - (\rho(x, 5) + \rho(x, 8) + \rho(x, 2))  \\
        = {} & x - (\rho(x, 8) + \rho(x, 2) + \rho(x, 5))  \\
        = {} & x - (\rho(x, 2) + \rho(x, 8) + \rho(x, 5))  \\
        = {} & x - (\rho(x, 5) + \rho(x, 2) + \rho(x, 8))  \\
        = {} & x - (\rho(x, 8) + \rho(x, 5) + \rho(x, 2)).
    \end{align*}
\end{example}

为方便, 我再介绍一次 \(\rho\)-记号.

\begin{definition}[\(\rho\)-记号]
    设 \(i\), \(j\) 为二个整数.
    定义
    \begin{align*}
        \rho(i, j)
        = \begin{cases}
              0, & i < j;    \\
              1, & i \geq j.
          \end{cases}
    \end{align*}
\end{definition}

不难看出, \(i \neq j\) 时,
\(\rho(i, j) + \rho(j, i) = 1\);
之后会用到的.
(当然了, \(i = j\) 时, \(\rho(i, j) + \rho(j, i) = 2\);
可是, 这没那么重要.)

\vspace{2ex}

一般化前面的例, 我们有

\begin{theorem}[缺项定位]
    设 \(n\) 为高于 \(1\) 的正整数.
    设文字列~I: \(1\), \(2\), \(\dots\), \(n\).
    设 \(k\) 是低于 \(n\) 的正整数.
    设 \(i_1\), \(i_2\), \(\dots\), \(i_k\) 是%
    不超过 \(n\) 的, 且\emph{互不相同的}正整数.
    去除文字列~I 的%
    第~\(i_1\), \(i_2\), \(\dots\), \(i_k\)~个文字
    (也就是说, 去除 \(i_1\), \(i_2\), \(\dots\), \(i_k\)),
    不改变文字的前后次序,
    得到文字列~II.
    那么,
    文字列~II 的文字 \(x\)
    在文字列~II 的位置
    \begin{align*}
        f(x) = x - (\rho(x, i_1) + \dots + \rho(x, i_k)).
    \end{align*}
\end{theorem}

\begin{proof}
    我们先从小到大地排
    \(i_1\), \(i_2\), \(\dots\), \(i_k\)
    为
    \(i_1'\), \(i_2'\), \(\dots\), \(i_k'\).

    为方便, 我们写
    \(i_0' = 0\),
    \(i_{k+1}' = n + 1\).
    注意到 \(i_0' < 1 \leq i_1'\),
    且 \(i_{k+1}' > n \geq i_k'\).

    文字列~II 的文字恰为%
    所有不等于
    \(i_1'\), \(i_2'\), \(\dots\), \(i_k'\)
    的, 且不超过 \(n\) 的正整数.
    任取文字列~II 的一个文字 \(x\).
    那么, 一定存在不超过 \(k\) 的非负整数 \(v\),
    使 \(i_v' < x < i_{v+1}'\).
    于是,
    \(x\) 前缺了 \(v\)~项.
    不难验证
    \begin{align*}
        \rho(x, i_1') + \dots + \rho(x, i_k') = v.
    \end{align*}
    故 \(x\) 在文字列~II 的位置
    \begin{align*}
        f(x)
        = {} &
        x - v
        \\
        = {} &
        x - (\rho(x, i_1') + \dots + \rho(x, i_k'))
        \\
        = {} &
        x - (\rho(x, i_1) + \dots + \rho(x, i_k)).
    \end{align*}
    我们用了加法的结合律与交换律.
\end{proof}

最后, 我想说,
在我的行列式教学里,
缺项定位其实不是十分重要,
但是,
我给的论证要用到此事.
若您希望理解我给出的论证,
我希望您理解它.

\KunAsteriskoEnEnhavtabelo
\section{排列}
\SenAsteriskoEnEnhavtabelo

\maldevigalegajxo

为方便说话, 我作一个新的定义.

\begin{definition}[排列]
    设 \(a_1\), \(a_2\), \(\dots\), \(a_n\) 是%
    \emph{互不相同的} \(n\)~个文字.
    我们说, 由这 \(n\)~个文字作成的任何一个有序的文字列
    \(a_{i_1}\), \(a_{i_2}\), \(\dots\), \(a_{i_n}\)
    (其中 \(i_1\), \(i_2\), \(\dots\), \(i_n\)
    是\emph{互不相同的}不超过 \(n\) 的 \(n\)~个正整数)
    是 \(a_1\), \(a_2\), \(\dots\), \(a_n\)
    的一个排列.
\end{definition}

不难算出 \(n\)~个互不相同的文字的排列的种数.
从 \(a_1\), \(a_2\), \(\dots\), \(a_n\) 里选一个为 \(1\)~号文字,
有 \(n\)~种选法;
从剩下的 \(n-1\)~个文字里选一个为 \(2\)~号文字,
有 \(n-1\)~种选法;
\(\dots \dots\);
从剩下的 \(2\)~个文字里选一个为 \(n-1\)~号文字,
有 \(2\)~种选法;
从剩下的 \(1\)~个文字里选一个为 \(n\)~号文字,
有 \(1\)~种选法.
由分步乘法计数原理, 知种数为
\begin{align*}
    n \cdot (n - 1) \cdot \dots \cdot 2 \cdot 1.
\end{align*}
不过, 这不是我的教学里的重点.

设文字列~I: \(1\), \(2\), \(\dots\), \(n\).
设 \(i_1\), \(i_2\), \(\dots\), \(i_n\) 是
\(1\), \(2\), \(\dots\), \(n\) 的一个排列.
我们去除文字列~I 的%
第~\(i_1\), \(i_2\), \(\dots\), \(i_{n-1}\) 个文字
(也就是说, 去除 \(i_1\), \(i_2\), \(\dots\), \(i_{n-1}\)),
不改变文字的前后次序,
得到文字列~II: \(i_n\)
(只剩一个文字了).
显然, \(i_n\) 的位置 \(f(i_n) = 1\).
另一方面, 利用 ``缺项定位公式'', 有
\begin{align*}
    f(i_n) = i_n - (\rho(i_n, i_1) + \dots + \rho(i_n, i_{n-1})).
\end{align*}
于是, 我们有

\begin{theorem}
    设
    \(i_1\), \(i_2\), \(\dots\), \(i_n\)
    是
    \(1\), \(2\), \(\dots\), \(n\)
    的一个排列;
    也就是说, 设
    \(i_1\), \(i_2\), \(\dots\), \(i_n\)
    是不超过 \(n\) 的正整数, 且\emph{互不相同}.
    则
    \begin{align*}
        i_n - (\rho(i_n, i_1) + \rho(i_n, i_2)
        + \dots + \rho(i_n, i_{n-1})) = 1,
    \end{align*}
    或
    \begin{align*}
        \rho(i_n, i_1) + \rho(i_n, i_2)
        + \dots + \rho(i_n, i_{n-1}) = i_n - 1.
    \end{align*}
\end{theorem}

\section{\texorpdfstring{\(-1\)~的整数次方}{-1 的整数次方}}

我想讲一讲 \(-1\) 的整数次方.
在证明行列式的公式时,
我们会经常用其性质.

设 \(i\), \(j\) 为整数.
首先, 我们都知道,
\begin{align*}
    (-1)^{i+j} = (-1)^{i} (-1)^{j}.
\end{align*}
因为 \(\pm 1\) 的倒数还是 \(\pm 1\), 故
\begin{align*}
    (-1)^{i} = (-1)^{-i}.
\end{align*}
所以
\begin{align*}
    (-1)^{i+j} = (-1)^{i} (-1)^{j}
    = (-1)^{i} (-1)^{-j}
    = (-1)^{i-j}.
\end{align*}
并且
\begin{align*}
    (-1)^{i-j} = (-1)^{-(i-j)} = (-1)^{-i+j}
    = (-1)^{-i-j}.
\end{align*}
值得一提,
\begin{align*}
    (-1)^{2j} = 1.
\end{align*}
所以
\begin{align*}
    (-1)^{i \pm 2j} = (-1)^{i}.
\end{align*}

我们知道, 任给一个偶数 \(n\),
一定有一个整数 \(k\) 使 \(n = 2k\)
(这是定义);
任给一个奇数 (也就是 ``不是偶数的整数'') \(n'\),
一定有一个整数 \(k'\) 使 \(n' = 2k' + 1\).
所以,
若整数 \(m\) 是偶数, 则 \((-1)^m = 1\);
若整数 \(m\) 是奇数, 则 \((-1)^m = -1\).
由此可知,
若整数 \(m\) 适合 \((-1)^m = 1\),
则 \(m\) 是偶数;
若整数 \(m\) 适合 \((-1)^m = -1\),
则 \(m\) 是奇数.

最后, 我想说,
对任何的整数 \(n\),
\(n(n-1)\) 一定是一个偶数.
由此可见, \((-1)^n = (-1)^{n^2}\).

\KunAsteriskoEnEnhavtabelo
\section{逆序数与符号}
\SenAsteriskoEnEnhavtabelo

\maldevigalegajxo

我们知道, 排列是特别的文字列:
排列的每一个文字是互不相同的.
本节, 我想介绍一些跟排列有关的量.

在研究排列时, 我们通常选 ``文字'' 为整数.
熟知, 整数有大小关系.
设 \(a_1\), \(a_2\), \(\dots\), \(a_n\) 是%
\emph{互不相同的} \(n\)~个整数.
我们从小到大地排
\(a_1\), \(a_2\), \(\dots\), \(a_n\)
为
\(b_1\), \(b_2\), \(\dots\), \(b_n\)
(\(b_1 < b_2 < \dots < b_n\)).
我们说, 这是整数 \(a_1\), \(a_2\), \(\dots\), \(a_n\)
的\emph{自然排列}.
显然, \(a_1\), \(a_2\), \(\dots\), \(a_n\)
的自然排列有且只有一个.

设
\(c_1\), \(c_2\), \(\dots\), \(c_n\)
是 \(a_1\), \(a_2\), \(\dots\), \(a_n\) 的一个排列.
若存在整数 \(i\), \(j\),
使 \(1 \leq i < j \leq n\),
且 \(c_i > c_j\),
我们说, 有序对 \((c_i, c_j)\) 为%
排列 \(c_1\), \(c_2\), \(\dots\), \(c_n\)
的一个\emph{逆序}.
比如, 排列 \(1\), \(2\), \(3\), \(4\), \(5\) 无逆序,
而 \(3\), \(2\), \(5\), \(1\), \(4\) 有 \(5\) 个逆序:
\((3, 2)\), \((3, 1)\), \((2, 1)\), \((5, 1)\), \((5, 4)\).

% 其实,
不难看出,
说 \((c_i, c_j)\) (其中 \(i < j\))
是排列
\(c_1\), \(c_2\), \(\dots\), \(c_n\)
的一个逆序,
\emph{相当于}说 \(\rho(c_i, c_j) = 1\).
(注意, 因为
\(c_1\), \(c_2\), \(\dots\), \(c_n\)
是互不相同的整数,
故不可能出现
\(i < j\),
但 \(c_i = c_j\) 的情形.)

我们说, 一个排列的所有的逆序的数目为其\emph{逆序数}.
比如, 排列 \(1\), \(2\), \(3\), \(4\), \(5\) 的逆序数为 \(0\),
而 \(3\), \(2\), \(5\), \(1\), \(4\) 的逆序数为 \(5\).

不难利用 \(\rho\)-记号写出排列
\(c_1\), \(c_2\), \(\dots\), \(c_n\)
的逆序数
\begin{align*}
    \tau(c_1, c_2, \dots, c_n)
    = {} & \hphantom{{} + {}} \rho(c_1, c_2)
    \\
         & + \rho(c_1, c_3) + \rho(c_2, c_3)
    \\
         & + \rho(c_1, c_4) + \rho(c_2, c_4)
    + \rho(c_3, c_4)
    \\
         & +
    \dots \dots \dots \dots
    \dots \dots \dots \dots
    \dots \dots \dots \dots
    \\
         & + \rho(c_1, c_n) + \rho(c_2, c_n) + \dots
    + \rho(c_{n-1}, c_n)
    \\
    = {} & \sum_{j = 2}^{n} {
    \sum_{i = 1}^{j - 1} {\rho(c_i, c_j)} }
    \\
    = {} & \sum_{1 \leq i < j \leq n} {\rho(c_i, c_j)}.
\end{align*}
比如,
当 \(c_1\), \(c_2\), \(c_3\), \(c_4\), \(c_5\) 为
\(1\), \(2\), \(3\), \(4\), \(5\) 时,
每一个 \(\rho(c_i, c_j)\)
(\(1 \leq i < j \leq 5\))
都是 \(0\);
当 \(c_1\), \(c_2\), \(c_3\), \(c_4\), \(c_5\) 为
\(3\), \(2\), \(5\), \(1\), \(4\) 时,
\(\rho(c_1, c_2)\), \(\rho(c_1, c_4)\),
\(\rho(c_2, c_4)\),
\(\rho(c_3, c_4)\), \(\rho(c_3, c_5)\)
都是 \(1\),
而
\(\rho(c_1, c_3)\), \(\rho(c_1, c_5)\),
\(\rho(c_2, c_3)\), \(\rho(c_2, c_5)\),
\(\rho(c_4, c_5)\)
都是 \(0\).

有一件小事值得一提.
若一个排列恰含一个文字, 我们说, 它没有逆序,
从而它的逆序数为 \(0\).
这跟前面的公式是一样的:
既然没有整数对 \((i, j)\) 适合 \(1 \leq i < j \leq 1\),
那这个和就是 \(0\).

\vspace{2ex}

利用 \(-1\)~的整数次方,
我们可再引入一些跟排列有关的量.

\begin{definition}[符号记号]
    设 \(x\) 为整数.
    定义
    \begin{align*}
        \operatorname{sgn} {(x)}
        =
        \begin{cases}
            1,  & x > 0; \\
            0,  & x = 0; \\
            -1, & x < 0.
        \end{cases}
    \end{align*}
\end{definition}

不难看出, 若 \(u\), \(v\) 是\emph{不相等的}整数,
则 \(\operatorname{sgn} {(v - u)} = (-1)^{\rho(u, v)}\).
并且, 说 \((c_i, c_j)\) (其中 \(i < j\))
是排列
\(c_1\), \(c_2\), \(\dots\), \(c_n\)
的一个逆序,
\emph{相当于}说
\(\operatorname{sgn} {(c_j - c_i)} = -1\).

\begin{definition}[整数文字列的符号]
    设 \(c_1\), \(c_2\), \(\dots\), \(c_n\) 是一个\emph{文字列},
    且其文字全为整数
    (也就是说, \(c_1\), \(c_2\), \(\dots\), \(c_n\) 是
    ``整数文字列'').
    定义其符号为
    \begin{align*}
        s(c_1, c_2, \dots, c_n)
        = {} & \hphantom{\cdot\,\,}
        \operatorname{sgn} {(c_2 - c_1)}
        \\
             & \cdot \operatorname{sgn} {(c_3 - c_1)}
        \cdot \operatorname{sgn} {(c_3 - c_2)}
        \\
             & \cdot \operatorname{sgn} {(c_4 - c_1)}
        \cdot \operatorname{sgn} {(c_4 - c_2)}
        \cdot \operatorname{sgn} {(c_4 - c_3)}
        \\
             & \cdot
        \dots \dots \dots \dots
        \dots \dots \dots \dots
        \dots \dots \dots \dots
        \dots \dots \dots \dots
        \\
             & \cdot \operatorname{sgn} {(c_n - c_1)}
        \cdot \operatorname{sgn} {(c_n - c_2)} \cdot \dots
        \cdot \operatorname{sgn} {(c_n - c_{n-1})}
        \\
        = {} & \prod_{j = 2}^{n} {
        \prod_{i = 1}^{j - 1} {\operatorname{sgn} {(c_j - c_i)}} }
        \\
        = {} & \prod_{1 \leq i < j \leq n}
        {\operatorname{sgn} {(c_j - c_i)}}.
    \end{align*}
\end{definition}

显然, 若 \(c_1\), \(c_2\), \(\dots\), \(c_n\) 里%
有二个相同的文字
(也就是说, 存在整数 \(i\), \(j\),
使 \(1 \leq i < j \leq n\),
且 \(c_i = c_j\)),
则此文字列的符号为零.
反过来, 因为非零的整数的积一定不是零,
故符号为零的文字列%
一定有二个相同的文字.

有一件小事值得一提.
若一个文字列恰含一个文字,
我们说, 它 (作为文字列) 的符号是 \(1\).
这跟前面的公式是一样的:
既然没有整数对 \((i, j)\) 适合 \(1 \leq i < j \leq 1\),
那这个积就是 \(1\).

不要混淆 \(\operatorname{sgn}\) 跟 \(s\):
比如, \(s(0) = 1\), 但 \(\operatorname{sgn} {(0)} = 0\);
又比如, \(s(-3) = 1\), 但 \(\operatorname{sgn} {(-3)} = -1\).

排列是特别的文字列
(排列的文字互不相同),
故排列也有符号,
且要么是 \(1\), 要么是 \(-1\).
根据 \(\operatorname{sgn}\) 与 \(\rho\) 的关系,
再利用 \(-1\) 的整数次方的性质,
不难得到排列的逆序数与符号的关系:

\begin{theorem}
    设 \(a_1\), \(a_2\), \(\dots\), \(a_n\) 是%
    \emph{互不相同的} \(n\)~个整数.
    设
    \(c_1\), \(c_2\), \(\dots\), \(c_n\)
    是 \(a_1\), \(a_2\), \(\dots\), \(a_n\) 的一个排列.
    则
    \begin{align*}
        s(c_1, c_2, \dots, c_n)
        = (-1)^{\tau(c_1, c_2, \dots, c_n)}.
    \end{align*}
\end{theorem}

一个恰含一个文字的文字列,
因其文字也是互不相同的,
故当然是一个排列.
我们说过, 恰含一个文字的排列的逆序数为 \(0\).
因为 \((-1)^0 = 1\),
故, 约定恰含一个文字的文字列的符号为 \(1\),
不是没有道理的.

\section{阵}

虽然, 理论地, 可不用阵 (矩形数表),
直接讲行列式,
但 ``为方便说话'', 我决定先介绍阵的基础.
相应地, 行列式, 会被定义为方阵
(正方形数表)
的一个属性.

\begin{definition}[阵]
    设 \(m\), \(n\) 是正整数.
    我们说,
    由 \(mn\) 个文字作成的 \(m \times n\) 矩形文字表
    (此处的 ``文字'', 一般是数,
    % 如整数、有理数、实数、复数等;
    如整数、有理数、实数等;
    当然, 也可是 ``跟数有关'', 但又不是数的对象,
    如整式、分式等)
    \begin{align*}
        A =
        \begin{bmatrix}
            [A]_{1,1} & [A]_{1,2} & \cdots & [A]_{1,n} \\
            [A]_{2,1} & [A]_{2,2} & \cdots & [A]_{2,n} \\
            \vdots    & \vdots    & {}     & \vdots    \\
            [A]_{m,1} & [A]_{m,2} & \cdots & [A]_{m,n} \\
        \end{bmatrix}
    \end{align*}
    是一个 \(m \times n\)~阵.

    我们说, \emph{有序对} \((m, n)\) 是 \(A\)~的尺寸.
    习惯地, 我们也可写 \((m, n)\) 为 \(m \times n\).
    注意这里的文字 \(\times\):
    一方面, 它可表示乘法, 表示此阵有 \(mn\)~个元;
    另一方面, 因为我们较少用 \(\times\) 表示乘法,
    故当我们用 \(\times\) 时,
    往往有某种特别的意思.
    比如,
    \(
    \begin{bmatrix}
        a & c & e \\
        b & d & f \\
    \end{bmatrix}
    \)
    跟
    \(
    \begin{bmatrix}
        a & d \\
        b & e \\
        c & f \\
    \end{bmatrix}
    \)
    的尺寸是不一样的,
    虽然这二个阵都含 \(6\)~个元.

    我们说, \(1 \times n\)~阵
    \(
    \begin{bmatrix}
        [A]_{i,1} & \cdots & [A]_{i,n}
    \end{bmatrix}
    \)
    是 \(A\) 的行~\(i\),
    \(m \times 1\)~阵
    \(
    \begin{bmatrix}
        [A]_{1,j} \\
        \vdots    \\
        [A]_{m,j} \\
    \end{bmatrix}
    \)
    是 \(A\) 的列~\(j\).
    我们说,
    行~\(i\), 列~\(j\) 交叉处的元 \([A]_{i,j}\)
    是 \(A\) 的 \((i, j)\)-元.

    我们也可写 \(1 \times n\)~阵
    \(
    \begin{bmatrix}
        [A]_{i,1} & \cdots & [A]_{i,n}
    \end{bmatrix}
    \)
    为
    \(
    [[A]_{i,1}, \dots, [A]_{i,n}]
    \).
    这种写法是较紧凑的.
    为了使二个或多个元不被认为是一个元,
    我们在最后一个元前的每一个元后,
    加了一个逗号
    (当然了, 逗号后, 也有一定的空白).

    若一个阵的尺寸 \((m, n)\) 适合 \(m = n\),
    我们说, 它是一个方阵.
    \(n \times n\)~阵的一个常用的名字是 \(n\)~级阵
    (\(n\)~级方阵).

    习惯地,
    % (尤其当我们研究元全是数的阵时),
    我们认为, \(1 \times 1\) 阵 \([a]\)
    跟 \(a\) 是同一个对象;
    形象地, 我们写 \(a = [a]\).

    若一个阵的元全是整数, 我们说, 它是一个整阵;
    若一个阵的元全是有理数, 我们说, 它是一个有理阵;
    若一个阵的元全是实数, 我们说, 它是一个实阵;
    % 若一个阵的元全是复数, 我们说, 它是一个复阵;
    若一个阵的元全是数, 我们说, 它是一个数阵.
    本课程研究的阵一般都是数阵,
    因为我要用数的运算定义阵的大多数运算.

    最后, 但并非不重要地, 说二个阵 \(A\), \(B\) 相等,
    就是说,
    \(A\) 的行数 (即其尺寸 \(m \times n\) 的第~1~分量 \(m\))
    等于 \(B\) 的行数,
    \(A\) 的列数 (即其尺寸 \(m \times n\) 的第~2~分量 \(n\))
    等于 \(B\) 的列数,
    且对任何不超过行数 \(m\) 的正整数 \(i\),
    与任何不超过列数 \(n\) 的正整数 \(j\),
    必 \([A]_{i,j} = [B]_{i,j}\).
    (通俗地, 二个阵相等, 相当于它们 ``完全一样''.)
    若二个阵 \(A\), \(B\) 相等,
    我们写 \(A = B\).
\end{definition}

我们用文字的相等
(当然, 还有数的相等; 不过, 数也算是文字),
定义了阵的相等.
文字的相等适合如下三条性质:

(1)
每一个文字 \(x\) 都跟自己相等, 即 \(x = x\).

(2)
若二个文字 \(x\), \(y\) 适合 \(x = y\), 则 \(y = x\).

(3)
若三个文字 \(x\), \(y\), \(z\) 适合 \(x = y\), 且 \(y = z\),
则 \(x = z\).

我们可以证明, 阵的相等也适合类似的三条性质.

\begin{theorem}
    阵的相等适合如下三条性质:

    (1)
    每一个阵 \(A\) 都跟自己相等, 即 \(A = A\).

    (2)
    若二个阵 \(A\), \(B\) 适合 \(A = B\), 则 \(B = A\).

    (3)
    若三个阵 \(A\), \(B\), \(C\) 适合 \(A = B\), 且 \(B = C\),
    则 \(A = C\).
\end{theorem}

\begin{proof}
    (1)
    设 \(A\)~的行数与列数分别是 \(m\), \(n\).
    那么,
    \(A\)~的行数~\(m\) 等于 \(A\)~的行数,
    \(A\)~的列数~\(n\) 等于 \(A\)~的列数.
    并且, 对任何不超过行数 \(m\) 的正整数 \(i\),
    与任何不超过列数 \(n\) 的正整数 \(j\),
    必 \([A]_{i,j} = [A]_{i,j}\).
    (我们用到了文字的相等的性质 (1).)
    所以, \(A = A\).

    (2)
    设二个阵 \(A\), \(B\) 适合 \(A = B\).
    设 \(A\)~的行数与列数分别是 \(m\), \(n\).
    那么,
    \(A\)~的行数~\(m\) 等于 \(B\)~的行数,
    \(A\)~的列数~\(n\) 等于 \(B\)~的列数.
    所以, \(B\)~的行数是 \(m\), \(B\)~的列数是 \(n\).
    并且, 对任何不超过行数 \(m\) 的正整数 \(i\),
    与任何不超过列数 \(n\) 的正整数 \(j\),
    必 \([A]_{i,j} = [B]_{i,j}\).

    由此可见,
    \(B\)~的行数~\(m\) 等于 \(A\)~的行数,
    \(B\)~的列数~\(n\) 等于 \(A\)~的列数.
    并且, 对任何不超过行数 \(m\) 的正整数 \(i\),
    与任何不超过列数 \(n\) 的正整数 \(j\),
    必 \([B]_{i,j} = [A]_{i,j}\).
    (我们用到了文字的相等的性质 (2).)
    所以, \(B = A\).

    (3)
    设三个阵 \(A\), \(B\), \(C\) 适合 \(A = B\), 且 \(B = C\).
    设 \(A\)~的行数与列数分别是 \(m\), \(n\).
    那么, 因为 \(A = B\),
    故
    \(A\)~的行数~\(m\) 等于 \(B\)~的行数,
    \(A\)~的列数~\(n\) 等于 \(B\)~的列数.
    从而,
    \(B\)~的行数是 \(m\), \(B\)~的列数是 \(n\).
    并且, 对任何不超过行数 \(m\) 的正整数 \(i\),
    与任何不超过列数 \(n\) 的正整数 \(j\),
    必 \([A]_{i,j} = [B]_{i,j}\).

    又因为 \(B = C\),
    故
    \(B\)~的行数~\(m\) 等于 \(C\)~的行数,
    \(B\)~的列数~\(n\) 等于 \(C\)~的列数.
    从而,
    \(C\)~的行数是 \(m\), \(C\)~的列数是 \(n\).
    并且, 对任何不超过行数 \(m\) 的正整数 \(i\),
    与任何不超过列数 \(n\) 的正整数 \(j\),
    必 \([B]_{i,j} = [C]_{i,j}\).

    由此可见,
    \(A\)~的行数~\(m\) 等于 \(C\)~的行数,
    \(A\)~的列数~\(n\) 等于 \(C\)~的列数.
    并且, 对任何不超过行数 \(m\) 的正整数 \(i\),
    与任何不超过列数 \(n\) 的正整数 \(j\),
    必 \([A]_{i,j} = [C]_{i,j}\).
    (我们用到了文字的相等的性质 (3).)
    所以, \(A = C\).
\end{proof}

在进入数阵的讨论前,
我先引入跟数较不紧密的阵的运算.

\begin{definition}[转置]
    设 \(A\) 是一个 \(m \times n\)~阵.
    定义 \(A\)~的\emph{转置}%
    为一个 \(n \times m\)~阵 \(A^{\mathrm{T}}\),
    其中, 对任何不超过 \(n\)~的正整数 \(i\)
    与任何不超过 \(m\)~的正整数 \(j\),
    \begin{align*}
        [A^{\mathrm{T}}]_{i,j} = [A]_{j,i}.
    \end{align*}
    % 作 \(n \times m\)~阵 \(A^{\mathrm{T}}\),
    % 使 \([A^{\mathrm{T}}]_{i,j} = [A]_{j,i}\)
    % (\(1 \leq i \leq n\), \(1 \leq j \leq m\)).
    % 我们说, \(A^{\mathrm{T}}\) 是 \(A\) 的\emph{转置}.
\end{definition}

% \begin{example}
%     设
%     \(
%     A =
%     \begin{bmatrix}
%         a & c & e \\
%         b & d & f \\
%     \end{bmatrix}.
%     \)
%     根据定义, \(A^{\mathrm{T}}\) 是一个 \(3 \times 2\)~阵, 且
%     \begin{align*}
%          & [A^{\mathrm{T}}]_{1,1} = [A]_{1,1} = a,
%         \quad [A^{\mathrm{T}}]_{1,2} = [A]_{2,1} = b, \\
%          & [A^{\mathrm{T}}]_{2,1} = [A]_{1,2} = c,
%         \quad [A^{\mathrm{T}}]_{2,2} = [A]_{2,2} = d, \\
%          & [A^{\mathrm{T}}]_{3,1} = [A]_{1,3} = e,
%         \quad [A^{\mathrm{T}}]_{3,2} = [A]_{2,3} = f.
%     \end{align*}
%     故
%     \(
%     A^{\mathrm{T}}
%     = \begin{bmatrix}
%         a & b \\
%         c & d \\
%         e & f \\
%     \end{bmatrix}.
%     \)
% \end{example}

\begin{example}
    设
    \begin{align*}
        A =
        \begin{bmatrix}
            [A]_{1,1} & [A]_{1,2} & \cdots & [A]_{1,n} \\
            [A]_{2,1} & [A]_{2,2} & \cdots & [A]_{2,n} \\
            \vdots    & \vdots    & {}     & \vdots    \\
            [A]_{m,1} & [A]_{m,2} & \cdots & [A]_{m,n} \\
        \end{bmatrix}
    \end{align*}
    是一个 \(m \times n\)~阵.
    则 \(A\)~的转置 \(A^{\mathrm{T}}\)
    是一个 \(n \times m\)~阵,
    且因 \([A^{\mathrm{T}}]_{i,j} = [A]_{j,i}\),
    故
    \begin{align*}
        A^{\mathrm{T}}
        =
        \begin{bmatrix}
            [A^{\mathrm{T}}]_{1,1} & [A^{\mathrm{T}}]_{1,2} &
            \cdots                 & [A^{\mathrm{T}}]_{1,m}   \\
            [A^{\mathrm{T}}]_{2,1} & [A^{\mathrm{T}}]_{2,2} &
            \cdots                 & [A^{\mathrm{T}}]_{2,m}   \\
            \vdots                 & \vdots                 &
            {}                     & \vdots                   \\
            [A^{\mathrm{T}}]_{n,1} & [A^{\mathrm{T}}]_{n,2} &
            \cdots                 & [A^{\mathrm{T}}]_{n,m}   \\
        \end{bmatrix}
        =
        \begin{bmatrix}
            [A]_{1,1} & [A]_{2,1} & \cdots & [A]_{m,1} \\
            [A]_{1,2} & [A]_{2,2} & \cdots & [A]_{m,2} \\
            \vdots    & \vdots    & {}     & \vdots    \\
            [A]_{1,n} & [A]_{2,n} & \cdots & [A]_{m,n} \\
        \end{bmatrix}.
    \end{align*}
    由此, 不难看出:
    (a)
    \(A^{\mathrm{T}}\)~的行~\(i\)
    跟 \(A\)~的列~\(i\) 对应
    (\(1 \leq i \leq n\)),
    且
    \(A^{\mathrm{T}}\)~的列~\(j\)
    跟 \(A\)~的行~\(j\) 对应
    (\(1 \leq j \leq m\));
    (b)
    互换 \(A\)~的行与列,
    即得 \(A^{\mathrm{T}}\).
\end{example}

关于转置, 我们有一个简单的结论.

\begin{theorem}
    设 \(A\) 是一个阵.
    则 \(A\) 的转置 \(A^{\mathrm{T}}\) 的转置
    \((A^{\mathrm{T}})^{\mathrm{T}}\)
    就是 \(A\),
    即
    \begin{align*}
        (A^{\mathrm{T}})^{\mathrm{T}} = A.
    \end{align*}
\end{theorem}

\begin{proof}
    设 \(A\) 是一个 \(m \times n\)~阵.
    则 \(A^{\mathrm{T}}\) 是一个 \(n \times m\)~阵.
    所以,
    \((A^{\mathrm{T}})^{\mathrm{T}}\) 是一个 \(m \times n\)~阵.
    不说元是否相等, 至少等式二侧的阵的尺寸是相等的.
    现在, 比较元是否相等:
    \begin{align*}
        [(A^{\mathrm{T}})^{\mathrm{T}}]_{i,j}
            = [A^{\mathrm{T}}]_{j,i}
            = [A]_{i,j}.
    \end{align*}
    看来, 元也相等.
\end{proof}

习惯地, 用转置, 我们可写 \(m \times 1\) 阵
\(
\begin{bmatrix}
    [A]_{1,j} \\
    \vdots    \\
    [A]_{m,j} \\
\end{bmatrix}
\)
为
\(
[ [A]_{1,j}, \dots, [A]_{m,j} ]^{\mathrm{T}}
\).
这种写法是有用的,
因为它可以节约一些%
% 纵向的%
空白:
对比
\(
\begin{bmatrix}
    1 \\
    2 \\
    3 \\
    4 \\
    5 \\
\end{bmatrix}
\)
与
\(
[1, 2, 3, 4, 5]^{\mathrm{T}}
\).

\vspace{2ex}

在学习行列式时, 我们经常要研究去除%
阵的若干行、若干列后的得到的阵.
这就是 ``子阵''.

\begin{definition}[子阵, 1]
    设 \(A\) 是一个 \(m \times n\)~阵.
    设 \(i_1\), \(\dots\), \(i_s\) 是%
    不超过 \(m\) 的互不相同的正整数.
    设 \(j_1\), \(\dots\), \(j_t\) 是%
    不超过 \(n\) 的互不相同的正整数.
    那么, 我们可去除 \(A\) 的行~\(i_1\), \(\dots\), \(i_s\),
    且去除 \(A\) 的列~\(j_1\), \(\dots\), \(j_t\).
    此时, 还剩 \(m - s\)~行与 \(n - t\)~列.
    不改变不被去除的元的位置,
    这作成了一个 \((m - s) \times (n - t)\)~阵,
    我们记其为
    \(A({i_1, \dots, i_s}|{j_1, \dots, j_t})\).
\end{definition}

\begin{example}
    设
    \(
    A =
    \begin{bmatrix}
        1 & 4 & 7 & 10 \\
        2 & 5 & 8 & 11 \\
        3 & 6 & 9 & 12 \\
    \end{bmatrix}.
    \)
    则
    \(
    A({1}|{2}) =
    \begin{bmatrix}
        2 & 8 & 11 \\
        3 & 9 & 12 \\
    \end{bmatrix}
    \),
    \(
    A({2,3}|{4}) =
    \begin{bmatrix}
        1 & 4 & 7 \\
    \end{bmatrix}
    \),
    且
    \(
    A({3,1}|{4,2}) = A({1,3}|{2,4}) =
    \begin{bmatrix}
        2 & 8 \\
    \end{bmatrix}
    \).
\end{example}

利用 ``缺项定位公式'', 我们不难写出,
若 \(i\) 不等于 \(i_1\), \(\dots\), \(i_s\) 的任何一个,
且 \(j\) 不等于 \(j_1\), \(\dots\), \(j_t\) 的任何一个,
则 \(A\)~的 \((i, j)\)-元一定是
\(A({i_1, \dots, i_s}|{j_1, \dots, j_t})\)~的
\((i - \rho(i, i_1) - \dots - \rho(i, i_s),
j - \rho(j, j_1) - \dots - \rho(j, j_t))\)-元.

前面, 我们 ``减法地'' 定义了子阵 (及记号),
因为我们 ``去除'' 若干行若干列.
有时, 考虑从原阵 ``取出'' 若干行若干列%
作成的阵是方便的;
也就是说, 我们也要 ``加法地'' 定义子阵 (及记号).

\begin{definition}[子阵, 2]
    设 \(A\) 为 \(m \times n\)~阵.
    设 \(i_1\), \(i_2\), \(\dots\), \(i_s\)
    是不超过 \(m\) 的正整数,
    且互不相同.
    设 \(j_1\), \(j_2\), \(\dots\), \(j_t\)
    是不超过 \(n\) 的正整数,
    且互不相同.
    那么, 我们记取 \(A\) 的行~\(i_1\), \(\dots\), \(i_s\)
    与列~\(j_1\), \(\dots\), \(j_t\) 交叉处的元%
    按\emph{原来的次序}排成的 \(s \times t\)~阵为
    \begin{align*}
        A\binom{i_1, i_2, \dots, i_s}{j_1, j_2, \dots, j_t}.
    \end{align*}
\end{definition}

\begin{example}
    设
    \(
    A =
    \begin{bmatrix}
        1 & 4 & 7 & 10 \\
        2 & 5 & 8 & 11 \\
        3 & 6 & 9 & 12 \\
    \end{bmatrix}.
    \)
    则
    \(
    {\displaystyle A\binom{2}{3}} =
        [8]
    \),
    \(
    {\displaystyle A\binom{1}{1,3}} =
    \begin{bmatrix}
        1 & 7
    \end{bmatrix}
    \),
    且
    \(
    {\displaystyle A\binom{3,1}{4,2} =
            A\binom{1,3}{2,4}} =
    \begin{bmatrix}
        4 & 10 \\
        6 & 12 \\
    \end{bmatrix}
    \).
    注意定义里的 \emph{``原来的次序''},
    故
    \(
    {\displaystyle A\binom{3,1}{4,2}}
    \)
    \emph{不等于}
    \(
    \begin{bmatrix}
        12 & 6 \\
        10 & 4 \\
    \end{bmatrix}.
    \)
\end{example}

设 \(A\) 为 \(m \times n\)~阵.
设
\(1 \leq i_1 < i_2 < \dots < i_s \leq m\),
且
\(1 \leq j_1 < j_2 < \dots < j_t \leq n\).
不难看出,
对任何不超过 \(s\)~的正整数 \(p\)
与任何不超过 \(t\)~的正整数 \(q\),
\begin{align*}
    \left[
        A\binom{i_1, i_2, \dots, i_s}{j_1, j_2, \dots, j_t}
        \right]_{p,q}
    = [A]_{i_p, j_q}.
\end{align*}

\vspace{2ex}

现在, 我们正式进入数阵的讨论.
\emph{从现在开始, 我所说的阵都是数阵.}

我说过, 我要利用数的运算定义阵的运算.
我们先从较简单的 ``加法'' ``减法'' ``数乘'' 开始.
我会在后面讨论阵的较复杂的运算.

\begin{definition}[阵的加法]
    设 \(A\), \(B\) \emph{都}是 \(m \times n\)~阵.
    定义 \(A + B\) \emph{也}是一个 \(m \times n\)~阵,
    其中, 对任何不超过 \(m\) 的正整数 \(i\)
    与任何不超过 \(n\) 的正整数 \(j\),
    \begin{align*}
        [A + B]_{i,j} = [A]_{i,j} + [B]_{i,j}.
    \end{align*}
    (通俗地, (同尺寸的) 阵的加法就是相应位置的元的加法.)
\end{definition}

\begin{example}
    设
    \(
    A = \begin{bmatrix}
        1 & 3 & 5 \\
        2 & 4 & 6 \\
    \end{bmatrix},
    \)
    \(
    B = \begin{bmatrix}
        7  & 8  & 9  \\
        10 & 11 & 12 \\
    \end{bmatrix}.
    \)
    则
    \begin{align*}
        A + B =
        \begin{bmatrix}
            8  & 11 & 14 \\
            12 & 15 & 18 \\
        \end{bmatrix}.
    \end{align*}
\end{example}

不难验证, 阵的加法有结合律与交换律.
具体地, 设 \(A\), \(B\), \(C\) 是三个同尺寸的阵.
那么,
\((A + B) + C = A + (B + C)\),
且
\(A + B = B + A\).

我验证结合律;
我留交换律为您的习题.

\begin{proof}
    设 \(A\), \(B\), \(C\) 的尺寸都是 \(m \times n\).
    那么, \(A + B\) 的尺寸也是 \(m \times n\),
    故 \((A + B) + C\) 的尺寸也是 \(m \times n\).
    同理, \(B + C\) 的尺寸也是 \(m \times n\),
    故 \(A + (B + C)\) 的尺寸也是 \(m \times n\).
    现在, 比较元是否相等:
    \begin{align*}
        [(A + B) + C]_{i,j}
        = {} & [A + B]_{i,j} + [C]_{i,j}           \\
        = {} & ([A]_{i,j} + [B]_{i,j}) + [C]_{i,j} \\
        = {} & [A]_{i,j} + ([B]_{i,j} + [C]_{i,j}) \\
        = {} & [A]_{i,j} + [B + C]_{i,j}           \\
        = {} & [A + (B + C)]_{i,j}.
    \end{align*}
    注意到, 我用到了数的加法的结合律.
\end{proof}

因为结合律, 我们可简单地写
\((A + B) + C\) 或 \(A + (B + C)\)
为 \(A + B + C\).

\vspace{2ex}

我们可用加法定义减法.
设 \(A\), \(B\) 为 \(m \times n\)~阵.
作一个 \(m \times n\)~阵 \(0\)
(即, \emph{零阵}),
其中 \([0]_{i,j} = 0\).
不难算出, \(B + 0 = 0 + B = B\)
(我留此为您的习题).
再作一个 \(m \times n\)~阵 \(-B\),
其中 \([-B]_{i,j} = -[B]_{i,j}\).
不难算出, \(B + (-B) = (-B) + B = 0\)
(我留此为您的习题).
我们说, 这么作出的 \(-B\) 是 \(B\) 的\emph{相反阵}.
我们知道, 数~I 减数~II, 就是数~I 加数~II 的相反数.
所以, 我们定义, \(A - B = A + (-B)\).
于是
\begin{align*}
    [A - B]_{i,j}
    = {} & [A + (-B)]_{i,j}         \\
    = {} & [A]_{i,j} + [-B]_{i,j}   \\
    = {} & [A]_{i,j} + (-[B]_{i,j}) \\
    = {} & [A]_{i,j} - [B]_{i,j}.
\end{align*}
(通俗地, (同尺寸的) 阵的减法就是相应位置的元的减法.)

我们知道, 二个数 \(x\), \(y\) 相等,
相当于 \(x - y = 0\).
由此可证:
二个同尺寸的阵 \(A\), \(B\) 相等,
相当于 \(A - B = 0\).

\vspace{2ex}

我以数乘运算结束本节.

\begin{definition}[阵的数乘]
    设 \(A\) 是一个 \(m \times n\)~阵.
    设 \(k\) 是一个数.
    定义 \(kA\) 也是一个 \(m \times n\)~阵,
    其中, 对任何不超过 \(m\) 的正整数 \(i\)
    与任何不超过 \(n\) 的正整数 \(j\),
    \begin{align*}
        [kA]_{i,j} = k[A]_{i,j}.
    \end{align*}
    (通俗地, 阵的数乘就是以一个数乘阵的每一个元.)
\end{definition}

\begin{example}
    设
    \(
    A = \begin{bmatrix}
        1 & 3 & 5 \\
        2 & 4 & 6 \\
    \end{bmatrix},
    \)
    设 \(k = 7\).
    则
    \begin{align*}
        kA = 7A =
        \begin{bmatrix}
            7  & 21 & 35 \\
            14 & 28 & 42 \\
        \end{bmatrix}.
    \end{align*}
\end{example}

设 \(k\), \(\ell\) 是数.
设 \(A\), \(B\) 是二个同尺寸的阵.
不难验证,
\begin{align*}
     & 1A = A,                     \\
     & k(\ell A) = (k\ell) A,      \\
     & (k + \ell) A = kA + \ell A, \\
     & k(A + B) = kA + kB.
\end{align*}
证明式~1 时, 要用到 \(1x = x\) (\(x\) 是数);
证明式~2 时, 要用到数的乘法的结合律;
证明式~3 与式~4 时, 要用到数的乘法与加法的分配律.

我验证式~2 与式~4;
您验证其他的式.

\begin{proof}
    设 \(k\), \(\ell\) 是数.
    设 \(A\), \(B\) 的尺寸都是 \(m \times n\).

    式~2:
    首先, 因为 \(A\) 是 \(m \times n\)~阵,
    故 \(\ell A\) 是 \(m \times n\)~阵,
    从而 \(k(\ell A)\) 也是 \(m \times n\)~阵.
    其次, 因为 \(A\) 是 \(m \times n\)~阵,
    故 \((k\ell) A\) 是 \(m \times n\)~阵.
    现在, 比较元是否相等:
    \begin{align*}
        [k(\ell A)]_{i,j}
        = k[\ell A]_{i,j}
        = k(\ell [A]_{i,j})
        = (k\ell)\, [A]_{i,j}
            = [(k\ell) A]_{i,j}.
    \end{align*}

    式~4:
    首先, 因为 \(A\), \(B\) 是 \(m \times n\)~阵,
    故 \(A + B\) 是 \(m \times n\)~阵,
    从而 \(k(A + B)\) 也是 \(m \times n\)~阵.
    其次, 因为 \(A\), \(B\) 是 \(m \times n\)~阵,
    故 \(kA\), \(kB\) 是 \(m \times n\)~阵,
    从而 \(kA + kB\) 也是 \(m \times n\)~阵.
    现在, 比较元是否相等:
    \begin{align*}
        [k(A + B)]_{i,j}
        = {} & k[A + B]_{i,j}           \\
        = {} & k([A]_{i,j} + [B]_{i,j}) \\
        = {} & k[A]_{i,j} + k[B]_{i,j}  \\
        = {} & [kA]_{i,j} + [kB]_{i,j}  \\
        = {} & [kA + kB]_{i,j}.
        \qedhere
    \end{align*}
\end{proof}

最后, 我提几件小事.

(1)
对任何阵 \(A\), \(0A = 0\).
这里, 等式左侧的 \(0\) 是数字零,
而等式右侧的 \(0\) 是元全为零的零阵
(当然, 它与 \(A\) 的尺寸相等).

(2)
对任何数 \(k\), \(k0 = 0\).
这里, 等式左右二侧的 \(0\) 都是零阵.

(3)
设 \(k\), \(\ell\) 是数.
设 \(A\), \(B\) 是二个同尺寸的阵.
则
\begin{align*}
     & (k - \ell) A = kA - \ell A, \\
     & k(A - B) = kA - kB.
\end{align*}

\section{行列式}

本节, 我要定义本章的主角, 即行列式
(determinanto).

值得注意的是, 我并不为每一个阵定义行列式;
我只为方阵定义行列式.

\begin{restatable}[行列式]{definition}{DefinitionDeterminants}
    设 \(A\) 是 \(n\)~级阵 (\(n \geq 1\)).
    定义 \(A\) 的行列式
    \begin{align*}
        \det {(A)}
        =
        \begin{dcases}
            [A]_{1,1},
             & n = 1;    \\
            \sum_{i = 1}^{n}
            {(-1)^{i+1} [A]_{i,1} \det {(A(i|1))}},
             & n \geq 2.
        \end{dcases}
    \end{align*}
\end{restatable}

我们看 4~个例.

\begin{example}
    设 \(A = [a]\) 是一个 \(1\)~级阵.
    根据定义, \(\det {(A)} = [A]_{1,1} = a\).
\end{example}

\begin{example}
    设
    \(
    A = \begin{bmatrix}
        a & c \\
        b & d \\
    \end{bmatrix}
    \)
    是一个 \(2\)~级阵.
    根据定义,
    \begin{align*}
        \det {(A)}
        = {} & (-1)^{1+1} [A]_{1,1} \det {(A(1|1))}
        + (-1)^{2+1} [A]_{2,1} \det {(A(2|1))}           \\
        = {} & [A]_{1,1} \det {[[A]_{2,2}]}
        - [A]_{2,1} \det {[[A]_{1,2}]}                   \\
        = {} & [A]_{1,1} [A]_{2,2} - [A]_{2,1} [A]_{1,2} \\
        = {} & ad - bc.
    \end{align*}
    此事是重要的;
    我们会经常用它.
    形象地, 我们可用 ``对角线'' 记
    \(2\)~级阵的行列式.
    \begin{tikzbildo*}[h!]
        \centering
        \begin{tikzpicture}[>=stealth]
            \matrix [%
                matrix of math nodes,
                column sep=1em,
                row sep=1em
            ] (s2) {%
                a & c \\
                b & d \\
            };

            \path
            (s2-1-1)    edge[->]        (s2-2-2)
            (s2-2-1)    edge[->,dashed] (s2-1-2);

            \node[anchor=east] at (s2-1-1.west) {\(+\)};
            \node[anchor=east] at (s2-2-1.west) {\(-\)};
        \end{tikzpicture}
    \end{tikzbildo*}
\end{example}

\begin{example}
    设 \(A\) 是一个 \(3\)~级阵.
    根据定义与上个例的结果,
    \begin{align*}
             & \det {(A)}
        \displaybreak[0]
        \\
        = {} &
        \hphantom{{} + {}}
        (-1)^{1+1} [A]_{1,1} \det {(A(1|1))}
        + (-1)^{2+1} [A]_{2,1} \det {(A(2|1))}
        \displaybreak[0]
        \\
             &
        + (-1)^{3+1} [A]_{3,1} \det {(A(3|1))}
        \displaybreak[0]
        \\
        = {} &
        \hphantom{{} + {}}
        [A]_{1,1}
        \det {\begin{bmatrix}
                      [A]_{2,2} & [A]_{2,3} \\
                      [A]_{3,2} & [A]_{3,3} \\
                  \end{bmatrix}}
        - [A]_{2,1}
        \det {\begin{bmatrix}
                      [A]_{1,2} & [A]_{1,3} \\
                      [A]_{3,2} & [A]_{3,3} \\
                  \end{bmatrix}}
        \displaybreak[0]
        \\
             &
        + [A]_{3,1}
        \det {\begin{bmatrix}
                      [A]_{1,2} & [A]_{1,3} \\
                      [A]_{2,2} & [A]_{2,3} \\
                  \end{bmatrix}}
        \displaybreak[0]
        \\
        = {} &
        \hphantom{{} + {}}
        [A]_{1,1}
        ([A]_{2,2} [A]_{3,3} - [A]_{3,2} [A]_{2,3})
        - [A]_{2,1}
        ([A]_{1,2} [A]_{3,3} - [A]_{3,2} [A]_{1,3})
        \displaybreak[0]
        \\
             &
        + [A]_{3,1}
        ([A]_{1,2} [A]_{2,3} - [A]_{2,2} [A]_{1,3})
        \displaybreak[0]
        \\
        = {} &
        \hphantom{{} + {}}
        [A]_{1,1} [A]_{2,2} [A]_{3,3}
        - [A]_{1,1} [A]_{3,2} [A]_{2,3}
        \displaybreak[0]
        \\
             &
        - [A]_{2,1} [A]_{1,2} [A]_{3,3}
        + [A]_{2,1} [A]_{3,2} [A]_{1,3}
        \displaybreak[0]
        \\
             &
        + [A]_{3,1} [A]_{1,2} [A]_{2,3}
        - [A]_{3,1} [A]_{2,2} [A]_{1,3}
        \displaybreak[0]
        \\
        = {} &
        \hphantom{{} + {}}
        [A]_{1,1} [A]_{2,2} [A]_{3,3}
        + [A]_{2,1} [A]_{3,2} [A]_{1,3}
        + [A]_{3,1} [A]_{1,2} [A]_{2,3}
        \displaybreak[0]
        \\
             &
        - [A]_{1,1} [A]_{3,2} [A]_{2,3}
        - [A]_{2,1} [A]_{1,2} [A]_{3,3}
        - [A]_{3,1} [A]_{2,2} [A]_{1,3}.
    \end{align*}
    % 法国%
    数学家 Pierre Frédéric Sarrus
    给了我们一种记 \(3\)~级阵的行列式的好方法.
    我们在阵的下方重写它.
    由左上至右下的对角线 (实线) 上的数的积的和%
    减由左下至右上的对角线 (虚线) 上的数的积的和%
    即为此阵的行列式.
    % https://tex.stackexchange.com/questions/32978/typesetting-a-matrix-with-crossing-arrows-on-it/#32981
    \begin{tikzbildo*}[h!]
        \centering
        \begin{tikzpicture}[>=stealth]
            \matrix [%
            matrix of math nodes,
            column sep=1em,
            row sep=1em
            ] (s3) {%
            {[}A{]}_{1,1} & {[}A{]}_{1,2} & {[}A{]}_{1,3} \\
            {[}A{]}_{2,1} & {[}A{]}_{2,2} & {[}A{]}_{2,3} \\
            {[}A{]}_{3,1} & {[}A{]}_{3,2} & {[}A{]}_{3,3} \\
            {[}A{]}_{1,1} & {[}A{]}_{1,2} & {[}A{]}_{1,3} \\
            {[}A{]}_{2,1} & {[}A{]}_{2,2} & {[}A{]}_{2,3} \\
            {[}A{]}_{3,1} & {[}A{]}_{3,2} & {[}A{]}_{3,3} \\
            };

            \path
            (s3-1-1)    edge            (s3-2-2)
            (s3-2-2)    edge[->]        (s3-3-3)
            (s3-2-1)    edge            (s3-3-2)
            (s3-3-2)    edge[->]        (s3-4-3)
            (s3-3-1)    edge            (s3-4-2)
            (s3-4-2)    edge[->]        (s3-5-3)
            (s3-4-1)    edge[dashed]    (s3-3-2)
            (s3-3-2)    edge[->,dashed] (s3-2-3)
            (s3-5-1)    edge[dashed]    (s3-4-2)
            (s3-4-2)    edge[->,dashed] (s3-3-3)
            (s3-6-1)    edge[dashed]    (s3-5-2)
            (s3-5-2)    edge[->,dashed] (s3-4-3);

            \foreach \r in {1,2,3}
                {\node[anchor=east] at (s3-\r-1.west) {\(+\)};}
            \foreach \r in {4,5,6}
                {\node[anchor=east] at (s3-\r-1.west) {\(-\)};}
        \end{tikzpicture}
    \end{tikzbildo*}
\end{example}

值得一提的是,
前面的对角线法则无法被推广到
\(n\)~级阵 (\(n \geq 4\)) 的行列式.
(当然, 我不要求您记 ``如此具体的公式'';
您记定义即可.)

\begin{example}
    设 \(A\) 是一个 \(4\)~级阵.
    根据定义,
    \begin{align*}
             & \det {(A)}
        \\
        = {} &
        \hphantom{{} + {}}
        (-1)^{1+1} [A]_{1,1} \det {(A(1|1))}
        + (-1)^{2+1} [A]_{2,1} \det {(A(2|1))}
        \\
             &
        + (-1)^{3+1} [A]_{3,1} \det {(A(3|1))}
        + (-1)^{4+1} [A]_{4,1} \det {(A(4|1))}.
    \end{align*}
    利用跟上个例相似的方法, 并利用上个例的结果,
    可知结果含 \(24\)~项:
    \begin{align*}
         & \hphantom{{} + {}}
        [A]_{1,1} [A]_{2,2} [A]_{3,3} [A]_{4,4}
        + [A]_{1,1} [A]_{3,2} [A]_{4,3} [A]_{2,4}
        + [A]_{1,1} [A]_{4,2} [A]_{2,3} [A]_{3,4}
        \\
         &
        - [A]_{1,1} [A]_{2,2} [A]_{4,3} [A]_{3,4}
        - [A]_{1,1} [A]_{3,2} [A]_{2,3} [A]_{4,4}
        - [A]_{1,1} [A]_{4,2} [A]_{3,3} [A]_{2,4}
        \\
         &
        - [A]_{2,1} [A]_{1,2} [A]_{3,3} [A]_{4,4}
        - [A]_{2,1} [A]_{3,2} [A]_{4,3} [A]_{1,4}
        - [A]_{2,1} [A]_{4,2} [A]_{1,3} [A]_{3,4}
        \\
         &
        + [A]_{2,1} [A]_{1,2} [A]_{4,3} [A]_{3,4}
        + [A]_{2,1} [A]_{3,2} [A]_{1,3} [A]_{4,4}
        + [A]_{2,1} [A]_{4,2} [A]_{3,3} [A]_{1,4}
        \\
         &
        + [A]_{3,1} [A]_{1,2} [A]_{2,3} [A]_{4,4}
        + [A]_{3,1} [A]_{2,2} [A]_{4,3} [A]_{1,4}
        + [A]_{3,1} [A]_{4,2} [A]_{1,3} [A]_{2,4}
        \\
         &
        - [A]_{3,1} [A]_{1,2} [A]_{4,3} [A]_{2,4}
        - [A]_{3,1} [A]_{2,2} [A]_{1,3} [A]_{4,4}
        - [A]_{3,1} [A]_{4,2} [A]_{2,3} [A]_{1,4}
        \\
         &
        - [A]_{4,1} [A]_{1,2} [A]_{2,3} [A]_{3,4}
        - [A]_{4,1} [A]_{2,2} [A]_{3,3} [A]_{1,4}
        - [A]_{4,1} [A]_{3,2} [A]_{1,3} [A]_{2,4}
        \\
         &
        + [A]_{4,1} [A]_{1,2} [A]_{3,3} [A]_{2,4}
        + [A]_{4,1} [A]_{2,2} [A]_{1,3} [A]_{3,4}
        + [A]_{4,1} [A]_{3,2} [A]_{2,3} [A]_{1,4}.
    \end{align*}
    根据 ``对角线'' 法则,
    我们应\emph{减}由左下至右上的对角线上的数的积.
    可是, 上式说,
    \([A]_{4,1} [A]_{3,2} [A]_{2,3} [A]_{1,4}\)
    前的符号实则为 \(1\),
    而不是 ``对角线'' 法则所认为的 \(-1\).
    \begin{tikzbildo*}[h!]
        \centering
        \begin{tikzpicture}[>=stealth]
            \matrix [%
            matrix of math nodes,
            column sep=1em,
            row sep=1em
            ] (s4) {%
            {[}A{]}_{1,1} & {[}A{]}_{1,2} &
            {[}A{]}_{1,3} & {[}A{]}_{1,4} \\
            {[}A{]}_{2,1} & {[}A{]}_{2,2} &
            {[}A{]}_{2,3} & {[}A{]}_{2,4} \\
            {[}A{]}_{3,1} & {[}A{]}_{3,2} &
            {[}A{]}_{3,3} & {[}A{]}_{3,4} \\
            {[}A{]}_{4,1} & {[}A{]}_{4,2} &
            {[}A{]}_{4,3} & {[}A{]}_{4,4} \\
            };

            \path
            % (s4-4-1)    edge[dashed]    (s4-3-2)
            % (s4-3-2)    edge[dashed]    (s4-2-3)
            % (s4-2-3)    edge[->,dashed] (s4-1-4);
            (s4-4-1)    edge[]    (s4-3-2)
            (s4-3-2)    edge[]    (s4-2-3)
            (s4-2-3)    edge[->]  (s4-1-4);

            \node[anchor=north] at (s4-4-1.south)
            % {It is \(+\).};
            {\(+\)};
        \end{tikzpicture}
    \end{tikzbildo*}
\end{example}

\section{按一列展开行列式}

本节, 我们学习按一列展开行列式.

我们先回想定义.

\DefinitionDeterminants*

不难看出,
\([A]_{1,1}\), \([A]_{2,1}\), \(\dots\), \([A]_{n,1}\)
全是 \(A\)~的列~\(1\) 的元,
故我们说,
此定义按列~\(1\) 展开行列式.
自然地, 我们会想,
我们能否 ``按其他的列展开行列式''.
% 幸运地,
此事的回答是 ``是''.

\begin{restatable}[]{theorem}{TheoremExpansionAboutAnyColumn}
    设 \(A\) 是 \(n\)~级阵 (\(n \geq 1\)).
    设 \(j\) 为整数, 且 \(1 \leq j \leq n\).
    则
    \begin{align*}
        \det {(A)} = \sum_{i = 1}^{n}
        {(-1)^{i+j} [A]_{i,j} \det {(A(i|j))}}.
    \end{align*}
\end{restatable}

或许, 我应解释,
当 \(A\) 是 \(1\)~级阵时,
\(\det {(A(1|1))}\) 是什么意思.
显然, 去除 \(A\) 的唯一的一行与唯一的一列后,
就不剩下任何一个元了.
我给出的阵的定义显然是未涉及此情形的.
我给出的行列式的定义也未涉及此情形.
% 此时,
我们作一个约定:
% 此时, 我们要一个约定.
% 就像
% ``允许集不含任何元''
% ``空集是任何集的子集''
% ``一个数的 \(0\)~次方等于 \(1\)''
% ``\(0\)~的阶乘等于 \(1\)''
% ``若 \(p\) 是假的, 则 `若 \(p\), 则 \(q\)' 是真的''
% 那样,
% 我们可大胆地约定,
\emph{\(1\)~级阵 \(A\) 的子阵 \(A(1|1)\) 是 ``\(0\)~级阵'',
    且 ``\(0\)~级阵'' 的行列式是 \(1\).}
这么看来, \((-1)^{1+1} [A]_{1,1} \det {(A(1|1))}\)
就是 \(1 \cdot [A]_{1,1} \cdot 1\),
即 \([A]_{1,1}\),
跟行列式的定义一样.
% 所以, 此约定非常合理, 没有什么问题.
% 以后, 我就不重复解释此约定了.

我们无妨先用 \(2\)~级阵验证此命题.
设 \(A\) 是 \(2\)~级阵 (也就是, 取 \(n = 2\)).
\(j = 1\) 时, 这就是定义;
\(j = 2\) 时,
\begin{align*}
    \det {(A)}
    = {} & [A]_{1,1} \det {[[A]_{2,2}]}
    - [A]_{2,1} \det {[[A]_{1,2}]}              \\
    = {} & {-[A]_{1,2}} \det {[[A]_{2,1}]}
    + [A]_{2,2} \det {[[A]_{1,1}]}              \\
    = {} & (-1)^{2+1} [A]_{2,1} \det {(A(2|1))}
    + (-1)^{2+2} [A]_{2,2} \det {(A(2|2))}.
\end{align*}
所以, \(n = 2\), \(j = 2\) 时, 命题是正确的.

\begin{proof}
    我们会用数学归纳法证明此事.
    具体地, 设 \(P(n)\) 为命题
    \begin{quotation}
        对\emph{任何} \(n\)~级阵 \(A\),
        对\emph{任何}不超过 \(n\) 的正整数 \(j\),
        \begin{align*}
            \det {(A)} = \sum_{i = 1}^{n}
            {(-1)^{i+j} [A]_{i,j} \det {(A(i|j))}}.
        \end{align*}
    \end{quotation}
    则, 我们的目标是:
    对任何正整数 \(n\), \(P(n)\) 是正确的.

    用数学归纳法不应是一个意外:
    毕竟, 我给出的行列式的定义就是%
    先定义小级阵的行列式,
    再用小级阵的行列式定义大级阵的行列式.
    值得一提的是, 我强调了 ``任何'' 二字;
    这一点, 在后面的论证里, 是重要的.

    设 \(n = 1\).
    \(j = 1\) 时, 显然.
    故 \(P(1)\) 是正确的.

    设 \(n = 2\).
    \(j = 1\) 时, 也显然 (定义).
    \(j = 2\) 时, 我们已经验证过了.
    故 \(P(2)\) 是正确的.

    现在, 我们假定 \(P(m-1)\) 是正确的.
    我们要证 \(P(m)\) 也是正确的.
    任取一个 \(m\)~级阵 \(A\).
    若 \(j = 1\), 这是定义, 不必证.
    现设 \(j \neq 1\).
    于是
    \begin{align*}
             & \det {(A)}                  \\
        = {} &
        \sum_{i = 1}^{m} {
        (-1)^{i+1} [A]_{i,1} \det {(A(i|1))}
        }
        \tag*{(1)}
        \\
        = {} &
        \sum_{i = 1}^{m} {
        (-1)^{i+1} [A]_{i,1}
        \sum_{\substack{1 \leq \ell \leq m \\ \ell \neq i}} {
        (-1)^{(\ell - \rho(\ell, i)) + (j - 1)}
            [A]_{\ell,j} \det {(A({i,\ell}|{1,j}))}
        }
        }
        \tag*{(2)}
        \\
        = {} &
        \sum_{i = 1}^{m} {
        \sum_{\substack{1 \leq \ell \leq m \\ \ell \neq i}} {
        (-1)^{(\ell - \rho(\ell, i)) + (j - 1)}
        (-1)^{i+1}
            [A]_{i,1} [A]_{\ell,j} \det {(A({i,\ell}|{1,j}))}
        }
        }
        \tag*{(3)}
        \\
        = {} &
        \sum_{\ell = 1}^{m} {
        \sum_{\substack{1 \leq i \leq m    \\ i \neq \ell}} {
        (-1)^{(\ell - \rho(\ell, i)) + (j - 1)}
        (-1)^{i+1}
            [A]_{i,1} [A]_{\ell,j} \det {(A({i,\ell}|{1,j}))}
        }
        }
        \tag*{(4)}
        \\
        = {} &
        \sum_{\ell = 1}^{m} {
        \sum_{\substack{1 \leq i \leq m    \\ i \neq \ell}} {
        (-1)^{\ell} (-1)^{\rho(\ell, i)} (-1)^{i} (-1)^{j}
            [A]_{i,1} [A]_{\ell,j} \det {(A({i,\ell}|{1,j}))}
        }
        }
        \tag*{(5)}
        \\
        = {} &
        \sum_{\ell = 1}^{m} {
        \sum_{\substack{1 \leq i \leq m    \\ i \neq \ell}} {
        (-1)^{\ell + j} (-1)^{i - \rho(i, \ell) + 1}
            [A]_{i,1} [A]_{\ell,j} \det {(A({i,\ell}|{1,j}))}
        }
        }
        \tag*{(6)}
        \\
        = {} &
        \sum_{\ell = 1}^{m} {(-1)^{\ell + j} [A]_{\ell,j}
        \sum_{\substack{1 \leq i \leq m    \\ i \neq \ell}} {
        (-1)^{i - \rho(i, \ell) + 1}
            [A]_{i,1} \det {(A({i,\ell}|{1,j}))}
        }
        }
        \tag*{(7)}
        \\
        = {} &
        \sum_{\ell = 1}^{m} {(-1)^{\ell + j} [A]_{\ell,j}
        \det {(A(\ell|j))}
        }.
        \tag*{(8)}
    \end{align*}
    所以, \(P(m)\) 是正确的.
    由数学归纳法原理, 待证命题成立.

    我想, 您一定看到了公式右侧的序号.
    这是方便我说话用的.
    我想, 适当地解释这几步,
    % 对喜欢钻研的初学行列式的读者而言,
    是有好处的.

    (1) 是行列式的定义.

    (2) 利用了假定.
    我们假定可按\emph{任何}列展开%
    \emph{每一个} \(m - 1\)~级阵的行列式.
    \(A(i|1)\) 不就是 \(m - 1\)~级阵吗?
    那么, 我们就按 \(A(i|1)\) 的列~\(j - 1\) 展开.
    \(A(i|1)\) 的列~\(j - 1\) 正好对应
    \(A\) 的列~\(j\).
    最后, 注意到, \(\ell \neq i\) 时,
    \(A\)~的 \((\ell, j)\)-元恰是
    \(A(i|1)\)~的 \((\ell - \rho(\ell, i), j - 1)\)-元.

    (3) 利用了分配律 (还有加法的结合律与交换律).

    (4) 利用了加法的结合律与交换律.
    (通俗地, 就是 ``求和号的次序可换''.)

    (5) 利用了 \(-1\)~的整数次方的性质.

    (6) 还是利用 \(-1\)~的整数次方的性质.
    当然了, 也用到了 \(\rho\)-记号的性质.

    (7) 又用了一次分配律 (还有加法的结合律与交换律).
    不过, 跟 (3) 对比, 这次是反着用.

    (8) 用到了行列式的定义.
    注意到, \(i \neq \ell\) 时,
    \(A\)~的 \((i, 1)\)-元恰是
    \(A(\ell|j)\)~的 \((i - \rho(i, \ell), 1)\)-元
    % (别忘了, 在这里, \(j \neq 1\)).
    (其中 \(j \neq 1\)).
\end{proof}

本节的定理十分重要.

\KunAsteriskoEnEnhavtabelo
\section{按多列展开行列式}
\SenAsteriskoEnEnhavtabelo

\maldevigalegajxo

本节, 我们学习按多列展开行列式.

我们已知, 我们可按任何一列展开行列式:

\TheoremExpansionAboutAnyColumn*

自然地, 我们会想, 我们能否 ``按多列展开行列式''.
% 幸运地,
此事的回答是 ``是''.

\begin{theorem}
    设 \(A\) 是 \(n\)~级阵 (\(n \geq 1\)).
    设 \(k\) 是不超过 \(n\) 的正整数.
    设 \(j_1\), \(j_2\), \(\dots\), \(j_k\) 是%
    不超过 \(n\) 的正整数,
    且 \(j_1 < j_2 < \dots < j_k\).
    则
    \begin{align*}
         &
        \det {(A)}
        = \sum_{1 \leq i_1 < i_2 < \dots < i_k \leq n}
        {\det {\left(
                A\binom{i_1, i_2, \dots, i_k}
                {j_1, j_2, \dots, j_k}
                \right)}}
        \\
         &
        \qquad \qquad \qquad
        \cdot (-1)^{i_1 + i_2 + \dots + i_k
            + j_1 + j_2 + \dots + j_k}
        \det {(A({i_1,i_2,\dots,i_k}|{j_1,j_2,\dots,j_k}))}.
    \end{align*}
\end{theorem}

若 \(k = 1\), 则
\(\displaystyle
A\binom{i_1}{j_1} = [[A]_{i_1,j_1}]\)
是一个 \(1\)~级阵.
回想, \(1\)~级阵 \([a]\) 的行列式就是 \(a\).
所以, 若 \(k = 1\), 则此定理就是
``按一列展开行列式''.

论证此事前, 我想用一个例助您理解, 此定理在说什么.

\begin{example}
    设
    \begin{align*}
        A =
        \begin{bmatrix}
            1 & 6 & 11 & 16 \\
            2 & 7 & 12 & 13 \\
            3 & 8 & 9  & 14 \\
            4 & 5 & 10 & 15 \\
        \end{bmatrix}.
    \end{align*}

    一方面,
    根据定义,
    \begin{align*}
             & \det {(A)} \\
        = {} &
        \hphantom{{} + {}}
        (-1)^{1+1} [A]_{1,1} \det {(A(1|1))}
        + (-1)^{2+1} [A]_{2,1} \det {(A(2|1))}
        \\
             &
        + (-1)^{3+1} [A]_{3,1} \det {(A(3|1))}
        + (-1)^{4+1} [A]_{4,1} \det {(A(4|1))}.
    \end{align*}
    可算出
    \begin{align*}
         & \det {(A(1|1))}
        = \det {\begin{bmatrix}
                        7 & 12 & 13 \\
                        8 & 9  & 14 \\
                        5 & 10 & 15 \\
                    \end{bmatrix}}
        = -180,            \\
         & \det {(A(2|1))}
        = \det {\begin{bmatrix}
                        6 & 11 & 16 \\
                        8 & 9  & 14 \\
                        5 & 10 & 15 \\
                    \end{bmatrix}}
        = -20,             \\
         & \det {(A(3|1))}
        = \det {\begin{bmatrix}
                        6 & 11 & 16 \\
                        7 & 12 & 13 \\
                        5 & 10 & 15 \\
                    \end{bmatrix}}
        = 20,              \\
         & \det {(A(4|1))}
        = \det {\begin{bmatrix}
                        6 & 11 & 16 \\
                        7 & 12 & 13 \\
                        8 & 9  & 14 \\
                    \end{bmatrix}}
        = -156.
    \end{align*}
    所以
    \begin{align*}
        \det {(A)}
        = 1 \cdot (-180) - 2 \cdot (-20)
        + 3 \cdot 20 - 4 \cdot (-156)
        = 544.
    \end{align*}

    另一方面,
    我们试按列~\(1\), \(2\) 展开.
    取 \(j_1\), \(j_2\) 为 \(1\), \(2\).
    不难写出, 适合条件
    ``\(1 \leq i_1 < i_2 \leq 4\)''
    的 \((i_1, i_2)\) 有 \(6\)~个:
    \((1, 2)\),
    \((1, 3)\), \((2, 3)\),
    \((1, 4)\), \((2, 4)\), \((3, 4)\).
    由此, 可写出
    \begin{align*}
         & A\binom{1,2}{1,2}
        = \begin{bmatrix}1&6\\2&7\\ \end{bmatrix},
        \quad
        (-1)^{1+2+1+2} = +1,
        \quad
        A(1,2|1,2)
        = \begin{bmatrix}9&14\\10&15\\ \end{bmatrix},  \\
         & A\binom{1,3}{1,2}
        = \begin{bmatrix}1&6\\3&8\\ \end{bmatrix},
        \quad
        (-1)^{1+3+1+2} = -1,
        \quad
        A(1,3|1,2)
        = \begin{bmatrix}12&13\\10&15\\ \end{bmatrix}, \\
         & A\binom{2,3}{1,2}
        = \begin{bmatrix}2&7\\3&8\\ \end{bmatrix},
        \quad
        (-1)^{2+3+1+2} = +1,
        \quad
        A(2,3|1,2)
        = \begin{bmatrix}11&16\\10&15\\ \end{bmatrix}, \\
         & A\binom{1,4}{1,2}
        = \begin{bmatrix}1&6\\4&5\\ \end{bmatrix},
        \quad
        (-1)^{1+4+1+2} = +1,
        \quad
        A(1,4|1,2)
        = \begin{bmatrix}12&13\\9&14\\ \end{bmatrix},  \\
         & A\binom{2,4}{1,2}
        = \begin{bmatrix}2&7\\4&5\\ \end{bmatrix},
        \quad
        (-1)^{2+4+1+2} = -1,
        \quad
        A(2,4|1,2)
        = \begin{bmatrix}11&16\\9&14\\ \end{bmatrix},  \\
         & A\binom{3,4}{1,2}
        = \begin{bmatrix}3&8\\4&5\\ \end{bmatrix},
        \quad
        (-1)^{3+4+1+2} = +1,
        \quad
        A(3,4|1,2)
        = \begin{bmatrix}11&16\\12&13\\ \end{bmatrix}.
    \end{align*}
    定理说,
    \begin{align*}
        \det {(A)}
        = {} & \hphantom{{} + {}}
        \det {\left( A\binom{1,2}{1,2} \right)}\,
        (-1)^{1+2+1+2} \det {(A(1,2|1,2))}
        \\
             & + \det {\left( A\binom{1,3}{1,2} \right)}\,
        (-1)^{1+3+1+2} \det {(A(1,3|1,2))}
        \\
             & + \det {\left( A\binom{2,3}{1,2} \right)}\,
        (-1)^{2+3+1+2} \det {(A(2,3|1,2))}
        \\
             & + \det {\left( A\binom{1,4}{1,2} \right)}\,
        (-1)^{1+4+1+2} \det {(A(1,4|1,2))}
        \\
             & + \det {\left( A\binom{2,4}{1,2} \right)}\,
        (-1)^{2+4+1+2} \det {(A(2,4|1,2))}
        \\
             & + \det {\left( A\binom{3,4}{1,2} \right)}\,
        (-1)^{3+4+1+2} \det {(A(3,4|1,2))}
        \\
        = {} & \hphantom{{} + {}}
        (-5) \cdot (-5) - (-10) \cdot 50
        \\
             & + (-5) \cdot 5 + (-19) \cdot 51
        \\
             & - (-18) \cdot 10 + (-17) \cdot (-49)
        \\
        = {} & 25 + 500 - 25 - 969 + 180 + 833
        \\
        = {} & 544.
    \end{align*}
\end{example}

\begin{proof}
    我们用数学归纳法证明此事.
    具体地, 设 \(P(k)\) 为命题
    \begin{quotation}
        对\emph{任何} \(n\)~级阵 \(A\)
        \emph{(其中 \(n \geq k\))},
        对\emph{任何}不超过 \(n\) 的 \(k\) 个正整数
        \(j_1\), \(j_2\), \(\dots\), \(j_k\)
        \emph{(其中 \(j_1 < j_2 < \dots < j_k\))},
        \begin{align*}
             &
            \det {(A)}
            = \sum_{1 \leq i_1 < i_2 < \dots < i_k \leq n}
            {\det {\left(
                    A\binom{i_1, i_2, \dots, i_k}%
                    {j_1, j_2, \dots, j_k}
                    \right)}}
            \\
             &
            \qquad \qquad \qquad
            \cdot (-1)^{i_1 + i_2 + \dots + i_k
                + j_1 + j_2 + \dots + j_k}
            \det {(A({i_1,i_2,\dots,i_k}|{j_1,j_2,\dots,j_k}))}.
        \end{align*}
    \end{quotation}
    则, 我们的目标是:
    对任何正整数 \(k\), \(P(k)\) 是正确的.

    \(k = 1\) 时, 这就是我们证过的 ``按一列展开行列式''.

    现在, 我们假定 \(P(k - 1)\) 是正确的.
    我们要证 \(P(k)\) 也是正确的.
    任取一个 \(n\)~级阵~\(A\) (\(n \geq k\)).
    任取不超过 \(n\) 的正整数
    \(j_1\), \(j_2\), \(\dots\), \(j_k\),
    且 \(j_1 < j_2 < \dots < j_k\).
    按列~\(j_1\) 展开行列式,
    知
    \begin{align*}
        \det {(A)} = \sum_{d_1 = 1}^{n}
        {[A]_{d_1,j_1}
        (-1)^{d_1 + j_1}
        \det {(A(d_1|j_1))}}.
    \end{align*}

    注意到,
    \(A\)~的列~\(j_2\), \(j_3\), \(\dots\), \(j_k\)
    跟 \(A(d_1|j_1)\) 的列~\(j_2 - 1\),
    \(j_3 - 1\), \(\dots\), \(j_k - 1\)
    对应.
    并且, \(A\)~的行~\(i\) (\(i \neq d_1\))
    跟 \(A(d_1|j_1)\)~的行~\(i - \rho(i, d_1)\) 对应.
    利用假定,
    我们按列~\(j_2 - 1\), \(j_3 - 1\), \(\dots\), \(j_k - 1\)
    展开每一个 \(\det {(A(d_1|j_1))}\),
    有
    \begin{align*}
         &
        \det {(A(d_1|j_1))}
        = \sum_{\substack{
        1 \leq d_2 < \dots < d_k \leq n \\
                d_1 \neq d_2,\dots,d_k
            }}
        {\det {\left(
                A\binom{d_2, \dots, d_k}{j_2, \dots, j_k}
                \right)}}
        \\
         &
        \qquad \qquad \qquad \qquad \qquad
        \cdot (-1)^{d_2 - \rho(d_2, d_1) + \dots
            + d_k - \rho(d_k, d_1)
            + j_2 + \dots + j_k - (k - 1)}
        \\
         &
        \qquad \qquad \qquad \qquad \qquad
        \cdot \det
        {(A({d_1,d_2,\dots,d_k}|{j_1,j_2,\dots,j_k}))}.
    \end{align*}
    利用分配律 (还有加法的结合律与交换律), 有
    \begin{align*}
         &
        \det {(A)}
        = \sum_{\substack{
        1 \leq d_1 \leq n               \\
        1 \leq d_2 < \dots < d_k \leq n \\
                d_1 \neq d_2,\dots,d_k
            }}
        {[A]_{d_1,j_1}
            (-1)^{d_1 + j_1}
            \det {\left(
                A\binom{d_2, \dots, d_k}{j_2, \dots, j_k}
                \right)}}
        \\
         &
        \qquad \qquad \qquad \qquad
        \cdot (-1)^{d_2 - \rho(d_2, d_1) + \dots
            + d_k - \rho(d_k, d_1)
            + j_2 + \dots + j_k - (k - 1)}
        \\
         &
        \qquad \qquad \qquad \qquad
        \cdot \det {(A({d_1,d_2,\dots,d_k}|{j_1,j_2,\dots,j_k}))}.
    \end{align*}
    注意到
    \begin{align*}
             &
        d_1 + j_1 + d_2 - \rho(d_2, d_1) + \dots
        + d_k - \rho(d_k, d_1)
        + j_2 + \dots + j_k - (k - 1)
        \\
        = {} &
        \hphantom{{} + {}}
        (d_1 + \dots + d_k) + (j_1 + \dots + d_k)
        \\
             &
        + (\rho(d_1, d_2) - 1) + \dots + (\rho(d_1, d_k) - 1)
        - (k - 1)
        \\
        = {} &
        (d_1 + \dots + d_k) + (j_1 + \dots + j_k)
        + (\rho(d_1, d_2) + \dots + \rho(d_1, d_k))
        - 2(k-1),
    \end{align*}
    故
    \begin{align*}
         &
        \det {(A)}
        = \sum_{\substack{
        1 \leq d_1 \leq n               \\
        1 \leq d_2 < \dots < d_k \leq n \\
                d_1 \neq d_2,\dots,d_k
            }}
        {(-1)^{\rho(d_1, d_2) + \dots + \rho(d_1, d_k)}
                [A]_{d_1,j_1}
            \det {\left(
                A\binom{d_2, \dots, d_k}{j_2, \dots, j_k}
                \right)}}
        \\
         &
        \qquad \qquad \qquad
        \cdot (-1)^{d_1 + \dots + d_k + j_1 + \dots + j_k}
        \det {(A({d_1,d_2,\dots,d_k}|{j_1,j_2,\dots,j_k}))}.
    \end{align*}

    \begingroup

    \makeatletter
    % https://tex.stackexchange.com/questions/198771/align-in-substack
    \providecommand*{\subalign}[1]{%
        \vcenter{%
            \Let@ \restore@math@cr \default@tag
            \baselineskip\fontdimen10 \scriptfont\tw@
            \advance\baselineskip\fontdimen12 \scriptfont\tw@
            \lineskip\thr@@\fontdimen8 \scriptfont\thr@@
            \lineskiplimit\lineskip
            \ialign{\hfil\(\m@th\scriptstyle##\)&%
                \(\m@th\scriptstyle{}##\)\hfil\crcr
                #1\crcr
            }%
        }%
    }
    \makeatother

    注意到
    \begin{align*}
        \sum_{\substack{
        1 \leq d_1 \leq n                              \\
        1 \leq d_2 < \dots < d_k \leq n                \\
                d_1 \neq d_2,\dots,d_k
            }} {\dots}
        =
        \sum_{1 \leq i_1 < i_2 < \dots < i_k \leq n}
        { \sum_{1 \leq s \leq k}
        { {({\dots})}\bigg|_{
        \subalign{
         & d_1 = i_s                                   \\
         & d_r = i_{r - \rho(s, r)}\,(2 \leq r \leq k)
        }
        }}}.
    \end{align*}
    % 负责地, 我来解释此式.
    我解释此式吧.
    要从 \(1\) 至 \(n\) 这 \(n\)~个整数中选出 \(k\)~个数
    \(d_1\), \(d_2\), \(\dots\), \(d_k\),
    适合条件 \(d_2 < \dots < d_k\),
    且 \(d_1\) 不跟 \(d_2\), \(\dots\), \(d_k\) 的任何一个相等,
    我们可以这样.
    先从 \(1\), \(2\), \(\dots\), \(n\) 里\emph{从小到大地}%
    挑 \(k\)~个数 \(i_1\), \(i_2\), \(\dots\), \(i_k\).
    然后, 我们从 \(i_1\), \(i_2\), \(\dots\), \(i_k\)
    选第~\(s\) \emph{小}的数 \(i_s\) 为 \(d_1\),
    再分别取 \(d_2\), \(d_3\), \(\dots\), \(d_k\)
    为
    \(i_1\), \(\dots\), \(i_{s-1}\),
    \(i_{s+1}\), \(\dots\), \(i_{k}\).
    (若 \(s = 1\), 则不出现 \(i_1\);
    若 \(s = k\), 则不出现 \(i_k\).
    下同.)
    更具体地, 我们使 \(d_1\) 为 \(i_s\),
    再使 \(d_r\) (\(r \geq 2\)) 为 \(i_{\ell (r)}\),
    其中 \(2 \leq r \leq s\) 时 \(\ell (r) = r-1\),
    而 \(r > s\) 时 \(\ell (r) = r\).
    利用 \(\rho\)-记号, 就是 \(d_1 = i_s\),
    且 \(d_r = i_{r - \rho(s, r)}\) (\(2 \leq r \leq k\)).

    \endgroup

    当 \(d_1 = i_s\),
    且 \(d_r = i_{r - \rho(s, r)}\)
    (\(2 \leq r \leq k\)) 时,
    \begin{align*}
         &
        d_1 = i_s,
        \\
         &
        \rho(d_1, d_2) + \dots + \rho(d_1, d_k)
        = s - 1 = (s + 1) - 2,
        \\
         &
        [A]_{d_1,j_1} = [A]_{i_s,j_1},
        \\
         &
        A\binom{d_2, \dots, d_k}{j_2, \dots, j_k}
        =
        A\binom{i_1, \dots, i_{s-1}, i_{s+1}, \dots, i_k}
        {j_2, \dots, j_k},
        \\
         &
        d_1 + \dots + d_k = i_1 + \dots + i_k,
        \\
         &
        A({d_1,d_2,\dots,d_k}|{j_1,j_2,\dots,j_k})
        =
        A({i_1,i_2,\dots,i_k}|{j_1,j_2,\dots,j_k}).
    \end{align*}
    故
    \begin{align*}
             &
        \det {(A)}
        \\
        = {} &
        \sum_{1 \leq i_1 < i_2 < \dots < i_k \leq n}
        {\sum_{1 \leq s \leq k}
            {(-1)^{s+1} [A]_{i_s,j_1}
                \det {\left(
                    A\binom{i_1, \dots, i_{s-1},
                        i_{s+1}, \dots, i_k}
                    {j_2, \dots, j_k}
                    \right)}}}
        \\
             &
        \qquad \qquad \qquad
        \cdot
        (-1)^{i_1 + \dots + i_k + j_1 + \dots + j_k}
        \det {(A({i_1,i_2,\dots,i_k}|{j_1,j_2,\dots,j_k}))}.
    \end{align*}
    注意到 \(i_1 + \dots + i_k\) 与
    \(A({i_1,i_2,\dots,i_k}|{j_1,j_2,\dots,j_k})\)
    都跟 \(s\) 无关,
    故由分配律 (还有加法的结合律与交换律),
    \begin{align*}
             &
        \det {(A)}
        \\
        = {} &
        \sum_{1 \leq i_1 < i_2 < \dots < i_k \leq n}
        {\left(
            \sum_{1 \leq s \leq k}
            {(-1)^{s+1} [A]_{i_s,j_1}
                \det {\left(
                    A\binom{i_1, \dots, i_{s-1},
                        i_{s+1}, \dots, i_k}
                    {j_2, \dots, j_k}
                    \right)}}
            \right)}
        \\
             &
        \qquad \qquad \qquad
        \cdot
        (-1)^{i_1 + \dots + i_k + j_1 + \dots + j_k}
        \det {(A({i_1,i_2,\dots,i_k}|{j_1,j_2,\dots,j_k}))}.
    \end{align*}
    注意到 \([A]_{i_s,j_1}\) 是
    \(\displaystyle
    A\binom{i_1, \dots, i_k}
    {j_1, \dots, j_k}\)~的
    \((s, 1)\)-元, 故
    \begin{align*}
             & \sum_{1 \leq s \leq k}
        {(-1)^{s+1} [A]_{i_s,j_1}
            \det {\left(
                A\binom{i_1, \dots, i_{s-1}, i_{s+1}, \dots, i_k}
                {j_2, \dots, j_k}
                \right)}}
        \\
        = {} & \det {\left(
            A\binom{i_1, \dots, i_k}
            {j_1, \dots, j_k}
            \right)}.
    \end{align*}
    综上, 我们有
    \begin{align*}
         &
        \det {(A)}
        = \sum_{1 \leq i_1 < i_2 < \dots < i_k \leq n}
        {\det {\left(
                A\binom{i_1, i_2, \dots, i_k}
                {j_1, j_2, \dots, j_k}
                \right)}}
        \\
         &
        \qquad \qquad \qquad
        \cdot (-1)^{i_1 + i_2 + \dots + i_k
            + j_1 + j_2 + \dots + j_k}
        \det {(A({i_1,i_2,\dots,i_k}|{j_1,j_2,\dots,j_k}))}.
    \end{align*}
    所以, \(P(k)\) 是正确的.
    由数学归纳法原理, 待证命题成立.
\end{proof}

\KunAsteriskoEnEnhavtabelo
\section{完全展开行列式}
\SenAsteriskoEnEnhavtabelo

\maldevigalegajxo

前面, 我们研究了按一列展开行列式.
它变大级阵的行列式为一些小级阵的行列式.
现在, 我们考虑 ``完全展开行列式'',
也就是, 用不含行列式的公式表示行列式.

其实, 我给出行列式的定义后,
我们立即算出了小级阵 (不超过 \(4\))
的行列式较具体的公式.
\(1\)~级阵的行列式非常简单, 有 \(1\)~项;
\(2\)~级阵的行列式不难, 有 \(2\)~项;
\(3\)~级阵的行列式较难, 但也不难记, 有 \(6\)~项;
\(4\)~级阵的行列式更复杂了, 不好记 (也不必记),
有 \(24\)~项.
本节, 我们的目标是,
较具体地写出较大级阵的行列式的公式.

\begin{restatable}[]{theorem}{TheoremFullExpansion}
    设 \(A\) 是 \(n\)~级阵 (\(n \geq 1\)).
    设 \(j_1\), \(j_2\), \(\dots\), \(j_n\) 是%
    不超过 \(n\) 的正整数,
    且\emph{互不相同}.
    则
    \begin{align*}
             & \det {(A)}
        \\
        = {} &
        \sum_{\substack{
        1 \leq i_1, i_2, \dots, i_n \leq n \\
                i_1, i_2, \dots, i_n\,\text{互不相同}
            }}
        {s(i_1, i_2, \dots, i_n)\,
            s(j_1, j_2, \dots, j_n)\,
            [A]_{i_1,j_1} [A]_{i_2,j_2} \dots [A]_{i_n,j_n}}.
    \end{align*}
\end{restatable}

\begin{proof}
    我们按列~\(j_1\) 展开 \(\det {(A)}\),
    有
    \begin{align*}
        \det {(A)}
        = \sum_{1 \leq i_1 \leq n}
        {(-1)^{i_1 + j_1} [A]_{i_1,j_1} \det {(A(i_1|j_1))}}.
    \end{align*}

    注意到, \(A\)~的列~\(j_2\) 对应
    \(A(i_1|j_1)\)~的列~\(j_2 - \rho(j_2, j_1)\),
    且 \(A\)~的行~\(i_2\) (\(i_2 \neq i_1\)) 对应
    \(A(i_1|j_1)\)~的行~\(i_2 - \rho(i_2, i_1)\).
    再注意到
    \begin{align*}
        i_2 - \rho(i_2, i_1) + j_2 - \rho(j_2, j_1)
        = i_2 + j_2 + \rho(i_1, i_2) + \rho(j_1, j_2) - 2,
    \end{align*}
    故
    \begin{align*}
             &
        \det {(A(i_1|j_1))}
        \\
        = {} &
        \sum_{\substack{
        1 \leq i_2 \leq n \\
            i_2 \neq i_1
        }}
        {(-1)^{i_2 + j_2 + \rho(i_1, i_2) + \rho(j_1, j_2)}
            [A]_{i_2,j_2} \det {(A({i_1,i_2}|{j_1,j_2}))}}.
    \end{align*}

    \(\dots \dots\)

    注意到, \(A\)~的列~\(j_k\) 对应
    \(A({i_1,\dots,i_{k-1}}|{j_1,\dots,j_{k-1}})\)~的%
    列~\(j_k - \rho(j_k, j_1) - \dots - \rho(j_k, j_{k-1})\),
    且 \(A\)~的行~\(i_k\)
    (\(i_k \neq i_1\), \(\dots\), \(i_{k-1}\)) 对应
    \(A({i_1,\dots,i_{k-1}}|{j_1,\dots,j_{k-1}})\)~的%
    行~\(i_k - \rho(i_k, i_1) - \dots - \rho(i_k, i_{k-1})\).
    再注意到
    \begin{align*}
             &
        i_k - \rho(i_k, i_1) - \dots - \rho(i_k, i_{k-1})
        + j_k - \rho(j_k, j_1) - \dots - \rho(j_k, j_{k-1})
        \\
        = {} &
        i_k + j_k
        + \rho(i_1, i_k) + \dots + \rho(i_{k-1}, i_k)
        + \rho(j_1, j_k) + \dots + \rho(j_{k-1}, j_k)
        - 2(k-1),
    \end{align*}
    故
    \begin{align*}
             &
        \det {(A({i_1,\dots,i_{k-1}}|{j_1,\dots,j_{k-1}}))}
        \\
        = {} &
        \sum_{\substack{
        1 \leq i_k \leq n \\
                i_k \neq i_1, \dots, i_{k-1}
            }}
        {(-1)^{i_k + j_k
                    + \rho(i_1, i_k) + \dots + \rho(i_{k-1}, i_k)
                    + \rho(j_1, j_k) + \dots
                    + \rho(j_{k-1}, j_k)}}
        \\
             &
        \qquad \qquad
        \cdot
        [A]_{i_k,j_k}
        \det {(A({i_1,\dots,i_{k}}|{j_1,\dots,j_{k}}))}.
    \end{align*}

    \(\dots \dots\)

    最后, 我们得到
    \begin{align*}
             &
        \det {(A({i_1,\dots,i_{n-1}}|{j_1,\dots,j_{n-1}}))}
        \\
        = {} &
        \sum_{\substack{
        1 \leq i_n \leq n \\
                i_n \neq i_1, \dots, i_{n-1}
            }}
        {(-1)^{i_n + j_n
                    + \rho(i_1, i_n) + \dots + \rho(i_{n-1}, i_n)
                    + \rho(j_1, j_n) + \dots + \rho(j_{n-1}, j_n)}
                [A]_{i_n,j_n}}.
    \end{align*}
    (其实, 这就是 \([A]_{i_n,j_n}\);
    % 不过, 为了不破坏规律,
    % 我们还是 ``复杂地'' 写.
    % 注意, 我们说过, 我们定义
    % ``零级阵'' 的行列式为 \(1\),
    % 所以我们不写
    % \(\det {(A({i_1,\dots,i_{n}}|{j_1,\dots,j_{n}}))}\).
    这里, 我们要注意到
    \begin{align*}
             & 1 + 1
        \\
        = {} &
        i_n - \rho(i_n, i_1) - \dots - \rho(i_n, i_{n-1})
        + j_n - \rho(j_n, j_1) - \dots - \rho(j_n, j_{n-1})
        \\
        = {} &
        i_n + j_n
        + \rho(i_1, i_n) + \dots + \rho(i_{n-1}, i_n)
        + \rho(j_1, j_n) + \dots + \rho(j_{n-1}, j_n)
        - 2(n-1),
    \end{align*}
    故
    \begin{align*}
        1 = (-1)^{1+1}
        = (-1)^{i_n + j_n
                + \rho(i_1, i_n) + \dots + \rho(i_{n-1}, i_n)
                + \rho(j_1, j_n) + \dots + \rho(j_{n-1}, j_n)}.
    \end{align*}
    您若无印象了, 则可看本章, 节~\sekcio{1},
    \malneprasekcio{2},
    \sekcio{3},
    \malneprasekcio{4}.)

    我们从后往前地代入, 有
    \begin{align*}
             & \det {(A)}
        \\
        = {} & \sum_{\substack{
        1 \leq i_1, i_2, \dots, i_n \leq n \\
                i_1, i_2, \dots, i_n\,\text{互不相同}
            }}
        {(-1)^{i_1 + j_1 + i_2 + j_2 + \dots + i_n + j_n
                    + \tau(i_1, i_2, \dots, i_n)
                    + \tau(j_1, j_2, \dots, j_n)}}
        \\
             &
        \qquad \qquad \qquad
        \cdot
        [A]_{i_1,j_1} [A]_{i_2,j_2} \dots [A]_{i_n,j_n}.
    \end{align*}
    最后, 注意到
    \begin{align*}
        i_1 + j_1 + i_2 + j_2 + \dots + i_n + j_n
        = 2(1 + 2 + \dots + n)
    \end{align*}
    是偶数,
    \(s\)-记号跟 \(\tau\)-记号的关系,
    以及 \(-1\) 的整数次方的性质,
    即得
    \begin{align*}
             & \det {(A)}
        \\
        = {} &
        \sum_{\substack{
        1 \leq i_1, i_2, \dots, i_n \leq n \\
                i_1, i_2, \dots, i_n\,\text{互不相同}
            }}
        {s(i_1, i_2, \dots, i_n)\,
            s(j_1, j_2, \dots, j_n)\,
            [A]_{i_1,j_1} [A]_{i_2,j_2} \dots [A]_{i_n,j_n}}.
    \end{align*}
\end{proof}

我们经常取 \(j_1\), \(j_2\), \(\dots\), \(j_n\)
为 \(1\), \(2\), \(\dots\), \(n\).
注意到 \(s(1, 2, \dots, n) = 1\),
即得
\begin{align*}
    \det {(A)}
    = {} &
    \sum_{\substack{
    1 \leq i_1, i_2, \dots, i_n \leq n \\
            i_1, i_2, \dots, i_n\,\text{互不相同}
        }}
    {s(i_1, i_2, \dots, i_n)\,
        [A]_{i_1,1} [A]_{i_2,2} \dots [A]_{i_n,n}}.
\end{align*}

\begin{example}
    设 \(A\) 为 \(3\)~级阵.
    那么, 适合条件
    ``\(1 \leq i_1, i_2, i_3 \leq 3\),
    且 \(i_1\), \(i_2\), \(i_3\) 互不相同'' 的
    \((i_1, i_2, i_3)\) 恰有 \(6\) 组:
    \begin{align*}
         & (1, 2, 3), \quad (2, 3, 1), \quad (3, 1, 2), \\
         & (1, 3, 2), \quad (2, 1, 3), \quad (3, 2, 1).
    \end{align*}
    回想
    \begin{align*}
        s(i_1, i_2, i_3)
        = \operatorname{sgn} {(i_2 - i_1)}
        \cdot \operatorname{sgn} {(i_3 - i_1)}
        \cdot \operatorname{sgn} {(i_3 - i_2)}.
    \end{align*}
    不难算出,
    \begin{align*}
         & s(1, 2, 3) = s(2, 3, 1) = s(3, 1, 2) = 1,  \\
         & s(1, 3, 2) = s(2, 1, 3) = s(3, 2, 1) = -1.
    \end{align*}
    故
    \begin{align*}
        \det {(A)}
        = {} &
        \hphantom{{} + {}}
        [A]_{1,1} [A]_{2,2} [A]_{3,3}
        + [A]_{2,1} [A]_{3,2} [A]_{1,3}
        + [A]_{3,1} [A]_{1,2} [A]_{2,3}
        \\
             &
        - [A]_{1,1} [A]_{3,2} [A]_{2,3}
        - [A]_{2,1} [A]_{1,2} [A]_{3,3}
        - [A]_{3,1} [A]_{2,2} [A]_{1,3}.
    \end{align*}
\end{example}

利用完全展开,
我们不难看出,
\(n\)~级阵的行列式有
\(n \cdot (n - 1) \cdot \dots \cdot 2 \cdot 1\)~项.
毕竟, \(1 \leq i_1, i_2, \dots, i_n \leq n\),
且 \(i_1\), \(i_2\), \(\dots\), \(i_n\) 互不相同%
相当于 \(i_1\), \(i_2\), \(\dots\), \(i_n\)
是 \(1\), \(2\), \(\dots\), \(n\) 的排列.
我们知道, \(n\)~个互不相同的文字共有
\(n \cdot (n - 1) \cdot \dots \cdot 2 \cdot 1\)~种%
排列.

% 乘热打铁地,
% 我要给出一个在某些场合有奇用的公式.

% 不过, 我们要一个新的记号;
% 而, 这个新的记号,
% 在学习必学内容时, 也是要用到的.
% 故, 我决定, ``广告之后, 精彩继续''.

\section{阵的一个新记号}

% 我们暂停行列式的讨论.
% 我要介绍阵的新写法.
本节, 我们学习阵的一个新记号.
它可使我们更好地理解阵与行列式.

\begin{definition}
    设 \(c_1\), \(c_2\), \(\dots\), \(c_n\)
    是 \(n\)~个 \(m \times 1\)~阵.
    定义
    \(
    \begin{bmatrix}
        c_1 & c_2 & \cdots & c_n \\
    \end{bmatrix}
    \)
    是一个 \(m \times n\) 阵,
    其中, 对任何不超过 \(m\) 的正整数 \(i\)
    与任何不超过 \(n\) 的正整数 \(j\),
    \begin{align*}
        \left[\,
            \begin{bmatrix}
                c_1 & c_2 & \cdots & c_n \\
            \end{bmatrix}
            \,\right]_{i,j}
        = [c_j]_{i,1}.
    \end{align*}
    (通俗地,
    \(
    \begin{bmatrix}
        c_1 & c_2 & \cdots & c_n \\
    \end{bmatrix}
    \)
    就是合 \(n\)~个 \(m \times 1\) ``小阵''
    \(c_1\), \(c_2\), \(\dots\), \(c_n\)
    为一个 \(m \times n\) ``大阵''
    的结果.)

    习惯地, 我们可写
    \(
    \begin{bmatrix}
        c_1 & c_2 & \cdots & c_n \\
    \end{bmatrix}
    \)
    为 \([c_1, c_2, \dots, c_n]\).
\end{definition}

这种写法, 自然强调了阵的列.
之后, 在研究行列式的性质时, 此写法十分有用.

\begin{example}
    设
    \(c_1 =  \begin{bmatrix}
        1 \\2\\3\\
    \end{bmatrix}\),
    \(c_2 =  \begin{bmatrix}
        4 \\5\\6\\
    \end{bmatrix}\),
    \(c_3 =  \begin{bmatrix}
        7 \\8\\9\\
    \end{bmatrix}\),
    \(c_4 =  \begin{bmatrix}
        10 \\11\\12\\
    \end{bmatrix}\).
    则
    \begin{align*}
        [c_1, c_2, c_3, c_4]
        =
        \begin{bmatrix}
            1 & 4 & 7 & 10 \\
            2 & 5 & 8 & 11 \\
            3 & 6 & 9 & 12 \\
        \end{bmatrix}.
    \end{align*}
\end{example}

论证下面的命题可被认为是一个好习题.

\begin{theorem}
    设
    \(c_1\), \(c_2\), \(\dots\), \(c_n\),
    \(d_1\), \(d_2\), \(\dots\), \(d_n\)
    是 \(2n\) 个 \(m \times 1\)~阵.
    设 \(k\) 是数.
    则
    \begin{align*}
         & [c_1, c_2, \dots, c_n] + [d_1, d_2, \dots, d_n]
        = [c_1 + d_1, c_2 + d_2, \dots, c_n + d_n],        \\
         & k[c_1, c_2, \dots, c_n]
        = [kc_1, kc_2, \dots, kc_n].
    \end{align*}
\end{theorem}

\begin{proof}
    我证式~1.
    论证式~2 比论证式~1 更简单,
    故我留它为您的习题.

    注意到 \([c_1, c_2, \dots, c_n]\)
    与 \([d_1, d_2, \dots, d_n]\)
    都是 \(m \times n\)~阵,
    它们自然可加,
    结果也是 \(m \times n\)~阵.
    并且, \(c_j\), \(d_j\) 都是 \(m \times 1\)~阵,
    故 \(c_j + d_j\) 也是.
    不说元是否相等, 至少等式二侧的阵的尺寸是相等的.
    现在, 比较元是否相等:
    \begin{align*}
             & [\, [c_1, c_2, \dots, c_n]
        + [d_1, d_2, \dots, d_n] \,]_{i,j}          \\
        = {} & [\, [c_1, c_2, \dots, c_n] \,]_{i,j}
        + [\, [d_1, d_2, \dots, d_n] \,]_{i,j}      \\
        = {} & [c_j]_{i,1} + [d_j]_{i,1}            \\
        = {} & [c_j + d_j]_{i,1}                    \\
        = {} & [\, [c_1 + d_1, c_2 + d_2, \dots,
                        c_n + d_n] \,]_{i,j}.
    \end{align*}
    看来, 元也相等.

    证明的要点有且只有一个: 用定义.
\end{proof}

设 \(c_1\), \(c_2\), \(\dots\), \(c_n\) 是
\(n\)~个 \(n \times 1\)~阵.
那么, \([c_1, c_2, \dots, c_n]\)
自然是一个 \(n\)~级阵,
从而有行列式
\(\det {([c_1, c_2, \dots, c_n])}\).
不难看到, 这里有 \(2\)~重括号.
习惯地, 我们写
\begin{align*}
    \det {([c_1, c_2, \dots, c_n])}
    =
    \det {[c_1, c_2, \dots, c_n]}.
\end{align*}
不写 \(( \ )\) 似乎并不会引起误解,
所以, 这是没问题的.
% (其实, 我干过类似的事;
% 若您细心, 您一定能发现它.)

\begin{example}
    设
    \(c_1 =  \begin{bmatrix}
        1 \\2\\3\\
    \end{bmatrix}\),
    \(c_2 =  \begin{bmatrix}
        5 \\6\\4\\
    \end{bmatrix}\),
    \(c_3 =  \begin{bmatrix}
        9 \\7\\8\\
    \end{bmatrix}\).
    则
    \begin{align*}
        \det {[c_1, c_2, c_3]}
        = {} &
        1 \cdot 6 \cdot 8
        + 2 \cdot 4 \cdot 9
        + 3 \cdot 5 \cdot 7
        - 1 \cdot 4 \cdot 7
        - 2 \cdot 5 \cdot 8
        - 3 \cdot 6 \cdot 9
        \\
        = {} &
        48 + 72 + 105 - 28 - 80 - 162
        \\
        = {} & {-45}.
    \end{align*}
\end{example}

前面, 我们合 ``竖排的阵'' 为一个大阵.
我们当然也可合 ``横排的阵'' 为一个大阵.

\begin{definition}
    设 \(r_1\), \(r_2\), \(\dots\), \(r_m\)
    是 \(m\)~个 \(1 \times n\)~阵.
    定义
    \(
    \begin{bmatrix}
        r_1    \\
        r_2    \\
        \vdots \\
        r_m    \\
    \end{bmatrix}
    \)
    是一个 \(m \times n\) 阵,
    其中, 对任何不超过 \(m\) 的正整数 \(i\)
    与任何不超过 \(n\) 的正整数 \(j\),
    \begin{align*}
        \left[\,\begin{bmatrix}
                        r_1    \\
                        r_2    \\
                        \vdots \\
                        r_m    \\
                    \end{bmatrix}\,\right]_{\,i,j}
        = [r_i]_{1,j}.
    \end{align*}
    (通俗地,
    \(
    \begin{bmatrix}
        r_1    \\
        r_2    \\
        \vdots \\
        r_m    \\
    \end{bmatrix}
    \)
    就是合 \(m\)~个 \(1 \times n\) ``小阵''
    \(r_1\), \(r_2\), \(\dots\), \(r_m\)
    为一个 \(m \times n\) ``大阵''
    的结果.)
\end{definition}

完全类似地, 我们有

\begin{theorem}
    设
    \(r_1\), \(r_2\), \(\dots\), \(r_m\),
    \(s_1\), \(s_2\), \(\dots\), \(s_m\)
    是 \(2m\) 个 \(1 \times n\)~阵.
    设 \(k\) 是数.
    则
    \begin{align*}
         & \begin{bmatrix}
               r_1    \\
               r_2    \\
               \vdots \\
               r_m    \\
           \end{bmatrix}
        + \begin{bmatrix}
              s_1    \\
              s_2    \\
              \vdots \\
              s_m    \\
          \end{bmatrix}
        = \begin{bmatrix}
              r_1 + s_1 \\
              r_2 + s_2 \\
              \vdots    \\
              r_m + s_m \\
          \end{bmatrix},   \\
         & k\begin{bmatrix}
                r_1    \\
                r_2    \\
                \vdots \\
                r_m    \\
            \end{bmatrix}
        = \begin{bmatrix}
              kr_1   \\
              kr_2   \\
              \vdots \\
              kr_m   \\
          \end{bmatrix}.
    \end{align*}
\end{theorem}

\begin{proof}
    完全类似.
\end{proof}

\KunAsteriskoEnEnhavtabelo
\section{完全展开行列式 (续)}
\SenAsteriskoEnEnhavtabelo

\maldevigalegajxo

% 我说, 我要给出一个在某些场合有奇用的公式.
% 不过, 为方便说话, 我 ``插入了广告''.
% 现在, ``精彩继续''.

设 \(A\) 是一个 \(n\)~级阵.
交换 \(A\)~的列的次序, 得 \(n\)~级阵~\(B\).

利用完全展开, 我们可得 \(A\), \(B\) 的行列式的关系.

\begin{restatable}[]{theorem}{TheoremColumnSwapAndSign}
    设 \(n\)~级阵 \(A\) 的列~\(1\), \(2\), \(\dots\), \(n\)
    分别为 \(a_1\), \(a_2\), \(\dots\), \(a_n\).
    设 \(\ell_1\), \(\ell_2\), \(\dots\), \(\ell_n\) 是%
    不超过 \(n\) 的正整数,
    且\emph{互不相同}.
    % (换句话说,
    % \(\ell_1\), \(\ell_2\), \(\dots\), \(\ell_n\) 是
    % \(1\), \(2\), \(\dots\), \(n\) 的排列).
    则
    \begin{align*}
        \det {[a_{\ell_1}, a_{\ell_2}, \dots, a_{\ell_n}]}
        = s(\ell_1, \ell_2, \dots, \ell_n) \det {(A)}.
    \end{align*}
\end{restatable}

不过, 论证此事前, 我想用一个例助您理解,
此定理在说什么.

\begin{example}
    设
    \(a_1 =  \begin{bmatrix}
        1 \\2\\3\\
    \end{bmatrix}\),
    \(a_2 =  \begin{bmatrix}
        5 \\6\\4\\
    \end{bmatrix}\),
    \(a_3 =  \begin{bmatrix}
        9 \\7\\8\\
    \end{bmatrix}\).
    则
    \begin{align*}
        \det {[a_1, a_2, a_3]}
        = {} &
        1 \cdot 6 \cdot 8
        + 2 \cdot 4 \cdot 9
        + 3 \cdot 5 \cdot 7
        - 1 \cdot 4 \cdot 7
        - 2 \cdot 5 \cdot 8
        - 3 \cdot 6 \cdot 9
        \\
        = {} &
        48 + 72 + 105 - 28 - 80 - 162
        \\
        = {} & {-45}.
    \end{align*}

    取 \(\ell_1\), \(\ell_2\), \(\ell_3\) 为 \(2\), \(3\), \(1\).
    则 \(s(2, 3, 1) = 1\).
    不难算出
    \begin{align*}
        \det {[a_2, a_3, a_1]}
        = {} &
        5 \cdot 7 \cdot 3
        + 6 \cdot 8 \cdot 1
        + 4 \cdot 9 \cdot 2
        - 5 \cdot 8 \cdot 2
        - 6 \cdot 9 \cdot 3
        - 4 \cdot 7 \cdot 1
        \\
        = {} &
        105 + 48 + 72 - 80 - 162 - 28
        \\
        = {} & {-45}.
    \end{align*}
    这就是 \(\det {[a_1, a_2, a_3]}\).

    再取 \(\ell_1\), \(\ell_2\), \(\ell_3\)
    为 \(1\), \(3\), \(2\).
    则 \(s(1, 3, 2) = -1\).
    不难算出
    \begin{align*}
        \det {[a_1, a_3, a_2]}
        = {} &
        1 \cdot 7 \cdot 4
        + 2 \cdot 8 \cdot 5
        + 3 \cdot 9 \cdot 6
        - 1 \cdot 8 \cdot 6
        - 2 \cdot 9 \cdot 4
        - 3 \cdot 7 \cdot 5
        \\
        = {} &
        28 + 80 + 162 - 48 - 72 - 105
        \\
        = {} & 45.
    \end{align*}
    这就是 \(-\det {[a_1, a_2, a_3]}\).
\end{example}

\begin{proof}
    作 \(n\)~级阵
    \(
    B = [a_{\ell_1}, a_{\ell_2}, \dots, a_{\ell_n}].
    \)
    注意到, \([B]_{u,v} = [a_{\ell_v}]_{u,1}
        = [A]_{u,\ell_v}\).

    我们完全展开 \(\det {(B)}\).
    取 \(j_1\), \(j_2\), \(\dots\), \(j_n\)
    为 \(1\), \(2\), \(\dots\), \(n\),
    并注意到 \(s(1, 2, \dots, n) = 1\),
    有
    \begin{align*}
        \det {(B)}
        = {} &
        \sum_{\substack{
        1 \leq i_1, i_2, \dots, i_n \leq n \\
                i_1, i_2, \dots, i_n\,\text{互不相同}
            }}
        {s(i_1, i_2, \dots, i_n)\,
            [B]_{i_1,1} [B]_{i_2,2}
            \dots [B]_{i_n,n}}.
    \end{align*}
    我们再完全展开 \(\det {(A)}\).
    取 \(j_1\), \(j_2\), \(\dots\), \(j_n\)
    为 \(\ell_1\), \(\ell_2\), \(\dots\), \(\ell_n\),
    有
    \begin{align*}
             & \det {(A)}
        \\
        = {} &
        s(\ell_1, \ell_2, \dots, \ell_n)\,
        \sum_{\substack{
        1 \leq i_1, i_2, \dots, i_n \leq n      \\
                i_1, i_2, \dots, i_n\,\text{互不相同}
            }}
        {s(i_1, i_2, \dots, i_n)\,
            [A]_{i_1,\ell_1} [A]_{i_2,\ell_2}
            \dots [A]_{i_n,\ell_n}}
        \\
        = {} &
        s(\ell_1, \ell_2, \dots, \ell_n)\,
        \sum_{\substack{
        1 \leq i_1, i_2, \dots, i_n \leq n      \\
                i_1, i_2, \dots, i_n\,\text{互不相同}
            }}
        {s(i_1, i_2, \dots, i_n)\,
            [B]_{i_1,1} [B]_{i_2,2}
            \dots [B]_{i_n,n}}
        \\
        = {} & s(\ell_1, \ell_2, \dots, \ell_n)
        \det {(B)}.
    \end{align*}
    注意到 \(s(\ell_1, \ell_2, \dots, \ell_n) = \pm 1\),
    故
    \begin{equation*}
        \det {(B)}
        = s(\ell_1, \ell_2, \dots, \ell_n)
        \det {(A)}.
        \qedhere
    \end{equation*}
\end{proof}

\section{单位阵}

在进一步讨论行列式前,
我要介绍一类特别的方阵.

\begin{definition}[\(\delta\)-记号]
    设 \(i\), \(j\) 是二个文字.
    定义
    \begin{align*}
        \delta(i, j)
        = \begin{cases}
              0, & i \neq j; \\
              1, & i = j.
          \end{cases}
    \end{align*}
\end{definition}

显然, \(\delta\)-记号是表示二个文字是否相等的一个量.
\(\delta(i, j) = 1\), 说明 \(i\), \(j\) 是同一个文字;
\(\delta(i, j) = 0\), 说明 \(i\), \(j\) 是不同的二个文字.

\begin{definition}[单位阵]
    设 \(n\) 为正整数.
    作 \(n\)~级阵 \(I_n\),
    使 \([I_n]_{i,j} = \delta(i, j)\),
    (\(1 \leq i, j \leq n\)).
    我们说, \(I_n\) 是 \(n\)~级单位阵.

    当我们不强调单位阵的尺寸时,
    我们可简单地写单位阵为 \(I\).
\end{definition}

\begin{example}
    \(1\)~级单位阵是 \(1\), 或 \([1]\).
    \(2\)~级单位阵是
    \(
    \begin{bmatrix}
        1 & 0 \\
        0 & 1 \\
    \end{bmatrix}.
    \)
    \(3\)~级单位阵是
    \(
    \begin{bmatrix}
        1 & 0 & 0 \\
        0 & 1 & 0 \\
        0 & 0 & 1 \\
    \end{bmatrix}.
    \)

    一般地, \(n\)~级单位阵含 \(n\)~个 \(1\)
    与 \(n^2 - n\)~个 \(0\),
    且 \(1\) \emph{恰在}行号与列号相等的地方
    (也就是说, \(0\) \emph{恰在}行号与列号不等的地方).
\end{example}

\emph{可}写一个 \(n \times 1\)~阵%
为 \(n\)~级单位阵 \(I_n\)~的列的数乘的和.
具体地, 我们设 \(x\) 是一个 \(n \times 1\)~阵,
且设 \(I_n = [e_1, e_2, \dots, e_n]\)
(也就是说,
设 \(e_1\), \(e_2\), \(\dots\), \(e_n\) 是
\(I_n\) 的列 \(1\), \(2\), \(\dots\), \(n\)).
记 \(k_i = [x]_{i,1}\).
则
\begin{align*}
    x
    = {} & k_1 e_1 + k_2 e_2 + \dots + k_n e_n
    \\
    = {} & \sum_{\ell = 1}^{n} {k_\ell e_\ell}.
\end{align*}
为证明此式, 我们要用%
单位阵的定义、加法的定义、数乘的定义
(不难看出, 等式二侧的尺寸都是 \(n \times 1\)):
\begin{align*}
    \left[ \sum_{\ell = 1}^{n} {k_\ell e_\ell} \right]_{i,1}
    = {} &
    \sum_{\ell = 1}^{n} {[k_\ell e_\ell]_{i,1}}
    \\
    = {} &
    \sum_{\ell = 1}^{n} {k_\ell [e_\ell]_{i,1}}
    \\
    = {} &
    \sum_{\ell = 1}^{n} {k_\ell [I_n]_{i,\ell}}
    \\
    = {} &
    \sum_{\ell = 1}^{n} {k_\ell \delta(i, \ell)}
    \\
    = {} &
    k_i \delta(i, i)
    + \sum_{\substack{1 \leq \ell \leq n \\ \ell \neq i}}
    {k_\ell \delta(i, \ell)}
    \\
    = {} &
    k_i
    + \sum_{\substack{1 \leq \ell \leq n \\ \ell \neq i}}
    {k_\ell 0}
    \\
    = {} &
    k_i.
\end{align*}

\emph{至多}以一种方式写一个 \(n \times 1\) 阵%
为 \(n\)~级单位阵 \(I_n\)~的列的数乘的和.
具体地, 设 \(e_1\), \(e_2\), \(\dots\), \(e_n\) 是
\(I_n\) 的列 \(1\), \(2\), \(\dots\), \(n\).
再设
\begin{align*}
    \sum_{\ell = 1}^{n} {k_\ell e_\ell}
    = \sum_{\ell = 1}^{n} {p_\ell e_\ell}.
\end{align*}
% 仿照前面的计算, 可知
不难算出
\begin{align*}
     & \sum_{\ell = 1}^{n} {k_\ell e_\ell}
    = \begin{bmatrix}
          k_1    \\
          \vdots \\
          k_n
      \end{bmatrix},                       \\
     & \sum_{\ell = 1}^{n} {p_\ell e_\ell}
    = \begin{bmatrix}
          p_1    \\
          \vdots \\
          p_n
      \end{bmatrix}.
\end{align*}
从而
\begin{align*}
    \begin{bmatrix}
        k_1    \\
        \vdots \\
        k_n
    \end{bmatrix}
    = \begin{bmatrix}
          p_1    \\
          \vdots \\
          p_n
      \end{bmatrix}.
\end{align*}
由此可知, \(k_i = p_i\).

综上, 我们有

\begin{theorem}
    可以, 且只能以一种方式写一个
    \(n \times 1\)~阵为 \(n\)~级单位阵 \(I_n\)~的%
    列的数乘的和.
\end{theorem}

\begin{example}
    设
    \(
    x = \begin{bmatrix}
        3 \\ 5 \\ 7 \\
    \end{bmatrix}.
    \)
    于是, 一方面,
    \begin{align*}
        x =
        3\begin{bmatrix}
             1 \\ 0 \\ 0 \\
         \end{bmatrix}
        + 5\begin{bmatrix}
               0 \\ 1 \\ 0 \\
           \end{bmatrix}
        + 7\begin{bmatrix}
               0 \\ 0 \\ 1 \\
           \end{bmatrix}.
    \end{align*}

    另一方面, 我们设
    \begin{align*}
        x =
        k_1 \begin{bmatrix}
                1 \\ 0 \\ 0 \\
            \end{bmatrix}
        + k_2 \begin{bmatrix}
                  0 \\ 1 \\ 0 \\
              \end{bmatrix}
        + k_3 \begin{bmatrix}
                  0 \\ 0 \\ 1 \\
              \end{bmatrix}.
    \end{align*}
    则
    \begin{align*}
        \begin{bmatrix}
            3 \\ 5 \\ 7 \\
        \end{bmatrix}
        =
        \begin{bmatrix}
            k_1 \\ 0 \\ 0 \\
        \end{bmatrix}
        + \begin{bmatrix}
              0 \\ k_2 \\ 0 \\
          \end{bmatrix}
        + \begin{bmatrix}
              0 \\ 0 \\ k_3 \\
          \end{bmatrix}
        = \begin{bmatrix}
              k_1 \\ k_2 \\ k_3 \\
          \end{bmatrix}.
    \end{align*}
    由此可知,
    \(k_1\), \(k_2\), \(k_3\)
    一定分别是
    \(3\), \(5\), \(7\).
\end{example}

顺便, 您可用完全类似的手法证明

\begin{theorem}
    可以, 且只能以一种方式写一个
    \(1 \times n\)~阵为 \(n\)~级单位阵 \(I_n\)~的%
    行的数乘的和.
\end{theorem}

\section{行列式的性质}

本节, 我们学习行列式的几条重要的性质.

\begin{theorem}[规范性]
    单位阵的行列式为 \(1\).
\end{theorem}

\begin{proof}
    我们用数学归纳法证明此事.
    具体地, 设 \(P(n)\) 为命题
    \begin{align*}
        \det {(I_n)} = 1.
    \end{align*}
    则, 我们的目标是:
    对任何正整数 \(n\), \(P(n)\) 是正确的.

    \(P(1)\) 显然是正确的.
    我就不解释原因了; 您解释它.

    现在, 我们假定 \(P(m-1)\) 是正确的.
    我们要证 \(P(m)\) 也是正确的;
    也就是说, \(\det {(I_m)} = 1\).
    按列~\(1\) 展开, 有
    \begin{align*}
             & \det {(I_m)}
        \\
        = {} & \sum_{i = 1}^{m}
        {(-1)^{i+1} [I_m]_{i,1} \det {(I_m (i|1))}}
        \\
        % = {} & \sum_{i = 1}^{m}
        % {(-1)^{i+1} \delta(i, 1) \det {(I_m (i|1))}}
        % \\
        = {} & (-1)^{1+1} \delta(1, 1) \det {(I_m (1|1))}
        + \sum_{i = 2}^{m}
        {(-1)^{i+1} \delta(i, 1) \det {(I_m (i|1))}}
        \\
        = {} & \det {(I_{m-1})}
        + \sum_{i = 2}^{m}
        {(-1)^{i+1} 0 \det {(I_m (i|1))}}
        \\
        = {} & \det {(I_{m-1})}
        \\
        = {} & 1.
    \end{align*}
    所以, \(P(m)\) 是正确的.
    由数学归纳法原理, 待证命题成立.
\end{proof}

我接下来要讲的二个性质的论证利用了%
按 (任何) 一列展开行列式%
的公式.
% 为方便您体会它,
请允许我引用此公式:

\TheoremExpansionAboutAnyColumn*

\begin{theorem}[多线性]
    行列式 (关于列) 是多线性的.
    具体地, 对任何不超过 \(n\) 的正整数 \(j\),
    任何 \(n-1\)~个 \(n \times 1\)~阵
    \(a_1\), \(\dots\), \(a_{j-1}\),
    \(a_{j+1}\), \(\dots\), \(a_n\),
    任何二个 \(n \times 1\)~阵 \(x\), \(y\),
    任何二个数 \(s\), \(t\),
    有
    \begin{align*}
             & \det
        {[a_1, \dots, a_{j-1}, sx + ty, a_{j+1}, \dots, a_n]}
        \\
        = {} &
        s
        \det {[a_1, \dots, a_{j-1}, x, a_{j+1}, \dots, a_n]}
        +
        t
        \det {[a_1, \dots, a_{j-1}, y, a_{j+1}, \dots, a_n]}.
    \end{align*}
    (若 \(j = 1\), 则 \(a_1\) 不出现;
    若 \(j = n\), 则 \(a_n\) 不出现.
    下同.)
\end{theorem}

\begin{proof}
    作三个 \(n\)~级阵 \(A\), \(B\), \(C\):
    \(A\), \(B\), \(C\)~的列~\(k\) 为 \(a_k\) (\(k \neq j\));
    \(A\)~的列~\(j\) 为 \(x\);
    \(B\)~的列~\(j\) 为 \(y\);
    \(C\)~的列~\(j\) 为 \(sx + ty\).
    那么,
    \begin{align*}
         & \det {(A)}
        = \det
        {[a_1, \dots, a_{j-1}, x, a_{j+1}, \dots, a_n]},
        \\
         & \det {(B)}
        = \det
        {[a_1, \dots, a_{j-1}, y, a_{j+1}, \dots, a_n]},
        \\
         & \det {(C)}
        = \det
        {[a_1, \dots, a_{j-1}, sx + ty, a_{j+1}, \dots, a_n]}.
    \end{align*}
    不难发现, 若 \(k \neq j\), 则
    \(
    [A]_{i,k} = [B]_{i,k} = [C]_{i,k},
    \)
    故
    \(
    A(i|j) = B(i|j) = C(i|j).
    \)
    再注意到
    \(
    [C]_{i,j} = s[A]_{i,j} + t[B]_{i,j},
    \)
    得
    \begin{align*}
             & \det {(C)}
        \\
        = {} &
        \sum_{i = 1}^{n} {
        (-1)^{i+j} [C]_{i,j} \det {(C(i|j))}
        }
        \\
        = {} &
        \sum_{i = 1}^{n} {
        (-1)^{i+j} (s[A]_{i,j} + t[B]_{i,j}) \det {(C(i|j))}
        }
        \\
        = {} &
        s\sum_{i = 1}^{n} {
        (-1)^{i+j} [A]_{i,j} \det {(C(i|j))}
        }
        +
        t\sum_{i = 1}^{n} {
        (-1)^{i+j} [B]_{i,j} \det {(C(i|j))}
        }
        \\
        = {} &
        s\sum_{i = 1}^{n} {
        (-1)^{i+j} [A]_{i,j} \det {(A(i|j))}
        }
        +
        t\sum_{i = 1}^{n} {
        (-1)^{i+j} [B]_{i,j} \det {(B(i|j))}
        }
        \\
        = {} &
        s \det{(A)} + t \det{(B)}.
        \qedhere
    \end{align*}
\end{proof}

多线性有一个较直白的推广.
具体地,
对任何不超过 \(n\) 的正整数 \(j\),
任何 \(n-1\)~个 \(n \times 1\)~阵
\(a_1\), \(\dots\), \(a_{j-1}\),
\(a_{j+1}\), \(\dots\), \(a_n\),
任何 \(\ell\)~个 \(n \times 1\)~阵
\(b_1\), \(b_2\), \(\dots\), \(b_\ell\),
任何 \(\ell\)~个数
\(s_1\), \(s_2\), \(\dots\), \(s_\ell\),
有
\begin{align*}
         & \det
    {\left[
            a_1, \dots, a_{j-1},
            \sum_{k = 1}^{\ell} {s_k b_k},
            a_{j+1}, \dots, a_n
            \right]}
    \\
    = {} &
    \sum_{k = 1}^{\ell}
    {s_k
    \det {[a_1, \dots, a_{j-1},
                b_k,
                a_{j+1}, \dots, a_n]}}.
\end{align*}
您可施数学归纳法于 \(\ell\) 以证此事
(注意到
\(
s_1 b_1 + \dots + s_\ell b_\ell
= (s_1 b_1 + \dots + s_{\ell-1} b_{\ell-1}) + s_\ell b_\ell
\)%
);
其实, 这就是多次用多线性的结果.

\begin{theorem}[交错性]
    行列式 (关于列) 是交错性的.
    具体地,
    若 \(n\)~级阵 \(A\) 有二列完全相同,
    则 \(\det {(A)} = 0\).
\end{theorem}

有一件事值得一提.
明显地, \(1\)~级阵不可能有二列完全相同.
但是,
根据
``若 \(p\) 是假的, 则 `若 \(p\), 则 \(q\)' 是真的''
原则,
我们仍认为 \(1\)~级阵的行列式有交错性.
(``若 \(p\), 则 \(q\)''
相当于
``对适合条件 \(p\) 的任何对象~\(o\),
\(o\) 也适合条件 \(q\)''.
形如 ``对任何 A, 必 B'' 的话%
的反面就是 ``存在某个 A, 使之非 B''.
所以, 若我们找不到 ``非 B'' 的 A,
我们应认为, ``对任何 A, 必 B'' 是对的.
特别地, 当 A 不存在时, 自然没有 ``非 B'' 的 A.
故, 我们认为, 即使 A 不存在,
``对任何 A, 必 B'' 是对的.)

\begin{proof}
    我们用数学归纳法证明此事.
    具体地, 设 \(P(n)\) 为命题
    \begin{quotation}
        对每一个有完全相同的二列的 \(n\)~级阵 \(A\),
        其行列式必为零.
    \end{quotation}
    则, 我们的目标是:
    对任何正整数~\(n\), \(P(n)\) 是正确的.

    前面解释过, 我们认为, \(P(1)\) 是真的.

    现在考虑 \(P(2)\).
    为此, 任取有完全相同的二列的 \(2\)~级阵
    \begin{align*}
        A = \begin{bmatrix}
                a & a \\
                b & b \\
            \end{bmatrix}.
    \end{align*}
    则
    \begin{align*}
        \det {(A)}
        = ab - ba
        = 0.
    \end{align*}
    故 \(P(2)\) 是真的.

    现在, 我们假定 \(P(m-1)\) 是正确的
    (注意, 此处 \(m \geq 3\)).
    我们要证 \(P(m)\) 也是正确的.
    任取一个有完全相同的二列的 \(m\)~级阵 \(A\).
    我们设 \(A\)~的列~\(p\) 等于列~\(q\),
    其中 \(1 \leq p < q \leq m\).
    在 \(1\), \(2\), \(\dots\), \(m\) 这 \(m\)~个数里,
    我们必定能找到一个数 \(j\),
    它既不等于 \(p\), 也不等于 \(q\).
    按列~\(j\) 展开行列式, 有
    \begin{align*}
        \det {(A)}
        = \sum_{i = 1}^{m} {(-1)^{i + j} [A]_{i,j}
        \det {(A(i|j))}}.
    \end{align*}
    注意到, \(m-1\)~级阵 \(A(i|j)\) 仍有完全相同的二列.
    所以, 根据假定, \(\det {(A(i|j))} = 0\).
    故 \(\det {(A)} = 0\).

    所以, \(P(m)\) 是正确的.
    由数学归纳法原理, 待证命题成立.
\end{proof}

行列式的一些性质是其多线性与交错性的推论.
比如

\begin{theorem}[反称性]
    行列式 (关于列) 是反称性的.
    具体地,
    设 \(A\) 是 \(n\)~级阵,
    设交换 \(A\)~的列~\(p\) 与列~\(q\) 后得到的阵为 \(B\)
    (\(p < q\)).
    则 \(\det {(B)} = -\det {(A)}\).
    (通俗地, 交换方阵的二列, 则其行列式变号.)
\end{theorem}

\begin{proof}
    设 \(A = [a_1, a_2, \dots, a_n]\).
    为方便说话, 我们写
    \begin{align*}
        f(x, y)
        = \det {[a_1, \dots, a_{p-1}, x, a_{p+1}, \dots,
                    a_{q-1}, y, a_{q+1}, \dots, a_n]}.
    \end{align*}
    则
    \begin{align*}
        0
        = {} & f(a_p + a_q, a_p + a_q)               \\
        = {} & f(a_p, a_p + a_q) + f(a_q, a_p + a_q) \\
        = {} & (f(a_p, a_p) + f(a_p, a_q))
        + (f(a_q, a_p) + f(a_q, a_q))                \\
        = {} & f(a_p, a_q) + f(a_q, a_p)             \\
        = {} & \det {(A)} + \det {(B)}.
        \qedhere
    \end{align*}
\end{proof}

\section{用行列式的性质确定行列式}

本节, 我们研究如何%
用行列式的一些性质确定行列式.

假定, 您\emph{熟悉}一些人
(而不只是 ``知道他们的存在'').
自然地, 您也知道他们的一些特征.
我从这些人里选一个,
我想, 您可以说出此人的一些特征.
但是, 反过来,
我说出某人的一些特征,
您就一定能确定此人吗?
那自然是不一定的.
不过, 当我给出较多的特征时,
那是有可能的.
% 您就一定能确定此人是谁吗?
% 那自然是不一定的.
% 但是, 当我给出较多的特征时,
% 那是有可能的.
% 比如, 我说, ``他是一个个人练习生'',
% 您可能无法肯定他是谁;
% 我继续说, ``他练习了 5/2~年了'',
% 此时, 您可缩小范围,
% 但不一定能给出判断;
% % 但不一定能给出判断
% % (毕竟, ``5/2'' 只是一个估计);
% 我又说, ``他喜欢唱、跳、说唱、篮球'',
% 此时, 您有可能可肯定地告诉我他是谁.

抽象地, \(n\)~级阵的行列式就是一个%
定义在全体 \(n\)~级阵上的函数.
就像一个人有多种特征那样,
行列式自然也有不少特征 (或者, 性质),
无论是 ``有名的'' (有名字的),
还是 ``无名的''.
值得注意的是,
有些特征并不是行列式特有的:
比如, 零函数 \(z(A) = 0\)
(其中 \(A\) 是任何的 \(n\)~级阵)
适合多线性、交错性、反称性,
恒取 \(1\) 的函数 \(u(A) = 1\)
适合规范性,
但它们都不是行列式.
不过, 若我们联合行列式的\emph{几个}特征,
则它们可\emph{确定}行列式.

% 我们已知行列式的 4~个性质:
% 规范性、多线性、交错性、反称性.
% 其中, 反称性是多线性与交错性的推论,
% 且规范性看上去跟多线性与交错性无关
% (毕竟,
% 这是一个简单的、用定义即可轻松解决的问题).
% 所以, 我们认为,
% 多线性与交错性有可能是 ``行列式之魂''.
% 幸运地, 事实的确如此.

\begin{theorem}
    设定义在全体 \(n\)~级阵上的函数 \(f\) 适合:

    (1)
    (多线性)
    对任何不超过 \(n\) 的正整数 \(j\),
    任何 \(n-1\)~个 \(n \times 1\)~阵
    \(a_1\), \(\dots\), \(a_{j-1}\),
    \(a_{j+1}\), \(\dots\), \(a_n\),
    任何二个 \(n \times 1\)~阵 \(x\), \(y\),
    任何二个数 \(s\), \(t\),
    有
    \begin{align*}
             & f
            {([a_1, \dots, a_{j-1}, sx + ty,
                        a_{j+1}, \dots, a_n])}
        \\
        = {} &
        s
        f {([a_1, \dots, a_{j-1}, x, a_{j+1}, \dots, a_n])}
        +
        t
        f {([a_1, \dots, a_{j-1}, y, a_{j+1}, \dots, a_n])}.
    \end{align*}

    (2)
    (交错性)
    若 \(n\)~级阵 \(A\) 有二列完全相同,
    则 \(f {(A)} = 0\).

    那么, 对任何 \(n\)~级阵 \(A\),
    \(f(A) = f(I) \det {(A)}\).

    特别地, 若 \(f(I) = 1\) (规范性),
    则 \(f\) 就是行列式.
\end{theorem}

\begin{proof}
    我们设 \(g(A) = f(A) - f(I) \det {(A)}\).
    不难验证, \(g\) 也有多线性、交错性、反称性
    (具体地, 设交换 \(n\)~级阵 \(A\)~的%
    列~\(p\) 与列~\(q\) 后得到的阵为 \(B\),
    且 \(p < q\),
    则 \(g(B) = -g(A)\);
    这是前二个性质的推论),
    且 \(g(I) = 0\).
    我们的目标是: 证 \(g\) 是零函数 (即, 恒为零).

    任取一个 \(n\)~级阵 \(A\).
    设 \(n\)~级单位阵的列~\(1\), \(2\), \(\dots\), \(n\)
    分别是 \(e_1\), \(e_2\), \(\dots\), \(e_n\).
    设 \(A\) 的列~\(1\), \(2\), \(\dots\), \(n\)
    分别是 \(a_1\), \(a_2\), \(\dots\), \(a_n\).
    于是
    \begin{align*}
        a_{k}
        = {} & [A]_{1,k} e_{1} + [A]_{2,k} e_{2}
        + \dots + [A]_{n,k} e_{n}
        \\
        = {} & \sum_{i_k = 1}^{n} {[A]_{i_k,k} e_{i_k}}.
    \end{align*}
    (注意, 这里, 我用了 \(n\)~个求和指标
    \(i_1\), \(i_2\), \(\dots\), \(i_n\).
    等会儿, 您就知道为什么我要这么写了.)
    从而, 利用多线性,
    \begin{align*}
        g(A)
        = {} &
        g([a_1, a_2, \dots, a_n])
        \\
        = {} &
        g\left(\left[ \sum_{i_1 = 1}^{n} {[A]_{i_1,1} e_{i_1},
        a_2, \dots, a_n} \right]\right)
        \\
        = {} &
        \sum_{i_1 = 1}^{n} {[A]_{i_1,1}\,
        g([ e_{i_1}, a_2, \dots, a_n ])}
        \\
        = {} &
        \sum_{i_1 = 1}^{n} {[A]_{i_1,1}\,
        g\left(\left[ e_{i_1},
            \sum_{i_2 = 1}^{n} [A]_{i_2,2} e_{i_2},
        \dots, a_n
        \right]\right)}
        \\
        = {} &
        \sum_{i_1 = 1}^{n} {
        \sum_{i_2 = 1}^{n} {[A]_{i_1,1} [A]_{i_2,2}\,
        g([ e_{i_1}, e_{i_2}, \dots, a_n ])}}
        \\
        = {} &
        \dots \dots \dots \dots
        \dots \dots \dots \dots
        \dots \dots \dots \dots
        \dots \dots \dots \dots
        \\
        = {} &
        \sum_{i_1 = 1}^{n} {
        \sum_{i_2 = 1}^{n} {
        \dots
        \sum_{i_n = 1}^{n} {
        [A]_{i_1,1} [A]_{i_2,2} \dots [A]_{i_n,n} \,
        g([e_{i_1}, e_{i_2}, \dots, e_{i_n}])}}}.
    \end{align*}

    利用交错性, 不难看出,
    若存在整数 \(p\), \(q\)
    使 \(1 \leq p < q \leq n\),
    且 \(i_p = i_q\),
    则 \([e_{i_1}, e_{i_2}, \dots, e_{i_n}]\)
    有二列相同,
    故 \(g([e_{i_1}, e_{i_2}, \dots, e_{i_n}]) = 0\).
    所以,
    \begin{align*}
        g(A)
        = \sum_{\substack{
        1 \leq i_1, i_2, \dots, i_n \leq n \\
                i_1, i_2, \dots, i_n\,\text{互不相同}
            }}
        {[A]_{i_1,1} [A]_{i_2,2} \dots [A]_{i_n,n}\,
            g([e_{i_1}, e_{i_2}, \dots, e_{i_n}])}.
    \end{align*}

    我们用反称性证明每一个
    \(g([e_{i_1}, e_{i_2}, \dots, e_{i_n}])\)
    (其中 \(i_1\), \(i_2\), \(\dots\), \(i_n\)
    是不超过 \(n\) 的正整数, 且互不相同)
    都是 \(0\).
    适当地交换
    \([e_{i_1}, e_{i_2}, \dots, e_{i_n}]\)~的列,
    可变其为 \([e_1, e_2, \dots, e_n]\),
    即 \(n\)~级单位阵.
    从而
    \begin{align*}
        g([e_{i_1}, e_{i_2}, \dots, e_{i_n}])
        = \pm g(I) = 0.
    \end{align*}

    所以 \(g(A) = 0\).
    故 \(f(A) = f(I) \det {(A)}\).
    (反过来, 不难验证, 若我们定义
    \(f(A) = f(I) \det {(A)}\),
    则 \(f\) 适合多线性与交错性.)
\end{proof}

\begin{theorem}
    若定义在全体 \(n\)~级阵上的函数 \(f\)
    适合规范性、多线性、交错性,
    则 \(f\) 就是 (\(n\)~级阵的) 行列式 (函数).
\end{theorem}

由此, 理论地, 我们可用%
行列式的这三条性质%
\emph{定义}行列式:

\begin{definition}
    设 \(f\) 是定义在全体 \(n\)~级阵上的函数.
    若 \(f\) 适合如下三条, 则说
    \(f\) 是 (\(n\)~级阵的) \emph{一个}行列式函数
    (自然地, 若 \(A\) 是 \(n\)~级阵,
    则 \(f(A)\) 是 \(A\) 的\emph{一个}行列式):

    (1)
    (规范性)
    若 \(I\) 是 \(n\)~级单位阵,
    则 \(f(I) = 1\).

    (2)
    (多线性)
    对任何不超过 \(n\) 的正整数 \(j\),
    任何 \(n-1\)~个 \(n \times 1\)~阵
    \(a_1\), \(\dots\), \(a_{j-1}\),
    \(a_{j+1}\), \(\dots\), \(a_n\),
    任何二个 \(n \times 1\)~阵 \(x\), \(y\),
    任何二个数 \(s\), \(t\),
    有
    \begin{align*}
             & f
            {([a_1, \dots, a_{j-1}, sx + ty,
                        a_{j+1}, \dots, a_n])}
        \\
        = {} &
        s
        f {([a_1, \dots, a_{j-1}, x, a_{j+1}, \dots, a_n])}
        +
        t
        f {([a_1, \dots, a_{j-1}, y, a_{j+1}, \dots, a_n])}.
    \end{align*}

    (3)
    (交错性)
    若 \(n\)~级阵 \(A\) 有二列完全相同,
    则 \(f {(A)} = 0\).
\end{definition}

不过, 我没有这么干,
因为由这个定义入手,
证明 (\(n\)~级阵的) 行列式函数存在且唯一%
是较难的.
(我认为, 我用的定义是较简单的.)

\section{``行'' 列式}

前面, 我们学习了行列式的一些公式与性质.
它们至少有一个共同点:
它们都是关于\emph{列}的命题.
毕竟, 行列式的定义就是按列~\(1\) 展开.
利用定义,
我们得到了按任何一列展开行列式的公式.
然后, 我们得到了行列式的一些 (关于列的) 性质.

自然地, 我们问:
行列式是否也有关于行的公式与性质?
此事的回答是 ``是''.
为此, 我们先按行~\(1\) 展开行列式.

\begin{restatable}[]{theorem}{TheoremExpansionAboutRowOne}
    设 \(A\) 是 \(n\)~级阵 (\(n \geq 1\)).
    则
    \begin{align*}
        \det {(A)} = \sum_{j = 1}^{n}
        {(-1)^{1+j} [A]_{1,j} \det {(A(1|j))}}.
    \end{align*}
\end{restatable}

我们无妨先用小级阵验证此命题.

设 \(A\) 是 \(1\)~级阵.
显然
(我作过一个关于 ``\(0\)~级阵'' 及其 ``行列式'' 的约定).

设 \(A\) 是 \(2\)~级阵 (也就是, 取 \(n = 2\)).
则
\begin{align*}
    \det {(A)}
    = {} & [A]_{1,1} \det {[[A]_{2,2}]}
    - [A]_{2,1} \det {[[A]_{1,2}]}              \\
    = {} & [A]_{1,1} \det {[[A]_{2,2}]}
    - [A]_{1,2} \det {[[A]_{2,1}]}              \\
    = {} & (-1)^{1+1} [A]_{1,1} \det {(A(1|1))}
    + (-1)^{1+2} [A]_{1,2} \det {(A(1|2))}.
\end{align*}
所以, \(n = 2\) 时, 命题是正确的.

\begin{proof}
    我们用数学归纳法证明此事.
    具体地, 设 \(P(n)\) 为命题
    \begin{quotation}
        对\emph{任何} \(n\)~级阵 \(A\),
        \begin{align*}
            \det {(A)} = \sum_{j = 1}^{n}
            {(-1)^{1+j} [A]_{1,j} \det {(A(1|j))}}.
        \end{align*}
    \end{quotation}
    则, 我们的目标是:
    对任何正整数 \(n\), \(P(n)\) 是正确的.

    我们已知 \(P(1)\), \(P(2)\) 都是正确的.

    现在, 我们假定 \(P(m-1)\) 是正确的.
    我们要证 \(P(m)\) 也是正确的.
    任取一个 \(m\)~级阵 \(A\).
    为方便, 我们写
    \((-1)^{1+1} [A]_{1,1} \det {(A(1|1))}\)
    为 \(f\).
    于是%
    \begin{align*}
             & \det {(A)} \\
        = {} &
        f
        +
        \sum_{i = 2}^{m} {
        (-1)^{i+1} [A]_{i,1} \det {(A(i|1))}
        }
        \tag*{(1)}
        \\
        = {} &
        f
        +
        \sum_{i = 2}^{m} {
        (-1)^{i+1} [A]_{i,1}
        \sum_{j = 2}^{m} {
        (-1)^{1+j-1} [A]_{1,j} \det {(A(i,1|1,j))}
        }
        }
        \tag*{(2)}
        \\
        = {} &
        f
        +
        \sum_{i = 2}^{m} {
        \sum_{j = 2}^{m} {
        (-1)^{i+1} [A]_{i,1}
        (-1)^{1+j-1} [A]_{1,j} \det {(A(i,1|1,j))}
        }
        }
        \tag*{(3)}
        \\
        = {} &
        f
        +
        \sum_{j = 2}^{m} {
        \sum_{i = 2}^{m} {
        (-1)^{i+1} [A]_{i,1}
        (-1)^{1+j-1} [A]_{1,j} \det {(A(i,1|1,j))}
        }
        }
        \tag*{(4)}
        \\
        = {} &
        f
        +
        \sum_{j = 2}^{m} {
        \sum_{i = 2}^{m} {
        (-1)^{1+j} [A]_{1,j}
        (-1)^{i-1+1} [A]_{i,1} \det {(A(i,1|1,j))}
        }
        }
        \tag*{(5)}
        \\
        = {} &
        f
        +
        \sum_{j = 2}^{m} {
        (-1)^{1+j} [A]_{1,j}
        \sum_{i = 2}^{m} {
        (-1)^{i-1+1} [A]_{i,1} \det {(A(i,1|1,j))}
        }
        }
        \tag*{(6)}
        \\
        = {} &
        f
        +
        \sum_{j = 2}^{m} {
        (-1)^{1+j} [A]_{1,j} \det {(A(1|j))}
        }.
        \tag*{(7)}
    \end{align*}%
    所以, \(P(m)\) 是正确的.
    由数学归纳法原理, 待证命题成立.

    我想, 您一定看到了公式右侧的序号.
    这是方便我说话用的.
    我想, 适当地解释这几步,
    % 对喜欢钻研的初学行列式的读者而言,
    是有好处的.

    (1) 是行列式的定义.

    (2) 利用了假定.
    设 \(i > 1\).
    我们假定可按行~\(1\) 展开每一个 \(m - 1\)~级阵的行列式.
    \(A(i|1)\) 不就是 \(m - 1\)~级阵吗?
    那么, 我们就按 \(A(i|1)\) 的行~\(1\) 展开.
    \(A(i|1)\) 的行~\(1\) 正好对应
    \(A\) 的行~\(1\).
    最后, 注意到,
    \(A\)~的 \((1, j)\)-元恰是
    \(A(i|1)\)~的 \((1, j - 1)\)-元.

    (3) 利用了分配律 (还有加法的结合律与交换律).

    (4) 利用了加法的结合律与交换律.
    (通俗地, 就是 ``求和号的次序可换''.)

    (5) 利用了 \(-1\)~的整数次方的性质 (我在前面说过).
    当然, 我还利用的乘法的结合律与交换律.

    (6) 又用了一次分配律 (还有加法的结合律与交换律).
    不过, 跟 (3) 对比, 这次是反着用.

    (7) 用到了行列式的定义.
    注意到, \(i > 1\) 时,
    \(A\)~的 \((i, 1)\)-元恰是
    \(A(1|j)\)~的 \((i - 1, 1)\)-元.
\end{proof}

不难看到, 按行~\(1\) 展开行列式的公式%
跟按列~\(1\) 展开行列式的公式是十分相似的.
为方便说话,
我作如下定义.
% 我作一个暂时的定义.

\begin{definition}[``行'' 列式]
    设 \(A\) 是 \(n\)~级阵 (\(n \geq 1\)).
    定义 \(A\) 的 ``行'' 列式
    \begin{align*}
        \operatorname{det}' {(A)}
        =
        \begin{dcases}
            [A]_{1,1},
             & n = 1;    \\
            \sum_{j = 1}^{n}
            {(-1)^{1+j} [A]_{1,j}
            \operatorname{det}' {(A(1|j))}},
             & n \geq 2.
        \end{dcases}
    \end{align*}
\end{definition}

由上个定理,
不难用数学归纳法证明,
``行'' 列式就是行列式.
作命题 \(P(n)\):
对任何 \(n\)~级阵 \(A\),
\(A\)~的 ``行'' 列式等于 \(A\)~的行列式.
\(P(1)\) 显然是正确的.
假定 \(P(m-1)\) 正确.
任取一个 \(m\)~级阵 \(A\).
则
\begin{align*}
    \operatorname{det}' {(A)}
    = {} & \sum_{j = 1}^{m}
    {(-1)^{1+j} [A]_{1,j} \operatorname{det}' {(A(1|j))}}
    \\
    = {} & \sum_{j = 1}^{m}
    {(-1)^{1+j} [A]_{1,j} \det {(A(1|j))}}
    \\
    = {} & \det {(A)}.
\end{align*}

如果, 我用
``行'' 列式~\(\operatorname{det}'\)
作行列式的定义,
那么, 我可用完全相似的方法,
根据新的定义,
证明按一行展开行列式的公式
(见本章, 节~\sekcio{7}):%
\begin{align*}
         & \det {(A)}                  \\
    = {} &
    \sum_{j = 1}^{m} {
    (-1)^{1+j} [A]_{1,j} \det {(A(1|j))}
    }
    % \tag*{(1)}
    \\
    = {} &
    \sum_{j = 1}^{m} {
    (-1)^{1+j} [A]_{1,j}
    \sum_{\substack{1 \leq \ell \leq m \\ \ell \neq j}} {
    (-1)^{(i - 1) + (\ell - \rho(\ell, j))}
        [A]_{i,\ell} \det {(A({1,i}|{j,\ell}))}
    }
    }
    % \tag*{(2)}
    \\
    = {} &
    \sum_{j = 1}^{m} {
    \sum_{\substack{1 \leq \ell \leq m \\ \ell \neq j}} {
    (-1)^{(i - 1) + (\ell - \rho(\ell, j))}
    (-1)^{1+j}
        [A]_{1,j} [A]_{i,\ell} \det {(A({1,i}|{j,\ell}))}
    }
    }
    % \tag*{(3)}
    \\
    = {} &
    \sum_{\ell = 1}^{m} {
    \sum_{\substack{1 \leq j \leq m    \\ j \neq \ell}} {
    (-1)^{(i - 1) + (\ell - \rho(\ell, j))}
    (-1)^{1+j}
        [A]_{1,j} [A]_{i,\ell} \det {(A({1,i}|{j,\ell}))}
    }
    }
    % \tag*{(4)}
    \\
    = {} &
    \sum_{\ell = 1}^{m} {
    \sum_{\substack{1 \leq j \leq m    \\ j \neq \ell}} {
    (-1)^{i} (-1)^{\ell} (-1)^{\rho(\ell, j)} (-1)^{j}
        [A]_{1,j} [A]_{i,\ell} \det {(A({1,i}|{j,\ell}))}
    }
    }
    % \tag*{(5)}
    \\
    = {} &
    \sum_{\ell = 1}^{m} {
    \sum_{\substack{1 \leq j \leq m    \\ j \neq \ell}} {
    (-1)^{i + \ell} (-1)^{1 + j - \rho(j, \ell)}
        [A]_{1,j} [A]_{i,\ell} \det {(A({1,i}|{j,\ell}))}
    }
    }
    % \tag*{(6)}
    \\
    = {} &
    \sum_{\ell = 1}^{m} {(-1)^{i + \ell} [A]_{i,\ell}
    \sum_{\substack{1 \leq j \leq m    \\ j \neq \ell}} {
    (-1)^{1 + j - \rho(j, \ell)}
        [A]_{1,j} \det {(A({1,i}|{j,\ell}))}
    }
    }
    % \tag*{(7)}
    \\
    = {} &
    \sum_{\ell = 1}^{m} {(-1)^{i + \ell} [A]_{i,\ell}
    \det {(A(i|\ell))}
    }.
    % \tag*{(8)}
\end{align*}%
不过, 写几乎一样的论证是较无趣的.
所以, 换个思路.

\begin{theorem}
    设 \(A\) 是 \(n\)~级阵 (\(n \geq 1\)).
    则 \(A\)~的转置, \(A^{\mathrm{T}}\),
    的行列式等于 \(A\)~的行列式.
    % 则 \(A\)~的行列式等于%
    % 其转置 \(A^{\mathrm{T}}\) 的行列式.
\end{theorem}

\begin{proof}
    我们用数学归纳法证明此事.
    具体地, 设 \(P(n)\) 为命题
    \begin{quotation}
        对\emph{任何} \(n\)~级阵 \(A\),
        \begin{align*}
            \det {(A^{\mathrm{T}})} = \det {(A)}.
        \end{align*}
    \end{quotation}
    则, 我们的目标是:
    对任何正整数 \(n\), \(P(n)\) 是正确的.

    \(P(1)\) 是正确的.
    毕竟, \(1\)~级阵的转置就是自己.

    现在, 我们假定 \(P(m-1)\) 是正确的.
    我们要证 \(P(m)\) 也是正确的.
    任取一个 \(m\)~级阵 \(A\).
    注意到 \([A]_{i,j} = [A^{\mathrm{T}}]_{j,i}\).
    于是, 可以验证,
    \(A(i|k) = (A^{\mathrm{T}}(k|i))^{\mathrm{T}}\).
    则
    \begin{align*}
        \det {(A^{\mathrm{T}})}
        = {} & \sum_{i = 1}^{m} {
        (-1)^{1+i} [A^{\mathrm{T}}]_{1,i}
        \det {(A^{\mathrm{T}}(1|i))}
        }
        \\
        = {} & \sum_{i = 1}^{m} {
        (-1)^{1+i} [A^{\mathrm{T}}]_{1,i}
        \det {((A^{\mathrm{T}}(1|i))^{\mathrm{T}})}
        }
        \\
        = {} & \sum_{i = 1}^{m} {
        (-1)^{i+1} [A]_{i,1}
        \det {(A(i|1))}
        }
        \\
        = {} & \det {(A)}.
    \end{align*}
    所以, \(P(m)\) 是正确的.
    由数学归纳法原理, 待证命题成立.
\end{proof}

此性质是重要的:
利用它,
我们可译行列式的关于列的命题%
为关于行的命题.
我用一个例助您理解;
它是按行~\(i\) 展开行列式的公式.

\begin{restatable}[]{theorem}{TheoremExpansionAboutAnyRow}
    设 \(A\) 是 \(n\)~级阵 (\(n \geq 1\)).
    设 \(i\) 为整数, 且 \(1 \leq i \leq n\).
    则
    \begin{align*}
        \det {(A)} = \sum_{j = 1}^{n}
        {(-1)^{i+j} [A]_{i,j} \det {(A(i|j))}}.
    \end{align*}
\end{restatable}

\begin{proof}
    基本的思路:
    先转置,
    施关于列的命题于转置,
    再转置回来.
    回想,
    \(A^{\mathrm{T}}\)~的列~\(i\)
    跟 \(A\)~的行~\(i\) 对应,
    故我们按列~\(i\) 展开 \(A^{\mathrm{T}}\)~的行列式.
    则
    \begin{align*}
        \det {(A)}
        = {} & \det {(A^{\mathrm{T}})}
        \\
        = {} & \sum_{j = 1}^{n}
        {(-1)^{j+i} [A^{\mathrm{T}}]_{j,i}
        \det {(A^{\mathrm{T}}(j|i))}
        }
        \\
        = {} & \sum_{j = 1}^{n}
        {(-1)^{j+i} [A]_{i,j}
        \det {((A(i|j))^{\mathrm{T}})}
        }
        \\
        = {} & \sum_{j = 1}^{n}
        {(-1)^{i+j} [A]_{i,j}
        \det {(A(i|j))}
        }.
        \qedhere
    \end{align*}
\end{proof}

为方便,
我译前面的行列式的关于列的性质%
为关于行的性质.
不过, 我就不证它们了.
我留它们的论证为您的习题吧.

\begin{theorem}[多线性]
    行列式 (关于行) 是多线性的.
    具体地, 对任何不超过 \(n\) 的正整数 \(i\),
    任何 \(n-1\)~个 \(1 \times n\)~阵
    \(a_1\), \(\dots\), \(a_{i-1}\),
    \(a_{i+1}\), \(\dots\), \(a_n\),
    任何二个 \(1 \times n\)~阵 \(x\), \(y\),
    任何二个数 \(s\), \(t\),
    有
    \begin{align*}
        \det
        {\begin{bmatrix}
                 a_1     \\
                 \vdots  \\
                 a_{i-1} \\
                 sx + ty \\
                 a_{i+1} \\
                 \vdots  \\
                 a_{n}   \\
             \end{bmatrix}}
        =
        s
        \det
        {\begin{bmatrix}
                 a_1     \\
                 \vdots  \\
                 a_{i-1} \\
                 x       \\
                 a_{i+1} \\
                 \vdots  \\
                 a_{n}   \\
             \end{bmatrix}}
        +
        t
        \det
        {\begin{bmatrix}
                 a_1     \\
                 \vdots  \\
                 a_{i-1} \\
                 y       \\
                 a_{i+1} \\
                 \vdots  \\
                 a_{n}   \\
             \end{bmatrix}}.
    \end{align*}
    (若 \(i = 1\), 则 \(a_1\) 不出现;
    若 \(i = n\), 则 \(a_n\) 不出现.
    下同.)
\end{theorem}

\begin{theorem}[交错性]
    行列式 (关于行) 是交错性的.
    具体地,
    若 \(n\)~级阵 \(A\) 有二行完全相同,
    则 \(\det {(A)} = 0\).
\end{theorem}

\begin{theorem}[反称性]
    行列式 (关于行) 是反称性的.
    具体地,
    设 \(A\) 是 \(n\)~级阵,
    设交换 \(A\)~的行~\(p\) 与行~\(q\) 后得到的阵为 \(B\)
    (\(p < q\)).
    则 \(\det {(B)} = -\det {(A)}\).
    (通俗地, 交换方阵的二行, 则其行列式变号.)
\end{theorem}

\begin{theorem}
    设定义在全体 \(n\)~级阵上的函数 \(f\) 适合:

    (1)
    (多线性)
    对任何不超过 \(n\) 的正整数 \(i\),
    任何 \(n-1\)~个 \(1 \times n\)~阵
    \(a_1\), \(\dots\), \(a_{i-1}\),
    \(a_{i+1}\), \(\dots\), \(a_n\),
    任何二个 \(1 \times n\)~阵 \(x\), \(y\),
    任何二个数 \(s\), \(t\),
    有
    \begin{align*}
        f
        {\left(\begin{bmatrix}
                       a_1     \\
                       \vdots  \\
                       a_{i-1} \\
                       sx + ty \\
                       a_{i+1} \\
                       \vdots  \\
                       a_{n}   \\
                   \end{bmatrix}\right)}
        =
        s
        f
            {\left(\begin{bmatrix}
                               a_1     \\
                               \vdots  \\
                               a_{i-1} \\
                               x       \\
                               a_{i+1} \\
                               \vdots  \\
                               a_{n}   \\
                           \end{bmatrix}\right)}
        +
        t
        f
            {\left(\begin{bmatrix}
                               a_1     \\
                               \vdots  \\
                               a_{i-1} \\
                               y       \\
                               a_{i+1} \\
                               \vdots  \\
                               a_{n}   \\
                           \end{bmatrix}\right)}.
    \end{align*}

    (2)
    (交错性)
    若 \(n\)~级阵 \(A\) 有二行完全相同,
    则 \(f {(A)} = 0\).

    那么, 对任何 \(n\)~级阵 \(A\),
    \(f(A) = f(I) \det {(A)}\).

    特别地, 若 \(f(I) = 1\) (规范性),
    则 \(f\) 就是行列式.
\end{theorem}

\begin{theorem}
    若定义在全体 \(n\)~级阵上的函数 \(f\)
    适合规范性、(关于行的) 多线性、(关于行的) 交错性,
    则 \(f\) 就是 (\(n\)~级阵的) 行列式 (函数).
\end{theorem}

由此, 理论地, 我们也可如此%
\emph{定义}行列式:

\begin{definition}
    设 \(f\) 是定义在全体 \(n\)~级阵上的函数.
    若 \(f\) 适合如下三条, 则说
    \(f\) 是 (\(n\)~级阵的) \emph{一个}行列式函数
    (自然地, 若 \(A\) 是 \(n\)~级阵,
    则 \(f(A)\) 是 \(A\) 的\emph{一个}行列式):

    (1)
    (规范性)
    若 \(I\) 是 \(n\)~级单位阵,
    则 \(f(I) = 1\).

    (2)
    (多线性)
    对任何不超过 \(n\) 的正整数 \(i\),
    任何 \(n-1\)~个 \(1 \times n\)~阵
    \(a_1\), \(\dots\), \(a_{i-1}\),
    \(a_{i+1}\), \(\dots\), \(a_n\),
    任何二个 \(1 \times n\)~阵 \(x\), \(y\),
    任何二个数 \(s\), \(t\),
    有
    \begin{align*}
        f
        {\left(\begin{bmatrix}
                       a_1     \\
                       \vdots  \\
                       a_{i-1} \\
                       sx + ty \\
                       a_{i+1} \\
                       \vdots  \\
                       a_{n}   \\
                   \end{bmatrix}\right)}
        =
        s
        f
            {\left(\begin{bmatrix}
                               a_1     \\
                               \vdots  \\
                               a_{i-1} \\
                               x       \\
                               a_{i+1} \\
                               \vdots  \\
                               a_{n}   \\
                           \end{bmatrix}\right)}
        +
        t
        f
            {\left(\begin{bmatrix}
                               a_1     \\
                               \vdots  \\
                               a_{i-1} \\
                               y       \\
                               a_{i+1} \\
                               \vdots  \\
                               a_{n}   \\
                           \end{bmatrix}\right)}.
    \end{align*}

    (3)
    (交错性)
    若 \(n\)~级阵 \(A\) 有二行完全相同,
    则 \(f {(A)} = 0\).
\end{definition}

\KunAsteriskoEnEnhavtabelo
\section{``行'' 列式 (续)}
\SenAsteriskoEnEnhavtabelo

\maldevigalegajxo

这里, 为方便,
我译前面的行列式的关于列的公式%
为关于行的公式.

我先引用我们见过的二个公式.

\TheoremExpansionAboutRowOne*

\TheoremExpansionAboutAnyRow*

现在, 我给出新的公式.
您至少有二种证明这些新公式的方法:
(a)
您可以用老方法, 几乎完全一样地论证;
(b)
您也可以利用转置, 译已有的公式.

\begin{theorem}
    设 \(A\) 是 \(n\)~级阵 (\(n \geq 1\)).
    设 \(k\) 是不超过 \(n\) 的正整数.
    设 \(i_1\), \(i_2\), \(\dots\), \(i_k\) 是%
    不超过 \(n\) 的正整数,
    且 \(i_1 < i_2 < \dots < i_k\).
    则
    \begin{align*}
         &
        \det {(A)}
        = \sum_{1 \leq j_1 < j_2 < \dots < j_k \leq n}
        {\det {\left(
                A\binom{i_1, i_2, \dots, i_k}
                {j_1, j_2, \dots, j_k}
                \right)}}
        \\
         &
        \qquad \qquad \qquad
        \cdot (-1)^{i_1 + i_2 + \dots + i_k
            + j_1 + j_2 + \dots + j_k}
        \det {(A({i_1,i_2,\dots,i_k}|{j_1,j_2,\dots,j_k}))}.
    \end{align*}
\end{theorem}

\begin{theorem}
    设 \(A\) 是 \(n\)~级阵 (\(n \geq 1\)).
    设 \(i_1\), \(i_2\), \(\dots\), \(i_n\) 是%
    不超过 \(n\) 的正整数,
    且\emph{互不相同}.
    则
    \begin{align*}
             & \det {(A)}                  \\
        = {} &
        \sum_{\substack{
        1 \leq j_1, j_2, \dots, j_n \leq n \\
                j_1, j_2, \dots, j_n\,\text{互不相同}
            }}
        {s(i_1, i_2, \dots, i_n)\,
            s(j_1, j_2, \dots, j_n)\,
            [A]_{i_1,j_1} [A]_{i_2,j_2} \dots [A]_{i_n,j_n}}.
    \end{align*}
\end{theorem}

特别地, 取
\(i_1\), \(i_2\), \(\dots\), \(i_n\)
为
\(1\), \(2\), \(\dots\), \(n\),
有,
\begin{align*}
    \det {(A)}
    = {} &
    \sum_{\substack{
    1 \leq j_1, j_2, \dots, j_n \leq n \\
            j_1, j_2, \dots, j_n\,\text{互不相同}
        }}
    { s(j_1, j_2, \dots, j_n)\,
        [A]_{1,j_1} [A]_{2,j_2} \dots [A]_{n,j_n}}.
\end{align*}
这是大多数教材\emph{定义}行列式的方式.
当我是初学者时, 我面对的就是它.

\begin{theorem}
    设 \(n\)~级阵 \(A\) 的行~\(1\), \(2\), \(\dots\), \(n\)
    分别为 \(a_1\), \(a_2\), \(\dots\), \(a_n\).
    设 \(\ell_1\), \(\ell_2\), \(\dots\), \(\ell_n\) 是%
    不超过 \(n\) 的正整数,
    且\emph{互不相同}.
    则
    \begin{align*}
        \det {\begin{bmatrix}
                      a_{\ell_1} \\
                      a_{\ell_2} \\
                      \vdots     \\
                      a_{\ell_n} \\
                  \end{bmatrix}
        }
        = s(\ell_1, \ell_2, \dots, \ell_n) \det {(A)}.
    \end{align*}
\end{theorem}

\section{阵的积}

本节, 我们学习阵的一种新运算, 即阵的乘法.

阵的乘法, 跟阵的加、减、数乘相比, 复杂一些.
不过,
我想说,
% 我要提醒您,
阵的加、减、数乘, 与本节要讨论的乘运算,
% 都是应实际需要引进的,
都不是数学家随意定义的.

我认为,
% 我的老师告诉我的一句话,
``长大了, 就明白了'',
不是没有道理的.
我初学数的减法时, 不明白为什么
``取 \(1\) 当 \(10\)'',
从而不理解 \(16 - 9 = 7\).
初学分数的加法时, 我不明白,
为什么 \(1/2 + 1/3 = 5/6\),
而不是 \(1/2 + 1/3 = 2/5\).
初学解方程时,
我不明白, 为什么 \(a - (b - c) = a - b + c\);
我当时认为, \(a - (b - c)\)
应等于 \(a - b - c\),
也就是 \((a - b) - c\).
不过, 现在我当然都明白了.
这样的故事还有不少.
% 但是, 某一日, 我突然就明白了这些问题.

我想, 您也有类似的体验.

好了, 我就说这么多.
我要开始引入阵的积了.

\begin{definition}
    设 \(A\) 是 \(m \times s\)~阵,
    \(B\) 是 \(s \times n\)~阵
    (也就是说, \(A\) 有多少列, \(B\) 就有多少行).
    我们定义 \(A\) 与 \(B\)~的积 (注意 \(A\), \(B\) 的次序) 为%
    一个 \(m \times n\)~阵 \(AB\),
    其中, 对任何不超过 \(m\) 的正整数 \(i\)
    与任何不超过 \(n\) 的正整数 \(j\),
    \begin{align*}
        [AB]_{i,j}
        = {} &
        [A]_{i,1} [B]_{1,j}
        + [A]_{i,2} [B]_{2,j}
        + \dots
        + [A]_{i,s} [B]_{s,j}
        \\
        = {} &
        \sum_{k = 1}^{s} {[A]_{i,k} [B]_{k,j}}.
    \end{align*}
    (通俗地, \(AB\) 是 \(m \times n\)~阵,
    其 \((i, j)\)-元就是
    \(A\)~的行~\(i\) 跟 \(B\)~的列~\(j\)~的%
    相应位置的元的积的和.)
\end{definition}

可以看到, 阵的乘法跟阵的加法比, 是有较大的区别的.
首先, 加法要求二个阵的尺寸相同, 而乘法要求%
第~1~个阵的列数等于第~2~个阵的行数.
其次, 阵的加法的定义用到的只不过是数的加法,
而阵的乘法的定义要同时用到数的加法与乘法.

阵的积, 不像数的积那样, 有可能是不可换的.
具体地,
设 \(A\), \(B\) 分别是 \(m \times s\) 与 \(s \times n\)~阵.
根据定义, \(AB\) 是有意义的.
但是, \(BA\) 无意义, 除非 \(n = m\).
这还只是一方面.
另一方面, 就算 \(n = m\),
\(AB\) 跟 \(BA\)~的尺寸也不一样,
除非 \(m = s = n\).
最后, 就算 \(m = s = n\),
\(AB\) 也可以不等于 \(BA\).

\begin{example}
    设
    \begin{align*}
        A = \begin{bmatrix}
                2 & 5 \\
                3 & 7 \\
            \end{bmatrix},
        \quad
        B = \begin{bmatrix}
                9 & 6 \\
                4 & 8 \\
            \end{bmatrix}.
    \end{align*}
    根据定义,
    \begin{align*}
         &
        AB = \begin{bmatrix}
                 2 \cdot 9 + 5 \cdot 4 & 2 \cdot 6 + 5 \cdot 8 \\
                 3 \cdot 9 + 7 \cdot 4 & 3 \cdot 6 + 7 \cdot 8 \\
             \end{bmatrix}
        = \begin{bmatrix}
              38 & 52 \\
              55 & 74 \\
          \end{bmatrix},
        \\
         &
        BA = \begin{bmatrix}
                 9 \cdot 2 + 6 \cdot 3 & 9 \cdot 5 + 6 \cdot 7 \\
                 4 \cdot 2 + 8 \cdot 3 & 4 \cdot 5 + 8 \cdot 7 \\
             \end{bmatrix}
        = \begin{bmatrix}
              36 & 87 \\
              32 & 76 \\
          \end{bmatrix}.
    \end{align*}
    所以, 即使 \(AB\), \(BA\) 都是 \(2\)~级阵,
    它们也不相等.
\end{example}

我们知道, 若 \(a\), \(b\) 是二个非零的数,
那么 \(ab \neq 0\).
但是, 对二个阵 \(A\), \(B\),
即使 \(A \neq 0\), \(B \neq 0\),
仍有可能 \(AB = 0\)
(此处的 \(0\) 是\emph{零阵},
即元全为零的阵).

\begin{example}
    设
    \begin{align*}
        A = \begin{bmatrix}
                0 & 0 \\
                1 & 0 \\
            \end{bmatrix},
        \quad
        B = \begin{bmatrix}
                0 & 0 \\
                0 & 1 \\
            \end{bmatrix}.
    \end{align*}
    根据定义,
    \begin{align*}
         &
        AB = \begin{bmatrix}
                 0 \cdot 0 + 0 \cdot 0 & 0 \cdot 0 + 0 \cdot 1 \\
                 1 \cdot 0 + 0 \cdot 0 & 1 \cdot 0 + 0 \cdot 1 \\
             \end{bmatrix}
        = \begin{bmatrix}
              0 & 0 \\
              0 & 0 \\
          \end{bmatrix}.
    \end{align*}
\end{example}

阵的积跟数的积虽有不同的地方,
但, 更重要地, 也有相同的 (或类似的) 地方.

\begin{theorem}
    设 \(A\), \(B\), \(C\) 分别是
    \(m \times s\), \(s \times n\), \(n \times p\)~阵.
    设 \(D\) 是 \(m \times s\) 阵,
    设 \(G\) 是 \(s \times n\) 阵.
    设 \(x\) 是数.
    则:

    (1)
    (结合律)
    \((AB)C = A(BC)\);

    (2)
    (分配律~1)
    \((A + D) B = AB + DB\);

    (3)
    (分配律~2)
    \(A (B + G) = AB + AG\);

    (4)
    (单位阵的作用)
    \(I_m A = A = A I_s\),
    其中 \(I_m\), \(I_s\) 分别是 \(m\), \(s\)~级单位阵;

    (5)
    (阵的数乘与乘法)
    \((xA)B = x(AB) = A(xB)\).
\end{theorem}

\begin{proof}
    (1)
    注意到 \(AB\) 是 \(m \times n\)~阵,
    故 \((AB)C\) 是 \(m \times p\)~阵;
    注意到 \(BC\) 是 \(s \times p\)~阵,
    故 \(A(BC)\) 也是 \(m \times p\)~阵.
    根据阵的积的定义,
    \begin{align*}
        [(AB)C]_{i,j}
        = {} &
        \sum_{\ell = 1}^{n} {[AB]_{i,\ell} [C]_{\ell,j}}
        \\
        = {} &
        \sum_{\ell = 1}^{n} {
        \left(\sum_{k = 1}^{s} {[A]_{i,k} [B]_{k,\ell}} \right)
        [C]_{\ell,j} }
        \\
        = {} &
        \sum_{\ell = 1}^{n} {
        \sum_{k = 1}^{s} {
        [A]_{i,k} [B]_{k,\ell} [C]_{\ell,j} }}
        \\
        = {} &
        \sum_{k = 1}^{s} {
        \sum_{\ell = 1}^{n} {
        [A]_{i,k} [B]_{k,\ell} [C]_{\ell,j}}}
        \\
        = {} &
        \sum_{k = 1}^{s} {
        [A]_{i,k}
        \left(
        \sum_{\ell = 1}^{n} {[B]_{k,\ell} [C]_{\ell,j}}
        \right) }
        \\
        = {} &
        \sum_{k = 1}^{s} {[A]_{i,k} [BC]_{k,j}}
        \\
        = {} &
        [A(BC)]_{i,j}.
    \end{align*}

    (2)
    您还是要先说明, 等式二侧的阵的尺寸都是 \(m \times n\)
    (我留此为您的习题).
    根据阵的和与积的定义,
    \begin{align*}
        [(A + D)B]_{i,j}
        = {} &
        \sum_{k = 1}^{s} {[A + D]_{i,k} [B]_{k,j}}
        \\
        = {} &
        \sum_{k = 1}^{s} {([A]_{i,k} + [D]_{i,k})\, [B]_{k,j}}
        \\
        = {} &
        \sum_{k = 1}^{s} {([A]_{i,k} [B]_{k,j}
        + [D]_{i,k} [B]_{k,j})}
        \\
        = {} &
        \sum_{k = 1}^{s} {[A]_{i,k} [B]_{k,j}}
        +
        \sum_{k = 1}^{s} {[D]_{i,k} [B]_{k,j}}
        \\
        = {} &
        [AB]_{i,j} + [DB]_{i,j}
        \\
        = {} &
        [AB + DB]_{i,j}.
    \end{align*}

    (3)
    这跟分配律~1 的论证完全类似;
    我留此为您的习题.

    (4)
    先证 \(I_m A = A\).
    您还是要先说明, 等式二侧的阵的尺寸都是 \(m \times s\)
    (我留此为您的习题).
    回想, 单位阵 \(I\) 的 \((i, j)\)-元是
    \(\delta(i, j)\);
    具体地, \(i = j\) 时, 它是 \(1\),
    而 \(i \neq j\) 时, 它是 \(0\).
    根据阵的积的定义,
    \begin{align*}
        [I_m A]_{i,j}
        = {} &
        \sum_{u = 1}^{m} {[I_m]_{i,u} [A]_{u,j}}
        \\
        = {} &
        \sum_{u = 1}^{m} {\delta(i, u)\, [A]_{u,j}}
        \\
        = {} &
        \delta(i, i)\, [A]_{i,j}
        + \sum_{\substack{1 \leq u \leq m \\ u \neq i}}
        {\delta(i, u)\, [A]_{u,j}}
        \\
        = {} &
        1\,[A]_{i,j}
        + \sum_{\substack{1 \leq u \leq m \\ u \neq i}}
        {0\,[A]_{u,j}}
        \\
        = {} & [A]_{i,j}.
    \end{align*}
    请允许我留另一半为您的习题.

    (5)
    先证 \((xA)B = x(AB)\).
    您还是要先说明, 等式二侧的阵的尺寸都是 \(m \times n\)
    (我留此为您的习题).
    根据阵的数乘与阵的积的定义,
    \begin{align*}
        [(xA)B]_{i,j}
        = {} &
        \sum_{k = 1}^{s} {[xA]_{i,k} [B]_{k,j}}
        \\
        = {} &
        \sum_{k = 1}^{s} {x[A]_{i,k} [B]_{k,j}}
        \\
        = {} &
        x \sum_{k = 1}^{s} {[A]_{i,k} [B]_{k,j}}
        \\
        = {} &
        x [AB]_{i,j}
        \\
        = {} &
        [x(AB)]_{i,j}.
    \end{align*}
    请允许我留另一半为您的习题.
\end{proof}

以后, 我们可简单地写 \((AB)C\) 或 \(A(BC)\) 为 \(ABC\),
也可简单地写 \(x(AB)\), \((xA)B\), \(A(xB)\)
为 \(xAB\)
(当然, \(x\) 是一个数).

\vspace{2ex}

我以二个重要的事实结束本节.

\begin{theorem}
    设 \(B\) 是 \(s \times m\)~阵.
    设 \(A\) 是 \(m \times n\)~阵.
    设 \(A\) 的列~\(1\), \(2\), \(\dots\), \(n\)
    分别是 \(a_1\), \(a_2\), \(\dots\), \(a_n\)
    (于是, 我们可写 \(A = [a_1, a_2, \dots, a_n]\)).
    则
    \begin{align*}
        BA = [Ba_1, Ba_2, \dots, Ba_n].
    \end{align*}
\end{theorem}

\begin{proof}
    记 \(C = [Ba_1, Ba_2, \dots, Ba_n]\).
    每一个 \(Ba_j\) 都是 \(s \times 1\)~阵,
    故 \(C\) 是 \(s \times n\)~阵.
    当然, \(BA\) 也是 \(s \times n\)~阵.
    根据阵的积的定义,
    \begin{align*}
        [C]_{i,j}
        = {} & [Ba_j]_{i,1}
        \\
        = {} & \sum_{k = 1}^{m} {[B]_{i,k} [a_j]_{k,1}}
        \\
        = {} & \sum_{k = 1}^{m} {[B]_{i,k} [A]_{k,j}}
        \\
        = {} & [BA]_{i,j}.
        \qedhere
    \end{align*}
\end{proof}

\begin{theorem}
    设 \(A\) 是 \(m \times n\)~阵.
    设 \(A\) 的列~\(1\), \(2\), \(\dots\), \(n\)
    分别是 \(a_1\), \(a_2\), \(\dots\), \(a_n\)
    (于是, 我们可写 \(A = [a_1, a_2, \dots, a_n]\)).
    设 \(X\) 是一个 \(n \times 1\)~阵.
    则
    \begin{align*}
        AX = [X]_{1,1} a_1 + [X]_{2,1} a_2 + \dots
        + [X]_{n,1} a_n.
    \end{align*}
\end{theorem}

\begin{proof}
    \(AX\) 自然是一个 \(m \times 1\)~阵.
    不难看出, 待证等式的右侧也是 \(m \times 1\)~阵.
    根据阵的积、和、数乘的定义,
    \begin{align*}
        [AX]_{i,1}
        = {} & \sum_{k = 1}^{n} {[A]_{i,k} [X]_{k,1}}
        \\
        = {} & \sum_{k = 1}^{n} {[a_k]_{i,1} [X]_{k,1}}
        \\
        = {} & \sum_{k = 1}^{n} {[[X]_{k,1} a_k]_{i,1}}
        \\
        = {} & \left[ \sum_{k = 1}^{n} {[X]_{k,1}
        a_k } \right]_{i,1}.
        \qedhere
    \end{align*}
\end{proof}

\section{Binet--Cauchy 公式 (青春版)}

设 \(A\), \(B\) 都是 \(n\)~级阵.
那么, \(BA\) 当然也是 \(n\)~级阵.
\(A\), \(B\), \(BA\) 都是 \(n\)~级阵,
故, 它们都有行列式.

本节, 我们学习三者的行列式的关系.

\begin{theorem}[Binet--Cauchy 公式, 青春版]
    设 \(A\), \(B\) 都是 \(n\)~级阵.
    则
    \begin{align*}
        \det {(BA)} = \det {(B)} \det {(A)}.
    \end{align*}
\end{theorem}

我用一个例助您理解, 此定理在说什么.

\begin{example}
    设
    \begin{align*}
        A = \begin{bmatrix}
                2 & 5 \\
                3 & 7 \\
            \end{bmatrix},
        \quad
        B = \begin{bmatrix}
                9 & 6 \\
                4 & 8 \\
            \end{bmatrix}.
    \end{align*}
    不难算出
    \begin{align*}
        AB
        = \begin{bmatrix}
              38 & 52 \\
              55 & 74 \\
          \end{bmatrix},
        \quad
        BA
        = \begin{bmatrix}
              36 & 87 \\
              32 & 76 \\
          \end{bmatrix}.
    \end{align*}
    可以看到, \(AB \neq BA\).

    不过, \(\det {(AB)} = \det {(BA)}\).
    一方面, 我们可直接验证:
    \begin{align*}
         & \det {(AB)} = 38 \cdot 74 - 55 \cdot 52 = -48, \\
         & \det {(BA)} = 36 \cdot 76 - 32 \cdot 87 = -48.
    \end{align*}
    另一方面, Binet--Cauchy 公式 (青春版) 指出,
    \begin{align*}
        \det {(AB)}
        = {} & \det {(A)} \det {(B)} \\
        = {} & \det {(B)} \det {(A)} \\
        = {} & \det {(BA)}.
    \end{align*}
    毕竟, 数的乘法是可换的.
\end{example}

\begin{proof}
    考虑定义在全体 \(n\)~级阵上的函数
    \(f(C) = \det {(BC)}\),
    其中 \(C = [c_1, c_2, \dots, c_n]\) 是 \(n\)~级阵.

    (1)
    \(f\) 是多线性的.
    对任何不超过 \(n\) 的正整数 \(j\),
    任何 \(n-1\)~个 \(n \times 1\)~阵
    \(c_1\), \(\dots\), \(c_{j-1}\),
    \(c_{j+1}\), \(\dots\), \(c_n\),
    任何二个 \(n \times 1\)~阵 \(x\), \(y\),
    任何二个数 \(s\), \(t\),
    有
    \begin{align*}
             &
        f([\dots, c_{j-1}, sx + ty, c_{j+1}, \dots])
        \\
        = {} &
        \det {[\dots, Bc_{j-1}, B(sx + ty), Bc_{j+1}, \dots]}
        \\
        = {} &
        \det {[\dots, Bc_{j-1}, sBx + tBy, Bc_{j+1}, \dots]}
        \\
        = {} &
        s\det {[\dots, Bc_{j-1}, Bx, Bc_{j+1}, \dots]}
        +
        t\det {[\dots, Bc_{j-1}, By, Bc_{j+1}, \dots]}
        \\
        = {} &
        s f([\dots, c_{j-1}, x, c_{j+1}, \dots])
        +
        t f([\dots, c_{j-1}, y, c_{j+1}, \dots]).
    \end{align*}
    我们用到了阵的运算律与行列式的多线性.

    (2)
    \(f\) 是交错性的.
    因为若 \(c_1\), \(c_2\), \(\dots\), \(c_n\) 中有二个相等,
    则 \(Bc_1\), \(Bc_2\), \(\dots\), \(Bc_n\) 中也有二个相等.
    再利用行列式的交错性,
    \(f(C) = \det {(BC)} = 0\).

    所以, 对任何 \(n\)~级阵 \(A\),
    \(f(A) = f(I) \det {(A)}\).
    注意到 \(BI = B\),
    故
    \begin{equation*}
        \det {(BA)} = f(A) = f(I) \det {(A)}
        = \det {(B)} \det {(A)}.
        \qedhere
    \end{equation*}
\end{proof}

或许, (方) 阵的行列式与阵的积的定义都是较复杂的
(跟阵的转置、加、减、数乘等对比).
但是, Binet--Cauchy 公式 (的青春版) 给出了一个联系:
二个同级的方阵的积的行列式%
等于这二个阵的行列式的积.
% 这, 或许就是它们的一种关联.

\section{按一列 (行) 展开行列式的公式的变体}

首先, 我们回想,
我们是可,
按任何一列, 也可按任何一行,
展开一个阵的行列式的:

\TheoremExpansionAboutAnyColumn*

\TheoremExpansionAboutAnyRow*

或许, 您还记得约定:
当 \(A\) 是 \(1\)~级阵时,
形如 \(A(1|1)\) 的记号表示 ``\(0\)~级阵'',
且 ``\(0\)~级阵'' 的行列式为 \(1\).
此约定可使我们较简单地写公式.

若我们改变公式的几个文字,
则可得到不一样的结果:

\begin{theorem}
    设 \(A\) 是 \(n\)~级阵 (\(n \geq 1\)).
    设 \(k\), \(j\) 都是不超过 \(n\) 的正整数,
    且 \(k \neq j\).
    则
    \begin{align*}
        \sum_{\ell = 1}^{n}
        {(-1)^{\ell+k} [A]_{\ell,j} \det {(A(\ell|k))}}
        = 0.
    \end{align*}
    类似地, 若 \(i\), \(k\) 都是不超过 \(n\) 的正整数,
    且 \(i \neq k\),
    则
    \begin{align*}
        \sum_{s = 1}^{n}
        {(-1)^{k+s} [A]_{i,s} \det {(A(k|s))}}
        = 0.
    \end{align*}
\end{theorem}

我用一个例助您理解, 此定理在说什么.

\begin{example}
    设
    \begin{align*}
        A =
        \begin{bmatrix}
            1 & 5 & 9 \\
            2 & 6 & 7 \\
            3 & 4 & 8 \\
        \end{bmatrix}.
    \end{align*}
    不难算出,
    \begin{align*}
         & \det {(A(1|1))} = 6 \cdot 8 - 4 \cdot 7 = 20,  \\
         & \det {(A(2|1))} = 5 \cdot 8 - 4 \cdot 9 = 4,   \\
         & \det {(A(3|1))} = 5 \cdot 7 - 6 \cdot 9 = -19.
    \end{align*}
    那么, 这个定理说, 当 \(j = 2\) 或 \(j = 3\) 时,
    \begin{align*}
         & \hphantom{{} + {}}
        (-1)^{1+1} [A]_{1,j} \det {(A(1|1))}
        + (-1)^{2+1} [A]_{2,j} \det {(A(2|1))}
        \\
         &
        + (-1)^{3+1} [A]_{3,j} \det {(A(3|1))} = 0,
    \end{align*}
    即
    \begin{align*}
        20\,[A]_{1,j} - 4\,[A]_{2,j} - 19\,[A]_{3,j} = 0.
    \end{align*}

    我们用较直接的方法验证此事.
    取 \(j = 2\), 有
    \begin{align*}
        20 \cdot 5 - 4 \cdot 6 - 19 \cdot 4 = 100 - 24 - 76 = 0.
    \end{align*}
    取 \(j = 3\), 有
    \begin{align*}
        20 \cdot 9 - 4 \cdot 7 - 19 \cdot 8 = 180 - 28 - 152 = 0.
    \end{align*}
\end{example}

\begin{proof}
    我证式~1, 您证式~2.

    取 \(n\)~个数 \(z_1\), \(z_2\), \(\dots\), \(z_n\).
    我们作 \(n\)~级阵 \(B\), 其中
    \begin{align*}
        [B]_{\ell, s} =
        \begin{cases}
            [A]_{\ell, s}, & s \neq k; \\
            z_\ell,        & s = k.
        \end{cases}
    \end{align*}
    (通俗地, \(B\)~的列~\(s\) 等于 \(A\)~的列 \(s\)
    (\(s \neq k\)),
    而 \(B\)~的列~\(k\) 的元分别是
    \(z_1\), \(z_2\), \(\dots\), \(z_n\).)
    由此可见, \(B(\ell|k) = A(\ell|k)\)
    (\(\ell = 1\), \(2\), \(\dots\), \(n\)).
    所以
    \begin{align*}
        \det {(B)}
        = {} & \sum_{\ell = 1}^{n}
        {(-1)^{\ell + k} [B]_{\ell, k} \det {(B(\ell|k))}}
        \\
        = {} & \sum_{\ell = 1}^{n}
        {(-1)^{\ell + k} z_\ell \det {(A(\ell|k))}}.
    \end{align*}
    注意看.
    现在, 特别地, 我们取
    \(z_1\), \(z_2\), \(\dots\), \(z_n\)
    为
    \([A]_{1,j}\), \([A]_{2,j}\), \(\dots\), \([A]_{n,j}\).
    此时, \(B\) 有相同的二列
    (\(B\)~的列~\(j\) 即为 \(B\)~的列~\(k\),
    并注意到 \(k \neq j\)).
    所以 \(\det {(B)} = 0\).
    故
    \begin{equation*}
        \sum_{\ell = 1}^{n}
        {(-1)^{\ell+k} [A]_{\ell,j} \det {(A(\ell|k))}}
        = 0.
        \qedhere
    \end{equation*}
\end{proof}

利用 \(\delta\)-记号, 有
(我留此事的论证为您的习题):

\begin{theorem}
    设 \(A\) 是 \(n\)~级阵 (\(n \geq 1\)).
    若 \(k\), \(j\) 都是不超过 \(n\) 的正整数,
    则
    \begin{align*}
        \sum_{\ell = 1}^{n}
        {(-1)^{\ell+k} \det {(A(\ell|k))}\, [A]_{\ell,j}}
        = \det {(A)}\, \delta(k, j).
    \end{align*}
    类似地, 若 \(i\), \(k\) 都是不超过 \(n\) 的正整数,
    则
    \begin{align*}
        \sum_{s = 1}^{n}
        {[A]_{i,s} (-1)^{k+s} \det {(A(k|s))}}
        = \det{(A)}\, \delta(i, k).
    \end{align*}
\end{theorem}

% 若您足够细心,
您可能注意到,
我用数的乘法的结合律与交换律改写了公式.
% 的确, 这是有玄机的.
为什么?

\begin{definition}[古伴]
    设 \(A\) 是 \(n\)~级阵 (\(n \geq 1\)).
    定义 \(A\)~的\emph{古伴}
    (或者, \emph{古典伴随阵})
    为 \(n\)~级阵 \(\operatorname{adj} {(A)}\),
    其中, 对任何不超过 \(n\) 的正整数 \(i\), \(j\),
    \begin{align*}
        [\operatorname{adj} {(A)}]_{i,j}
        = (-1)^{j+i} \det {(A(j|i))}
    \end{align*}
    (注意等式右侧的 \(i\), \(j\) 的次序).
\end{definition}

有一件小事值得一提.
根据约定, 一个 \(1\)~级阵的古伴%
就是 \([1]\),
也就是 \(1\)~级单位阵.

\begin{example}
    设
    \begin{align*}
        A =
        \begin{bmatrix}
            1 & 5 & 9 \\
            2 & 6 & 7 \\
            3 & 4 & 8 \\
        \end{bmatrix}.
    \end{align*}
    我们计算 \(A\)~的古伴 \(\operatorname{adj} {(A)}\).
    根据定义, 我们要计算 \(9\)~个 \(2\)~级阵的行列式:
    \begin{align*}
         & \det {(A(1|1))} = 6 \cdot 8 - 4 \cdot 7 = 20,  \\
         & \det {(A(1|2))} = 2 \cdot 8 - 3 \cdot 7 = -5,  \\
         & \det {(A(1|3))} = 2 \cdot 4 - 3 \cdot 6 = -10, \\
         & \det {(A(2|1))} = 5 \cdot 8 - 4 \cdot 9 = 4,   \\
         & \det {(A(2|2))} = 1 \cdot 8 - 3 \cdot 9 = -19, \\
         & \det {(A(2|3))} = 1 \cdot 4 - 3 \cdot 5 = -11, \\
         & \det {(A(3|1))} = 5 \cdot 7 - 6 \cdot 9 = -19, \\
         & \det {(A(3|2))} = 1 \cdot 7 - 2 \cdot 9 = -11, \\
         & \det {(A(3|3))} = 1 \cdot 6 - 2 \cdot 5 = -4.
    \end{align*}
    所以
    \begin{align*}
             & \operatorname{adj} {(A)}
        \\
        = {} &
        \begin{bmatrix}
            (-1)^{1+1} \det {(A(1|1))}
             & (-1)^{2+1} \det {(A(2|1))}
             & (-1)^{3+1} \det {(A(3|1))} \\
            (-1)^{1+2} \det {(A(1|2))}
             & (-1)^{2+2} \det {(A(2|2))}
             & (-1)^{3+2} \det {(A(3|2))} \\
            (-1)^{1+3} \det {(A(1|3))}
             & (-1)^{2+3} \det {(A(2|3))}
             & (-1)^{3+3} \det {(A(3|3))} \\
        \end{bmatrix}
        \\
        = {} &
        \begin{bmatrix}
            +\det {(A(1|1))}
             & -\det {(A(2|1))}
             & +\det {(A(3|1))} \\
            -\det {(A(1|2))}
             & +\det {(A(2|2))}
             & -\det {(A(3|2))} \\
            +\det {(A(1|3))}
             & -\det {(A(2|3))}
             & +\det {(A(3|3))} \\
        \end{bmatrix}
        \\
        = {} &
        \begin{bmatrix}
            20  & -4  & -19 \\
            5   & -19 & 11  \\
            -10 & 11  & -4  \\
        \end{bmatrix}.
    \end{align*}
\end{example}

\begin{theorem}
    设 \(A\) 是 \(n\)~级阵 (\(n \geq 1\)).
    设 \(\operatorname{adj} {(A)}\) 为其古伴.
    则
    \begin{align*}
        \operatorname{adj} {(A)}\, A
        = \det {(A)}\, I
        = A \operatorname{adj} {(A)}.
    \end{align*}
\end{theorem}

\begin{proof}
    首先, 二个等式的左右二侧都是 \(n\)~级阵.

    由上个定理 (的前半部分),
    \begin{align*}
        [\operatorname{adj} {(A)}\, A]_{i,j}
        = {} &
        \sum_{\ell = 1}^{n}
        {[\operatorname{adj} {(A)}]_{i,\ell} [A]_{\ell,j}}
        \\
        = {} &
        \sum_{\ell = 1}^{n}
        {(-1)^{\ell+i} \det {(A(\ell|i))}\, [A]_{\ell,j}}
        \\
        = {} & \det {(A)}\, \delta(i, j)
        \\
        = {} & \det {(A)}\, [I]_{i,j}
        \\
        = {} & [\det {(A)}\, I]_{i,j}.
    \end{align*}
    同理,
    \begin{align*}
        [A \operatorname{adj} {(A)}]_{i,j}
        = {} &
        \sum_{s = 1}^{n}
        {[A]_{i,s} [\operatorname{adj} {(A)}]_{s,j}}
        \\
        = {} &
        \sum_{s = 1}^{n}
        {[A]_{i,s} (-1)^{j+s} \det {(A(j|s))}}
        \\
        = {} & \det {(A)}\, \delta(i, j)
        \\
        = {} & \det {(A)}\, [I]_{i,j}
        \\
        = {} & [\det {(A)}\, I]_{i,j}.
        \qedhere
    \end{align*}
\end{proof}

\begin{example}
    设
    \begin{align*}
        A =
        \begin{bmatrix}
            1 & 5 & 9 \\
            2 & 6 & 7 \\
            3 & 4 & 8 \\
        \end{bmatrix}.
    \end{align*}
    我们知道, \(A\) 的行列式是 \(-45\).
    我们还知道, \(A\)~的古伴
    \begin{align*}
        \operatorname{adj} {(A)} =
        \begin{bmatrix}
            20  & -4  & -19 \\
            5   & -19 & 11  \\
            -10 & 11  & -4  \\
        \end{bmatrix}.
    \end{align*}
    可以验证
    \begin{align*}
        \operatorname{adj} {(A)}\, A
        = -45I
        = A\operatorname{adj} {(A)}.
    \end{align*}
\end{example}

设 \(A\) 是 \(n\)~级阵,
且其行列式 \(\det {(A)} \neq 0\).
作 \(n\)~级阵
\begin{align*}
    B = \frac{1}{\det {(A)}} \operatorname{adj} {(A)}.
\end{align*}
从而
\begin{align*}
    BA
    = {} & \Big( \frac{1}{\det {(A)}}
    \operatorname{adj} {(A)} \Big) A
    \\
    = {} & \frac{1}{\det {(A)}} (\operatorname{adj} {(A)}\, A)
    \\
    = {} & \frac{1}{\det {(A)}} (\det {(A)}\, I)
    \\
    = {} & \Big( \frac{1}{\det {(A)}} \det {(A)} \Big) I
    \\
    = {} & 1I
    \\
    = {} & I.
\end{align*}
类似地, 可知 \(AB = I\).
所以, 我们有

\begin{theorem}
    设 \(A\) 是 \(n\)~级阵 (\(n \geq 1\)).
    若 \(\det {(A)} \neq 0\),
    则存在 \(n\)~级阵 \(B\) 使
    \begin{align*}
        BA = I = AB.
    \end{align*}
\end{theorem}

反之, 若有 \(n\)~级阵 \(B\) 使 \(BA = I = AB\),
则,
因为二个同级的方阵的积的行列式%
等于这二个阵的行列式的积,
\begin{align*}
    1 = \det {(I)} = \det {(AB)} = \det {(A)} \det {(B)}.
\end{align*}
从而 \(\det {(A)} \neq 0\).

我们熟知, 任取不等于 \(0\) 的数 \(a\),
必有一个数 \(b\) 使 \(ba = 1 = ab\);
反过来, 若 \(ba = 1 = ab\), 则 \(a \neq 0\).
这么看来,
行列式非零的阵跟非零的数 ``像'';
甚至, 特别地,
行列式非零的 \(1\)~级阵就是非零的数.

% Systems of linear equations.
% "\include" cannot be nested.
% Use "\input" instead.
% The following text
% was intended to be directly inserted in "determinantoj.tex".
% However, because it is too long,
% and it is not really on determinants,
% I decide to create a new file to contain it.

\section{线性方程组}

在接下来的若干节里, 我想用行列式讨论%
线性方程组的解.
这是行列式的一个应用.
或许, 您可以在这些讨论里体会到,
``行列式是一个\emph{工具}''
``行列式是方阵的一个\emph{属性}''
的意思.

本节, 我们学习一些基本的概念.

设 \(a_1\), \(a_2\), \(\dots\), \(a_n\) 是 \(n\)~个常数,
\(c\) 是常数,
\(x_1\), \(x_2\), \(\dots\), \(x_n\) 是 \(n\)~个未知数.
我们说, 形如
\begin{align*}
    a_1 x_1 + a_2 x_2 + \dots + a_n x_n + c
\end{align*}
的式是一个 \emph{\(n\)~元 \({\leq} 1\)~次式}.
如果 \(a_1\), \(a_2\), \(\dots\), \(a_n\)
中有一个数不为零,
我们说, 这个式是一个 \emph{\(n\)~元 \(1\)~次式}.
如果 \(a_1\), \(a_2\), \(\dots\), \(a_n\)
全为零,
那么, 这个式就是一个常数.

形如
\begin{align*}
    a_1 x_1 + a_2 x_2 + \dots + a_n x_n + c = 0
\end{align*}
的方程是一个 \emph{\(n\)~元 \({\leq} 1\)~次方程}.
不过, 习惯地, 我们移常数项 \(c\) 到等式的右侧.
也就是说, 形如
\begin{align*}
    a_1 x_1 + a_2 x_2 + \dots + a_n x_n = b
\end{align*}
的方程也是一个 \(n\)~元 \({\leq} 1\)~次方程,
其中 \(b\) 就是常数 \(-c\).
若我们不想强调未知数之数~\(n\),
我们说, 这是一个 \emph{\({\leq} 1\)~次方程};
我们也可说, 这是一个\emph{线性方程}.

\emph{由 \(m\)~个 \(n\)~元 \({\leq} 1\)~次方程作成的方程组},
是形如
\begin{align*}
    \begin{cases}
        a_{1,1} x_1 + a_{1,2} x_2 + \dots + a_{1,n} x_n = b_1, \\
        a_{2,1} x_1 + a_{2,2} x_2 + \dots + a_{2,n} x_n = b_2, \\
        \dots \dots \dots \dots
        \dots \dots \dots \dots
        \dots \dots \dots \dots
        ,                                                      \\
        a_{m,1} x_1 + a_{m,2} x_2 + \dots + a_{m,n} x_n = b_m  \\
    \end{cases}
\end{align*}
的方程组,
其中
\(a_{1,1}\), \(a_{1,2}\), \(\dots\), \(a_{1,n}\),
\(a_{2,1}\), \(a_{2,2}\), \(\dots\), \(a_{2,n}\),
\(\dots\),
\(a_{m,1}\), \(a_{m,2}\), \(\dots\), \(a_{m,n}\),
\(b_1\), \(b_2\), \(\dots\), \(b_m\)
是事先指定的 \(mn + m\) 个数,
而 \(x_1\), \(x_2\), \(\dots\), \(x_n\)
都是未知数.
若我们不想强调方程数~\(m\),
我们可说, 这是一个 \emph{\(n\)~元 \({\leq} 1\)~次方程组};
若我们不想强调未知数之数~\(n\),
我们可说, 这是一个%
\emph{由 \(m\)~个 \({\leq} 1\)~次方程作成的方程组}.
既然 \({\leq} 1\)~次方程的另一个名字是线性方程,
我们也可说, 这是一个\emph{线性方程组}.

若数 \(c_1\), \(c_2\), \(\dots\), \(c_n\) 适合
\begin{align*}
    a_{1,1} c_1 + a_{1,2} c_2 + \dots + a_{1,n} c_n = b_1, \\
    a_{2,1} c_1 + a_{2,2} c_2 + \dots + a_{2,n} c_n = b_2, \\
    \dots \dots \dots \dots
    \dots \dots \dots \dots
    \dots \dots \dots \dots
    ,                                                      \\
    a_{m,1} c_1 + a_{m,2} c_2 + \dots + a_{m,n} c_n = b_m,
\end{align*}
我们说, \((c_1, c_2, \dots, c_n)\)
是此方程组的一个\emph{解}.
有时, 我们也说, 形如
\(x_1 = c_1\),
\(x_2 = c_2\),
\(\dots\),
\(x_n = c_n\)
的 \(n\)~个等式 (的联合)
是此方程组的一个解.

若 \(c_1 = c_2 = \dots = c_n = 0\),
且 \((c_1, c_2, \dots, c_n)\) 是此方程组的一个解,
我们说, 这是此方程组的\emph{零解}.
若 \(c_1\), \(c_2\), \(\dots\), \(c_n\) 不全为零,
且 \((c_1, c_2, \dots, c_n)\) 是此方程组的一个解,
我们说, 这是此方程组的一个\emph{非零解}.

\begin{example}
    假定有若干只鸡与若干只兔被关在某处.
    假定每一只鸡有 \(1\)~个头与 \(2\)~只腿;
    假定每一只兔有 \(1\)~个头与 \(4\)~只腿.
    假定, 我们知道,
    这些鸡与兔一共有 \(35\)~个头与 \(94\)~只腿.
    我们能由此算出鸡与兔的数目吗?

    我们代数地思考此事.
    设有 \(x_1\)~只鸡, 有 \(x_2\)~只兔.
    那么,
    这些鸡与兔%
    一共有 \(x_1 + x_2\)~个头,
    与 \(2x_1 + 4x_2\)~只腿.
    注意到, 这二个式都是 \(2\)~元 \({\leq} 1\)~次式.
    我们可列出%
    由 \(2\)~个 \(2\)~元 \({\leq} 1\)~次方程作成的方程组
    \begin{align*}
        \begin{cases}
            x_1 + x_2 = 35, \\
            2x_1 + 4x_2 = 94.
        \end{cases}
    \end{align*}
    接下来的问题, 就是解这个方程组.
    不过,
    % 不要急;
    此例的目的是使您熟悉概念.
    我想之后讲如何解这个方程组.

    可以验证, \((23, 12)\) 是此方程组的一个解:
    因为 \(23 + 12 = 35\),
    且 \(2 \cdot 23 + 4 \cdot 12 = 94\).
    因为 \(23\), \(12\) 里有一个数不是零,
    故 \((23, 12)\) 是此方程组的一个非零解.
    我们也可说,
    \(x_1 = 23\),
    \(x_2 = 12\)
    是此方程组的一个 (非零) 解.

    不过, \((12, 23)\) 不是此方程组的一个解:
    虽然 \(12 + 23 = 35\),
    但是 \(2 \cdot 12 + 4 \cdot 23 = 116 \neq 94\).
\end{example}

\begin{example}
    考虑%
    由 \(2\)~个 \(3\)~元 \({\leq} 1\)~次方程作成的方程组
    \begin{align*}
        \begin{cases}
            x_1 + 2x_2 + 3x_3 = 0, \\
            4x_1 + 5x_2 + 6x_3 = 0.
        \end{cases}
    \end{align*}
    可以验证, 对每一个数~\(k\),
    \((k, -2k, k)\) 是此方程组的一个解:
    \begin{align*}
        k + 2(-2k) + 3k = k - 4k + 3k = 0, \\
        4k + 5(-2k) + 6k = 4k - 10k + 6k = 0.
    \end{align*}
    当 \(k = 0\) 时, 这是零解;
    当 \(k \neq 0\) 时, 因为
    \(k\), \(-2k\), \(k\) 里有一个数 \(k\) 不是零,
    故这是一个非零解.

    我们也可说,
    \(x_1 = k\),
    \(x_2 = -2k\),
    \(x_3 = k\)
    是此方程组的一个解.
\end{example}

利用阵的积, 我们可简单地写一个线性方程组.
设
\begin{align*}
    A =
    \begin{bmatrix}
        a_{1,1} & a_{1,2} & \cdots & a_{1,n} \\
        a_{2,1} & a_{2,2} & \cdots & a_{2,n} \\
        \vdots  & \vdots  & {}     & \vdots  \\
        a_{m,1} & a_{m,2} & \cdots & a_{m,n} \\
    \end{bmatrix},
    \quad
    X =
    \begin{bmatrix}
        x_1 \\ x_2 \\ \vdots \\ x_n
    \end{bmatrix},
    \quad
    B =
    \begin{bmatrix}
        b_1 \\ b_2 \\ \vdots \\ b_m
    \end{bmatrix}.
\end{align*}
则, 对不超过 \(m\) 的正整数~\(i\),
\begin{align*}
    [B]_{i,1}
    = {} &
    b_i
    \\
    = {} &
    a_{i,1} x_1 + a_{i,2} x_2 + \dots + a_{i,n} x_n
    \\
    = {} &
    [A]_{i,1} [X]_{1,1} + [A]_{i,2} [X]_{2,1}
    + \dots + [A]_{i,n} [X]_{n,1}
    \\
    = {} &
    [AX]_{i,1}.
\end{align*}
注意到 \(AX\) 的尺寸也是 \(m \times 1\), 故
\begin{align*}
    AX = B.
\end{align*}
所以, 我们也说形如 \(AX = B\) 的阵等式
(\(X\) 的元是未知数, \(X\) 是恰有 \(1\)~列的阵)
是一个线性方程组.
相应地, 若 \(n \times 1\)~阵 \(C\)
(其元跟 \(X\)~的元相比, 自然都是已知数)
适合 \(AC = B\),
我们说, \(C\) 是此方程组的一个解;
若 \(C = 0\)
(也就是说, \(C\) 的每一个元都是零)
是此方程组的一个解,
我们说, \(C\) 是此方程组的零解;
若不等于零的 \(C\)
(也就是说, \(C\) 有一个不为零的元)
是此方程组的一个解,
我们说, \(C\) 是此方程组的一个非零解.

\begin{example}
    我们可改写
    \begin{align*}
        \begin{cases}
            x_1 + x_2 = 35, \\
            2x_1 + 4x_2 = 94.
        \end{cases}
    \end{align*}
    为
    \begin{align*}
        \begin{bmatrix}
            1 & 1 \\
            2 & 4 \\
        \end{bmatrix}
        \begin{bmatrix}
            x_1 \\
            x_2 \\
        \end{bmatrix}
        =
        \begin{bmatrix}
            35 \\
            94 \\
        \end{bmatrix}.
    \end{align*}
    可以验证,
    \(
    C = \begin{bmatrix}
        23 \\
        12 \\
    \end{bmatrix}
    \)
    是此方程组的一个非零解:
    \(C \neq 0\), 且
    \begin{align*}
        \begin{bmatrix}
            1 & 1 \\
            2 & 4 \\
        \end{bmatrix}
        \begin{bmatrix}
            23 \\
            12 \\
        \end{bmatrix}
        =
        \begin{bmatrix}
            1 \cdot 23 + 1 \cdot 12 \\
            2 \cdot 23 + 4 \cdot 12 \\
        \end{bmatrix}
        =
        \begin{bmatrix}
            35 \\
            94 \\
        \end{bmatrix}.
    \end{align*}

    不过,
    \(
    D = \begin{bmatrix}
        12 \\
        23 \\
    \end{bmatrix}
    \)
    不是此方程组的一个解:
    \begin{align*}
        \begin{bmatrix}
            1 & 1 \\
            2 & 4 \\
        \end{bmatrix}
        \begin{bmatrix}
            12 \\
            23 \\
        \end{bmatrix}
        =
        \begin{bmatrix}
            1 \cdot 12 + 1 \cdot 23 \\
            2 \cdot 12 + 4 \cdot 23 \\
        \end{bmatrix}
        =
        \begin{bmatrix}
            35  \\
            116 \\
        \end{bmatrix}
        \neq
        \begin{bmatrix}
            35 \\
            94 \\
        \end{bmatrix}.
    \end{align*}
\end{example}

\begin{example}
    我们可改写
    \begin{align*}
        \begin{cases}
            x_1 + 2x_2 + 3x_3 = 0, \\
            4x_1 + 5x_2 + 6x_3 = 0.
        \end{cases}
    \end{align*}
    为
    \begin{align*}
        \begin{bmatrix}
            1 & 2 & 3 \\
            4 & 5 & 6 \\
        \end{bmatrix}
        \begin{bmatrix}
            x_1 \\
            x_2 \\
            x_3 \\
        \end{bmatrix}
        =
        \begin{bmatrix}
            0 \\
            0 \\
        \end{bmatrix}.
    \end{align*}
    可以验证, 对每一个数~\(k\),
    \(
    C = k\begin{bmatrix}
        1  \\
        -2 \\
        1  \\
    \end{bmatrix}
    \)
    是此方程组的一个解:
    \begin{align*}
        \begin{bmatrix}
            1 & 2 & 3 \\
            4 & 5 & 6 \\
        \end{bmatrix}
        \left(
        k\begin{bmatrix}
                 1  \\
                 -2 \\
                 1  \\
             \end{bmatrix}
        \right)
        = {} &
        k
        \left(
        \begin{bmatrix}
                1 & 2 & 3 \\
                4 & 5 & 6 \\
            \end{bmatrix}
        \begin{bmatrix}
                1  \\
                -2 \\
                1  \\
            \end{bmatrix}
        \right)
        \\
        = {} &
        k
        \begin{bmatrix}
            1 \cdot 1 + 2 \cdot (-2) + 3 \cdot 1 \\
            4 \cdot 1 + 5 \cdot (-2) + 6 \cdot 1 \\
        \end{bmatrix}
        \\
        = {} &
        k
        \begin{bmatrix}
            0 \\
            0 \\
        \end{bmatrix}
        =
        \begin{bmatrix}
            0 \\
            0 \\
        \end{bmatrix}.
    \end{align*}
    当 \(k = 0\) 时, 它是零解;
    当 \(k \neq 0\) 时, 它是一个非零解.
\end{example}

最后, 有一件小事值得一提.
前面, 我们写一个%
由 \(m\)~个 \(n\)~元 \({\leq} 1\)~次方程作成的方程组%
为阵等式.
反过来,
设 \(A\) 是 \(m \times n\)~阵,
\(B\) 是 \(m \times 1\)~阵,
\(X\) 是未知的 \(n \times 1\)~阵.
那么, 我们也可还原形如 \(AX = B\) 的阵等式为%
由 \(m\)~个 \(n\)~元 \({\leq} 1\)~次方程作成的方程组.
比如, 我们可写
\begin{align*}
    \begin{bmatrix}
        2 & 3  & -5 & 7  & 0   \\
        0 & 1  & 11 & 13 & -17 \\
        4 & -6 & 8  & -9 & 12  \\
    \end{bmatrix}
    \begin{bmatrix}
        x_1 \\
        x_2 \\
        x_3 \\
        x_4 \\
        x_5 \\
    \end{bmatrix}
    =
    \begin{bmatrix}
        7 \\
        8 \\
        9 \\
    \end{bmatrix}
\end{align*}
为
\begin{align*}
    \begin{cases}
        2x_1 + 3x_2 - 5x_3 + 7x_4 = 7,   \\
        x_2 + 11x_3 + 13x_4 - 17x_5 = 8, \\
        4x_1 - 6x_2 + 8x_3 - 9x_4 + 12x_5 = 9.
    \end{cases}
\end{align*}
(注意到, 我们未写系数为 \(0\) 的项.)

\section{\texorpdfstring{由 \(n\)~个 \(n\)~元
      \({\leq} 1\)~次方程作成的方程组 (1)}%
  {由 n 个 n 元 ≤1 次方程作成的方程组 (1)}}

在上节, 我们学习了线性方程组的一些基本的概念.
并且, 我们知道, 可用阵等式简单地写一个线性方程组
(或者说, 形如 \(AX = B\) 的阵等式就是一个线性方程组).

从本节起, 我们讨论%
由 \(n\)~个 \(n\)~元 \({\leq} 1\)~次方程作成的方程组.
用阵等式表示这种方程组,
就是 \(AX = B\),
其中 \(A\) 是 \(n\)~级阵,
\(B\) 是 \(n \times 1\)~阵,
\(X\) 是未知的 \(n \times 1\)~阵.
\(A\) 是方阵, 故它有行列式.
\(AX = B\) 的解跟 \(A\)~的行列式是否有关系?
此事的回答是 ``是''.
本节, 我们讨论一种特别的情形.

\begin{theorem}[Cramer 公式, 1]
    设 \(A\) 是 \(n\)~级阵 (\(n \geq 1\)).
    设 \(B\) 是 \(n \times 1\)~阵.
    设 \(X\) 是未知的 \(n \times 1\)~阵.
    若 \(\det {(A)} \neq 0\),
    则线性方程组 \(AX = B\) 有唯一的解
    \begin{align*}
        X = \frac{1}{\det {(A)}} \operatorname{adj} {(A)}\,B.
    \end{align*}
\end{theorem}

在论证此事前, 我想用一个简单的例助您理解它.

\begin{example}
    设 \(a\), \(b\) 是常数.
    一个 \(1\)~元 \({\leq} 1\)~次方程
    \(ax = b\)
    当然也是一个线性方程组:
    这是方程数为 \(1\) 时的特别情形.
    我们在中学就知道, 当 \(a \neq 0\) 时,
    \(ax = b\) 有唯一的解 \(x = \frac{1}{a}\, b\).
    一方面, \(x = \frac{1}{a}\, b\) 是一个解:
    \begin{align*}
        a \Big(\frac{1}{a}\, b \Big)
        = \Big(a\, \frac{1}{a} \Big) b
        = 1b
        = b.
    \end{align*}
    另一方面, 若数~\(y\) 也适合 \(ay = b\), 则
    \begin{align*}
        y
        = 1y
        = \Big(\frac{1}{a}\, a\Big)y
        = \frac{1}{a} (ay)
        = \frac{1}{a}\, b.
    \end{align*}

    我们说, 此事是 Cramer 公式的一个特例.
    首先, 我们可写 \(ax = b\)
    为阵等式 \([a]\, [x] = [b]\).
    (注意到, \([a]\), \([x]\), \([b]\) 都是 \(1\)~级阵.)
    Cramer 公式说, 若 \(\det {[a]} \neq 0\),
    则 \([a]\, [x] = [b]\)
    有唯一的解
    (注意, \(1\)~级阵的古伴是 \([1]\))
    \begin{align*}
        [x] = \frac{1}{\det {[a]}}
        \operatorname{adj} {([a])}\,[b]
        = \frac{1}{a}\, [1]\,[b]
        = \frac{1}{a}\, [b]
        = \Big[ \frac{1}{a}\, b \Big].
    \end{align*}
    这跟我们已知的结论是一样的.
\end{example}

\begin{proof}
    我们先验证
    \(C = \frac{1}{\det {(A)}} \operatorname{adj} {(A)}\,B\)
    适合 \(AC = B\):
    \begin{align*}
        AC
        = {} & A\Big( \frac{1}{\det {(A)}}
        \operatorname{adj} {(A)}\,B \Big)
        \\
        = {} & \frac{1}{\det {(A)}} (A\operatorname{adj} {(A)}\,B)
        \\
        = {} & \frac{1}{\det {(A)}} (\det {(A)}\,I_n B)
        \\
        = {} & B.
    \end{align*}

    我们再证 \(AX = B\) 至多有一个解.
    设 \(n \times 1\)~阵 \(D\) 也适合 \(AD = B\).
    则
    \begin{align*}
        D
        = {} & I_n D
        \\
        = {} & \Big( \frac{1}{\det {(A)}}
        \operatorname{adj} {(A)}\,A \Big) D
        \\
        = {} & \frac{1}{\det {(A)}}
        \operatorname{adj} {(A)}\, (AD)
        \\
        = {} & \frac{1}{\det {(A)}}
        \operatorname{adj} {(A)}\,B
        \\
        = {} & C.
        \qedhere
    \end{align*}
\end{proof}

\begin{example}
    考虑线性方程组 \(AX = B\),
    其中
    \begin{align*}
        A = \begin{bmatrix}
                1 & 1 \\
                2 & 4 \\
            \end{bmatrix},
        \quad
        X =
        \begin{bmatrix}
            x_1 \\
            x_2 \\
        \end{bmatrix},
        \quad
        B =
        \begin{bmatrix}
            35 \\
            94 \\
        \end{bmatrix}.
    \end{align*}
    不难算出
    \begin{align*}
        \det {(A)} = 1 \cdot 4 - 2 \cdot 1 = 2 \neq 0,
    \end{align*}
    所以, 此方程组有唯一的解
    \begin{align*}
        X
        = {} &
        \frac{1}{\det {(A)}} \operatorname{adj} {(A)}\,B
        \\
        = {} &
        \frac{1}{2}
        \begin{bmatrix}
            4  & -1 \\
            -2 & 1  \\
        \end{bmatrix}
        \begin{bmatrix}
            35 \\
            94 \\
        \end{bmatrix}
        =
        \frac{1}{2}
        \begin{bmatrix}
            4 \cdot 35 - 1 \cdot 94  \\
            -2 \cdot 35 + 1 \cdot 94 \\
        \end{bmatrix}
        \\
        = {} &
        \frac{1}{2}
        \begin{bmatrix}
            46 \\
            24 \\
        \end{bmatrix}
        =
        \begin{bmatrix}
            23 \\
            12 \\
        \end{bmatrix}.
    \end{align*}
\end{example}

我们可进一步地改写 Cramer 公式.
我们先具体地算出
\(\operatorname{adj} {(A)}\,B\)~的每一个元:
\begin{align*}
    [\operatorname{adj} {(A)}\,B]_{i,1}
    = {} &
    \sum_{\ell = 1}^{n}
    {[\operatorname{adj} {(A)}]_{i,\ell} [B]_{\ell,1}}
    \\
    = {} &
    \sum_{\ell = 1}^{n}
    {(-1)^{\ell+i} \det {(A(\ell|i))}\, [B]_{\ell,1}}.
\end{align*}
设 \(A \{i, B\}\) 是%
以 \(B\) 代阵~\(A\) 的列~\(i\) 后所得的阵,
即
\begin{align*}
    [A \{i, B\}]_{\ell,k}
    = \begin{cases}
          [A]_{\ell,k}, & k \neq i; \\
          [B]_{\ell,1}, & k = i.
      \end{cases}
\end{align*}
由此可见,
\(A(\ell|i) = (A \{i, B\})(\ell|i)\)
(\(\ell = 1\), \(2\), \(\dots\), \(n\)).
所以
\begin{align*}
         &
    \sum_{\ell = 1}^{n}
    {(-1)^{\ell+i} \det {(A(\ell|i))}\, [B]_{\ell,1}}
    \\
    = {} &
    \sum_{\ell = 1}^{n}
    {(-1)^{\ell+i} \det {((A \{i, B\})(\ell|i))}\,
    [A \{i, B\}]_{\ell,1}}
    \\
    = {} &
    \det {(A \{i, B\})}.
\end{align*}
从而
\begin{align*}
    [X]_{i,1}
    = {} &
    \Big[ \frac{1}{\det {(A)}}
        \operatorname{adj} {(A)}\,B \Big]_{i,1}
    \\
    = {} &
    \frac{1}{\det {(A)}}
    [ \operatorname{adj} {(A)}\,B ]_{i,1}
    \\
    = {} &
    \frac{1}{\det {(A)}}
    \det {(A \{i, B\})}
    \\
    = {} &
    \frac{\det {(A \{i, B\})}}{\det {(A)}}.
\end{align*}
由此, 我们得到 Cramer 公式的另一个形式:

\begin{theorem}[Cramer 公式, 2]
    设线性方程组
    \begin{align*}
        \begin{cases}
            a_{1,1} x_1 + a_{1,2} x_2 + \dots
            + a_{1,n} x_n = b_1, \\
            a_{2,1} x_1 + a_{2,2} x_2 + \dots
            + a_{2,n} x_n = b_2, \\
            \dots \dots \dots \dots
            \dots \dots \dots \dots
            \dots \dots \dots \dots,
            \\
            a_{n,1} x_1 + a_{n,2} x_2 + \dots
            + a_{n,n} x_n = b_n. \\
        \end{cases}
    \end{align*}
    (注意, 方程数等于未知数之数.)
    记
    \begin{align*}
        A =
        \begin{bmatrix}
            a_{1,1} & a_{1,2} & \cdots & a_{1,n} \\
            a_{2,1} & a_{2,2} & \cdots & a_{2,n} \\
            \vdots  & \vdots  & {}     & \vdots  \\
            a_{n,1} & a_{n,2} & \cdots & a_{n,n} \\
        \end{bmatrix},
        \quad
        B =
        \begin{bmatrix}
            b_1 \\ b_2 \\ \vdots \\ b_n
        \end{bmatrix}.
    \end{align*}
    若 \(\det {(A)} \neq 0\),
    则此线性方程组有唯一的解
    \begin{align*}
        x_i = \frac{\det {(A \{i, B\})}}{\det {(A)}},
        \quad
        \text{\(i = 1\), \(2\), \(\dots\), \(n\)},
    \end{align*}
    其中 \(A \{i, B\}\) 是%
    以 \(B\) 代阵~\(A\) 的列~\(i\) 后所得的阵.
\end{theorem}

\begin{example}
    考虑由 \(2\)~个 \(2\)~元 \({\leq} 1\)~次方程作成的方程组
    \begin{align*}
        \begin{cases}
            a_{1,1} x_1 + a_{1,2} x_2 = b_1, \\
            a_{2,1} x_1 + a_{2,2} x_2 = b_2. \\
        \end{cases}
    \end{align*}
    根据 Cramer 公式, 若
    \(d = \det {
        \begin{bmatrix}
            a_{1,1} & a_{1,2} \\
            a_{2,1} & a_{2,2} \\
        \end{bmatrix}
    }
    = a_{1,1} a_{2,2} - a_{2,1} a_{1,2} \neq 0\),
    则此方程组有唯一的解
    \begin{align*}
        x_1
        = \frac{\det {
                \begin{bmatrix}
                    b_{1} & a_{1,2} \\
                    b_{2} & a_{2,2} \\
                \end{bmatrix}
            }}{d}
        = \frac{b_1 a_{2,2} - b_2 a_{1,2}}{d},
        \\
        x_2
        = \frac{\det {
                \begin{bmatrix}
                    a_{1,1} & b_{1} \\
                    a_{2,1} & b_{2} \\
                \end{bmatrix}
            }}{d}
        = \frac{a_{1,1} b_2 - a_{2,1} b_1}{d}.
    \end{align*}
\end{example}

\begin{example}
    考虑线性方程组
    \begin{align*}
        \begin{cases}
            x_1 + x_2 = 35, \\
            2x_1 + 4x_2 = 94.
        \end{cases}
    \end{align*}
    因为 \(d = 1 \cdot 4 - 2 \cdot 1 = 2 \neq 0\),
    故, 由上个例,
    此方程组有唯一的解
    \begin{align*}
        x_1 = \frac{35 \cdot 4 - 94 \cdot 1}{2} = 23,
        \\
        x_2 = \frac{1 \cdot 94 - 2 \cdot 35}{2} = 12.
    \end{align*}
\end{example}

一般地, Cramer 公式, 只是一个理论的公式.
这是因为, 一般地, 计算行列式是较复杂的事
(或许, 您还记得,
\(3\)~级阵的行列式的较具体的公式含 \(6\)~项,
而 \(4\)~级阵的行列式的较具体的公式含 \(24\)~项).
所以, 我们一般用别的方法
(如代入消元法、加减消元法等)
解线性方程组.

还是以
\begin{align*}
    \begin{cases}
        x_1 + x_2 = 35, \\
        2x_1 + 4x_2 = 94
    \end{cases}
\end{align*}
为例.
由方程~\(1\), 有 \(x_1 = 35 - x_2\).
代入其到方程~\(2\), 有
\(2(35 - x_2) + 4x_2 = 94\),
即 \(2x_2 = 24\).
由此可知 \(x_2 = 12\).
代入其到 \(x_1 = 35 - x_2\),
有 \(x_1 = 23\).
最后, 经验证, \((23, 12)\) 的确是此方程组的一个解.

\KunAsteriskoEnEnhavtabelo

\section{\texorpdfstring{由 \(n\)~个 \(n\)~元
      \({\leq} 1\)~次方程作成的方程组 (2)}%
  {由 n 个 n 元 ≤1 次方程作成的方程组 (2)}}

\maldevigalegajxo

在上节, 我们\emph{定量地}研究了%
由 \(n\)~个 \(n\)~元 \({\leq} 1\)~次方程作成的方程组%
的解.
具体地, 当 \(n\)~级阵 \(A\) 的行列式不为零时,
我们用行列式写出了 \(AX = B\) 的唯一的解,
其中
\(B\) 是 \(n \times 1\)~阵,
\(X\) 是未知的 \(n \times 1\)~阵.
但是, 若 \(\det {(A)} = 0\),
则 Cramer 公式不可用
(因为分母不可为零).
其实, 当 \(\det {(A)} = 0\) 时,
\(AX = B\) 是否有解是一个较复杂的问题.

\begin{example}
    设
    \begin{align*}
        A = \begin{bmatrix}
                1  & 1  \\
                -1 & -1 \\
            \end{bmatrix},
        \quad
        X =
        \begin{bmatrix}
            x_1 \\
            x_2 \\
        \end{bmatrix},
        \quad
        B =
        \begin{bmatrix}
            1 \\
            b \\
        \end{bmatrix},
    \end{align*}
    其中 \(b\) 是常数.
    不难算出, \(\det {(A)} = 0\).
    当 \(b\) 取某些数时,
    我们讨论 \(AX = B\) 的解.

    当 \(b = -1\) 时, 此方程组有解:
    \begin{align*}
        A
        \begin{bmatrix}
            1 \\
            0 \\
        \end{bmatrix}
        =
        \begin{bmatrix}
            1 \cdot 1 + 1 \cdot 0     \\
            -1 \cdot 1 + (-1) \cdot 0 \\
        \end{bmatrix}
        =
        B.
    \end{align*}
    甚至, \(AX = B\) 有无限多个解.
    设 \(t\) 是常数.
    则
    \begin{align*}
        A
        \begin{bmatrix}
            1 - t \\
            t     \\
        \end{bmatrix}
        =
        \begin{bmatrix}
            1 \cdot (1 - t) + 1 \cdot t     \\
            -1 \cdot (1 - t) + (-1) \cdot t \\
        \end{bmatrix}
        =
        B.
    \end{align*}
    取不同的 \(t\),
    就有不一样的
    \(
    \begin{bmatrix}
        1 - t \\
        t     \\
    \end{bmatrix}
    \).
    我们知道, 有无限多个数.
    所以, \(AX = B\) 有无限多个解.

    当 \(b = 0\) 时, 此方程组无解.
    用反证法.
    反设
    \(
    A \begin{bmatrix}
        y_1 \\
        y_2 \\
    \end{bmatrix}
    = B
    \),
    即
    \begin{align*}
        y_1 + y_2 = 1, \\
        -y_1 - y_2 = 0.
    \end{align*}
    从而
    \begin{align*}
        1 + 0
        = {} & (y_1 + y_2) + (-y_1 - y_2)     \\
        = {} & (y_1 - y_1) + (y_2 - y_2) = 0.
    \end{align*}
    这是矛盾.
\end{example}

从本节开始, 我们\emph{定性地}研究%
由 \(n\)~个 \(n\)~元 \({\leq} 1\)~次方程作成的方程组%
的解.
具体地, 我们主要讨论解的性质,
而不是解的公式.

根据 Cramer 公式, 我们不难得到如下事实:

\begin{theorem}
    设 \(A\) 是 \(n\)~级阵.
    设 \(\det {(A)} \neq 0\).
    那么, 对任何的 \(n \times 1\)~阵 \(B\),
    存在一个 \(n \times 1\)~阵 \(C\),
    使 \(AC = B\).
    (换句话说,
    若存在某 \(n \times 1\)~阵 \(B\),
    使对任何的 \(n \times 1\)~阵 \(C\),
    必 \(AC \neq B\),
    则 \(\det {(A)} = 0\).)
\end{theorem}

重要地, 此事反过来也对.

\begin{theorem}
    设 \(A\) 是 \(n\)~级阵.
    设对任何的 \(n \times 1\)~阵 \(B\),
    存在一个 \(n \times 1\)~阵 \(C\),
    使 \(AC = B\).
    则 \(\det {(A)} \neq 0\).
    (换句话说,
    若 \(\det {(A)} = 0\),
    则存在某 \(n \times 1\)~阵 \(B\),
    使对任何的 \(n \times 1\)~阵 \(C\),
    必 \(AC \neq B\).)
\end{theorem}

\begin{proof}
    设 \(n\)~级单位阵~\(I\) 的%
    列~\(1\), \(2\), \(\dots\), \(n\)
    分别是
    \(e_1\), \(e_2\), \(\dots\), \(e_n\).
    那么, 每一个 \(e_i\) 都是 \(n \times 1\)~阵.
    根据假定, 存在一个 \(n \times 1\)~阵 \(f_i\)
    使 \(Af_i = e_i\).
    作 \(n\)~级阵
    \(F = [f_1, f_2, \dots, f_n]\).
    则
    \begin{align*}
        AF
        = {} & A\, [f_1, f_2, \dots, f_n] \\
        = {} & [Af_1, Af_2, \dots, Af_n]  \\
        = {} & [e_1, e_2, \dots, e_n]     \\
        = {} & I.
    \end{align*}
    因为二个同级的方阵的积的行列式%
    等于这二个阵的行列式的积,
    \begin{align*}
        1 = \det {(I)} = \det {(A)} \det {(F)}.
    \end{align*}
    从而 \(\det {(A)} \neq 0\).
\end{proof}

\section{\texorpdfstring{由 \(n\)~个 \(n\)~元
      \({\leq} 1\)~次方程作成的方程组 (3)}%
  {由 n 个 n 元 ≤1 次方程作成的方程组 (3)}}

\maldevigalegajxo

我们进一步地\emph{定性地}研究%
由 \(n\)~个 \(n\)~元 \({\leq} 1\)~次方程作成的方程组%
的解.

前面, 我们知道,
若 \(n\)~级阵 \(A\) 的行列式为零,
则存在某 \(n \times 1\)~阵 \(B\),
使线性方程组 \(AX = B\) 无解.
当然, 对某些 \(B\),
\(AX = B\) 还是有解的:
取 \(B = 0\),
则 \(AX = B\) 至少有一个解 \(X = 0\)
(\(0\) 是元全为零的 \(n \times 1\)~阵).

本节, 我们研究, 若 \(AX = B\) 有解,
则它是否有唯一的解.

我们先看一个较简单的事实.

\begin{theorem}
    设 \(A\) 是 \(n\)~级阵.
    设 \(B\) 是 \(n \times 1\)~阵.
    设 \(n \times 1\)~阵 \(C\) 适合 \(AC = B\).

    (1)
    若 \(AX = 0\) 只有零解, 则 \(AX = B\) 的解是唯一的;

    (2)
    若存在非零的 \(n \times 1\)~阵 \(D\)
    使 \(AD = 0\),
    则 \(AX = B\) 有无限多个解.
\end{theorem}

\begin{proof}
    (1)
    设 \(n \times 1\)~阵 \(Y\) 适合 \(AY = B\).
    则
    \begin{align*}
        A(Y - C) = AY - AC = B - B = 0.
    \end{align*}
    故 \(Y - C\) 是 \(AX = 0\) 的一个解.
    因为 \(AX = 0\) 只有零解,
    故 \(Y - C = 0\).
    从而 \(Y = C\).

    (2)
    要证 \(AX = B\) 有无限多个解,
    只要证:
    任取一个正整数 \(m\),
    我们一定能找到 \(AX = B\) 的
    \(m\) 个\emph{互不相同的}解.

    作 \(m\)~个 \(n \times 1\)~阵
    \(C_1\), \(C_2\), \(\dots\), \(C_m\),
    其中 \(C_j = C + jD\).
    首先, 每一个 \(C_j\) 都是 \(AX = B\) 的解:
    \begin{align*}
        AC_j = A(C + jD) = AC + A(jD) = B + j(AD) = B + j0 = B.
    \end{align*}
    然后, 我们证,
    若 \(p\), \(q\) 是二个不相等的%
    且不超过 \(m\) 的正整数,
    则 \(C_p \neq C_q\).
    用反证法.
    反设 \(C_p = C_q\),
    则
    \begin{align*}
        0 = C_p - C_q = (C + pD) - (C + qD) = (p - q)D.
    \end{align*}
    因为 \(p \neq q\), 故 \(p - q \neq 0\).
    从而
    \begin{align*}
        0 = \frac{1}{p-q}\, 0
        = \frac{1}{p-q}\, ((p-q)D)
        = \Big( \frac{1}{p-q}\, (p-q) \Big) D
        = 1D
        = D.
    \end{align*}
    这是矛盾.
    所以, \(C_p \neq C_q\).
\end{proof}

由此可见, 若 \(AX = 0\) 有非零解,
则 \(AX = B\) 有解时,
其解不但不唯一, 还有无限多个.
若 \(AX = 0\) 只有零解,
则 \(AX = B\) 有解时,
其解是唯一的.

根据 Cramer 公式, 我们不难得到如下事实:

\begin{theorem}
    设 \(A\) 是 \(n\)~级阵.
    设 \(AX = 0\) 有非零解.
    则 \(\det {(A)} = 0\).
    (换句话说,
    若 \(\det {(A)} \neq 0\),
    则 \(AX = 0\) 只有零解.)
\end{theorem}

重要地, 此事反过来也对.

\begin{theorem}
    设 \(A\) 是 \(n\)~级阵.
    设 \(\det {(A)} = 0\).
    则 \(AX = 0\) 有非零解.
    (换句话说,
    若 \(AX = 0\) 只有零解,
    则 \(\det {(A)} \neq 0\).)
\end{theorem}

为说话方便, 我要引入一个小概念.
说阵~\(Z\) 是 \(A\) 的一个 \emph{\(p\)~级子阵},
就是说, \(Z\) 是 \(A\) 的一个子阵
(见本章, 节~\sekcio{5}),
且 \(Z\) 是一个 \(p\)~级阵.

在论证此事前, 我想用三个例%
助您理解此事为什么是正确的.

\begin{example}
    设 \(A\) 是一个 \(5\)~级阵,
    且 \(A = 0\).
    那么, 我们任取一个非零的 \(5 \times 1\)~阵 \(S\),
    必有 \(AS = 0\).
\end{example}

\begin{example}
    设 \(A\) 是一个 \(5\)~级阵,
    且 \(\det {(A)} = 0\),
    但 \(A \neq 0\).

    我们知道,
    \(A \operatorname{adj} {(A)} = \det {(A)}\, I\),
    其中 \(\operatorname{adj} {(A)}\)
    是 \(A\) 的古伴,
    \(I\) 是 \(5\)~级单位阵.
    因为 \(\det {(A)} = 0\),
    故 \(A \operatorname{adj} {(A)} = 0\).

    若 \(\operatorname{adj} {(A)} \neq 0\),
    则存在 \(k\), \(v\), 使
    \([\operatorname{adj} {(A)}]_{k,v} \neq 0\).
    我们设 \(\operatorname{adj} {(A)}\) 的%
    列~\(v\) 是 \(Y\).
    那么, \([Y]_{k,1}
        = [\operatorname{adj} {(A)}]_{k,v} \neq 0\),
    从而 \(Y \neq 0\).
    我们证明 \(AY = 0\):
    \begin{align*}
        [AY]_{i,1}
        = {} &
        \sum_{\ell = 1}^{5}
        {[A]_{i,\ell} [Y]_{\ell,1}}
        \\
        = {} &
        \sum_{\ell = 1}^{5}
        {[A]_{i,\ell} [\operatorname{adj} {(A)}]_{\ell,v}}
        \\
        = {} &
        [A \operatorname{adj} {(A)}]_{i,v}
        \\
        = {} &
        [0]_{i,v}
        \\
        = {} &
        0.
    \end{align*}
\end{example}

\begin{example}
    仍设 \(A\) 是一个 \(5\)~级阵,
    且 \(\det {(A)} = 0\),
    但 \(A \neq 0\).

    在上个例里, 我们假定
    \(\operatorname{adj} {(A)} \neq 0\).
    可是, 它也有可能为 \(0\), 即使 \(A \neq 0\)
    (比如, 可以验证
    \begin{align*}
        M = \begin{bmatrix}
                1  & 1 & -1 & -1 & 1 \\
                1  & 2 & 0  & 0  & 2 \\
                2  & 1 & -3 & -3 & 1 \\
                -1 & 0 & 2  & 2  & 0 \\
                -1 & 1 & 3  & 3  & 1 \\
            \end{bmatrix}
    \end{align*}
    的古伴就是 \(0\)).

    现在, 我们假定
    \(\operatorname{adj} {(A)} = 0\).
    此时, 我们不能利用
    \(\operatorname{adj} {(A)}\)
    写出 \(AX = 0\) 的非零解.
    我们应如何作?

    若 \(\operatorname{adj} {(A)} = 0\),
    则 \(A\) 的每一个 \(4\)~级子阵的行列式都是零.
    从而, 一定存在一个%
    低于 \(4\) 的正整数 \(r\)
    适合性质:
    \emph{\(A\) 有一个 \(r\)~级子阵, 其行列式非零,
        且 \(A\) 的每一个 \(r+1\)~级子阵的行列式都是零.}
    (我们已知 \(A\) 的每一个 \(4\)~级子阵的行列式都是零.
    我们考虑 \(A\) 的 \(3\)~级子阵.
    若有一个 \(3\)~级子阵的行列式不是零,
    我们就取 \(r = 3\).
    若不然, \(A\) 的每一个 \(3\)~级子阵的行列式都是零.
    我们考虑 \(A\) 的 \(2\)~级子阵.
    若有一个 \(2\)~级子阵的行列式不是零,
    我们就取 \(r = 2\).
    若不然, \(A\) 的每一个 \(2\)~级子阵的行列式都是零.
    我们考虑 \(A\) 的 \(1\)~级子阵.
    因为 \(A \neq 0\),
    故 \(A\) 一定有 \(1\)~级子阵的行列式不是零.
    此时, 取 \(r = 1\) 即可.)

    我以 \(r = 2\) 为例%
    演示找 \(AX = 0\) 的非零解的方法
    (可以验证,
    前面的 \(M\) 的行~\(1\), \(2\)
    与列~\(1\), \(2\) 作成的
    \(2\)~级子阵的行列式不是零,
    但 \(M\) 的每一个 \(3\)~级子阵的行列式都是零);
    \(r = 3\) 或 \(r = 1\) 时的情形是类似的.

    我们设
    \(\displaystyle
    A_2 =
    A\binom{1,2}{1,2}
    \)
    的行列式不为零,
    且 \(A\) 的每一个 \(3\)~级子阵的行列式都是零.
    再设
    \(\displaystyle
    E =
    A\binom{1,2,3}{1,2,3}
    \).
    因为 \(E\) 是 \(A\) 的一个 \(3\)~级子阵,
    故 \(\det {(E)} = 0\).
    从而
    \(E \operatorname{adj} {(E)} = 0\).
    注意到
    \([\operatorname{adj} {(E)}]_{3,3}
    = (-1)^{3+3} \det {(A_2)} \neq 0\).
    设 \(\operatorname{adj} {(E)}\)
    的列~\(3\) 为 \(T\).
    那么 \(T \neq 0\).
    并且, 我们可用完全类似的方法, 证明
    \(ET = 0\).

    接下来, 我们设法由此得到一个 \(AX = 0\)
    的非零解.
    我们作一个 \(5 \times 1\)~阵 \(W\), 其中
    \begin{align*}
        [W]_{\ell,1}
        = \begin{cases}
              [\operatorname{adj} {(E)}]_{\ell,3},
                 & \ell \leq 3; \\
              0, & \ell > 3.
          \end{cases}
    \end{align*}
    (通俗地, 我们在 \(T\) 的最后一个元后加 \(2\)~个 \(0\),
    变 \(3 \times 1\)~阵 \(T\)
    为一个 \(5 \times 1\)~阵 \(W\).)
    因为 \([W]_{3,1} \neq 0\),
    故 \(W \neq 0\).
    我们证明 \(AW = 0\).

    取不超过 \(5\) 的正整数 \(q\).
    则
    \begin{align*}
        [AW]_{q,1}
        = {} &
        \sum_{\ell = 1}^{5}
        {[A]_{q,\ell} [W]_{\ell,1}}
        \\
        = {} &
        \sum_{\ell = 1}^{3}
        {[A]_{q,\ell} [W]_{\ell,1}}
        +
        \sum_{\ell = 4}^{5}
        {[A]_{q,\ell} [W]_{\ell,1}}
        \\
        = {} &
        \sum_{\ell = 1}^{3}
        {[A]_{q,\ell} [\operatorname{adj} {(E)}]_{\ell,3}}
        +
        \sum_{\ell = 4}^{5}
        {[A]_{q,\ell}\, 0}
        \\
        = {} &
        \sum_{\ell = 1}^{3}
        {[A]_{q,\ell} [\operatorname{adj} {(E)}]_{\ell,3}}.
    \end{align*}
    若 \(q \leq 3\), 则
    \begin{align*}
        [AW]_{q,1}
        = {} &
        \sum_{\ell=1}^{3}
        {[A]_{q,\ell}
            [\operatorname{adj} {(E)}]_{\ell,3}}
        \\
        = {} &
        \sum_{p=1}^{3}
        {[E]_{q,\ell}
            [\operatorname{adj} {(E)}]_{\ell,3}}
        \\
        = {} &
        [E \operatorname{adj} {(E)}]_{q,3}
        \\
        = {} &
        [0]_{q,3}
        \\
        = {} & 0.
    \end{align*}
    设 \(q > 3\).
    则
    \begin{align*}
        [AW]_{q,1}
        = {} &
        \sum_{\ell=1}^{3}
        {[A]_{q,\ell}
            [\operatorname{adj} {(E)}]_{\ell,3}}
        \\
        = {} &
        \sum_{\ell=1}^{3}
        {[A]_{q,\ell}
        (-1)^{3+\ell} \det {(E(3|\ell))}}.
    \end{align*}
    作一个 \(3\)~级阵 \(C_q\), 其中
    \begin{align*}
        [C_q]_{h,\ell}
        = \begin{cases}
              [E]_{h,\ell},
               & h \neq 3; \\
              [A]_{q,\ell},
               & h = 3.
          \end{cases}
    \end{align*}
    于是, \(C_q(3|\ell) = E(3|\ell)\).
    从而
    \begin{align*}
        [AW]_{q,1}
        = {} &
        \sum_{\ell=1}^{3}
        {[A]_{q,\ell}
        (-1)^{3+\ell} \det {(E(3|\ell))}}
        \\
        = {} &
        \sum_{\ell=1}^{3}
        {[C_q]_{3,\ell}
        (-1)^{3+\ell} \det {(C_q (3|\ell))}}
        \\
        = {} &
        \det {(C_q)}.
    \end{align*}
    注意到
    \(
    \displaystyle
    C_q = A\binom{1,2,q}{1,2,3}
    \)
    是 \(A\)~的一个 \(3\)~级子阵,
    故
    \begin{align*}
        [AW]_{q,1} = \det {(C_q)} = 0.
    \end{align*}

    综上, \(AW = 0\), 且 \(W \neq 0\).
\end{example}

下面, 我给出此事的论证.
证明分三种情形:
(a)
\(A = 0\);
(b)
\(A \neq 0\), 且
\(\operatorname{adj} {(A)} \neq 0\);
(c)
\(A \neq 0\), 且
\(\operatorname{adj} {(A)} = 0\).
情形 (a) 最简单,
我们随便找一个非零的 \(n \times 1\)~阵即可.
情形 (b) 也不难,
我们取 \(A\) 的古伴的一个非零的列即可.
情形 (c) 是最复杂的.
在上个例里, 我们假定%
行列式非零的子阵%
是由 \(A\) 的前 \(r\)~行与前 \(r\)~列作成的,
且每一个 \(r+1\)~级子阵的行列式都是零
(此处 \(r = 2\)).
(通俗地, 我们假定行列式非零的子阵在左上角.)
不过, 对一般的阵来说,
行列式非零的子阵不一定在左上角,
这是论证的最大的挑战.

\begin{proof}
    设 \(A\) 是 \(n\)~级阵,
    且 \(\det {(A)} = 0\).

    若 \(A = 0\),
    我们任取一个非零的 \(n \times 1\)~阵 \(S\).
    则 \(AS = 0\).

    以下, 我们假定 \(A \neq 0\);
    也就是说, \(A\) 有一个元不是零
    (同时, 这也相当于 \(n > 1\):
    回想 \(1\)~级阵的行列式是什么).

    我们知道,
    \(A \operatorname{adj} {(A)} = \det {(A)}\, I\),
    其中 \(\operatorname{adj} {(A)}\)
    是 \(A\) 的古伴,
    \(I\) 是 \(n\)~级单位阵.
    因为 \(\det {(A)} = 0\),
    故 \(A \operatorname{adj} {(A)} = 0\).

    若 \(\operatorname{adj} {(A)} \neq 0\),
    则存在 \(k\), \(v\), 使
    \([\operatorname{adj} {(A)}]_{k,v} \neq 0\).
    我们设 \(\operatorname{adj} {(A)}\) 的%
    列~\(v\) 是 \(Y\).
    那么, \([Y]_{k,1}
        = [\operatorname{adj} {(A)}]_{k,v} \neq 0\),
    从而 \(Y \neq 0\).
    我们证明 \(AY = 0\):
    \begin{align*}
        [AY]_{i,1}
        = {} &
        \sum_{\ell = 1}^{n}
        {[A]_{i,\ell} [Y]_{\ell,1}}
        \\
        = {} &
        \sum_{\ell = 1}^{n}
        {[A]_{i,\ell} [\operatorname{adj} {(A)}]_{\ell,v}}
        \\
        = {} &
        [A \operatorname{adj} {(A)}]_{i,v}
        \\
        = {} &
        [0]_{i,v}
        \\
        = {} &
        0.
    \end{align*}

    若 \(\operatorname{adj} {(A)} = 0\),
    则 \(A\) 的每一个 \(n-1\)~级子阵的行列式都是零
    (同时, 这也相当于 \(n > 2\):
    回想 \(2\)~级阵的古伴是什么;
    再回想前面作过的假定 \(A \neq 0\)).
    从而, 一定存在一个%
    低于 \(n-1\) 的正整数 \(r\)
    适合性质:
    \emph{\(A\) 有一个 \(r\)~级子阵, 其行列式非零,
        且 \(A\) 的每一个 \(r+1\)~级子阵的行列式都是零.}
    (我们已知 \(A\) 的每一个 \(n-1\)~级子阵的行列式都是零.
    我们考虑 \(A\) 的 \(n-2\)~级子阵.
    若有一个 \(n-2\)~级子阵的行列式不是零,
    我们就取 \(r = n-2\).
    若不然, \(A\) 的每一个 \(n-2\)~级子阵的行列式都是零.
    我们考虑 \(A\) 的 \(n-3\)~级子阵.
    若有一个 \(n-3\)~级子阵的行列式不是零,
    我们就取 \(r = n-3\).
    \(\dots \dots\)
    若不然, \(A\) 的每一个 \(2\)~级子阵的行列式都是零.
    我们考虑 \(A\) 的 \(1\)~级子阵.
    因为 \(A \neq 0\),
    故 \(A\) 一定有 \(1\)~级子阵的行列式不是零.
    此时, 取 \(r = 1\) 即可.)

    我们设 \(A\)~的 \(r\)~级子阵
    \(
    \displaystyle
    A_r = A\binom{i_1,\dots,i_r}{j_1,\dots,j_r}
    \)
    的行列式非零,
    其中 \(i_1 < i_2 < \dots < i_r\),
    且 \(j_1 < j_2 < \dots < j_r\).
    我们从 \(1\), \(2\), \(\dots\), \(n\) 中
    去除 \(r\)~个整数
    \(i_1\), \(i_2\), \(\dots\), \(i_r\);
    此时, 还剩下 \(n-r\)~个整数,
    我们从中选一个为 \(i_{r+1}\).
    类似地,
    我们再从 \(1\), \(2\), \(\dots\), \(n\) 中
    去除 \(r\)~个整数
    \(j_1\), \(j_2\), \(\dots\), \(j_r\);
    此时, 还剩下 \(n-r\)~个整数,
    我们从中选一个为 \(j_{r+1}\).
    作 \(r+1\)~级阵
    \(
    \displaystyle
    E =
    A\binom{i_1,\dots,i_r,i_{r+1}}
    {j_1,\dots,j_r,j_{r+1}}
    \).
    我们设 \(i_p\) 在
    \(i_1\), \(\dots\), \(i_r\), \(i_{r+1}\) 中%
    是第~\(f(i_p)\) 小的数.
    再设 \(j_p\) 在
    \(j_1\), \(\dots\), \(j_r\), \(j_{r+1}\) 中%
    是第~\(g(j_p)\) 小的数.
    因为 \(E\) 是 \(A\) 的一个 \(r+1\)~级子阵,
    故 \(\det {(E)} = 0\).
    从而
    \(E \operatorname{adj} {(E)} = 0\).
    注意到
    \([\operatorname{adj} {(E)}]_{g(j_{r+1}), f(i_{r+1})}
    = (-1)^{f(i_{r+1})+g(j_{r+1})} \det {(A_r)} \neq 0\).
    设 \(\operatorname{adj} {(E)}\)
    的列~\(f(i_{r+1})\) 为 \(T\).
    那么 \(T \neq 0\).
    并且, 我们可用完全类似的方法, 证明
    \(ET = 0\).

    接下来, 我们设法由此得到一个 \(AX = 0\)
    的非零解.
    我们作一个 \(n \times 1\)~阵 \(W\), 其中
    \begin{align*}
        [W]_{\ell,1}
        = \begin{cases}
              [\operatorname{adj} {(E)}]_{g(j_p),f(i_{r+1})},
                 & \text{\(\ell = j_p\),
              \(p = 1\), \(\dots\), \(r\), \(r+1\)}; \\
              0, & \text{其他}.
          \end{cases}
    \end{align*}
    (通俗地, 我们在适当的位置写零,
    变 \((r+1) \times 1\)~阵 \(T\)
    为一个 \(n \times 1\)~阵 \(W\).)
    因为 \([W]_{j_{r+1},1} \neq 0\),
    故 \(W \neq 0\).
    我们证明 \(AW = 0\).

    取不超过 \(n\) 的正整数 \(q\).
    则
    \begin{align*}
        [AW]_{q,1}
        = {} &
        \sum_{\ell = 1}^{n}
        {[A]_{q,\ell} [W]_{\ell,1}}
        \\
        = {} &
        \sum_{\substack{
        1 \leq \ell \leq n \\
        \ell = j_p         \\
                1 \leq p \leq r+1
            }}
        {[A]_{q,\ell} [W]_{\ell,1}}
        +
        \sum_{\substack{
        1 \leq \ell \leq n \\
                \text{其他}
            }}
        {[A]_{q,\ell} [W]_{\ell,1}}
        \\
        = {} &
        \sum_{\substack{
        1 \leq \ell \leq n \\
        \ell = j_p         \\
                1 \leq p \leq r+1
            }}
        {[A]_{q,j_p}
                [\operatorname{adj} {(E)}]_{g(j_p),f(i_{r+1})}}
        +
        \sum_{\substack{
        1 \leq \ell \leq n \\
                \text{其他}
            }}
        {[A]_{q,\ell}\, 0}
        \\
        = {} &
        \sum_{p=1}^{r+1}
        {[A]_{q,j_p}
            [\operatorname{adj} {(E)}]_{g(j_p),f(i_{r+1})}}.
    \end{align*}
    若 \(q\) 等于某个 \(i_t\), 则
    \begin{align*}
        [AW]_{i_t,1}
        = {} &
        \sum_{p=1}^{r+1}
        {[A]_{i_t,j_p}
            [\operatorname{adj} {(E)}]_{g(j_p),f(i_{r+1})}}
        \\
        = {} &
        \sum_{p=1}^{r+1}
        {[E]_{f(i_t),g(j_p)}
            [\operatorname{adj} {(E)}]_{g(j_p),f(i_{r+1})}}
        \\
        = {} &
        [E \operatorname{adj} {(E)}]_{f(i_t),f(i_{r+1})}
        \\
        = {} &
        [0]_{f(i_t),f(i_{r+1})}
        \\
        = {} & 0.
    \end{align*}
    若 \(q\) 不等于
    \(i_1\), \(\dots\), \(i_r\), \(i_{r+1}\)
    中的任何一个,
    则
    \begin{align*}
        [AW]_{q,1}
        = {} &
        \sum_{p=1}^{r+1}
        {[A]_{q,j_p}
            [\operatorname{adj} {(E)}]_{g(j_p),f(i_{r+1})}}
        \\
        = {} &
        \sum_{p=1}^{r+1}
        {[A]_{q,j_p}
        (-1)^{f(i_{r+1})+g(j_p)}
        \det {(E(f(i_{r+1})|g(j_p)))}}.
    \end{align*}
    作一个 \(r+1\)~级阵 \(C_q\), 其中
    \begin{align*}
        [C_q]_{h,g(j_p)}
        = \begin{cases}
              [E]_{h,g(j_p)},
               & h \neq f(i_{r+1}); \\
              [A]_{q,j_p},
               & h = f(i_{r+1}).
          \end{cases}
    \end{align*}
    于是, \(C_q(f(i_{r+1})|g(j_p))
    = E(f(i_{r+1})|g(j_p))\).
    从而
    \begin{align*}
        [AW]_{q,1}
        = {} &
        \sum_{p=1}^{r+1}
        {[A]_{q,j_p}
        (-1)^{f(i_{r+1})+g(j_p)}
        \det {(E(f(i_{r+1})|g(j_p)))}}
        \\
        = {} &
        \sum_{p=1}^{r+1}
        {[C_q]_{f(i_{r+1}),g(j_p)}
        (-1)^{f(i_{r+1})+g(j_p)}
        \det {(C_q (f(i_{r+1})|g(j_p)))}}
        \\
        = {} &
        \det {(C_q)}.
    \end{align*}

    再作一个 \(r+1\)~级阵
    \(
    \displaystyle
    D_q =
    A\binom{i_1,\dots,i_r,q}
    {j_1,\dots,j_r,j_{r+1}}
    \).
    不难看出, 适当地交换 \(C_q\) 的行的次序,
    即可变 \(C_q\) 为 \(D_q\).
    根据反称性,
    \(\det {(C_q)} = \pm \det {(D_q)}\).
    (具体地, 设 \(q\) 在
    \(i_1\), \(\dots\), \(i_r\), \(q\) 中%
    是第~\(u\) 小的数.
    那么, 当 \(f(i_{r+1}) = u\) 时, \(C_q = D_q\).
    当 \(f(i_{r+1}) < u\) 时,
    我们交换行~\(f(i_{r+1})\) 与 \(f(i_{r+1})+1\),
    再交换行~\(f(i_{r+1})+1\) 与 \(f(i_{r+1})+2\),
    \(\dots \dots\),
    再交换行~\(u-1\) 与 \(u\),
    作 \(u - f(i_{r+1})\)~次相邻行的交换,
    即可变 \(C_q\) 为 \(D_q\).
    当 \(f(i_{r+1}) > u\) 时,
    我们交换行~\(f(i_{r+1})\) 与 \(f(i_{r+1})-1\),
    再交换行~\(f(i_{r+1})-1\) 与 \(f(i_{r+1})-2\),
    \(\dots \dots\),
    再交换行~\(u+1\) 与 \(u\),
    作 \(f(i_{r+1}) - u\)~次相邻行的交换,
    即可变 \(C_q\) 为 \(D_q\).)
    因为 \(D_q\) 是 \(A\) 的一个 \(r+1\)~级子阵,
    故其行列式为零.
    从而
    \begin{align*}
        [AW]_{q,1}
        = \det {(C_q)}
        = \pm \det {(D_q)}
        = 0.
    \end{align*}

    综上, 我们找到了一个 \(n \times 1\)~阵 \(W\)
    使 \(AW = 0\), 且 \(W \neq 0\).
\end{proof}

\section{\texorpdfstring{由 \(n\)~个 \(n\)~元
      \({\leq} 1\)~次方程作成的方程组 (4)}%
  {由 n 个 n 元 ≤1 次方程作成的方程组 (4)}}

\maldevigalegajxo

本节没有新的知识.

综合前几节的讨论, 我们有如下定理.

\begin{theorem}
    设 \(A\) 是 \(n\)~级阵,
    \(B\) 是 \(n \times 1\)~阵,
    \(X\) 是未知的 \(n \times 1\)~阵.

    (1)
    当 \(\det {(A)} \neq 0\) 时,
    \(AX = B\) 有唯一的解;

    (2)
    当 \(\det {(A)} = 0\) 时,
    \(AX = B\) 要么无解,
    要么有无限多个解.
\end{theorem}

这就是%
由 \(n\)~个 \(n\)~元 \({\leq} 1\)~次方程作成的方程组%
的解的定性的理论 (青春版).
不过,
% 不过:
% (a)
当 \(\det {(A)} = 0\) 时,
如何进一步地\emph{判断} \(AX = B\) 是否有解?
这是此理论无法解答的问题.
% (b)
% 此理论能否被推广到一般的%
% 由 \(m\)~个 \(n\)~元 \({\leq} 1\)~次方程作成的方程组?
% (c)
% 若 \(AX = B\) 有解,
% 但 \(\det {(A)} = 0\),
% 能否给出解的公式?
% 这是定量的理论.

若 \(\det {(A)} \neq 0\),
则, 定量地,
我们可用 Cramer 公式写出
\(AX = B\) 的 (唯一的) 解.
不过, 若 \(AX = B\),
但 \(\det {(A)} = 0\),
我们如何写出它的 (所有的) 解?

我会在接下来的几节里,
讨论一般的%
由 \(m\)~个 \(n\)~元 \({\leq} 1\)~次方程作成的方程组,
并解决这些问题.

\section{\texorpdfstring{由 \(m\)~个 \(n\)~元
      \({\leq} 1\)~次方程作成的方程组 (1)}%
  {由 m 个 n 元 ≤1 次方程作成的方程组 (1)}}

\maldevigalegajxo

在接下来的若干节里, 我想用行列式讨论%
由 \(m\)~个 \(n\)~元 \({\leq} 1\)~次方程作成的方程组%
的解.
我希望您注意到, 方程数 \(m\) 不一定等于未知数之数 \(n\).

讨论%
由 \(n\)~个 \(n\)~元 \({\leq} 1\)~次方程作成的方程组%
的解时,
我们先对一种特别情形作了定量的讨论
(Cramer 公式),
再对较一般的情形作了定性的讨论;
换句话说, 我们当时是先定量, 再定性.
不过, 这次,
讨论%
由 \(m\)~个 \(n\)~元 \({\leq} 1\)~次方程作成的方程组%
时, 我们先定性, 再定量.
这么作, 会较方便一些.

或许, 您记得,
若一个 \(n\)~级阵 \(A\) 的行列式为零,
则 \(AX = 0\) 有非零解.
或许, 您还记得,
\(AX = 0\) 是否有非零解%
跟 \(AX = B\) (有解时) 是否有唯一的解%
有关.
对一般的%
由 \(m\)~个 \(n\)~元 \({\leq} 1\)~次方程作成的方程组%
来说, 我们也有类似的结论.

本节, 我们讨论, 线性方程组有解时, 解是否唯一.

我们先看一个较简单的事实.

\begin{theorem}
    设 \(A\) 是 \(m \times n\)~阵.
    设 \(B\) 是 \(m \times 1\)~阵.
    设 \(n \times 1\)~阵 \(C\) 适合 \(AC = B\).

    (1)
    若 \(AX = 0\) 只有零解, 则 \(AX = B\) 的解是唯一的;

    (2)
    若存在非零的 \(n \times 1\)~阵 \(D\)
    使 \(AD = 0\),
    则 \(AX = B\) 有无限多个解.
\end{theorem}

\begin{proof}
    (1)
    设 \(n \times 1\)~阵 \(Y\) 适合 \(AY = B\).
    则
    \begin{align*}
        A(Y - C) = AY - AC = B - B = 0.
    \end{align*}
    故 \(Y - C\) 是 \(AX = 0\) 的一个解.
    因为 \(AX = 0\) 只有零解,
    故 \(Y - C = 0\).
    从而 \(Y = C\).

    (2)
    要证 \(AX = B\) 有无限多个解,
    只要证:
    任取一个正整数 \(\ell\),
    我们一定能找到 \(AX = B\) 的
    \(\ell\) 个\emph{互不相同的}解.

    作 \(\ell\)~个 \(n \times 1\)~阵
    \(C_1\), \(C_2\), \(\dots\), \(C_\ell\),
    其中 \(C_j = C + jD\).
    首先, 每一个 \(C_j\) 都是 \(AX = B\) 的解:
    \begin{align*}
        AC_j = A(C + jD) = AC + A(jD) = B + j(AD) = B + j0 = B.
    \end{align*}
    然后, 我们证,
    若 \(p\), \(q\) 是二个不相等的%
    且不超过 \(\ell\) 的正整数,
    则 \(C_p \neq C_q\).
    用反证法.
    反设 \(C_p = C_q\),
    则
    \begin{align*}
        0 = C_p - C_q = (C + pD) - (C + qD) = (p - q)D.
    \end{align*}
    因为 \(p \neq q\), 故 \(p - q \neq 0\).
    从而
    \begin{align*}
        0 = \frac{1}{p-q}\, 0
        = \frac{1}{p-q}\, ((p-q)D)
        = \Big( \frac{1}{p-q}\, (p-q) \Big) D
        = 1D
        = D.
    \end{align*}
    这是矛盾.
    所以, \(C_p \neq C_q\).
\end{proof}

由此可见, 若 \(AX = 0\) 有非零解,
则 \(AX = B\) 有解时,
其解不但不唯一, 还有无限多个.
若 \(AX = 0\) 只有零解,
则 \(AX = B\) 有解时,
其解是唯一的.

接着, 我们讨论, \(AX = 0\) 是否有非零解.
不过, 为了使讨论简单一些, 我们要作一些准备.

\begin{theorem}
    设 \(A\) 是 \(m \times n\)~阵.
    若 \(A\) 没有行列式非零的 \(r+1\)~级子阵,
    则对任何高于 \(r\)~的整数 \(s\),
    \(A\) 没有行列式非零的 \(s\)~级子阵.
\end{theorem}

\begin{proof}
    我们用数学归纳法证明此事.
    具体地, 设命题 \(P(s)\) 为:
    \begin{quotation}
        \(A\) 没有行列式非零的 \(s\)~级子阵.
    \end{quotation}
    则我们的目标是:
    对任何高于 \(r\)~的整数 \(s\),
    \(P(s)\) 是正确的.

    \(P(r+1)\) 是正确的.

    假定 \(P(t)\) 是正确的.
    我们由此证, \(P(t+1)\) 也是正确的.
    若 \(A\) 不存在 \(t+1\)~级子阵,
    此事自然是正确的.
    若 \(A\) 存在 \(t+1\)~级子阵,
    % 利用按一列 (或一行) 展开行列式,
    利用定义, 按列~\(1\) 展开其行列式,
    可知,
    此子阵的行列式%
    是 \(A\)~的 \(t+1\)~个 \(t\)~级子阵的行列式的倍的和.
    由此可知,
    \(A\) 没有行列式非零的 \(t+1\)~级子阵.

    所以, \(P(t+1)\) 是正确的.
    由数学归纳法原理, 待证命题成立.
\end{proof}

\begin{theorem}
    设 \(A\) 是 \(m \times n\)~阵.
    存在一个唯一的非负整数 \(r\),
    使:

    (1)
    \(A\) 有一个行列式非零的 \(r\)~级子阵;

    (2)
    \(A\) 没有行列式非零的 \(r+1\)~级子阵.
\end{theorem}

\begin{proof}
    唯一性是较简单的.
    设还有非负整数 \(s\) 适合条件:

    (1\ensuremath{'})
    \(A\) 有一个行列式非零的 \(s\)~级子阵;

    (2\ensuremath{'})
    \(A\) 没有行列式非零的 \(s+1\)~级子阵.

    反设 \(s > r\).
    既然 \(A\) 没有行列式非零的 \(r+1\)~级子阵,
    故 \(A\) 没有行列式非零的 \(s\)~级子阵.
    这跟 (1\ensuremath{'}) 矛盾.
    所以, \(s \leq r\).
    类似地, 我们可证 \(r \leq s\).
    从而 \(s = r\).

    下面, 我们说明存在性.
    我们设 \(t\) 是 \(m\) 与 \(n\) 中的较小者.
    % 毕竟, 若此 \(r\) 存在, 则它既不超过 \(m\), 也不超过 \(n\).
    若 \(A\) 有行列式非零的 \(t\)~级子阵,
    我们可取 \(r = t\);
    否则, 我们代 \(t\) 以 \(t - 1\).
    若 \(A\) 有行列式非零的 \(t - 1\)~级子阵,
    我们可取 \(r = t - 1\);
    否则, 我们代 \(t - 1\) 以 \(t - 2\).
    \(\dots \dots\)
    反复地作下去,
    我们一定能找到此 \(r\).
    (注意到, 我们约定 ``\(0\)~级阵'' 的行列式为 \(1\),
    所以这个过程一定能结束.)
\end{proof}

\begin{theorem}[零的作用]
    设 \(A\) 是 \(m \times n\)~阵.
    设 \(B\) 是 \(m \times 1\)~阵.
    设 \(k\) 是一个不超过 \(m\)~的正整数.
    作一个 \((m + 1) \times n\)~阵 \(\underbar{A}\),
    其中
    \begin{align*}
        [\underbar{A}]_{i,j}
        = \begin{cases}
              [A]_{i,j},   & i \leq k;  \\
              0,           & i = k + 1; \\
              [A]_{i-1,j}, & i > k + 1.
          \end{cases}
    \end{align*}
    (通俗地, \(\underbar{A}\)
    是在 \(A\)~的行~\(k\) 下写零行作成的阵.)
    再作一个 \((m + 1) \times 1\)~阵 \(\underbar{B}\),
    其中
    \begin{align*}
        [\underbar{B}]_{i,1}
        = \begin{cases}
              [B]_{i,1},   & i \leq k;  \\
              0,           & i = k + 1; \\
              [B]_{i-1,1}, & i > k + 1.
          \end{cases}
    \end{align*}
    (通俗地, \(\underbar{B}\)
    是在 \(B\)~的行~\(k\) 下写零作成的阵.)

    (1)
    若 \(n \times 1\)~阵 \(C\) 适合
    \(AC = B\),
    则 \(\underbar{A}C = \underbar{B}\).

    (2)
    若 \(n \times 1\)~阵 \(C\) 适合
    \(\underbar{A}C = \underbar{B}\),
    则 \(AC = B\).

    (3)
    若 \(A\) 有一个行列式非零的 \(r\)~级子阵,
    但没有行列式非零的 \(r+1\)~级子阵,
    则 \(\underbar{A}\) 有一个行列式非零的 \(r\)~级子阵,
    但没有行列式非零的 \(r+1\)~级子阵.

    (4)
    若 \(\underbar{A}\) 有一个行列式非零的 \(r\)~级子阵,
    但没有行列式非零的 \(r+1\)~级子阵,
    则 \(A\) 有一个行列式非零的 \(r\)~级子阵,
    但没有行列式非零的 \(r+1\)~级子阵.
\end{theorem}

\begin{proof}
    (1)
    设 \(AC = B\).
    我们要证 \(\underbar{A}C = \underbar{B}\).

    首先, \(\underbar{A}C\) 与 \(\underbar{B}\)
    的尺寸都是 \((m+1) \times 1\).

    若 \(i \leq k\), 则
    \begin{align*}
        [\underbar{A}C]_{i,1}
        = {} &
        \sum_{\ell = 1}^{n}
        {[\underbar{A}]_{i,\ell} [C]_{\ell,1}}
        \\
        = {} &
        \sum_{\ell = 1}^{n}
        {[A]_{i,\ell} [C]_{\ell,1}}
        \\
        = {} & [AC]_{i,1}
        \\
        = {} & [B]_{i,1}
        % \\
        % = {} &
            =
            [\underbar{B}]_{i,1}.
    \end{align*}
    若 \(i = k + 1\), 则
    \begin{align*}
        [\underbar{A}C]_{i,1}
        = {} &
        \sum_{\ell = 1}^{n}
        {[\underbar{A}]_{i,\ell} [C]_{\ell,1}}
        \\
        = {} &
        \sum_{\ell = 1}^{n}
        {0\,[C]_{\ell,1}}
        \\
        = {} & 0
        % \\
        % = {} &
        =
        [\underbar{B}]_{i,1}.
    \end{align*}
    若 \(i > k + 1\), 则
    \begin{align*}
        [\underbar{A}C]_{i,1}
        = {} &
        \sum_{\ell = 1}^{n}
        {[\underbar{A}]_{i,\ell} [C]_{\ell,1}}
        \\
        = {} &
        \sum_{\ell = 1}^{n}
        {[A]_{i-1,\ell} [C]_{\ell,1}}
        \\
        = {} & [AC]_{i-1,1}
        \\
        = {} & [B]_{i-1,1}
        % \\
        % = {} &
            =
            [\underbar{B}]_{i,1}.
    \end{align*}
    由此可见, \(\underbar{A}C = \underbar{B}\).

    (2)
    设 \(\underbar{A}C = \underbar{B}\).
    我们要证 \(AC = B\).

    首先, \(AC\) 与 \(B\)
    的尺寸都是 \(m \times 1\).

    其次, 注意到
    \begin{align*}
        [A]_{i,j}
        = \begin{cases}
              [\underbar{A}]_{i,j},   & i \leq k; \\
              [\underbar{A}]_{i+1,j}, & i > k.
          \end{cases}
    \end{align*}
    类似地,
    \begin{align*}
        [B]_{i,1}
        = \begin{cases}
              [\underbar{B}]_{i,1},   & i \leq k; \\
              [\underbar{B}]_{i+1,1}, & i > k.
          \end{cases}
    \end{align*}

    若 \(i \leq k\), 则
    \begin{align*}
        [AC]_{i,1}
        = {} &
        \sum_{\ell = 1}^{n}
        {[A]_{i,\ell} [C]_{\ell,1}}
        \\
        = {} &
        \sum_{\ell = 1}^{n}
        {[\underbar{A}]_{i,\ell} [C]_{\ell,1}}
        \\
        = {} & [\underbar{A}C]_{i,1}
        \\
        = {} & [\underbar{B}]_{i,1}
        % \\
        % = {} &
            =
            [B]_{i,1}.
    \end{align*}
    若 \(i > k\), 则
    \begin{align*}
        [AC]_{i,1}
        = {} &
        \sum_{\ell = 1}^{n}
        {[A]_{i,\ell} [C]_{\ell,1}}
        \\
        = {} &
        \sum_{\ell = 1}^{n}
        {[\underbar{A}]_{i+1,\ell} [C]_{\ell,1}}
        \\
        = {} & [\underbar{A}C]_{i+1,1}
        \\
        = {} & [\underbar{B}]_{i+1,1}
        % \\
        % = {} &
            =
            [B]_{i,1}.
    \end{align*}
    由此可见, \(AC = B\).

    (3)
    注意到, \(A\)~的子阵都是 \(\underbar{A}\)~的子阵.
    所以, 若 \(A\) 有一个行列式非零的 \(r\)~级子阵,
    则 \(\underbar{A}\) 也有一个行列式非零的 \(r\)~级子阵.
    对 \(\underbar{A}\)~的每一个子阵来说,
    要么 \(\underbar{A}\) 的行~\(k+1\) 被选中,
    要么 \(\underbar{A}\) 的行~\(k+1\) 不被选中.
    所以, 那些 \(\underbar{A}\) 的行~\(k+1\) 不被选中的
    \(r+1\)~级子阵 (若存在)
    一定是 \(A\)~的 \(r+1\)~级子阵 (若存在),
    故其行列式为零.
    而那些 \(\underbar{A}\) 的行~\(k+1\) 被选中的
    \(r+1\)~级子阵 (若存在)
    有一行的元全是零,
    故其行列式为零.
    从而, \(\underbar{A}\) 也没有行列式非零的 \(r+1\)~级子阵.

    (4)
    对 \(\underbar{A}\)~的每一个子阵来说,
    要么 \(\underbar{A}\) 的行~\(k+1\) 被选中,
    要么 \(\underbar{A}\) 的行~\(k+1\) 不被选中.
    设 \(\underbar{A}\) 有一个行列式非零的 \(r\)~级子阵.
    那么, 这个子阵一定不含元全为零的行.
    特别地, 对此子阵而言,
    \(\underbar{A}\) 的行~\(k+1\) 一定不被选中.
    所以, 这也是 \(A\)~的一个 \(r\)~级子阵.
    所以, \(A\) 也有一个行列式非零的 \(r\)~级子阵.
    注意到, \(A\)~的子阵都是 \(\underbar{A}\)~的子阵.
    既然 \(\underbar{A}\) 没有行列式非零的 \(r+1\)~级子阵,
    那么 \(A\) 也没有行列式非零的 \(r+1\)~级子阵.
\end{proof}

这是本节的首个重要的结论.

\begin{theorem}
    设 \(A\) 是 \(m \times n\)~阵.
    设 \(A\) 有一个行列式非零的 \(r\)~级子阵,
    但没有行列式非零的 \(r+1\)~级子阵.
    设 \(r < n\).
    则 \(AX = 0\) 有非零解.
\end{theorem}

\begin{proof}
    设 \(A\) 是 \(m \times n\)~阵.

    我们无妨设 \(m \geq n\).
    若 \(m < n\), 我们作一个 \(n\)~级阵 \(\underbar{A}\),
    其中
    \begin{align*}
        [\underbar{A}]_{i,j}
        = \begin{cases}
              [A]_{i,j}, & i \leq m; \\
              0,         & i > m.
          \end{cases}
    \end{align*}
    (通俗地, \(\underbar{A}\)
    是在 \(A\)~的行~\(m\) 下写 \(n-m\)~个零行作成的阵.)
    由此, 不难得到:
    (a)
    若 \(n \times 1\)~阵 \(C\) 适合 \(\underbar{A}C = 0\),
    则 \(AC = 0\);
    (b)
    \(\underbar{A}\) 有一个行列式非零的 \(r\)~级子阵,
    但没有行列式非零的 \(r+1\)~级子阵.
    所以, 若我们找到了 \(\underbar{A}X = 0\)
    的非零解,
    则我们也找到了 \(AX = 0\) 的非零解.
    以下, 我们设 \(m \geq n\).

    若 \(A = 0\),
    我们任取一个非零的 \(n \times 1\)~阵 \(S\).
    则 \(AS = 0\).

    以下, 我们假定 \(A \neq 0\);
    也就是说, \(A\) 有一个元不是零
    (同时, 这也相当于 \(n > 1\):
    回想 \(1\)~级阵的行列式是什么).
    从而 \(r \geq 1\).

    我们设 \(A\)~的 \(r\)~级子阵
    \(
    \displaystyle
    A_r = A\binom{i_1,\dots,i_r}{j_1,\dots,j_r}
    \)
    的行列式非零,
    其中 \(i_1 < i_2 < \dots < i_r\),
    且 \(j_1 < j_2 < \dots < j_r\).
    我们从 \(1\), \(2\), \(\dots\), \(m\) 中
    去除 \(r\)~个整数
    \(i_1\), \(i_2\), \(\dots\), \(i_r\);
    此时, 还剩下 \(m-r\)~个整数,
    我们从中选一个为 \(i_{r+1}\).
    类似地,
    我们再从 \(1\), \(2\), \(\dots\), \(n\) 中
    去除 \(r\)~个整数
    \(j_1\), \(j_2\), \(\dots\), \(j_r\);
    此时, 还剩下 \(n-r\)~个整数,
    我们从中选一个为 \(j_{r+1}\).
    作 \(r+1\)~级阵
    \(
    \displaystyle
    E =
    A\binom{i_1,\dots,i_r,i_{r+1}}
    {j_1,\dots,j_r,j_{r+1}}
    \).
    我们设 \(i_p\) 在
    \(i_1\), \(\dots\), \(i_r\), \(i_{r+1}\) 中%
    是第~\(f(i_p)\) 小的数.
    再设 \(j_p\) 在
    \(j_1\), \(\dots\), \(j_r\), \(j_{r+1}\) 中%
    是第~\(g(j_p)\) 小的数.
    因为 \(E\) 是 \(A\) 的一个 \(r+1\)~级子阵,
    故 \(\det {(E)} = 0\).
    从而
    \(E \operatorname{adj} {(E)} = 0\).
    注意到
    \([\operatorname{adj} {(E)}]_{g(j_{r+1}), f(i_{r+1})}
    = (-1)^{f(i_{r+1})+g(j_{r+1})} \det {(A_r)} \neq 0\).

    接下来, 我们设法由此得到一个 \(AX = 0\)
    的非零解.
    我们作一个 \(n \times 1\)~阵 \(W\), 其中
    \begin{align*}
        [W]_{\ell,1}
        = \begin{cases}
              [\operatorname{adj} {(E)}]_{g(j_p),f(i_{r+1})},
                 & \text{\(\ell = j_p\),
              \(p = 1\), \(\dots\), \(r\), \(r+1\)}; \\
              0, & \text{其他}.
          \end{cases}
    \end{align*}
    (通俗地, 我们在适当的位置写零,
    变 \(\operatorname{adj} {(E)}\)
    的列~\(f(i_{r+1})\)
    为一个 \(n \times 1\)~阵 \(W\).)
    因为 \([W]_{j_{r+1},1} \neq 0\),
    故 \(W \neq 0\).
    我们证明 \(AW = 0\).

    取不超过 \(m\) 的正整数 \(q\).
    则
    \begin{align*}
        [AW]_{q,1}
        = {} &
        \sum_{\ell = 1}^{n}
        {[A]_{q,\ell} [W]_{\ell,1}}
        \\
        = {} &
        \sum_{\substack{
        1 \leq \ell \leq n \\
        \ell = j_p         \\
                1 \leq p \leq r+1
            }}
        {[A]_{q,\ell} [W]_{\ell,1}}
        +
        \sum_{\substack{
        1 \leq \ell \leq n \\
                \text{其他}
            }}
        {[A]_{q,\ell} [W]_{\ell,1}}
        \\
        = {} &
        \sum_{\substack{
        1 \leq \ell \leq n \\
        \ell = j_p         \\
                1 \leq p \leq r+1
            }}
        {[A]_{q,j_p}
                [\operatorname{adj} {(E)}]_{g(j_p),f(i_{r+1})}}
        +
        \sum_{\substack{
        1 \leq \ell \leq n \\
                \text{其他}
            }}
        {[A]_{q,\ell}\, 0}
        \\
        = {} &
        \sum_{p=1}^{r+1}
        {[A]_{q,j_p}
            [\operatorname{adj} {(E)}]_{g(j_p),f(i_{r+1})}}.
    \end{align*}
    若 \(q\) 等于某个 \(i_t\), 则
    \begin{align*}
        [AW]_{i_t,1}
        = {} &
        \sum_{p=1}^{r+1}
        {[A]_{i_t,j_p}
            [\operatorname{adj} {(E)}]_{g(j_p),f(i_{r+1})}}
        \\
        = {} &
        \sum_{p=1}^{r+1}
        {[E]_{f(i_t),g(j_p)}
            [\operatorname{adj} {(E)}]_{g(j_p),f(i_{r+1})}}
        \\
        = {} &
        [E \operatorname{adj} {(E)}]_{f(i_t),f(i_{r+1})}
        \\
        = {} &
        [0]_{f(i_t),f(i_{r+1})}
        \\
        = {} & 0.
    \end{align*}
    若 \(q\) 不等于
    \(i_1\), \(\dots\), \(i_r\), \(i_{r+1}\)
    中的任何一个,
    则
    \begin{align*}
        [AW]_{q,1}
        = {} &
        \sum_{p=1}^{r+1}
        {[A]_{q,j_p}
            [\operatorname{adj} {(E)}]_{g(j_p),f(i_{r+1})}}
        \\
        = {} &
        \sum_{p=1}^{r+1}
        {[A]_{q,j_p}
        (-1)^{f(i_{r+1})+g(j_p)}
        \det {(E(f(i_{r+1})|g(j_p)))}}.
    \end{align*}
    作一个 \(r+1\)~级阵 \(C_q\), 其中
    \begin{align*}
        [C_q]_{h,g(j_p)}
        = \begin{cases}
              [E]_{h,g(j_p)},
               & h \neq f(i_{r+1}); \\
              [A]_{q,j_p},
               & h = f(i_{r+1}).
          \end{cases}
    \end{align*}
    于是, \(C_q(f(i_{r+1})|g(j_p))
    = E(f(i_{r+1})|g(j_p))\).
    从而
    \begin{align*}
        [AW]_{q,1}
        = {} &
        \sum_{p=1}^{r+1}
        {[A]_{q,j_p}
        (-1)^{f(i_{r+1})+g(j_p)}
        \det {(E(f(i_{r+1})|g(j_p)))}}
        \\
        = {} &
        \sum_{p=1}^{r+1}
        {[C_q]_{f(i_{r+1}),g(j_p)}
        (-1)^{f(i_{r+1})+g(j_p)}
        \det {(C_q (f(i_{r+1})|g(j_p)))}}
        \\
        = {} &
        \det {(C_q)}.
    \end{align*}

    再作一个 \(r+1\)~级阵
    \(
    \displaystyle
    D_q =
    A\binom{i_1,\dots,i_r,q}
    {j_1,\dots,j_r,j_{r+1}}
    \).
    不难看出, 适当地交换 \(C_q\) 的行的次序,
    即可变 \(C_q\) 为 \(D_q\).
    根据反称性,
    \(\det {(C_q)} = \pm \det {(D_q)}\).
    (具体地, 设 \(q\) 在
    \(i_1\), \(\dots\), \(i_r\), \(q\) 中%
    是第~\(u\) 小的数.
    那么, 当 \(f(i_{r+1}) = u\) 时, \(C_q = D_q\).
    当 \(f(i_{r+1}) < u\) 时,
    我们交换行~\(f(i_{r+1})\) 与 \(f(i_{r+1})+1\),
    再交换行~\(f(i_{r+1})+1\) 与 \(f(i_{r+1})+2\),
    \(\dots \dots\),
    再交换行~\(u-1\) 与 \(u\),
    作 \(u - f(i_{r+1})\)~次相邻行的交换,
    即可变 \(C_q\) 为 \(D_q\).
    当 \(f(i_{r+1}) > u\) 时,
    我们交换行~\(f(i_{r+1})\) 与 \(f(i_{r+1})-1\),
    再交换行~\(f(i_{r+1})-1\) 与 \(f(i_{r+1})-2\),
    \(\dots \dots\),
    再交换行~\(u+1\) 与 \(u\),
    作 \(f(i_{r+1}) - u\)~次相邻行的交换,
    即可变 \(C_q\) 为 \(D_q\).)
    因为 \(D_q\) 是 \(A\) 的一个 \(r+1\)~级子阵,
    故其行列式为零.
    从而
    \begin{align*}
        [AW]_{q,1}
        = \det {(C_q)}
        = \pm \det {(D_q)}
        = 0.
    \end{align*}

    综上, 我们找到了一个 \(n \times 1\)~阵 \(W\)
    使 \(AW = 0\), 且 \(W \neq 0\).
\end{proof}

这是本节的另一个重要的结论.

\begin{theorem}
    设 \(A\) 是 \(m \times n\)~阵.
    设 \(A\) 有一个行列式非零的 \(n\)~级子阵
    (此时, \(A\) 当然没有行列式非零的 \(n+1\)~级子阵).
    则 \(AX = 0\) 只有零解.
\end{theorem}

\begin{proof}
    设 \(n \times 1\)~阵 \(C\) 适合 \(AC = 0\).
    设 \(A\)~的 \(n\)~级子阵
    \begin{align*}
        E = A\binom{i_1,i_2,\dots,i_n}{1,2,\dots,n}
    \end{align*}
    的行列式非零,
    其中
    \(1 \leq i_1 < i_2 < \dots < i_n \leq m\).
    因为 \(C\) 适合
    \begin{align*}
        [A]_{1,1} [C]_{1,1} + [A]_{1,2} [C]_{2,1}
        + \dots + [A]_{1,n} [C]_{n,1} = 0, \\
        [A]_{2,1} [C]_{1,1} + [A]_{2,2} [C]_{2,1}
        + \dots + [A]_{2,n} [C]_{n,1} = 0, \\
        \dots \dots \dots
        \dots \dots \dots \dots
        \dots \dots \dots \dots
        \dots \dots \dots \dots
        \dots \dots \dots \dots,
        \\
        [A]_{m,1} [C]_{1,1} + [A]_{m,2} [C]_{2,1}
        + \dots + [A]_{m,n} [C]_{n,1} = 0,
    \end{align*}
    故 \(C\) 当然也适合
    \begin{align*}
        [A]_{i_1,1} [C]_{1,1} + [A]_{i_1,2} [C]_{2,1}
        + \dots + [A]_{i_1,n} [C]_{n,1} = 0, \\
        [A]_{i_2,1} [C]_{1,1} + [A]_{i_2,2} [C]_{2,1}
        + \dots + [A]_{i_2,n} [C]_{n,1} = 0, \\
        \dots \dots \dots
        \dots \dots \dots \dots
        \dots \dots \dots \dots
        \dots \dots \dots \dots
        \dots \dots \dots \dots,
        \\
        [A]_{i_n,1} [C]_{1,1} + [A]_{i_n,2} [C]_{2,1}
        + \dots + [A]_{i_n,n} [C]_{n,1} = 0,
    \end{align*}
    即 \(EC = 0\).
    因为 \(\det {(E)} \neq 0\),
    故, 由 Cramer 公式,
    有 \(C = 0\).
\end{proof}

现在, 我们作一个小结.

\begin{theorem}
    设 \(A\) 是 \(m \times n\)~阵.
    设 \(A\) 有一个行列式非零的 \(r\)~级子阵,
    但没有行列式非零的 \(r+1\)~级子阵.

    (1)
    若 \(r < n\),
    则 \(AX = 0\) 有非零解;

    (2)
    若 \(r = n\),
    则 \(AX = 0\) 只有零解.
\end{theorem}

\begin{theorem}
    设 \(A\) 是 \(m \times n\)~阵.
    设 \(B\) 是 \(m \times 1\)~阵.
    设 \(n \times 1\)~阵 \(C\) 适合 \(AC = B\);
    也就是说, \(AX = B\) 有解.
    设 \(A\) 有一个行列式非零的 \(r\)~级子阵,
    但没有行列式非零的 \(r+1\)~级子阵.

    (1)
    若 \(r < n\),
    则 \(AX = B\) 有无限多个解;

    (2)
    若 \(r = n\),
    则 \(AX = B\) 有唯一的解.
\end{theorem}

\section{\texorpdfstring{由 \(m\)~个 \(n\)~元
      \({\leq} 1\)~次方程作成的方程组 (2)}%
  {由 m 个 n 元 ≤1 次方程作成的方程组 (2)}}

\maldevigalegajxo

前面, 我们讨论了
``\(AX = B\) 有解时, 解是否唯一''
问题.
此事的回答跟 \(A\)~的子阵的行列式有关.
设 \(A\) 有 \(n\)~列.
当 \(A\) 有一个行列式非零的 \(n\)~级子阵时,
此方程组有唯一的解;
当 \(A\) 没有行列式非零的 \(n\)~级子阵时,
其解不但不唯一, 还有无限多个.

接着, 自然地, 我们想讨论线性方程组何时有解.
或许, 一个好的想法是:
我们先讨论 \(AX = B\) 有解时,
\(A\), \(B\) 应适合什么条件;
然后, 反过来,
我们再讨论适合这些条件的 \(A\), \(B\)
是否能使 \(AX = B\) 有解.

我们会用此思想研究此事.
不过, 在此之前, 请允许我介绍一件有用的小事.

\begin{theorem}
    设 \(A\) 是 \(n\)~级阵.
    设 \(p\), \(q\) 是二个不超过 \(n\)~的正整数,
    且 \(p \neq q\).
    设 \(x\) 是一个数.
    作 \(n\)~级阵 \(E\), 其中
    \begin{align*}
        [E]_{i,j}
        = \begin{cases}
              [A]_{i,j},              & j \neq q; \\
              [A]_{i,q} + x[A]_{i,p}, & j = q.
          \end{cases}
    \end{align*}
    (通俗地,
    我们加 \(A\)~的列~\(p\) 的 \(x\)~倍于列~\(q\),
    不改变其他的列,
    得阵~\(E\).)
    则 \(\det {(E)} = \det {(A)}\).

    类似地,
    若我们加 \(A\)~的行~\(p\) 的 \(x\)~倍于行~\(q\),
    不改变其他的行,
    得阵~\(F\),
    则我们也有 \(\det {(F)} = \det {(A)}\).
\end{theorem}

\begin{proof}
    我证明关于列的事;
    我们可用类似的方法证明关于行的事
    (或者, 利用行列式与转置的关系).

    设 \(A = [a_1, a_2, \dots, a_n]\).

    先设 \(p < q\).
    为方便说话, 我们写
    \begin{align*}
        f(u, v)
        = \det {[a_1, \dots, a_{p-1}, u, a_{p+1}, \dots,
                    a_{q-1}, v, a_{q+1}, \dots, a_n]}.
    \end{align*}
    于是, \(f(a_p, a_q)\) 就是 \(\det {(A)}\),
    而 \(f(a_p, a_q + xa_p)\) 就是 \(\det {(E)}\).
    利用多线性与交错性,
    \begin{align*}
        \det {(E)}
        = {} & f(a_p, a_q + xa_p)         \\
        = {} & f(a_p, a_q) + xf(a_p, a_p) \\
        = {} & f(a_p, a_q) + x0           \\
        = {} & f(a_p, a_q)                \\
        = {} & \det {(A)}.
    \end{align*}

    再设 \(p > q\).
    为方便说话, 我们写
    \begin{align*}
        g(u, v)
        = \det {[a_1, \dots, a_{q-1}, u, a_{q+1}, \dots,
                    a_{p-1}, v, a_{p+1}, \dots, a_n]}.
    \end{align*}
    于是, \(g(a_q, a_p)\) 就是 \(\det {(A)}\),
    而 \(g(a_q + xa_p, a_p)\) 就是 \(\det {(E)}\).
    利用多线性与交错性,
    \begin{align*}
        \det {(E)}
        = {} & g(a_q + xa_p, a_p)         \\
        = {} & g(a_q, a_p) + xg(a_p, a_p) \\
        = {} & g(a_q, a_p) + x0           \\
        = {} & g(a_q, a_p)                \\
        = {} & \det {(A)}.
        \qedhere
    \end{align*}
\end{proof}

现在, 我们可以研究,
若 \(AX = B\) 有解,
则 \(A\), \(B\) 应适合什么条件.

\begin{restatable}[]{theorem}{TheoremNecessityForConsistency}
    设 \(A\) 是 \(m \times n\)~阵.
    设 \(B\) 是 \(m \times 1\)~阵.
    设存在 \(n \times 1\)~阵 \(C\)
    适合 \(AC = B\).
    作一个 \(m \times (n+1)\)~阵 \(G\),
    其中
    \begin{align*}
        [G]_{i,j}
        = \begin{cases}
              [A]_{i,j}, & j \leq n;  \\
              [B]_{i,1}, & j = n + 1.
          \end{cases}
    \end{align*}
    (通俗地, 在 \(A\)~的最后一列的右侧加入一列 \(B\),
    得到尺寸较大的阵 \(G\).)
    则一定存在一个非负整数 \(r\), 使
    \(A\) 有一个行列式非零的 \(r\)~级子阵
    (从而 \(G\) 也有一个行列式非零的 \(r\)~级子阵),
    但 \(G\) 没有行列式非零的 \(r+1\)~级子阵
    (从而 \(A\) 也没有行列式非零的 \(r+1\)~级子阵).
\end{restatable}

\begin{proof}
    我们知道, 存在一个非负整数 \(r\),
    使 \(A\) 有一个行列式非零的 \(r\)~级子阵,
    但 \(A\) 没有行列式非零的 \(r+1\)~级子阵.
    我们证明,
    \(G\) 也没有行列式非零的 \(r+1\)~级子阵.

    若 \(G\) 根本没有 \(r+1\)~级子阵,
    则 \(G\) 当然也没有行列式非零的 \(r+1\)~级子阵.

    若 \(G\) 有 \(r+1\)~级子阵,
    那么,
    对 \(G\)~的每一个 \(r+1\)~级子阵来说,
    要么 \(G\) 的列~\(n+1\) 被选中,
    要么 \(G\) 的列~\(n+1\) 不被选中.

    若 \(G\) 的列~\(n+1\) 不被选中,
    那么, 这个子阵就是 \(A\)~的 \(r+1\)~级子阵,
    故其行列式为零.

    若 \(G\) 的列~\(n+1\) 被选中,
    我们说明,
    这个子阵的行列式是
    \(A\)~的一些 \(r+1\)~级子阵的行列式的倍的和,
    故其行列式仍为零.
    因为 \(AC = B\), 故,
    对任何不超过 \(m\) 的正整数 \(i\),
    \begin{align*}
        [B]_{i,1}
            = [A]_{i,1} [C]_{1,1} + [A]_{i,2} [C]_{2,1}
        + \dots + [A]_{i,n} [C]_{n,1},
    \end{align*}
    即
    \begin{align*}
        [G]_{i,n+1}
        = [G]_{i,1} [C]_{1,1} + [G]_{i,2} [C]_{2,1}
        + \dots + [G]_{i,n} [C]_{n,1}.
    \end{align*}
    设这个子阵为
    \begin{align*}
        D = G\binom{i_1,\dots,i_r,i_{r+1}}{j_1,\dots,j_r,n+1},
    \end{align*}
    其中
    \(1 \leq i_1 < \dots < i_r < i_{r+1} \leq m\),
    \(1 \leq j_1 < \dots < j_r \leq n\).
    我们加 \(D\) 的列~\(1\) 的 \(-[C]_{j_1,1}\)~倍%
    于列~\(r+1\),
    得阵~\(D_1\),
    则 \(\det {(D_1)} = \det {(D)}\).
    我们加 \(D_1\) 的列~\(2\) 的 \(-[C]_{j_2,1}\)~倍%
    于列~\(r+1\),
    得阵~\(D_2\),
    则 \(\det {(D_2)} = \det {(D_1)} = \det {(D)}\).
    \(\dots \dots\)
    我们加 \(D_{r-1}\) 的列~\(r\) 的 \(-[C]_{j_r,1}\)~倍%
    于列~\(r+1\),
    得阵~\(D_r\),
    则 \(\det {(D_r)} = \det {(D_{r-1})} = \det {(D)}\).
    注意到, \(D_r\)~的列
    \(1\), \(2\), \(\dots\), \(r\)
    分别跟 \(D\)~的列
    \(1\), \(2\), \(\dots\), \(r\)
    相等;
    不过,
    \begin{align*}
        [D_r]_{s,r+1}
        = {} &
        [G]_{i_s,n+1}
        - \sum_{\ell = 1}^{r}
        {[G]_{i_s,j_\ell} [C]_{j_\ell,1}}
        \\
        = {} &
        \sum_{1 \leq t \leq n}
        {[G]_{i_s,t} [C]_{t,1}}
        -
        \sum_{\substack{1 \leq t \leq n \\
                \text{\(t\) 等于某个 \(j_\ell\)}}}
        {[G]_{i_s,t} [C]_{t,1}}
        \\
        = {} &
        \sum_{\substack{1 \leq t \leq n \\
                \text{\(t\) 不等于任何 \(j_\ell\)}}}
        {[G]_{i_s,t} [C]_{t,1}}.
    \end{align*}
    为方便说话, 我们记 \((r+1) \times n\)~阵
    \begin{align*}
        A\binom{i_1,\dots,i_r,i_{r+1}}{1,2,\dots,n}
    \end{align*}
    的列~\(1\), \(2\), \(\dots\), \(n\) 为
    \(h_1\), \(h_2\), \(\dots\), \(h_n\).
    于是, \(D_r\)~的列~\(1\), \(2\), \(\dots\), \(r\) 就是
    \(h_{j_1}\), \(h_{j_2}\), \(\dots\), \(h_{j_r}\),
    而 \(D_r\)~的列~\(r+1\) 是
    \begin{align*}
        \sum_{\substack{1 \leq t \leq n \\
                \text{\(t\) 不等于任何 \(j_\ell\)}}}
        {[C]_{t,1} h_t}.
    \end{align*}
    从而, 利用多线性,
    \begin{align*}
        \det {(D)}
        = {} &
        \det {(D_r)}
        \\
        = {} &
        \det {
            \left[
                h_{j_1}, \dots, h_{j_r},
        \sum_{\substack{1 \leq t \leq n \\
                        \text{\(t\) 不等于任何 \(j_\ell\)}}}
                {[C]_{t,1} h_t}
                \right]
        }
        \\
        = {} &
        \sum_{\substack{1 \leq t \leq n \\
                \text{\(t\) 不等于任何 \(j_\ell\)}}}
        {
            [C]_{t,1} \det {[h_{j_1}, \dots, h_{j_r}, h_t]}
        }.
    \end{align*}
    适当地交换
    \([h_{j_1}, \dots, h_{j_r}, h_t]\)
    的列的次序,
    利用反称性, 有
    \begin{align*}
        \det {(D)}
        = {} &
        \sum_{\substack{1 \leq t \leq n \\
                \text{\(t\) 不等于任何 \(j_\ell\)}}}
        {
            [C]_{t,1} \det {[h_{j_1}, \dots, h_{j_r}, h_t]}
        }
        \\
        = {} &
        \sum_{\substack{1 \leq t \leq n \\
                \text{\(t\) 不等于任何 \(j_\ell\)}}}
        {
            (\pm)\, [C]_{t,1}
            \det {\left(
                A\binom{i_1,\dots,i_r,i_{r+1}}{j_1,\dots,j_r,t}
                \right)}
        }
        \\
        = {} &
        \sum_{\substack{1 \leq t \leq n \\
                \text{\(t\) 不等于任何 \(j_\ell\)}}}
        {
            (\pm)\, [C]_{t,1} \,0
        }
        \\
        = {} &
        0.
        \qedhere
    \end{align*}
\end{proof}

\section{\texorpdfstring{由 \(m\)~个 \(n\)~元
      \({\leq} 1\)~次方程作成的方程组 (3)}%
  {由 m 个 n 元 ≤1 次方程作成的方程组 (3)}}

\maldevigalegajxo

前面, 我们讨论了
\(AX = B\) 有解时,
\(A\), \(B\) 应适合的条件:

\TheoremNecessityForConsistency*

那么, 反过来,
若 \(A\) 有一个行列式非零的 \(r\)~级子阵,
但 \(G\) 没有行列式非零的 \(r+1\)~级子阵,
则 \(AX = B\) 是否有解?
% 幸运地,
此事的回答是 ``是''.
不过, 为了论证此事,
我们要作一些准备.

\begin{theorem}
    设 \(A\) 是 \(m \times n\)~阵,
    且 \(A \neq 0\).
    设 \(A\) 有一个行列式非零的 \(r\)~级子阵
    \begin{align*}
        A_r = A\binom{i_1,\dots,i_r}{j_1,\dots,j_r}
    \end{align*}
    (其中
    \(1 \leq i_1 < \dots < i_r \leq m\),
    \(1 \leq j_1 < \dots < j_r \leq n\)),
    但 \(A\) 没有行列式非零的 \(r+1\)~级子阵.
    那么, 对任何不超过 \(m\) 的正整数 \(p\),
    一定存在 \(r\)~个数
    \(d_{p,1}\), \(d_{p,2}\), \(\dots\), \(d_{p,r}\),
    使对任何不超过 \(n\) 的正整数 \(q\),
    \begin{align*}
        [A]_{p,q}
        = {} &
        [A]_{i_1,q} d_{p,1}
        + [A]_{i_2,q} d_{p,2}
        + \dots
        + [A]_{i_r,q} d_{p,r}
        \\
        = {} &
        \sum_{s = 1}^{r} {[A]_{i_s,q} d_{p,s}}.
    \end{align*}
\end{theorem}

我们也可如此说前面的结论:

\begin{theorem}
    设 \(A\) 是 \(m \times n\)~阵,
    且 \(A \neq 0\).
    设 \(A\) 有一个行列式非零的 \(r\)~级子阵
    \begin{align*}
        A_r = A\binom{i_1,\dots,i_r}{j_1,\dots,j_r}
    \end{align*}
    (其中
    \(1 \leq i_1 < \dots < i_r \leq m\),
    \(1 \leq j_1 < \dots < j_r \leq n\)),
    但 \(A\) 没有行列式非零的 \(r+1\)~级子阵.
    设 \(A\)~的%
    行~\(1\), \(2\), \(\dots\), \(m\)
    为 \(a_1\), \(a_2\), \(\dots\), \(a_m\).
    那么, 对任何不超过 \(m\) 的正整数 \(p\),
    一定存在 \(r\)~个数
    \(d_{p,1}\), \(d_{p,2}\), \(\dots\), \(d_{p,r}\),
    使
    \begin{align*}
        a_p
        = {} &
        d_{p,1} a_{i_1}
        + d_{p,2} a_{i_2}
        + \dots
        + d_{p,r} a_{i_r}
        \\
        = {} &
        \sum_{s = 1}^{r} {d_{p,s} a_{i_s}}.
    \end{align*}
\end{theorem}

通俗地, 这个定理说,
任给一个非零阵 \(A\),
我们总能找出它的某 \(r\)~行
\(a_{i_1}\), \(a_{i_2}\), \(\dots\), \(a_{i_r}\),
使 \(A\)~的每一行都可被写为%
这 \(r\)~行的数乘的和,
且由这 \(r\)~行作成的子阵一定有一个%
行列式非零的 \(r\)~级子阵.

\begin{example}
    设
    \begin{align*}
        A =
        \begin{bmatrix}
            1 & 0 & 0 & 0 & 0 & -1 \\
            0 & 1 & 0 & 0 & 0 & 1  \\
            0 & 0 & 1 & 0 & 0 & 0  \\
            0 & 1 & 1 & 0 & 0 & 1  \\
        \end{bmatrix}.
    \end{align*}
    不难看出, \(A\) 有一个 \(3\)~级子阵, 其行列式非零:
    \begin{align*}
        \det {\left(
            A\binom{1,2,3}{1,2,3}
            \right)} = 1.
    \end{align*}
    不过, \(A\) 没有行列式非零的 \(4\)~级子阵.
    我们考虑 \(A\) 的列~\(4\), \(5\).
    任取 \(A\)~的一个 \(4\)~级子阵.
    若 \(A\) 的列~\(4\) 或列~\(5\) 被选中,
    那么其行列式显然为零.
    若 \(A\) 的列~\(4\) 与列~\(5\) 都不被选中,
    则这个子阵一定是
    \begin{align*}
        A\binom{1,2,3,4}{1,2,3,6}
        = \begin{bmatrix}
              1 & 0 & 0 & -1 \\
              0 & 1 & 0 & 1  \\
              0 & 0 & 1 & 0  \\
              0 & 1 & 1 & 1  \\
          \end{bmatrix}.
    \end{align*}
    不难算出, 它的行列式为零.

    我们说, \(A\) 的每一行一定可被写为
    \(A\) 的前 \(3\)~行的数乘的和.
    设 \(A\)~的%
    行~\(1\), \(2\), \(3\), \(4\)
    为 \(a_1\), \(a_2\), \(a_3\), \(a_4\).
    则
    \begin{align*}
        a_1 = 1a_1 + 0a_2 + 0a_3, \\
        a_2 = 0a_1 + 1a_2 + 0a_3, \\
        a_3 = 0a_1 + 0a_2 + 1a_3, \\
        a_4 = 0a_1 + 1a_2 + 1a_3.
    \end{align*}
    并且, 由这 \(3\)~行作成的子阵一定有一个%
    行列式非零的 \(3\)~级子阵:
    \(3\)~级单位阵就是一个.

    顺便一提,
    \(A\) 的每一行当然可被写为
    \(A\) 的前 \(4\)~行的数乘的和:
    \begin{align*}
        a_1 = 1a_1 + 0a_2 + 0a_3 + 0a_4, \\
        a_2 = 0a_1 + 1a_2 + 0a_3 + 0a_4, \\
        a_3 = 0a_1 + 0a_2 + 1a_3 + 0a_4, \\
        a_4 = 0a_1 + 0a_2 + 0a_3 + 1a_4.
    \end{align*}
    可是, 由这 \(4\)~行作成的子阵没有%
    行列式非零的 \(4\)~级子阵.
\end{example}

\begin{proof}
    任取不超过 \(m\) 的正整数 \(p\).

    若 \(p\) 等于某个 \(i_s\)
    (\(s = 1\), \(2\), \(\dots\), \(r\)),
    我们取
    \begin{align*}
        d_{p,v}
        = \begin{cases}
              1, & v = s;    \\
              0, & v \neq s.
          \end{cases}
    \end{align*}
    于是, 对任何不超过 \(n\) 的正整数 \(q\),
    \begin{align*}
        [A]_{i_1,q} d_{p,1}
        + [A]_{i_2,q} d_{p,2}
        + \dots
        + [A]_{i_r,q} d_{p,r}
            = [A]_{i_s,q} 1
            = [A]_{p,q}.
    \end{align*}

    下设 \(p\) 不等于任何 \(i_s\).

    考虑%
    由 \(r\)~个 \(r\)~元 \({\leq} 1\)~次方程作成的方程组
    \begin{align*}
        \begin{cases}
            [A]_{i_1,j_1} x_1 + [A]_{i_2,j_1} x_2
            + \dots + [A]_{i_r,j_1} x_r = [A]_{p,j_1},
            \\
            [A]_{i_1,j_2} x_1 + [A]_{i_2,j_2} x_2
            + \dots + [A]_{i_r,j_2} x_r = [A]_{p,j_2},
            \\
            \dots
            \dots \dots \dots \dots
            \dots \dots \dots \dots
            \dots \dots \dots \dots
            \dots \dots \dots \dots,
            \\
            [A]_{i_1,j_r} x_1 + [A]_{i_2,j_r} x_2
            + \dots + [A]_{i_r,j_r} x_r = [A]_{p,j_r},
        \end{cases}
    \end{align*}
    即
    \begin{align*}
        A_r^{\mathrm{T}}
        \begin{bmatrix}
            x_1    \\
            x_2    \\
            \vdots \\
            x_r    \\
        \end{bmatrix}
        =
        \begin{bmatrix}
            [A]_{p,j_1} \\
            [A]_{p,j_2} \\
            \vdots      \\
            [A]_{p,j_r} \\
        \end{bmatrix}.
    \end{align*}
    因为
    \(\det {(A_r^{\mathrm{T}})}
    = \det {(A_r)} \neq 0\),
    故, 由 Cramer 公式,
    存在 \(r\)~个数
    \(d_{p,1}\), \(d_{p,2}\), \(\dots\), \(d_{p,r}\),
    使
    \begin{align*}
        [A]_{i_1,j_1} d_{p,1} + [A]_{i_2,j_1} d_{p,2}
        + \dots + [A]_{i_r,j_1} d_{p,r} = [A]_{p,j_1},
        \\
        [A]_{i_1,j_2} d_{p,1} + [A]_{i_2,j_2} d_{p,2}
        + \dots + [A]_{i_r,j_2} d_{p,r} = [A]_{p,j_2},
        \\
        \dots \dots \dots
        \dots \dots \dots \dots
        \dots \dots \dots \dots
        \dots \dots \dots \dots
        \dots \dots \dots \dots,
        \\
        [A]_{i_1,j_r} d_{p,1} + [A]_{i_2,j_r} d_{p,2}
        + \dots + [A]_{i_r,j_r} d_{p,r} = [A]_{p,j_r}.
    \end{align*}
    我们由此证明, 对任何不超过 \(n\) 的正整数 \(q\),
    \begin{align*}
        [A]_{p,q}
            =
            [A]_{i_1,q} d_{p,1}
        + [A]_{i_2,q} d_{p,2}
        + \dots
        + [A]_{i_r,q} d_{p,r}.
    \end{align*}

    若 \(q\) 等于某个 \(j_\ell\), 显然.
    下设 \(q\) 不等于任何 \(j_\ell\).

    考虑 \(A\) 的 \(r+1\)~级子阵
    \begin{align*}
        E =
        A\binom{i_1,\dots,i_r,p}{j_1,\dots,j_r,q}.
    \end{align*}
    我们用 ``算二次 (dufoje)'' 思想,
    证我们想要的等式.

    一方面, 我们知道, 既然
    \(A\) 没有行列式非零的 \(r+1\)~级子阵,
    故 \(\det {(E)} = 0\).

    另一方面, 我们也可适当地作一些辅助阵,
    它们的行列式等于 \(E\) 的行列式;
    从而, 计算这些辅助阵的行列式,
    也就相当于计算 \(E\) 的行列式.
    为方便, 我们记
    \(i_{r+1} = p\),
    \(j_{r+1} = q\).
    设 \(i_s\) 是在
    \(i_1\), \(\dots\), \(i_r\), \(i_{r+1}\)
    中第~\(f(i_s)\)~小的数;
    设 \(j_\ell\) 是在
    \(j_1\), \(\dots\), \(j_r\), \(j_{r+1}\)
    中第~\(g(j_\ell)\)~小的数.
    我们加 \(E\) 的行~\(f(i_1)\) 的 \(-d_{p,1}\)~倍%
    于行~\(f(p)\),
    得阵~\(E_1\),
    则 \(\det {(E_1)} = \det {(E)}\).
    我们加 \(E_1\) 的行~\(f(i_2)\) 的 \(-d_{p,2}\)~倍%
    于行~\(f(p)\),
    得阵~\(E_2\),
    则 \(\det {(E_2)} = \det {(E_1)} = \det {(E)}\).
    \(\dots \dots\)
    我们加 \(E_{r-1}\) 的行~\(f(i_r)\) 的 \(-d_{p,r}\)~倍%
    于行~\(f(p)\),
    得阵~\(E_r\),
    则 \(\det {(E_r)} = \det {(E_{r-1})} = \det {(E)}\).
    注意到, \(E_r\)~的行
    \(f(i_1)\), \(f(i_2)\), \(\dots\), \(f(i_r)\)
    分别跟 \(E\)~的行
    \(f(i_1)\), \(f(i_2)\), \(\dots\), \(f(i_r)\)
    相等;
    不过,
    \begin{align*}
        [E_r]_{f(p),g(v)}
        = [A]_{p,v} -
        \sum_{s = 1}^{r}
        {[A]_{i_s,v} d_{p,s}},
    \end{align*}
    其中 \(v = j_1\), \(\dots\), \(j_r\), \(q\).
    所以, 当
    % \(t \neq r+1\)
    \(g(v) \neq g(q)\)
    时,
    % \([E_r]_{f(p),g(j_t)} = 0\).
    \([E_r]_{f(p),g(v)} = 0\).
    我们按行~\(f(p)\) 展开 \(E_r\) 的行列式,
    有
    \begin{align*}
        \det {(E_r)}
        = {} &
        (-1)^{f(p)+g(q)} [E_r]_{f(p),g(q)}
        \det {(E_r (f(p)|g(q)))}
        \\
        = {} &
        (-1)^{f(p)+g(q)} \det {(A_r)}
        \left(
        [A]_{p,q} -
        \sum_{s = 1}^{r}
        {[A]_{i_s,q} d_{p,s}}
        \right).
    \end{align*}

    回想, \(0 = \det {(E)}\),
    且 \(\det {(E)} = \det {(E_r)}\).
    比较二次计算的结果, 我们应有
    \begin{align*}
        0 =
        (-1)^{f(p)+g(q)} \det {(A_r)}
        \left(
        [A]_{p,q} -
        \sum_{s = 1}^{r}
        {[A]_{i_s,q} d_{p,s}}
        \right).
    \end{align*}
    注意到,
    \((-1)^{f(p)+g(q)} \det {(A_r)} \neq 0\),
    故
    \begin{equation*}
        [A]_{p,q} -
        \sum_{s = 1}^{r}
        {[A]_{i_s,q} d_{p,s}} = 0.
        \qedhere
    \end{equation*}
\end{proof}

现在, 我们可以证明本节的重要结论了.

\begin{theorem}
    设 \(A\) 是 \(m \times n\)~阵.
    设 \(B\) 是 \(m \times 1\)~阵.
    作一个 \(m \times (n+1)\)~阵 \(G\),
    其中
    \begin{align*}
        [G]_{i,j}
        = \begin{cases}
              [A]_{i,j}, & j \leq n;  \\
              [B]_{i,1}, & j = n + 1.
          \end{cases}
    \end{align*}
    (通俗地, 在 \(A\)~的最后一列的右侧加入一列 \(B\),
    得到尺寸较大的阵 \(G\).)
    设
    \(A\) 有一个行列式非零的 \(r\)~级子阵
    (从而 \(G\) 也有一个行列式非零的 \(r\)~级子阵),
    但 \(G\) 没有行列式非零的 \(r+1\)~级子阵
    (从而 \(A\) 也没有行列式非零的 \(r+1\)~级子阵).
    则存在一个 \(n \times 1\)~阵 \(C\),
    使 \(AC = B\).
\end{theorem}

\begin{proof}
    先设 \(A = 0\).
    那么, \(A\) 有一个行列式非零的 \(0\)~级子阵,
    但 \(G\) 没有行列式非零的 \(1\)~级子阵.
    所以 \(G = 0\).
    所以 \(B = 0\).
    那么, 任何一个 \(n \times 1\)~阵 \(C\)
    都适合 \(AC = B\).

    下设 \(A \neq 0\).
    设 \(A\) 的 \(r\)~级子阵
    \begin{align*}
        A_r = A\binom{i_1,\dots,i_r}{j_1,\dots,j_r}
    \end{align*}
    (其中
    \(1 \leq i_1 < \dots < i_r \leq m\),
    \(1 \leq j_1 < \dots < j_r \leq n\))
    的行列式非零.
    \(A_r\) 当然也是 \(G\)~的子阵, 且
    \begin{align*}
        A\binom{i_1,\dots,i_r}{j_1,\dots,j_r}
        =
        G\binom{i_1,\dots,i_r}{j_1,\dots,j_r}.
    \end{align*}
    所以, 对任何不超过 \(m\) 的正整数 \(p\),
    一定存在 \(r\)~个数
    \(d_{p,1}\), \(d_{p,2}\), \(\dots\), \(d_{p,r}\),
    使对任何不超过 \(n+1\) 的正整数 \(q\),
    \begin{align*}
        [G]_{p,q}
        = {} &
        [G]_{i_1,q} d_{p,1}
        + [G]_{i_2,q} d_{p,2}
        + \dots
        + [G]_{i_r,q} d_{p,r}
        \\
        = {} &
        \sum_{s = 1}^{r} {[G]_{i_s,q} d_{p,s}}.
    \end{align*}

    考虑%
    由 \(r\)~个 \(n\)~元 \({\leq} 1\)~次方程作成的方程组
    \begin{align*}
        \begin{cases}
            [A]_{i_1,1} x_1 + [A]_{i_1,2} x_2 + \dots
            + [A]_{i_1,n} x_n = [B]_{i_1,1},
            \\
            [A]_{i_2,1} x_1 + [A]_{i_2,2} x_2 + \dots
            + [A]_{i_2,n} x_n = [B]_{i_2,1},
            \\
            \dots
            \dots \dots \dots \dots
            \dots \dots \dots \dots
            \dots \dots \dots \dots
            \dots \dots \dots \dots,
            \\
            [A]_{i_r,1} x_1 + [A]_{i_r,2} x_2 + \dots
            + [A]_{i_r,n} x_n = [B]_{i_r,1}.
        \end{cases}
    \end{align*}
    我们说, 这个方程组一定有解.

    若 \(r = n\), 则这是一个%
    由 \(n\)~个 \(n\)~元 \({\leq} 1\)~次方程作成的方程组.
    因为 \(\det {(A_r)} \neq 0\),
    故, 由 Cramer 公式,
    此方程组有一个 (唯一的) 解.

    若 \(r < n\),
    从 \(1\), \(2\), \(\dots\), \(n\)
    去除 \(j_1\), \(j_2\), \(\dots\), \(j_r\)
    后, 还剩 \(n - r\)~个数.
    我们从小到大地叫这 \(n - r\)~个数为
    \(j_{r+1}\), \(\dots\), \(j_n\).
    我们可改写此方程组为
    \begin{align*}
        \sum_{\ell = 1}^{r}
        {[A]_{p,j_\ell} x_{j_\ell}}
            = [B]_{p,1}
        - \sum_{r < \ell \leq n}
        {[A]_{p,j_\ell} x_{j_\ell}},
    \end{align*}
    其中 \(p = i_1\), \(i_2\), \(\dots\), \(i_r\),
    下同.

    设
    \(f_{j_{r+1}}\), \(\dots\), \(f_{j_r}\)
    为任何的 \(n - r\) 个数.
    我们考虑%
    由 \(r\)~个 \(r\)~元 \({\leq} 1\)~次方程作成的方程组
    \begin{align*}
        \sum_{\ell = 1}^{r}
        {[A]_{p,j_\ell} y_\ell}
        = [B]_{p,1}
        - \sum_{r < \ell \leq n}
        {[A]_{p,j_\ell} f_{j_\ell}}.
    \end{align*}
    利用阵等式, 我们可写
    \begin{align*}
        A_r
        \begin{bmatrix}
            y_1    \\
            y_2    \\
            \vdots \\
            y_r    \\
        \end{bmatrix}
        =
        \begin{bmatrix}
            z_{i_1} \\
            z_{i_2} \\
            \vdots  \\
            z_{i_r} \\
        \end{bmatrix},
    \end{align*}
    其中
    \begin{align*}
        z_p = [B]_{p,1}
        - \sum_{r < \ell \leq n}
        {[A]_{p,j_\ell} f_{j_\ell}}.
    \end{align*}
    因为 \(\det {(A_r)} \neq 0\),
    故, 由 Cramer 公式,
    存在 \(r\) 个数
    \(f_{j_1}\), \(f_{j_2}\), \(\dots\), \(f_{j_r}\),
    使
    \begin{align*}
        \sum_{\ell = 1}^{r}
        {[A]_{p,j_\ell} f_{j_\ell}}
            = [B]_{p,1}
        - \sum_{r < \ell \leq n}
        {[A]_{p,j_\ell} f_{j_\ell}}.
    \end{align*}
    从而
    \begin{align*}
        \sum_{\ell = 1}^{n}
        {[A]_{p,j_\ell} f_{j_\ell}}
            = [B]_{p,1}.
    \end{align*}

    设 \(n\) 个数
    \(c_1\), \(c_2\), \(\dots\), \(c_n\)
    适合
    \begin{align*}
        [A]_{i_1,1} c_1 + [A]_{i_1,2} c_2 + \dots
        + [A]_{i_1,n} c_n = [B]_{i_1,1},
        \\
        [A]_{i_2,1} c_1 + [A]_{i_2,2} c_2 + \dots
        + [A]_{i_2,n} c_n = [B]_{i_2,1},
        \\
        \dots
        \dots \dots \dots \dots
        \dots \dots \dots \dots
        \dots \dots \dots \dots
        \dots \dots \dots \dots,
        \\
        [A]_{i_r,1} c_1 + [A]_{i_r,2} c_2 + \dots
        + [A]_{i_r,n} c_n = [B]_{i_r,1}.
    \end{align*}
    作 \(n \times 1\)~阵 \(C\),
    其中 \([C]_{i,1} = c_i\).
    我们证明, \(AC = B\).

    若 \(i\) 等于某 \(i_s\), 则
    \begin{align*}
        [AC]_{i,1}
        = {} &
        \sum_{\ell = 1}^{n}
        {[A]_{i,\ell} [C]_{\ell,1}}
        \\
        = {} &
        \sum_{\ell = 1}^{n}
        {[A]_{i,\ell} c_\ell}
        \\
        = {} & [B]_{i,1}.
    \end{align*}
    若 \(i\) 不等于任何一个 \(i_s\), 则
    \begin{align*}
        [AC]_{i,1}
        = {} &
        \sum_{\ell = 1}^{n}
        {[A]_{i,\ell} [C]_{\ell,1}}
        \\
        = {} &
        \sum_{\ell = 1}^{n}
        {[G]_{i,\ell} [C]_{\ell,1}}
        \\
        = {} &
        \sum_{\ell = 1}^{n}
        {
        \left( \sum_{s = 1}^{r}
        {
            [G]_{i_s,\ell} d_{i,s}
        } \right) [C]_{\ell,1}
        }
        \\
        = {} &
        \sum_{\ell = 1}^{n}
        {
        \sum_{s = 1}^{r}
        {
        [G]_{i_s,\ell} d_{i,s} [C]_{\ell,1}
        }
        }
        \\
        = {} &
        \sum_{s = 1}^{r}
        {
        \sum_{\ell = 1}^{n}
        {
        [G]_{i_s,\ell} d_{i,s} [C]_{\ell,1}
        }
        }
        \\
        = {} &
        \sum_{s = 1}^{r}
        {
        \sum_{\ell = 1}^{n}
        {
        [A]_{i_s,\ell} d_{i,s} c_\ell
        }
        }
        \\
        = {} &
        \sum_{s = 1}^{r}
        {
        \sum_{\ell = 1}^{n}
        {
        [A]_{i_s,\ell} c_\ell d_{i,s}
        }
        }
        \\
        = {} &
        \sum_{s = 1}^{r}
        {
        \left(\sum_{\ell = 1}^{n}
        {
            [A]_{i_s,\ell} c_\ell
        }\right)
        d_{i,s}
        }
        \\
        = {} &
        \sum_{s = 1}^{r}
        {[B]_{i_s,1} d_{i,s}}
        \\
        = {} &
        \sum_{s = 1}^{r}
        {[G]_{i_s,n+1} d_{i,s}}
        \\
        = {} &
        [G]_{i,n+1}
        \\
        = {} &
        [B]_{i,1}.
        \qedhere
    \end{align*}
\end{proof}

结合前几节的讨论, 我们可以得到判断
\(AX = B\) 是否有解,
与有解时其解是否唯一的方法
(定性的理论, 完整版):

\begin{restatable}[]{theorem}%
    {TheoremQualitativeTheoryOfLinearSystemOfEquations}
    设 \(A\) 是 \(m \times n\)~阵.
    设 \(B\) 是 \(m \times 1\)~阵.
    作一个 \(m \times (n+1)\)~阵 \(G\),
    其中
    \begin{align*}
        [G]_{i,j}
        = \begin{cases}
              [A]_{i,j}, & j \leq n;  \\
              [B]_{i,1}, & j = n + 1.
          \end{cases}
    \end{align*}
    (通俗地, 在 \(A\)~的最后一列的右侧加入一列 \(B\),
    得到尺寸较大的阵 \(G\).)
    设
    \(A\) 有一个行列式非零的 \(r\)~级子阵,
    但没有行列式非零的 \(r+1\)~级子阵.

    (1)
    若 \(G\)~也没有行列式非零的 \(r+1\)~级子阵,
    则 \(AX = B\) 有解.
    进一步地, 若 \(r < n\),
    则 \(AX = B\) 有无限多个解;
    若 \(r = n\),
    则 \(AX = B\) 有唯一的解.

    (2)
    若 \(G\)~有一个行列式非零的 \(r+1\)~级子阵,
    则 \(AX = B\) 无解.
\end{restatable}

\section{\texorpdfstring{由 \(m\)~个 \(n\)~元
      \({\leq} 1\)~次方程作成的方程组 (4)}%
  {由 m 个 n 元 ≤1 次方程作成的方程组 (4)}}

% 本节是 ``选学内容中的选学内容'';
% 换句话说, 您不学它, 不影响您理解别的选学内容.
% 毕竟, 本节的公式, 几乎无用,
% 因为, 若您想用它计算线性方程组的解,
% 您要计算好几个阵的行列式.
% 计算一个阵的行列式已经是一件较复杂的事了,
% 计算好几个阵的行列式自然就是一件更复杂的事.
% 不过, 本节自然是可以作为阅读材料的.

\maldevigalegajxo

前面, 我们得到了线性方程组的解的定性的理论:

\TheoremQualitativeTheoryOfLinearSystemOfEquations*

现在, 我们作定量的讨论:
当 \(AX = B\) 有解时,
我们试作出公式, 以表示它的解.

设 \(A\) 是一个 \(m \times n\)~阵.
设 \(B\) 是一个 \(m \times 1\)~阵.
设 \(AX = B\) 有解.

若 \(A = 0\),
则因 \(AX = B\) 有解,
必有 \(B = 0\).
此时, 显然, 每一个 \(n \times 1\)~阵都是解.
下设 \(A \neq 0\).
设 \(A\) 有一个行列式非零的 \(r\)~级子阵
\begin{align*}
    T = A\binom{i_1,\dots,i_r}{j_1,\dots,j_r}
\end{align*}
(其中
\(1 \leq i_1 < \dots < i_r \leq m\),
\(1 \leq j_1 < \dots < j_r \leq n\)),
但没有行列式非零的 \(r+1\)~级子阵.
为说话方便, 我们作一个 \(r \times n\)~阵
\begin{align*}
    U = A\binom{i_1,\dots,i_r}{1,\dots,n},
\end{align*}
与一个 \(r \times 1\)~阵
\begin{align*}
    V = B\binom{i_1,\dots,i_r}{1}
    =
    \begin{bmatrix}
        [B]_{i_1,1} \\
        [B]_{i_2,1} \\
        \vdots      \\
        [B]_{i_r,1} \\
    \end{bmatrix}.
\end{align*}
注意到, \(U\) 有一个行列式非零的 \(r\)~级子阵 \(T\).

我们回想上节的讨论.
为了解方程组 \(AX = B\), 即
\begin{align*}
    \begin{cases}
        [A]_{1,1} x_1 + [A]_{1,2} x_2 + \dots
        + [A]_{1,n} x_n = [B]_{1,1},
        \\
        [A]_{2,1} x_1 + [A]_{2,2} x_2 + \dots
        + [A]_{2,n} x_n = [B]_{2,1},
        \\
        \dots
        \dots \dots \dots \dots
        \dots \dots \dots \dots
        \dots \dots \dots \dots
        \dots \dots \dots \dots,
        \\
        [A]_{m,1} x_1 + [A]_{m,2} x_2 + \dots
        + [A]_{m,n} x_n = [B]_{m,1},
    \end{cases}
\end{align*}
我们考虑了由这 \(m\)~个方程的%
第~\(i_1\), \(i_2\), \(\dots\), \(i_r\)~个方程%
作成的方程组
\begin{align*}
    \begin{cases}
        [A]_{i_1,1} x_1 + [A]_{i_1,2} x_2 + \dots
        + [A]_{i_1,n} x_n = [B]_{i_1,1},
        \\
        [A]_{i_2,1} x_1 + [A]_{i_2,2} x_2 + \dots
        + [A]_{i_2,n} x_n = [B]_{i_2,1},
        \\
        \dots
        \dots \dots \dots \dots
        \dots \dots \dots \dots
        \dots \dots \dots \dots
        \dots \dots \dots \dots,
        \\
        [A]_{i_r,1} x_1 + [A]_{i_r,2} x_2 + \dots
        + [A]_{i_r,n} x_n = [B]_{i_r,1},
    \end{cases}
\end{align*}
即 \(UX = V\).
当时, 我们已经证明了,
\(UX = V\) 的解都是 \(AX = B\) 的解.
反过来, \(AX = B\) 的解显然都是 \(UX = V\) 的解,
因为后者的方程全部都是来自前者的.
这么看来, \(AX = B\) 跟 \(UX = V\) 有着相同的解.
所以, 研究 \(AX = B\) 的解的公式,
相当于研究 \(UX = V\) 的解的公式.

若 \(r = n\),
则我们直接用 Cramer 公式,
即可写出 \(UX = V\) 的唯一的解
(注意, 此时 \(U\) 是一个 \(r \times r\)~阵)
\begin{align*}
    X = \frac{1}{\det {(U)}} \operatorname{adj} {(U)}\,V.
\end{align*}
我们也可较直接地表达此解.
设 \(U\{k,V\}\) 是以 \(r \times 1\)~阵 \(V\)
代 \(U\) 的列~\(k\) 后得到的阵.
则
\begin{align*}
    x_k = \frac{\det {(U\{k,V\})}}{\det {(U)}}.
\end{align*}

下设 \(r < n\).
我们试写出方程组 \(UX = V\) 的所有的解.

\begin{theorem}
    设 \(V\) 是一个 \(r \times 1\)~阵.
    设 \(U\) 是一个 \(r \times n\)~阵,
    其中 \(r < n\),
    且 \(U\) 有一个行列式非零的 \(r\)~级子阵
    \begin{align*}
        T = U\binom{1,2,\dots,r}{j_1,j_2,\dots,j_r},
    \end{align*}
    其中 \(1 \leq j_1 < \dots < j_r \leq n\).
    从 \(1\), \(2\), \(\dots\), \(n\)
    去除 \(j_1\), \(j_2\), \(\dots\), \(j_r\)
    后, 还剩 \(n - r\)~个数.
    我们从小到大地叫这 \(n - r\)~个数为
    \(j_{r+1}\), \(\dots\), \(j_n\).
    再设 \(U\)~的%
    列~\(1\), \(2\), \(\dots\), \(n\)
    分别是 \(u_1\), \(u_2\), \(\dots\), \(u_n\).

    (1)
    设 \(c_{j_{r+1}}\), \(\dots\), \(c_{j_n}\)
    是 \(n-r\)~个常数.
    作 \(n \times 1\)~阵 \(C\), 其中
    \begin{align*}
        [C]_{j_k,1}
        = \begin{dcases}
              \frac{\det {(T\{k,V\})}}{\det {(T)}}
              - \sum_{r < \ell \leq n}
              {c_{j_\ell}
              \frac{\det {(T\{k,u_{j_\ell}\})}}{\det {(T)}}},
               & k \leq r; \\
              c_{j_k},
               & k > r.
          \end{dcases}
    \end{align*}
    其中 \(T\{k,Y\}\) 是以 \(r \times 1\)~阵 \(Y\)
    代 \(T\) 的列~\(k\) 后得到的阵.
    则 \(UC = V\).

    (2)
    若 \(n \times 1\)~阵 \(D\) 适合 \(UD = V\),
    则存在 \(n-r\)~个数
    \(c_{j_{r+1}}\), \(\dots\), \(c_{j_n}\),
    使
    \begin{align*}
        [D]_{j_k,1}
        = \begin{dcases}
              \frac{\det {(T\{k,V\})}}{\det {(T)}}
              - \sum_{r < \ell \leq n}
              {c_{j_\ell}
              \frac{\det {(T\{k,u_{j_\ell}\})}}{\det {(T)}}},
               & k \leq r; \\
              c_{j_k},
               & k > r.
          \end{dcases}
    \end{align*}
    换句话说, \(UX = V\) 的每一个解%
    都可被 (1) 中的公式表达.
\end{theorem}

\begin{proof}
    (1)
    我们可改写方程组 \(UX = V\) 为
    \begin{align*}
        \sum_{\ell = 1}^{r}
        {[U]_{p,j_\ell} x_{j_\ell}}
            = [V]_{p,1}
        - \sum_{r < \ell \leq n}
        {[U]_{p,j_\ell} x_{j_\ell}},
    \end{align*}
    其中 \(p = 1\), \(2\), \(\dots\), \(r\),
    下同.
    考虑%
    由 \(r\)~个 \(r\)~元 \({\leq} 1\)~次方程作成的方程组
    \begin{align*}
        \sum_{\ell = 1}^{r}
        {[U]_{p,j_\ell} y_\ell}
        = [V]_{p,1}
        - \sum_{r < \ell \leq n}
        {[U]_{p,j_\ell} c_{j_\ell}}.
    \end{align*}
    利用阵等式, 我们可写
    \begin{align*}
        T
        \begin{bmatrix}
            y_1    \\
            y_2    \\
            \vdots \\
            y_r    \\
        \end{bmatrix}
        =
        V -
        \sum_{r < \ell \leq n}
        {c_{j_\ell} u_{j_\ell}}.
    \end{align*}
    因为 \(\det {(T)} \neq 0\),
    故, 由 Cramer 公式,
    存在 \(r\) 个数
    \(c_{j_1}\), \(c_{j_2}\), \(\dots\), \(c_{j_r}\),
    使
    \begin{align*}
        \sum_{\ell = 1}^{r}
        {[U]_{p,j_\ell} c_{j_\ell}}
            = [V]_{p,1}
        - \sum_{r < \ell \leq n}
        {[U]_{p,j_\ell} c_{j_\ell}},
    \end{align*}
    其中
    \begin{align*}
        c_{j_k}
        = {} &
        \frac{1}{\det {(T)}}
        \det {
            \left(
            T\left\{
            k,
            V -
            \sum_{r < \ell \leq n}
            {c_{j_\ell} u_{j_\ell}}
            \right\}
            \right)
        }
        \\
        = {} &
        \frac{1}{\det {(T)}} \det {(T\{k,V\})}
        -
        \frac{1}{\det {(T)}}
        \sum_{r < \ell \leq n}
        {c_{j_\ell}
        \det {(T\{k,u_{j_\ell}\})}
        }
        \\
        = {} &
        \frac{\det {(T\{k,V\})}}{\det {(T)}}
        - \sum_{r < \ell \leq n}
        {c_{j_\ell}
        \frac{\det {(T\{k,u_{j_\ell}\})}}{\det {(T)}}}.
    \end{align*}
    从而
    \begin{align*}
        \sum_{\ell = 1}^{n}
        {[U]_{p,j_\ell} c_{j_\ell}}
            = [V]_{p,1}.
    \end{align*}
    注意到 \([C]_{j_k,1} = c_{j_k}\).
    所以, \(UC = V\).

    (2)
    设 \(D\) 适合 \(UD = V\).
    则
    \begin{align*}
        \sum_{\ell = 1}^{n}
        {[U]_{p,j_\ell} [D]_{j_\ell,1}}
            = [V]_{p,1}.
    \end{align*}
    从而
    \begin{align*}
        \sum_{\ell = 1}^{r}
        {[U]_{p,j_\ell} [D]_{j_\ell,1}}
            = [V]_{p,1}
        - \sum_{r < \ell \leq n}
        {[U]_{p,j_\ell} [D]_{j_\ell,1}}.
    \end{align*}

    取 \(c_{j_\ell} = [D]_{j_\ell,1}\),
    \(\ell > r\).
    考虑方程组
    \begin{align*}
        \sum_{\ell = 1}^{r}
        {[U]_{p,j_\ell} y_\ell}
        = [V]_{p,1}
        - \sum_{r < \ell \leq n}
        {[U]_{p,j_\ell} c_{j_\ell}}.
    \end{align*}
    由 (1), 我们知道
    \begin{align*}
        y_k =
        \frac{\det {(T\{k,V\})}}{\det {(T)}}
        - \sum_{r < \ell \leq n}
        {c_{j_\ell}
        \frac{\det {(T\{k,u_{j_\ell}\})}}{\det {(T)}}}
    \end{align*}
    (其中 \(k = 1\), \(2\), \(\dots\), \(r\), 下同)
    是一个解;
    另一方面,
    \(y_k = [D]_{j_k,1}\) 也是一个解.
    因为 \(\det {(T)} \neq 0\),
    故, 由 Cramer 公式,
    这二组解应是相同的,
    即
    \begin{equation*}
        [D]_{j_k,1} =
        \frac{\det {(T\{k,V\})}}{\det {(T)}}
        - \sum_{r < \ell \leq n}
        {c_{j_\ell}
        \frac{\det {(T\{k,u_{j_\ell}\})}}{\det {(T)}}}.
        \qedhere
    \end{equation*}
\end{proof}

现在, 我们作一个小结.

注意到, 当 \(r = n\) 时,
不可能有数 \(\ell\) 适合 \(r < \ell \leq n\).
既然没有数被加, 那么, 形如
\begin{align*}
    \sum_{r < \ell \leq n} {f(\ell)}
\end{align*}
的式是零.
用这种约定,
我们可统一地写%
在 \(r = n\) 与 \(r < n\) 这二种情形下%
解的公式.

\begin{theorem}
    设 \(V\) 是一个 \(r \times 1\)~阵.
    设 \(U\) 是一个 \(r \times n\)~阵,
    其中 \(r \leq n\),
    且 \(U\) 有一个行列式非零的 \(r\)~级子阵
    \begin{align*}
        T = U\binom{1,2,\dots,r}{j_1,j_2,\dots,j_r},
    \end{align*}
    其中 \(1 \leq j_1 < \dots < j_r \leq n\).
    从 \(1\), \(2\), \(\dots\), \(n\)
    去除 \(j_1\), \(j_2\), \(\dots\), \(j_r\)
    后, 还剩 \(n - r\)~个数.
    我们从小到大地叫这 \(n - r\)~个数为
    \(j_{r+1}\), \(\dots\), \(j_n\).
    再设 \(U\)~的%
    列~\(1\), \(2\), \(\dots\), \(n\)
    分别是 \(u_1\), \(u_2\), \(\dots\), \(u_n\).
    那么, \(UX = V\) 的解\emph{恰}为
    \begin{align*}
        x_{j_k}
        = \begin{dcases}
              \frac{\det {(T\{k,V\})}}{\det {(T)}}
              - \sum_{r < \ell \leq n}
              {c_{j_\ell}
              \frac{\det {(T\{k,u_{j_\ell}\})}}{\det {(T)}}},
               & k \leq r; \\
              c_{j_k},
               & k > r,
          \end{dcases}
    \end{align*}
    其中 \(c_{j_{r+1}}\), \(\dots\), \(c_{j_n}\)
    是任何的 \(n-r\) 个数.
\end{theorem}

\section{\texorpdfstring{由 \(m\)~个 \(n\)~元
      \({\leq} 1\)~次方程作成的方程组 (5)}%
  {由 m 个 n 元 ≤1 次方程作成的方程组 (5)}}

\maldevigalegajxo

线性方程组的故事要结束了.
毕竟, 我们已解决线性方程组的三个较重要的问题:

(1)
一个线性方程组何时有解?

(2)
一个线性方程组有解时, 其解是否唯一?

(3)
一个线性方程组有解时, 我们如何找到它的全部的解?

我们用行列式回答了这三个问题,
作了一个较完整的线性方程组的理论.
所以, 理论地, 给了一个线性方程组,
我们可以计算一些行列式以确定它是否有解,
且有解时其解为何.

不过,
计算一个方阵的行列式并不是什么简单的事情:
\(3\)~级阵的行列式的较具体的公式含 \(6\)~项,
而 \(4\)~级阵的行列式的较具体的公式含 \(24\)~项.
所以, 像 Cramer 公式那样,
我们又作了一个 ``较理论的'' 理论.

但是, 此理论仍然是重要的.
或许, 您还记得,
任给一个阵~\(A\),
必存在唯一的非负整数 \(r\),
使 \(A\)~有一个行列式非零的 \(r\)~级子阵,
但 \(A\)~没有行列式非零的 \(r+1\)~级子阵.
聪明的数学家意识到,
此 \(r\) 十分地重要,
故专门为它起了一个名字, ``\emph{秩}''.
然后, 他们研究了秩,
发现:
(a)
可以施适当的变换于阵~\(A\),
得到一个新阵~\(E\),
且 \(A\) 与 \(E\) 有相同的秩;
(b)
``适当的变换'' 跟计算方阵的行列式比,
有着较少的计算量
(具体地, 加、减、乘、除的次数);
(c)
\(E\)~的秩不难计算,
甚至可被直接看出.
于是, 联合这个发现与我们的理论,
就是一个更好的线性方程组的理论.
若您对此事感兴趣,
您可以见线性代数 (或高等代数) 教材.
% 此事虽好,
% 但可能不适合在一门主讲行列式的课程里被具体地讨论.

我就说这么多吧.

\SenAsteriskoEnEnhavtabelo


\KunAsteriskoEnEnhavtabelo
\section{Binet--Cauchy 公式}
\SenAsteriskoEnEnhavtabelo

\maldevigalegajxo

或许, (方) 阵的行列式与阵的积的定义都是较复杂的
(跟阵的转置、加、减、数乘等对比).
不过, 这二种较复杂的运算有着较不平凡的联系.

设 \(A\) 是一个 \(m \times n\)~阵,
\(B\) 是一个 \(n \times m\)~阵.
根据阵的积的定义,
\(BA\) 是一个 \(n\)~级阵,
故它有行列式.

\begin{theorem}[Binet--Cauchy 公式]
    设 \(A\), \(B\) 分别是 \(m \times n\), \(n \times m\)~阵.

    (1)
    若 \(n > m\), 则 \(\det {(BA)} = 0\).

    (2)
    若 \(n \leq m\),
    则
    \begin{align*}
             & \det {(BA)}
        \\
        = {} & \sum_{1 \leq j_1 < j_2 < \dots < j_n \leq m}
        {
            \det {\left(
                B\binom{1, 2, \dots, n}{j_1, j_2, \dots, j_n}
                \right)}
            \det {\left(
                A\binom{j_1, j_2, \dots, j_n}{1, 2, \dots, n}
                \right)}
        }.
    \end{align*}
    特别地, 若 \(n = m\), 则因适合条件
    \(1 \leq j_1 < j_2 < \dots < j_n \leq m\)
    的 \(j_1\), \(j_2\), \(\dots\), \(j_n\),
    是, 且只能是,
    \(1\), \(2\), \(\dots\), \(n\),
    故
    \begin{align*}
        \det {(BA)}
        = {} &
        \det {\left(
            B\binom{1, 2, \dots, n}{1, 2, \dots, n}
            \right)}
        \det {\left(
            A\binom{1, 2, \dots, n}{1, 2, \dots, n}
            \right)}
        \\
        = {} & \det {(B)} \det {(A)}.
    \end{align*}
\end{theorem}

我用二个例助您理解, 此定理在说什么.

\begin{example}
    设
    \begin{align*}
        A =
        \begin{bmatrix}
            1 & 3 & 5 \\
            2 & 4 & 6 \\
        \end{bmatrix},
        \quad
        B =
        \begin{bmatrix}
            7 & 10 \\
            8 & 11 \\
            9 & 12 \\
        \end{bmatrix}.
    \end{align*}
    不难算出,
    \begin{align*}
        AB =
        \begin{bmatrix}
            76  & 103 \\
            100 & 136 \\
        \end{bmatrix},
        \quad
        BA =
        \begin{bmatrix}
            27 & 61 & 95  \\
            30 & 68 & 106 \\
            33 & 75 & 117 \\
        \end{bmatrix}.
    \end{align*}
    直接计算, 可知
    \begin{align*}
        \det {(AB)} = 76 \cdot 136 - 100 \cdot 103 = 36,
    \end{align*}
    而
    \begin{align*}
        \det {(BA)}
        = {} & \hphantom{{} + {}}
        27 \cdot 68 \cdot 117
        + 30 \cdot 75 \cdot 95
        + 33 \cdot 61 \cdot 106
        \\
             &
        - 27 \cdot 75 \cdot 106
        - 30 \cdot 61 \cdot 117
        - 33 \cdot 68 \cdot 95
        \\
        = {} & 0.
    \end{align*}

    或许, 前面的计算是较复杂的.
    我们试用 Binet--Cauchy 公式, 再计算它们.

    因为 \(A\), \(B\) 的尺寸分别是 \(2 \times 3\), \(3 \times 2\),
    而 \(3 > 2\),
    根据 (1), \(\det {(BA)} = 0\)
    (注意到, \(BA\) 是 \(3\)~级阵).

    因为 \(2 \leq 3\), 故, 根据 (2),
    \begin{align*}
             & \det {(AB)}
        \\
        = {} &
        \hphantom{{} + {}}
        \det {\left(A\binom{1,2}{1,2}\right)}
        \det {\left(B\binom{1,2}{1,2}\right)}
        + \det {\left(A\binom{1,2}{1,3}\right)}
        \det {\left(B\binom{1,3}{1,2}\right)}
        \\
             &
        + \det {\left(A\binom{1,2}{2,3}\right)}
        \det {\left(B\binom{2,3}{1,2}\right)}.
    \end{align*}
    这里, 要注意到, 适合条件
    ``\(1 \leq j_1 < j_2 \leq 3\)''
    的 \((j_1, j_2)\)
    恰有三组:
    \((1, 2)\), \((1, 3)\), \((2, 3)\).
    不难写出
    \begin{align*}
         & A\binom{1,2}{1,2}
        = \begin{bmatrix} 1&3\\2&4\\ \end{bmatrix},
        \quad
        B\binom{1,2}{1,2}
        = \begin{bmatrix} 7&10\\8&11\\ \end{bmatrix}, \\
         & A\binom{1,2}{1,3}
        = \begin{bmatrix} 1&5\\2&6\\ \end{bmatrix},
        \quad
        B\binom{1,3}{1,2}
        = \begin{bmatrix} 7&10\\9&12\\ \end{bmatrix}, \\
         & A\binom{1,2}{2,3}
        = \begin{bmatrix} 3&5\\4&6\\ \end{bmatrix},
        \quad
        B\binom{2,3}{1,2}
        = \begin{bmatrix} 8&11\\9&12\\ \end{bmatrix}.
    \end{align*}
    从而
    \begin{align*}
        \det {(AB)}
        = (-2) \cdot (-3) + (-4) \cdot (-6) + (-2) \cdot (-3)
        = 36.
    \end{align*}
\end{example}

\begin{example}
    设
    \begin{align*}
        A = \begin{bmatrix}
                2 & 5 \\
                3 & 7 \\
            \end{bmatrix},
        \quad
        B = \begin{bmatrix}
                9 & 6 \\
                4 & 8 \\
            \end{bmatrix}.
    \end{align*}
    不难算出
    \begin{align*}
        AB
        = \begin{bmatrix}
              38 & 52 \\
              55 & 74 \\
          \end{bmatrix},
        \quad
        BA
        = \begin{bmatrix}
              36 & 87 \\
              32 & 76 \\
          \end{bmatrix}.
    \end{align*}
    可以看到, \(AB \neq BA\).

    不过, \(\det {(AB)} = \det {(BA)}\).
    一方面, 我们可直接验证:
    \begin{align*}
         & \det {(AB)} = 38 \cdot 74 - 55 \cdot 52 = -48, \\
         & \det {(BA)} = 36 \cdot 76 - 32 \cdot 87 = -48.
    \end{align*}
    另一方面, Binet--Cauchy 公式指出,
    \begin{align*}
        \det {(AB)}
        = {} & \det {(A)} \det {(B)} \\
        = {} & \det {(B)} \det {(A)} \\
        = {} & \det {(BA)}.
    \end{align*}
    毕竟, 数的乘法是可换的.
\end{example}

为较简单地论证公式,
我们要三个新事实;
为证明这三个新事实的前二个,
我们要一些老的公式.

\TheoremFullExpansion*

取 \(j_1\), \(j_2\), \(\dots\), \(j_n\)
为 \(1\), \(2\), \(\dots\), \(n\),
并注意到 \(s(1, 2, \dots, n) = 1\),
有
\begin{align*}
    \det {(A)}
    = {} &
    \sum_{\substack{
    1 \leq i_1, i_2, \dots, i_n \leq n \\
            i_1, i_2, \dots, i_n\,\text{互不相同}
        }}
    {s(i_1, i_2, \dots, i_n)\,
        [A]_{i_1,1} [A]_{i_2,2} \dots [A]_{i_n,n}}.
\end{align*}

\TheoremColumnSwapAndSign*

现在, 我要开始展现三个新事实了.

\begin{theorem}
    设 \(C\) 为 \(n \times m\)~阵,
    且 \(m \geq n\).
    设 \(C\)~的列~\(1\), \(2\), \(\dots\), \(m\)
    分别为 \(c_1\), \(c_2\), \(\dots\), \(c_m\).
    设 \(k_1\), \(k_2\), \(\dots\), \(k_n\) 是%
    不超过 \(m\)~的正整数,
    且互不相同.
    设 \(j_1\), \(j_2\), \(\dots\), \(j_n\) 是
    \(k_1\), \(k_2\), \(\dots\), \(k_n\) 的自然排列.
    则
    \begin{align*}
        \det {[c_{k_1}, c_{k_2}, \dots, c_{k_n}]}
        = s(k_1, k_2, \dots, k_n)
        \det {[c_{j_1}, c_{j_2}, \dots, c_{j_n}]}.
    \end{align*}
\end{theorem}

\begin{proof}
    设 \(k_t\) 在 \(k_1\), \(k_2\), \(\dots\), \(k_n\)
    的自然排列
    \(j_1\), \(j_2\), \(\dots\), \(j_n\)
    里的位置为 \(f(k_t)\)
    (\(t = 1\), \(2\), \(\dots\), \(n\)).
    注意, \(f(j_t) = t\).
    并且, 若 \(k_t < k_v\), 必有 \(f(k_t) < f(k_v)\).
    反过来, 若 \(f(k_t) < f(k_v)\), 必有 \(k_t < k_v\).
    (我用一个简单的例助您理解.
    设 \(m = 9\), \(n = 3\).
    设 \(k_1\), \(k_2\), \(k_3\) 分别为 \(8\), \(9\), \(6\).
    那么, \(k_1\), \(k_2\), \(k_3\) 的自然排列
    \(j_1\), \(j_2\), \(j_3\)
    是
    \(6\), \(8\), \(9\).
    故 \(f(k_1) = f(j_2) = f(8) = 2\),
    \(f(k_2) = f(j_3) = f(9) = 3\),
    \(f(k_3) = f(j_1) = f(6) = 1\).)
    从而
    \(\operatorname{sgn} {(k_v - k_t)}
    = \operatorname{sgn} {(f(k_v) - f(k_t))}\).

    作阵~\(G = [c_{j_1}, c_{j_2}, \dots, c_{j_n}]\).
    设 \(G\) 的列
    \(1\), \(2\), \(\dots\), \(n\)
    分别是
    \(g_1\), \(g_2\), \(\dots\), \(g_n\).
    则
    \begin{align*}
        [c_{k_1}, c_{k_2}, \dots, c_{k_n}]
        = [g_{f(k_1)}, g_{f(k_2)}, \dots, g_{f(k_n)}].
    \end{align*}
    故
    \begin{align*}
        \det {[c_{k_1}, c_{k_2}, \dots, c_{k_n}]}
        = {} &
        \det {[g_{f(k_1)}, g_{f(k_2)}, \dots, g_{f(k_n)}]}
        \\
        = {} &
        s(f(k_1), f(k_2), \dots, f(k_n))
        \det {[g_1, g_2, \dots, g_n]}
        \\
        = {} &
        s(k_1, k_2, \dots, k_n)
        \det {[c_{j_1}, c_{j_2}, \dots, c_{j_n}]}.
        \qedhere
    \end{align*}
\end{proof}

\begin{theorem}
    设 \(D\) 为 \(m \times n\)~阵,
    且 \(m \geq n\).
    % 设 \(D\) 的行~\(1\), \(2\), \(\dots\), \(m\)
    % 分别为 \(d_1\), \(d_2\), \(\dots\), \(d_m\).
    设 \(i_1\), \(i_2\), \(\dots\), \(i_n\) 是%
    不超过 \(m\) 的正整数,
    且 \(i_1 < i_2 < \dots < i_n\).
    则
    \begin{align*}
             &
        \det {\left(
            D\binom{i_1, i_2, \dots, i_n}{1, 2, \dots, n}
            \right)}
        % \det {
        %     \begin{bmatrix}
        %         d_{i_1} \\
        %         d_{i_2} \\
        %         \vdots  \\
        %         d_{i_n}
        %     \end{bmatrix}
        % }
        % =
        \\
        = {} &
        \sum_{\substack{
        k_1, k_2, \dots, k_n\,\text{是} \\
                i_1, i_2, \dots, i_n\,\text{的排列}
            }}
        {s(k_1, k_2, \dots, k_n)\,
            [D]_{k_1,1} [D]_{k_2,2} \dots [D]_{k_n,n}}.
    \end{align*}
\end{theorem}

\begin{proof}
    设
    \begin{align*}
        H =
        % \begin{bmatrix}
        %     d_{i_1} \\
        %     d_{i_2} \\
        %     \vdots  \\
        %     d_{i_n}
        % \end{bmatrix}.
        D\binom{i_1, i_2, \dots, i_n}{1, 2, \dots, n}.
    \end{align*}
    不难看出,
    % \([H]_{u,v} = [d_{i_u}]_{1,v} = [D]_{i_u,v}\).
    \([H]_{u,v} = [D]_{i_u,v}\).

    记 \(f(i_t) = t\)
    (\(t = 1\), \(2\), \(\dots\), \(n\)).
    设 \(k_1\), \(k_2\), \(\dots\), \(k_n\)
    是 \(i_1\), \(i_2\), \(\dots\), \(i_n\)
    的一个排列.
    那么, \(f(k_1)\), \(f(k_2)\), \(\dots\), \(f(k_n)\)
    一定是 \(1\), \(2\), \(\dots\), \(n\) 的一个排列.
    反过来, 若 \(v_1\), \(v_2\), \(\dots\), \(v_n\) 是
    \(1\), \(2\), \(\dots\), \(n\) 的一个排列,
    则 \(i_{v_1}\), \(i_{v_2}\), \(\dots\), \(i_{v_n}\)
    一定是 \(i_1\), \(i_2\), \(\dots\), \(i_n\) 的一个排列.
    并且, 若 \(k_t < k_v\), 必有 \(f(k_t) < f(k_v)\).
    反过来, 若 \(f(k_t) < f(k_v)\), 必有 \(k_t < k_v\).
    从而
    \(\operatorname{sgn} {(k_v - k_t)}
    = \operatorname{sgn} {(f(k_v) - f(k_t))}\).
    故
    \begin{align*}
             &
        % \det {
        %     \begin{bmatrix}
        %         d_{i_1} \\
        %         d_{i_2} \\
        %         \vdots  \\
        %         d_{i_n}
        %     \end{bmatrix}
        % }
        \det {\left(
            D\binom{i_1, i_2, \dots, i_n}{1, 2, \dots, n}
            \right)}
        \\
        = {} &
        \det {(H)}
        \\
        = {} &
        \sum_{\substack{
        k_1, k_2, \dots, k_n\,\text{是} \\
                i_1, i_2, \dots, i_n\,\text{的排列}
            }}
        {s(f(k_1), f(k_2), \dots, f(k_n))\,
            [H]_{f(k_1),1} [H]_{f(k_2),2} \dots [H]_{f(k_n),n}}
        \\
        = {} &
        \sum_{\substack{
        k_1, k_2, \dots, k_n\,\text{是} \\
                i_1, i_2, \dots, i_n\,\text{的排列}
            }}
        {s(k_1, k_2, \dots, k_n)\,
            [D]_{k_1,1} [D]_{k_2,2} \dots [D]_{k_n,n}}.
        \qedhere
    \end{align*}
\end{proof}

\begin{theorem}
    设 \(n \leq m\).
    设%
    % \(m^n\)~个%
    数 \(f(k_1, k_2, \dots, k_n)\)
    (其中
    \(k_1\), \(k_2\), \(\dots\), \(k_n\) 过
    \(1\), \(2\), \(\dots\), \(m\))
    适合:
    若 \(k_p = k_q\) (\(1 \leq p < q \leq n\)),
    则 \(f(k_1, k_2, \dots, k_n) = 0\).
    那么
    \begin{align*}
             &
        \sum_{1 \leq k_1, k_2, \dots, k_n \leq m}
        {f(k_1, k_2, \dots, k_n)}
        \\
        = {} &
        \sum_{\substack{
        1 \leq k_1, k_2, \dots, k_n \leq m \\
                k_1, k_2, \dots, k_n\,\text{互不相同}
            }}
        {f(k_1, k_2, \dots, k_n)}
        \\
        = {} &
        \sum_{1 \leq j_1 < j_2 < \dots < j_n \leq m}
        {\sum_{\substack{
        k_1, k_2, \dots, k_n\,\text{是}     \\
                    j_1, j_2, \dots, j_n\,\text{的排列}
                }}
            {f(k_1, k_2, \dots, k_n)}}.
    \end{align*}
\end{theorem}

\begin{proof}
    由 \(f\) 的性质可知, 前一个等式成立.
    后一个等式也不难.
    注意到,
    要从 \(m\)~个不同的数里\emph{有前后次序地}选
    \(n\)~个不同的数
    \(k_1\), \(k_2\), \(\dots\), \(k_n\),
    我们可以这样:
    先\emph{从小到大地}从这 \(m\)~个数里选 \(n\)~个%
    不同的数
    \(j_1\), \(j_2\), \(\dots\), \(j_n\),
    再\emph{有前后次序地}作这 \(n\)~个数的排列
    \(k_1\), \(k_2\), \(\dots\), \(k_n\).
    所以后一个等式也成立.
\end{proof}

现在, 我们证明 Binet--Cauchy 公式.

\begin{proof}
    设 \(B\) 的列
    \(1\), \(2\), \(\dots\), \(m\)
    分别是
    \(b_1\), \(b_2\), \(\dots\), \(b_m\).
    设 \(A\) 的列
    \(1\), \(2\), \(\dots\), \(n\)
    分别是
    \(a_1\), \(a_2\), \(\dots\), \(a_n\).
    那么
    \begin{align*}
        BA
        = {} & [Ba_1, Ba_2, \dots, Ba_n] \\
        = {} &
        \left[
        \sum_{k_1 = 1}^{m} {[A]_{k_1,1} b_{k_1}},
        \sum_{k_2 = 1}^{m} {[A]_{k_2,2} b_{k_2}},
        \dots,
        \sum_{k_n = 1}^{m} {[A]_{k_n,n} b_{k_n}}
        \right].
    \end{align*}
    利用行列式的多线性, 有
    \begin{align*}
             & \det {(BA)}
        \\
        = {} &
        \det {\left[
        \sum_{k_1 = 1}^{m} {[A]_{k_1,1} b_{k_1}},
        \sum_{k_2 = 1}^{m} {[A]_{k_2,2} b_{k_2}},
        \dots,
        \sum_{k_n = 1}^{m} {[A]_{k_n,n} b_{k_n}}
        \right]}
        \\
        = {} &
        \sum_{k_1 = 1}^{m} {[A]_{k_1,1}
        \det {\left[
        b_{k_1},
        \sum_{k_2 = 1}^{m} {[A]_{k_2,2} b_{k_2}},
        \dots,
        \sum_{k_n = 1}^{m} {[A]_{k_n,n} b_{k_n}}
        \right]}}
        \\
        = {} &
        \sum_{k_1 = 1}^{m} {
        \sum_{k_2 = 1}^{m} {
        [A]_{k_1,1} [A]_{k_2,2}
        \det {\left[
        b_{k_1},
        b_{k_2},
        \dots,
        \sum_{k_n = 1}^{m} {[A]_{k_n,n} b_{k_n}}
        \right]}}}
        \\
        = {} &
        \dots \dots \dots \dots
        \dots \dots \dots \dots
        \dots \dots \dots \dots
        \dots \dots \dots \dots
        \dots \dots \dots \dots
        \\
        = {} &
        \sum_{1 \leq k_1, k_2, \dots, k_n \leq m}
        {[A]_{k_1,1} [A]_{k_2,2} \dots [A]_{k_n,n}
            \det {[b_{k_1}, b_{k_2}, \dots, b_{k_n}]}}.
    \end{align*}

    记
    \(
    f(k_1, k_2, \dots, k_n)
    = [A]_{k_1,1} [A]_{k_2,2} \dots [A]_{k_n,n}
    \det {[b_{k_1}, b_{k_2}, \dots, b_{k_n}]}.
    \)
    根据行列式的交错性,
    当 \(k_1\), \(k_2\), \(\dots\), \(k_n\) 中有二个数相同时,
    \(f(k_1, k_2, \dots, k_n) = 0\).

    (1)
    若 \(n > m\), 则 \(k_1\), \(k_2\), \(\dots\), \(k_n\)
    中总是有二个相同
    (抽屉原理).
    所以, 每一项都是零.
    故 \(\det {(BA)} = 0\).

    (2)
    若 \(n \leq m\),
    那么 \(\det {(BA)}\) 等于
    \begin{align*}
        \sum_{1 \leq j_1 < j_2 < \dots < j_n \leq m}
        {\sum_{\substack{
        k_1, k_2, \dots, k_n\,\text{是} \\
                    j_1, j_2, \dots, j_n\,\text{的排列}
                }}
            {[A]_{k_1,1} [A]_{k_2,2} \dots [A]_{k_n,n}
                \det {[b_{k_1}, b_{k_2}, \dots, b_{k_n}]}}}.
    \end{align*}
    因为
    \(1 \leq j_1 < j_2 < \dots < j_n \leq m\),
    且
    \(k_1\), \(k_2\), \(\dots\), \(k_n\)
    是
    \(j_1\), \(j_2\), \(\dots\), \(j_n\)
    的排列,
    故
    \begin{align*}
        \det {[b_{k_1}, b_{k_2}, \dots, b_{k_n}]}
        = s(k_1, k_2, \dots, k_n)
        \det {[b_{j_1}, b_{j_2}, \dots, b_{j_n}]},
    \end{align*}
    从而 \(\det {(BA)}\) 等于
    \begin{align*}
         &
        \sum_{1 \leq j_1 < j_2 < \dots < j_n \leq m}
        {\det {[b_{j_1}, b_{j_2}, \dots, b_{j_n}]}}
        \\
         & \qquad \qquad \qquad
        \cdot
        \sum_{\substack{
        k_1, k_2, \dots, k_n\,\text{是} \\
                j_1, j_2, \dots, j_n\,\text{的排列}
            }}
        {s(k_1, k_2, \dots, k_n)\,
            [A]_{k_1,1} [A]_{k_2,2} \dots [A]_{k_n,n}}.
    \end{align*}
    注意到
    \begin{align*}
             &
        \sum_{\substack{
        k_1, k_2, \dots, k_n\,\text{是} \\
                j_1, j_2, \dots, j_n\,\text{的排列}
            }}
        {s(k_1, k_2, \dots, k_n)\,
            [A]_{k_1,1} [A]_{k_2,2} \dots [A]_{k_n,n}}
        \\
        = {} &
        \det {\left(
            A\binom{j_1, j_2, \dots, j_n}{1, 2, \dots, n}
            \right)}.
    \end{align*}
    再注意到,
    因为
    \(1 \leq j_1 < j_2 < \dots < j_n \leq m\),
    故,
    由子阵的记号的定义,
    \begin{align*}
        [b_{j_1}, b_{j_2}, \dots, b_{j_n}]
        = B\binom{1, 2, \dots, n}{j_1, j_2, \dots, j_n}.
    \end{align*}
    从而
    \begin{align*}
             & \det {(BA)}
        \\
        = {} & \sum_{1 \leq j_1 < j_2 < \dots < j_n \leq m}
        {
            \det {\left(
                B\binom{1, 2, \dots, n}{j_1, j_2, \dots, j_n}
                \right)}
            \det {\left(
                A\binom{j_1, j_2, \dots, j_n}{1, 2, \dots, n}
                \right)}
        }.
    \end{align*}
\end{proof}

\section{杂例}

\begin{example}
    设 \(A\) 是 \(n\)~级阵.
    设 \(k\) 是数.
    求证: \(\det {(kA)} = k^n \det {(A)}\).
\end{example}

这里, 我展现一种用行列式定义的方法.

不过, 先回想定义:

\DefinitionDeterminants*

\begin{proof}
    我们用数学归纳法证明此事.
    取数 \(k\).
    具体地, 设 \(P(n)\) 为命题
    \begin{quotation}
        对任何 \(n\)~级阵 \(A\),
        \begin{align*}
            \det {(kA)} = k^n \det {(A)}.
        \end{align*}
    \end{quotation}
    则, 我们的目标是:
    对任何正整数 \(n\), \(P(n)\) 是正确的.

    \(P(1)\) 是正确的.
    任取 \(1\)~级阵 \(A = [a]\).
    那么, \(kA = [ka]\).
    所以,
    \begin{align*}
        \det {(kA)} = ka = k^1 \det {(A)}.
    \end{align*}

    现在, 我们假定 \(P(m-1)\) 是正确的.
    我们要证 \(P(m)\) 也是正确的.
    任取 \(m\)~级阵 \(A\).
    按 \(kA\)~的列~\(1\) 展开, 有
    \begin{align*}
        \det {(kA)}
        = {} & \sum_{i = 1}^{m}
        {(-1)^{i+1} [kA]_{i,1} \det {((kA) (i|1))}}
        \\
        = {} & \sum_{i = 1}^{m}
        {(-1)^{i+1} k [A]_{i,1} \det {((kA) (i|1))}}.
    \end{align*}
    注意到, 每个 \((kA)(i|1)\) 都是 \(m-1\)~级阵.
    并且, 不难验证, \((kA)(i|1) = k(A(i|1))\).
    从而, 根据假定,
    \begin{align*}
        \det {((kA)(i|1))} = k^{m-1} \det {(A(i|1))}.
    \end{align*}
    所以
    \begin{align*}
        \det {(kA)}
        = {} & \sum_{i = 1}^{m}
        {(-1)^{i+1} k [A]_{i,1} k^{m-1} \det {(A(i|1))}}
        \\
        = {} & k k^{m-1}
        \sum_{i = 1}^{m}
        {(-1)^{i+1} [A]_{i,1} \det {(A(i|1))}}
        \\
        = {} & k^m \det {(A)}.
    \end{align*}
    所以, \(P(m)\) 是正确的.
    由数学归纳法原理, 待证命题成立.
\end{proof}

% 当然, 我认为, 这么解题, 是有一些复杂的.
% 您可以考虑用别的方法 (或想法);
% 不过, 我就不在这里写出来了.

当然, 本题也有其他的解法.
若您适当地运用行列式的性质,
您或许能较简单地解此题.

\vspace{2ex}

在讲后一个例前,
我有必要提及阵的新记号.

\begin{definition}
    设 \(A\), \(B\), \(C\), \(D\)
    分别是 \(m \times s\), \(n \times s\),
    \(m \times t\), \(n \times t\)~阵.
    我们约定, 记号
    \begin{align*}
        \begin{bmatrix}
            A & C \\
            B & D \\
        \end{bmatrix}
    \end{align*}
    表示一个 \((m + n) \times (s + t)\)-阵 \(M\),
    且对任何不超过 \(m + n\)~的正整数 \(i\)
    与不超过 \(s + t\)~的正整数 \(j\),
    \begin{align*}
        [M]_{i,j}
        = \begin{cases}
              [A]_{i,j},
               & \text{\(i \leq m\), \(j \leq s\)}; \\
              [B]_{i-m,j},
               & \text{\(i > m\), \(j \leq s\)};    \\
              [C]_{i,j-s},
               & \text{\(i \leq m\), \(j > s\)};    \\
              [D]_{i-m,j-s},
               & \text{\(i > m\), \(j > s\)}.
          \end{cases}
    \end{align*}
\end{definition}

比如, 设
\begin{align*}
     &
    A = \begin{bmatrix}
            10 & 12 & 14 \\
            11 & 13 & 15 \\
        \end{bmatrix},
    \\
     &
    B = \begin{bmatrix}
            16 & 20 & 24 \\
            17 & 21 & 25 \\
            18 & 22 & 26 \\
            19 & 23 & 27 \\
        \end{bmatrix},
    \\
     &
    C = \begin{bmatrix}
            28 & 30 & 32 & 34 & 36 \\
            29 & 31 & 33 & 35 & 37 \\
        \end{bmatrix},
    \\
     &
    D = \begin{bmatrix}
            38 & 42 & 46 & 50 & 54 \\
            39 & 43 & 47 & 51 & 55 \\
            40 & 44 & 48 & 52 & 56 \\
            41 & 45 & 49 & 53 & 57 \\
        \end{bmatrix}.
\end{align*}
则
\begin{align*}
    \begin{bmatrix}
        A & C \\
        B & D \\
    \end{bmatrix}
    =
    \begin{bmatrix}
        10 & 12 & 14 & 28 & 30 & 32 & 34 & 36 \\
        11 & 13 & 15 & 29 & 31 & 33 & 35 & 37 \\
        16 & 20 & 24 & 38 & 42 & 46 & 50 & 54 \\
        17 & 21 & 25 & 39 & 43 & 47 & 51 & 55 \\
        18 & 22 & 26 & 40 & 44 & 48 & 52 & 56 \\
        19 & 23 & 27 & 41 & 45 & 49 & 53 & 57 \\
    \end{bmatrix}.
\end{align*}

设 \(A\), \(B\), \(C\), \(D\)
分别是 \(m \times s\), \(n \times s\),
\(m \times t\), \(n \times t\)~阵.
再设
\begin{align*}
    M = \begin{bmatrix}
            A & C \\
            B & D \\
        \end{bmatrix}.
\end{align*}
不难验证, 若 \(i \leq m\), 且 \(j \leq s\),
则
\begin{align*}
    M(i|j)
    =
    \begin{bmatrix}
        A(i|j) & C(i|) \\
        B(|j)  & D     \\
    \end{bmatrix},
\end{align*}
其中 \(B(|j)\) 表示去除 \(B\)~的列~\(j\) 后得到的阵
(不去除它的行),
\(C(i|)\) 表示去除 \(C\)~的行~\(i\) 后得到的阵
(不去除它的列).

\begin{example}
    设 \(A\), \(D\) 分别是 \(n\), \(t\)~级阵.
    设 \(C\) 是 \(n \times t\)~阵.
    求证:
    \begin{align*}
        \det {
            \begin{bmatrix}
                A & C \\
                0 & D \\
            \end{bmatrix}
        }
        = \det {(A)} \det {(D)},
    \end{align*}
    其中左下角的 \(0\) 指 \(t \times n\)~零阵.
\end{example}

% 若您学过 ``按多列 (行) 展开行列式'',
% 再利用行列式的多线性
% (具体地, 若一个方阵含一列 \(0\), 或含一行 \(0\),
% 则其行列式为 \(0\)),
% 本题是十分简单的;
% 不过, 为降低学习难度,
% 我标记它为选学内容.
% 故, 我还是用定义解此题.

我还是用定义解此题.
若您学过 ``按多列 (行) 展开行列式'',
您或许能较简单地解此题.

\begin{proof}
    我们用数学归纳法证明此事.
    取 \(t\)~级阵 \(D\).
    具体地, 设 \(P(n)\) 为命题
    \begin{quotation}
        对任何 \(n\)~级阵 \(A\), 任何 \(n \times t\)~阵 \(C\),
        \begin{align*}
            \det {
                \begin{bmatrix}
                    A & C \\
                    0 & D \\
                \end{bmatrix}
            }
            = \det {(A)} \det {(D)},
        \end{align*}
        其中左下角的 \(0\) 指 \(t \times n\)~零阵.
    \end{quotation}
    则, 我们的目标是:
    对任何正整数 \(n\), \(P(n)\) 是正确的.

    \(P(1)\) 是正确的.
    设 \(A = [a]\).
    记
    \begin{align*}
        J =
        \begin{bmatrix}
            A & C \\
            0 & D \\
        \end{bmatrix},
    \end{align*}
    其中左下角的 \(0\) 指 \(t \times 1\)~零阵.
    则
    \begin{align*}
             & \det {(J)}
        \\
        = {} & \sum_{i = 1}^{1+t}
        {(-1)^{i+1} [J]_{i,1} \det {(J(i|1))}}
        \\
        = {} & \sum_{i = 1}^{1}
        {(-1)^{i+1} [J]_{i,1} \det {(J(i|1))}}
        +
        \sum_{i = 2}^{1+t}
        {(-1)^{i+1} [J]_{i,1} \det {(J(i|1))}}
        \\
        = {} &
        (-1)^{1+1} [J]_{1,1} \det {(J(1|1))}
        +
        \sum_{i = 2}^{1+t}
        {(-1)^{i+1} 0 \det {(J(i|1))}}
        \\
        = {} &
        [A]_{1,1} \det {(D)}
        \\
        = {} & \det {(A)} \det {(D)}.
    \end{align*}

    现在, 我们假定 \(P(m-1)\) 是正确的.
    我们要证 \(P(m)\) 也是正确的.
    任取 \(m\)~级阵 \(A\).
    记
    \begin{align*}
        M = \begin{bmatrix}
                A & C \\
                0 & D \\
            \end{bmatrix},
    \end{align*}
    其中左下角的 \(0\) 指 \(t \times m\)~零阵.
    则
    \begin{align*}
             & \det {(M)}
        \\
        = {} & \sum_{i = 1}^{m+t}
        {(-1)^{i+1} [M]_{i,1} \det {(M(i|1))}}
        \\
        = {} & \sum_{i = 1}^{m}
        {(-1)^{i+1} [M]_{i,1} \det {(M(i|1))}}
        +
        \sum_{i = m+1}^{m+t}
        {(-1)^{i+1} [M]_{i,1} \det {(M(i|1))}}
        \\
        = {} & \sum_{i = 1}^{m}
        {(-1)^{i+1} [A]_{i,1} \det {(M(i|1))}}
        +
        \sum_{i = m+1}^{m+t}
        {(-1)^{i+1} 0 \det {(M(i|1))}}
        \\
        = {} & \sum_{i = 1}^{m}
        {(-1)^{i+1} [A]_{i,1} \det {(M(i|1))}}.
    \end{align*}
    注意到, \(1 \leq i \leq m\) 时,
    \begin{align*}
        M(i|1)
        = \begin{bmatrix}
              A(i|1) & C(i|) \\
              0      & D     \\
          \end{bmatrix},
    \end{align*}
    其中左下角的 \(0\) 指 \(t \times (m-1)\)~零阵.
    根据假定,
    \begin{align*}
        \det {(M(i|1))} = \det {(A(i|1))} \det {(D)}.
    \end{align*}
    从而
    \begin{align*}
        \det {(M)}
        = {} & \sum_{i = 1}^{m}
        {(-1)^{i+1} [A]_{i,1} \det {(A(i|1))} \det {(D)}}
        \\
        = {} & \left(\sum_{i = 1}^{m}
        {(-1)^{i+1} [A]_{i,1} \det {(A(i|1))}}\right)
        \det {(D)}
        \\
        = {} & \det {(A)} \det {(D)}.
    \end{align*}

    所以, \(P(m)\) 是正确的.
    由数学归纳法原理, 待证命题成立.
\end{proof}

\KunAsteriskoEnEnhavtabelo
\section{La lasta leciono}
\SenAsteriskoEnEnhavtabelo

\maldevigalegajxo

这是 ``最后一课'', la lasta leciono.

恭喜!
您学完了本章;
或者说, 我的行列式课程完了.
您可能会想:
``我还能再学些什么?''
我认为, 这个问题, 较难回答:
毕竟, 行列式只是一个工具而已.
比如,
学完我的课程后,
可以进一步地学习线性代数 (或高等代数).

若您想见别的文献,
以下的说明或许对您是有用的.

习惯地, 几乎每一个 (说汉语的) 作者都说 ``矩阵'',
而不是 ``阵''
(当然, ``方阵'' 是一个例外).

习惯地, 人们叫一个 \(1 \times n\)~阵为一个行向量,
叫一个 \(m \times 1\)~阵为一个列向量.
行向量与列向量可被统一地叫作向量.
不过, 当初我写作时, 不想给您较多挑战,
故我未正式地引入 ``向量'' 二字.

我一般表
\(A\)~的 \((i, j)\)-元%
以 \([A]_{i,j}\).
此记号自然是较准确的;
并且,
% 因为括号 \({[} \ {]}\) 的天然的属性,
我们甚至可自由地代括号里的单文字 \(A\)
以较复杂的文字, 如 \(3A + 4B\)
(当 \(A\), \(B\) 是同尺寸的阵时),
且较方便地作计算:
\begin{align*}
    [3A + 4B]_{i,j}
        = [3A]_{i,j} + [4B]_{i,j}
        = 3[A]_{i,j} + 4[B]_{i,j}.
\end{align*}
不过, 此记号自然是较复杂的.

其实,
大多数作者,
习惯地,
不用记号 \([A]_{i,j}\),
而是用较简单的记号 \(a_{ij}\)
(注意到, 这儿没有逗号):
\begin{align*}
    \begin{bmatrix}
        a_{11} & a_{12} & \cdots & a_{1n} \\
        a_{21} & a_{22} & \cdots & a_{2n} \\
        \vdots & \vdots & {}     & \vdots \\
        a_{m1} & a_{m2} & \cdots & a_{mn} \\
    \end{bmatrix}
    =
    \begin{bmatrix}
        [A]_{1,1} & [A]_{1,2} & \cdots & [A]_{1,n} \\
        [A]_{2,1} & [A]_{2,2} & \cdots & [A]_{2,n} \\
        \vdots    & \vdots    & {}     & \vdots    \\
        [A]_{m,1} & [A]_{m,2} & \cdots & [A]_{m,n} \\
    \end{bmatrix}.
\end{align*}
于是, 在此写法下,
\(A\)~的 \((2, 3)\)-元是 \(a_{23}\),
\(B\)~的 \((4, 2)\)-元是 \(b_{42}\)
(但是, 若 \(i\), \(j\) 的一个不能被 ``单文字'' 表示时,
逗号仍被用:
比如, \(C\)~的 \((11, 9)\)-元是 \(c_{11,9}\)).
这种记号可被较方便地写,
但是, 这种记号, 要求阵被 ``单文字'' 表示.
比如, 设 \(A\), \(B\) 分别是
\(m \times s\), \(s \times n\)~阵.
习惯地, 不写 \(AB\)~的 \((i, j)\)-元为
\({ab}_{ij}\)
(因为, \({ab}_{ij}\),
习惯地,
会被理解为%
一个数~\(a\) 跟
\(B\)~的 \((i, j)\)-元 \(b_{ij}\)
的积),
而是记 \(C = AB\)
(也就是说, 为 \(AB\) 取 ``小名'' \(C\)),
再说 \(AB\)~的 \((i, j)\)-元
\begin{align*}
    c_{ij}
    = a_{i1} b_{1j} + a_{i2} b_{2j} + \dots
    + a_{is} b_{sj}.
\end{align*}

有的作者不用括号 \({[} \ {]}\),
而是用括号 \(( \ )\),
以包围数表, 作成一个阵:
\begin{align*}
    \begin{pmatrix}
        a_{11} & a_{12} & \cdots & a_{1n} \\
        a_{21} & a_{22} & \cdots & a_{2n} \\
        \vdots & \vdots & {}     & \vdots \\
        a_{m1} & a_{m2} & \cdots & a_{mn} \\
    \end{pmatrix}
    =
    \begin{bmatrix}
        a_{11} & a_{12} & \cdots & a_{1n} \\
        a_{21} & a_{22} & \cdots & a_{2n} \\
        \vdots & \vdots & {}     & \vdots \\
        a_{m1} & a_{m2} & \cdots & a_{mn} \\
    \end{bmatrix}.
\end{align*}

有的作者用大写拉丁字母 \(O\) 表示一个零阵.

有的作者用大写拉丁字母 \(E\) 表示一个单位阵.
也有的作者用阿拉伯数字 \(1\) 表示一个单位阵.

有的作者用加粗的文字表示阵.
具体地, 他们写
\(\mathbf{A}\),
\(\mathbf{B}\),
\(\mathbf{C}\),
\(\mathbf{E}\),
\(\mathbf{I}\),
\(\mathbf{1}\),
\(\mathbf{O}\),
\(\mathbf{0}\),
\(\mathbf{x}\),
\(\dots\),
而不是
\(A\), \(B\), \(C\),
\(E\), \(I\), \(1\),
\(O\), \(0\),
\(x\), \(\dots\).
% 我可以理解此事:
这较好地区分了阵与不是阵的对象.
不过, 加粗的文字不便手写,
故此类记号一般是印刷专用的.

一些作者, 习惯地, 写%
一个方阵~\(A\) 的行列式
\(\det {(A)}\) 为 \(|A|\).
若 \(A\)~的元被具体地写出,
则有记号
\begin{align*}
    \begin{vmatrix}
        a_{11} & a_{12} & \cdots & a_{1n} \\
        a_{21} & a_{22} & \cdots & a_{2n} \\
        \vdots & \vdots & {}     & \vdots \\
        a_{n1} & a_{n2} & \cdots & a_{nn} \\
    \end{vmatrix}
    = {} &
    \left|\,
    \begin{bmatrix}
        a_{11} & a_{12} & \cdots & a_{1n} \\
        a_{21} & a_{22} & \cdots & a_{2n} \\
        \vdots & \vdots & {}     & \vdots \\
        a_{n1} & a_{n2} & \cdots & a_{nn} \\
    \end{bmatrix}
    \,\right|
    \\
    = {} &
    \det {
        \begin{bmatrix}
            a_{11} & a_{12} & \cdots & a_{1n} \\
            a_{21} & a_{22} & \cdots & a_{2n} \\
            \vdots & \vdots & {}     & \vdots \\
            a_{n1} & a_{n2} & \cdots & a_{nn} \\
        \end{bmatrix}
    }.
\end{align*}
其实, 历史地, 行列式比阵早出现;
于是, 行列式有专门的竖线记号.

一些作者讲行列式时, 会提及 ``余子式'' 与
``代数余子式''.
通俗地, 一个方阵 \(A\)~的 \((i, j)\)-元
\(a_{i,j}\) 的%
余子式 \(M_{i,j}\),
% (这里, 为尽可能地消除歧义, 我用了逗号),
是行列式
\begin{align*}
    \begin{vmatrix}
        a_{1,1}     & \cdots & a_{1,j-1}   &
        a_{1,j+1}   & \cdots & a_{1,n}       \\
        \vdots      & {}     & \vdots      &
        \vdots      & {}     & \vdots        \\
        a_{i-1,1}   & \cdots & a_{i-1,j-1} &
        a_{i-1,j+1} & \cdots & a_{i-1,n}     \\
        a_{i+1,1}   & \cdots & a_{i+1,j-1} &
        a_{i+1,j+1} & \cdots & a_{i+1,n}     \\
        \vdots      & {}     & \vdots      &
        \vdots      & {}     & \vdots        \\
        a_{n,1}     & \cdots & a_{n,j-1}   &
        a_{n,j+1}   & \cdots & a_{n,n}       \\
    \end{vmatrix}.
\end{align*}
一个方阵 \(A\)~的 \((i, j)\)-元
\(a_{i,j}\) 的%
代数余子式 \(A_{i,j}\)
(有的作者记其为 \(C_{i,j}\)),
是
\((-1)^{i+j} M_{i,j}\).
于是, 不难看出,
对任何不超过 \(n\)~的正整数 \(i\), \(j\),
\begin{align*}
    |A|
    = {} &
    (-1)^{i+1} a_{i,1} M_{i,1}
    + (-1)^{i+2} a_{i,2} M_{i,2}
    + \dots
    + (-1)^{i+n} a_{i,n} M_{i,n}
    \\
    = {} &
    (-1)^{1+j} a_{1,j} M_{1,j}
    + (-1)^{2+j} a_{2,j} M_{2,j}
    + \dots
    + (-1)^{n+j} a_{n,j} M_{n,j},
\end{align*}
或
\begin{align*}
    |A|
    = {} &
    a_{i,1} A_{i,1}
    + a_{i,2} A_{i,2}
    + \dots
    + a_{i,n} A_{i,n}
    \\
    = {} &
    a_{1,j} A_{1,j}
    + a_{2,j} A_{2,j}
    + \dots
    + a_{n,j} A_{n,j}.
\end{align*}
我没有讲这二个概念,
因为我引入了子阵\emph{及其记号}
(有些作者不介绍表示子阵的记号).
其实,
余子式就是子阵的行列式:
\(
M_{i,j} = \det {(A(i|j))}
\).
这么看来, 我不必引入余子式,
也不必引入代数余子式了.
并且, 我 ``不想给您较多挑战''.

有的作者写方阵 \(A\)~的古伴
\(\operatorname{adj} {(A)}\)
为 \(A^{\ast}\),
且叫 ``古伴'' (即 ``古典伴随阵'')
为 ``伴随'' ``伴随阵''.

我希望, 这些说明,
可助您较顺利地适应其他的文献上的符号.
或许, 就像嬴政统一文字那样,
数学的符号也应被统一;
可是, 此事有些难.
我能作的事, 至多是助您适应这些符号.

最后, 我想说,
行列式并不只是代数的工具.
行列式在微积分与几何里也是有用的;
您可以去相关的文献里找到更多的讨论.
% (也可以直接去互联网看看).
% 不过, 我对此了解不深;
% 再者, 若我进一步讲下去, 我就要
% ``一台电脑, 一个键盘, 一个图画一天''
% 了
% (乳胶可不是什么简单的东西,
% 但因为, 跟词儿比,
% 乳胶对我, 用鼠标有些困难的人,
% 更友好一些,
% 我忍了).

我就说这么多.
% 或许, 后会有期.
再见.
