% Specify the fonts used by the document.

\IfFontExistsTF{XITS}{%
    \IfFontExistsTF{XITS Italic}{%
        \IfFontExistsTF{Fira Sans SemiBold}{%
            \IfFontExistsTF{Fira Sans SemiBold Italic}{%
                \setmainfont{XITS}[
                    UprightFont = *,
                    BoldFont = Fira Sans SemiBold,
                    ItalicFont = * Italic,
                    BoldItalicFont = Fira Sans SemiBold Italic
                ]
            }{%
                \setmainfont{XITS}
            }
        }{%
            \setmainfont{XITS}
        }
    }{}
}{}

\IfFontExistsTF{Fira Sans Light}{%
    \IfFontExistsTF{Fira Sans Light Italic}{%
        \IfFontExistsTF{Fira Sans SemiBold}{%
            \IfFontExistsTF{Fira Sans SemiBold Italic}{%
                \setsansfont{Fira Sans}[
                    UprightFont = * Light,
                    BoldFont = * SemiBold,
                    ItalicFont = * Light Italic,
                    BoldItalicFont = * SemiBold Italic
                ]
            }{}
        }{}
    }{}
}{}

\IfFontExistsTF{Sarasa Mono SC Light}{%
    \IfFontExistsTF{Sarasa Mono SC Light Italic}{%
        \IfFontExistsTF{Sarasa Mono SC SemiBold}{%
            \IfFontExistsTF{Sarasa Mono SC SemiBold Italic}{%
                \setmonofont{Sarasa Mono SC}[
                    UprightFont = * Light,
                    BoldFont = * SemiBold,
                    ItalicFont = * Light Italic,
                    BoldItalicFont = * SemiBold Italic
                ]
            }{}
        }{}
    }{}
}{}

\IfFontExistsTF{Source Han Serif CN Light}{%
    \IfFontExistsTF{Sarasa Mono SC Light Italic}{%
        \IfFontExistsTF{Sarasa Mono SC SemiBold}{%
            \IfFontExistsTF{Sarasa Mono SC SemiBold Italic}{%
                \PassOptionsToPackage{fontset = none}{ctexbook}
                \setCJKmainfont{Source Han Serif CN}[
                    UprightFont = * Light,
                    BoldFont = Sarasa Mono SC SemiBold,
                    ItalicFont = Sarasa Mono SC Light Italic,
                    BoldItalicFont = Sarasa Mono SC SemiBold Italic
                ]
                \setCJKsansfont{Sarasa Mono SC}[
                    UprightFont = * Light,
                    BoldFont = * SemiBold,
                    ItalicFont = * Light Italic,
                    BoldItalicFont = * SemiBold Italic
                ]
                \setCJKmonofont{Sarasa Mono SC}[
                    UprightFont = * Light,
                    BoldFont = * SemiBold,
                    ItalicFont = * Light Italic,
                    BoldItalicFont = * SemiBold Italic
                ]
            }{}
        }{}
    }{}
}{}

% Redefine the emphasis style.
\let\emph\relax
\DeclareTextFontCommand{\emph}{\bfseries}

% These are packages which deal with mathematical formulae.
\usepackage{amsmath}
\usepackage{mathtools}
\usepackage[
    math-style=ISO,
    bold-style=ISO,
    mathrm=sym,
    mathbf=sym,
    partial=upright,
    warnings-off={mathtools-colon,mathtools-overbracket}
]{unicode-math}
\allowdisplaybreaks[3]

% Specify the mathematical fonts used by the document.
\IfFontExistsTF{XITS Math}{%
    % Standing-up-straight integral symbol (For XITS Math).
    \setmathfont{XITS Math}[StylisticSet=8]%
}{}
\IfFontExistsTF{TeX Gyre Termes Math}{%
\setmathfont{TeX Gyre Termes Math}[range=bb/{latin,Latin}]%
}{}
\IfFontExistsTF{STIX Two Math}{%
    \setmathfont{STIX Two Math}[range={"2218-"2218}]
}{}


% These are packages which deal with theorem environments.
\usepackage{amsthm}
\usepackage{thmtools}
\usepackage{thm-restate}

\declaretheoremstyle[%
    spaceabove=2ex,%
    spacebelow=2ex,%
    headfont=\normalfont\bfseries,%
    notefont=\normalfont,%
    notebraces={(}{)},%
    bodyfont=\normalfont,%
    headindent=\parindent,%
    headpunct={},%
    postheadspace=1em%
]{TheoremStyle}
\declaretheoremstyle[%
    spaceabove=2ex,%
    spacebelow=2ex,%
    headfont=\normalfont\bfseries,%
    notefont=\normalfont,%
    notebraces={(}{)},%
    bodyfont=\normalfont,%
    headindent=\parindent,%
    headpunct={},%
    postheadspace=1em,%
    qed=\qedsymbol%
]{ProofStyle}

\declaretheorem[%
    style=TheoremStyle,%
    name=定理,%
    numberwithin=chapter%
]{theorem}
\declaretheorem[%
    style=TheoremStyle,%
    name=定义,%
    sibling=theorem%
]{definition}
\declaretheorem[%
    style=TheoremStyle,%
    name=例,%
    sibling=theorem%
]{example}
\declaretheorem[%
    style=TheoremStyle,%
    name=注,%
    numbered=no%
]{remark*}

% Undefine the proof environment provided by `amsthm`.
\let\proof\relax
\let\endproof\relax
\declaretheorem[%
    style=ProofStyle,%
    name=证,
    numbered=no%
]{proof}
\renewcommand*{\qedsymbol}{证毕.}

% Comment some words.
% Note that this package automatically loads hyperref.
\usepackage{pdfcomment}

% If "pdfcomment" is not available:
\providecommand{\pdfmarkupcomment}[3][]{#2}

% These are packages which deal with hyperlinks.
% % If "pdfcomment" is not used:
% \usepackage[hidelinks,hyperindex,breaklinks]{hyperref}
% If "pdfcomment" is used:
\hypersetup{hidelinks,breaklinks}

% Add hyperlinks to the table of contents.
\usepackage{bookmark}

\makeatletter
\providecommand*{\midsloppy}{%
    \tolerance 5000%
    \hbadness 4000%
    \emergencystretch 1.5em%
    \hfuzz .1\p@%
    \vfuzz\hfuzz}
\makeatother

% \makeatletter
% \providecommand*{\leqnomode}{\tagsleft@true\let\veqno\@@leqno}
% \providecommand*{\reqnomode}{\tagsleft@false\let\veqno\@@eqno}
% \makeatother

% Some xeCJK settings.
% However, they only work in XeLaTeX.
% https://tex.stackexchange.com/questions/13172/detect-which-tex-engine-is-used
\usepackage{ifxetex}
\usepackage{ifluatex}
\ifxetex
    \usepackage{xpatch}
    \ExplSyntaxOn
    % https://github.com/CTeX-org/ctex-kit/issues/636
    % Apply the patch only if the package date is after 2022-08-05
    % _and_ before 2022-08-31.
    % The bug was introduced in v3.9.1 (on 2022-08-05)
    % and fixed in dev (on 2022-08-31).
    % See
    % https://github.com/CTeX-org/ctex-kit/commit/a841dabaac7c2dccacd1858ba96f6eb38adfb335
    % https://github.com/CTeX-org/ctex-kit/commit/aa5f00db05f12034845f8e5c4119971b5aae33ff
    \IfPackageAtLeastTF{xeCJK}{2022-08-05}{
        \IfPackageAtLeastTF{xeCJK}{2022-08-31}{}{
            \xpatchcmd \__xeCJK_check_for_xglue:
            {\__xeCJK_if_last_glue:TF}
            {\__xeCJK_if_last_glue:T}
            % {}{\PatchFailed}
            {}{}
        }
    }{}
    \ExplSyntaxOff

    \xeCJKsetup{%
        % % https://github.com/CTeX-org/ctex-kit/issues/636
        xCJKecglue = true,%
        CJKspace = true,%
        CheckSingle = true,%
    }
\fi

% These are modifications which are
% to be inserted after \begin{document}.
\AtBeginDocument{%
    % Switch off extra space after punctuation.
    \frenchspacing

    % A magic that kills as many overfull lines as possible.
    \midsloppy

    \renewcommand*{\UrlFont}{\ttfamily}

    % \let\umathpi\pi
    % \renewcommand*{\pi}{\symup\umathpi}
    % \let\umathiota\iota
    % \renewcommand*{\iota}{\symup\umathiota}

    \renewcommand*{\leq}{\leqslant}
    \renewcommand*{\geq}{\geqslant}
    \renewcommand*{\mathellipsis}{\cdots}

    % \renewcommand*{\Re}{\operatorname{Re}}
    % \renewcommand*{\Im}{\operatorname{Im}}
}

\usepackage{tikz}
\usetikzlibrary{calc,matrix}

% The packages enables one to create new environments.
\usepackage{newfloat}
% Create an environment exclusively for tikzpictures.
\DeclareFloatingEnvironment[
    fileext=los,
    listname={Listo de Tikzbildoj},
    name={Tikzbildo},
    placement=tbhp,
    within=section,
]{tikzbildo}
% % Create an environment exclusively for illustrations.
% \DeclareFloatingEnvironment[
%     fileext=los,
%     listname={Listo de Ilustraĵoj},
%     name={Ilustraĵo},
%     placement=tbhp,
%     within=section,
% ]{ilustrajxo}

% Insert figures.
\usepackage{graphicx}
% % Insert and configure captions.
% \usepackage{caption}
% % Insert sub-figures.
% \usepackage{subcaption}

% Make quotations wider.
\makeatletter
\renewenvironment{quotation}
{\list{}{\listparindent 2\ccwd%
        \itemindent\listparindent
        % \rightmargin\leftmargin
        \leftmargin\rightmargin
        \parsep\z@ \@plus\p@}%
    \item\relax}
{\endlist}
\makeatother

% Start a new page with each section.
% https://tex.stackexchange.com/questions/9497/start-new-page-with-each-section
\providecommand*{\phantomsection}{}
\usepackage{titlesec}
\newcommand*{\sectionbreak}{\clearpage\phantomsection}

\makeatletter
\if@twoside % if the document is two-sided
    % https://latex.org/forum/viewtopic.php?t=6965
    \raggedbottom

    % Use "\usepackage[pass]{geometry}"
    % to see the information of page margins
    % in the log file
    % (Do it in a separate file,
    % since the current file is so big
    % that it will take unnecessarily longer time.)
    \setlength\oddsidemargin{31pt}
    \setlength\evensidemargin{31pt}
    \setlength\marginparwidth{38pt}
\else % if the document is one-sided
    % Redefine \cleardoublepage.
    \def\cleardoublepage{\clearpage\ifodd\c@page\else
            \hbox{}\newpage\if@twocolumn\hbox{}\newpage\fi\fi}
\fi
\makeatother

% Ten heavenly stems: 甲乙丙丁戊己庚辛壬癸
% Twelve earthly branches: 子丑寅卯辰巳午未申酉戌亥
% Listing comma: 、

% Insert the table of bibliography.
\usepackage[%
    backend=biber,%
    style=gb7714-2015,%
    % gbalign=gb7714-2015,%
    defernumbers=true,%
]{biblatex}
% Use a null heading for each bibliography.
\defbibheading{bibliography}[]{}

\newcommand*{\maldevigalegajxo}{%
    \begin{remark*}
        本节是选学内容;
        换句话说, 您不学本节,
        并不会影响您对任何必学内容
        (也就是, 未声明为 ``选学内容'' 的节)
        的理解.
    \end{remark*}%
}

% Make long tables.
\usepackage{longtable}
% A cell that allows line breaks.
\providecommand*{\cxelo}[2][c]{%
    \begin{tabular}[#1]{@{}l@{}}#2\end{tabular}}

% Commands for displaying section numbers:
% \sekcio{} and \malneprasekcio{}
\providecommand*{\sekcio}[1]{#1}
\providecommand*{\malneprasekcio}[1]{\textit{#1}}

% % The following code hides all instances of "ilustrajxo*".
% % ========================================
% % This provides the `\comment` block.
% \usepackage{verbatim}
% % Redefine the \figure* environment to
% % apply the `\comment` block to its contents,
% % essentially remove them all from the document.
% \renewenvironment{ilustrajxo*}[1][]{%
%     \expandafter\comment%
% }{%
%     \expandafter\endcomment%
% }

% This package deals with headers and footers.
\usepackage{fancyhdr}
\pagestyle{fancy}
\fancyhf{}
\setlength{\headheight}{15pt}
\fancyhead[LO]{\nouppercase{\rightmark}}
\fancyhead[RO]{\thepage}
\fancyhead[RE]{\nouppercase{\leftmark}}
\fancyhead[LE]{\thepage}
\renewcommand*{\headrulewidth}{0pt}
\renewcommand*{\footrulewidth}{0pt}
% \fancypagestyle{plain}{%
%     \fancyhf{}
%     \fancyfoot[C]{\sffamily\thepage}
%     \renewcommand*{\headrulewidth}{0pt}
%     \renewcommand*{\footrulewidth}{0pt}
% }

% If a chapter has no section,
% mark the header on the odd pages as well.
% Put the redefinition after fancyhdr, if it is imported.
\renewcommand*{\chaptermark}[1]{%
    \markboth{\CTEXifname{\CTEXthechapter\hspace{\ccwd}}{%
        }#1}{\CTEXifname{\CTEXthechapter\hspace{\ccwd}}{%
        }#1}%
}

\usepackage{ifdraft}

\ifdraft{% draft case
    \pagestyle{plain}

    % https://tex.stackexchange.com/questions/448296/
    \ifxetex
        \special{dvipdfmx:config z 0}
        \special{dvipdfmx:config C 0x40}
    \fi
    \ifluatex
        \pdfvariable compresslevel 0
        \pdfvariable objcompresslevel 0
    \fi
}%
{% final case
}

% Inspired by https://tex.stackexchange.com/a/688793
\newcommand\EbleEstasAsterisko{}% check definable
\let\EbleEstasAsterisko\relax
\renewcommand\thesection{\EbleEstasAsterisko\thechapter.\arabic{section}}
\newcommand\KunAsteriskoEnEnhavtabelo{%
    \addtocontents{toc}%
    {\def\EbleEstasAsterisko{\let\EbleEstasAsterisko\relax\string\llap{*}}}%
}
\newcommand\SenAsteriskoEnEnhavtabelo{%
    \addtocontents{toc}%
    {\def\EbleEstasAsterisko{\let\EbleEstasAsterisko\relax}}%
}
